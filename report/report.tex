\documentclass[11pt, a4paper]{article}

\usepackage{stylesheet}

\institution{EPFL}
\project{Semester Project}
\title{Akantu implementation of the INTERNODES method for contact mechanics}
\author{Bruno Ploumhans \& Fabio Matti}
\supervisor{Dr Guillaume Anciaux \\ Raquel Dantas Batista}
\date{\today}

\begin{document}

\maketitle

\tableofcontents

\clearpage
\section{Introduction}
\label{sec:intro}
INTERNODES (INTERpolation for NOn-conforming DEcompositionS) is a recently developed method for numerical contact mechanics. It was initially proposed in \cite{deparis} and further developed for two-body contact problems in \cite{voet}. It is believed to be simpler and more robust than alternatives such as the mortar finite element method. Continuing the work of previous students, we fully implemented and tested the INTERNODES method in the framework of Akantu\footnote{\url{https://gitlab.com/akantu/akantu}}, a generic and efficient finite element solver developed by EPFL written in C++, with the main goal of making our implementation reliable for broader usage.

In this paper, we briefly explain how the method works. Then, we go through important details of our implementation, and we dicuss the tests that we wrote to validate it. Finally, we discuss further improvements that can be made to the computational contact mechanics package of Akantu.

\clearpage
\section{The INTERNODES method for contact mechanics}
\label{sec:internodes}

In essence, the INTERNODES method makes use of two different interpolants which are used to transfer information from one interface to another. To this purpose, radial basis function (RBF) interpolants are introduced.

\subsection{Radial basis interpolation}
\label{subsec:rbf-interpolation}

The engine of radial basis interpolation are the radial basis functions (RBF) $\phi: \mathbb{R}^d \to \mathbb{R}$. These functions take values that are isometric, i.e. only depend on the distance from their corresponding node $\boldsymbol{\xi}$ and are parametrized by the radius parameter $r$.

The radial basis interpolant of $g : \mathbb{R}^d \to \mathbb{R}$ at interpolation nodes $\boldsymbol{\xi}_1, \dots, \boldsymbol{\xi}_M$ with associated radius parameters $r_1, \dots, r_M$ will be defined as
\begin{equation}
    \Pi(\mathbf{x}) = \sum_{m=1}^M g(\boldsymbol{\xi}_m) \phi(\left\| \mathbf{x} - \boldsymbol{\xi}_m \right\|, r_m)
\end{equation}
One can easily see that the interpolant is merely a superposition of rescaled RBFs. This fact, as well as the parameters used to define the radial basis function are depicted in \reffig{fig:radialbasis}.

\begin{figure}[ht]
    \centering
    \begin{tikzpicture}
    \shade[inner color=darkblue, outer color=white] (0, 0) circle (0.8);
    \fill[darkblue] (0, 0) circle (0.1) node[anchor=north east] {$\boldsymbol{\xi}_{j}$};
    \shade[inner color=darkblue, outer color=white] (-2, 1) circle (0.8);
    \fill[darkblue] (-2, 1) circle (0.1) node[anchor=north east] {$\boldsymbol{\xi}_{j+1}$};
    \shade[inner color=darkblue, outer color=white] (1.6, 2.1) circle (0.8);
    \fill[darkblue] (1.6, 2.1) circle (0.1) node[anchor=north east] {$\boldsymbol{\xi}_{j-1}$};
    \draw[darkblue, <->] (0.1, 0) to node[midway, below] {$r$} (0.8, 0);
\end{tikzpicture}
    \caption{Visualization of the parameters which are used to define radial basis functions. Notice that the radius parameters $r_j$ must not necessarily coincide with the radius of the support of the function, as is the case for the Wendland $C^2$ RBF.}\label{fig:radialbasis}
\end{figure}

It is established in \cite{voet} that the Wendland $C^2$ RBF possesses particularly desirable properties. The aforementioned RBF is defined as
\begin{equation}
    \phi(\delta) = (1 - \delta)_{+}^4(1 + 4\delta) 
\end{equation}
with $\delta = \left\| \mathbf{x} - \boldsymbol{\xi}_m \right\| / r$.

Denoting with $\mathbf{g}_{\zeta} = (g(\boldsymbol{\zeta}_1), \dots, g(\boldsymbol{\zeta}_N))^{\top}$ the evaluations of the function $g$ at positions $\boldsymbol{\zeta}_1, \dots, \boldsymbol{\zeta}_N$ collected in a vector, and with $\mathbf{g}_{\xi} = (g(\boldsymbol{\xi}_1), \dots, g(\boldsymbol{\xi}_M))^{\top}$ the function values at the interpolation nodes $\boldsymbol{\xi}_1, \dots, \boldsymbol{\xi}_M$, we may write
\begin{equation}
    \mathbf{g}_{\zeta} = \mathbf{D}_{NN}^{-1} \boldsymbol{\Phi}_{NM} \boldsymbol{\Phi}_{MM}^{-1} \mathbf{g}_{\xi}
\end{equation}
where the radial basis matrices are defined as
\begin{align}
    (\boldsymbol{\Phi}_{MM})_{ij} &= \phi(\left\| \boldsymbol{\xi}_i - \boldsymbol{\xi}_j \right\|, r_j) \hspace{10px} i, j \in \{1, \dots, M\} \\
    (\boldsymbol{\Phi}_{NM})_{ij} &= \phi(\left\| \boldsymbol{\zeta}_i - \boldsymbol{\xi}_j \right\|, r_j) \hspace{10px} i \in \{1, \dots, N\},~ j \in \{1, \dots, M\}
\end{align}
Therefore, the interpolation matrix is identified as
\begin{equation}
\mathbf{R}_{NM} = \mathbf{D}_{NN}^{-1} \boldsymbol{\Phi}_{NM} \boldsymbol{\Phi}_{MM}^{-1} \label{equ:interpolation-matrix}
\end{equation}

There are two particularities about the above formulation of an interpolation which distinguish it from classical radial basis interpolaion. The two modifications which were proposed in Deparis et. al. \cite{deparis} are:

\begin{itemize}
    \item Localized radius parameters for each node $r_{j}$, $j \in \{1, \dots, M\}$
    \item Rescaling with $\mathbf{D}_{NN}^{-1}$ to recover exact interpolation of constant functions, where we define
    % TODO: Find a simpler expression
    \begin{equation}
        (\mathbf{D}_{NN})_{ij} =
        \begin{cases}
            (\boldsymbol{\Phi}_{NM}\boldsymbol{\Phi}_{MM}^{-1}\boldsymbol{1}_{\zeta})_i, &\text{ if } i = j \\
            0, &\text{ if } i \neq j
        \end{cases}
    \end{equation}
\end{itemize}

These two modifications decisively impact the stability and well-posedness of the interpolation. To achieve a \enquote{good} interpolation, multiple conditions on the radius parameters $r_j$ need to be satisfied. These will be written down and discussed in the next subsection.

\subsection{Radius parameters}
\label{subsec:radius-parameters}

In order to ensure the invertibility of both $\boldsymbol{\Phi}_{MM}$ and $\mathbf{D}_{NN}$ the following conditions derived in \cite{voet} should hold:

\begin{block}{Conditions on radius parameters}
There exist $c \in (0, 1)$ and $C \in (c, 1)$ such that
\begin{align}
    \forall i: \# \{j \neq i: \left\| \boldsymbol{\xi}_i - \boldsymbol{\xi}_j \right\| < r_i \} &< 1 / \phi(c) \label{con:support-condition}  \tag{Condition 1} \\
    \forall i \neq j:~\left\| \boldsymbol{\xi}_i - \boldsymbol{\xi}_j \right\| &\geq c r_j \label{con:c-condition}  \tag{Condition 2}  \\
    \forall i, \exists j:~\left\| \boldsymbol{\zeta}_i - \boldsymbol{\xi}_j \right\| &\leq C r_j \label{con:C-condition}  \tag{Condition 3} 
\end{align}
\end{block}

\refcon{con:support-condition} says that there can only be a limited number of other interpolation nodes within the support of the radial basis function sitting at each interpolation nodes. \refcon{con:c-condition} prohibits two interpolation nodes from being chosen too close to each other. Finally, \refcon{con:C-condition} ensures that at each point at which the interpolant will be evaluated is within the support of an interpolation nodes. 

In \reffig{fig:radiusparameters} these conditions are visualized using examples which violate them.

\begin{figure}[ht]
    \centering
    \begin{subfigure}[b]{.32\linewidth}
        \input{figures/radiusparameters2.tex}
        \caption{Violates \refcon{con:c-condition}.}\label{fig:radiusparameters1}
    \end{subfigure}
    \begin{subfigure}[b]{.32\linewidth}
        \centering
        \begin{tikzpicture}[scale=0.5]
    \fill[darkblue!10!white] (0, 0) circle (0.75);
    \fill[darkblue!30!white] (0, 0) circle (0.35);

    \draw[darkblue, opacity=1] (-2, 1) to (0, 0) to (0.6, -1.1);
    \draw[darkblue, path fading=west, opacity=1] (-3.9, 1.5) to (-2, 1);
    \draw[darkblue, path fading=west, opacity=1] (-3.1, -0.5) to (-2, 1);
    \draw[darkblue, path fading=south, opacity=1] (-1.6, -0.5) to (-2, 1);
    \draw[darkblue, path fading=south, opacity=1] (0, 0) to (-1.4, -0.8);
    \draw[darkblue, path fading=south, opacity=1] (0.6, -1.1) to (1, -2.5);
    \draw[darkblue, path fading=west, opacity=1] (0.6, -1.1) to (-0.8, -1);
    \draw[darkblue, path fading=west, opacity=1] (0.6, -1.1) to (-0.2, -2.4);

    \draw[darkblue, opacity=1] (-0.5, 2) to (0.75, 0.45) to (2, -1);
    \draw[darkblue, path fading=west, opacity=1] (-0.5, 2) to (-1.2, 3);
    \draw[darkblue, path fading=north, opacity=1] (-0.5, 2) to (0, 3);
    \draw[darkblue, path fading=east, opacity=1] (-0.5, 2) to (0.8, 1.9);
    \draw[darkblue, path fading=north, opacity=1] (0.75, 0.45) to (0.9, 1.6);
    \draw[darkblue, path fading=east, opacity=1] (0.75, 0.45) to (2, 0.5);
    \draw[darkblue, path fading=north, opacity=1] (2, -1) to (2.1, 0.3);
    \draw[darkblue, path fading=east, opacity=1] (2, -1) to (3, -1.75);
    \draw[darkblue, path fading=east, opacity=1] (2, -1) to (3, -0.7);

    \fill[darkblue] (0, 0) circle (0.1) node[anchor=north east] {$\boldsymbol{\xi}_{j}$};
    \fill[darkblue] (-2, 1) circle (0.1) node[anchor=north east] {$\boldsymbol{\xi}_{j+1}$};
    \fill[darkblue] (0.6, -1.1) circle (0.1) node[anchor=north east] {$\boldsymbol{\xi}_{j-1}$};
    %\fill[white, draw=darkblue, thick] (0.75, 0.45) circle (0.08) node[anchor=south west, text=darkblue] {$\boldsymbol{\zeta}_{i}$};
    \fill[white, draw=darkblue, thick] (-0.5, 2) circle (0.08) node[anchor=south west, text=darkblue] {$\boldsymbol{\zeta}_{i+1}$};
    \fill[white, draw=darkblue, thick] (2, -1) circle (0.08) node[anchor=south west, text=darkblue] {$\boldsymbol{\zeta}_{i-1}$};

    \draw[white, draw=red, thick] (0.75, 0.45) circle (0.08) node[anchor=south west, text=red] {$\boldsymbol{\zeta}_{i}$};


    %\fill[darkblue!30!white] (3, -3.5) circle (0.4);
    %\fill[darkblue!10!white] (2, -3.5) circle (0.4);
    %\draw[darkblue, thick, <->] (2.0, -3.5) to (2.4, -3.5);
    %\draw[darkblue, thick, <->] (3.0, -3.5) to (3.4, -3.5);
    %\node[darkblue] at (3, -3) {$cr_j$};
    %\node[darkblue] at (2, -3) {$Cr_j$};
\end{tikzpicture}
        \caption{Violates \refcon{con:C-condition}.}\label{fig:radiusparameters2}
    \end{subfigure}
    \begin{subfigure}[b]{.32\linewidth}
        \input{figures/radiusparameters3.tex}
        \caption{Valid parameter choice.}\label{fig:radiusparameters3}
    \end{subfigure}
    \caption{Visualization of the process of finding suitable radius parameters. The inner circle has radius $cr_j$ while the outer one has radius $Cr_j$. Marked in orange are nodes which in each configuration cause one of the conditions to be violated.}
    \label{fig:radiusparameters}
\end{figure}

Our task is now to find suitable radius parameters which satisfy the above conditions. To first simplify the notation and later use it to reduce the complexity of the algorithm we introduce the distance matrix between two sets of nodes $\{\boldsymbol{\xi}_1, \boldsymbol{\xi}_2, \dots\}$
and $\{\boldsymbol{\zeta}_1, \boldsymbol{\zeta}_2, \dots\}$ which we define as

\begin{equation}
    \mathbf{D}^{\boldsymbol{\xi}, \boldsymbol{\zeta}}(i, j) =
        \left\| \boldsymbol{\xi}_i - \boldsymbol{\zeta}_j \right\|
    \label{equ:distance-matrix}
\end{equation}

Making use of this definition, we now propose a procedure, which, given an arbitrary set of interpolation nodes $\{\boldsymbol{\xi}_1, \boldsymbol{\xi}_2, \dots\}$ finds a suitable set of radius parameters $\{r_1, r_2, \dots \}$. The procedure is detailed in \refalg{alg:radiusparameters}.

\begin{algorithm}[H]
    \caption{Computation of radius parameters}
    \begin{algorithmic}[1]
    \Require Positions of interpolation nodes $\{\boldsymbol{\xi}_1, \boldsymbol{\xi}_2, \dots\}$
    \Require Radial basis function $\phi: \mathbb{R} \to \mathbb{R}_{\geq 0}$
    \Require Constant $c \in (0, C)$
    \State Compute distance matrix $\mathbf{D}^{\boldsymbol{\xi}, \boldsymbol{\xi}}$ defined in \refequ{equ:distance-matrix}
    \State For each node $i$, compute minimal distance $d_i = \min_{j \neq i} \mathbf{D}^{\boldsymbol{\xi}, \boldsymbol{\xi}}(i, j)$
    \While{$c \leq C$}
        \State For each node $i$, let $r_i \gets d_i / c$
        \State For each node $i$, count $n_i = \# \{j \neq i: \mathbf{D}^{\boldsymbol{\xi}, \boldsymbol{\xi}}(i, j) < r_i \}$
        \If{For all nodes $i$, $n_i < 1/\phi(c)$}
            \State \textbf{break}
        \EndIf
        \State Increase $c$
    \EndWhile
    \State \textbf{return} Radius parameters $\{r_1, r_2, \dots \}$
\end{algorithmic}
    \label{alg:radiusparameters}
\end{algorithm}

In the setting of finite element approximations for contact mechanics, the interpolation nodes are usually taken to be the mesh points corresponding to the interface between two bodies that are in contact. Consequently, $\{\boldsymbol{\xi}_1, \boldsymbol{\xi}_2, \dots\}$ are the mesh points which correspond to one body, named the \enquote{primary},
and $\{\boldsymbol{\zeta}_1, \boldsymbol{\zeta}_2, \dots\}$ correspond to the other body, named the \enquote{secondary}. Based on \refalg{alg:radiusparameters}, we now propose a second procedure which ensures that for a given contact problem \refcon{con:C-condition} is satisfied. In multiple iterations an index set $\mathcal{I}$ of primary interface nodes and an index set $\mathcal{J}$ of secondary interface nodes are determined by removing nodes from the original set of nodes which are isolated, i.e. not clearly within the support of an interpolation node from the opposite interface. The procedure is given in \refalg{alg:nodesearch}.

\begin{algorithm}[H]
    \caption{Search for interpolation nodes}
    \begin{algorithmic}[1]
    \Require Positions of primary nodes $\{\boldsymbol{\xi}_1, \boldsymbol{\xi}_2, \dots\}$
    \Require Positions of secondary nodes $\{\boldsymbol{\zeta}_1, \boldsymbol{\zeta}_2, \dots\}$
    \Require Radial basis function $\phi: \mathbb{R}^d \to \mathbb{R}_{\geq 0}$
    \Require Constant $C \in (0, 1)$
    \State Let $\mathcal{I} = \{1, 2, \dots \}$ and $\mathcal{J} = \{1, 2, \dots \}$ denote an index set of active nodes
    \State Compute distance matrix $\mathbf{D}^{\boldsymbol{\xi}, \boldsymbol{\zeta}}$ defined in \refequ{equ:distance-matrix}
    \While{$\mathcal{I}$ or $\mathcal{J}$ were modified in the previous iteration}
        \State Obtain radial basis parameters $r^{\boldsymbol{\xi}}_i$, $i \in \mathcal{I}$ and
        $r^{\boldsymbol{\zeta}}_j$, $j \in \mathcal{J}$ using \refalg{alg:radiusparameters}
        \State Remove all isolated nodes $i \in \mathcal{I}$, for which $\min_{j \in \mathcal{J}} \mathbf{D}^{\boldsymbol{\xi}, \boldsymbol{\zeta}}(i, j) / r^{\boldsymbol{\zeta}}_j \geq C$
        \State Remove all isolated nodes $j \in \mathcal{J}$, for which $\min_{i \in \mathcal{I}} \mathbf{D}^{\boldsymbol{\xi}, \boldsymbol{\zeta}}(i, j) / r^{\boldsymbol{\xi}}_i \geq C$
    \EndWhile
    \State \textbf{return} Sets of active nodes $\mathcal{I}$ and $\mathcal{J}$ with radius parameters $r^{\boldsymbol{\xi}}_i, i \in \mathcal{I}$ and $r^{\boldsymbol{\zeta}}_j, j \in \mathcal{J}$
\end{algorithmic}
    \label{alg:nodesearch}
\end{algorithm}

These two algorithms (\refalg{alg:radiusparameters} and \refalg{alg:nodesearch}) hold three main benefits over the implementation found in \cite{voet}:

\begin{itemize}
    \item For uniform meshes (constant element size) all radial basis parameters will be the exact same, i.e. $r^{\boldsymbol{\xi}}_1 = r^{\boldsymbol{\xi}}_2 = \cdots = r^{\boldsymbol{\zeta}}_1 = r^{\boldsymbol{\zeta}}_2 = \cdots $.
    \item The computation of the distance matrix is done outside the \texttt{while}-loop, and hence reduces the compuational complexity by a factor which corresponds to the number of iterations required to find the suitable sets of interpolation nodes
    \item The algorithm will find interface nodes for a wider class of examples.
\end{itemize}

This procedure can further be optimized by introducing a \texttt{SpatialGrid}, which we have done in our implementations in Akantu.

\subsection{Strong form}
\label{subsec:strong}

A stereotypical example of a contact problem which we aim to solve is sketched in \reffig{fig:contactproblem}.

\begin{figure}[H]
    \centering
    \input{figures/contactproblem.tex}
    \caption{Contact problem where a semisphere $\Omega_2$ interfaces with a half-space $\Omega_1$.}\label{fig:contactproblem}
\end{figure}

The strong form of the problem is formulated for the displacement field 
$\mathbf{u}: \Omega_{1, 2} \to \mathbb{R}^d$. Each body participating in the contact needs to individually satisfy the following equilibrium equations \cite{voet}:

\begin{block}{Equilibrium equations}
A solid subject to an external force $\mathbf{f}$ satisfies the following boundary problem
\begin{align}
     \text{div}(\boldsymbol{\sigma}(\mathbf{u})) &= \mathbf{f} \label{con:stress-tensor} \tag{Differential equation with Cauchy stress tensor $\boldsymbol{\sigma}$} \\
    \mathbf{u} &= \mathbf{g} \label{con:dirichlet}  \tag{Dirichlet boundary conditions with displacement field $\mathbf{g}$}  \\
    \boldsymbol{\sigma}(\mathbf{u}) \mathbf{n} &= \mathbf{t} \label{con:neumann}  \tag{Neumann boundary conditions with surface traction $\mathbf{t}$}  \\
    \boldsymbol{\sigma}(\mathbf{u}) \mathbf{n} &= \boldsymbol{\lambda} \label{con:lagrange}  \tag{Lagrange multipliers $\boldsymbol{\lambda}$ defined along interface $\Gamma$}  \\
    \boldsymbol{\lambda} \cdot \mathbf{n} &\leq 0 \label{con:hertz} \tag{Hertz-Signorini-Moreau condition along interface $\Gamma$}
\end{align}
\end{block}

In order to numerically approximate solutions for these problems, a finite element discretization of the domain needs to be computed, and the strong form converted into a weak form of the problem. This will be done in the next section.

\subsection{Weak form}
\label{subsec:weak}

For the weak formulation, the Hertz-Signorini-Moreau inequality constraint is relaxed, which results in a linear system for the displacements $\mathbf{u}$ and Lagrange multipliers $\boldsymbol{\lambda}$ \cite{voet}. The linear system of equations takes the form of a saddle point problem.

\begin{block}{INTERNODES system of equations}
For a contact problem between two bodies $\Omega_1$ and $\Omega_2$ with interfaces $\Gamma_1$ and $\Gamma_2$ the following system of equations is solved:
\begin{equation}
\underbrace{
\begin{bmatrix}
\setlength\arrayrulewidth{.1pt}
\begin{array}{cc|cc|c}
    \mathbf{K}_{\Omega_1\Omega_1} & \mathbf{K}_{\Omega_1\Gamma_1} & & &  \\
    \mathbf{K}_{\Gamma_1\Omega_1} & \mathbf{K}_{\Gamma_1\Gamma_1} & &  & - \mathbf{M}_{\Gamma_1} \\ \hline
     & & \mathbf{K}_{\Omega_2\Omega_2} & \mathbf{K}_{\Omega_2\Gamma_2} &  \\
     & & \mathbf{K}_{\Gamma_2\Omega_2} & \mathbf{K}_{\Gamma_2\Gamma_2} & -\mathbf{M}_{\Gamma_2} \mathbf{R}_{\Gamma_2 \Gamma_1} \\ \hline
     & & & -\mathbf{R}_{\Gamma_1 \Gamma_2} & 
\end{array}
\end{bmatrix}
}_{= \mathbf{A}}
\begin{bmatrix}
\setlength\arrayrulewidth{.1pt}
\begin{array}{c}
    \mathbf{u}_{\Omega_1} \\
    \mathbf{u}_{\Gamma_1} \\ \hline
    \mathbf{u}_{\Omega_2} \\
    \mathbf{u}_{\Gamma_2} \\ \hline
    \boldsymbol{\lambda} \\
\end{array}
\end{bmatrix}
=
\underbrace{
\begin{bmatrix}
\setlength\arrayrulewidth{.1pt}
\begin{array}{c}
    \mathbf{f}_{\Omega_1} \\
    \mathbf{f}_{\Gamma_1} \\ \hline
    \mathbf{f}_{\Omega_2} \\
    \mathbf{f}_{\Gamma_2} \\ \hline
    \mathbf{d} \\
\end{array}
\end{bmatrix}
}_{\mathbf{b}}
\label{equ:linear-system}
\end{equation}
The subscripts indicate which degrees of freedom are collected in the objects. $\mathbf{M}$ and $\mathbf{K}$ denote the finite element mass and stiffness matrices respectively. The interpolation matrices $\mathbf{R}$ are defined in \refequ{equ:interpolation-matrix}. The external force is $\mathbf{f}$ and the nodal gaps between the interfaces is $\mathbf{d}$.
\end{block}

\subsection{INTERNODES Contact algorithm}
\label{subsec:contact-algorithm}

In \cite{voet} an algorithm is suggested for solving problems in contact
mechanics. For an initial guess of the interface $\Gamma$, the algorithm solves 
the system \refequ{equ:linear-system} and in each iteration updates the interface
by removing interface nodes that are in tension or adding nodes that are
interpenetrating to the interface. In all available reference implementations, these two operations of adding and removing interface nodes were treated as mutually exclusive in each iteration \cite{voetthesis, voet, moritz}. In order to speed up the convergence, we have modified the procedure to be able to simultaneously remove nodes from and add back nodes to the interface.

The algorithm is said to have converged, i.e. a solution to the problem has been found, if

\begin{enumerate}
    \item No nodes belonging to the interface are in tension, i.e.
        \begin{equation} \label{equ:convcheck1}
            \boldsymbol{\lambda} \cdot \mathbf{n} \leq 0
        \end{equation}
    \item No interpenetrating nodes exist, i.e. the nodal gaps $\mathbf{d}$ after solving the system all satisfy
        \begin{equation} \label{equ:convcheck2}
            \mathbf{d} \cdot \mathbf{n} \geq 0
        \end{equation}
\end{enumerate}

This procedure is summarized in \refalg{alg:internodes}.

\begin{algorithm}
    \caption{Contact algorithm for interrnodes method}
    \begin{algorithmic}[1]
    \Require Positions of primary nodes $\{\boldsymbol{\xi}_1, \boldsymbol{\xi}_2, \dots\}$
    \Require Positions of secondary nodes $\{\boldsymbol{\zeta}_1, \boldsymbol{\zeta}_2, \dots\}$
    \Require Interface candidate index sets $\mathcal{I}^C$ and $\mathcal{J}^C$
    \While{$\mathcal{I}^C$ or $\mathcal{J}^C$ were modified in the previous iteration}
        \State Determine interface nodes $\mathcal{I}$ and $\mathcal{J}$ with radius parameters $r^{\boldsymbol{\xi}}_i, i \in \mathcal{I}$
        and $r^{\boldsymbol{\zeta}}_j, j \in \mathcal{J}$  using \refalg{alg:nodesearch} on the candidate index sets $\mathcal{I}^C$ and $\mathcal{J}^C$
        \State Assemble the matrix $\mathbf{A}$ and right-hand side $\mathbf{b}$ of \refequ{equ:linear-system}
        \State Solve \refequ{equ:linear-system} to obtain displacements $\mathbf{u}$ and Lagarnge multipliers $\boldsymbol{\lambda}$
        \State Update the interface candidate nodes $\mathcal{I}^C$ and $\mathcal{J}^C$ with
        $\mathcal{I}$ and $\mathcal{J}$, respectively, where all nodes in tension
        have been removed and all interpenetrating nodes have been added 
    \EndWhile
\end{algorithmic}
    \label{alg:internodes}
\end{algorithm}

In order to understand this algorithm more thoroughly, we run the algorithm on an example problem and visualize the different steps in \reffig{fig:example}. Notice that for producing this figure we use a simpler version of the algorithm where the iterations are only used to define the interface nodes. The Python implementation for prototyping uses this version for simplicity. The alternative is to use the \enquote{unphysical} intermediary solutions as the starting points for the next iteration in the algorithm. This is done in the Akantu implementation.

\begin{figure}[H]
    \centering
    \begin{subfigure}[b]{.32\linewidth}
        \centering
        \scalebox{0.8}{%% Creator: Matplotlib, PGF backend
%%
%% To include the figure in your LaTeX document, write
%%   \input{<filename>.pgf}
%%
%% Make sure the required packages are loaded in your preamble
%%   \usepackage{pgf}
%%
%% Also ensure that all the required font packages are loaded; for instance,
%% the lmodern package is sometimes necessary when using math font.
%%   \usepackage{lmodern}
%%
%% Figures using additional raster images can only be included by \input if
%% they are in the same directory as the main LaTeX file. For loading figures
%% from other directories you can use the `import` package
%%   \usepackage{import}
%%
%% and then include the figures with
%%   \import{<path to file>}{<filename>.pgf}
%%
%% Matplotlib used the following preamble
%%   
%%   \usepackage{fontspec}
%%   \setmainfont{DejaVuSans.ttf}[Path=\detokenize{/home/fabio/Internodes-CM/.venv/lib/python3.8/site-packages/matplotlib/mpl-data/fonts/ttf/}]
%%   \setsansfont{DejaVuSans.ttf}[Path=\detokenize{/home/fabio/Internodes-CM/.venv/lib/python3.8/site-packages/matplotlib/mpl-data/fonts/ttf/}]
%%   \setmonofont{DejaVuSansMono.ttf}[Path=\detokenize{/home/fabio/Internodes-CM/.venv/lib/python3.8/site-packages/matplotlib/mpl-data/fonts/ttf/}]
%%   \makeatletter\@ifpackageloaded{underscore}{}{\usepackage[strings]{underscore}}\makeatother
%%
\begingroup%
\makeatletter%
\begin{pgfpicture}%
\pgfpathrectangle{\pgfpointorigin}{\pgfqpoint{1.982500in}{1.432000in}}%
\pgfusepath{use as bounding box, clip}%
\begin{pgfscope}%
\pgfsetbuttcap%
\pgfsetmiterjoin%
\definecolor{currentfill}{rgb}{1.000000,1.000000,1.000000}%
\pgfsetfillcolor{currentfill}%
\pgfsetlinewidth{0.000000pt}%
\definecolor{currentstroke}{rgb}{1.000000,1.000000,1.000000}%
\pgfsetstrokecolor{currentstroke}%
\pgfsetdash{}{0pt}%
\pgfpathmoveto{\pgfqpoint{0.000000in}{0.000000in}}%
\pgfpathlineto{\pgfqpoint{1.982500in}{0.000000in}}%
\pgfpathlineto{\pgfqpoint{1.982500in}{1.432000in}}%
\pgfpathlineto{\pgfqpoint{0.000000in}{1.432000in}}%
\pgfpathlineto{\pgfqpoint{0.000000in}{0.000000in}}%
\pgfpathclose%
\pgfusepath{fill}%
\end{pgfscope}%
\begin{pgfscope}%
\pgfpathrectangle{\pgfqpoint{0.100000in}{0.100000in}}{\pgfqpoint{1.782500in}{1.232000in}}%
\pgfusepath{clip}%
\pgfsetrectcap%
\pgfsetroundjoin%
\pgfsetlinewidth{0.250937pt}%
\definecolor{currentstroke}{rgb}{0.054902,0.262745,0.486275}%
\pgfsetstrokecolor{currentstroke}%
\pgfsetdash{}{0pt}%
\pgfpathmoveto{\pgfqpoint{1.882500in}{0.452000in}}%
\pgfpathlineto{\pgfqpoint{1.892500in}{0.452000in}}%
\pgfpathmoveto{\pgfqpoint{1.729714in}{0.452000in}}%
\pgfpathlineto{\pgfqpoint{1.882500in}{0.452000in}}%
\pgfpathmoveto{\pgfqpoint{1.576929in}{0.452000in}}%
\pgfpathlineto{\pgfqpoint{1.729714in}{0.452000in}}%
\pgfpathmoveto{\pgfqpoint{1.424143in}{0.452000in}}%
\pgfpathlineto{\pgfqpoint{1.576929in}{0.452000in}}%
\pgfpathmoveto{\pgfqpoint{1.271357in}{0.452000in}}%
\pgfpathlineto{\pgfqpoint{1.424143in}{0.452000in}}%
\pgfpathmoveto{\pgfqpoint{1.118571in}{0.452000in}}%
\pgfpathlineto{\pgfqpoint{1.271357in}{0.452000in}}%
\pgfpathmoveto{\pgfqpoint{0.965786in}{0.452000in}}%
\pgfpathlineto{\pgfqpoint{0.813000in}{0.452000in}}%
\pgfpathmoveto{\pgfqpoint{0.965786in}{0.452000in}}%
\pgfpathlineto{\pgfqpoint{1.118571in}{0.452000in}}%
\pgfpathmoveto{\pgfqpoint{0.456500in}{0.452000in}}%
\pgfpathlineto{\pgfqpoint{0.813000in}{0.452000in}}%
\pgfpathmoveto{\pgfqpoint{0.456500in}{0.452000in}}%
\pgfpathlineto{\pgfqpoint{0.100000in}{0.452000in}}%
\pgfpathmoveto{\pgfqpoint{0.100000in}{0.090000in}}%
\pgfpathlineto{\pgfqpoint{0.100000in}{0.452000in}}%
\pgfpathmoveto{\pgfqpoint{1.805208in}{0.175249in}}%
\pgfpathlineto{\pgfqpoint{1.892500in}{0.121802in}}%
\pgfpathmoveto{\pgfqpoint{1.805208in}{0.175249in}}%
\pgfpathlineto{\pgfqpoint{1.637355in}{0.090000in}}%
\pgfpathmoveto{\pgfqpoint{1.195930in}{0.177192in}}%
\pgfpathlineto{\pgfqpoint{1.374206in}{0.090000in}}%
\pgfpathmoveto{\pgfqpoint{1.195930in}{0.177192in}}%
\pgfpathlineto{\pgfqpoint{1.042511in}{0.090000in}}%
\pgfpathmoveto{\pgfqpoint{1.892500in}{0.189285in}}%
\pgfpathlineto{\pgfqpoint{1.805208in}{0.175249in}}%
\pgfpathmoveto{\pgfqpoint{1.501834in}{0.227677in}}%
\pgfpathlineto{\pgfqpoint{1.506188in}{0.090000in}}%
\pgfpathmoveto{\pgfqpoint{1.501834in}{0.227677in}}%
\pgfpathlineto{\pgfqpoint{1.805208in}{0.175249in}}%
\pgfpathmoveto{\pgfqpoint{1.501834in}{0.227677in}}%
\pgfpathlineto{\pgfqpoint{1.195930in}{0.177192in}}%
\pgfpathmoveto{\pgfqpoint{0.856997in}{0.215161in}}%
\pgfpathlineto{\pgfqpoint{0.876107in}{0.090000in}}%
\pgfpathmoveto{\pgfqpoint{0.856997in}{0.215161in}}%
\pgfpathlineto{\pgfqpoint{1.195930in}{0.177192in}}%
\pgfpathmoveto{\pgfqpoint{0.485360in}{0.139977in}}%
\pgfpathlineto{\pgfqpoint{0.100000in}{0.452000in}}%
\pgfpathmoveto{\pgfqpoint{0.485360in}{0.139977in}}%
\pgfpathlineto{\pgfqpoint{0.456500in}{0.452000in}}%
\pgfpathmoveto{\pgfqpoint{0.485360in}{0.139977in}}%
\pgfpathlineto{\pgfqpoint{0.334871in}{0.090000in}}%
\pgfpathmoveto{\pgfqpoint{0.485360in}{0.139977in}}%
\pgfpathlineto{\pgfqpoint{0.632751in}{0.090000in}}%
\pgfpathmoveto{\pgfqpoint{0.485360in}{0.139977in}}%
\pgfpathlineto{\pgfqpoint{0.856997in}{0.215161in}}%
\pgfpathmoveto{\pgfqpoint{1.892500in}{0.434693in}}%
\pgfpathlineto{\pgfqpoint{1.882500in}{0.452000in}}%
\pgfpathmoveto{\pgfqpoint{1.892500in}{0.257344in}}%
\pgfpathlineto{\pgfqpoint{1.805208in}{0.175249in}}%
\pgfpathmoveto{\pgfqpoint{1.653321in}{0.319800in}}%
\pgfpathlineto{\pgfqpoint{1.729714in}{0.452000in}}%
\pgfpathmoveto{\pgfqpoint{1.653321in}{0.319800in}}%
\pgfpathlineto{\pgfqpoint{1.576929in}{0.452000in}}%
\pgfpathmoveto{\pgfqpoint{1.653321in}{0.319800in}}%
\pgfpathlineto{\pgfqpoint{1.805208in}{0.175249in}}%
\pgfpathmoveto{\pgfqpoint{1.653321in}{0.319800in}}%
\pgfpathlineto{\pgfqpoint{1.501834in}{0.227677in}}%
\pgfpathmoveto{\pgfqpoint{1.347750in}{0.319800in}}%
\pgfpathlineto{\pgfqpoint{1.424143in}{0.452000in}}%
\pgfpathmoveto{\pgfqpoint{1.347750in}{0.319800in}}%
\pgfpathlineto{\pgfqpoint{1.271357in}{0.452000in}}%
\pgfpathmoveto{\pgfqpoint{1.347750in}{0.319800in}}%
\pgfpathlineto{\pgfqpoint{1.195930in}{0.177192in}}%
\pgfpathmoveto{\pgfqpoint{1.347750in}{0.319800in}}%
\pgfpathlineto{\pgfqpoint{1.501834in}{0.227677in}}%
\pgfpathmoveto{\pgfqpoint{1.042179in}{0.319800in}}%
\pgfpathlineto{\pgfqpoint{1.118571in}{0.452000in}}%
\pgfpathmoveto{\pgfqpoint{1.042179in}{0.319800in}}%
\pgfpathlineto{\pgfqpoint{0.965786in}{0.452000in}}%
\pgfpathmoveto{\pgfqpoint{1.042179in}{0.319800in}}%
\pgfpathlineto{\pgfqpoint{1.195930in}{0.177192in}}%
\pgfpathmoveto{\pgfqpoint{1.042179in}{0.319800in}}%
\pgfpathlineto{\pgfqpoint{0.856997in}{0.215161in}}%
\pgfpathmoveto{\pgfqpoint{1.793890in}{0.090000in}}%
\pgfpathlineto{\pgfqpoint{1.805208in}{0.175249in}}%
\pgfpathmoveto{\pgfqpoint{0.655483in}{0.306719in}}%
\pgfpathlineto{\pgfqpoint{0.813000in}{0.452000in}}%
\pgfpathmoveto{\pgfqpoint{0.655483in}{0.306719in}}%
\pgfpathlineto{\pgfqpoint{0.456500in}{0.452000in}}%
\pgfpathmoveto{\pgfqpoint{0.655483in}{0.306719in}}%
\pgfpathlineto{\pgfqpoint{0.856997in}{0.215161in}}%
\pgfpathmoveto{\pgfqpoint{0.655483in}{0.306719in}}%
\pgfpathlineto{\pgfqpoint{0.485360in}{0.139977in}}%
\pgfpathmoveto{\pgfqpoint{1.805927in}{0.343767in}}%
\pgfpathlineto{\pgfqpoint{1.882500in}{0.452000in}}%
\pgfpathmoveto{\pgfqpoint{1.805927in}{0.343767in}}%
\pgfpathlineto{\pgfqpoint{1.729714in}{0.452000in}}%
\pgfpathmoveto{\pgfqpoint{1.805927in}{0.343767in}}%
\pgfpathlineto{\pgfqpoint{1.805208in}{0.175249in}}%
\pgfpathmoveto{\pgfqpoint{1.805927in}{0.343767in}}%
\pgfpathlineto{\pgfqpoint{1.892500in}{0.330194in}}%
\pgfpathmoveto{\pgfqpoint{1.805927in}{0.343767in}}%
\pgfpathlineto{\pgfqpoint{1.653321in}{0.319800in}}%
\pgfpathmoveto{\pgfqpoint{1.500668in}{0.351969in}}%
\pgfpathlineto{\pgfqpoint{1.576929in}{0.452000in}}%
\pgfpathmoveto{\pgfqpoint{1.500668in}{0.351969in}}%
\pgfpathlineto{\pgfqpoint{1.424143in}{0.452000in}}%
\pgfpathmoveto{\pgfqpoint{1.500668in}{0.351969in}}%
\pgfpathlineto{\pgfqpoint{1.501834in}{0.227677in}}%
\pgfpathmoveto{\pgfqpoint{1.500668in}{0.351969in}}%
\pgfpathlineto{\pgfqpoint{1.653321in}{0.319800in}}%
\pgfpathmoveto{\pgfqpoint{1.500668in}{0.351969in}}%
\pgfpathlineto{\pgfqpoint{1.347750in}{0.319800in}}%
\pgfpathmoveto{\pgfqpoint{1.206988in}{0.090000in}}%
\pgfpathlineto{\pgfqpoint{1.195930in}{0.177192in}}%
\pgfpathmoveto{\pgfqpoint{1.195157in}{0.344158in}}%
\pgfpathlineto{\pgfqpoint{1.271357in}{0.452000in}}%
\pgfpathmoveto{\pgfqpoint{1.195157in}{0.344158in}}%
\pgfpathlineto{\pgfqpoint{1.118571in}{0.452000in}}%
\pgfpathmoveto{\pgfqpoint{1.195157in}{0.344158in}}%
\pgfpathlineto{\pgfqpoint{1.195930in}{0.177192in}}%
\pgfpathmoveto{\pgfqpoint{1.195157in}{0.344158in}}%
\pgfpathlineto{\pgfqpoint{1.347750in}{0.319800in}}%
\pgfpathmoveto{\pgfqpoint{1.195157in}{0.344158in}}%
\pgfpathlineto{\pgfqpoint{1.042179in}{0.319800in}}%
\pgfpathmoveto{\pgfqpoint{0.889393in}{0.349588in}}%
\pgfpathlineto{\pgfqpoint{0.813000in}{0.452000in}}%
\pgfpathmoveto{\pgfqpoint{0.889393in}{0.349588in}}%
\pgfpathlineto{\pgfqpoint{0.965786in}{0.452000in}}%
\pgfpathmoveto{\pgfqpoint{0.889393in}{0.349588in}}%
\pgfpathlineto{\pgfqpoint{0.856997in}{0.215161in}}%
\pgfpathmoveto{\pgfqpoint{0.889393in}{0.349588in}}%
\pgfpathlineto{\pgfqpoint{1.042179in}{0.319800in}}%
\pgfpathmoveto{\pgfqpoint{0.889393in}{0.349588in}}%
\pgfpathlineto{\pgfqpoint{0.655483in}{0.306719in}}%
\pgfpathmoveto{\pgfqpoint{0.494080in}{0.090000in}}%
\pgfpathlineto{\pgfqpoint{0.485360in}{0.139977in}}%
\pgfpathlineto{\pgfqpoint{0.485360in}{0.139977in}}%
\pgfusepath{stroke}%
\end{pgfscope}%
\begin{pgfscope}%
\pgfpathrectangle{\pgfqpoint{0.100000in}{0.100000in}}{\pgfqpoint{1.782500in}{1.232000in}}%
\pgfusepath{clip}%
\pgfsetrectcap%
\pgfsetroundjoin%
\pgfsetlinewidth{0.250937pt}%
\definecolor{currentstroke}{rgb}{0.835294,0.321569,0.035294}%
\pgfsetstrokecolor{currentstroke}%
\pgfsetdash{}{0pt}%
\pgfpathmoveto{\pgfqpoint{0.813000in}{0.860417in}}%
\pgfpathlineto{\pgfqpoint{0.694167in}{1.244000in}}%
\pgfpathmoveto{\pgfqpoint{1.288333in}{1.244000in}}%
\pgfpathlineto{\pgfqpoint{1.882500in}{1.244000in}}%
\pgfpathmoveto{\pgfqpoint{1.288333in}{1.244000in}}%
\pgfpathlineto{\pgfqpoint{0.694167in}{1.244000in}}%
\pgfpathmoveto{\pgfqpoint{0.909162in}{0.739160in}}%
\pgfpathlineto{\pgfqpoint{0.813000in}{0.860417in}}%
\pgfpathmoveto{\pgfqpoint{1.030178in}{0.630794in}}%
\pgfpathlineto{\pgfqpoint{0.909162in}{0.739160in}}%
\pgfpathmoveto{\pgfqpoint{1.172958in}{0.538087in}}%
\pgfpathlineto{\pgfqpoint{1.030178in}{0.630794in}}%
\pgfpathmoveto{\pgfqpoint{1.333857in}{0.463404in}}%
\pgfpathlineto{\pgfqpoint{1.172958in}{0.538087in}}%
\pgfpathmoveto{\pgfqpoint{1.508764in}{0.408655in}}%
\pgfpathlineto{\pgfqpoint{1.333857in}{0.463404in}}%
\pgfpathmoveto{\pgfqpoint{1.693215in}{0.375235in}}%
\pgfpathlineto{\pgfqpoint{1.508764in}{0.408655in}}%
\pgfpathmoveto{\pgfqpoint{1.882500in}{0.364000in}}%
\pgfpathlineto{\pgfqpoint{1.693215in}{0.375235in}}%
\pgfpathmoveto{\pgfqpoint{1.892500in}{0.364594in}}%
\pgfpathlineto{\pgfqpoint{1.882500in}{0.364000in}}%
\pgfpathmoveto{\pgfqpoint{1.892500in}{1.244000in}}%
\pgfpathlineto{\pgfqpoint{1.882500in}{1.244000in}}%
\pgfpathmoveto{\pgfqpoint{1.544975in}{0.898385in}}%
\pgfpathlineto{\pgfqpoint{1.882500in}{1.244000in}}%
\pgfpathmoveto{\pgfqpoint{1.544975in}{0.898385in}}%
\pgfpathlineto{\pgfqpoint{1.288333in}{1.244000in}}%
\pgfpathmoveto{\pgfqpoint{1.817713in}{0.645445in}}%
\pgfpathlineto{\pgfqpoint{1.892500in}{0.675390in}}%
\pgfpathmoveto{\pgfqpoint{1.817713in}{0.645445in}}%
\pgfpathlineto{\pgfqpoint{1.544975in}{0.898385in}}%
\pgfpathmoveto{\pgfqpoint{1.186381in}{0.929127in}}%
\pgfpathlineto{\pgfqpoint{1.288333in}{1.244000in}}%
\pgfpathmoveto{\pgfqpoint{1.186381in}{0.929127in}}%
\pgfpathlineto{\pgfqpoint{1.544975in}{0.898385in}}%
\pgfpathmoveto{\pgfqpoint{1.892500in}{0.628836in}}%
\pgfpathlineto{\pgfqpoint{1.817713in}{0.645445in}}%
\pgfpathmoveto{\pgfqpoint{1.511108in}{0.615050in}}%
\pgfpathlineto{\pgfqpoint{1.333857in}{0.463404in}}%
\pgfpathmoveto{\pgfqpoint{1.511108in}{0.615050in}}%
\pgfpathlineto{\pgfqpoint{1.508764in}{0.408655in}}%
\pgfpathmoveto{\pgfqpoint{1.511108in}{0.615050in}}%
\pgfpathlineto{\pgfqpoint{1.544975in}{0.898385in}}%
\pgfpathmoveto{\pgfqpoint{1.511108in}{0.615050in}}%
\pgfpathlineto{\pgfqpoint{1.817713in}{0.645445in}}%
\pgfpathmoveto{\pgfqpoint{1.278705in}{0.730225in}}%
\pgfpathlineto{\pgfqpoint{1.030178in}{0.630794in}}%
\pgfpathmoveto{\pgfqpoint{1.278705in}{0.730225in}}%
\pgfpathlineto{\pgfqpoint{1.172958in}{0.538087in}}%
\pgfpathmoveto{\pgfqpoint{1.278705in}{0.730225in}}%
\pgfpathlineto{\pgfqpoint{1.544975in}{0.898385in}}%
\pgfpathmoveto{\pgfqpoint{1.278705in}{0.730225in}}%
\pgfpathlineto{\pgfqpoint{1.186381in}{0.929127in}}%
\pgfpathmoveto{\pgfqpoint{1.278705in}{0.730225in}}%
\pgfpathlineto{\pgfqpoint{1.511108in}{0.615050in}}%
\pgfpathmoveto{\pgfqpoint{1.798690in}{0.469691in}}%
\pgfpathlineto{\pgfqpoint{1.693215in}{0.375235in}}%
\pgfpathmoveto{\pgfqpoint{1.798690in}{0.469691in}}%
\pgfpathlineto{\pgfqpoint{1.882500in}{0.364000in}}%
\pgfpathmoveto{\pgfqpoint{1.798690in}{0.469691in}}%
\pgfpathlineto{\pgfqpoint{1.817713in}{0.645445in}}%
\pgfpathmoveto{\pgfqpoint{1.334493in}{0.597404in}}%
\pgfpathlineto{\pgfqpoint{1.172958in}{0.538087in}}%
\pgfpathmoveto{\pgfqpoint{1.334493in}{0.597404in}}%
\pgfpathlineto{\pgfqpoint{1.333857in}{0.463404in}}%
\pgfpathmoveto{\pgfqpoint{1.334493in}{0.597404in}}%
\pgfpathlineto{\pgfqpoint{1.511108in}{0.615050in}}%
\pgfpathmoveto{\pgfqpoint{1.334493in}{0.597404in}}%
\pgfpathlineto{\pgfqpoint{1.278705in}{0.730225in}}%
\pgfpathmoveto{\pgfqpoint{1.095926in}{0.762297in}}%
\pgfpathlineto{\pgfqpoint{0.909162in}{0.739160in}}%
\pgfpathmoveto{\pgfqpoint{1.095926in}{0.762297in}}%
\pgfpathlineto{\pgfqpoint{1.030178in}{0.630794in}}%
\pgfpathmoveto{\pgfqpoint{1.095926in}{0.762297in}}%
\pgfpathlineto{\pgfqpoint{1.186381in}{0.929127in}}%
\pgfpathmoveto{\pgfqpoint{1.095926in}{0.762297in}}%
\pgfpathlineto{\pgfqpoint{1.278705in}{0.730225in}}%
\pgfpathmoveto{\pgfqpoint{0.960375in}{1.053168in}}%
\pgfpathlineto{\pgfqpoint{0.694167in}{1.244000in}}%
\pgfpathmoveto{\pgfqpoint{0.960375in}{1.053168in}}%
\pgfpathlineto{\pgfqpoint{0.813000in}{0.860417in}}%
\pgfpathmoveto{\pgfqpoint{0.960375in}{1.053168in}}%
\pgfpathlineto{\pgfqpoint{1.288333in}{1.244000in}}%
\pgfpathmoveto{\pgfqpoint{0.960375in}{1.053168in}}%
\pgfpathlineto{\pgfqpoint{1.186381in}{0.929127in}}%
\pgfpathmoveto{\pgfqpoint{1.892500in}{0.376929in}}%
\pgfpathlineto{\pgfqpoint{1.882500in}{0.364000in}}%
\pgfpathmoveto{\pgfqpoint{1.892500in}{0.558019in}}%
\pgfpathlineto{\pgfqpoint{1.817713in}{0.645445in}}%
\pgfpathmoveto{\pgfqpoint{1.892500in}{0.471002in}}%
\pgfpathlineto{\pgfqpoint{1.798690in}{0.469691in}}%
\pgfpathmoveto{\pgfqpoint{1.633962in}{0.491742in}}%
\pgfpathlineto{\pgfqpoint{1.508764in}{0.408655in}}%
\pgfpathmoveto{\pgfqpoint{1.633962in}{0.491742in}}%
\pgfpathlineto{\pgfqpoint{1.693215in}{0.375235in}}%
\pgfpathmoveto{\pgfqpoint{1.633962in}{0.491742in}}%
\pgfpathlineto{\pgfqpoint{1.817713in}{0.645445in}}%
\pgfpathmoveto{\pgfqpoint{1.633962in}{0.491742in}}%
\pgfpathlineto{\pgfqpoint{1.511108in}{0.615050in}}%
\pgfpathmoveto{\pgfqpoint{1.633962in}{0.491742in}}%
\pgfpathlineto{\pgfqpoint{1.798690in}{0.469691in}}%
\pgfpathmoveto{\pgfqpoint{1.892500in}{1.184050in}}%
\pgfpathlineto{\pgfqpoint{1.882500in}{1.244000in}}%
\pgfpathmoveto{\pgfqpoint{1.892500in}{0.919808in}}%
\pgfpathlineto{\pgfqpoint{1.544975in}{0.898385in}}%
\pgfpathmoveto{\pgfqpoint{1.892500in}{0.820421in}}%
\pgfpathlineto{\pgfqpoint{1.817713in}{0.645445in}}%
\pgfpathmoveto{\pgfqpoint{1.892500in}{1.238184in}}%
\pgfpathlineto{\pgfqpoint{1.882500in}{1.244000in}}%
\pgfpathmoveto{\pgfqpoint{0.992969in}{0.868834in}}%
\pgfpathlineto{\pgfqpoint{0.813000in}{0.860417in}}%
\pgfpathmoveto{\pgfqpoint{0.992969in}{0.868834in}}%
\pgfpathlineto{\pgfqpoint{0.909162in}{0.739160in}}%
\pgfpathmoveto{\pgfqpoint{0.992969in}{0.868834in}}%
\pgfpathlineto{\pgfqpoint{1.186381in}{0.929127in}}%
\pgfpathmoveto{\pgfqpoint{0.992969in}{0.868834in}}%
\pgfpathlineto{\pgfqpoint{1.095926in}{0.762297in}}%
\pgfpathmoveto{\pgfqpoint{0.992969in}{0.868834in}}%
\pgfpathlineto{\pgfqpoint{0.960375in}{1.053168in}}%
\pgfpathlineto{\pgfqpoint{0.960375in}{1.053168in}}%
\pgfusepath{stroke}%
\end{pgfscope}%
\begin{pgfscope}%
\pgfpathrectangle{\pgfqpoint{0.100000in}{0.100000in}}{\pgfqpoint{1.782500in}{1.232000in}}%
\pgfusepath{clip}%
\pgfsetbuttcap%
\pgfsetroundjoin%
\pgfsetlinewidth{0.501875pt}%
\definecolor{currentstroke}{rgb}{0.054902,0.262745,0.486275}%
\pgfsetstrokecolor{currentstroke}%
\pgfsetdash{}{0pt}%
\pgfpathmoveto{\pgfqpoint{2.952000in}{0.433627in}}%
\pgfpathcurveto{\pgfqpoint{2.956873in}{0.433627in}}{\pgfqpoint{2.961546in}{0.435563in}}{\pgfqpoint{2.964992in}{0.439008in}}%
\pgfpathcurveto{\pgfqpoint{2.968437in}{0.442454in}}{\pgfqpoint{2.970373in}{0.447127in}}{\pgfqpoint{2.970373in}{0.452000in}}%
\pgfpathcurveto{\pgfqpoint{2.970373in}{0.456873in}}{\pgfqpoint{2.968437in}{0.461546in}}{\pgfqpoint{2.964992in}{0.464992in}}%
\pgfpathcurveto{\pgfqpoint{2.961546in}{0.468437in}}{\pgfqpoint{2.956873in}{0.470373in}}{\pgfqpoint{2.952000in}{0.470373in}}%
\pgfpathcurveto{\pgfqpoint{2.947127in}{0.470373in}}{\pgfqpoint{2.942454in}{0.468437in}}{\pgfqpoint{2.939008in}{0.464992in}}%
\pgfpathcurveto{\pgfqpoint{2.935563in}{0.461546in}}{\pgfqpoint{2.933627in}{0.456873in}}{\pgfqpoint{2.933627in}{0.452000in}}%
\pgfpathcurveto{\pgfqpoint{2.933627in}{0.447127in}}{\pgfqpoint{2.935563in}{0.442454in}}{\pgfqpoint{2.939008in}{0.439008in}}%
\pgfpathcurveto{\pgfqpoint{2.942454in}{0.435563in}}{\pgfqpoint{2.947127in}{0.433627in}}{\pgfqpoint{2.952000in}{0.433627in}}%
\pgfusepath{stroke}%
\end{pgfscope}%
\begin{pgfscope}%
\pgfpathrectangle{\pgfqpoint{0.100000in}{0.100000in}}{\pgfqpoint{1.782500in}{1.232000in}}%
\pgfusepath{clip}%
\pgfsetbuttcap%
\pgfsetroundjoin%
\pgfsetlinewidth{0.501875pt}%
\definecolor{currentstroke}{rgb}{0.054902,0.262745,0.486275}%
\pgfsetstrokecolor{currentstroke}%
\pgfsetdash{}{0pt}%
\pgfpathmoveto{\pgfqpoint{0.813000in}{0.433627in}}%
\pgfpathcurveto{\pgfqpoint{0.817873in}{0.433627in}}{\pgfqpoint{0.822546in}{0.435563in}}{\pgfqpoint{0.825992in}{0.439008in}}%
\pgfpathcurveto{\pgfqpoint{0.829437in}{0.442454in}}{\pgfqpoint{0.831373in}{0.447127in}}{\pgfqpoint{0.831373in}{0.452000in}}%
\pgfpathcurveto{\pgfqpoint{0.831373in}{0.456873in}}{\pgfqpoint{0.829437in}{0.461546in}}{\pgfqpoint{0.825992in}{0.464992in}}%
\pgfpathcurveto{\pgfqpoint{0.822546in}{0.468437in}}{\pgfqpoint{0.817873in}{0.470373in}}{\pgfqpoint{0.813000in}{0.470373in}}%
\pgfpathcurveto{\pgfqpoint{0.808127in}{0.470373in}}{\pgfqpoint{0.803454in}{0.468437in}}{\pgfqpoint{0.800008in}{0.464992in}}%
\pgfpathcurveto{\pgfqpoint{0.796563in}{0.461546in}}{\pgfqpoint{0.794627in}{0.456873in}}{\pgfqpoint{0.794627in}{0.452000in}}%
\pgfpathcurveto{\pgfqpoint{0.794627in}{0.447127in}}{\pgfqpoint{0.796563in}{0.442454in}}{\pgfqpoint{0.800008in}{0.439008in}}%
\pgfpathcurveto{\pgfqpoint{0.803454in}{0.435563in}}{\pgfqpoint{0.808127in}{0.433627in}}{\pgfqpoint{0.813000in}{0.433627in}}%
\pgfpathlineto{\pgfqpoint{0.813000in}{0.433627in}}%
\pgfpathclose%
\pgfusepath{stroke}%
\end{pgfscope}%
\begin{pgfscope}%
\pgfpathrectangle{\pgfqpoint{0.100000in}{0.100000in}}{\pgfqpoint{1.782500in}{1.232000in}}%
\pgfusepath{clip}%
\pgfsetbuttcap%
\pgfsetroundjoin%
\pgfsetlinewidth{0.501875pt}%
\definecolor{currentstroke}{rgb}{0.054902,0.262745,0.486275}%
\pgfsetstrokecolor{currentstroke}%
\pgfsetdash{}{0pt}%
\pgfpathmoveto{\pgfqpoint{2.799214in}{0.433627in}}%
\pgfpathcurveto{\pgfqpoint{2.804087in}{0.433627in}}{\pgfqpoint{2.808761in}{0.435563in}}{\pgfqpoint{2.812206in}{0.439008in}}%
\pgfpathcurveto{\pgfqpoint{2.815652in}{0.442454in}}{\pgfqpoint{2.817588in}{0.447127in}}{\pgfqpoint{2.817588in}{0.452000in}}%
\pgfpathcurveto{\pgfqpoint{2.817588in}{0.456873in}}{\pgfqpoint{2.815652in}{0.461546in}}{\pgfqpoint{2.812206in}{0.464992in}}%
\pgfpathcurveto{\pgfqpoint{2.808761in}{0.468437in}}{\pgfqpoint{2.804087in}{0.470373in}}{\pgfqpoint{2.799214in}{0.470373in}}%
\pgfpathcurveto{\pgfqpoint{2.794342in}{0.470373in}}{\pgfqpoint{2.789668in}{0.468437in}}{\pgfqpoint{2.786222in}{0.464992in}}%
\pgfpathcurveto{\pgfqpoint{2.782777in}{0.461546in}}{\pgfqpoint{2.780841in}{0.456873in}}{\pgfqpoint{2.780841in}{0.452000in}}%
\pgfpathcurveto{\pgfqpoint{2.780841in}{0.447127in}}{\pgfqpoint{2.782777in}{0.442454in}}{\pgfqpoint{2.786222in}{0.439008in}}%
\pgfpathcurveto{\pgfqpoint{2.789668in}{0.435563in}}{\pgfqpoint{2.794342in}{0.433627in}}{\pgfqpoint{2.799214in}{0.433627in}}%
\pgfusepath{stroke}%
\end{pgfscope}%
\begin{pgfscope}%
\pgfpathrectangle{\pgfqpoint{0.100000in}{0.100000in}}{\pgfqpoint{1.782500in}{1.232000in}}%
\pgfusepath{clip}%
\pgfsetbuttcap%
\pgfsetroundjoin%
\pgfsetlinewidth{0.501875pt}%
\definecolor{currentstroke}{rgb}{0.054902,0.262745,0.486275}%
\pgfsetstrokecolor{currentstroke}%
\pgfsetdash{}{0pt}%
\pgfpathmoveto{\pgfqpoint{2.646429in}{0.433627in}}%
\pgfpathcurveto{\pgfqpoint{2.651301in}{0.433627in}}{\pgfqpoint{2.655975in}{0.435563in}}{\pgfqpoint{2.659420in}{0.439008in}}%
\pgfpathcurveto{\pgfqpoint{2.662866in}{0.442454in}}{\pgfqpoint{2.664802in}{0.447127in}}{\pgfqpoint{2.664802in}{0.452000in}}%
\pgfpathcurveto{\pgfqpoint{2.664802in}{0.456873in}}{\pgfqpoint{2.662866in}{0.461546in}}{\pgfqpoint{2.659420in}{0.464992in}}%
\pgfpathcurveto{\pgfqpoint{2.655975in}{0.468437in}}{\pgfqpoint{2.651301in}{0.470373in}}{\pgfqpoint{2.646429in}{0.470373in}}%
\pgfpathcurveto{\pgfqpoint{2.641556in}{0.470373in}}{\pgfqpoint{2.636882in}{0.468437in}}{\pgfqpoint{2.633437in}{0.464992in}}%
\pgfpathcurveto{\pgfqpoint{2.629991in}{0.461546in}}{\pgfqpoint{2.628055in}{0.456873in}}{\pgfqpoint{2.628055in}{0.452000in}}%
\pgfpathcurveto{\pgfqpoint{2.628055in}{0.447127in}}{\pgfqpoint{2.629991in}{0.442454in}}{\pgfqpoint{2.633437in}{0.439008in}}%
\pgfpathcurveto{\pgfqpoint{2.636882in}{0.435563in}}{\pgfqpoint{2.641556in}{0.433627in}}{\pgfqpoint{2.646429in}{0.433627in}}%
\pgfusepath{stroke}%
\end{pgfscope}%
\begin{pgfscope}%
\pgfpathrectangle{\pgfqpoint{0.100000in}{0.100000in}}{\pgfqpoint{1.782500in}{1.232000in}}%
\pgfusepath{clip}%
\pgfsetbuttcap%
\pgfsetroundjoin%
\pgfsetlinewidth{0.501875pt}%
\definecolor{currentstroke}{rgb}{0.054902,0.262745,0.486275}%
\pgfsetstrokecolor{currentstroke}%
\pgfsetdash{}{0pt}%
\pgfpathmoveto{\pgfqpoint{2.493643in}{0.433627in}}%
\pgfpathcurveto{\pgfqpoint{2.498516in}{0.433627in}}{\pgfqpoint{2.503189in}{0.435563in}}{\pgfqpoint{2.506635in}{0.439008in}}%
\pgfpathcurveto{\pgfqpoint{2.510080in}{0.442454in}}{\pgfqpoint{2.512016in}{0.447127in}}{\pgfqpoint{2.512016in}{0.452000in}}%
\pgfpathcurveto{\pgfqpoint{2.512016in}{0.456873in}}{\pgfqpoint{2.510080in}{0.461546in}}{\pgfqpoint{2.506635in}{0.464992in}}%
\pgfpathcurveto{\pgfqpoint{2.503189in}{0.468437in}}{\pgfqpoint{2.498516in}{0.470373in}}{\pgfqpoint{2.493643in}{0.470373in}}%
\pgfpathcurveto{\pgfqpoint{2.488770in}{0.470373in}}{\pgfqpoint{2.484096in}{0.468437in}}{\pgfqpoint{2.480651in}{0.464992in}}%
\pgfpathcurveto{\pgfqpoint{2.477206in}{0.461546in}}{\pgfqpoint{2.475270in}{0.456873in}}{\pgfqpoint{2.475270in}{0.452000in}}%
\pgfpathcurveto{\pgfqpoint{2.475270in}{0.447127in}}{\pgfqpoint{2.477206in}{0.442454in}}{\pgfqpoint{2.480651in}{0.439008in}}%
\pgfpathcurveto{\pgfqpoint{2.484096in}{0.435563in}}{\pgfqpoint{2.488770in}{0.433627in}}{\pgfqpoint{2.493643in}{0.433627in}}%
\pgfusepath{stroke}%
\end{pgfscope}%
\begin{pgfscope}%
\pgfpathrectangle{\pgfqpoint{0.100000in}{0.100000in}}{\pgfqpoint{1.782500in}{1.232000in}}%
\pgfusepath{clip}%
\pgfsetbuttcap%
\pgfsetroundjoin%
\pgfsetlinewidth{0.501875pt}%
\definecolor{currentstroke}{rgb}{0.054902,0.262745,0.486275}%
\pgfsetstrokecolor{currentstroke}%
\pgfsetdash{}{0pt}%
\pgfpathmoveto{\pgfqpoint{2.340857in}{0.433627in}}%
\pgfpathcurveto{\pgfqpoint{2.345730in}{0.433627in}}{\pgfqpoint{2.350404in}{0.435563in}}{\pgfqpoint{2.353849in}{0.439008in}}%
\pgfpathcurveto{\pgfqpoint{2.357294in}{0.442454in}}{\pgfqpoint{2.359230in}{0.447127in}}{\pgfqpoint{2.359230in}{0.452000in}}%
\pgfpathcurveto{\pgfqpoint{2.359230in}{0.456873in}}{\pgfqpoint{2.357294in}{0.461546in}}{\pgfqpoint{2.353849in}{0.464992in}}%
\pgfpathcurveto{\pgfqpoint{2.350404in}{0.468437in}}{\pgfqpoint{2.345730in}{0.470373in}}{\pgfqpoint{2.340857in}{0.470373in}}%
\pgfpathcurveto{\pgfqpoint{2.335984in}{0.470373in}}{\pgfqpoint{2.331311in}{0.468437in}}{\pgfqpoint{2.327865in}{0.464992in}}%
\pgfpathcurveto{\pgfqpoint{2.324420in}{0.461546in}}{\pgfqpoint{2.322484in}{0.456873in}}{\pgfqpoint{2.322484in}{0.452000in}}%
\pgfpathcurveto{\pgfqpoint{2.322484in}{0.447127in}}{\pgfqpoint{2.324420in}{0.442454in}}{\pgfqpoint{2.327865in}{0.439008in}}%
\pgfpathcurveto{\pgfqpoint{2.331311in}{0.435563in}}{\pgfqpoint{2.335984in}{0.433627in}}{\pgfqpoint{2.340857in}{0.433627in}}%
\pgfusepath{stroke}%
\end{pgfscope}%
\begin{pgfscope}%
\pgfpathrectangle{\pgfqpoint{0.100000in}{0.100000in}}{\pgfqpoint{1.782500in}{1.232000in}}%
\pgfusepath{clip}%
\pgfsetbuttcap%
\pgfsetroundjoin%
\pgfsetlinewidth{0.501875pt}%
\definecolor{currentstroke}{rgb}{0.054902,0.262745,0.486275}%
\pgfsetstrokecolor{currentstroke}%
\pgfsetdash{}{0pt}%
\pgfpathmoveto{\pgfqpoint{2.188071in}{0.433627in}}%
\pgfpathcurveto{\pgfqpoint{2.192944in}{0.433627in}}{\pgfqpoint{2.197618in}{0.435563in}}{\pgfqpoint{2.201063in}{0.439008in}}%
\pgfpathcurveto{\pgfqpoint{2.204509in}{0.442454in}}{\pgfqpoint{2.206445in}{0.447127in}}{\pgfqpoint{2.206445in}{0.452000in}}%
\pgfpathcurveto{\pgfqpoint{2.206445in}{0.456873in}}{\pgfqpoint{2.204509in}{0.461546in}}{\pgfqpoint{2.201063in}{0.464992in}}%
\pgfpathcurveto{\pgfqpoint{2.197618in}{0.468437in}}{\pgfqpoint{2.192944in}{0.470373in}}{\pgfqpoint{2.188071in}{0.470373in}}%
\pgfpathcurveto{\pgfqpoint{2.183199in}{0.470373in}}{\pgfqpoint{2.178525in}{0.468437in}}{\pgfqpoint{2.175080in}{0.464992in}}%
\pgfpathcurveto{\pgfqpoint{2.171634in}{0.461546in}}{\pgfqpoint{2.169698in}{0.456873in}}{\pgfqpoint{2.169698in}{0.452000in}}%
\pgfpathcurveto{\pgfqpoint{2.169698in}{0.447127in}}{\pgfqpoint{2.171634in}{0.442454in}}{\pgfqpoint{2.175080in}{0.439008in}}%
\pgfpathcurveto{\pgfqpoint{2.178525in}{0.435563in}}{\pgfqpoint{2.183199in}{0.433627in}}{\pgfqpoint{2.188071in}{0.433627in}}%
\pgfusepath{stroke}%
\end{pgfscope}%
\begin{pgfscope}%
\pgfpathrectangle{\pgfqpoint{0.100000in}{0.100000in}}{\pgfqpoint{1.782500in}{1.232000in}}%
\pgfusepath{clip}%
\pgfsetbuttcap%
\pgfsetroundjoin%
\pgfsetlinewidth{0.501875pt}%
\definecolor{currentstroke}{rgb}{0.054902,0.262745,0.486275}%
\pgfsetstrokecolor{currentstroke}%
\pgfsetdash{}{0pt}%
\pgfpathmoveto{\pgfqpoint{2.035286in}{0.433627in}}%
\pgfpathcurveto{\pgfqpoint{2.040158in}{0.433627in}}{\pgfqpoint{2.044832in}{0.435563in}}{\pgfqpoint{2.048278in}{0.439008in}}%
\pgfpathcurveto{\pgfqpoint{2.051723in}{0.442454in}}{\pgfqpoint{2.053659in}{0.447127in}}{\pgfqpoint{2.053659in}{0.452000in}}%
\pgfpathcurveto{\pgfqpoint{2.053659in}{0.456873in}}{\pgfqpoint{2.051723in}{0.461546in}}{\pgfqpoint{2.048278in}{0.464992in}}%
\pgfpathcurveto{\pgfqpoint{2.044832in}{0.468437in}}{\pgfqpoint{2.040158in}{0.470373in}}{\pgfqpoint{2.035286in}{0.470373in}}%
\pgfpathcurveto{\pgfqpoint{2.030413in}{0.470373in}}{\pgfqpoint{2.025739in}{0.468437in}}{\pgfqpoint{2.022294in}{0.464992in}}%
\pgfpathcurveto{\pgfqpoint{2.018848in}{0.461546in}}{\pgfqpoint{2.016912in}{0.456873in}}{\pgfqpoint{2.016912in}{0.452000in}}%
\pgfpathcurveto{\pgfqpoint{2.016912in}{0.447127in}}{\pgfqpoint{2.018848in}{0.442454in}}{\pgfqpoint{2.022294in}{0.439008in}}%
\pgfpathcurveto{\pgfqpoint{2.025739in}{0.435563in}}{\pgfqpoint{2.030413in}{0.433627in}}{\pgfqpoint{2.035286in}{0.433627in}}%
\pgfusepath{stroke}%
\end{pgfscope}%
\begin{pgfscope}%
\pgfpathrectangle{\pgfqpoint{0.100000in}{0.100000in}}{\pgfqpoint{1.782500in}{1.232000in}}%
\pgfusepath{clip}%
\pgfsetbuttcap%
\pgfsetroundjoin%
\pgfsetlinewidth{0.501875pt}%
\definecolor{currentstroke}{rgb}{0.054902,0.262745,0.486275}%
\pgfsetstrokecolor{currentstroke}%
\pgfsetdash{}{0pt}%
\pgfpathmoveto{\pgfqpoint{1.882500in}{0.433627in}}%
\pgfpathcurveto{\pgfqpoint{1.887373in}{0.433627in}}{\pgfqpoint{1.892046in}{0.435563in}}{\pgfqpoint{1.895492in}{0.439008in}}%
\pgfpathcurveto{\pgfqpoint{1.898937in}{0.442454in}}{\pgfqpoint{1.900873in}{0.447127in}}{\pgfqpoint{1.900873in}{0.452000in}}%
\pgfpathcurveto{\pgfqpoint{1.900873in}{0.456873in}}{\pgfqpoint{1.898937in}{0.461546in}}{\pgfqpoint{1.895492in}{0.464992in}}%
\pgfpathcurveto{\pgfqpoint{1.892046in}{0.468437in}}{\pgfqpoint{1.887373in}{0.470373in}}{\pgfqpoint{1.882500in}{0.470373in}}%
\pgfpathcurveto{\pgfqpoint{1.877627in}{0.470373in}}{\pgfqpoint{1.872954in}{0.468437in}}{\pgfqpoint{1.869508in}{0.464992in}}%
\pgfpathcurveto{\pgfqpoint{1.866063in}{0.461546in}}{\pgfqpoint{1.864127in}{0.456873in}}{\pgfqpoint{1.864127in}{0.452000in}}%
\pgfpathcurveto{\pgfqpoint{1.864127in}{0.447127in}}{\pgfqpoint{1.866063in}{0.442454in}}{\pgfqpoint{1.869508in}{0.439008in}}%
\pgfpathcurveto{\pgfqpoint{1.872954in}{0.435563in}}{\pgfqpoint{1.877627in}{0.433627in}}{\pgfqpoint{1.882500in}{0.433627in}}%
\pgfpathlineto{\pgfqpoint{1.882500in}{0.433627in}}%
\pgfpathclose%
\pgfusepath{stroke}%
\end{pgfscope}%
\begin{pgfscope}%
\pgfpathrectangle{\pgfqpoint{0.100000in}{0.100000in}}{\pgfqpoint{1.782500in}{1.232000in}}%
\pgfusepath{clip}%
\pgfsetbuttcap%
\pgfsetroundjoin%
\pgfsetlinewidth{0.501875pt}%
\definecolor{currentstroke}{rgb}{0.054902,0.262745,0.486275}%
\pgfsetstrokecolor{currentstroke}%
\pgfsetdash{}{0pt}%
\pgfpathmoveto{\pgfqpoint{1.729714in}{0.433627in}}%
\pgfpathcurveto{\pgfqpoint{1.734587in}{0.433627in}}{\pgfqpoint{1.739261in}{0.435563in}}{\pgfqpoint{1.742706in}{0.439008in}}%
\pgfpathcurveto{\pgfqpoint{1.746152in}{0.442454in}}{\pgfqpoint{1.748088in}{0.447127in}}{\pgfqpoint{1.748088in}{0.452000in}}%
\pgfpathcurveto{\pgfqpoint{1.748088in}{0.456873in}}{\pgfqpoint{1.746152in}{0.461546in}}{\pgfqpoint{1.742706in}{0.464992in}}%
\pgfpathcurveto{\pgfqpoint{1.739261in}{0.468437in}}{\pgfqpoint{1.734587in}{0.470373in}}{\pgfqpoint{1.729714in}{0.470373in}}%
\pgfpathcurveto{\pgfqpoint{1.724842in}{0.470373in}}{\pgfqpoint{1.720168in}{0.468437in}}{\pgfqpoint{1.716722in}{0.464992in}}%
\pgfpathcurveto{\pgfqpoint{1.713277in}{0.461546in}}{\pgfqpoint{1.711341in}{0.456873in}}{\pgfqpoint{1.711341in}{0.452000in}}%
\pgfpathcurveto{\pgfqpoint{1.711341in}{0.447127in}}{\pgfqpoint{1.713277in}{0.442454in}}{\pgfqpoint{1.716722in}{0.439008in}}%
\pgfpathcurveto{\pgfqpoint{1.720168in}{0.435563in}}{\pgfqpoint{1.724842in}{0.433627in}}{\pgfqpoint{1.729714in}{0.433627in}}%
\pgfpathlineto{\pgfqpoint{1.729714in}{0.433627in}}%
\pgfpathclose%
\pgfusepath{stroke}%
\end{pgfscope}%
\begin{pgfscope}%
\pgfpathrectangle{\pgfqpoint{0.100000in}{0.100000in}}{\pgfqpoint{1.782500in}{1.232000in}}%
\pgfusepath{clip}%
\pgfsetbuttcap%
\pgfsetroundjoin%
\pgfsetlinewidth{0.501875pt}%
\definecolor{currentstroke}{rgb}{0.054902,0.262745,0.486275}%
\pgfsetstrokecolor{currentstroke}%
\pgfsetdash{}{0pt}%
\pgfpathmoveto{\pgfqpoint{1.576929in}{0.433627in}}%
\pgfpathcurveto{\pgfqpoint{1.581801in}{0.433627in}}{\pgfqpoint{1.586475in}{0.435563in}}{\pgfqpoint{1.589920in}{0.439008in}}%
\pgfpathcurveto{\pgfqpoint{1.593366in}{0.442454in}}{\pgfqpoint{1.595302in}{0.447127in}}{\pgfqpoint{1.595302in}{0.452000in}}%
\pgfpathcurveto{\pgfqpoint{1.595302in}{0.456873in}}{\pgfqpoint{1.593366in}{0.461546in}}{\pgfqpoint{1.589920in}{0.464992in}}%
\pgfpathcurveto{\pgfqpoint{1.586475in}{0.468437in}}{\pgfqpoint{1.581801in}{0.470373in}}{\pgfqpoint{1.576929in}{0.470373in}}%
\pgfpathcurveto{\pgfqpoint{1.572056in}{0.470373in}}{\pgfqpoint{1.567382in}{0.468437in}}{\pgfqpoint{1.563937in}{0.464992in}}%
\pgfpathcurveto{\pgfqpoint{1.560491in}{0.461546in}}{\pgfqpoint{1.558555in}{0.456873in}}{\pgfqpoint{1.558555in}{0.452000in}}%
\pgfpathcurveto{\pgfqpoint{1.558555in}{0.447127in}}{\pgfqpoint{1.560491in}{0.442454in}}{\pgfqpoint{1.563937in}{0.439008in}}%
\pgfpathcurveto{\pgfqpoint{1.567382in}{0.435563in}}{\pgfqpoint{1.572056in}{0.433627in}}{\pgfqpoint{1.576929in}{0.433627in}}%
\pgfpathlineto{\pgfqpoint{1.576929in}{0.433627in}}%
\pgfpathclose%
\pgfusepath{stroke}%
\end{pgfscope}%
\begin{pgfscope}%
\pgfpathrectangle{\pgfqpoint{0.100000in}{0.100000in}}{\pgfqpoint{1.782500in}{1.232000in}}%
\pgfusepath{clip}%
\pgfsetbuttcap%
\pgfsetroundjoin%
\pgfsetlinewidth{0.501875pt}%
\definecolor{currentstroke}{rgb}{0.054902,0.262745,0.486275}%
\pgfsetstrokecolor{currentstroke}%
\pgfsetdash{}{0pt}%
\pgfpathmoveto{\pgfqpoint{1.424143in}{0.433627in}}%
\pgfpathcurveto{\pgfqpoint{1.429016in}{0.433627in}}{\pgfqpoint{1.433689in}{0.435563in}}{\pgfqpoint{1.437135in}{0.439008in}}%
\pgfpathcurveto{\pgfqpoint{1.440580in}{0.442454in}}{\pgfqpoint{1.442516in}{0.447127in}}{\pgfqpoint{1.442516in}{0.452000in}}%
\pgfpathcurveto{\pgfqpoint{1.442516in}{0.456873in}}{\pgfqpoint{1.440580in}{0.461546in}}{\pgfqpoint{1.437135in}{0.464992in}}%
\pgfpathcurveto{\pgfqpoint{1.433689in}{0.468437in}}{\pgfqpoint{1.429016in}{0.470373in}}{\pgfqpoint{1.424143in}{0.470373in}}%
\pgfpathcurveto{\pgfqpoint{1.419270in}{0.470373in}}{\pgfqpoint{1.414596in}{0.468437in}}{\pgfqpoint{1.411151in}{0.464992in}}%
\pgfpathcurveto{\pgfqpoint{1.407706in}{0.461546in}}{\pgfqpoint{1.405770in}{0.456873in}}{\pgfqpoint{1.405770in}{0.452000in}}%
\pgfpathcurveto{\pgfqpoint{1.405770in}{0.447127in}}{\pgfqpoint{1.407706in}{0.442454in}}{\pgfqpoint{1.411151in}{0.439008in}}%
\pgfpathcurveto{\pgfqpoint{1.414596in}{0.435563in}}{\pgfqpoint{1.419270in}{0.433627in}}{\pgfqpoint{1.424143in}{0.433627in}}%
\pgfpathlineto{\pgfqpoint{1.424143in}{0.433627in}}%
\pgfpathclose%
\pgfusepath{stroke}%
\end{pgfscope}%
\begin{pgfscope}%
\pgfpathrectangle{\pgfqpoint{0.100000in}{0.100000in}}{\pgfqpoint{1.782500in}{1.232000in}}%
\pgfusepath{clip}%
\pgfsetbuttcap%
\pgfsetroundjoin%
\pgfsetlinewidth{0.501875pt}%
\definecolor{currentstroke}{rgb}{0.054902,0.262745,0.486275}%
\pgfsetstrokecolor{currentstroke}%
\pgfsetdash{}{0pt}%
\pgfpathmoveto{\pgfqpoint{1.271357in}{0.433627in}}%
\pgfpathcurveto{\pgfqpoint{1.276230in}{0.433627in}}{\pgfqpoint{1.280904in}{0.435563in}}{\pgfqpoint{1.284349in}{0.439008in}}%
\pgfpathcurveto{\pgfqpoint{1.287794in}{0.442454in}}{\pgfqpoint{1.289730in}{0.447127in}}{\pgfqpoint{1.289730in}{0.452000in}}%
\pgfpathcurveto{\pgfqpoint{1.289730in}{0.456873in}}{\pgfqpoint{1.287794in}{0.461546in}}{\pgfqpoint{1.284349in}{0.464992in}}%
\pgfpathcurveto{\pgfqpoint{1.280904in}{0.468437in}}{\pgfqpoint{1.276230in}{0.470373in}}{\pgfqpoint{1.271357in}{0.470373in}}%
\pgfpathcurveto{\pgfqpoint{1.266484in}{0.470373in}}{\pgfqpoint{1.261811in}{0.468437in}}{\pgfqpoint{1.258365in}{0.464992in}}%
\pgfpathcurveto{\pgfqpoint{1.254920in}{0.461546in}}{\pgfqpoint{1.252984in}{0.456873in}}{\pgfqpoint{1.252984in}{0.452000in}}%
\pgfpathcurveto{\pgfqpoint{1.252984in}{0.447127in}}{\pgfqpoint{1.254920in}{0.442454in}}{\pgfqpoint{1.258365in}{0.439008in}}%
\pgfpathcurveto{\pgfqpoint{1.261811in}{0.435563in}}{\pgfqpoint{1.266484in}{0.433627in}}{\pgfqpoint{1.271357in}{0.433627in}}%
\pgfpathlineto{\pgfqpoint{1.271357in}{0.433627in}}%
\pgfpathclose%
\pgfusepath{stroke}%
\end{pgfscope}%
\begin{pgfscope}%
\pgfpathrectangle{\pgfqpoint{0.100000in}{0.100000in}}{\pgfqpoint{1.782500in}{1.232000in}}%
\pgfusepath{clip}%
\pgfsetbuttcap%
\pgfsetroundjoin%
\pgfsetlinewidth{0.501875pt}%
\definecolor{currentstroke}{rgb}{0.054902,0.262745,0.486275}%
\pgfsetstrokecolor{currentstroke}%
\pgfsetdash{}{0pt}%
\pgfpathmoveto{\pgfqpoint{1.118571in}{0.433627in}}%
\pgfpathcurveto{\pgfqpoint{1.123444in}{0.433627in}}{\pgfqpoint{1.128118in}{0.435563in}}{\pgfqpoint{1.131563in}{0.439008in}}%
\pgfpathcurveto{\pgfqpoint{1.135009in}{0.442454in}}{\pgfqpoint{1.136945in}{0.447127in}}{\pgfqpoint{1.136945in}{0.452000in}}%
\pgfpathcurveto{\pgfqpoint{1.136945in}{0.456873in}}{\pgfqpoint{1.135009in}{0.461546in}}{\pgfqpoint{1.131563in}{0.464992in}}%
\pgfpathcurveto{\pgfqpoint{1.128118in}{0.468437in}}{\pgfqpoint{1.123444in}{0.470373in}}{\pgfqpoint{1.118571in}{0.470373in}}%
\pgfpathcurveto{\pgfqpoint{1.113699in}{0.470373in}}{\pgfqpoint{1.109025in}{0.468437in}}{\pgfqpoint{1.105580in}{0.464992in}}%
\pgfpathcurveto{\pgfqpoint{1.102134in}{0.461546in}}{\pgfqpoint{1.100198in}{0.456873in}}{\pgfqpoint{1.100198in}{0.452000in}}%
\pgfpathcurveto{\pgfqpoint{1.100198in}{0.447127in}}{\pgfqpoint{1.102134in}{0.442454in}}{\pgfqpoint{1.105580in}{0.439008in}}%
\pgfpathcurveto{\pgfqpoint{1.109025in}{0.435563in}}{\pgfqpoint{1.113699in}{0.433627in}}{\pgfqpoint{1.118571in}{0.433627in}}%
\pgfpathlineto{\pgfqpoint{1.118571in}{0.433627in}}%
\pgfpathclose%
\pgfusepath{stroke}%
\end{pgfscope}%
\begin{pgfscope}%
\pgfpathrectangle{\pgfqpoint{0.100000in}{0.100000in}}{\pgfqpoint{1.782500in}{1.232000in}}%
\pgfusepath{clip}%
\pgfsetbuttcap%
\pgfsetroundjoin%
\pgfsetlinewidth{0.501875pt}%
\definecolor{currentstroke}{rgb}{0.054902,0.262745,0.486275}%
\pgfsetstrokecolor{currentstroke}%
\pgfsetdash{}{0pt}%
\pgfpathmoveto{\pgfqpoint{0.965786in}{0.433627in}}%
\pgfpathcurveto{\pgfqpoint{0.970658in}{0.433627in}}{\pgfqpoint{0.975332in}{0.435563in}}{\pgfqpoint{0.978778in}{0.439008in}}%
\pgfpathcurveto{\pgfqpoint{0.982223in}{0.442454in}}{\pgfqpoint{0.984159in}{0.447127in}}{\pgfqpoint{0.984159in}{0.452000in}}%
\pgfpathcurveto{\pgfqpoint{0.984159in}{0.456873in}}{\pgfqpoint{0.982223in}{0.461546in}}{\pgfqpoint{0.978778in}{0.464992in}}%
\pgfpathcurveto{\pgfqpoint{0.975332in}{0.468437in}}{\pgfqpoint{0.970658in}{0.470373in}}{\pgfqpoint{0.965786in}{0.470373in}}%
\pgfpathcurveto{\pgfqpoint{0.960913in}{0.470373in}}{\pgfqpoint{0.956239in}{0.468437in}}{\pgfqpoint{0.952794in}{0.464992in}}%
\pgfpathcurveto{\pgfqpoint{0.949348in}{0.461546in}}{\pgfqpoint{0.947412in}{0.456873in}}{\pgfqpoint{0.947412in}{0.452000in}}%
\pgfpathcurveto{\pgfqpoint{0.947412in}{0.447127in}}{\pgfqpoint{0.949348in}{0.442454in}}{\pgfqpoint{0.952794in}{0.439008in}}%
\pgfpathcurveto{\pgfqpoint{0.956239in}{0.435563in}}{\pgfqpoint{0.960913in}{0.433627in}}{\pgfqpoint{0.965786in}{0.433627in}}%
\pgfpathlineto{\pgfqpoint{0.965786in}{0.433627in}}%
\pgfpathclose%
\pgfusepath{stroke}%
\end{pgfscope}%
\begin{pgfscope}%
\pgfpathrectangle{\pgfqpoint{0.100000in}{0.100000in}}{\pgfqpoint{1.782500in}{1.232000in}}%
\pgfusepath{clip}%
\pgfsetbuttcap%
\pgfsetroundjoin%
\pgfsetlinewidth{0.501875pt}%
\definecolor{currentstroke}{rgb}{0.835294,0.321569,0.035294}%
\pgfsetstrokecolor{currentstroke}%
\pgfsetdash{}{0pt}%
\pgfpathmoveto{\pgfqpoint{0.813000in}{0.842044in}}%
\pgfpathcurveto{\pgfqpoint{0.817873in}{0.842044in}}{\pgfqpoint{0.822546in}{0.843980in}}{\pgfqpoint{0.825992in}{0.847425in}}%
\pgfpathcurveto{\pgfqpoint{0.829437in}{0.850871in}}{\pgfqpoint{0.831373in}{0.855544in}}{\pgfqpoint{0.831373in}{0.860417in}}%
\pgfpathcurveto{\pgfqpoint{0.831373in}{0.865290in}}{\pgfqpoint{0.829437in}{0.869963in}}{\pgfqpoint{0.825992in}{0.873409in}}%
\pgfpathcurveto{\pgfqpoint{0.822546in}{0.876854in}}{\pgfqpoint{0.817873in}{0.878790in}}{\pgfqpoint{0.813000in}{0.878790in}}%
\pgfpathcurveto{\pgfqpoint{0.808127in}{0.878790in}}{\pgfqpoint{0.803454in}{0.876854in}}{\pgfqpoint{0.800008in}{0.873409in}}%
\pgfpathcurveto{\pgfqpoint{0.796563in}{0.869963in}}{\pgfqpoint{0.794627in}{0.865290in}}{\pgfqpoint{0.794627in}{0.860417in}}%
\pgfpathcurveto{\pgfqpoint{0.794627in}{0.855544in}}{\pgfqpoint{0.796563in}{0.850871in}}{\pgfqpoint{0.800008in}{0.847425in}}%
\pgfpathcurveto{\pgfqpoint{0.803454in}{0.843980in}}{\pgfqpoint{0.808127in}{0.842044in}}{\pgfqpoint{0.813000in}{0.842044in}}%
\pgfpathlineto{\pgfqpoint{0.813000in}{0.842044in}}%
\pgfpathclose%
\pgfusepath{stroke}%
\end{pgfscope}%
\begin{pgfscope}%
\pgfpathrectangle{\pgfqpoint{0.100000in}{0.100000in}}{\pgfqpoint{1.782500in}{1.232000in}}%
\pgfusepath{clip}%
\pgfsetbuttcap%
\pgfsetroundjoin%
\pgfsetlinewidth{0.501875pt}%
\definecolor{currentstroke}{rgb}{0.835294,0.321569,0.035294}%
\pgfsetstrokecolor{currentstroke}%
\pgfsetdash{}{0pt}%
\pgfpathmoveto{\pgfqpoint{2.952000in}{0.842044in}}%
\pgfpathcurveto{\pgfqpoint{2.956873in}{0.842044in}}{\pgfqpoint{2.961546in}{0.843980in}}{\pgfqpoint{2.964992in}{0.847425in}}%
\pgfpathcurveto{\pgfqpoint{2.968437in}{0.850871in}}{\pgfqpoint{2.970373in}{0.855544in}}{\pgfqpoint{2.970373in}{0.860417in}}%
\pgfpathcurveto{\pgfqpoint{2.970373in}{0.865290in}}{\pgfqpoint{2.968437in}{0.869963in}}{\pgfqpoint{2.964992in}{0.873409in}}%
\pgfpathcurveto{\pgfqpoint{2.961546in}{0.876854in}}{\pgfqpoint{2.956873in}{0.878790in}}{\pgfqpoint{2.952000in}{0.878790in}}%
\pgfpathcurveto{\pgfqpoint{2.947127in}{0.878790in}}{\pgfqpoint{2.942454in}{0.876854in}}{\pgfqpoint{2.939008in}{0.873409in}}%
\pgfpathcurveto{\pgfqpoint{2.935563in}{0.869963in}}{\pgfqpoint{2.933627in}{0.865290in}}{\pgfqpoint{2.933627in}{0.860417in}}%
\pgfpathcurveto{\pgfqpoint{2.933627in}{0.855544in}}{\pgfqpoint{2.935563in}{0.850871in}}{\pgfqpoint{2.939008in}{0.847425in}}%
\pgfpathcurveto{\pgfqpoint{2.942454in}{0.843980in}}{\pgfqpoint{2.947127in}{0.842044in}}{\pgfqpoint{2.952000in}{0.842044in}}%
\pgfusepath{stroke}%
\end{pgfscope}%
\begin{pgfscope}%
\pgfpathrectangle{\pgfqpoint{0.100000in}{0.100000in}}{\pgfqpoint{1.782500in}{1.232000in}}%
\pgfusepath{clip}%
\pgfsetbuttcap%
\pgfsetroundjoin%
\pgfsetlinewidth{0.501875pt}%
\definecolor{currentstroke}{rgb}{0.835294,0.321569,0.035294}%
\pgfsetstrokecolor{currentstroke}%
\pgfsetdash{}{0pt}%
\pgfpathmoveto{\pgfqpoint{0.909162in}{0.720787in}}%
\pgfpathcurveto{\pgfqpoint{0.914035in}{0.720787in}}{\pgfqpoint{0.918708in}{0.722723in}}{\pgfqpoint{0.922154in}{0.726168in}}%
\pgfpathcurveto{\pgfqpoint{0.925599in}{0.729614in}}{\pgfqpoint{0.927535in}{0.734287in}}{\pgfqpoint{0.927535in}{0.739160in}}%
\pgfpathcurveto{\pgfqpoint{0.927535in}{0.744033in}}{\pgfqpoint{0.925599in}{0.748706in}}{\pgfqpoint{0.922154in}{0.752152in}}%
\pgfpathcurveto{\pgfqpoint{0.918708in}{0.755597in}}{\pgfqpoint{0.914035in}{0.757533in}}{\pgfqpoint{0.909162in}{0.757533in}}%
\pgfpathcurveto{\pgfqpoint{0.904289in}{0.757533in}}{\pgfqpoint{0.899616in}{0.755597in}}{\pgfqpoint{0.896170in}{0.752152in}}%
\pgfpathcurveto{\pgfqpoint{0.892725in}{0.748706in}}{\pgfqpoint{0.890789in}{0.744033in}}{\pgfqpoint{0.890789in}{0.739160in}}%
\pgfpathcurveto{\pgfqpoint{0.890789in}{0.734287in}}{\pgfqpoint{0.892725in}{0.729614in}}{\pgfqpoint{0.896170in}{0.726168in}}%
\pgfpathcurveto{\pgfqpoint{0.899616in}{0.722723in}}{\pgfqpoint{0.904289in}{0.720787in}}{\pgfqpoint{0.909162in}{0.720787in}}%
\pgfpathlineto{\pgfqpoint{0.909162in}{0.720787in}}%
\pgfpathclose%
\pgfusepath{stroke}%
\end{pgfscope}%
\begin{pgfscope}%
\pgfpathrectangle{\pgfqpoint{0.100000in}{0.100000in}}{\pgfqpoint{1.782500in}{1.232000in}}%
\pgfusepath{clip}%
\pgfsetbuttcap%
\pgfsetroundjoin%
\pgfsetlinewidth{0.501875pt}%
\definecolor{currentstroke}{rgb}{0.835294,0.321569,0.035294}%
\pgfsetstrokecolor{currentstroke}%
\pgfsetdash{}{0pt}%
\pgfpathmoveto{\pgfqpoint{1.030178in}{0.612421in}}%
\pgfpathcurveto{\pgfqpoint{1.035051in}{0.612421in}}{\pgfqpoint{1.039725in}{0.614357in}}{\pgfqpoint{1.043170in}{0.617802in}}%
\pgfpathcurveto{\pgfqpoint{1.046616in}{0.621248in}}{\pgfqpoint{1.048551in}{0.625922in}}{\pgfqpoint{1.048551in}{0.630794in}}%
\pgfpathcurveto{\pgfqpoint{1.048551in}{0.635667in}}{\pgfqpoint{1.046616in}{0.640341in}}{\pgfqpoint{1.043170in}{0.643786in}}%
\pgfpathcurveto{\pgfqpoint{1.039725in}{0.647232in}}{\pgfqpoint{1.035051in}{0.649167in}}{\pgfqpoint{1.030178in}{0.649167in}}%
\pgfpathcurveto{\pgfqpoint{1.025306in}{0.649167in}}{\pgfqpoint{1.020632in}{0.647232in}}{\pgfqpoint{1.017186in}{0.643786in}}%
\pgfpathcurveto{\pgfqpoint{1.013741in}{0.640341in}}{\pgfqpoint{1.011805in}{0.635667in}}{\pgfqpoint{1.011805in}{0.630794in}}%
\pgfpathcurveto{\pgfqpoint{1.011805in}{0.625922in}}{\pgfqpoint{1.013741in}{0.621248in}}{\pgfqpoint{1.017186in}{0.617802in}}%
\pgfpathcurveto{\pgfqpoint{1.020632in}{0.614357in}}{\pgfqpoint{1.025306in}{0.612421in}}{\pgfqpoint{1.030178in}{0.612421in}}%
\pgfpathlineto{\pgfqpoint{1.030178in}{0.612421in}}%
\pgfpathclose%
\pgfusepath{stroke}%
\end{pgfscope}%
\begin{pgfscope}%
\pgfpathrectangle{\pgfqpoint{0.100000in}{0.100000in}}{\pgfqpoint{1.782500in}{1.232000in}}%
\pgfusepath{clip}%
\pgfsetbuttcap%
\pgfsetroundjoin%
\pgfsetlinewidth{0.501875pt}%
\definecolor{currentstroke}{rgb}{0.835294,0.321569,0.035294}%
\pgfsetstrokecolor{currentstroke}%
\pgfsetdash{}{0pt}%
\pgfpathmoveto{\pgfqpoint{1.172958in}{0.519713in}}%
\pgfpathcurveto{\pgfqpoint{1.177831in}{0.519713in}}{\pgfqpoint{1.182505in}{0.521649in}}{\pgfqpoint{1.185950in}{0.525095in}}%
\pgfpathcurveto{\pgfqpoint{1.189396in}{0.528540in}}{\pgfqpoint{1.191332in}{0.533214in}}{\pgfqpoint{1.191332in}{0.538087in}}%
\pgfpathcurveto{\pgfqpoint{1.191332in}{0.542959in}}{\pgfqpoint{1.189396in}{0.547633in}}{\pgfqpoint{1.185950in}{0.551078in}}%
\pgfpathcurveto{\pgfqpoint{1.182505in}{0.554524in}}{\pgfqpoint{1.177831in}{0.556460in}}{\pgfqpoint{1.172958in}{0.556460in}}%
\pgfpathcurveto{\pgfqpoint{1.168086in}{0.556460in}}{\pgfqpoint{1.163412in}{0.554524in}}{\pgfqpoint{1.159966in}{0.551078in}}%
\pgfpathcurveto{\pgfqpoint{1.156521in}{0.547633in}}{\pgfqpoint{1.154585in}{0.542959in}}{\pgfqpoint{1.154585in}{0.538087in}}%
\pgfpathcurveto{\pgfqpoint{1.154585in}{0.533214in}}{\pgfqpoint{1.156521in}{0.528540in}}{\pgfqpoint{1.159966in}{0.525095in}}%
\pgfpathcurveto{\pgfqpoint{1.163412in}{0.521649in}}{\pgfqpoint{1.168086in}{0.519713in}}{\pgfqpoint{1.172958in}{0.519713in}}%
\pgfpathlineto{\pgfqpoint{1.172958in}{0.519713in}}%
\pgfpathclose%
\pgfusepath{stroke}%
\end{pgfscope}%
\begin{pgfscope}%
\pgfpathrectangle{\pgfqpoint{0.100000in}{0.100000in}}{\pgfqpoint{1.782500in}{1.232000in}}%
\pgfusepath{clip}%
\pgfsetbuttcap%
\pgfsetroundjoin%
\pgfsetlinewidth{0.501875pt}%
\definecolor{currentstroke}{rgb}{0.835294,0.321569,0.035294}%
\pgfsetstrokecolor{currentstroke}%
\pgfsetdash{}{0pt}%
\pgfpathmoveto{\pgfqpoint{1.333857in}{0.445031in}}%
\pgfpathcurveto{\pgfqpoint{1.338729in}{0.445031in}}{\pgfqpoint{1.343403in}{0.446967in}}{\pgfqpoint{1.346848in}{0.450412in}}%
\pgfpathcurveto{\pgfqpoint{1.350294in}{0.453858in}}{\pgfqpoint{1.352230in}{0.458532in}}{\pgfqpoint{1.352230in}{0.463404in}}%
\pgfpathcurveto{\pgfqpoint{1.352230in}{0.468277in}}{\pgfqpoint{1.350294in}{0.472951in}}{\pgfqpoint{1.346848in}{0.476396in}}%
\pgfpathcurveto{\pgfqpoint{1.343403in}{0.479842in}}{\pgfqpoint{1.338729in}{0.481778in}}{\pgfqpoint{1.333857in}{0.481778in}}%
\pgfpathcurveto{\pgfqpoint{1.328984in}{0.481778in}}{\pgfqpoint{1.324310in}{0.479842in}}{\pgfqpoint{1.320865in}{0.476396in}}%
\pgfpathcurveto{\pgfqpoint{1.317419in}{0.472951in}}{\pgfqpoint{1.315483in}{0.468277in}}{\pgfqpoint{1.315483in}{0.463404in}}%
\pgfpathcurveto{\pgfqpoint{1.315483in}{0.458532in}}{\pgfqpoint{1.317419in}{0.453858in}}{\pgfqpoint{1.320865in}{0.450412in}}%
\pgfpathcurveto{\pgfqpoint{1.324310in}{0.446967in}}{\pgfqpoint{1.328984in}{0.445031in}}{\pgfqpoint{1.333857in}{0.445031in}}%
\pgfpathlineto{\pgfqpoint{1.333857in}{0.445031in}}%
\pgfpathclose%
\pgfusepath{stroke}%
\end{pgfscope}%
\begin{pgfscope}%
\pgfpathrectangle{\pgfqpoint{0.100000in}{0.100000in}}{\pgfqpoint{1.782500in}{1.232000in}}%
\pgfusepath{clip}%
\pgfsetbuttcap%
\pgfsetroundjoin%
\pgfsetlinewidth{0.501875pt}%
\definecolor{currentstroke}{rgb}{0.835294,0.321569,0.035294}%
\pgfsetstrokecolor{currentstroke}%
\pgfsetdash{}{0pt}%
\pgfpathmoveto{\pgfqpoint{1.508764in}{0.390281in}}%
\pgfpathcurveto{\pgfqpoint{1.513637in}{0.390281in}}{\pgfqpoint{1.518311in}{0.392217in}}{\pgfqpoint{1.521756in}{0.395663in}}%
\pgfpathcurveto{\pgfqpoint{1.525202in}{0.399108in}}{\pgfqpoint{1.527138in}{0.403782in}}{\pgfqpoint{1.527138in}{0.408655in}}%
\pgfpathcurveto{\pgfqpoint{1.527138in}{0.413527in}}{\pgfqpoint{1.525202in}{0.418201in}}{\pgfqpoint{1.521756in}{0.421646in}}%
\pgfpathcurveto{\pgfqpoint{1.518311in}{0.425092in}}{\pgfqpoint{1.513637in}{0.427028in}}{\pgfqpoint{1.508764in}{0.427028in}}%
\pgfpathcurveto{\pgfqpoint{1.503892in}{0.427028in}}{\pgfqpoint{1.499218in}{0.425092in}}{\pgfqpoint{1.495772in}{0.421646in}}%
\pgfpathcurveto{\pgfqpoint{1.492327in}{0.418201in}}{\pgfqpoint{1.490391in}{0.413527in}}{\pgfqpoint{1.490391in}{0.408655in}}%
\pgfpathcurveto{\pgfqpoint{1.490391in}{0.403782in}}{\pgfqpoint{1.492327in}{0.399108in}}{\pgfqpoint{1.495772in}{0.395663in}}%
\pgfpathcurveto{\pgfqpoint{1.499218in}{0.392217in}}{\pgfqpoint{1.503892in}{0.390281in}}{\pgfqpoint{1.508764in}{0.390281in}}%
\pgfpathlineto{\pgfqpoint{1.508764in}{0.390281in}}%
\pgfpathclose%
\pgfusepath{stroke}%
\end{pgfscope}%
\begin{pgfscope}%
\pgfpathrectangle{\pgfqpoint{0.100000in}{0.100000in}}{\pgfqpoint{1.782500in}{1.232000in}}%
\pgfusepath{clip}%
\pgfsetbuttcap%
\pgfsetroundjoin%
\pgfsetlinewidth{0.501875pt}%
\definecolor{currentstroke}{rgb}{0.835294,0.321569,0.035294}%
\pgfsetstrokecolor{currentstroke}%
\pgfsetdash{}{0pt}%
\pgfpathmoveto{\pgfqpoint{1.693215in}{0.356862in}}%
\pgfpathcurveto{\pgfqpoint{1.698088in}{0.356862in}}{\pgfqpoint{1.702762in}{0.358798in}}{\pgfqpoint{1.706207in}{0.362244in}}%
\pgfpathcurveto{\pgfqpoint{1.709653in}{0.365689in}}{\pgfqpoint{1.711589in}{0.370363in}}{\pgfqpoint{1.711589in}{0.375235in}}%
\pgfpathcurveto{\pgfqpoint{1.711589in}{0.380108in}}{\pgfqpoint{1.709653in}{0.384782in}}{\pgfqpoint{1.706207in}{0.388227in}}%
\pgfpathcurveto{\pgfqpoint{1.702762in}{0.391673in}}{\pgfqpoint{1.698088in}{0.393609in}}{\pgfqpoint{1.693215in}{0.393609in}}%
\pgfpathcurveto{\pgfqpoint{1.688343in}{0.393609in}}{\pgfqpoint{1.683669in}{0.391673in}}{\pgfqpoint{1.680224in}{0.388227in}}%
\pgfpathcurveto{\pgfqpoint{1.676778in}{0.384782in}}{\pgfqpoint{1.674842in}{0.380108in}}{\pgfqpoint{1.674842in}{0.375235in}}%
\pgfpathcurveto{\pgfqpoint{1.674842in}{0.370363in}}{\pgfqpoint{1.676778in}{0.365689in}}{\pgfqpoint{1.680224in}{0.362244in}}%
\pgfpathcurveto{\pgfqpoint{1.683669in}{0.358798in}}{\pgfqpoint{1.688343in}{0.356862in}}{\pgfqpoint{1.693215in}{0.356862in}}%
\pgfpathlineto{\pgfqpoint{1.693215in}{0.356862in}}%
\pgfpathclose%
\pgfusepath{stroke}%
\end{pgfscope}%
\begin{pgfscope}%
\pgfpathrectangle{\pgfqpoint{0.100000in}{0.100000in}}{\pgfqpoint{1.782500in}{1.232000in}}%
\pgfusepath{clip}%
\pgfsetbuttcap%
\pgfsetroundjoin%
\pgfsetlinewidth{0.501875pt}%
\definecolor{currentstroke}{rgb}{0.835294,0.321569,0.035294}%
\pgfsetstrokecolor{currentstroke}%
\pgfsetdash{}{0pt}%
\pgfpathmoveto{\pgfqpoint{1.882500in}{0.345627in}}%
\pgfpathcurveto{\pgfqpoint{1.887373in}{0.345627in}}{\pgfqpoint{1.892046in}{0.347563in}}{\pgfqpoint{1.895492in}{0.351008in}}%
\pgfpathcurveto{\pgfqpoint{1.898937in}{0.354454in}}{\pgfqpoint{1.900873in}{0.359127in}}{\pgfqpoint{1.900873in}{0.364000in}}%
\pgfpathcurveto{\pgfqpoint{1.900873in}{0.368873in}}{\pgfqpoint{1.898937in}{0.373546in}}{\pgfqpoint{1.895492in}{0.376992in}}%
\pgfpathcurveto{\pgfqpoint{1.892046in}{0.380437in}}{\pgfqpoint{1.887373in}{0.382373in}}{\pgfqpoint{1.882500in}{0.382373in}}%
\pgfpathcurveto{\pgfqpoint{1.877627in}{0.382373in}}{\pgfqpoint{1.872954in}{0.380437in}}{\pgfqpoint{1.869508in}{0.376992in}}%
\pgfpathcurveto{\pgfqpoint{1.866063in}{0.373546in}}{\pgfqpoint{1.864127in}{0.368873in}}{\pgfqpoint{1.864127in}{0.364000in}}%
\pgfpathcurveto{\pgfqpoint{1.864127in}{0.359127in}}{\pgfqpoint{1.866063in}{0.354454in}}{\pgfqpoint{1.869508in}{0.351008in}}%
\pgfpathcurveto{\pgfqpoint{1.872954in}{0.347563in}}{\pgfqpoint{1.877627in}{0.345627in}}{\pgfqpoint{1.882500in}{0.345627in}}%
\pgfpathlineto{\pgfqpoint{1.882500in}{0.345627in}}%
\pgfpathclose%
\pgfusepath{stroke}%
\end{pgfscope}%
\begin{pgfscope}%
\pgfpathrectangle{\pgfqpoint{0.100000in}{0.100000in}}{\pgfqpoint{1.782500in}{1.232000in}}%
\pgfusepath{clip}%
\pgfsetbuttcap%
\pgfsetroundjoin%
\pgfsetlinewidth{0.501875pt}%
\definecolor{currentstroke}{rgb}{0.835294,0.321569,0.035294}%
\pgfsetstrokecolor{currentstroke}%
\pgfsetdash{}{0pt}%
\pgfpathmoveto{\pgfqpoint{2.071785in}{0.356862in}}%
\pgfpathcurveto{\pgfqpoint{2.076657in}{0.356862in}}{\pgfqpoint{2.081331in}{0.358798in}}{\pgfqpoint{2.084776in}{0.362244in}}%
\pgfpathcurveto{\pgfqpoint{2.088222in}{0.365689in}}{\pgfqpoint{2.090158in}{0.370363in}}{\pgfqpoint{2.090158in}{0.375235in}}%
\pgfpathcurveto{\pgfqpoint{2.090158in}{0.380108in}}{\pgfqpoint{2.088222in}{0.384782in}}{\pgfqpoint{2.084776in}{0.388227in}}%
\pgfpathcurveto{\pgfqpoint{2.081331in}{0.391673in}}{\pgfqpoint{2.076657in}{0.393609in}}{\pgfqpoint{2.071785in}{0.393609in}}%
\pgfpathcurveto{\pgfqpoint{2.066912in}{0.393609in}}{\pgfqpoint{2.062238in}{0.391673in}}{\pgfqpoint{2.058793in}{0.388227in}}%
\pgfpathcurveto{\pgfqpoint{2.055347in}{0.384782in}}{\pgfqpoint{2.053411in}{0.380108in}}{\pgfqpoint{2.053411in}{0.375235in}}%
\pgfpathcurveto{\pgfqpoint{2.053411in}{0.370363in}}{\pgfqpoint{2.055347in}{0.365689in}}{\pgfqpoint{2.058793in}{0.362244in}}%
\pgfpathcurveto{\pgfqpoint{2.062238in}{0.358798in}}{\pgfqpoint{2.066912in}{0.356862in}}{\pgfqpoint{2.071785in}{0.356862in}}%
\pgfusepath{stroke}%
\end{pgfscope}%
\begin{pgfscope}%
\pgfpathrectangle{\pgfqpoint{0.100000in}{0.100000in}}{\pgfqpoint{1.782500in}{1.232000in}}%
\pgfusepath{clip}%
\pgfsetbuttcap%
\pgfsetroundjoin%
\pgfsetlinewidth{0.501875pt}%
\definecolor{currentstroke}{rgb}{0.835294,0.321569,0.035294}%
\pgfsetstrokecolor{currentstroke}%
\pgfsetdash{}{0pt}%
\pgfpathmoveto{\pgfqpoint{2.256236in}{0.390281in}}%
\pgfpathcurveto{\pgfqpoint{2.261108in}{0.390281in}}{\pgfqpoint{2.265782in}{0.392217in}}{\pgfqpoint{2.269228in}{0.395663in}}%
\pgfpathcurveto{\pgfqpoint{2.272673in}{0.399108in}}{\pgfqpoint{2.274609in}{0.403782in}}{\pgfqpoint{2.274609in}{0.408655in}}%
\pgfpathcurveto{\pgfqpoint{2.274609in}{0.413527in}}{\pgfqpoint{2.272673in}{0.418201in}}{\pgfqpoint{2.269228in}{0.421646in}}%
\pgfpathcurveto{\pgfqpoint{2.265782in}{0.425092in}}{\pgfqpoint{2.261108in}{0.427028in}}{\pgfqpoint{2.256236in}{0.427028in}}%
\pgfpathcurveto{\pgfqpoint{2.251363in}{0.427028in}}{\pgfqpoint{2.246689in}{0.425092in}}{\pgfqpoint{2.243244in}{0.421646in}}%
\pgfpathcurveto{\pgfqpoint{2.239798in}{0.418201in}}{\pgfqpoint{2.237862in}{0.413527in}}{\pgfqpoint{2.237862in}{0.408655in}}%
\pgfpathcurveto{\pgfqpoint{2.237862in}{0.403782in}}{\pgfqpoint{2.239798in}{0.399108in}}{\pgfqpoint{2.243244in}{0.395663in}}%
\pgfpathcurveto{\pgfqpoint{2.246689in}{0.392217in}}{\pgfqpoint{2.251363in}{0.390281in}}{\pgfqpoint{2.256236in}{0.390281in}}%
\pgfusepath{stroke}%
\end{pgfscope}%
\begin{pgfscope}%
\pgfpathrectangle{\pgfqpoint{0.100000in}{0.100000in}}{\pgfqpoint{1.782500in}{1.232000in}}%
\pgfusepath{clip}%
\pgfsetbuttcap%
\pgfsetroundjoin%
\pgfsetlinewidth{0.501875pt}%
\definecolor{currentstroke}{rgb}{0.835294,0.321569,0.035294}%
\pgfsetstrokecolor{currentstroke}%
\pgfsetdash{}{0pt}%
\pgfpathmoveto{\pgfqpoint{2.431143in}{0.445031in}}%
\pgfpathcurveto{\pgfqpoint{2.436016in}{0.445031in}}{\pgfqpoint{2.440690in}{0.446967in}}{\pgfqpoint{2.444135in}{0.450412in}}%
\pgfpathcurveto{\pgfqpoint{2.447581in}{0.453858in}}{\pgfqpoint{2.449517in}{0.458532in}}{\pgfqpoint{2.449517in}{0.463404in}}%
\pgfpathcurveto{\pgfqpoint{2.449517in}{0.468277in}}{\pgfqpoint{2.447581in}{0.472951in}}{\pgfqpoint{2.444135in}{0.476396in}}%
\pgfpathcurveto{\pgfqpoint{2.440690in}{0.479842in}}{\pgfqpoint{2.436016in}{0.481778in}}{\pgfqpoint{2.431143in}{0.481778in}}%
\pgfpathcurveto{\pgfqpoint{2.426271in}{0.481778in}}{\pgfqpoint{2.421597in}{0.479842in}}{\pgfqpoint{2.418152in}{0.476396in}}%
\pgfpathcurveto{\pgfqpoint{2.414706in}{0.472951in}}{\pgfqpoint{2.412770in}{0.468277in}}{\pgfqpoint{2.412770in}{0.463404in}}%
\pgfpathcurveto{\pgfqpoint{2.412770in}{0.458532in}}{\pgfqpoint{2.414706in}{0.453858in}}{\pgfqpoint{2.418152in}{0.450412in}}%
\pgfpathcurveto{\pgfqpoint{2.421597in}{0.446967in}}{\pgfqpoint{2.426271in}{0.445031in}}{\pgfqpoint{2.431143in}{0.445031in}}%
\pgfusepath{stroke}%
\end{pgfscope}%
\begin{pgfscope}%
\pgfpathrectangle{\pgfqpoint{0.100000in}{0.100000in}}{\pgfqpoint{1.782500in}{1.232000in}}%
\pgfusepath{clip}%
\pgfsetbuttcap%
\pgfsetroundjoin%
\pgfsetlinewidth{0.501875pt}%
\definecolor{currentstroke}{rgb}{0.835294,0.321569,0.035294}%
\pgfsetstrokecolor{currentstroke}%
\pgfsetdash{}{0pt}%
\pgfpathmoveto{\pgfqpoint{2.592042in}{0.519713in}}%
\pgfpathcurveto{\pgfqpoint{2.596914in}{0.519713in}}{\pgfqpoint{2.601588in}{0.521649in}}{\pgfqpoint{2.605034in}{0.525095in}}%
\pgfpathcurveto{\pgfqpoint{2.608479in}{0.528540in}}{\pgfqpoint{2.610415in}{0.533214in}}{\pgfqpoint{2.610415in}{0.538087in}}%
\pgfpathcurveto{\pgfqpoint{2.610415in}{0.542959in}}{\pgfqpoint{2.608479in}{0.547633in}}{\pgfqpoint{2.605034in}{0.551078in}}%
\pgfpathcurveto{\pgfqpoint{2.601588in}{0.554524in}}{\pgfqpoint{2.596914in}{0.556460in}}{\pgfqpoint{2.592042in}{0.556460in}}%
\pgfpathcurveto{\pgfqpoint{2.587169in}{0.556460in}}{\pgfqpoint{2.582495in}{0.554524in}}{\pgfqpoint{2.579050in}{0.551078in}}%
\pgfpathcurveto{\pgfqpoint{2.575604in}{0.547633in}}{\pgfqpoint{2.573668in}{0.542959in}}{\pgfqpoint{2.573668in}{0.538087in}}%
\pgfpathcurveto{\pgfqpoint{2.573668in}{0.533214in}}{\pgfqpoint{2.575604in}{0.528540in}}{\pgfqpoint{2.579050in}{0.525095in}}%
\pgfpathcurveto{\pgfqpoint{2.582495in}{0.521649in}}{\pgfqpoint{2.587169in}{0.519713in}}{\pgfqpoint{2.592042in}{0.519713in}}%
\pgfusepath{stroke}%
\end{pgfscope}%
\begin{pgfscope}%
\pgfpathrectangle{\pgfqpoint{0.100000in}{0.100000in}}{\pgfqpoint{1.782500in}{1.232000in}}%
\pgfusepath{clip}%
\pgfsetbuttcap%
\pgfsetroundjoin%
\pgfsetlinewidth{0.501875pt}%
\definecolor{currentstroke}{rgb}{0.835294,0.321569,0.035294}%
\pgfsetstrokecolor{currentstroke}%
\pgfsetdash{}{0pt}%
\pgfpathmoveto{\pgfqpoint{2.734822in}{0.612421in}}%
\pgfpathcurveto{\pgfqpoint{2.739694in}{0.612421in}}{\pgfqpoint{2.744368in}{0.614357in}}{\pgfqpoint{2.747814in}{0.617802in}}%
\pgfpathcurveto{\pgfqpoint{2.751259in}{0.621248in}}{\pgfqpoint{2.753195in}{0.625922in}}{\pgfqpoint{2.753195in}{0.630794in}}%
\pgfpathcurveto{\pgfqpoint{2.753195in}{0.635667in}}{\pgfqpoint{2.751259in}{0.640341in}}{\pgfqpoint{2.747814in}{0.643786in}}%
\pgfpathcurveto{\pgfqpoint{2.744368in}{0.647232in}}{\pgfqpoint{2.739694in}{0.649167in}}{\pgfqpoint{2.734822in}{0.649167in}}%
\pgfpathcurveto{\pgfqpoint{2.729949in}{0.649167in}}{\pgfqpoint{2.725275in}{0.647232in}}{\pgfqpoint{2.721830in}{0.643786in}}%
\pgfpathcurveto{\pgfqpoint{2.718384in}{0.640341in}}{\pgfqpoint{2.716449in}{0.635667in}}{\pgfqpoint{2.716449in}{0.630794in}}%
\pgfpathcurveto{\pgfqpoint{2.716449in}{0.625922in}}{\pgfqpoint{2.718384in}{0.621248in}}{\pgfqpoint{2.721830in}{0.617802in}}%
\pgfpathcurveto{\pgfqpoint{2.725275in}{0.614357in}}{\pgfqpoint{2.729949in}{0.612421in}}{\pgfqpoint{2.734822in}{0.612421in}}%
\pgfusepath{stroke}%
\end{pgfscope}%
\begin{pgfscope}%
\pgfpathrectangle{\pgfqpoint{0.100000in}{0.100000in}}{\pgfqpoint{1.782500in}{1.232000in}}%
\pgfusepath{clip}%
\pgfsetbuttcap%
\pgfsetroundjoin%
\pgfsetlinewidth{0.501875pt}%
\definecolor{currentstroke}{rgb}{0.835294,0.321569,0.035294}%
\pgfsetstrokecolor{currentstroke}%
\pgfsetdash{}{0pt}%
\pgfpathmoveto{\pgfqpoint{2.855838in}{0.720787in}}%
\pgfpathcurveto{\pgfqpoint{2.860711in}{0.720787in}}{\pgfqpoint{2.865384in}{0.722723in}}{\pgfqpoint{2.868830in}{0.726168in}}%
\pgfpathcurveto{\pgfqpoint{2.872275in}{0.729614in}}{\pgfqpoint{2.874211in}{0.734287in}}{\pgfqpoint{2.874211in}{0.739160in}}%
\pgfpathcurveto{\pgfqpoint{2.874211in}{0.744033in}}{\pgfqpoint{2.872275in}{0.748706in}}{\pgfqpoint{2.868830in}{0.752152in}}%
\pgfpathcurveto{\pgfqpoint{2.865384in}{0.755597in}}{\pgfqpoint{2.860711in}{0.757533in}}{\pgfqpoint{2.855838in}{0.757533in}}%
\pgfpathcurveto{\pgfqpoint{2.850965in}{0.757533in}}{\pgfqpoint{2.846292in}{0.755597in}}{\pgfqpoint{2.842846in}{0.752152in}}%
\pgfpathcurveto{\pgfqpoint{2.839401in}{0.748706in}}{\pgfqpoint{2.837465in}{0.744033in}}{\pgfqpoint{2.837465in}{0.739160in}}%
\pgfpathcurveto{\pgfqpoint{2.837465in}{0.734287in}}{\pgfqpoint{2.839401in}{0.729614in}}{\pgfqpoint{2.842846in}{0.726168in}}%
\pgfpathcurveto{\pgfqpoint{2.846292in}{0.722723in}}{\pgfqpoint{2.850965in}{0.720787in}}{\pgfqpoint{2.855838in}{0.720787in}}%
\pgfusepath{stroke}%
\end{pgfscope}%
\end{pgfpicture}%
\makeatother%
\endgroup%
}
        \caption{Initial configuration}\label{fig:example-initial}
    \end{subfigure}
    \begin{subfigure}[b]{.64\linewidth}
        \begin{tikzpicture}
            \draw[thick, mainorange] (1.25, 3.25) circle (0.07);
            \node[anchor=west] at (1.5, 3.25) {Candidate nodes of primary $\mathcal{I}^C$};
            \draw[thick, mainblue] (1.25, 2.75) circle (0.07);
            \node[anchor=west] at (1.5, 2.75) {Candidate nodes of secondary $\mathcal{J}^C$};
            \fill[thick, mainorange] (1.25,  2.25) circle (0.07);
            \node[anchor=west] at (1.5, 2.25) {Interface nodes of primary $\mathcal{I}$};
            \fill[thick, mainblue] (1.25,  1.75) circle (0.07);
            \node[anchor=west] at (1.5, 1.75) {Interface nodes of secondary $\mathcal{J}$};
            \draw[ultra thick, mainorange] (1.2, 1.2) to (1.3, 1.3);
            \draw[ultra thick, mainorange] (1.3, 1.2) to (1.2, 1.3);
            \node[anchor=west] at (1.5, 1.25) {Removed nodes from primary};
            \draw[ultra thick, mainblue] (1.2, 0.7) to (1.3, 0.8);
            \draw[ultra thick, mainblue] (1.3, 0.7) to (1.2, 0.8);
            \node[anchor=west] at (1.5, 0.75) {Removed nodes from secondary};
            \fill[white] (0, 0) rectangle (0.1, 0.1);
        \end{tikzpicture}
    \end{subfigure}
    \begin{subfigure}[b]{.32\linewidth}
        \scalebox{0.8}{\input{plots/example_iter0_interface.pgf}}
        \caption{Iteration 1: Find interface}\label{fig:example-iter0-interface}
    \end{subfigure}
    \begin{subfigure}[b]{.32\linewidth}
        \scalebox{0.8}{%% Creator: Matplotlib, PGF backend
%%
%% To include the figure in your LaTeX document, write
%%   \input{<filename>.pgf}
%%
%% Make sure the required packages are loaded in your preamble
%%   \usepackage{pgf}
%%
%% Also ensure that all the required font packages are loaded; for instance,
%% the lmodern package is sometimes necessary when using math font.
%%   \usepackage{lmodern}
%%
%% Figures using additional raster images can only be included by \input if
%% they are in the same directory as the main LaTeX file. For loading figures
%% from other directories you can use the `import` package
%%   \usepackage{import}
%%
%% and then include the figures with
%%   \import{<path to file>}{<filename>.pgf}
%%
%% Matplotlib used the following preamble
%%   
%%   \usepackage{fontspec}
%%   \setmainfont{DejaVuSans.ttf}[Path=\detokenize{/home/fabio/Internodes-CM/.venv/lib/python3.8/site-packages/matplotlib/mpl-data/fonts/ttf/}]
%%   \setsansfont{DejaVuSans.ttf}[Path=\detokenize{/home/fabio/Internodes-CM/.venv/lib/python3.8/site-packages/matplotlib/mpl-data/fonts/ttf/}]
%%   \setmonofont{DejaVuSansMono.ttf}[Path=\detokenize{/home/fabio/Internodes-CM/.venv/lib/python3.8/site-packages/matplotlib/mpl-data/fonts/ttf/}]
%%   \makeatletter\@ifpackageloaded{underscore}{}{\usepackage[strings]{underscore}}\makeatother
%%
\begingroup%
\makeatletter%
\begin{pgfpicture}%
\pgfpathrectangle{\pgfpointorigin}{\pgfqpoint{1.982500in}{1.432000in}}%
\pgfusepath{use as bounding box, clip}%
\begin{pgfscope}%
\pgfsetbuttcap%
\pgfsetmiterjoin%
\definecolor{currentfill}{rgb}{1.000000,1.000000,1.000000}%
\pgfsetfillcolor{currentfill}%
\pgfsetlinewidth{0.000000pt}%
\definecolor{currentstroke}{rgb}{1.000000,1.000000,1.000000}%
\pgfsetstrokecolor{currentstroke}%
\pgfsetdash{}{0pt}%
\pgfpathmoveto{\pgfqpoint{0.000000in}{0.000000in}}%
\pgfpathlineto{\pgfqpoint{1.982500in}{0.000000in}}%
\pgfpathlineto{\pgfqpoint{1.982500in}{1.432000in}}%
\pgfpathlineto{\pgfqpoint{0.000000in}{1.432000in}}%
\pgfpathlineto{\pgfqpoint{0.000000in}{0.000000in}}%
\pgfpathclose%
\pgfusepath{fill}%
\end{pgfscope}%
\begin{pgfscope}%
\pgfpathrectangle{\pgfqpoint{0.100000in}{0.100000in}}{\pgfqpoint{1.782500in}{1.232000in}}%
\pgfusepath{clip}%
\pgfsetrectcap%
\pgfsetroundjoin%
\pgfsetlinewidth{0.250937pt}%
\definecolor{currentstroke}{rgb}{0.054902,0.262745,0.486275}%
\pgfsetstrokecolor{currentstroke}%
\pgfsetdash{}{0pt}%
\pgfpathmoveto{\pgfqpoint{0.451098in}{0.100000in}}%
\pgfpathlineto{\pgfqpoint{0.181023in}{0.100000in}}%
\pgfpathmoveto{\pgfqpoint{0.721174in}{0.100000in}}%
\pgfpathlineto{\pgfqpoint{0.451098in}{0.100000in}}%
\pgfpathmoveto{\pgfqpoint{0.991250in}{0.100000in}}%
\pgfpathlineto{\pgfqpoint{0.721174in}{0.100000in}}%
\pgfpathmoveto{\pgfqpoint{1.261326in}{0.100000in}}%
\pgfpathlineto{\pgfqpoint{0.991250in}{0.100000in}}%
\pgfpathmoveto{\pgfqpoint{1.531402in}{0.100000in}}%
\pgfpathlineto{\pgfqpoint{1.801477in}{0.100000in}}%
\pgfpathmoveto{\pgfqpoint{1.531402in}{0.100000in}}%
\pgfpathlineto{\pgfqpoint{1.261326in}{0.100000in}}%
\pgfpathmoveto{\pgfqpoint{1.784788in}{0.431960in}}%
\pgfpathlineto{\pgfqpoint{1.801477in}{0.100000in}}%
\pgfpathmoveto{\pgfqpoint{1.784788in}{0.431960in}}%
\pgfpathlineto{\pgfqpoint{1.740685in}{0.754365in}}%
\pgfpathmoveto{\pgfqpoint{1.575146in}{0.731227in}}%
\pgfpathlineto{\pgfqpoint{1.740685in}{0.754365in}}%
\pgfpathmoveto{\pgfqpoint{1.575146in}{0.731227in}}%
\pgfpathlineto{\pgfqpoint{1.377693in}{0.703727in}}%
\pgfpathmoveto{\pgfqpoint{1.333507in}{0.678982in}}%
\pgfpathlineto{\pgfqpoint{1.377693in}{0.703727in}}%
\pgfpathmoveto{\pgfqpoint{1.286136in}{0.655839in}}%
\pgfpathlineto{\pgfqpoint{1.333507in}{0.678982in}}%
\pgfpathmoveto{\pgfqpoint{1.234900in}{0.638023in}}%
\pgfpathlineto{\pgfqpoint{1.286136in}{0.655839in}}%
\pgfpathmoveto{\pgfqpoint{1.180840in}{0.624920in}}%
\pgfpathlineto{\pgfqpoint{1.234900in}{0.638023in}}%
\pgfpathmoveto{\pgfqpoint{1.119285in}{0.614311in}}%
\pgfpathlineto{\pgfqpoint{1.180840in}{0.624920in}}%
\pgfpathmoveto{\pgfqpoint{1.054943in}{0.605752in}}%
\pgfpathlineto{\pgfqpoint{1.119285in}{0.614311in}}%
\pgfpathmoveto{\pgfqpoint{0.990043in}{0.598875in}}%
\pgfpathlineto{\pgfqpoint{1.054943in}{0.605752in}}%
\pgfpathmoveto{\pgfqpoint{0.924598in}{0.600413in}}%
\pgfpathlineto{\pgfqpoint{0.990043in}{0.598875in}}%
\pgfpathmoveto{\pgfqpoint{0.861703in}{0.610513in}}%
\pgfpathlineto{\pgfqpoint{0.924598in}{0.600413in}}%
\pgfpathmoveto{\pgfqpoint{0.800345in}{0.623839in}}%
\pgfpathlineto{\pgfqpoint{0.861703in}{0.610513in}}%
\pgfpathmoveto{\pgfqpoint{0.746146in}{0.638036in}}%
\pgfpathlineto{\pgfqpoint{0.800345in}{0.623839in}}%
\pgfpathmoveto{\pgfqpoint{0.695822in}{0.656509in}}%
\pgfpathlineto{\pgfqpoint{0.746146in}{0.638036in}}%
\pgfpathmoveto{\pgfqpoint{0.647969in}{0.682601in}}%
\pgfpathlineto{\pgfqpoint{0.602566in}{0.711351in}}%
\pgfpathmoveto{\pgfqpoint{0.647969in}{0.682601in}}%
\pgfpathlineto{\pgfqpoint{0.695822in}{0.656509in}}%
\pgfpathmoveto{\pgfqpoint{0.407330in}{0.734379in}}%
\pgfpathlineto{\pgfqpoint{0.602566in}{0.711351in}}%
\pgfpathmoveto{\pgfqpoint{0.407330in}{0.734379in}}%
\pgfpathlineto{\pgfqpoint{0.241845in}{0.758819in}}%
\pgfpathmoveto{\pgfqpoint{0.197059in}{0.435954in}}%
\pgfpathlineto{\pgfqpoint{0.181023in}{0.100000in}}%
\pgfpathmoveto{\pgfqpoint{0.197059in}{0.435954in}}%
\pgfpathlineto{\pgfqpoint{0.241845in}{0.758819in}}%
\pgfpathmoveto{\pgfqpoint{1.370566in}{0.368055in}}%
\pgfpathlineto{\pgfqpoint{1.261326in}{0.100000in}}%
\pgfpathmoveto{\pgfqpoint{1.081725in}{0.349008in}}%
\pgfpathlineto{\pgfqpoint{0.991250in}{0.100000in}}%
\pgfpathmoveto{\pgfqpoint{1.223541in}{0.458479in}}%
\pgfpathlineto{\pgfqpoint{1.370566in}{0.368055in}}%
\pgfpathmoveto{\pgfqpoint{1.223541in}{0.458479in}}%
\pgfpathlineto{\pgfqpoint{1.081725in}{0.349008in}}%
\pgfpathmoveto{\pgfqpoint{0.956373in}{0.443934in}}%
\pgfpathlineto{\pgfqpoint{1.081725in}{0.349008in}}%
\pgfpathmoveto{\pgfqpoint{0.956373in}{0.443934in}}%
\pgfpathlineto{\pgfqpoint{0.821549in}{0.363441in}}%
\pgfpathmoveto{\pgfqpoint{0.690710in}{0.470697in}}%
\pgfpathlineto{\pgfqpoint{0.821549in}{0.363441in}}%
\pgfpathmoveto{\pgfqpoint{0.690710in}{0.470697in}}%
\pgfpathlineto{\pgfqpoint{0.544697in}{0.384052in}}%
\pgfpathmoveto{\pgfqpoint{1.514560in}{0.493546in}}%
\pgfpathlineto{\pgfqpoint{1.784788in}{0.431960in}}%
\pgfpathmoveto{\pgfqpoint{1.514560in}{0.493546in}}%
\pgfpathlineto{\pgfqpoint{1.370566in}{0.368055in}}%
\pgfpathmoveto{\pgfqpoint{1.354379in}{0.509631in}}%
\pgfpathlineto{\pgfqpoint{1.370566in}{0.368055in}}%
\pgfpathmoveto{\pgfqpoint{1.354379in}{0.509631in}}%
\pgfpathlineto{\pgfqpoint{1.223541in}{0.458479in}}%
\pgfpathmoveto{\pgfqpoint{1.354379in}{0.509631in}}%
\pgfpathlineto{\pgfqpoint{1.514560in}{0.493546in}}%
\pgfpathmoveto{\pgfqpoint{1.088111in}{0.476529in}}%
\pgfpathlineto{\pgfqpoint{1.081725in}{0.349008in}}%
\pgfpathmoveto{\pgfqpoint{1.088111in}{0.476529in}}%
\pgfpathlineto{\pgfqpoint{1.223541in}{0.458479in}}%
\pgfpathmoveto{\pgfqpoint{1.088111in}{0.476529in}}%
\pgfpathlineto{\pgfqpoint{0.956373in}{0.443934in}}%
\pgfpathmoveto{\pgfqpoint{0.825301in}{0.482755in}}%
\pgfpathlineto{\pgfqpoint{0.821549in}{0.363441in}}%
\pgfpathmoveto{\pgfqpoint{0.825301in}{0.482755in}}%
\pgfpathlineto{\pgfqpoint{0.956373in}{0.443934in}}%
\pgfpathmoveto{\pgfqpoint{0.825301in}{0.482755in}}%
\pgfpathlineto{\pgfqpoint{0.690710in}{0.470697in}}%
\pgfpathmoveto{\pgfqpoint{0.551323in}{0.531974in}}%
\pgfpathlineto{\pgfqpoint{0.544697in}{0.384052in}}%
\pgfpathmoveto{\pgfqpoint{0.551323in}{0.531974in}}%
\pgfpathlineto{\pgfqpoint{0.690710in}{0.470697in}}%
\pgfpathmoveto{\pgfqpoint{0.381486in}{0.510355in}}%
\pgfpathlineto{\pgfqpoint{0.241845in}{0.758819in}}%
\pgfpathmoveto{\pgfqpoint{0.381486in}{0.510355in}}%
\pgfpathlineto{\pgfqpoint{0.407330in}{0.734379in}}%
\pgfpathmoveto{\pgfqpoint{0.381486in}{0.510355in}}%
\pgfpathlineto{\pgfqpoint{0.197059in}{0.435954in}}%
\pgfpathmoveto{\pgfqpoint{0.381486in}{0.510355in}}%
\pgfpathlineto{\pgfqpoint{0.544697in}{0.384052in}}%
\pgfpathmoveto{\pgfqpoint{0.381486in}{0.510355in}}%
\pgfpathlineto{\pgfqpoint{0.551323in}{0.531974in}}%
\pgfpathmoveto{\pgfqpoint{1.276566in}{0.559152in}}%
\pgfpathlineto{\pgfqpoint{1.286136in}{0.655839in}}%
\pgfpathmoveto{\pgfqpoint{1.276566in}{0.559152in}}%
\pgfpathlineto{\pgfqpoint{1.234900in}{0.638023in}}%
\pgfpathmoveto{\pgfqpoint{1.276566in}{0.559152in}}%
\pgfpathlineto{\pgfqpoint{1.223541in}{0.458479in}}%
\pgfpathmoveto{\pgfqpoint{1.276566in}{0.559152in}}%
\pgfpathlineto{\pgfqpoint{1.354379in}{0.509631in}}%
\pgfpathmoveto{\pgfqpoint{1.152718in}{0.539270in}}%
\pgfpathlineto{\pgfqpoint{1.180840in}{0.624920in}}%
\pgfpathmoveto{\pgfqpoint{1.152718in}{0.539270in}}%
\pgfpathlineto{\pgfqpoint{1.119285in}{0.614311in}}%
\pgfpathmoveto{\pgfqpoint{1.152718in}{0.539270in}}%
\pgfpathlineto{\pgfqpoint{1.223541in}{0.458479in}}%
\pgfpathmoveto{\pgfqpoint{1.152718in}{0.539270in}}%
\pgfpathlineto{\pgfqpoint{1.088111in}{0.476529in}}%
\pgfpathmoveto{\pgfqpoint{1.023657in}{0.527880in}}%
\pgfpathlineto{\pgfqpoint{1.054943in}{0.605752in}}%
\pgfpathmoveto{\pgfqpoint{1.023657in}{0.527880in}}%
\pgfpathlineto{\pgfqpoint{0.990043in}{0.598875in}}%
\pgfpathmoveto{\pgfqpoint{1.023657in}{0.527880in}}%
\pgfpathlineto{\pgfqpoint{0.956373in}{0.443934in}}%
\pgfpathmoveto{\pgfqpoint{1.023657in}{0.527880in}}%
\pgfpathlineto{\pgfqpoint{1.088111in}{0.476529in}}%
\pgfpathmoveto{\pgfqpoint{0.891682in}{0.530129in}}%
\pgfpathlineto{\pgfqpoint{0.924598in}{0.600413in}}%
\pgfpathmoveto{\pgfqpoint{0.891682in}{0.530129in}}%
\pgfpathlineto{\pgfqpoint{0.861703in}{0.610513in}}%
\pgfpathmoveto{\pgfqpoint{0.891682in}{0.530129in}}%
\pgfpathlineto{\pgfqpoint{0.956373in}{0.443934in}}%
\pgfpathmoveto{\pgfqpoint{0.891682in}{0.530129in}}%
\pgfpathlineto{\pgfqpoint{0.825301in}{0.482755in}}%
\pgfpathmoveto{\pgfqpoint{0.765740in}{0.547058in}}%
\pgfpathlineto{\pgfqpoint{0.800345in}{0.623839in}}%
\pgfpathmoveto{\pgfqpoint{0.765740in}{0.547058in}}%
\pgfpathlineto{\pgfqpoint{0.746146in}{0.638036in}}%
\pgfpathmoveto{\pgfqpoint{0.765740in}{0.547058in}}%
\pgfpathlineto{\pgfqpoint{0.690710in}{0.470697in}}%
\pgfpathmoveto{\pgfqpoint{0.765740in}{0.547058in}}%
\pgfpathlineto{\pgfqpoint{0.825301in}{0.482755in}}%
\pgfpathmoveto{\pgfqpoint{0.644521in}{0.578018in}}%
\pgfpathlineto{\pgfqpoint{0.695822in}{0.656509in}}%
\pgfpathmoveto{\pgfqpoint{0.644521in}{0.578018in}}%
\pgfpathlineto{\pgfqpoint{0.647969in}{0.682601in}}%
\pgfpathmoveto{\pgfqpoint{0.644521in}{0.578018in}}%
\pgfpathlineto{\pgfqpoint{0.690710in}{0.470697in}}%
\pgfpathmoveto{\pgfqpoint{0.644521in}{0.578018in}}%
\pgfpathlineto{\pgfqpoint{0.551323in}{0.531974in}}%
\pgfpathmoveto{\pgfqpoint{1.407587in}{0.599878in}}%
\pgfpathlineto{\pgfqpoint{1.377693in}{0.703727in}}%
\pgfpathmoveto{\pgfqpoint{1.407587in}{0.599878in}}%
\pgfpathlineto{\pgfqpoint{1.333507in}{0.678982in}}%
\pgfpathmoveto{\pgfqpoint{1.407587in}{0.599878in}}%
\pgfpathlineto{\pgfqpoint{1.514560in}{0.493546in}}%
\pgfpathmoveto{\pgfqpoint{1.407587in}{0.599878in}}%
\pgfpathlineto{\pgfqpoint{1.354379in}{0.509631in}}%
\pgfpathmoveto{\pgfqpoint{0.938577in}{0.290022in}}%
\pgfpathlineto{\pgfqpoint{0.991250in}{0.100000in}}%
\pgfpathmoveto{\pgfqpoint{0.938577in}{0.290022in}}%
\pgfpathlineto{\pgfqpoint{1.081725in}{0.349008in}}%
\pgfpathmoveto{\pgfqpoint{0.938577in}{0.290022in}}%
\pgfpathlineto{\pgfqpoint{0.821549in}{0.363441in}}%
\pgfpathmoveto{\pgfqpoint{0.938577in}{0.290022in}}%
\pgfpathlineto{\pgfqpoint{0.956373in}{0.443934in}}%
\pgfpathmoveto{\pgfqpoint{0.481120in}{0.621443in}}%
\pgfpathlineto{\pgfqpoint{0.602566in}{0.711351in}}%
\pgfpathmoveto{\pgfqpoint{0.481120in}{0.621443in}}%
\pgfpathlineto{\pgfqpoint{0.407330in}{0.734379in}}%
\pgfpathmoveto{\pgfqpoint{0.481120in}{0.621443in}}%
\pgfpathlineto{\pgfqpoint{0.551323in}{0.531974in}}%
\pgfpathmoveto{\pgfqpoint{0.481120in}{0.621443in}}%
\pgfpathlineto{\pgfqpoint{0.381486in}{0.510355in}}%
\pgfpathmoveto{\pgfqpoint{1.332349in}{0.597525in}}%
\pgfpathlineto{\pgfqpoint{1.333507in}{0.678982in}}%
\pgfpathmoveto{\pgfqpoint{1.332349in}{0.597525in}}%
\pgfpathlineto{\pgfqpoint{1.286136in}{0.655839in}}%
\pgfpathmoveto{\pgfqpoint{1.332349in}{0.597525in}}%
\pgfpathlineto{\pgfqpoint{1.354379in}{0.509631in}}%
\pgfpathmoveto{\pgfqpoint{1.332349in}{0.597525in}}%
\pgfpathlineto{\pgfqpoint{1.276566in}{0.559152in}}%
\pgfpathmoveto{\pgfqpoint{1.332349in}{0.597525in}}%
\pgfpathlineto{\pgfqpoint{1.407587in}{0.599878in}}%
\pgfpathmoveto{\pgfqpoint{1.087875in}{0.552554in}}%
\pgfpathlineto{\pgfqpoint{1.119285in}{0.614311in}}%
\pgfpathmoveto{\pgfqpoint{1.087875in}{0.552554in}}%
\pgfpathlineto{\pgfqpoint{1.054943in}{0.605752in}}%
\pgfpathmoveto{\pgfqpoint{1.087875in}{0.552554in}}%
\pgfpathlineto{\pgfqpoint{1.088111in}{0.476529in}}%
\pgfpathmoveto{\pgfqpoint{1.087875in}{0.552554in}}%
\pgfpathlineto{\pgfqpoint{1.152718in}{0.539270in}}%
\pgfpathmoveto{\pgfqpoint{1.087875in}{0.552554in}}%
\pgfpathlineto{\pgfqpoint{1.023657in}{0.527880in}}%
\pgfpathmoveto{\pgfqpoint{1.213366in}{0.566671in}}%
\pgfpathlineto{\pgfqpoint{1.234900in}{0.638023in}}%
\pgfpathmoveto{\pgfqpoint{1.213366in}{0.566671in}}%
\pgfpathlineto{\pgfqpoint{1.180840in}{0.624920in}}%
\pgfpathmoveto{\pgfqpoint{1.213366in}{0.566671in}}%
\pgfpathlineto{\pgfqpoint{1.223541in}{0.458479in}}%
\pgfpathmoveto{\pgfqpoint{1.213366in}{0.566671in}}%
\pgfpathlineto{\pgfqpoint{1.276566in}{0.559152in}}%
\pgfpathmoveto{\pgfqpoint{1.213366in}{0.566671in}}%
\pgfpathlineto{\pgfqpoint{1.152718in}{0.539270in}}%
\pgfpathmoveto{\pgfqpoint{0.957452in}{0.539685in}}%
\pgfpathlineto{\pgfqpoint{0.990043in}{0.598875in}}%
\pgfpathmoveto{\pgfqpoint{0.957452in}{0.539685in}}%
\pgfpathlineto{\pgfqpoint{0.924598in}{0.600413in}}%
\pgfpathmoveto{\pgfqpoint{0.957452in}{0.539685in}}%
\pgfpathlineto{\pgfqpoint{0.956373in}{0.443934in}}%
\pgfpathmoveto{\pgfqpoint{0.957452in}{0.539685in}}%
\pgfpathlineto{\pgfqpoint{1.023657in}{0.527880in}}%
\pgfpathmoveto{\pgfqpoint{0.957452in}{0.539685in}}%
\pgfpathlineto{\pgfqpoint{0.891682in}{0.530129in}}%
\pgfpathmoveto{\pgfqpoint{0.828623in}{0.557099in}}%
\pgfpathlineto{\pgfqpoint{0.861703in}{0.610513in}}%
\pgfpathmoveto{\pgfqpoint{0.828623in}{0.557099in}}%
\pgfpathlineto{\pgfqpoint{0.800345in}{0.623839in}}%
\pgfpathmoveto{\pgfqpoint{0.828623in}{0.557099in}}%
\pgfpathlineto{\pgfqpoint{0.825301in}{0.482755in}}%
\pgfpathmoveto{\pgfqpoint{0.828623in}{0.557099in}}%
\pgfpathlineto{\pgfqpoint{0.891682in}{0.530129in}}%
\pgfpathmoveto{\pgfqpoint{0.828623in}{0.557099in}}%
\pgfpathlineto{\pgfqpoint{0.765740in}{0.547058in}}%
\pgfpathmoveto{\pgfqpoint{1.188848in}{0.271931in}}%
\pgfpathlineto{\pgfqpoint{0.991250in}{0.100000in}}%
\pgfpathmoveto{\pgfqpoint{1.188848in}{0.271931in}}%
\pgfpathlineto{\pgfqpoint{1.261326in}{0.100000in}}%
\pgfpathmoveto{\pgfqpoint{1.188848in}{0.271931in}}%
\pgfpathlineto{\pgfqpoint{1.370566in}{0.368055in}}%
\pgfpathmoveto{\pgfqpoint{1.188848in}{0.271931in}}%
\pgfpathlineto{\pgfqpoint{1.081725in}{0.349008in}}%
\pgfpathmoveto{\pgfqpoint{1.188848in}{0.271931in}}%
\pgfpathlineto{\pgfqpoint{1.223541in}{0.458479in}}%
\pgfpathmoveto{\pgfqpoint{0.693813in}{0.285882in}}%
\pgfpathlineto{\pgfqpoint{0.721174in}{0.100000in}}%
\pgfpathmoveto{\pgfqpoint{0.693813in}{0.285882in}}%
\pgfpathlineto{\pgfqpoint{0.821549in}{0.363441in}}%
\pgfpathmoveto{\pgfqpoint{0.693813in}{0.285882in}}%
\pgfpathlineto{\pgfqpoint{0.544697in}{0.384052in}}%
\pgfpathmoveto{\pgfqpoint{0.693813in}{0.285882in}}%
\pgfpathlineto{\pgfqpoint{0.690710in}{0.470697in}}%
\pgfpathmoveto{\pgfqpoint{0.707873in}{0.576304in}}%
\pgfpathlineto{\pgfqpoint{0.746146in}{0.638036in}}%
\pgfpathmoveto{\pgfqpoint{0.707873in}{0.576304in}}%
\pgfpathlineto{\pgfqpoint{0.695822in}{0.656509in}}%
\pgfpathmoveto{\pgfqpoint{0.707873in}{0.576304in}}%
\pgfpathlineto{\pgfqpoint{0.690710in}{0.470697in}}%
\pgfpathmoveto{\pgfqpoint{0.707873in}{0.576304in}}%
\pgfpathlineto{\pgfqpoint{0.765740in}{0.547058in}}%
\pgfpathmoveto{\pgfqpoint{0.707873in}{0.576304in}}%
\pgfpathlineto{\pgfqpoint{0.644521in}{0.578018in}}%
\pgfpathmoveto{\pgfqpoint{1.505766in}{0.629724in}}%
\pgfpathlineto{\pgfqpoint{1.377693in}{0.703727in}}%
\pgfpathmoveto{\pgfqpoint{1.505766in}{0.629724in}}%
\pgfpathlineto{\pgfqpoint{1.575146in}{0.731227in}}%
\pgfpathmoveto{\pgfqpoint{1.505766in}{0.629724in}}%
\pgfpathlineto{\pgfqpoint{1.514560in}{0.493546in}}%
\pgfpathmoveto{\pgfqpoint{1.505766in}{0.629724in}}%
\pgfpathlineto{\pgfqpoint{1.407587in}{0.599878in}}%
\pgfpathmoveto{\pgfqpoint{0.592313in}{0.620955in}}%
\pgfpathlineto{\pgfqpoint{0.602566in}{0.711351in}}%
\pgfpathmoveto{\pgfqpoint{0.592313in}{0.620955in}}%
\pgfpathlineto{\pgfqpoint{0.647969in}{0.682601in}}%
\pgfpathmoveto{\pgfqpoint{0.592313in}{0.620955in}}%
\pgfpathlineto{\pgfqpoint{0.551323in}{0.531974in}}%
\pgfpathmoveto{\pgfqpoint{0.592313in}{0.620955in}}%
\pgfpathlineto{\pgfqpoint{0.644521in}{0.578018in}}%
\pgfpathmoveto{\pgfqpoint{0.592313in}{0.620955in}}%
\pgfpathlineto{\pgfqpoint{0.481120in}{0.621443in}}%
\pgfpathmoveto{\pgfqpoint{0.380766in}{0.291207in}}%
\pgfpathlineto{\pgfqpoint{0.181023in}{0.100000in}}%
\pgfpathmoveto{\pgfqpoint{0.380766in}{0.291207in}}%
\pgfpathlineto{\pgfqpoint{0.451098in}{0.100000in}}%
\pgfpathmoveto{\pgfqpoint{0.380766in}{0.291207in}}%
\pgfpathlineto{\pgfqpoint{0.197059in}{0.435954in}}%
\pgfpathmoveto{\pgfqpoint{0.380766in}{0.291207in}}%
\pgfpathlineto{\pgfqpoint{0.544697in}{0.384052in}}%
\pgfpathmoveto{\pgfqpoint{0.380766in}{0.291207in}}%
\pgfpathlineto{\pgfqpoint{0.381486in}{0.510355in}}%
\pgfpathmoveto{\pgfqpoint{1.529279in}{0.294989in}}%
\pgfpathlineto{\pgfqpoint{1.801477in}{0.100000in}}%
\pgfpathmoveto{\pgfqpoint{1.529279in}{0.294989in}}%
\pgfpathlineto{\pgfqpoint{1.261326in}{0.100000in}}%
\pgfpathmoveto{\pgfqpoint{1.529279in}{0.294989in}}%
\pgfpathlineto{\pgfqpoint{1.531402in}{0.100000in}}%
\pgfpathmoveto{\pgfqpoint{1.529279in}{0.294989in}}%
\pgfpathlineto{\pgfqpoint{1.784788in}{0.431960in}}%
\pgfpathmoveto{\pgfqpoint{1.529279in}{0.294989in}}%
\pgfpathlineto{\pgfqpoint{1.370566in}{0.368055in}}%
\pgfpathmoveto{\pgfqpoint{1.529279in}{0.294989in}}%
\pgfpathlineto{\pgfqpoint{1.514560in}{0.493546in}}%
\pgfpathmoveto{\pgfqpoint{1.624417in}{0.609671in}}%
\pgfpathlineto{\pgfqpoint{1.740685in}{0.754365in}}%
\pgfpathmoveto{\pgfqpoint{1.624417in}{0.609671in}}%
\pgfpathlineto{\pgfqpoint{1.784788in}{0.431960in}}%
\pgfpathmoveto{\pgfqpoint{1.624417in}{0.609671in}}%
\pgfpathlineto{\pgfqpoint{1.575146in}{0.731227in}}%
\pgfpathmoveto{\pgfqpoint{1.624417in}{0.609671in}}%
\pgfpathlineto{\pgfqpoint{1.514560in}{0.493546in}}%
\pgfpathmoveto{\pgfqpoint{1.624417in}{0.609671in}}%
\pgfpathlineto{\pgfqpoint{1.505766in}{0.629724in}}%
\pgfpathmoveto{\pgfqpoint{0.840111in}{0.209031in}}%
\pgfpathlineto{\pgfqpoint{0.721174in}{0.100000in}}%
\pgfpathmoveto{\pgfqpoint{0.840111in}{0.209031in}}%
\pgfpathlineto{\pgfqpoint{0.991250in}{0.100000in}}%
\pgfpathmoveto{\pgfqpoint{0.840111in}{0.209031in}}%
\pgfpathlineto{\pgfqpoint{0.821549in}{0.363441in}}%
\pgfpathmoveto{\pgfqpoint{0.840111in}{0.209031in}}%
\pgfpathlineto{\pgfqpoint{0.938577in}{0.290022in}}%
\pgfpathmoveto{\pgfqpoint{0.840111in}{0.209031in}}%
\pgfpathlineto{\pgfqpoint{0.693813in}{0.285882in}}%
\pgfpathmoveto{\pgfqpoint{0.554609in}{0.230276in}}%
\pgfpathlineto{\pgfqpoint{0.451098in}{0.100000in}}%
\pgfpathmoveto{\pgfqpoint{0.554609in}{0.230276in}}%
\pgfpathlineto{\pgfqpoint{0.721174in}{0.100000in}}%
\pgfpathmoveto{\pgfqpoint{0.554609in}{0.230276in}}%
\pgfpathlineto{\pgfqpoint{0.544697in}{0.384052in}}%
\pgfpathmoveto{\pgfqpoint{0.554609in}{0.230276in}}%
\pgfpathlineto{\pgfqpoint{0.693813in}{0.285882in}}%
\pgfpathmoveto{\pgfqpoint{0.554609in}{0.230276in}}%
\pgfpathlineto{\pgfqpoint{0.380766in}{0.291207in}}%
\pgfpathlineto{\pgfqpoint{0.380766in}{0.291207in}}%
\pgfusepath{stroke}%
\end{pgfscope}%
\begin{pgfscope}%
\pgfpathrectangle{\pgfqpoint{0.100000in}{0.100000in}}{\pgfqpoint{1.782500in}{1.232000in}}%
\pgfusepath{clip}%
\pgfsetrectcap%
\pgfsetroundjoin%
\pgfsetlinewidth{0.250937pt}%
\definecolor{currentstroke}{rgb}{0.835294,0.321569,0.035294}%
\pgfsetstrokecolor{currentstroke}%
\pgfsetdash{}{0pt}%
\pgfpathmoveto{\pgfqpoint{0.496062in}{0.835205in}}%
\pgfpathlineto{\pgfqpoint{0.451098in}{1.085600in}}%
\pgfpathmoveto{\pgfqpoint{1.531402in}{1.085600in}}%
\pgfpathlineto{\pgfqpoint{1.486187in}{0.833488in}}%
\pgfpathmoveto{\pgfqpoint{0.721174in}{1.085600in}}%
\pgfpathlineto{\pgfqpoint{0.991250in}{1.085600in}}%
\pgfpathmoveto{\pgfqpoint{0.721174in}{1.085600in}}%
\pgfpathlineto{\pgfqpoint{0.451098in}{1.085600in}}%
\pgfpathmoveto{\pgfqpoint{0.546727in}{0.758322in}}%
\pgfpathlineto{\pgfqpoint{0.496062in}{0.835205in}}%
\pgfpathmoveto{\pgfqpoint{0.613861in}{0.704199in}}%
\pgfpathlineto{\pgfqpoint{0.546727in}{0.758322in}}%
\pgfpathmoveto{\pgfqpoint{0.684747in}{0.662330in}}%
\pgfpathlineto{\pgfqpoint{0.613861in}{0.704199in}}%
\pgfpathmoveto{\pgfqpoint{0.763773in}{0.632922in}}%
\pgfpathlineto{\pgfqpoint{0.684747in}{0.662330in}}%
\pgfpathmoveto{\pgfqpoint{0.842615in}{0.614483in}}%
\pgfpathlineto{\pgfqpoint{0.763773in}{0.632922in}}%
\pgfpathmoveto{\pgfqpoint{0.916422in}{0.601060in}}%
\pgfpathlineto{\pgfqpoint{0.842615in}{0.614483in}}%
\pgfpathmoveto{\pgfqpoint{0.990080in}{0.598631in}}%
\pgfpathlineto{\pgfqpoint{0.916422in}{0.601060in}}%
\pgfpathmoveto{\pgfqpoint{1.063069in}{0.606844in}}%
\pgfpathlineto{\pgfqpoint{0.990080in}{0.598631in}}%
\pgfpathmoveto{\pgfqpoint{1.138803in}{0.617203in}}%
\pgfpathlineto{\pgfqpoint{1.063069in}{0.606844in}}%
\pgfpathmoveto{\pgfqpoint{1.217115in}{0.633115in}}%
\pgfpathlineto{\pgfqpoint{1.138803in}{0.617203in}}%
\pgfpathmoveto{\pgfqpoint{1.297142in}{0.661043in}}%
\pgfpathlineto{\pgfqpoint{1.217115in}{0.633115in}}%
\pgfpathmoveto{\pgfqpoint{1.366701in}{0.697571in}}%
\pgfpathlineto{\pgfqpoint{1.297142in}{0.661043in}}%
\pgfpathmoveto{\pgfqpoint{1.435831in}{0.755419in}}%
\pgfpathlineto{\pgfqpoint{1.486187in}{0.833488in}}%
\pgfpathmoveto{\pgfqpoint{1.435831in}{0.755419in}}%
\pgfpathlineto{\pgfqpoint{1.366701in}{0.697571in}}%
\pgfpathmoveto{\pgfqpoint{1.261326in}{1.085600in}}%
\pgfpathlineto{\pgfqpoint{0.991250in}{1.085600in}}%
\pgfpathmoveto{\pgfqpoint{1.261326in}{1.085600in}}%
\pgfpathlineto{\pgfqpoint{1.531402in}{1.085600in}}%
\pgfpathmoveto{\pgfqpoint{0.833368in}{0.890343in}}%
\pgfpathlineto{\pgfqpoint{0.991250in}{1.085600in}}%
\pgfpathmoveto{\pgfqpoint{0.833368in}{0.890343in}}%
\pgfpathlineto{\pgfqpoint{0.721174in}{1.085600in}}%
\pgfpathmoveto{\pgfqpoint{0.961724in}{0.757684in}}%
\pgfpathlineto{\pgfqpoint{1.113897in}{0.827208in}}%
\pgfpathmoveto{\pgfqpoint{0.961724in}{0.757684in}}%
\pgfpathlineto{\pgfqpoint{0.833368in}{0.890343in}}%
\pgfpathmoveto{\pgfqpoint{1.279672in}{0.859587in}}%
\pgfpathlineto{\pgfqpoint{1.113897in}{0.827208in}}%
\pgfpathmoveto{\pgfqpoint{0.668050in}{0.896214in}}%
\pgfpathlineto{\pgfqpoint{0.721174in}{1.085600in}}%
\pgfpathmoveto{\pgfqpoint{0.668050in}{0.896214in}}%
\pgfpathlineto{\pgfqpoint{0.833368in}{0.890343in}}%
\pgfpathmoveto{\pgfqpoint{1.087918in}{0.720099in}}%
\pgfpathlineto{\pgfqpoint{1.063069in}{0.606844in}}%
\pgfpathmoveto{\pgfqpoint{1.087918in}{0.720099in}}%
\pgfpathlineto{\pgfqpoint{1.138803in}{0.617203in}}%
\pgfpathmoveto{\pgfqpoint{1.087918in}{0.720099in}}%
\pgfpathlineto{\pgfqpoint{1.113897in}{0.827208in}}%
\pgfpathmoveto{\pgfqpoint{1.087918in}{0.720099in}}%
\pgfpathlineto{\pgfqpoint{0.961724in}{0.757684in}}%
\pgfpathmoveto{\pgfqpoint{1.208497in}{0.752670in}}%
\pgfpathlineto{\pgfqpoint{1.217115in}{0.633115in}}%
\pgfpathmoveto{\pgfqpoint{1.208497in}{0.752670in}}%
\pgfpathlineto{\pgfqpoint{1.297142in}{0.661043in}}%
\pgfpathmoveto{\pgfqpoint{1.208497in}{0.752670in}}%
\pgfpathlineto{\pgfqpoint{1.113897in}{0.827208in}}%
\pgfpathmoveto{\pgfqpoint{1.208497in}{0.752670in}}%
\pgfpathlineto{\pgfqpoint{1.279672in}{0.859587in}}%
\pgfpathmoveto{\pgfqpoint{1.208497in}{0.752670in}}%
\pgfpathlineto{\pgfqpoint{1.087918in}{0.720099in}}%
\pgfpathmoveto{\pgfqpoint{0.826322in}{0.731329in}}%
\pgfpathlineto{\pgfqpoint{0.763773in}{0.632922in}}%
\pgfpathmoveto{\pgfqpoint{0.826322in}{0.731329in}}%
\pgfpathlineto{\pgfqpoint{0.842615in}{0.614483in}}%
\pgfpathmoveto{\pgfqpoint{0.826322in}{0.731329in}}%
\pgfpathlineto{\pgfqpoint{0.833368in}{0.890343in}}%
\pgfpathmoveto{\pgfqpoint{0.826322in}{0.731329in}}%
\pgfpathlineto{\pgfqpoint{0.961724in}{0.757684in}}%
\pgfpathmoveto{\pgfqpoint{0.715986in}{0.784595in}}%
\pgfpathlineto{\pgfqpoint{0.613861in}{0.704199in}}%
\pgfpathmoveto{\pgfqpoint{0.715986in}{0.784595in}}%
\pgfpathlineto{\pgfqpoint{0.684747in}{0.662330in}}%
\pgfpathmoveto{\pgfqpoint{0.715986in}{0.784595in}}%
\pgfpathlineto{\pgfqpoint{0.833368in}{0.890343in}}%
\pgfpathmoveto{\pgfqpoint{0.715986in}{0.784595in}}%
\pgfpathlineto{\pgfqpoint{0.668050in}{0.896214in}}%
\pgfpathmoveto{\pgfqpoint{0.715986in}{0.784595in}}%
\pgfpathlineto{\pgfqpoint{0.826322in}{0.731329in}}%
\pgfpathmoveto{\pgfqpoint{1.369718in}{0.959100in}}%
\pgfpathlineto{\pgfqpoint{1.486187in}{0.833488in}}%
\pgfpathmoveto{\pgfqpoint{1.369718in}{0.959100in}}%
\pgfpathlineto{\pgfqpoint{1.531402in}{1.085600in}}%
\pgfpathmoveto{\pgfqpoint{1.369718in}{0.959100in}}%
\pgfpathlineto{\pgfqpoint{1.261326in}{1.085600in}}%
\pgfpathmoveto{\pgfqpoint{1.369718in}{0.959100in}}%
\pgfpathlineto{\pgfqpoint{1.279672in}{0.859587in}}%
\pgfpathmoveto{\pgfqpoint{0.955632in}{0.657941in}}%
\pgfpathlineto{\pgfqpoint{0.916422in}{0.601060in}}%
\pgfpathmoveto{\pgfqpoint{0.955632in}{0.657941in}}%
\pgfpathlineto{\pgfqpoint{0.990080in}{0.598631in}}%
\pgfpathmoveto{\pgfqpoint{0.955632in}{0.657941in}}%
\pgfpathlineto{\pgfqpoint{0.961724in}{0.757684in}}%
\pgfpathmoveto{\pgfqpoint{1.292856in}{0.745563in}}%
\pgfpathlineto{\pgfqpoint{1.297142in}{0.661043in}}%
\pgfpathmoveto{\pgfqpoint{1.292856in}{0.745563in}}%
\pgfpathlineto{\pgfqpoint{1.366701in}{0.697571in}}%
\pgfpathmoveto{\pgfqpoint{1.292856in}{0.745563in}}%
\pgfpathlineto{\pgfqpoint{1.279672in}{0.859587in}}%
\pgfpathmoveto{\pgfqpoint{1.292856in}{0.745563in}}%
\pgfpathlineto{\pgfqpoint{1.208497in}{0.752670in}}%
\pgfpathmoveto{\pgfqpoint{1.161401in}{0.688685in}}%
\pgfpathlineto{\pgfqpoint{1.138803in}{0.617203in}}%
\pgfpathmoveto{\pgfqpoint{1.161401in}{0.688685in}}%
\pgfpathlineto{\pgfqpoint{1.217115in}{0.633115in}}%
\pgfpathmoveto{\pgfqpoint{1.161401in}{0.688685in}}%
\pgfpathlineto{\pgfqpoint{1.087918in}{0.720099in}}%
\pgfpathmoveto{\pgfqpoint{1.161401in}{0.688685in}}%
\pgfpathlineto{\pgfqpoint{1.208497in}{0.752670in}}%
\pgfpathmoveto{\pgfqpoint{1.382877in}{0.853640in}}%
\pgfpathlineto{\pgfqpoint{1.486187in}{0.833488in}}%
\pgfpathmoveto{\pgfqpoint{1.382877in}{0.853640in}}%
\pgfpathlineto{\pgfqpoint{1.435831in}{0.755419in}}%
\pgfpathmoveto{\pgfqpoint{1.382877in}{0.853640in}}%
\pgfpathlineto{\pgfqpoint{1.279672in}{0.859587in}}%
\pgfpathmoveto{\pgfqpoint{1.382877in}{0.853640in}}%
\pgfpathlineto{\pgfqpoint{1.369718in}{0.959100in}}%
\pgfpathmoveto{\pgfqpoint{0.750725in}{0.710054in}}%
\pgfpathlineto{\pgfqpoint{0.684747in}{0.662330in}}%
\pgfpathmoveto{\pgfqpoint{0.750725in}{0.710054in}}%
\pgfpathlineto{\pgfqpoint{0.763773in}{0.632922in}}%
\pgfpathmoveto{\pgfqpoint{0.750725in}{0.710054in}}%
\pgfpathlineto{\pgfqpoint{0.826322in}{0.731329in}}%
\pgfpathmoveto{\pgfqpoint{0.750725in}{0.710054in}}%
\pgfpathlineto{\pgfqpoint{0.715986in}{0.784595in}}%
\pgfpathmoveto{\pgfqpoint{0.631249in}{0.790338in}}%
\pgfpathlineto{\pgfqpoint{0.546727in}{0.758322in}}%
\pgfpathmoveto{\pgfqpoint{0.631249in}{0.790338in}}%
\pgfpathlineto{\pgfqpoint{0.613861in}{0.704199in}}%
\pgfpathmoveto{\pgfqpoint{0.631249in}{0.790338in}}%
\pgfpathlineto{\pgfqpoint{0.668050in}{0.896214in}}%
\pgfpathmoveto{\pgfqpoint{0.631249in}{0.790338in}}%
\pgfpathlineto{\pgfqpoint{0.715986in}{0.784595in}}%
\pgfpathmoveto{\pgfqpoint{0.565758in}{0.964836in}}%
\pgfpathlineto{\pgfqpoint{0.451098in}{1.085600in}}%
\pgfpathmoveto{\pgfqpoint{0.565758in}{0.964836in}}%
\pgfpathlineto{\pgfqpoint{0.496062in}{0.835205in}}%
\pgfpathmoveto{\pgfqpoint{0.565758in}{0.964836in}}%
\pgfpathlineto{\pgfqpoint{0.721174in}{1.085600in}}%
\pgfpathmoveto{\pgfqpoint{0.565758in}{0.964836in}}%
\pgfpathlineto{\pgfqpoint{0.668050in}{0.896214in}}%
\pgfpathmoveto{\pgfqpoint{1.024300in}{0.661701in}}%
\pgfpathlineto{\pgfqpoint{0.990080in}{0.598631in}}%
\pgfpathmoveto{\pgfqpoint{1.024300in}{0.661701in}}%
\pgfpathlineto{\pgfqpoint{1.063069in}{0.606844in}}%
\pgfpathmoveto{\pgfqpoint{1.024300in}{0.661701in}}%
\pgfpathlineto{\pgfqpoint{0.961724in}{0.757684in}}%
\pgfpathmoveto{\pgfqpoint{1.024300in}{0.661701in}}%
\pgfpathlineto{\pgfqpoint{1.087918in}{0.720099in}}%
\pgfpathmoveto{\pgfqpoint{1.024300in}{0.661701in}}%
\pgfpathlineto{\pgfqpoint{0.955632in}{0.657941in}}%
\pgfpathmoveto{\pgfqpoint{0.887232in}{0.667312in}}%
\pgfpathlineto{\pgfqpoint{0.842615in}{0.614483in}}%
\pgfpathmoveto{\pgfqpoint{0.887232in}{0.667312in}}%
\pgfpathlineto{\pgfqpoint{0.916422in}{0.601060in}}%
\pgfpathmoveto{\pgfqpoint{0.887232in}{0.667312in}}%
\pgfpathlineto{\pgfqpoint{0.961724in}{0.757684in}}%
\pgfpathmoveto{\pgfqpoint{0.887232in}{0.667312in}}%
\pgfpathlineto{\pgfqpoint{0.826322in}{0.731329in}}%
\pgfpathmoveto{\pgfqpoint{0.887232in}{0.667312in}}%
\pgfpathlineto{\pgfqpoint{0.955632in}{0.657941in}}%
\pgfpathmoveto{\pgfqpoint{1.359582in}{0.777756in}}%
\pgfpathlineto{\pgfqpoint{1.366701in}{0.697571in}}%
\pgfpathmoveto{\pgfqpoint{1.359582in}{0.777756in}}%
\pgfpathlineto{\pgfqpoint{1.435831in}{0.755419in}}%
\pgfpathmoveto{\pgfqpoint{1.359582in}{0.777756in}}%
\pgfpathlineto{\pgfqpoint{1.279672in}{0.859587in}}%
\pgfpathmoveto{\pgfqpoint{1.359582in}{0.777756in}}%
\pgfpathlineto{\pgfqpoint{1.292856in}{0.745563in}}%
\pgfpathmoveto{\pgfqpoint{1.359582in}{0.777756in}}%
\pgfpathlineto{\pgfqpoint{1.382877in}{0.853640in}}%
\pgfpathmoveto{\pgfqpoint{1.015543in}{0.908082in}}%
\pgfpathlineto{\pgfqpoint{0.991250in}{1.085600in}}%
\pgfpathmoveto{\pgfqpoint{1.015543in}{0.908082in}}%
\pgfpathlineto{\pgfqpoint{1.113897in}{0.827208in}}%
\pgfpathmoveto{\pgfqpoint{1.015543in}{0.908082in}}%
\pgfpathlineto{\pgfqpoint{0.833368in}{0.890343in}}%
\pgfpathmoveto{\pgfqpoint{1.015543in}{0.908082in}}%
\pgfpathlineto{\pgfqpoint{0.961724in}{0.757684in}}%
\pgfpathmoveto{\pgfqpoint{1.174253in}{0.955154in}}%
\pgfpathlineto{\pgfqpoint{0.991250in}{1.085600in}}%
\pgfpathmoveto{\pgfqpoint{1.174253in}{0.955154in}}%
\pgfpathlineto{\pgfqpoint{1.261326in}{1.085600in}}%
\pgfpathmoveto{\pgfqpoint{1.174253in}{0.955154in}}%
\pgfpathlineto{\pgfqpoint{1.113897in}{0.827208in}}%
\pgfpathmoveto{\pgfqpoint{1.174253in}{0.955154in}}%
\pgfpathlineto{\pgfqpoint{1.279672in}{0.859587in}}%
\pgfpathmoveto{\pgfqpoint{1.174253in}{0.955154in}}%
\pgfpathlineto{\pgfqpoint{1.369718in}{0.959100in}}%
\pgfpathmoveto{\pgfqpoint{1.174253in}{0.955154in}}%
\pgfpathlineto{\pgfqpoint{1.015543in}{0.908082in}}%
\pgfpathmoveto{\pgfqpoint{0.579353in}{0.848924in}}%
\pgfpathlineto{\pgfqpoint{0.496062in}{0.835205in}}%
\pgfpathmoveto{\pgfqpoint{0.579353in}{0.848924in}}%
\pgfpathlineto{\pgfqpoint{0.546727in}{0.758322in}}%
\pgfpathmoveto{\pgfqpoint{0.579353in}{0.848924in}}%
\pgfpathlineto{\pgfqpoint{0.668050in}{0.896214in}}%
\pgfpathmoveto{\pgfqpoint{0.579353in}{0.848924in}}%
\pgfpathlineto{\pgfqpoint{0.631249in}{0.790338in}}%
\pgfpathmoveto{\pgfqpoint{0.579353in}{0.848924in}}%
\pgfpathlineto{\pgfqpoint{0.565758in}{0.964836in}}%
\pgfpathlineto{\pgfqpoint{0.565758in}{0.964836in}}%
\pgfusepath{stroke}%
\end{pgfscope}%
\begin{pgfscope}%
\pgfpathrectangle{\pgfqpoint{0.100000in}{0.100000in}}{\pgfqpoint{1.782500in}{1.232000in}}%
\pgfusepath{clip}%
\pgfsetbuttcap%
\pgfsetroundjoin%
\definecolor{currentfill}{rgb}{0.054902,0.262745,0.486275}%
\pgfsetfillcolor{currentfill}%
\pgfsetlinewidth{1.003750pt}%
\definecolor{currentstroke}{rgb}{0.054902,0.262745,0.486275}%
\pgfsetstrokecolor{currentstroke}%
\pgfsetdash{}{0pt}%
\pgfsys@defobject{currentmarker}{\pgfqpoint{-0.018373in}{-0.018373in}}{\pgfqpoint{0.018373in}{0.018373in}}{%
\pgfpathmoveto{\pgfqpoint{0.000000in}{-0.018373in}}%
\pgfpathcurveto{\pgfqpoint{0.004873in}{-0.018373in}}{\pgfqpoint{0.009546in}{-0.016437in}}{\pgfqpoint{0.012992in}{-0.012992in}}%
\pgfpathcurveto{\pgfqpoint{0.016437in}{-0.009546in}}{\pgfqpoint{0.018373in}{-0.004873in}}{\pgfqpoint{0.018373in}{0.000000in}}%
\pgfpathcurveto{\pgfqpoint{0.018373in}{0.004873in}}{\pgfqpoint{0.016437in}{0.009546in}}{\pgfqpoint{0.012992in}{0.012992in}}%
\pgfpathcurveto{\pgfqpoint{0.009546in}{0.016437in}}{\pgfqpoint{0.004873in}{0.018373in}}{\pgfqpoint{0.000000in}{0.018373in}}%
\pgfpathcurveto{\pgfqpoint{-0.004873in}{0.018373in}}{\pgfqpoint{-0.009546in}{0.016437in}}{\pgfqpoint{-0.012992in}{0.012992in}}%
\pgfpathcurveto{\pgfqpoint{-0.016437in}{0.009546in}}{\pgfqpoint{-0.018373in}{0.004873in}}{\pgfqpoint{-0.018373in}{0.000000in}}%
\pgfpathcurveto{\pgfqpoint{-0.018373in}{-0.004873in}}{\pgfqpoint{-0.016437in}{-0.009546in}}{\pgfqpoint{-0.012992in}{-0.012992in}}%
\pgfpathcurveto{\pgfqpoint{-0.009546in}{-0.016437in}}{\pgfqpoint{-0.004873in}{-0.018373in}}{\pgfqpoint{0.000000in}{-0.018373in}}%
\pgfpathlineto{\pgfqpoint{0.000000in}{-0.018373in}}%
\pgfpathclose%
\pgfusepath{stroke,fill}%
}%
\begin{pgfscope}%
\pgfsys@transformshift{1.377693in}{0.703727in}%
\pgfsys@useobject{currentmarker}{}%
\end{pgfscope}%
\begin{pgfscope}%
\pgfsys@transformshift{0.602566in}{0.711351in}%
\pgfsys@useobject{currentmarker}{}%
\end{pgfscope}%
\begin{pgfscope}%
\pgfsys@transformshift{1.333507in}{0.678982in}%
\pgfsys@useobject{currentmarker}{}%
\end{pgfscope}%
\begin{pgfscope}%
\pgfsys@transformshift{1.286136in}{0.655839in}%
\pgfsys@useobject{currentmarker}{}%
\end{pgfscope}%
\begin{pgfscope}%
\pgfsys@transformshift{1.234900in}{0.638023in}%
\pgfsys@useobject{currentmarker}{}%
\end{pgfscope}%
\begin{pgfscope}%
\pgfsys@transformshift{1.180840in}{0.624920in}%
\pgfsys@useobject{currentmarker}{}%
\end{pgfscope}%
\begin{pgfscope}%
\pgfsys@transformshift{1.119285in}{0.614311in}%
\pgfsys@useobject{currentmarker}{}%
\end{pgfscope}%
\begin{pgfscope}%
\pgfsys@transformshift{1.054943in}{0.605752in}%
\pgfsys@useobject{currentmarker}{}%
\end{pgfscope}%
\begin{pgfscope}%
\pgfsys@transformshift{0.990043in}{0.598875in}%
\pgfsys@useobject{currentmarker}{}%
\end{pgfscope}%
\begin{pgfscope}%
\pgfsys@transformshift{0.924598in}{0.600413in}%
\pgfsys@useobject{currentmarker}{}%
\end{pgfscope}%
\begin{pgfscope}%
\pgfsys@transformshift{0.861703in}{0.610513in}%
\pgfsys@useobject{currentmarker}{}%
\end{pgfscope}%
\begin{pgfscope}%
\pgfsys@transformshift{0.800345in}{0.623839in}%
\pgfsys@useobject{currentmarker}{}%
\end{pgfscope}%
\begin{pgfscope}%
\pgfsys@transformshift{0.746146in}{0.638036in}%
\pgfsys@useobject{currentmarker}{}%
\end{pgfscope}%
\begin{pgfscope}%
\pgfsys@transformshift{0.695822in}{0.656509in}%
\pgfsys@useobject{currentmarker}{}%
\end{pgfscope}%
\begin{pgfscope}%
\pgfsys@transformshift{0.647969in}{0.682601in}%
\pgfsys@useobject{currentmarker}{}%
\end{pgfscope}%
\end{pgfscope}%
\begin{pgfscope}%
\pgfpathrectangle{\pgfqpoint{0.100000in}{0.100000in}}{\pgfqpoint{1.782500in}{1.232000in}}%
\pgfusepath{clip}%
\pgfsetbuttcap%
\pgfsetroundjoin%
\definecolor{currentfill}{rgb}{0.835294,0.321569,0.035294}%
\pgfsetfillcolor{currentfill}%
\pgfsetlinewidth{1.003750pt}%
\definecolor{currentstroke}{rgb}{0.835294,0.321569,0.035294}%
\pgfsetstrokecolor{currentstroke}%
\pgfsetdash{}{0pt}%
\pgfsys@defobject{currentmarker}{\pgfqpoint{-0.018373in}{-0.018373in}}{\pgfqpoint{0.018373in}{0.018373in}}{%
\pgfpathmoveto{\pgfqpoint{0.000000in}{-0.018373in}}%
\pgfpathcurveto{\pgfqpoint{0.004873in}{-0.018373in}}{\pgfqpoint{0.009546in}{-0.016437in}}{\pgfqpoint{0.012992in}{-0.012992in}}%
\pgfpathcurveto{\pgfqpoint{0.016437in}{-0.009546in}}{\pgfqpoint{0.018373in}{-0.004873in}}{\pgfqpoint{0.018373in}{0.000000in}}%
\pgfpathcurveto{\pgfqpoint{0.018373in}{0.004873in}}{\pgfqpoint{0.016437in}{0.009546in}}{\pgfqpoint{0.012992in}{0.012992in}}%
\pgfpathcurveto{\pgfqpoint{0.009546in}{0.016437in}}{\pgfqpoint{0.004873in}{0.018373in}}{\pgfqpoint{0.000000in}{0.018373in}}%
\pgfpathcurveto{\pgfqpoint{-0.004873in}{0.018373in}}{\pgfqpoint{-0.009546in}{0.016437in}}{\pgfqpoint{-0.012992in}{0.012992in}}%
\pgfpathcurveto{\pgfqpoint{-0.016437in}{0.009546in}}{\pgfqpoint{-0.018373in}{0.004873in}}{\pgfqpoint{-0.018373in}{0.000000in}}%
\pgfpathcurveto{\pgfqpoint{-0.018373in}{-0.004873in}}{\pgfqpoint{-0.016437in}{-0.009546in}}{\pgfqpoint{-0.012992in}{-0.012992in}}%
\pgfpathcurveto{\pgfqpoint{-0.009546in}{-0.016437in}}{\pgfqpoint{-0.004873in}{-0.018373in}}{\pgfqpoint{0.000000in}{-0.018373in}}%
\pgfpathlineto{\pgfqpoint{0.000000in}{-0.018373in}}%
\pgfpathclose%
\pgfusepath{stroke,fill}%
}%
\begin{pgfscope}%
\pgfsys@transformshift{0.613861in}{0.704199in}%
\pgfsys@useobject{currentmarker}{}%
\end{pgfscope}%
\begin{pgfscope}%
\pgfsys@transformshift{0.684747in}{0.662330in}%
\pgfsys@useobject{currentmarker}{}%
\end{pgfscope}%
\begin{pgfscope}%
\pgfsys@transformshift{0.763773in}{0.632922in}%
\pgfsys@useobject{currentmarker}{}%
\end{pgfscope}%
\begin{pgfscope}%
\pgfsys@transformshift{0.842615in}{0.614483in}%
\pgfsys@useobject{currentmarker}{}%
\end{pgfscope}%
\begin{pgfscope}%
\pgfsys@transformshift{0.916422in}{0.601060in}%
\pgfsys@useobject{currentmarker}{}%
\end{pgfscope}%
\begin{pgfscope}%
\pgfsys@transformshift{0.990080in}{0.598631in}%
\pgfsys@useobject{currentmarker}{}%
\end{pgfscope}%
\begin{pgfscope}%
\pgfsys@transformshift{1.063069in}{0.606844in}%
\pgfsys@useobject{currentmarker}{}%
\end{pgfscope}%
\begin{pgfscope}%
\pgfsys@transformshift{1.138803in}{0.617203in}%
\pgfsys@useobject{currentmarker}{}%
\end{pgfscope}%
\begin{pgfscope}%
\pgfsys@transformshift{1.217115in}{0.633115in}%
\pgfsys@useobject{currentmarker}{}%
\end{pgfscope}%
\begin{pgfscope}%
\pgfsys@transformshift{1.297142in}{0.661043in}%
\pgfsys@useobject{currentmarker}{}%
\end{pgfscope}%
\begin{pgfscope}%
\pgfsys@transformshift{1.366701in}{0.697571in}%
\pgfsys@useobject{currentmarker}{}%
\end{pgfscope}%
\end{pgfscope}%
\end{pgfpicture}%
\makeatother%
\endgroup%
}
        \caption{Iteration 1: Solve system}\label{fig:example-iter0-solution}
    \end{subfigure}
    \begin{subfigure}[b]{.32\linewidth}
        \scalebox{0.8}{%% Creator: Matplotlib, PGF backend
%%
%% To include the figure in your LaTeX document, write
%%   \input{<filename>.pgf}
%%
%% Make sure the required packages are loaded in your preamble
%%   \usepackage{pgf}
%%
%% Also ensure that all the required font packages are loaded; for instance,
%% the lmodern package is sometimes necessary when using math font.
%%   \usepackage{lmodern}
%%
%% Figures using additional raster images can only be included by \input if
%% they are in the same directory as the main LaTeX file. For loading figures
%% from other directories you can use the `import` package
%%   \usepackage{import}
%%
%% and then include the figures with
%%   \import{<path to file>}{<filename>.pgf}
%%
%% Matplotlib used the following preamble
%%   
%%   \usepackage{fontspec}
%%   \setmainfont{DejaVuSans.ttf}[Path=\detokenize{/home/fabio/Internodes-CM/.venv/lib/python3.8/site-packages/matplotlib/mpl-data/fonts/ttf/}]
%%   \setsansfont{DejaVuSans.ttf}[Path=\detokenize{/home/fabio/Internodes-CM/.venv/lib/python3.8/site-packages/matplotlib/mpl-data/fonts/ttf/}]
%%   \setmonofont{DejaVuSansMono.ttf}[Path=\detokenize{/home/fabio/Internodes-CM/.venv/lib/python3.8/site-packages/matplotlib/mpl-data/fonts/ttf/}]
%%   \makeatletter\@ifpackageloaded{underscore}{}{\usepackage[strings]{underscore}}\makeatother
%%
\begingroup%
\makeatletter%
\begin{pgfpicture}%
\pgfpathrectangle{\pgfpointorigin}{\pgfqpoint{1.982500in}{1.432000in}}%
\pgfusepath{use as bounding box, clip}%
\begin{pgfscope}%
\pgfsetbuttcap%
\pgfsetmiterjoin%
\definecolor{currentfill}{rgb}{1.000000,1.000000,1.000000}%
\pgfsetfillcolor{currentfill}%
\pgfsetlinewidth{0.000000pt}%
\definecolor{currentstroke}{rgb}{1.000000,1.000000,1.000000}%
\pgfsetstrokecolor{currentstroke}%
\pgfsetdash{}{0pt}%
\pgfpathmoveto{\pgfqpoint{0.000000in}{0.000000in}}%
\pgfpathlineto{\pgfqpoint{1.982500in}{0.000000in}}%
\pgfpathlineto{\pgfqpoint{1.982500in}{1.432000in}}%
\pgfpathlineto{\pgfqpoint{0.000000in}{1.432000in}}%
\pgfpathlineto{\pgfqpoint{0.000000in}{0.000000in}}%
\pgfpathclose%
\pgfusepath{fill}%
\end{pgfscope}%
\begin{pgfscope}%
\pgfpathrectangle{\pgfqpoint{0.100000in}{0.100000in}}{\pgfqpoint{1.782500in}{1.232000in}}%
\pgfusepath{clip}%
\pgfsetrectcap%
\pgfsetroundjoin%
\pgfsetlinewidth{0.250937pt}%
\definecolor{currentstroke}{rgb}{0.054902,0.262745,0.486275}%
\pgfsetstrokecolor{currentstroke}%
\pgfsetdash{}{0pt}%
\pgfpathmoveto{\pgfqpoint{1.879844in}{0.284679in}}%
\pgfpathlineto{\pgfqpoint{1.892500in}{0.285550in}}%
\pgfpathmoveto{\pgfqpoint{1.735866in}{0.286876in}}%
\pgfpathlineto{\pgfqpoint{1.879844in}{0.284679in}}%
\pgfpathmoveto{\pgfqpoint{1.597497in}{0.301305in}}%
\pgfpathlineto{\pgfqpoint{1.735866in}{0.286876in}}%
\pgfpathmoveto{\pgfqpoint{1.462510in}{0.320341in}}%
\pgfpathlineto{\pgfqpoint{1.597497in}{0.301305in}}%
\pgfpathmoveto{\pgfqpoint{1.343271in}{0.340622in}}%
\pgfpathlineto{\pgfqpoint{1.462510in}{0.320341in}}%
\pgfpathmoveto{\pgfqpoint{1.232559in}{0.367013in}}%
\pgfpathlineto{\pgfqpoint{1.343271in}{0.340622in}}%
\pgfpathmoveto{\pgfqpoint{1.127282in}{0.404287in}}%
\pgfpathlineto{\pgfqpoint{1.027396in}{0.445358in}}%
\pgfpathmoveto{\pgfqpoint{1.127282in}{0.404287in}}%
\pgfpathlineto{\pgfqpoint{1.232559in}{0.367013in}}%
\pgfpathmoveto{\pgfqpoint{0.597876in}{0.478255in}}%
\pgfpathlineto{\pgfqpoint{1.027396in}{0.445358in}}%
\pgfpathmoveto{\pgfqpoint{0.597876in}{0.478255in}}%
\pgfpathlineto{\pgfqpoint{0.233809in}{0.513170in}}%
\pgfpathmoveto{\pgfqpoint{0.143412in}{0.090000in}}%
\pgfpathlineto{\pgfqpoint{0.233809in}{0.513170in}}%
\pgfpathmoveto{\pgfqpoint{1.221312in}{0.101567in}}%
\pgfpathlineto{\pgfqpoint{1.243042in}{0.090000in}}%
\pgfpathmoveto{\pgfqpoint{1.221312in}{0.101567in}}%
\pgfpathlineto{\pgfqpoint{1.191293in}{0.090000in}}%
\pgfpathmoveto{\pgfqpoint{1.517412in}{0.118793in}}%
\pgfpathlineto{\pgfqpoint{1.516017in}{0.090000in}}%
\pgfpathmoveto{\pgfqpoint{1.517412in}{0.118793in}}%
\pgfpathlineto{\pgfqpoint{1.667121in}{0.090000in}}%
\pgfpathmoveto{\pgfqpoint{1.517412in}{0.118793in}}%
\pgfpathlineto{\pgfqpoint{1.221312in}{0.101567in}}%
\pgfpathmoveto{\pgfqpoint{0.914662in}{0.189106in}}%
\pgfpathlineto{\pgfqpoint{0.907825in}{0.090000in}}%
\pgfpathmoveto{\pgfqpoint{0.914662in}{0.189106in}}%
\pgfpathlineto{\pgfqpoint{1.221312in}{0.101567in}}%
\pgfpathmoveto{\pgfqpoint{0.541019in}{0.158221in}}%
\pgfpathlineto{\pgfqpoint{0.233809in}{0.513170in}}%
\pgfpathmoveto{\pgfqpoint{0.541019in}{0.158221in}}%
\pgfpathlineto{\pgfqpoint{0.597876in}{0.478255in}}%
\pgfpathmoveto{\pgfqpoint{0.541019in}{0.158221in}}%
\pgfpathlineto{\pgfqpoint{0.280592in}{0.090000in}}%
\pgfpathmoveto{\pgfqpoint{0.541019in}{0.158221in}}%
\pgfpathlineto{\pgfqpoint{0.676780in}{0.090000in}}%
\pgfpathmoveto{\pgfqpoint{0.541019in}{0.158221in}}%
\pgfpathlineto{\pgfqpoint{0.914662in}{0.189106in}}%
\pgfpathmoveto{\pgfqpoint{1.892500in}{0.267322in}}%
\pgfpathlineto{\pgfqpoint{1.879844in}{0.284679in}}%
\pgfpathmoveto{\pgfqpoint{1.892500in}{0.133599in}}%
\pgfpathlineto{\pgfqpoint{1.838685in}{0.090000in}}%
\pgfpathmoveto{\pgfqpoint{1.663450in}{0.186470in}}%
\pgfpathlineto{\pgfqpoint{1.735866in}{0.286876in}}%
\pgfpathmoveto{\pgfqpoint{1.663450in}{0.186470in}}%
\pgfpathlineto{\pgfqpoint{1.597497in}{0.301305in}}%
\pgfpathmoveto{\pgfqpoint{1.663450in}{0.186470in}}%
\pgfpathlineto{\pgfqpoint{1.774949in}{0.090000in}}%
\pgfpathmoveto{\pgfqpoint{1.663450in}{0.186470in}}%
\pgfpathlineto{\pgfqpoint{1.517412in}{0.118793in}}%
\pgfpathmoveto{\pgfqpoint{1.386379in}{0.210654in}}%
\pgfpathlineto{\pgfqpoint{1.462510in}{0.320341in}}%
\pgfpathmoveto{\pgfqpoint{1.386379in}{0.210654in}}%
\pgfpathlineto{\pgfqpoint{1.343271in}{0.340622in}}%
\pgfpathmoveto{\pgfqpoint{1.386379in}{0.210654in}}%
\pgfpathlineto{\pgfqpoint{1.221312in}{0.101567in}}%
\pgfpathmoveto{\pgfqpoint{1.386379in}{0.210654in}}%
\pgfpathlineto{\pgfqpoint{1.517412in}{0.118793in}}%
\pgfpathmoveto{\pgfqpoint{1.119697in}{0.254883in}}%
\pgfpathlineto{\pgfqpoint{1.232559in}{0.367013in}}%
\pgfpathmoveto{\pgfqpoint{1.119697in}{0.254883in}}%
\pgfpathlineto{\pgfqpoint{1.127282in}{0.404287in}}%
\pgfpathmoveto{\pgfqpoint{1.119697in}{0.254883in}}%
\pgfpathlineto{\pgfqpoint{1.221312in}{0.101567in}}%
\pgfpathmoveto{\pgfqpoint{1.119697in}{0.254883in}}%
\pgfpathlineto{\pgfqpoint{0.914662in}{0.189106in}}%
\pgfpathmoveto{\pgfqpoint{0.760214in}{0.316918in}}%
\pgfpathlineto{\pgfqpoint{1.027396in}{0.445358in}}%
\pgfpathmoveto{\pgfqpoint{0.760214in}{0.316918in}}%
\pgfpathlineto{\pgfqpoint{0.597876in}{0.478255in}}%
\pgfpathmoveto{\pgfqpoint{0.760214in}{0.316918in}}%
\pgfpathlineto{\pgfqpoint{0.914662in}{0.189106in}}%
\pgfpathmoveto{\pgfqpoint{0.760214in}{0.316918in}}%
\pgfpathlineto{\pgfqpoint{0.541019in}{0.158221in}}%
\pgfpathmoveto{\pgfqpoint{1.808144in}{0.200122in}}%
\pgfpathlineto{\pgfqpoint{1.879844in}{0.284679in}}%
\pgfpathmoveto{\pgfqpoint{1.808144in}{0.200122in}}%
\pgfpathlineto{\pgfqpoint{1.735866in}{0.286876in}}%
\pgfpathmoveto{\pgfqpoint{1.808144in}{0.200122in}}%
\pgfpathlineto{\pgfqpoint{1.806233in}{0.090000in}}%
\pgfpathmoveto{\pgfqpoint{1.808144in}{0.200122in}}%
\pgfpathlineto{\pgfqpoint{1.892500in}{0.190355in}}%
\pgfpathmoveto{\pgfqpoint{1.808144in}{0.200122in}}%
\pgfpathlineto{\pgfqpoint{1.663450in}{0.186470in}}%
\pgfpathmoveto{\pgfqpoint{1.524720in}{0.224999in}}%
\pgfpathlineto{\pgfqpoint{1.597497in}{0.301305in}}%
\pgfpathmoveto{\pgfqpoint{1.524720in}{0.224999in}}%
\pgfpathlineto{\pgfqpoint{1.462510in}{0.320341in}}%
\pgfpathmoveto{\pgfqpoint{1.524720in}{0.224999in}}%
\pgfpathlineto{\pgfqpoint{1.517412in}{0.118793in}}%
\pgfpathmoveto{\pgfqpoint{1.524720in}{0.224999in}}%
\pgfpathlineto{\pgfqpoint{1.663450in}{0.186470in}}%
\pgfpathmoveto{\pgfqpoint{1.524720in}{0.224999in}}%
\pgfpathlineto{\pgfqpoint{1.386379in}{0.210654in}}%
\pgfpathmoveto{\pgfqpoint{1.221611in}{0.090000in}}%
\pgfpathlineto{\pgfqpoint{1.221312in}{0.101567in}}%
\pgfpathmoveto{\pgfqpoint{1.259070in}{0.252435in}}%
\pgfpathlineto{\pgfqpoint{1.343271in}{0.340622in}}%
\pgfpathmoveto{\pgfqpoint{1.259070in}{0.252435in}}%
\pgfpathlineto{\pgfqpoint{1.232559in}{0.367013in}}%
\pgfpathmoveto{\pgfqpoint{1.259070in}{0.252435in}}%
\pgfpathlineto{\pgfqpoint{1.221312in}{0.101567in}}%
\pgfpathmoveto{\pgfqpoint{1.259070in}{0.252435in}}%
\pgfpathlineto{\pgfqpoint{1.386379in}{0.210654in}}%
\pgfpathmoveto{\pgfqpoint{1.259070in}{0.252435in}}%
\pgfpathlineto{\pgfqpoint{1.119697in}{0.254883in}}%
\pgfpathmoveto{\pgfqpoint{1.004838in}{0.316222in}}%
\pgfpathlineto{\pgfqpoint{1.027396in}{0.445358in}}%
\pgfpathmoveto{\pgfqpoint{1.004838in}{0.316222in}}%
\pgfpathlineto{\pgfqpoint{1.127282in}{0.404287in}}%
\pgfpathmoveto{\pgfqpoint{1.004838in}{0.316222in}}%
\pgfpathlineto{\pgfqpoint{0.914662in}{0.189106in}}%
\pgfpathmoveto{\pgfqpoint{1.004838in}{0.316222in}}%
\pgfpathlineto{\pgfqpoint{1.119697in}{0.254883in}}%
\pgfpathmoveto{\pgfqpoint{1.004838in}{0.316222in}}%
\pgfpathlineto{\pgfqpoint{0.760214in}{0.316918in}}%
\pgfpathmoveto{\pgfqpoint{0.540673in}{0.090000in}}%
\pgfpathlineto{\pgfqpoint{0.541019in}{0.158221in}}%
\pgfpathlineto{\pgfqpoint{0.541019in}{0.158221in}}%
\pgfusepath{stroke}%
\end{pgfscope}%
\begin{pgfscope}%
\pgfpathrectangle{\pgfqpoint{0.100000in}{0.100000in}}{\pgfqpoint{1.782500in}{1.232000in}}%
\pgfusepath{clip}%
\pgfsetrectcap%
\pgfsetroundjoin%
\pgfsetlinewidth{0.250937pt}%
\definecolor{currentstroke}{rgb}{0.835294,0.321569,0.035294}%
\pgfsetstrokecolor{currentstroke}%
\pgfsetdash{}{0pt}%
\pgfpathmoveto{\pgfqpoint{0.793086in}{0.622292in}}%
\pgfpathlineto{\pgfqpoint{0.694167in}{0.980000in}}%
\pgfpathmoveto{\pgfqpoint{1.288333in}{0.980000in}}%
\pgfpathlineto{\pgfqpoint{1.882500in}{0.980000in}}%
\pgfpathmoveto{\pgfqpoint{1.288333in}{0.980000in}}%
\pgfpathlineto{\pgfqpoint{0.694167in}{0.980000in}}%
\pgfpathmoveto{\pgfqpoint{0.904550in}{0.512460in}}%
\pgfpathlineto{\pgfqpoint{0.793086in}{0.622292in}}%
\pgfpathmoveto{\pgfqpoint{1.052243in}{0.435141in}}%
\pgfpathlineto{\pgfqpoint{0.904550in}{0.512460in}}%
\pgfpathmoveto{\pgfqpoint{1.208193in}{0.375328in}}%
\pgfpathlineto{\pgfqpoint{1.052243in}{0.435141in}}%
\pgfpathmoveto{\pgfqpoint{1.382051in}{0.333317in}}%
\pgfpathlineto{\pgfqpoint{1.208193in}{0.375328in}}%
\pgfpathmoveto{\pgfqpoint{1.555503in}{0.306975in}}%
\pgfpathlineto{\pgfqpoint{1.382051in}{0.333317in}}%
\pgfpathmoveto{\pgfqpoint{1.717879in}{0.287801in}}%
\pgfpathlineto{\pgfqpoint{1.555503in}{0.306975in}}%
\pgfpathmoveto{\pgfqpoint{1.879926in}{0.284330in}}%
\pgfpathlineto{\pgfqpoint{1.717879in}{0.287801in}}%
\pgfpathmoveto{\pgfqpoint{1.892500in}{0.285248in}}%
\pgfpathlineto{\pgfqpoint{1.879926in}{0.284330in}}%
\pgfpathmoveto{\pgfqpoint{1.892500in}{0.980000in}}%
\pgfpathlineto{\pgfqpoint{1.882500in}{0.980000in}}%
\pgfpathmoveto{\pgfqpoint{1.535159in}{0.701061in}}%
\pgfpathlineto{\pgfqpoint{1.882500in}{0.980000in}}%
\pgfpathmoveto{\pgfqpoint{1.535159in}{0.701061in}}%
\pgfpathlineto{\pgfqpoint{1.288333in}{0.980000in}}%
\pgfpathmoveto{\pgfqpoint{1.817543in}{0.511549in}}%
\pgfpathlineto{\pgfqpoint{1.892500in}{0.533786in}}%
\pgfpathmoveto{\pgfqpoint{1.817543in}{0.511549in}}%
\pgfpathlineto{\pgfqpoint{1.535159in}{0.701061in}}%
\pgfpathmoveto{\pgfqpoint{1.171459in}{0.709448in}}%
\pgfpathlineto{\pgfqpoint{1.288333in}{0.980000in}}%
\pgfpathmoveto{\pgfqpoint{1.171459in}{0.709448in}}%
\pgfpathlineto{\pgfqpoint{1.535159in}{0.701061in}}%
\pgfpathmoveto{\pgfqpoint{1.892500in}{0.497052in}}%
\pgfpathlineto{\pgfqpoint{1.817543in}{0.511549in}}%
\pgfpathmoveto{\pgfqpoint{1.519659in}{0.473898in}}%
\pgfpathlineto{\pgfqpoint{1.382051in}{0.333317in}}%
\pgfpathmoveto{\pgfqpoint{1.519659in}{0.473898in}}%
\pgfpathlineto{\pgfqpoint{1.555503in}{0.306975in}}%
\pgfpathmoveto{\pgfqpoint{1.519659in}{0.473898in}}%
\pgfpathlineto{\pgfqpoint{1.535159in}{0.701061in}}%
\pgfpathmoveto{\pgfqpoint{1.519659in}{0.473898in}}%
\pgfpathlineto{\pgfqpoint{1.817543in}{0.511549in}}%
\pgfpathmoveto{\pgfqpoint{1.276919in}{0.549994in}}%
\pgfpathlineto{\pgfqpoint{1.052243in}{0.435141in}}%
\pgfpathmoveto{\pgfqpoint{1.276919in}{0.549994in}}%
\pgfpathlineto{\pgfqpoint{1.208193in}{0.375328in}}%
\pgfpathmoveto{\pgfqpoint{1.276919in}{0.549994in}}%
\pgfpathlineto{\pgfqpoint{1.535159in}{0.701061in}}%
\pgfpathmoveto{\pgfqpoint{1.276919in}{0.549994in}}%
\pgfpathlineto{\pgfqpoint{1.171459in}{0.709448in}}%
\pgfpathmoveto{\pgfqpoint{1.276919in}{0.549994in}}%
\pgfpathlineto{\pgfqpoint{1.519659in}{0.473898in}}%
\pgfpathmoveto{\pgfqpoint{1.804140in}{0.369059in}}%
\pgfpathlineto{\pgfqpoint{1.717879in}{0.287801in}}%
\pgfpathmoveto{\pgfqpoint{1.804140in}{0.369059in}}%
\pgfpathlineto{\pgfqpoint{1.879926in}{0.284330in}}%
\pgfpathmoveto{\pgfqpoint{1.804140in}{0.369059in}}%
\pgfpathlineto{\pgfqpoint{1.817543in}{0.511549in}}%
\pgfpathmoveto{\pgfqpoint{1.353344in}{0.443506in}}%
\pgfpathlineto{\pgfqpoint{1.208193in}{0.375328in}}%
\pgfpathmoveto{\pgfqpoint{1.353344in}{0.443506in}}%
\pgfpathlineto{\pgfqpoint{1.382051in}{0.333317in}}%
\pgfpathmoveto{\pgfqpoint{1.353344in}{0.443506in}}%
\pgfpathlineto{\pgfqpoint{1.519659in}{0.473898in}}%
\pgfpathmoveto{\pgfqpoint{1.353344in}{0.443506in}}%
\pgfpathlineto{\pgfqpoint{1.276919in}{0.549994in}}%
\pgfpathmoveto{\pgfqpoint{1.090498in}{0.558197in}}%
\pgfpathlineto{\pgfqpoint{0.904550in}{0.512460in}}%
\pgfpathmoveto{\pgfqpoint{1.090498in}{0.558197in}}%
\pgfpathlineto{\pgfqpoint{1.052243in}{0.435141in}}%
\pgfpathmoveto{\pgfqpoint{1.090498in}{0.558197in}}%
\pgfpathlineto{\pgfqpoint{1.171459in}{0.709448in}}%
\pgfpathmoveto{\pgfqpoint{1.090498in}{0.558197in}}%
\pgfpathlineto{\pgfqpoint{1.276919in}{0.549994in}}%
\pgfpathmoveto{\pgfqpoint{0.946418in}{0.807480in}}%
\pgfpathlineto{\pgfqpoint{0.694167in}{0.980000in}}%
\pgfpathmoveto{\pgfqpoint{0.946418in}{0.807480in}}%
\pgfpathlineto{\pgfqpoint{0.793086in}{0.622292in}}%
\pgfpathmoveto{\pgfqpoint{0.946418in}{0.807480in}}%
\pgfpathlineto{\pgfqpoint{1.288333in}{0.980000in}}%
\pgfpathmoveto{\pgfqpoint{0.946418in}{0.807480in}}%
\pgfpathlineto{\pgfqpoint{1.171459in}{0.709448in}}%
\pgfpathmoveto{\pgfqpoint{1.892500in}{0.299379in}}%
\pgfpathlineto{\pgfqpoint{1.879926in}{0.284330in}}%
\pgfpathmoveto{\pgfqpoint{1.892500in}{0.436890in}}%
\pgfpathlineto{\pgfqpoint{1.817543in}{0.511549in}}%
\pgfpathmoveto{\pgfqpoint{1.892500in}{0.372201in}}%
\pgfpathlineto{\pgfqpoint{1.804140in}{0.369059in}}%
\pgfpathmoveto{\pgfqpoint{1.653660in}{0.382445in}}%
\pgfpathlineto{\pgfqpoint{1.555503in}{0.306975in}}%
\pgfpathmoveto{\pgfqpoint{1.653660in}{0.382445in}}%
\pgfpathlineto{\pgfqpoint{1.717879in}{0.287801in}}%
\pgfpathmoveto{\pgfqpoint{1.653660in}{0.382445in}}%
\pgfpathlineto{\pgfqpoint{1.817543in}{0.511549in}}%
\pgfpathmoveto{\pgfqpoint{1.653660in}{0.382445in}}%
\pgfpathlineto{\pgfqpoint{1.519659in}{0.473898in}}%
\pgfpathmoveto{\pgfqpoint{1.653660in}{0.382445in}}%
\pgfpathlineto{\pgfqpoint{1.804140in}{0.369059in}}%
\pgfpathmoveto{\pgfqpoint{1.892500in}{0.932550in}}%
\pgfpathlineto{\pgfqpoint{1.882500in}{0.980000in}}%
\pgfpathmoveto{\pgfqpoint{1.892500in}{0.723655in}}%
\pgfpathlineto{\pgfqpoint{1.535159in}{0.701061in}}%
\pgfpathmoveto{\pgfqpoint{1.892500in}{0.647567in}}%
\pgfpathlineto{\pgfqpoint{1.817543in}{0.511549in}}%
\pgfpathmoveto{\pgfqpoint{1.892500in}{0.975371in}}%
\pgfpathlineto{\pgfqpoint{1.882500in}{0.980000in}}%
\pgfpathmoveto{\pgfqpoint{0.976326in}{0.641891in}}%
\pgfpathlineto{\pgfqpoint{0.793086in}{0.622292in}}%
\pgfpathmoveto{\pgfqpoint{0.976326in}{0.641891in}}%
\pgfpathlineto{\pgfqpoint{0.904550in}{0.512460in}}%
\pgfpathmoveto{\pgfqpoint{0.976326in}{0.641891in}}%
\pgfpathlineto{\pgfqpoint{1.171459in}{0.709448in}}%
\pgfpathmoveto{\pgfqpoint{0.976326in}{0.641891in}}%
\pgfpathlineto{\pgfqpoint{1.090498in}{0.558197in}}%
\pgfpathmoveto{\pgfqpoint{0.976326in}{0.641891in}}%
\pgfpathlineto{\pgfqpoint{0.946418in}{0.807480in}}%
\pgfpathlineto{\pgfqpoint{0.946418in}{0.807480in}}%
\pgfusepath{stroke}%
\end{pgfscope}%
\begin{pgfscope}%
\pgfpathrectangle{\pgfqpoint{0.100000in}{0.100000in}}{\pgfqpoint{1.782500in}{1.232000in}}%
\pgfusepath{clip}%
\pgfsetbuttcap%
\pgfsetroundjoin%
\definecolor{currentfill}{rgb}{0.054902,0.262745,0.486275}%
\pgfsetfillcolor{currentfill}%
\pgfsetlinewidth{1.003750pt}%
\definecolor{currentstroke}{rgb}{0.054902,0.262745,0.486275}%
\pgfsetstrokecolor{currentstroke}%
\pgfsetdash{}{0pt}%
\pgfsys@defobject{currentmarker}{\pgfqpoint{-0.018373in}{-0.018373in}}{\pgfqpoint{0.018373in}{0.018373in}}{%
\pgfpathmoveto{\pgfqpoint{0.000000in}{-0.018373in}}%
\pgfpathcurveto{\pgfqpoint{0.004873in}{-0.018373in}}{\pgfqpoint{0.009546in}{-0.016437in}}{\pgfqpoint{0.012992in}{-0.012992in}}%
\pgfpathcurveto{\pgfqpoint{0.016437in}{-0.009546in}}{\pgfqpoint{0.018373in}{-0.004873in}}{\pgfqpoint{0.018373in}{0.000000in}}%
\pgfpathcurveto{\pgfqpoint{0.018373in}{0.004873in}}{\pgfqpoint{0.016437in}{0.009546in}}{\pgfqpoint{0.012992in}{0.012992in}}%
\pgfpathcurveto{\pgfqpoint{0.009546in}{0.016437in}}{\pgfqpoint{0.004873in}{0.018373in}}{\pgfqpoint{0.000000in}{0.018373in}}%
\pgfpathcurveto{\pgfqpoint{-0.004873in}{0.018373in}}{\pgfqpoint{-0.009546in}{0.016437in}}{\pgfqpoint{-0.012992in}{0.012992in}}%
\pgfpathcurveto{\pgfqpoint{-0.016437in}{0.009546in}}{\pgfqpoint{-0.018373in}{0.004873in}}{\pgfqpoint{-0.018373in}{0.000000in}}%
\pgfpathcurveto{\pgfqpoint{-0.018373in}{-0.004873in}}{\pgfqpoint{-0.016437in}{-0.009546in}}{\pgfqpoint{-0.012992in}{-0.012992in}}%
\pgfpathcurveto{\pgfqpoint{-0.009546in}{-0.016437in}}{\pgfqpoint{-0.004873in}{-0.018373in}}{\pgfqpoint{0.000000in}{-0.018373in}}%
\pgfpathlineto{\pgfqpoint{0.000000in}{-0.018373in}}%
\pgfpathclose%
\pgfusepath{stroke,fill}%
}%
\begin{pgfscope}%
\pgfsys@transformshift{2.635466in}{0.399118in}%
\pgfsys@useobject{currentmarker}{}%
\end{pgfscope}%
\begin{pgfscope}%
\pgfsys@transformshift{2.531249in}{0.366056in}%
\pgfsys@useobject{currentmarker}{}%
\end{pgfscope}%
\begin{pgfscope}%
\pgfsys@transformshift{2.418530in}{0.340604in}%
\pgfsys@useobject{currentmarker}{}%
\end{pgfscope}%
\begin{pgfscope}%
\pgfsys@transformshift{2.299598in}{0.321885in}%
\pgfsys@useobject{currentmarker}{}%
\end{pgfscope}%
\begin{pgfscope}%
\pgfsys@transformshift{2.164176in}{0.306730in}%
\pgfsys@useobject{currentmarker}{}%
\end{pgfscope}%
\begin{pgfscope}%
\pgfsys@transformshift{2.022625in}{0.294504in}%
\pgfsys@useobject{currentmarker}{}%
\end{pgfscope}%
\begin{pgfscope}%
\pgfsys@transformshift{1.879844in}{0.284679in}%
\pgfsys@useobject{currentmarker}{}%
\end{pgfscope}%
\begin{pgfscope}%
\pgfsys@transformshift{1.735866in}{0.286876in}%
\pgfsys@useobject{currentmarker}{}%
\end{pgfscope}%
\begin{pgfscope}%
\pgfsys@transformshift{1.597497in}{0.301305in}%
\pgfsys@useobject{currentmarker}{}%
\end{pgfscope}%
\begin{pgfscope}%
\pgfsys@transformshift{1.462510in}{0.320341in}%
\pgfsys@useobject{currentmarker}{}%
\end{pgfscope}%
\begin{pgfscope}%
\pgfsys@transformshift{1.343271in}{0.340622in}%
\pgfsys@useobject{currentmarker}{}%
\end{pgfscope}%
\begin{pgfscope}%
\pgfsys@transformshift{1.232559in}{0.367013in}%
\pgfsys@useobject{currentmarker}{}%
\end{pgfscope}%
\end{pgfscope}%
\begin{pgfscope}%
\pgfpathrectangle{\pgfqpoint{0.100000in}{0.100000in}}{\pgfqpoint{1.782500in}{1.232000in}}%
\pgfusepath{clip}%
\pgfsetbuttcap%
\pgfsetroundjoin%
\definecolor{currentfill}{rgb}{0.835294,0.321569,0.035294}%
\pgfsetfillcolor{currentfill}%
\pgfsetlinewidth{1.003750pt}%
\definecolor{currentstroke}{rgb}{0.835294,0.321569,0.035294}%
\pgfsetstrokecolor{currentstroke}%
\pgfsetdash{}{0pt}%
\pgfsys@defobject{currentmarker}{\pgfqpoint{-0.018373in}{-0.018373in}}{\pgfqpoint{0.018373in}{0.018373in}}{%
\pgfpathmoveto{\pgfqpoint{0.000000in}{-0.018373in}}%
\pgfpathcurveto{\pgfqpoint{0.004873in}{-0.018373in}}{\pgfqpoint{0.009546in}{-0.016437in}}{\pgfqpoint{0.012992in}{-0.012992in}}%
\pgfpathcurveto{\pgfqpoint{0.016437in}{-0.009546in}}{\pgfqpoint{0.018373in}{-0.004873in}}{\pgfqpoint{0.018373in}{0.000000in}}%
\pgfpathcurveto{\pgfqpoint{0.018373in}{0.004873in}}{\pgfqpoint{0.016437in}{0.009546in}}{\pgfqpoint{0.012992in}{0.012992in}}%
\pgfpathcurveto{\pgfqpoint{0.009546in}{0.016437in}}{\pgfqpoint{0.004873in}{0.018373in}}{\pgfqpoint{0.000000in}{0.018373in}}%
\pgfpathcurveto{\pgfqpoint{-0.004873in}{0.018373in}}{\pgfqpoint{-0.009546in}{0.016437in}}{\pgfqpoint{-0.012992in}{0.012992in}}%
\pgfpathcurveto{\pgfqpoint{-0.016437in}{0.009546in}}{\pgfqpoint{-0.018373in}{0.004873in}}{\pgfqpoint{-0.018373in}{0.000000in}}%
\pgfpathcurveto{\pgfqpoint{-0.018373in}{-0.004873in}}{\pgfqpoint{-0.016437in}{-0.009546in}}{\pgfqpoint{-0.012992in}{-0.012992in}}%
\pgfpathcurveto{\pgfqpoint{-0.009546in}{-0.016437in}}{\pgfqpoint{-0.004873in}{-0.018373in}}{\pgfqpoint{0.000000in}{-0.018373in}}%
\pgfpathlineto{\pgfqpoint{0.000000in}{-0.018373in}}%
\pgfpathclose%
\pgfusepath{stroke,fill}%
}%
\begin{pgfscope}%
\pgfsys@transformshift{1.052243in}{0.435141in}%
\pgfsys@useobject{currentmarker}{}%
\end{pgfscope}%
\begin{pgfscope}%
\pgfsys@transformshift{1.208193in}{0.375328in}%
\pgfsys@useobject{currentmarker}{}%
\end{pgfscope}%
\begin{pgfscope}%
\pgfsys@transformshift{1.382051in}{0.333317in}%
\pgfsys@useobject{currentmarker}{}%
\end{pgfscope}%
\begin{pgfscope}%
\pgfsys@transformshift{1.555503in}{0.306975in}%
\pgfsys@useobject{currentmarker}{}%
\end{pgfscope}%
\begin{pgfscope}%
\pgfsys@transformshift{1.717879in}{0.287801in}%
\pgfsys@useobject{currentmarker}{}%
\end{pgfscope}%
\begin{pgfscope}%
\pgfsys@transformshift{1.879926in}{0.284330in}%
\pgfsys@useobject{currentmarker}{}%
\end{pgfscope}%
\begin{pgfscope}%
\pgfsys@transformshift{2.040503in}{0.296062in}%
\pgfsys@useobject{currentmarker}{}%
\end{pgfscope}%
\begin{pgfscope}%
\pgfsys@transformshift{2.207117in}{0.310861in}%
\pgfsys@useobject{currentmarker}{}%
\end{pgfscope}%
\begin{pgfscope}%
\pgfsys@transformshift{2.379404in}{0.333593in}%
\pgfsys@useobject{currentmarker}{}%
\end{pgfscope}%
\begin{pgfscope}%
\pgfsys@transformshift{2.555461in}{0.373491in}%
\pgfsys@useobject{currentmarker}{}%
\end{pgfscope}%
\begin{pgfscope}%
\pgfsys@transformshift{2.708493in}{0.425674in}%
\pgfsys@useobject{currentmarker}{}%
\end{pgfscope}%
\end{pgfscope}%
\begin{pgfscope}%
\pgfpathrectangle{\pgfqpoint{0.100000in}{0.100000in}}{\pgfqpoint{1.782500in}{1.232000in}}%
\pgfusepath{clip}%
\pgfsetbuttcap%
\pgfsetroundjoin%
\pgfsetlinewidth{1.003750pt}%
\definecolor{currentstroke}{rgb}{0.054902,0.262745,0.486275}%
\pgfsetstrokecolor{currentstroke}%
\pgfsetdash{}{0pt}%
\pgfpathmoveto{\pgfqpoint{0.000000in}{-0.018373in}}%
\pgfpathcurveto{\pgfqpoint{0.004873in}{-0.018373in}}{\pgfqpoint{0.009546in}{-0.016437in}}{\pgfqpoint{0.012992in}{-0.012992in}}%
\pgfpathcurveto{\pgfqpoint{0.016437in}{-0.009546in}}{\pgfqpoint{0.018373in}{-0.004873in}}{\pgfqpoint{0.018373in}{0.000000in}}%
\pgfpathcurveto{\pgfqpoint{0.018373in}{0.004873in}}{\pgfqpoint{0.016437in}{0.009546in}}{\pgfqpoint{0.012992in}{0.012992in}}%
\pgfpathcurveto{\pgfqpoint{0.009546in}{0.016437in}}{\pgfqpoint{0.004873in}{0.018373in}}{\pgfqpoint{0.000000in}{0.018373in}}%
\pgfpathcurveto{\pgfqpoint{-0.004873in}{0.018373in}}{\pgfqpoint{-0.009546in}{0.016437in}}{\pgfqpoint{-0.012992in}{0.012992in}}%
\pgfpathcurveto{\pgfqpoint{-0.016437in}{0.009546in}}{\pgfqpoint{-0.018373in}{0.004873in}}{\pgfqpoint{-0.018373in}{0.000000in}}%
\pgfpathcurveto{\pgfqpoint{-0.018373in}{-0.004873in}}{\pgfqpoint{-0.016437in}{-0.009546in}}{\pgfqpoint{-0.012992in}{-0.012992in}}%
\pgfpathcurveto{\pgfqpoint{-0.009546in}{-0.016437in}}{\pgfqpoint{-0.004873in}{-0.018373in}}{\pgfqpoint{0.000000in}{-0.018373in}}%
\pgfusepath{stroke}%
\end{pgfscope}%
\begin{pgfscope}%
\pgfpathrectangle{\pgfqpoint{0.100000in}{0.100000in}}{\pgfqpoint{1.782500in}{1.232000in}}%
\pgfusepath{clip}%
\pgfsetbuttcap%
\pgfsetroundjoin%
\pgfsetlinewidth{1.003750pt}%
\definecolor{currentstroke}{rgb}{0.835294,0.321569,0.035294}%
\pgfsetstrokecolor{currentstroke}%
\pgfsetdash{}{0pt}%
\pgfpathmoveto{\pgfqpoint{0.000000in}{-0.018373in}}%
\pgfpathcurveto{\pgfqpoint{0.004873in}{-0.018373in}}{\pgfqpoint{0.009546in}{-0.016437in}}{\pgfqpoint{0.012992in}{-0.012992in}}%
\pgfpathcurveto{\pgfqpoint{0.016437in}{-0.009546in}}{\pgfqpoint{0.018373in}{-0.004873in}}{\pgfqpoint{0.018373in}{0.000000in}}%
\pgfpathcurveto{\pgfqpoint{0.018373in}{0.004873in}}{\pgfqpoint{0.016437in}{0.009546in}}{\pgfqpoint{0.012992in}{0.012992in}}%
\pgfpathcurveto{\pgfqpoint{0.009546in}{0.016437in}}{\pgfqpoint{0.004873in}{0.018373in}}{\pgfqpoint{0.000000in}{0.018373in}}%
\pgfpathcurveto{\pgfqpoint{-0.004873in}{0.018373in}}{\pgfqpoint{-0.009546in}{0.016437in}}{\pgfqpoint{-0.012992in}{0.012992in}}%
\pgfpathcurveto{\pgfqpoint{-0.016437in}{0.009546in}}{\pgfqpoint{-0.018373in}{0.004873in}}{\pgfqpoint{-0.018373in}{0.000000in}}%
\pgfpathcurveto{\pgfqpoint{-0.018373in}{-0.004873in}}{\pgfqpoint{-0.016437in}{-0.009546in}}{\pgfqpoint{-0.012992in}{-0.012992in}}%
\pgfpathcurveto{\pgfqpoint{-0.009546in}{-0.016437in}}{\pgfqpoint{-0.004873in}{-0.018373in}}{\pgfqpoint{0.000000in}{-0.018373in}}%
\pgfusepath{stroke}%
\end{pgfscope}%
\begin{pgfscope}%
\pgfpathrectangle{\pgfqpoint{0.100000in}{0.100000in}}{\pgfqpoint{1.782500in}{1.232000in}}%
\pgfusepath{clip}%
\pgfsetbuttcap%
\pgfsetroundjoin%
\definecolor{currentfill}{rgb}{0.054902,0.262745,0.486275}%
\pgfsetfillcolor{currentfill}%
\pgfsetlinewidth{1.505625pt}%
\definecolor{currentstroke}{rgb}{0.054902,0.262745,0.486275}%
\pgfsetstrokecolor{currentstroke}%
\pgfsetdash{}{0pt}%
\pgfsys@defobject{currentmarker}{\pgfqpoint{-0.018373in}{-0.018373in}}{\pgfqpoint{0.018373in}{0.018373in}}{%
\pgfpathmoveto{\pgfqpoint{-0.018373in}{-0.018373in}}%
\pgfpathlineto{\pgfqpoint{0.018373in}{0.018373in}}%
\pgfpathmoveto{\pgfqpoint{-0.018373in}{0.018373in}}%
\pgfpathlineto{\pgfqpoint{0.018373in}{-0.018373in}}%
\pgfusepath{stroke,fill}%
}%
\begin{pgfscope}%
\pgfsys@transformshift{2.732674in}{0.434467in}%
\pgfsys@useobject{currentmarker}{}%
\end{pgfscope}%
\begin{pgfscope}%
\pgfsys@transformshift{1.027396in}{0.445358in}%
\pgfsys@useobject{currentmarker}{}%
\end{pgfscope}%
\begin{pgfscope}%
\pgfsys@transformshift{1.127282in}{0.404287in}%
\pgfsys@useobject{currentmarker}{}%
\end{pgfscope}%
\end{pgfscope}%
\begin{pgfscope}%
\pgfpathrectangle{\pgfqpoint{0.100000in}{0.100000in}}{\pgfqpoint{1.782500in}{1.232000in}}%
\pgfusepath{clip}%
\pgfsetbuttcap%
\pgfsetroundjoin%
\definecolor{currentfill}{rgb}{0.835294,0.321569,0.035294}%
\pgfsetfillcolor{currentfill}%
\pgfsetlinewidth{1.505625pt}%
\definecolor{currentstroke}{rgb}{0.835294,0.321569,0.035294}%
\pgfsetstrokecolor{currentstroke}%
\pgfsetdash{}{0pt}%
\pgfsys@defobject{currentmarker}{\pgfqpoint{-0.018373in}{-0.018373in}}{\pgfqpoint{0.018373in}{0.018373in}}{%
\pgfpathmoveto{\pgfqpoint{-0.018373in}{-0.018373in}}%
\pgfpathlineto{\pgfqpoint{0.018373in}{0.018373in}}%
\pgfpathmoveto{\pgfqpoint{-0.018373in}{0.018373in}}%
\pgfpathlineto{\pgfqpoint{0.018373in}{-0.018373in}}%
\pgfusepath{stroke,fill}%
}%
\end{pgfscope}%
\end{pgfpicture}%
\makeatother%
\endgroup%
}
        \caption{Iteration 1: Update interface}\label{fig:example-iter0-dumping}
    \end{subfigure}
    \begin{subfigure}[b]{.32\linewidth}
        \scalebox{0.8}{\input{plots/example_iter1_interface.pgf}}
        \caption{Iteration 2: Find interface}\label{fig:example-iter1-interface}
    \end{subfigure}
    \begin{subfigure}[b]{.32\linewidth}
        \scalebox{0.8}{%% Creator: Matplotlib, PGF backend
%%
%% To include the figure in your LaTeX document, write
%%   \input{<filename>.pgf}
%%
%% Make sure the required packages are loaded in your preamble
%%   \usepackage{pgf}
%%
%% Also ensure that all the required font packages are loaded; for instance,
%% the lmodern package is sometimes necessary when using math font.
%%   \usepackage{lmodern}
%%
%% Figures using additional raster images can only be included by \input if
%% they are in the same directory as the main LaTeX file. For loading figures
%% from other directories you can use the `import` package
%%   \usepackage{import}
%%
%% and then include the figures with
%%   \import{<path to file>}{<filename>.pgf}
%%
%% Matplotlib used the following preamble
%%   
%%   \usepackage{fontspec}
%%   \setmainfont{DejaVuSans.ttf}[Path=\detokenize{/home/fabio/Internodes-CM/.venv/lib/python3.8/site-packages/matplotlib/mpl-data/fonts/ttf/}]
%%   \setsansfont{DejaVuSans.ttf}[Path=\detokenize{/home/fabio/Internodes-CM/.venv/lib/python3.8/site-packages/matplotlib/mpl-data/fonts/ttf/}]
%%   \setmonofont{DejaVuSansMono.ttf}[Path=\detokenize{/home/fabio/Internodes-CM/.venv/lib/python3.8/site-packages/matplotlib/mpl-data/fonts/ttf/}]
%%   \makeatletter\@ifpackageloaded{underscore}{}{\usepackage[strings]{underscore}}\makeatother
%%
\begingroup%
\makeatletter%
\begin{pgfpicture}%
\pgfpathrectangle{\pgfpointorigin}{\pgfqpoint{1.982500in}{1.432000in}}%
\pgfusepath{use as bounding box, clip}%
\begin{pgfscope}%
\pgfsetbuttcap%
\pgfsetmiterjoin%
\definecolor{currentfill}{rgb}{1.000000,1.000000,1.000000}%
\pgfsetfillcolor{currentfill}%
\pgfsetlinewidth{0.000000pt}%
\definecolor{currentstroke}{rgb}{1.000000,1.000000,1.000000}%
\pgfsetstrokecolor{currentstroke}%
\pgfsetdash{}{0pt}%
\pgfpathmoveto{\pgfqpoint{0.000000in}{0.000000in}}%
\pgfpathlineto{\pgfqpoint{1.982500in}{0.000000in}}%
\pgfpathlineto{\pgfqpoint{1.982500in}{1.432000in}}%
\pgfpathlineto{\pgfqpoint{0.000000in}{1.432000in}}%
\pgfpathlineto{\pgfqpoint{0.000000in}{0.000000in}}%
\pgfpathclose%
\pgfusepath{fill}%
\end{pgfscope}%
\begin{pgfscope}%
\pgfpathrectangle{\pgfqpoint{0.100000in}{0.100000in}}{\pgfqpoint{1.782500in}{1.232000in}}%
\pgfusepath{clip}%
\pgfsetrectcap%
\pgfsetroundjoin%
\pgfsetlinewidth{0.250937pt}%
\definecolor{currentstroke}{rgb}{0.054902,0.262745,0.486275}%
\pgfsetstrokecolor{currentstroke}%
\pgfsetdash{}{0pt}%
\pgfpathmoveto{\pgfqpoint{0.458635in}{0.100000in}}%
\pgfpathlineto{\pgfqpoint{0.190067in}{0.100000in}}%
\pgfpathmoveto{\pgfqpoint{0.727204in}{0.100000in}}%
\pgfpathlineto{\pgfqpoint{0.458635in}{0.100000in}}%
\pgfpathmoveto{\pgfqpoint{0.995772in}{0.100000in}}%
\pgfpathlineto{\pgfqpoint{0.727204in}{0.100000in}}%
\pgfpathmoveto{\pgfqpoint{1.264340in}{0.100000in}}%
\pgfpathlineto{\pgfqpoint{0.995772in}{0.100000in}}%
\pgfpathmoveto{\pgfqpoint{1.532909in}{0.100000in}}%
\pgfpathlineto{\pgfqpoint{1.801477in}{0.100000in}}%
\pgfpathmoveto{\pgfqpoint{1.532909in}{0.100000in}}%
\pgfpathlineto{\pgfqpoint{1.264340in}{0.100000in}}%
\pgfpathmoveto{\pgfqpoint{1.796841in}{0.415884in}}%
\pgfpathlineto{\pgfqpoint{1.801477in}{0.100000in}}%
\pgfpathmoveto{\pgfqpoint{1.796841in}{0.415884in}}%
\pgfpathlineto{\pgfqpoint{1.774230in}{0.729694in}}%
\pgfpathmoveto{\pgfqpoint{1.611368in}{0.716211in}}%
\pgfpathlineto{\pgfqpoint{1.774230in}{0.729694in}}%
\pgfpathmoveto{\pgfqpoint{1.611368in}{0.716211in}}%
\pgfpathlineto{\pgfqpoint{1.438592in}{0.697903in}}%
\pgfpathmoveto{\pgfqpoint{1.358413in}{0.681892in}}%
\pgfpathlineto{\pgfqpoint{1.438592in}{0.697903in}}%
\pgfpathmoveto{\pgfqpoint{1.300610in}{0.658103in}}%
\pgfpathlineto{\pgfqpoint{1.358413in}{0.681892in}}%
\pgfpathmoveto{\pgfqpoint{1.245113in}{0.640065in}}%
\pgfpathlineto{\pgfqpoint{1.300610in}{0.658103in}}%
\pgfpathmoveto{\pgfqpoint{1.187563in}{0.626883in}}%
\pgfpathlineto{\pgfqpoint{1.245113in}{0.640065in}}%
\pgfpathmoveto{\pgfqpoint{1.121983in}{0.616327in}}%
\pgfpathlineto{\pgfqpoint{1.187563in}{0.626883in}}%
\pgfpathmoveto{\pgfqpoint{1.054136in}{0.608017in}}%
\pgfpathlineto{\pgfqpoint{1.121983in}{0.616327in}}%
\pgfpathmoveto{\pgfqpoint{0.985700in}{0.601411in}}%
\pgfpathlineto{\pgfqpoint{1.054136in}{0.608017in}}%
\pgfpathmoveto{\pgfqpoint{0.916417in}{0.603252in}}%
\pgfpathlineto{\pgfqpoint{0.985700in}{0.601411in}}%
\pgfpathmoveto{\pgfqpoint{0.848834in}{0.613810in}}%
\pgfpathlineto{\pgfqpoint{0.916417in}{0.603252in}}%
\pgfpathmoveto{\pgfqpoint{0.783038in}{0.628109in}}%
\pgfpathlineto{\pgfqpoint{0.848834in}{0.613810in}}%
\pgfpathmoveto{\pgfqpoint{0.721820in}{0.643258in}}%
\pgfpathlineto{\pgfqpoint{0.783038in}{0.628109in}}%
\pgfpathmoveto{\pgfqpoint{0.661196in}{0.663049in}}%
\pgfpathlineto{\pgfqpoint{0.721820in}{0.643258in}}%
\pgfpathmoveto{\pgfqpoint{0.590312in}{0.684620in}}%
\pgfpathlineto{\pgfqpoint{0.517884in}{0.696169in}}%
\pgfpathmoveto{\pgfqpoint{0.590312in}{0.684620in}}%
\pgfpathlineto{\pgfqpoint{0.661196in}{0.663049in}}%
\pgfpathmoveto{\pgfqpoint{0.352544in}{0.709066in}}%
\pgfpathlineto{\pgfqpoint{0.517884in}{0.696169in}}%
\pgfpathmoveto{\pgfqpoint{0.352544in}{0.709066in}}%
\pgfpathlineto{\pgfqpoint{0.190332in}{0.716109in}}%
\pgfpathmoveto{\pgfqpoint{0.181023in}{0.405438in}}%
\pgfpathlineto{\pgfqpoint{0.190067in}{0.100000in}}%
\pgfpathmoveto{\pgfqpoint{0.181023in}{0.405438in}}%
\pgfpathlineto{\pgfqpoint{0.190332in}{0.716109in}}%
\pgfpathmoveto{\pgfqpoint{1.377207in}{0.371708in}}%
\pgfpathlineto{\pgfqpoint{1.264340in}{0.100000in}}%
\pgfpathmoveto{\pgfqpoint{1.083002in}{0.353907in}}%
\pgfpathlineto{\pgfqpoint{0.995772in}{0.100000in}}%
\pgfpathmoveto{\pgfqpoint{1.230784in}{0.463627in}}%
\pgfpathlineto{\pgfqpoint{1.377207in}{0.371708in}}%
\pgfpathmoveto{\pgfqpoint{1.230784in}{0.463627in}}%
\pgfpathlineto{\pgfqpoint{1.083002in}{0.353907in}}%
\pgfpathmoveto{\pgfqpoint{0.951674in}{0.449240in}}%
\pgfpathlineto{\pgfqpoint{1.083002in}{0.353907in}}%
\pgfpathmoveto{\pgfqpoint{0.951674in}{0.449240in}}%
\pgfpathlineto{\pgfqpoint{0.815316in}{0.370020in}}%
\pgfpathmoveto{\pgfqpoint{0.672687in}{0.479220in}}%
\pgfpathlineto{\pgfqpoint{0.815316in}{0.370020in}}%
\pgfpathmoveto{\pgfqpoint{0.672687in}{0.479220in}}%
\pgfpathlineto{\pgfqpoint{0.534657in}{0.384219in}}%
\pgfpathmoveto{\pgfqpoint{1.529457in}{0.488685in}}%
\pgfpathlineto{\pgfqpoint{1.796841in}{0.415884in}}%
\pgfpathmoveto{\pgfqpoint{1.529457in}{0.488685in}}%
\pgfpathlineto{\pgfqpoint{1.377207in}{0.371708in}}%
\pgfpathmoveto{\pgfqpoint{1.369377in}{0.514759in}}%
\pgfpathlineto{\pgfqpoint{1.377207in}{0.371708in}}%
\pgfpathmoveto{\pgfqpoint{1.369377in}{0.514759in}}%
\pgfpathlineto{\pgfqpoint{1.230784in}{0.463627in}}%
\pgfpathmoveto{\pgfqpoint{1.369377in}{0.514759in}}%
\pgfpathlineto{\pgfqpoint{1.529457in}{0.488685in}}%
\pgfpathmoveto{\pgfqpoint{1.089290in}{0.480619in}}%
\pgfpathlineto{\pgfqpoint{1.083002in}{0.353907in}}%
\pgfpathmoveto{\pgfqpoint{1.089290in}{0.480619in}}%
\pgfpathlineto{\pgfqpoint{1.230784in}{0.463627in}}%
\pgfpathmoveto{\pgfqpoint{1.089290in}{0.480619in}}%
\pgfpathlineto{\pgfqpoint{0.951674in}{0.449240in}}%
\pgfpathmoveto{\pgfqpoint{0.812716in}{0.489158in}}%
\pgfpathlineto{\pgfqpoint{0.815316in}{0.370020in}}%
\pgfpathmoveto{\pgfqpoint{0.812716in}{0.489158in}}%
\pgfpathlineto{\pgfqpoint{0.951674in}{0.449240in}}%
\pgfpathmoveto{\pgfqpoint{0.812716in}{0.489158in}}%
\pgfpathlineto{\pgfqpoint{0.672687in}{0.479220in}}%
\pgfpathmoveto{\pgfqpoint{0.523931in}{0.529297in}}%
\pgfpathlineto{\pgfqpoint{0.534657in}{0.384219in}}%
\pgfpathmoveto{\pgfqpoint{0.523931in}{0.529297in}}%
\pgfpathlineto{\pgfqpoint{0.672687in}{0.479220in}}%
\pgfpathmoveto{\pgfqpoint{0.356202in}{0.490685in}}%
\pgfpathlineto{\pgfqpoint{0.190332in}{0.716109in}}%
\pgfpathmoveto{\pgfqpoint{0.356202in}{0.490685in}}%
\pgfpathlineto{\pgfqpoint{0.352544in}{0.709066in}}%
\pgfpathmoveto{\pgfqpoint{0.356202in}{0.490685in}}%
\pgfpathlineto{\pgfqpoint{0.181023in}{0.405438in}}%
\pgfpathmoveto{\pgfqpoint{0.356202in}{0.490685in}}%
\pgfpathlineto{\pgfqpoint{0.534657in}{0.384219in}}%
\pgfpathmoveto{\pgfqpoint{0.356202in}{0.490685in}}%
\pgfpathlineto{\pgfqpoint{0.523931in}{0.529297in}}%
\pgfpathmoveto{\pgfqpoint{1.289734in}{0.563347in}}%
\pgfpathlineto{\pgfqpoint{1.300610in}{0.658103in}}%
\pgfpathmoveto{\pgfqpoint{1.289734in}{0.563347in}}%
\pgfpathlineto{\pgfqpoint{1.245113in}{0.640065in}}%
\pgfpathmoveto{\pgfqpoint{1.289734in}{0.563347in}}%
\pgfpathlineto{\pgfqpoint{1.230784in}{0.463627in}}%
\pgfpathmoveto{\pgfqpoint{1.289734in}{0.563347in}}%
\pgfpathlineto{\pgfqpoint{1.369377in}{0.514759in}}%
\pgfpathmoveto{\pgfqpoint{1.157523in}{0.542874in}}%
\pgfpathlineto{\pgfqpoint{1.187563in}{0.626883in}}%
\pgfpathmoveto{\pgfqpoint{1.157523in}{0.542874in}}%
\pgfpathlineto{\pgfqpoint{1.121983in}{0.616327in}}%
\pgfpathmoveto{\pgfqpoint{1.157523in}{0.542874in}}%
\pgfpathlineto{\pgfqpoint{1.230784in}{0.463627in}}%
\pgfpathmoveto{\pgfqpoint{1.157523in}{0.542874in}}%
\pgfpathlineto{\pgfqpoint{1.089290in}{0.480619in}}%
\pgfpathmoveto{\pgfqpoint{1.021299in}{0.531634in}}%
\pgfpathlineto{\pgfqpoint{1.054136in}{0.608017in}}%
\pgfpathmoveto{\pgfqpoint{1.021299in}{0.531634in}}%
\pgfpathlineto{\pgfqpoint{0.985700in}{0.601411in}}%
\pgfpathmoveto{\pgfqpoint{1.021299in}{0.531634in}}%
\pgfpathlineto{\pgfqpoint{0.951674in}{0.449240in}}%
\pgfpathmoveto{\pgfqpoint{1.021299in}{0.531634in}}%
\pgfpathlineto{\pgfqpoint{1.089290in}{0.480619in}}%
\pgfpathmoveto{\pgfqpoint{0.881689in}{0.534959in}}%
\pgfpathlineto{\pgfqpoint{0.916417in}{0.603252in}}%
\pgfpathmoveto{\pgfqpoint{0.881689in}{0.534959in}}%
\pgfpathlineto{\pgfqpoint{0.848834in}{0.613810in}}%
\pgfpathmoveto{\pgfqpoint{0.881689in}{0.534959in}}%
\pgfpathlineto{\pgfqpoint{0.951674in}{0.449240in}}%
\pgfpathmoveto{\pgfqpoint{0.881689in}{0.534959in}}%
\pgfpathlineto{\pgfqpoint{0.812716in}{0.489158in}}%
\pgfpathmoveto{\pgfqpoint{0.745570in}{0.554174in}}%
\pgfpathlineto{\pgfqpoint{0.783038in}{0.628109in}}%
\pgfpathmoveto{\pgfqpoint{0.745570in}{0.554174in}}%
\pgfpathlineto{\pgfqpoint{0.721820in}{0.643258in}}%
\pgfpathmoveto{\pgfqpoint{0.745570in}{0.554174in}}%
\pgfpathlineto{\pgfqpoint{0.672687in}{0.479220in}}%
\pgfpathmoveto{\pgfqpoint{0.745570in}{0.554174in}}%
\pgfpathlineto{\pgfqpoint{0.812716in}{0.489158in}}%
\pgfpathmoveto{\pgfqpoint{0.610324in}{0.585562in}}%
\pgfpathlineto{\pgfqpoint{0.661196in}{0.663049in}}%
\pgfpathmoveto{\pgfqpoint{0.610324in}{0.585562in}}%
\pgfpathlineto{\pgfqpoint{0.590312in}{0.684620in}}%
\pgfpathmoveto{\pgfqpoint{0.610324in}{0.585562in}}%
\pgfpathlineto{\pgfqpoint{0.672687in}{0.479220in}}%
\pgfpathmoveto{\pgfqpoint{0.610324in}{0.585562in}}%
\pgfpathlineto{\pgfqpoint{0.523931in}{0.529297in}}%
\pgfpathmoveto{\pgfqpoint{1.433326in}{0.603291in}}%
\pgfpathlineto{\pgfqpoint{1.438592in}{0.697903in}}%
\pgfpathmoveto{\pgfqpoint{1.433326in}{0.603291in}}%
\pgfpathlineto{\pgfqpoint{1.358413in}{0.681892in}}%
\pgfpathmoveto{\pgfqpoint{1.433326in}{0.603291in}}%
\pgfpathlineto{\pgfqpoint{1.529457in}{0.488685in}}%
\pgfpathmoveto{\pgfqpoint{1.433326in}{0.603291in}}%
\pgfpathlineto{\pgfqpoint{1.369377in}{0.514759in}}%
\pgfpathmoveto{\pgfqpoint{0.938271in}{0.294479in}}%
\pgfpathlineto{\pgfqpoint{0.995772in}{0.100000in}}%
\pgfpathmoveto{\pgfqpoint{0.938271in}{0.294479in}}%
\pgfpathlineto{\pgfqpoint{1.083002in}{0.353907in}}%
\pgfpathmoveto{\pgfqpoint{0.938271in}{0.294479in}}%
\pgfpathlineto{\pgfqpoint{0.815316in}{0.370020in}}%
\pgfpathmoveto{\pgfqpoint{0.938271in}{0.294479in}}%
\pgfpathlineto{\pgfqpoint{0.951674in}{0.449240in}}%
\pgfpathmoveto{\pgfqpoint{0.437987in}{0.601518in}}%
\pgfpathlineto{\pgfqpoint{0.517884in}{0.696169in}}%
\pgfpathmoveto{\pgfqpoint{0.437987in}{0.601518in}}%
\pgfpathlineto{\pgfqpoint{0.352544in}{0.709066in}}%
\pgfpathmoveto{\pgfqpoint{0.437987in}{0.601518in}}%
\pgfpathlineto{\pgfqpoint{0.523931in}{0.529297in}}%
\pgfpathmoveto{\pgfqpoint{0.437987in}{0.601518in}}%
\pgfpathlineto{\pgfqpoint{0.356202in}{0.490685in}}%
\pgfpathmoveto{\pgfqpoint{1.351528in}{0.601963in}}%
\pgfpathlineto{\pgfqpoint{1.358413in}{0.681892in}}%
\pgfpathmoveto{\pgfqpoint{1.351528in}{0.601963in}}%
\pgfpathlineto{\pgfqpoint{1.300610in}{0.658103in}}%
\pgfpathmoveto{\pgfqpoint{1.351528in}{0.601963in}}%
\pgfpathlineto{\pgfqpoint{1.369377in}{0.514759in}}%
\pgfpathmoveto{\pgfqpoint{1.351528in}{0.601963in}}%
\pgfpathlineto{\pgfqpoint{1.289734in}{0.563347in}}%
\pgfpathmoveto{\pgfqpoint{1.351528in}{0.601963in}}%
\pgfpathlineto{\pgfqpoint{1.433326in}{0.603291in}}%
\pgfpathmoveto{\pgfqpoint{1.088974in}{0.555599in}}%
\pgfpathlineto{\pgfqpoint{1.121983in}{0.616327in}}%
\pgfpathmoveto{\pgfqpoint{1.088974in}{0.555599in}}%
\pgfpathlineto{\pgfqpoint{1.054136in}{0.608017in}}%
\pgfpathmoveto{\pgfqpoint{1.088974in}{0.555599in}}%
\pgfpathlineto{\pgfqpoint{1.089290in}{0.480619in}}%
\pgfpathmoveto{\pgfqpoint{1.088974in}{0.555599in}}%
\pgfpathlineto{\pgfqpoint{1.157523in}{0.542874in}}%
\pgfpathmoveto{\pgfqpoint{1.088974in}{0.555599in}}%
\pgfpathlineto{\pgfqpoint{1.021299in}{0.531634in}}%
\pgfpathmoveto{\pgfqpoint{1.222194in}{0.569805in}}%
\pgfpathlineto{\pgfqpoint{1.245113in}{0.640065in}}%
\pgfpathmoveto{\pgfqpoint{1.222194in}{0.569805in}}%
\pgfpathlineto{\pgfqpoint{1.187563in}{0.626883in}}%
\pgfpathmoveto{\pgfqpoint{1.222194in}{0.569805in}}%
\pgfpathlineto{\pgfqpoint{1.230784in}{0.463627in}}%
\pgfpathmoveto{\pgfqpoint{1.222194in}{0.569805in}}%
\pgfpathlineto{\pgfqpoint{1.289734in}{0.563347in}}%
\pgfpathmoveto{\pgfqpoint{1.222194in}{0.569805in}}%
\pgfpathlineto{\pgfqpoint{1.157523in}{0.542874in}}%
\pgfpathmoveto{\pgfqpoint{0.951314in}{0.543497in}}%
\pgfpathlineto{\pgfqpoint{0.985700in}{0.601411in}}%
\pgfpathmoveto{\pgfqpoint{0.951314in}{0.543497in}}%
\pgfpathlineto{\pgfqpoint{0.916417in}{0.603252in}}%
\pgfpathmoveto{\pgfqpoint{0.951314in}{0.543497in}}%
\pgfpathlineto{\pgfqpoint{0.951674in}{0.449240in}}%
\pgfpathmoveto{\pgfqpoint{0.951314in}{0.543497in}}%
\pgfpathlineto{\pgfqpoint{1.021299in}{0.531634in}}%
\pgfpathmoveto{\pgfqpoint{0.951314in}{0.543497in}}%
\pgfpathlineto{\pgfqpoint{0.881689in}{0.534959in}}%
\pgfpathmoveto{\pgfqpoint{0.813693in}{0.562058in}}%
\pgfpathlineto{\pgfqpoint{0.848834in}{0.613810in}}%
\pgfpathmoveto{\pgfqpoint{0.813693in}{0.562058in}}%
\pgfpathlineto{\pgfqpoint{0.783038in}{0.628109in}}%
\pgfpathmoveto{\pgfqpoint{0.813693in}{0.562058in}}%
\pgfpathlineto{\pgfqpoint{0.812716in}{0.489158in}}%
\pgfpathmoveto{\pgfqpoint{0.813693in}{0.562058in}}%
\pgfpathlineto{\pgfqpoint{0.881689in}{0.534959in}}%
\pgfpathmoveto{\pgfqpoint{0.813693in}{0.562058in}}%
\pgfpathlineto{\pgfqpoint{0.745570in}{0.554174in}}%
\pgfpathmoveto{\pgfqpoint{1.192290in}{0.275687in}}%
\pgfpathlineto{\pgfqpoint{0.995772in}{0.100000in}}%
\pgfpathmoveto{\pgfqpoint{1.192290in}{0.275687in}}%
\pgfpathlineto{\pgfqpoint{1.264340in}{0.100000in}}%
\pgfpathmoveto{\pgfqpoint{1.192290in}{0.275687in}}%
\pgfpathlineto{\pgfqpoint{1.377207in}{0.371708in}}%
\pgfpathmoveto{\pgfqpoint{1.192290in}{0.275687in}}%
\pgfpathlineto{\pgfqpoint{1.083002in}{0.353907in}}%
\pgfpathmoveto{\pgfqpoint{1.192290in}{0.275687in}}%
\pgfpathlineto{\pgfqpoint{1.230784in}{0.463627in}}%
\pgfpathmoveto{\pgfqpoint{0.691426in}{0.290497in}}%
\pgfpathlineto{\pgfqpoint{0.727204in}{0.100000in}}%
\pgfpathmoveto{\pgfqpoint{0.691426in}{0.290497in}}%
\pgfpathlineto{\pgfqpoint{0.815316in}{0.370020in}}%
\pgfpathmoveto{\pgfqpoint{0.691426in}{0.290497in}}%
\pgfpathlineto{\pgfqpoint{0.534657in}{0.384219in}}%
\pgfpathmoveto{\pgfqpoint{0.691426in}{0.290497in}}%
\pgfpathlineto{\pgfqpoint{0.672687in}{0.479220in}}%
\pgfpathmoveto{\pgfqpoint{0.680324in}{0.583817in}}%
\pgfpathlineto{\pgfqpoint{0.721820in}{0.643258in}}%
\pgfpathmoveto{\pgfqpoint{0.680324in}{0.583817in}}%
\pgfpathlineto{\pgfqpoint{0.661196in}{0.663049in}}%
\pgfpathmoveto{\pgfqpoint{0.680324in}{0.583817in}}%
\pgfpathlineto{\pgfqpoint{0.672687in}{0.479220in}}%
\pgfpathmoveto{\pgfqpoint{0.680324in}{0.583817in}}%
\pgfpathlineto{\pgfqpoint{0.745570in}{0.554174in}}%
\pgfpathmoveto{\pgfqpoint{0.680324in}{0.583817in}}%
\pgfpathlineto{\pgfqpoint{0.610324in}{0.585562in}}%
\pgfpathmoveto{\pgfqpoint{1.535895in}{0.617892in}}%
\pgfpathlineto{\pgfqpoint{1.438592in}{0.697903in}}%
\pgfpathmoveto{\pgfqpoint{1.535895in}{0.617892in}}%
\pgfpathlineto{\pgfqpoint{1.611368in}{0.716211in}}%
\pgfpathmoveto{\pgfqpoint{1.535895in}{0.617892in}}%
\pgfpathlineto{\pgfqpoint{1.529457in}{0.488685in}}%
\pgfpathmoveto{\pgfqpoint{1.535895in}{0.617892in}}%
\pgfpathlineto{\pgfqpoint{1.433326in}{0.603291in}}%
\pgfpathmoveto{\pgfqpoint{0.545528in}{0.620176in}}%
\pgfpathlineto{\pgfqpoint{0.517884in}{0.696169in}}%
\pgfpathmoveto{\pgfqpoint{0.545528in}{0.620176in}}%
\pgfpathlineto{\pgfqpoint{0.590312in}{0.684620in}}%
\pgfpathmoveto{\pgfqpoint{0.545528in}{0.620176in}}%
\pgfpathlineto{\pgfqpoint{0.523931in}{0.529297in}}%
\pgfpathmoveto{\pgfqpoint{0.545528in}{0.620176in}}%
\pgfpathlineto{\pgfqpoint{0.610324in}{0.585562in}}%
\pgfpathmoveto{\pgfqpoint{0.545528in}{0.620176in}}%
\pgfpathlineto{\pgfqpoint{0.437987in}{0.601518in}}%
\pgfpathmoveto{\pgfqpoint{0.379348in}{0.283472in}}%
\pgfpathlineto{\pgfqpoint{0.190067in}{0.100000in}}%
\pgfpathmoveto{\pgfqpoint{0.379348in}{0.283472in}}%
\pgfpathlineto{\pgfqpoint{0.458635in}{0.100000in}}%
\pgfpathmoveto{\pgfqpoint{0.379348in}{0.283472in}}%
\pgfpathlineto{\pgfqpoint{0.181023in}{0.405438in}}%
\pgfpathmoveto{\pgfqpoint{0.379348in}{0.283472in}}%
\pgfpathlineto{\pgfqpoint{0.534657in}{0.384219in}}%
\pgfpathmoveto{\pgfqpoint{0.379348in}{0.283472in}}%
\pgfpathlineto{\pgfqpoint{0.356202in}{0.490685in}}%
\pgfpathmoveto{\pgfqpoint{1.533495in}{0.293969in}}%
\pgfpathlineto{\pgfqpoint{1.801477in}{0.100000in}}%
\pgfpathmoveto{\pgfqpoint{1.533495in}{0.293969in}}%
\pgfpathlineto{\pgfqpoint{1.264340in}{0.100000in}}%
\pgfpathmoveto{\pgfqpoint{1.533495in}{0.293969in}}%
\pgfpathlineto{\pgfqpoint{1.532909in}{0.100000in}}%
\pgfpathmoveto{\pgfqpoint{1.533495in}{0.293969in}}%
\pgfpathlineto{\pgfqpoint{1.796841in}{0.415884in}}%
\pgfpathmoveto{\pgfqpoint{1.533495in}{0.293969in}}%
\pgfpathlineto{\pgfqpoint{1.377207in}{0.371708in}}%
\pgfpathmoveto{\pgfqpoint{1.533495in}{0.293969in}}%
\pgfpathlineto{\pgfqpoint{1.529457in}{0.488685in}}%
\pgfpathmoveto{\pgfqpoint{1.649933in}{0.594471in}}%
\pgfpathlineto{\pgfqpoint{1.774230in}{0.729694in}}%
\pgfpathmoveto{\pgfqpoint{1.649933in}{0.594471in}}%
\pgfpathlineto{\pgfqpoint{1.796841in}{0.415884in}}%
\pgfpathmoveto{\pgfqpoint{1.649933in}{0.594471in}}%
\pgfpathlineto{\pgfqpoint{1.611368in}{0.716211in}}%
\pgfpathmoveto{\pgfqpoint{1.649933in}{0.594471in}}%
\pgfpathlineto{\pgfqpoint{1.529457in}{0.488685in}}%
\pgfpathmoveto{\pgfqpoint{1.649933in}{0.594471in}}%
\pgfpathlineto{\pgfqpoint{1.535895in}{0.617892in}}%
\pgfpathmoveto{\pgfqpoint{0.841552in}{0.212299in}}%
\pgfpathlineto{\pgfqpoint{0.727204in}{0.100000in}}%
\pgfpathmoveto{\pgfqpoint{0.841552in}{0.212299in}}%
\pgfpathlineto{\pgfqpoint{0.995772in}{0.100000in}}%
\pgfpathmoveto{\pgfqpoint{0.841552in}{0.212299in}}%
\pgfpathlineto{\pgfqpoint{0.815316in}{0.370020in}}%
\pgfpathmoveto{\pgfqpoint{0.841552in}{0.212299in}}%
\pgfpathlineto{\pgfqpoint{0.938271in}{0.294479in}}%
\pgfpathmoveto{\pgfqpoint{0.841552in}{0.212299in}}%
\pgfpathlineto{\pgfqpoint{0.691426in}{0.290497in}}%
\pgfpathmoveto{\pgfqpoint{0.556128in}{0.231085in}}%
\pgfpathlineto{\pgfqpoint{0.458635in}{0.100000in}}%
\pgfpathmoveto{\pgfqpoint{0.556128in}{0.231085in}}%
\pgfpathlineto{\pgfqpoint{0.727204in}{0.100000in}}%
\pgfpathmoveto{\pgfqpoint{0.556128in}{0.231085in}}%
\pgfpathlineto{\pgfqpoint{0.534657in}{0.384219in}}%
\pgfpathmoveto{\pgfqpoint{0.556128in}{0.231085in}}%
\pgfpathlineto{\pgfqpoint{0.691426in}{0.290497in}}%
\pgfpathmoveto{\pgfqpoint{0.556128in}{0.231085in}}%
\pgfpathlineto{\pgfqpoint{0.379348in}{0.283472in}}%
\pgfpathlineto{\pgfqpoint{0.379348in}{0.283472in}}%
\pgfusepath{stroke}%
\end{pgfscope}%
\begin{pgfscope}%
\pgfpathrectangle{\pgfqpoint{0.100000in}{0.100000in}}{\pgfqpoint{1.782500in}{1.232000in}}%
\pgfusepath{clip}%
\pgfsetrectcap%
\pgfsetroundjoin%
\pgfsetlinewidth{0.250937pt}%
\definecolor{currentstroke}{rgb}{0.835294,0.321569,0.035294}%
\pgfsetstrokecolor{currentstroke}%
\pgfsetdash{}{0pt}%
\pgfpathmoveto{\pgfqpoint{0.488920in}{0.850184in}}%
\pgfpathlineto{\pgfqpoint{0.458635in}{1.085600in}}%
\pgfpathmoveto{\pgfqpoint{1.532909in}{1.085600in}}%
\pgfpathlineto{\pgfqpoint{1.494399in}{0.840973in}}%
\pgfpathmoveto{\pgfqpoint{0.727204in}{1.085600in}}%
\pgfpathlineto{\pgfqpoint{0.995772in}{1.085600in}}%
\pgfpathmoveto{\pgfqpoint{0.727204in}{1.085600in}}%
\pgfpathlineto{\pgfqpoint{0.458635in}{1.085600in}}%
\pgfpathmoveto{\pgfqpoint{0.529202in}{0.776277in}}%
\pgfpathlineto{\pgfqpoint{0.488920in}{0.850184in}}%
\pgfpathmoveto{\pgfqpoint{0.576443in}{0.718597in}}%
\pgfpathlineto{\pgfqpoint{0.529202in}{0.776277in}}%
\pgfpathmoveto{\pgfqpoint{0.649179in}{0.669718in}}%
\pgfpathlineto{\pgfqpoint{0.576443in}{0.718597in}}%
\pgfpathmoveto{\pgfqpoint{0.742562in}{0.637869in}}%
\pgfpathlineto{\pgfqpoint{0.649179in}{0.669718in}}%
\pgfpathmoveto{\pgfqpoint{0.828308in}{0.618003in}}%
\pgfpathlineto{\pgfqpoint{0.742562in}{0.637869in}}%
\pgfpathmoveto{\pgfqpoint{0.907725in}{0.603940in}}%
\pgfpathlineto{\pgfqpoint{0.828308in}{0.618003in}}%
\pgfpathmoveto{\pgfqpoint{0.985741in}{0.601166in}}%
\pgfpathlineto{\pgfqpoint{0.907725in}{0.603940in}}%
\pgfpathmoveto{\pgfqpoint{1.062700in}{0.609075in}}%
\pgfpathlineto{\pgfqpoint{0.985741in}{0.601166in}}%
\pgfpathmoveto{\pgfqpoint{1.142662in}{0.619169in}}%
\pgfpathlineto{\pgfqpoint{1.062700in}{0.609075in}}%
\pgfpathmoveto{\pgfqpoint{1.226275in}{0.635125in}}%
\pgfpathlineto{\pgfqpoint{1.142662in}{0.619169in}}%
\pgfpathmoveto{\pgfqpoint{1.313856in}{0.663445in}}%
\pgfpathlineto{\pgfqpoint{1.226275in}{0.635125in}}%
\pgfpathmoveto{\pgfqpoint{1.401434in}{0.701100in}}%
\pgfpathlineto{\pgfqpoint{1.313856in}{0.663445in}}%
\pgfpathmoveto{\pgfqpoint{1.451289in}{0.764757in}}%
\pgfpathlineto{\pgfqpoint{1.494399in}{0.840973in}}%
\pgfpathmoveto{\pgfqpoint{1.451289in}{0.764757in}}%
\pgfpathlineto{\pgfqpoint{1.401434in}{0.701100in}}%
\pgfpathmoveto{\pgfqpoint{1.264340in}{1.085600in}}%
\pgfpathlineto{\pgfqpoint{0.995772in}{1.085600in}}%
\pgfpathmoveto{\pgfqpoint{1.264340in}{1.085600in}}%
\pgfpathlineto{\pgfqpoint{1.532909in}{1.085600in}}%
\pgfpathmoveto{\pgfqpoint{0.830943in}{0.887574in}}%
\pgfpathlineto{\pgfqpoint{0.995772in}{1.085600in}}%
\pgfpathmoveto{\pgfqpoint{0.830943in}{0.887574in}}%
\pgfpathlineto{\pgfqpoint{0.727204in}{1.085600in}}%
\pgfpathmoveto{\pgfqpoint{0.957322in}{0.756420in}}%
\pgfpathlineto{\pgfqpoint{1.117688in}{0.824660in}}%
\pgfpathmoveto{\pgfqpoint{0.957322in}{0.756420in}}%
\pgfpathlineto{\pgfqpoint{0.830943in}{0.887574in}}%
\pgfpathmoveto{\pgfqpoint{1.287084in}{0.857316in}}%
\pgfpathlineto{\pgfqpoint{1.117688in}{0.824660in}}%
\pgfpathmoveto{\pgfqpoint{0.665556in}{0.898299in}}%
\pgfpathlineto{\pgfqpoint{0.727204in}{1.085600in}}%
\pgfpathmoveto{\pgfqpoint{0.665556in}{0.898299in}}%
\pgfpathlineto{\pgfqpoint{0.830943in}{0.887574in}}%
\pgfpathmoveto{\pgfqpoint{1.090371in}{0.720244in}}%
\pgfpathlineto{\pgfqpoint{1.062700in}{0.609075in}}%
\pgfpathmoveto{\pgfqpoint{1.090371in}{0.720244in}}%
\pgfpathlineto{\pgfqpoint{1.142662in}{0.619169in}}%
\pgfpathmoveto{\pgfqpoint{1.090371in}{0.720244in}}%
\pgfpathlineto{\pgfqpoint{1.117688in}{0.824660in}}%
\pgfpathmoveto{\pgfqpoint{1.090371in}{0.720244in}}%
\pgfpathlineto{\pgfqpoint{0.957322in}{0.756420in}}%
\pgfpathmoveto{\pgfqpoint{1.219108in}{0.751293in}}%
\pgfpathlineto{\pgfqpoint{1.226275in}{0.635125in}}%
\pgfpathmoveto{\pgfqpoint{1.219108in}{0.751293in}}%
\pgfpathlineto{\pgfqpoint{1.313856in}{0.663445in}}%
\pgfpathmoveto{\pgfqpoint{1.219108in}{0.751293in}}%
\pgfpathlineto{\pgfqpoint{1.117688in}{0.824660in}}%
\pgfpathmoveto{\pgfqpoint{1.219108in}{0.751293in}}%
\pgfpathlineto{\pgfqpoint{1.287084in}{0.857316in}}%
\pgfpathmoveto{\pgfqpoint{1.219108in}{0.751293in}}%
\pgfpathlineto{\pgfqpoint{1.090371in}{0.720244in}}%
\pgfpathmoveto{\pgfqpoint{0.812724in}{0.731232in}}%
\pgfpathlineto{\pgfqpoint{0.742562in}{0.637869in}}%
\pgfpathmoveto{\pgfqpoint{0.812724in}{0.731232in}}%
\pgfpathlineto{\pgfqpoint{0.828308in}{0.618003in}}%
\pgfpathmoveto{\pgfqpoint{0.812724in}{0.731232in}}%
\pgfpathlineto{\pgfqpoint{0.830943in}{0.887574in}}%
\pgfpathmoveto{\pgfqpoint{0.812724in}{0.731232in}}%
\pgfpathlineto{\pgfqpoint{0.957322in}{0.756420in}}%
\pgfpathmoveto{\pgfqpoint{0.701489in}{0.784036in}}%
\pgfpathlineto{\pgfqpoint{0.576443in}{0.718597in}}%
\pgfpathmoveto{\pgfqpoint{0.701489in}{0.784036in}}%
\pgfpathlineto{\pgfqpoint{0.649179in}{0.669718in}}%
\pgfpathmoveto{\pgfqpoint{0.701489in}{0.784036in}}%
\pgfpathlineto{\pgfqpoint{0.830943in}{0.887574in}}%
\pgfpathmoveto{\pgfqpoint{0.701489in}{0.784036in}}%
\pgfpathlineto{\pgfqpoint{0.665556in}{0.898299in}}%
\pgfpathmoveto{\pgfqpoint{0.701489in}{0.784036in}}%
\pgfpathlineto{\pgfqpoint{0.812724in}{0.731232in}}%
\pgfpathmoveto{\pgfqpoint{1.373352in}{0.959810in}}%
\pgfpathlineto{\pgfqpoint{1.494399in}{0.840973in}}%
\pgfpathmoveto{\pgfqpoint{1.373352in}{0.959810in}}%
\pgfpathlineto{\pgfqpoint{1.532909in}{1.085600in}}%
\pgfpathmoveto{\pgfqpoint{1.373352in}{0.959810in}}%
\pgfpathlineto{\pgfqpoint{1.264340in}{1.085600in}}%
\pgfpathmoveto{\pgfqpoint{1.373352in}{0.959810in}}%
\pgfpathlineto{\pgfqpoint{1.287084in}{0.857316in}}%
\pgfpathmoveto{\pgfqpoint{0.949513in}{0.659395in}}%
\pgfpathlineto{\pgfqpoint{0.907725in}{0.603940in}}%
\pgfpathmoveto{\pgfqpoint{0.949513in}{0.659395in}}%
\pgfpathlineto{\pgfqpoint{0.985741in}{0.601166in}}%
\pgfpathmoveto{\pgfqpoint{0.949513in}{0.659395in}}%
\pgfpathlineto{\pgfqpoint{0.957322in}{0.756420in}}%
\pgfpathmoveto{\pgfqpoint{1.310269in}{0.744148in}}%
\pgfpathlineto{\pgfqpoint{1.313856in}{0.663445in}}%
\pgfpathmoveto{\pgfqpoint{1.310269in}{0.744148in}}%
\pgfpathlineto{\pgfqpoint{1.401434in}{0.701100in}}%
\pgfpathmoveto{\pgfqpoint{1.310269in}{0.744148in}}%
\pgfpathlineto{\pgfqpoint{1.287084in}{0.857316in}}%
\pgfpathmoveto{\pgfqpoint{1.310269in}{0.744148in}}%
\pgfpathlineto{\pgfqpoint{1.219108in}{0.751293in}}%
\pgfpathmoveto{\pgfqpoint{1.167827in}{0.689125in}}%
\pgfpathlineto{\pgfqpoint{1.142662in}{0.619169in}}%
\pgfpathmoveto{\pgfqpoint{1.167827in}{0.689125in}}%
\pgfpathlineto{\pgfqpoint{1.226275in}{0.635125in}}%
\pgfpathmoveto{\pgfqpoint{1.167827in}{0.689125in}}%
\pgfpathlineto{\pgfqpoint{1.090371in}{0.720244in}}%
\pgfpathmoveto{\pgfqpoint{1.167827in}{0.689125in}}%
\pgfpathlineto{\pgfqpoint{1.219108in}{0.751293in}}%
\pgfpathmoveto{\pgfqpoint{1.390634in}{0.855971in}}%
\pgfpathlineto{\pgfqpoint{1.494399in}{0.840973in}}%
\pgfpathmoveto{\pgfqpoint{1.390634in}{0.855971in}}%
\pgfpathlineto{\pgfqpoint{1.451289in}{0.764757in}}%
\pgfpathmoveto{\pgfqpoint{1.390634in}{0.855971in}}%
\pgfpathlineto{\pgfqpoint{1.287084in}{0.857316in}}%
\pgfpathmoveto{\pgfqpoint{1.390634in}{0.855971in}}%
\pgfpathlineto{\pgfqpoint{1.373352in}{0.959810in}}%
\pgfpathmoveto{\pgfqpoint{0.728960in}{0.711312in}}%
\pgfpathlineto{\pgfqpoint{0.649179in}{0.669718in}}%
\pgfpathmoveto{\pgfqpoint{0.728960in}{0.711312in}}%
\pgfpathlineto{\pgfqpoint{0.742562in}{0.637869in}}%
\pgfpathmoveto{\pgfqpoint{0.728960in}{0.711312in}}%
\pgfpathlineto{\pgfqpoint{0.812724in}{0.731232in}}%
\pgfpathmoveto{\pgfqpoint{0.728960in}{0.711312in}}%
\pgfpathlineto{\pgfqpoint{0.701489in}{0.784036in}}%
\pgfpathmoveto{\pgfqpoint{0.616584in}{0.797742in}}%
\pgfpathlineto{\pgfqpoint{0.529202in}{0.776277in}}%
\pgfpathmoveto{\pgfqpoint{0.616584in}{0.797742in}}%
\pgfpathlineto{\pgfqpoint{0.576443in}{0.718597in}}%
\pgfpathmoveto{\pgfqpoint{0.616584in}{0.797742in}}%
\pgfpathlineto{\pgfqpoint{0.665556in}{0.898299in}}%
\pgfpathmoveto{\pgfqpoint{0.616584in}{0.797742in}}%
\pgfpathlineto{\pgfqpoint{0.701489in}{0.784036in}}%
\pgfpathmoveto{\pgfqpoint{0.567574in}{0.969290in}}%
\pgfpathlineto{\pgfqpoint{0.458635in}{1.085600in}}%
\pgfpathmoveto{\pgfqpoint{0.567574in}{0.969290in}}%
\pgfpathlineto{\pgfqpoint{0.488920in}{0.850184in}}%
\pgfpathmoveto{\pgfqpoint{0.567574in}{0.969290in}}%
\pgfpathlineto{\pgfqpoint{0.727204in}{1.085600in}}%
\pgfpathmoveto{\pgfqpoint{0.567574in}{0.969290in}}%
\pgfpathlineto{\pgfqpoint{0.665556in}{0.898299in}}%
\pgfpathmoveto{\pgfqpoint{1.022401in}{0.663058in}}%
\pgfpathlineto{\pgfqpoint{0.985741in}{0.601166in}}%
\pgfpathmoveto{\pgfqpoint{1.022401in}{0.663058in}}%
\pgfpathlineto{\pgfqpoint{1.062700in}{0.609075in}}%
\pgfpathmoveto{\pgfqpoint{1.022401in}{0.663058in}}%
\pgfpathlineto{\pgfqpoint{0.957322in}{0.756420in}}%
\pgfpathmoveto{\pgfqpoint{1.022401in}{0.663058in}}%
\pgfpathlineto{\pgfqpoint{1.090371in}{0.720244in}}%
\pgfpathmoveto{\pgfqpoint{1.022401in}{0.663058in}}%
\pgfpathlineto{\pgfqpoint{0.949513in}{0.659395in}}%
\pgfpathmoveto{\pgfqpoint{0.876764in}{0.668875in}}%
\pgfpathlineto{\pgfqpoint{0.828308in}{0.618003in}}%
\pgfpathmoveto{\pgfqpoint{0.876764in}{0.668875in}}%
\pgfpathlineto{\pgfqpoint{0.907725in}{0.603940in}}%
\pgfpathmoveto{\pgfqpoint{0.876764in}{0.668875in}}%
\pgfpathlineto{\pgfqpoint{0.957322in}{0.756420in}}%
\pgfpathmoveto{\pgfqpoint{0.876764in}{0.668875in}}%
\pgfpathlineto{\pgfqpoint{0.812724in}{0.731232in}}%
\pgfpathmoveto{\pgfqpoint{0.876764in}{0.668875in}}%
\pgfpathlineto{\pgfqpoint{0.949513in}{0.659395in}}%
\pgfpathmoveto{\pgfqpoint{1.374566in}{0.778570in}}%
\pgfpathlineto{\pgfqpoint{1.401434in}{0.701100in}}%
\pgfpathmoveto{\pgfqpoint{1.374566in}{0.778570in}}%
\pgfpathlineto{\pgfqpoint{1.451289in}{0.764757in}}%
\pgfpathmoveto{\pgfqpoint{1.374566in}{0.778570in}}%
\pgfpathlineto{\pgfqpoint{1.287084in}{0.857316in}}%
\pgfpathmoveto{\pgfqpoint{1.374566in}{0.778570in}}%
\pgfpathlineto{\pgfqpoint{1.310269in}{0.744148in}}%
\pgfpathmoveto{\pgfqpoint{1.374566in}{0.778570in}}%
\pgfpathlineto{\pgfqpoint{1.390634in}{0.855971in}}%
\pgfpathmoveto{\pgfqpoint{1.016954in}{0.905905in}}%
\pgfpathlineto{\pgfqpoint{0.995772in}{1.085600in}}%
\pgfpathmoveto{\pgfqpoint{1.016954in}{0.905905in}}%
\pgfpathlineto{\pgfqpoint{1.117688in}{0.824660in}}%
\pgfpathmoveto{\pgfqpoint{1.016954in}{0.905905in}}%
\pgfpathlineto{\pgfqpoint{0.830943in}{0.887574in}}%
\pgfpathmoveto{\pgfqpoint{1.016954in}{0.905905in}}%
\pgfpathlineto{\pgfqpoint{0.957322in}{0.756420in}}%
\pgfpathmoveto{\pgfqpoint{1.178001in}{0.952692in}}%
\pgfpathlineto{\pgfqpoint{0.995772in}{1.085600in}}%
\pgfpathmoveto{\pgfqpoint{1.178001in}{0.952692in}}%
\pgfpathlineto{\pgfqpoint{1.264340in}{1.085600in}}%
\pgfpathmoveto{\pgfqpoint{1.178001in}{0.952692in}}%
\pgfpathlineto{\pgfqpoint{1.117688in}{0.824660in}}%
\pgfpathmoveto{\pgfqpoint{1.178001in}{0.952692in}}%
\pgfpathlineto{\pgfqpoint{1.287084in}{0.857316in}}%
\pgfpathmoveto{\pgfqpoint{1.178001in}{0.952692in}}%
\pgfpathlineto{\pgfqpoint{1.373352in}{0.959810in}}%
\pgfpathmoveto{\pgfqpoint{1.178001in}{0.952692in}}%
\pgfpathlineto{\pgfqpoint{1.016954in}{0.905905in}}%
\pgfpathmoveto{\pgfqpoint{0.572757in}{0.858321in}}%
\pgfpathlineto{\pgfqpoint{0.488920in}{0.850184in}}%
\pgfpathmoveto{\pgfqpoint{0.572757in}{0.858321in}}%
\pgfpathlineto{\pgfqpoint{0.529202in}{0.776277in}}%
\pgfpathmoveto{\pgfqpoint{0.572757in}{0.858321in}}%
\pgfpathlineto{\pgfqpoint{0.665556in}{0.898299in}}%
\pgfpathmoveto{\pgfqpoint{0.572757in}{0.858321in}}%
\pgfpathlineto{\pgfqpoint{0.616584in}{0.797742in}}%
\pgfpathmoveto{\pgfqpoint{0.572757in}{0.858321in}}%
\pgfpathlineto{\pgfqpoint{0.567574in}{0.969290in}}%
\pgfpathlineto{\pgfqpoint{0.567574in}{0.969290in}}%
\pgfusepath{stroke}%
\end{pgfscope}%
\begin{pgfscope}%
\pgfpathrectangle{\pgfqpoint{0.100000in}{0.100000in}}{\pgfqpoint{1.782500in}{1.232000in}}%
\pgfusepath{clip}%
\pgfsetbuttcap%
\pgfsetroundjoin%
\definecolor{currentfill}{rgb}{0.054902,0.262745,0.486275}%
\pgfsetfillcolor{currentfill}%
\pgfsetlinewidth{1.003750pt}%
\definecolor{currentstroke}{rgb}{0.054902,0.262745,0.486275}%
\pgfsetstrokecolor{currentstroke}%
\pgfsetdash{}{0pt}%
\pgfsys@defobject{currentmarker}{\pgfqpoint{-0.018373in}{-0.018373in}}{\pgfqpoint{0.018373in}{0.018373in}}{%
\pgfpathmoveto{\pgfqpoint{0.000000in}{-0.018373in}}%
\pgfpathcurveto{\pgfqpoint{0.004873in}{-0.018373in}}{\pgfqpoint{0.009546in}{-0.016437in}}{\pgfqpoint{0.012992in}{-0.012992in}}%
\pgfpathcurveto{\pgfqpoint{0.016437in}{-0.009546in}}{\pgfqpoint{0.018373in}{-0.004873in}}{\pgfqpoint{0.018373in}{0.000000in}}%
\pgfpathcurveto{\pgfqpoint{0.018373in}{0.004873in}}{\pgfqpoint{0.016437in}{0.009546in}}{\pgfqpoint{0.012992in}{0.012992in}}%
\pgfpathcurveto{\pgfqpoint{0.009546in}{0.016437in}}{\pgfqpoint{0.004873in}{0.018373in}}{\pgfqpoint{0.000000in}{0.018373in}}%
\pgfpathcurveto{\pgfqpoint{-0.004873in}{0.018373in}}{\pgfqpoint{-0.009546in}{0.016437in}}{\pgfqpoint{-0.012992in}{0.012992in}}%
\pgfpathcurveto{\pgfqpoint{-0.016437in}{0.009546in}}{\pgfqpoint{-0.018373in}{0.004873in}}{\pgfqpoint{-0.018373in}{0.000000in}}%
\pgfpathcurveto{\pgfqpoint{-0.018373in}{-0.004873in}}{\pgfqpoint{-0.016437in}{-0.009546in}}{\pgfqpoint{-0.012992in}{-0.012992in}}%
\pgfpathcurveto{\pgfqpoint{-0.009546in}{-0.016437in}}{\pgfqpoint{-0.004873in}{-0.018373in}}{\pgfqpoint{0.000000in}{-0.018373in}}%
\pgfpathlineto{\pgfqpoint{0.000000in}{-0.018373in}}%
\pgfpathclose%
\pgfusepath{stroke,fill}%
}%
\begin{pgfscope}%
\pgfsys@transformshift{1.358413in}{0.681892in}%
\pgfsys@useobject{currentmarker}{}%
\end{pgfscope}%
\begin{pgfscope}%
\pgfsys@transformshift{1.300610in}{0.658103in}%
\pgfsys@useobject{currentmarker}{}%
\end{pgfscope}%
\begin{pgfscope}%
\pgfsys@transformshift{1.245113in}{0.640065in}%
\pgfsys@useobject{currentmarker}{}%
\end{pgfscope}%
\begin{pgfscope}%
\pgfsys@transformshift{1.187563in}{0.626883in}%
\pgfsys@useobject{currentmarker}{}%
\end{pgfscope}%
\begin{pgfscope}%
\pgfsys@transformshift{1.121983in}{0.616327in}%
\pgfsys@useobject{currentmarker}{}%
\end{pgfscope}%
\begin{pgfscope}%
\pgfsys@transformshift{1.054136in}{0.608017in}%
\pgfsys@useobject{currentmarker}{}%
\end{pgfscope}%
\begin{pgfscope}%
\pgfsys@transformshift{0.985700in}{0.601411in}%
\pgfsys@useobject{currentmarker}{}%
\end{pgfscope}%
\begin{pgfscope}%
\pgfsys@transformshift{0.916417in}{0.603252in}%
\pgfsys@useobject{currentmarker}{}%
\end{pgfscope}%
\begin{pgfscope}%
\pgfsys@transformshift{0.848834in}{0.613810in}%
\pgfsys@useobject{currentmarker}{}%
\end{pgfscope}%
\begin{pgfscope}%
\pgfsys@transformshift{0.783038in}{0.628109in}%
\pgfsys@useobject{currentmarker}{}%
\end{pgfscope}%
\begin{pgfscope}%
\pgfsys@transformshift{0.721820in}{0.643258in}%
\pgfsys@useobject{currentmarker}{}%
\end{pgfscope}%
\begin{pgfscope}%
\pgfsys@transformshift{0.661196in}{0.663049in}%
\pgfsys@useobject{currentmarker}{}%
\end{pgfscope}%
\end{pgfscope}%
\begin{pgfscope}%
\pgfpathrectangle{\pgfqpoint{0.100000in}{0.100000in}}{\pgfqpoint{1.782500in}{1.232000in}}%
\pgfusepath{clip}%
\pgfsetbuttcap%
\pgfsetroundjoin%
\definecolor{currentfill}{rgb}{0.835294,0.321569,0.035294}%
\pgfsetfillcolor{currentfill}%
\pgfsetlinewidth{1.003750pt}%
\definecolor{currentstroke}{rgb}{0.835294,0.321569,0.035294}%
\pgfsetstrokecolor{currentstroke}%
\pgfsetdash{}{0pt}%
\pgfsys@defobject{currentmarker}{\pgfqpoint{-0.018373in}{-0.018373in}}{\pgfqpoint{0.018373in}{0.018373in}}{%
\pgfpathmoveto{\pgfqpoint{0.000000in}{-0.018373in}}%
\pgfpathcurveto{\pgfqpoint{0.004873in}{-0.018373in}}{\pgfqpoint{0.009546in}{-0.016437in}}{\pgfqpoint{0.012992in}{-0.012992in}}%
\pgfpathcurveto{\pgfqpoint{0.016437in}{-0.009546in}}{\pgfqpoint{0.018373in}{-0.004873in}}{\pgfqpoint{0.018373in}{0.000000in}}%
\pgfpathcurveto{\pgfqpoint{0.018373in}{0.004873in}}{\pgfqpoint{0.016437in}{0.009546in}}{\pgfqpoint{0.012992in}{0.012992in}}%
\pgfpathcurveto{\pgfqpoint{0.009546in}{0.016437in}}{\pgfqpoint{0.004873in}{0.018373in}}{\pgfqpoint{0.000000in}{0.018373in}}%
\pgfpathcurveto{\pgfqpoint{-0.004873in}{0.018373in}}{\pgfqpoint{-0.009546in}{0.016437in}}{\pgfqpoint{-0.012992in}{0.012992in}}%
\pgfpathcurveto{\pgfqpoint{-0.016437in}{0.009546in}}{\pgfqpoint{-0.018373in}{0.004873in}}{\pgfqpoint{-0.018373in}{0.000000in}}%
\pgfpathcurveto{\pgfqpoint{-0.018373in}{-0.004873in}}{\pgfqpoint{-0.016437in}{-0.009546in}}{\pgfqpoint{-0.012992in}{-0.012992in}}%
\pgfpathcurveto{\pgfqpoint{-0.009546in}{-0.016437in}}{\pgfqpoint{-0.004873in}{-0.018373in}}{\pgfqpoint{0.000000in}{-0.018373in}}%
\pgfpathlineto{\pgfqpoint{0.000000in}{-0.018373in}}%
\pgfpathclose%
\pgfusepath{stroke,fill}%
}%
\begin{pgfscope}%
\pgfsys@transformshift{0.576443in}{0.718597in}%
\pgfsys@useobject{currentmarker}{}%
\end{pgfscope}%
\begin{pgfscope}%
\pgfsys@transformshift{0.649179in}{0.669718in}%
\pgfsys@useobject{currentmarker}{}%
\end{pgfscope}%
\begin{pgfscope}%
\pgfsys@transformshift{0.742562in}{0.637869in}%
\pgfsys@useobject{currentmarker}{}%
\end{pgfscope}%
\begin{pgfscope}%
\pgfsys@transformshift{0.828308in}{0.618003in}%
\pgfsys@useobject{currentmarker}{}%
\end{pgfscope}%
\begin{pgfscope}%
\pgfsys@transformshift{0.907725in}{0.603940in}%
\pgfsys@useobject{currentmarker}{}%
\end{pgfscope}%
\begin{pgfscope}%
\pgfsys@transformshift{0.985741in}{0.601166in}%
\pgfsys@useobject{currentmarker}{}%
\end{pgfscope}%
\begin{pgfscope}%
\pgfsys@transformshift{1.062700in}{0.609075in}%
\pgfsys@useobject{currentmarker}{}%
\end{pgfscope}%
\begin{pgfscope}%
\pgfsys@transformshift{1.142662in}{0.619169in}%
\pgfsys@useobject{currentmarker}{}%
\end{pgfscope}%
\begin{pgfscope}%
\pgfsys@transformshift{1.226275in}{0.635125in}%
\pgfsys@useobject{currentmarker}{}%
\end{pgfscope}%
\begin{pgfscope}%
\pgfsys@transformshift{1.313856in}{0.663445in}%
\pgfsys@useobject{currentmarker}{}%
\end{pgfscope}%
\begin{pgfscope}%
\pgfsys@transformshift{1.401434in}{0.701100in}%
\pgfsys@useobject{currentmarker}{}%
\end{pgfscope}%
\end{pgfscope}%
\end{pgfpicture}%
\makeatother%
\endgroup%
}
        \caption{Iteration 2: Solve system}\label{fig:example-iter1-solution}
    \end{subfigure}
    \begin{subfigure}[b]{.32\linewidth}
        \scalebox{0.8}{%% Creator: Matplotlib, PGF backend
%%
%% To include the figure in your LaTeX document, write
%%   \input{<filename>.pgf}
%%
%% Make sure the required packages are loaded in your preamble
%%   \usepackage{pgf}
%%
%% Also ensure that all the required font packages are loaded; for instance,
%% the lmodern package is sometimes necessary when using math font.
%%   \usepackage{lmodern}
%%
%% Figures using additional raster images can only be included by \input if
%% they are in the same directory as the main LaTeX file. For loading figures
%% from other directories you can use the `import` package
%%   \usepackage{import}
%%
%% and then include the figures with
%%   \import{<path to file>}{<filename>.pgf}
%%
%% Matplotlib used the following preamble
%%   
%%   \usepackage{fontspec}
%%   \setmainfont{DejaVuSans.ttf}[Path=\detokenize{/home/fabio/Internodes-CM/.venv/lib/python3.8/site-packages/matplotlib/mpl-data/fonts/ttf/}]
%%   \setsansfont{DejaVuSans.ttf}[Path=\detokenize{/home/fabio/Internodes-CM/.venv/lib/python3.8/site-packages/matplotlib/mpl-data/fonts/ttf/}]
%%   \setmonofont{DejaVuSansMono.ttf}[Path=\detokenize{/home/fabio/Internodes-CM/.venv/lib/python3.8/site-packages/matplotlib/mpl-data/fonts/ttf/}]
%%   \makeatletter\@ifpackageloaded{underscore}{}{\usepackage[strings]{underscore}}\makeatother
%%
\begingroup%
\makeatletter%
\begin{pgfpicture}%
\pgfpathrectangle{\pgfpointorigin}{\pgfqpoint{1.982500in}{1.432000in}}%
\pgfusepath{use as bounding box, clip}%
\begin{pgfscope}%
\pgfsetbuttcap%
\pgfsetmiterjoin%
\definecolor{currentfill}{rgb}{1.000000,1.000000,1.000000}%
\pgfsetfillcolor{currentfill}%
\pgfsetlinewidth{0.000000pt}%
\definecolor{currentstroke}{rgb}{1.000000,1.000000,1.000000}%
\pgfsetstrokecolor{currentstroke}%
\pgfsetdash{}{0pt}%
\pgfpathmoveto{\pgfqpoint{0.000000in}{0.000000in}}%
\pgfpathlineto{\pgfqpoint{1.982500in}{0.000000in}}%
\pgfpathlineto{\pgfqpoint{1.982500in}{1.432000in}}%
\pgfpathlineto{\pgfqpoint{0.000000in}{1.432000in}}%
\pgfpathlineto{\pgfqpoint{0.000000in}{0.000000in}}%
\pgfpathclose%
\pgfusepath{fill}%
\end{pgfscope}%
\begin{pgfscope}%
\pgfpathrectangle{\pgfqpoint{0.100000in}{0.100000in}}{\pgfqpoint{1.782500in}{1.232000in}}%
\pgfusepath{clip}%
\pgfsetrectcap%
\pgfsetroundjoin%
\pgfsetlinewidth{0.250937pt}%
\definecolor{currentstroke}{rgb}{0.054902,0.262745,0.486275}%
\pgfsetstrokecolor{currentstroke}%
\pgfsetdash{}{0pt}%
\pgfpathmoveto{\pgfqpoint{1.860217in}{0.288302in}}%
\pgfpathlineto{\pgfqpoint{1.892500in}{0.290314in}}%
\pgfpathmoveto{\pgfqpoint{1.706938in}{0.290931in}}%
\pgfpathlineto{\pgfqpoint{1.860217in}{0.288302in}}%
\pgfpathmoveto{\pgfqpoint{1.557421in}{0.306015in}}%
\pgfpathlineto{\pgfqpoint{1.706938in}{0.290931in}}%
\pgfpathmoveto{\pgfqpoint{1.411859in}{0.326442in}}%
\pgfpathlineto{\pgfqpoint{1.557421in}{0.306015in}}%
\pgfpathmoveto{\pgfqpoint{1.276422in}{0.348083in}}%
\pgfpathlineto{\pgfqpoint{1.411859in}{0.326442in}}%
\pgfpathmoveto{\pgfqpoint{1.142300in}{0.376355in}}%
\pgfpathlineto{\pgfqpoint{1.276422in}{0.348083in}}%
\pgfpathmoveto{\pgfqpoint{0.985482in}{0.407171in}}%
\pgfpathlineto{\pgfqpoint{0.825246in}{0.423670in}}%
\pgfpathmoveto{\pgfqpoint{0.985482in}{0.407171in}}%
\pgfpathlineto{\pgfqpoint{1.142300in}{0.376355in}}%
\pgfpathmoveto{\pgfqpoint{0.459457in}{0.442094in}}%
\pgfpathlineto{\pgfqpoint{0.825246in}{0.423670in}}%
\pgfpathmoveto{\pgfqpoint{0.459457in}{0.442094in}}%
\pgfpathlineto{\pgfqpoint{0.100586in}{0.452156in}}%
\pgfpathmoveto{\pgfqpoint{0.090000in}{0.224033in}}%
\pgfpathlineto{\pgfqpoint{0.100586in}{0.452156in}}%
\pgfpathmoveto{\pgfqpoint{1.167724in}{0.113743in}}%
\pgfpathlineto{\pgfqpoint{1.215749in}{0.090000in}}%
\pgfpathmoveto{\pgfqpoint{1.167724in}{0.113743in}}%
\pgfpathlineto{\pgfqpoint{1.114301in}{0.090000in}}%
\pgfpathmoveto{\pgfqpoint{1.477516in}{0.127940in}}%
\pgfpathlineto{\pgfqpoint{1.478798in}{0.090000in}}%
\pgfpathmoveto{\pgfqpoint{1.477516in}{0.127940in}}%
\pgfpathlineto{\pgfqpoint{1.682049in}{0.090000in}}%
\pgfpathmoveto{\pgfqpoint{1.477516in}{0.127940in}}%
\pgfpathlineto{\pgfqpoint{1.167724in}{0.113743in}}%
\pgfpathmoveto{\pgfqpoint{0.838623in}{0.185281in}}%
\pgfpathlineto{\pgfqpoint{0.849532in}{0.090000in}}%
\pgfpathmoveto{\pgfqpoint{0.838623in}{0.185281in}}%
\pgfpathlineto{\pgfqpoint{1.167724in}{0.113743in}}%
\pgfpathmoveto{\pgfqpoint{0.467549in}{0.130122in}}%
\pgfpathlineto{\pgfqpoint{0.100586in}{0.452156in}}%
\pgfpathmoveto{\pgfqpoint{0.467549in}{0.130122in}}%
\pgfpathlineto{\pgfqpoint{0.459457in}{0.442094in}}%
\pgfpathmoveto{\pgfqpoint{0.467549in}{0.130122in}}%
\pgfpathlineto{\pgfqpoint{0.339865in}{0.090000in}}%
\pgfpathmoveto{\pgfqpoint{0.467549in}{0.130122in}}%
\pgfpathlineto{\pgfqpoint{0.571696in}{0.090000in}}%
\pgfpathmoveto{\pgfqpoint{0.467549in}{0.130122in}}%
\pgfpathlineto{\pgfqpoint{0.838623in}{0.185281in}}%
\pgfpathmoveto{\pgfqpoint{1.892500in}{0.247442in}}%
\pgfpathlineto{\pgfqpoint{1.860217in}{0.288302in}}%
\pgfpathmoveto{\pgfqpoint{1.892500in}{0.153105in}}%
\pgfpathlineto{\pgfqpoint{1.809918in}{0.090000in}}%
\pgfpathmoveto{\pgfqpoint{1.630108in}{0.193370in}}%
\pgfpathlineto{\pgfqpoint{1.706938in}{0.290931in}}%
\pgfpathmoveto{\pgfqpoint{1.630108in}{0.193370in}}%
\pgfpathlineto{\pgfqpoint{1.557421in}{0.306015in}}%
\pgfpathmoveto{\pgfqpoint{1.630108in}{0.193370in}}%
\pgfpathlineto{\pgfqpoint{1.760808in}{0.090000in}}%
\pgfpathmoveto{\pgfqpoint{1.630108in}{0.193370in}}%
\pgfpathlineto{\pgfqpoint{1.477516in}{0.127940in}}%
\pgfpathmoveto{\pgfqpoint{1.328966in}{0.220820in}}%
\pgfpathlineto{\pgfqpoint{1.411859in}{0.326442in}}%
\pgfpathmoveto{\pgfqpoint{1.328966in}{0.220820in}}%
\pgfpathlineto{\pgfqpoint{1.276422in}{0.348083in}}%
\pgfpathmoveto{\pgfqpoint{1.328966in}{0.220820in}}%
\pgfpathlineto{\pgfqpoint{1.167724in}{0.113743in}}%
\pgfpathmoveto{\pgfqpoint{1.328966in}{0.220820in}}%
\pgfpathlineto{\pgfqpoint{1.477516in}{0.127940in}}%
\pgfpathmoveto{\pgfqpoint{1.029755in}{0.265660in}}%
\pgfpathlineto{\pgfqpoint{1.142300in}{0.376355in}}%
\pgfpathmoveto{\pgfqpoint{1.029755in}{0.265660in}}%
\pgfpathlineto{\pgfqpoint{0.985482in}{0.407171in}}%
\pgfpathmoveto{\pgfqpoint{1.029755in}{0.265660in}}%
\pgfpathlineto{\pgfqpoint{1.167724in}{0.113743in}}%
\pgfpathmoveto{\pgfqpoint{1.029755in}{0.265660in}}%
\pgfpathlineto{\pgfqpoint{0.838623in}{0.185281in}}%
\pgfpathmoveto{\pgfqpoint{0.648485in}{0.288454in}}%
\pgfpathlineto{\pgfqpoint{0.825246in}{0.423670in}}%
\pgfpathmoveto{\pgfqpoint{0.648485in}{0.288454in}}%
\pgfpathlineto{\pgfqpoint{0.459457in}{0.442094in}}%
\pgfpathmoveto{\pgfqpoint{0.648485in}{0.288454in}}%
\pgfpathlineto{\pgfqpoint{0.838623in}{0.185281in}}%
\pgfpathmoveto{\pgfqpoint{0.648485in}{0.288454in}}%
\pgfpathlineto{\pgfqpoint{0.467549in}{0.130122in}}%
\pgfpathmoveto{\pgfqpoint{1.784144in}{0.205567in}}%
\pgfpathlineto{\pgfqpoint{1.860217in}{0.288302in}}%
\pgfpathmoveto{\pgfqpoint{1.784144in}{0.205567in}}%
\pgfpathlineto{\pgfqpoint{1.706938in}{0.290931in}}%
\pgfpathmoveto{\pgfqpoint{1.784144in}{0.205567in}}%
\pgfpathlineto{\pgfqpoint{1.784828in}{0.090000in}}%
\pgfpathmoveto{\pgfqpoint{1.784144in}{0.205567in}}%
\pgfpathlineto{\pgfqpoint{1.892500in}{0.193707in}}%
\pgfpathmoveto{\pgfqpoint{1.784144in}{0.205567in}}%
\pgfpathlineto{\pgfqpoint{1.630108in}{0.193370in}}%
\pgfpathmoveto{\pgfqpoint{1.479677in}{0.232082in}}%
\pgfpathlineto{\pgfqpoint{1.557421in}{0.306015in}}%
\pgfpathmoveto{\pgfqpoint{1.479677in}{0.232082in}}%
\pgfpathlineto{\pgfqpoint{1.411859in}{0.326442in}}%
\pgfpathmoveto{\pgfqpoint{1.479677in}{0.232082in}}%
\pgfpathlineto{\pgfqpoint{1.477516in}{0.127940in}}%
\pgfpathmoveto{\pgfqpoint{1.479677in}{0.232082in}}%
\pgfpathlineto{\pgfqpoint{1.630108in}{0.193370in}}%
\pgfpathmoveto{\pgfqpoint{1.479677in}{0.232082in}}%
\pgfpathlineto{\pgfqpoint{1.328966in}{0.220820in}}%
\pgfpathmoveto{\pgfqpoint{1.171375in}{0.090000in}}%
\pgfpathlineto{\pgfqpoint{1.167724in}{0.113743in}}%
\pgfpathmoveto{\pgfqpoint{1.184619in}{0.263168in}}%
\pgfpathlineto{\pgfqpoint{1.276422in}{0.348083in}}%
\pgfpathmoveto{\pgfqpoint{1.184619in}{0.263168in}}%
\pgfpathlineto{\pgfqpoint{1.142300in}{0.376355in}}%
\pgfpathmoveto{\pgfqpoint{1.184619in}{0.263168in}}%
\pgfpathlineto{\pgfqpoint{1.167724in}{0.113743in}}%
\pgfpathmoveto{\pgfqpoint{1.184619in}{0.263168in}}%
\pgfpathlineto{\pgfqpoint{1.328966in}{0.220820in}}%
\pgfpathmoveto{\pgfqpoint{1.184619in}{0.263168in}}%
\pgfpathlineto{\pgfqpoint{1.029755in}{0.265660in}}%
\pgfpathmoveto{\pgfqpoint{0.886403in}{0.315109in}}%
\pgfpathlineto{\pgfqpoint{0.825246in}{0.423670in}}%
\pgfpathmoveto{\pgfqpoint{0.886403in}{0.315109in}}%
\pgfpathlineto{\pgfqpoint{0.985482in}{0.407171in}}%
\pgfpathmoveto{\pgfqpoint{0.886403in}{0.315109in}}%
\pgfpathlineto{\pgfqpoint{0.838623in}{0.185281in}}%
\pgfpathmoveto{\pgfqpoint{0.886403in}{0.315109in}}%
\pgfpathlineto{\pgfqpoint{1.029755in}{0.265660in}}%
\pgfpathmoveto{\pgfqpoint{0.886403in}{0.315109in}}%
\pgfpathlineto{\pgfqpoint{0.648485in}{0.288454in}}%
\pgfpathmoveto{\pgfqpoint{0.474489in}{0.090000in}}%
\pgfpathlineto{\pgfqpoint{0.467549in}{0.130122in}}%
\pgfpathlineto{\pgfqpoint{0.467549in}{0.130122in}}%
\pgfusepath{stroke}%
\end{pgfscope}%
\begin{pgfscope}%
\pgfpathrectangle{\pgfqpoint{0.100000in}{0.100000in}}{\pgfqpoint{1.782500in}{1.232000in}}%
\pgfusepath{clip}%
\pgfsetrectcap%
\pgfsetroundjoin%
\pgfsetlinewidth{0.250937pt}%
\definecolor{currentstroke}{rgb}{0.835294,0.321569,0.035294}%
\pgfsetstrokecolor{currentstroke}%
\pgfsetdash{}{0pt}%
\pgfpathmoveto{\pgfqpoint{0.761166in}{0.643692in}}%
\pgfpathlineto{\pgfqpoint{0.694167in}{0.980000in}}%
\pgfpathmoveto{\pgfqpoint{1.288333in}{0.980000in}}%
\pgfpathlineto{\pgfqpoint{1.882500in}{0.980000in}}%
\pgfpathmoveto{\pgfqpoint{1.288333in}{0.980000in}}%
\pgfpathlineto{\pgfqpoint{0.694167in}{0.980000in}}%
\pgfpathmoveto{\pgfqpoint{0.850285in}{0.538110in}}%
\pgfpathlineto{\pgfqpoint{0.761166in}{0.643692in}}%
\pgfpathmoveto{\pgfqpoint{0.954799in}{0.455711in}}%
\pgfpathlineto{\pgfqpoint{0.850285in}{0.538110in}}%
\pgfpathmoveto{\pgfqpoint{1.115716in}{0.385883in}}%
\pgfpathlineto{\pgfqpoint{0.954799in}{0.455711in}}%
\pgfpathmoveto{\pgfqpoint{1.322310in}{0.340385in}}%
\pgfpathlineto{\pgfqpoint{1.115716in}{0.385883in}}%
\pgfpathmoveto{\pgfqpoint{1.512012in}{0.312005in}}%
\pgfpathlineto{\pgfqpoint{1.322310in}{0.340385in}}%
\pgfpathmoveto{\pgfqpoint{1.687709in}{0.291915in}}%
\pgfpathlineto{\pgfqpoint{1.512012in}{0.312005in}}%
\pgfpathmoveto{\pgfqpoint{1.860307in}{0.287952in}}%
\pgfpathlineto{\pgfqpoint{1.687709in}{0.291915in}}%
\pgfpathmoveto{\pgfqpoint{1.892500in}{0.290088in}}%
\pgfpathlineto{\pgfqpoint{1.860307in}{0.287952in}}%
\pgfpathmoveto{\pgfqpoint{1.892500in}{0.980000in}}%
\pgfpathlineto{\pgfqpoint{1.882500in}{0.980000in}}%
\pgfpathmoveto{\pgfqpoint{1.517841in}{0.697106in}}%
\pgfpathlineto{\pgfqpoint{1.882500in}{0.980000in}}%
\pgfpathmoveto{\pgfqpoint{1.517841in}{0.697106in}}%
\pgfpathlineto{\pgfqpoint{1.288333in}{0.980000in}}%
\pgfpathmoveto{\pgfqpoint{1.797436in}{0.509742in}}%
\pgfpathlineto{\pgfqpoint{1.892500in}{0.535864in}}%
\pgfpathmoveto{\pgfqpoint{1.797436in}{0.509742in}}%
\pgfpathlineto{\pgfqpoint{1.517841in}{0.697106in}}%
\pgfpathmoveto{\pgfqpoint{1.151948in}{0.712427in}}%
\pgfpathlineto{\pgfqpoint{1.288333in}{0.980000in}}%
\pgfpathmoveto{\pgfqpoint{1.151948in}{0.712427in}}%
\pgfpathlineto{\pgfqpoint{1.517841in}{0.697106in}}%
\pgfpathmoveto{\pgfqpoint{1.892500in}{0.493052in}}%
\pgfpathlineto{\pgfqpoint{1.797436in}{0.509742in}}%
\pgfpathmoveto{\pgfqpoint{1.477534in}{0.473760in}}%
\pgfpathlineto{\pgfqpoint{1.322310in}{0.340385in}}%
\pgfpathmoveto{\pgfqpoint{1.477534in}{0.473760in}}%
\pgfpathlineto{\pgfqpoint{1.512012in}{0.312005in}}%
\pgfpathmoveto{\pgfqpoint{1.477534in}{0.473760in}}%
\pgfpathlineto{\pgfqpoint{1.517841in}{0.697106in}}%
\pgfpathmoveto{\pgfqpoint{1.477534in}{0.473760in}}%
\pgfpathlineto{\pgfqpoint{1.797436in}{0.509742in}}%
\pgfpathmoveto{\pgfqpoint{1.231443in}{0.549194in}}%
\pgfpathlineto{\pgfqpoint{0.954799in}{0.455711in}}%
\pgfpathmoveto{\pgfqpoint{1.231443in}{0.549194in}}%
\pgfpathlineto{\pgfqpoint{1.115716in}{0.385883in}}%
\pgfpathmoveto{\pgfqpoint{1.231443in}{0.549194in}}%
\pgfpathlineto{\pgfqpoint{1.517841in}{0.697106in}}%
\pgfpathmoveto{\pgfqpoint{1.231443in}{0.549194in}}%
\pgfpathlineto{\pgfqpoint{1.151948in}{0.712427in}}%
\pgfpathmoveto{\pgfqpoint{1.231443in}{0.549194in}}%
\pgfpathlineto{\pgfqpoint{1.477534in}{0.473760in}}%
\pgfpathmoveto{\pgfqpoint{1.780159in}{0.371136in}}%
\pgfpathlineto{\pgfqpoint{1.687709in}{0.291915in}}%
\pgfpathmoveto{\pgfqpoint{1.780159in}{0.371136in}}%
\pgfpathlineto{\pgfqpoint{1.860307in}{0.287952in}}%
\pgfpathmoveto{\pgfqpoint{1.780159in}{0.371136in}}%
\pgfpathlineto{\pgfqpoint{1.797436in}{0.509742in}}%
\pgfpathmoveto{\pgfqpoint{1.292218in}{0.445303in}}%
\pgfpathlineto{\pgfqpoint{1.115716in}{0.385883in}}%
\pgfpathmoveto{\pgfqpoint{1.292218in}{0.445303in}}%
\pgfpathlineto{\pgfqpoint{1.322310in}{0.340385in}}%
\pgfpathmoveto{\pgfqpoint{1.292218in}{0.445303in}}%
\pgfpathlineto{\pgfqpoint{1.477534in}{0.473760in}}%
\pgfpathmoveto{\pgfqpoint{1.292218in}{0.445303in}}%
\pgfpathlineto{\pgfqpoint{1.231443in}{0.549194in}}%
\pgfpathmoveto{\pgfqpoint{1.043604in}{0.568774in}}%
\pgfpathlineto{\pgfqpoint{0.850285in}{0.538110in}}%
\pgfpathmoveto{\pgfqpoint{1.043604in}{0.568774in}}%
\pgfpathlineto{\pgfqpoint{0.954799in}{0.455711in}}%
\pgfpathmoveto{\pgfqpoint{1.043604in}{0.568774in}}%
\pgfpathlineto{\pgfqpoint{1.151948in}{0.712427in}}%
\pgfpathmoveto{\pgfqpoint{1.043604in}{0.568774in}}%
\pgfpathlineto{\pgfqpoint{1.231443in}{0.549194in}}%
\pgfpathmoveto{\pgfqpoint{0.935177in}{0.813843in}}%
\pgfpathlineto{\pgfqpoint{0.694167in}{0.980000in}}%
\pgfpathmoveto{\pgfqpoint{0.935177in}{0.813843in}}%
\pgfpathlineto{\pgfqpoint{0.761166in}{0.643692in}}%
\pgfpathmoveto{\pgfqpoint{0.935177in}{0.813843in}}%
\pgfpathlineto{\pgfqpoint{1.288333in}{0.980000in}}%
\pgfpathmoveto{\pgfqpoint{0.935177in}{0.813843in}}%
\pgfpathlineto{\pgfqpoint{1.151948in}{0.712427in}}%
\pgfpathmoveto{\pgfqpoint{1.892500in}{0.323046in}}%
\pgfpathlineto{\pgfqpoint{1.860307in}{0.287952in}}%
\pgfpathmoveto{\pgfqpoint{1.892500in}{0.421679in}}%
\pgfpathlineto{\pgfqpoint{1.797436in}{0.509742in}}%
\pgfpathmoveto{\pgfqpoint{1.892500in}{0.374781in}}%
\pgfpathlineto{\pgfqpoint{1.780159in}{0.371136in}}%
\pgfpathmoveto{\pgfqpoint{1.619213in}{0.384678in}}%
\pgfpathlineto{\pgfqpoint{1.512012in}{0.312005in}}%
\pgfpathmoveto{\pgfqpoint{1.619213in}{0.384678in}}%
\pgfpathlineto{\pgfqpoint{1.687709in}{0.291915in}}%
\pgfpathmoveto{\pgfqpoint{1.619213in}{0.384678in}}%
\pgfpathlineto{\pgfqpoint{1.797436in}{0.509742in}}%
\pgfpathmoveto{\pgfqpoint{1.619213in}{0.384678in}}%
\pgfpathlineto{\pgfqpoint{1.477534in}{0.473760in}}%
\pgfpathmoveto{\pgfqpoint{1.619213in}{0.384678in}}%
\pgfpathlineto{\pgfqpoint{1.780159in}{0.371136in}}%
\pgfpathmoveto{\pgfqpoint{1.892500in}{0.925220in}}%
\pgfpathlineto{\pgfqpoint{1.882500in}{0.980000in}}%
\pgfpathmoveto{\pgfqpoint{1.892500in}{0.720948in}}%
\pgfpathlineto{\pgfqpoint{1.517841in}{0.697106in}}%
\pgfpathmoveto{\pgfqpoint{1.892500in}{0.663625in}}%
\pgfpathlineto{\pgfqpoint{1.797436in}{0.509742in}}%
\pgfpathmoveto{\pgfqpoint{1.892500in}{0.975290in}}%
\pgfpathlineto{\pgfqpoint{1.882500in}{0.980000in}}%
\pgfpathmoveto{\pgfqpoint{0.946644in}{0.655315in}}%
\pgfpathlineto{\pgfqpoint{0.761166in}{0.643692in}}%
\pgfpathmoveto{\pgfqpoint{0.946644in}{0.655315in}}%
\pgfpathlineto{\pgfqpoint{0.850285in}{0.538110in}}%
\pgfpathmoveto{\pgfqpoint{0.946644in}{0.655315in}}%
\pgfpathlineto{\pgfqpoint{1.151948in}{0.712427in}}%
\pgfpathmoveto{\pgfqpoint{0.946644in}{0.655315in}}%
\pgfpathlineto{\pgfqpoint{1.043604in}{0.568774in}}%
\pgfpathmoveto{\pgfqpoint{0.946644in}{0.655315in}}%
\pgfpathlineto{\pgfqpoint{0.935177in}{0.813843in}}%
\pgfpathlineto{\pgfqpoint{0.935177in}{0.813843in}}%
\pgfusepath{stroke}%
\end{pgfscope}%
\begin{pgfscope}%
\pgfpathrectangle{\pgfqpoint{0.100000in}{0.100000in}}{\pgfqpoint{1.782500in}{1.232000in}}%
\pgfusepath{clip}%
\pgfsetbuttcap%
\pgfsetroundjoin%
\definecolor{currentfill}{rgb}{0.054902,0.262745,0.486275}%
\pgfsetfillcolor{currentfill}%
\pgfsetlinewidth{1.003750pt}%
\definecolor{currentstroke}{rgb}{0.054902,0.262745,0.486275}%
\pgfsetstrokecolor{currentstroke}%
\pgfsetdash{}{0pt}%
\pgfsys@defobject{currentmarker}{\pgfqpoint{-0.018373in}{-0.018373in}}{\pgfqpoint{0.018373in}{0.018373in}}{%
\pgfpathmoveto{\pgfqpoint{0.000000in}{-0.018373in}}%
\pgfpathcurveto{\pgfqpoint{0.004873in}{-0.018373in}}{\pgfqpoint{0.009546in}{-0.016437in}}{\pgfqpoint{0.012992in}{-0.012992in}}%
\pgfpathcurveto{\pgfqpoint{0.016437in}{-0.009546in}}{\pgfqpoint{0.018373in}{-0.004873in}}{\pgfqpoint{0.018373in}{0.000000in}}%
\pgfpathcurveto{\pgfqpoint{0.018373in}{0.004873in}}{\pgfqpoint{0.016437in}{0.009546in}}{\pgfqpoint{0.012992in}{0.012992in}}%
\pgfpathcurveto{\pgfqpoint{0.009546in}{0.016437in}}{\pgfqpoint{0.004873in}{0.018373in}}{\pgfqpoint{0.000000in}{0.018373in}}%
\pgfpathcurveto{\pgfqpoint{-0.004873in}{0.018373in}}{\pgfqpoint{-0.009546in}{0.016437in}}{\pgfqpoint{-0.012992in}{0.012992in}}%
\pgfpathcurveto{\pgfqpoint{-0.016437in}{0.009546in}}{\pgfqpoint{-0.018373in}{0.004873in}}{\pgfqpoint{-0.018373in}{0.000000in}}%
\pgfpathcurveto{\pgfqpoint{-0.018373in}{-0.004873in}}{\pgfqpoint{-0.016437in}{-0.009546in}}{\pgfqpoint{-0.012992in}{-0.012992in}}%
\pgfpathcurveto{\pgfqpoint{-0.009546in}{-0.016437in}}{\pgfqpoint{-0.004873in}{-0.018373in}}{\pgfqpoint{0.000000in}{-0.018373in}}%
\pgfpathlineto{\pgfqpoint{0.000000in}{-0.018373in}}%
\pgfpathclose%
\pgfusepath{stroke,fill}%
}%
\begin{pgfscope}%
\pgfsys@transformshift{2.556907in}{0.369290in}%
\pgfsys@useobject{currentmarker}{}%
\end{pgfscope}%
\begin{pgfscope}%
\pgfsys@transformshift{2.434128in}{0.343521in}%
\pgfsys@useobject{currentmarker}{}%
\end{pgfscope}%
\begin{pgfscope}%
\pgfsys@transformshift{2.306807in}{0.324690in}%
\pgfsys@useobject{currentmarker}{}%
\end{pgfscope}%
\begin{pgfscope}%
\pgfsys@transformshift{2.161723in}{0.309610in}%
\pgfsys@useobject{currentmarker}{}%
\end{pgfscope}%
\begin{pgfscope}%
\pgfsys@transformshift{2.011620in}{0.297739in}%
\pgfsys@useobject{currentmarker}{}%
\end{pgfscope}%
\begin{pgfscope}%
\pgfsys@transformshift{1.860217in}{0.288302in}%
\pgfsys@useobject{currentmarker}{}%
\end{pgfscope}%
\begin{pgfscope}%
\pgfsys@transformshift{1.706938in}{0.290931in}%
\pgfsys@useobject{currentmarker}{}%
\end{pgfscope}%
\begin{pgfscope}%
\pgfsys@transformshift{1.557421in}{0.306015in}%
\pgfsys@useobject{currentmarker}{}%
\end{pgfscope}%
\begin{pgfscope}%
\pgfsys@transformshift{1.411859in}{0.326442in}%
\pgfsys@useobject{currentmarker}{}%
\end{pgfscope}%
\begin{pgfscope}%
\pgfsys@transformshift{1.276422in}{0.348083in}%
\pgfsys@useobject{currentmarker}{}%
\end{pgfscope}%
\begin{pgfscope}%
\pgfsys@transformshift{1.142300in}{0.376355in}%
\pgfsys@useobject{currentmarker}{}%
\end{pgfscope}%
\end{pgfscope}%
\begin{pgfscope}%
\pgfpathrectangle{\pgfqpoint{0.100000in}{0.100000in}}{\pgfqpoint{1.782500in}{1.232000in}}%
\pgfusepath{clip}%
\pgfsetbuttcap%
\pgfsetroundjoin%
\definecolor{currentfill}{rgb}{0.835294,0.321569,0.035294}%
\pgfsetfillcolor{currentfill}%
\pgfsetlinewidth{1.003750pt}%
\definecolor{currentstroke}{rgb}{0.835294,0.321569,0.035294}%
\pgfsetstrokecolor{currentstroke}%
\pgfsetdash{}{0pt}%
\pgfsys@defobject{currentmarker}{\pgfqpoint{-0.018373in}{-0.018373in}}{\pgfqpoint{0.018373in}{0.018373in}}{%
\pgfpathmoveto{\pgfqpoint{0.000000in}{-0.018373in}}%
\pgfpathcurveto{\pgfqpoint{0.004873in}{-0.018373in}}{\pgfqpoint{0.009546in}{-0.016437in}}{\pgfqpoint{0.012992in}{-0.012992in}}%
\pgfpathcurveto{\pgfqpoint{0.016437in}{-0.009546in}}{\pgfqpoint{0.018373in}{-0.004873in}}{\pgfqpoint{0.018373in}{0.000000in}}%
\pgfpathcurveto{\pgfqpoint{0.018373in}{0.004873in}}{\pgfqpoint{0.016437in}{0.009546in}}{\pgfqpoint{0.012992in}{0.012992in}}%
\pgfpathcurveto{\pgfqpoint{0.009546in}{0.016437in}}{\pgfqpoint{0.004873in}{0.018373in}}{\pgfqpoint{0.000000in}{0.018373in}}%
\pgfpathcurveto{\pgfqpoint{-0.004873in}{0.018373in}}{\pgfqpoint{-0.009546in}{0.016437in}}{\pgfqpoint{-0.012992in}{0.012992in}}%
\pgfpathcurveto{\pgfqpoint{-0.016437in}{0.009546in}}{\pgfqpoint{-0.018373in}{0.004873in}}{\pgfqpoint{-0.018373in}{0.000000in}}%
\pgfpathcurveto{\pgfqpoint{-0.018373in}{-0.004873in}}{\pgfqpoint{-0.016437in}{-0.009546in}}{\pgfqpoint{-0.012992in}{-0.012992in}}%
\pgfpathcurveto{\pgfqpoint{-0.009546in}{-0.016437in}}{\pgfqpoint{-0.004873in}{-0.018373in}}{\pgfqpoint{0.000000in}{-0.018373in}}%
\pgfpathlineto{\pgfqpoint{0.000000in}{-0.018373in}}%
\pgfpathclose%
\pgfusepath{stroke,fill}%
}%
\begin{pgfscope}%
\pgfsys@transformshift{1.115716in}{0.385883in}%
\pgfsys@useobject{currentmarker}{}%
\end{pgfscope}%
\begin{pgfscope}%
\pgfsys@transformshift{1.322310in}{0.340385in}%
\pgfsys@useobject{currentmarker}{}%
\end{pgfscope}%
\begin{pgfscope}%
\pgfsys@transformshift{1.512012in}{0.312005in}%
\pgfsys@useobject{currentmarker}{}%
\end{pgfscope}%
\begin{pgfscope}%
\pgfsys@transformshift{1.687709in}{0.291915in}%
\pgfsys@useobject{currentmarker}{}%
\end{pgfscope}%
\begin{pgfscope}%
\pgfsys@transformshift{1.860307in}{0.287952in}%
\pgfsys@useobject{currentmarker}{}%
\end{pgfscope}%
\begin{pgfscope}%
\pgfsys@transformshift{2.030568in}{0.299250in}%
\pgfsys@useobject{currentmarker}{}%
\end{pgfscope}%
\begin{pgfscope}%
\pgfsys@transformshift{2.207472in}{0.313671in}%
\pgfsys@useobject{currentmarker}{}%
\end{pgfscope}%
\begin{pgfscope}%
\pgfsys@transformshift{2.392452in}{0.336464in}%
\pgfsys@useobject{currentmarker}{}%
\end{pgfscope}%
\begin{pgfscope}%
\pgfsys@transformshift{2.586212in}{0.376922in}%
\pgfsys@useobject{currentmarker}{}%
\end{pgfscope}%
\end{pgfscope}%
\begin{pgfscope}%
\pgfpathrectangle{\pgfqpoint{0.100000in}{0.100000in}}{\pgfqpoint{1.782500in}{1.232000in}}%
\pgfusepath{clip}%
\pgfsetbuttcap%
\pgfsetroundjoin%
\pgfsetlinewidth{1.003750pt}%
\definecolor{currentstroke}{rgb}{0.054902,0.262745,0.486275}%
\pgfsetstrokecolor{currentstroke}%
\pgfsetdash{}{0pt}%
\pgfpathmoveto{\pgfqpoint{0.000000in}{-0.018373in}}%
\pgfpathcurveto{\pgfqpoint{0.004873in}{-0.018373in}}{\pgfqpoint{0.009546in}{-0.016437in}}{\pgfqpoint{0.012992in}{-0.012992in}}%
\pgfpathcurveto{\pgfqpoint{0.016437in}{-0.009546in}}{\pgfqpoint{0.018373in}{-0.004873in}}{\pgfqpoint{0.018373in}{0.000000in}}%
\pgfpathcurveto{\pgfqpoint{0.018373in}{0.004873in}}{\pgfqpoint{0.016437in}{0.009546in}}{\pgfqpoint{0.012992in}{0.012992in}}%
\pgfpathcurveto{\pgfqpoint{0.009546in}{0.016437in}}{\pgfqpoint{0.004873in}{0.018373in}}{\pgfqpoint{0.000000in}{0.018373in}}%
\pgfpathcurveto{\pgfqpoint{-0.004873in}{0.018373in}}{\pgfqpoint{-0.009546in}{0.016437in}}{\pgfqpoint{-0.012992in}{0.012992in}}%
\pgfpathcurveto{\pgfqpoint{-0.016437in}{0.009546in}}{\pgfqpoint{-0.018373in}{0.004873in}}{\pgfqpoint{-0.018373in}{0.000000in}}%
\pgfpathcurveto{\pgfqpoint{-0.018373in}{-0.004873in}}{\pgfqpoint{-0.016437in}{-0.009546in}}{\pgfqpoint{-0.012992in}{-0.012992in}}%
\pgfpathcurveto{\pgfqpoint{-0.009546in}{-0.016437in}}{\pgfqpoint{-0.004873in}{-0.018373in}}{\pgfqpoint{0.000000in}{-0.018373in}}%
\pgfusepath{stroke}%
\end{pgfscope}%
\begin{pgfscope}%
\pgfpathrectangle{\pgfqpoint{0.100000in}{0.100000in}}{\pgfqpoint{1.782500in}{1.232000in}}%
\pgfusepath{clip}%
\pgfsetbuttcap%
\pgfsetroundjoin%
\pgfsetlinewidth{1.003750pt}%
\definecolor{currentstroke}{rgb}{0.835294,0.321569,0.035294}%
\pgfsetstrokecolor{currentstroke}%
\pgfsetdash{}{0pt}%
\pgfpathmoveto{\pgfqpoint{0.000000in}{-0.018373in}}%
\pgfpathcurveto{\pgfqpoint{0.004873in}{-0.018373in}}{\pgfqpoint{0.009546in}{-0.016437in}}{\pgfqpoint{0.012992in}{-0.012992in}}%
\pgfpathcurveto{\pgfqpoint{0.016437in}{-0.009546in}}{\pgfqpoint{0.018373in}{-0.004873in}}{\pgfqpoint{0.018373in}{0.000000in}}%
\pgfpathcurveto{\pgfqpoint{0.018373in}{0.004873in}}{\pgfqpoint{0.016437in}{0.009546in}}{\pgfqpoint{0.012992in}{0.012992in}}%
\pgfpathcurveto{\pgfqpoint{0.009546in}{0.016437in}}{\pgfqpoint{0.004873in}{0.018373in}}{\pgfqpoint{0.000000in}{0.018373in}}%
\pgfpathcurveto{\pgfqpoint{-0.004873in}{0.018373in}}{\pgfqpoint{-0.009546in}{0.016437in}}{\pgfqpoint{-0.012992in}{0.012992in}}%
\pgfpathcurveto{\pgfqpoint{-0.016437in}{0.009546in}}{\pgfqpoint{-0.018373in}{0.004873in}}{\pgfqpoint{-0.018373in}{0.000000in}}%
\pgfpathcurveto{\pgfqpoint{-0.018373in}{-0.004873in}}{\pgfqpoint{-0.016437in}{-0.009546in}}{\pgfqpoint{-0.012992in}{-0.012992in}}%
\pgfpathcurveto{\pgfqpoint{-0.009546in}{-0.016437in}}{\pgfqpoint{-0.004873in}{-0.018373in}}{\pgfqpoint{0.000000in}{-0.018373in}}%
\pgfusepath{stroke}%
\end{pgfscope}%
\begin{pgfscope}%
\pgfpathrectangle{\pgfqpoint{0.100000in}{0.100000in}}{\pgfqpoint{1.782500in}{1.232000in}}%
\pgfusepath{clip}%
\pgfsetbuttcap%
\pgfsetroundjoin%
\definecolor{currentfill}{rgb}{0.054902,0.262745,0.486275}%
\pgfsetfillcolor{currentfill}%
\pgfsetlinewidth{1.505625pt}%
\definecolor{currentstroke}{rgb}{0.054902,0.262745,0.486275}%
\pgfsetstrokecolor{currentstroke}%
\pgfsetdash{}{0pt}%
\pgfsys@defobject{currentmarker}{\pgfqpoint{-0.018373in}{-0.018373in}}{\pgfqpoint{0.018373in}{0.018373in}}{%
\pgfpathmoveto{\pgfqpoint{-0.018373in}{-0.018373in}}%
\pgfpathlineto{\pgfqpoint{0.018373in}{0.018373in}}%
\pgfpathmoveto{\pgfqpoint{-0.018373in}{0.018373in}}%
\pgfpathlineto{\pgfqpoint{0.018373in}{-0.018373in}}%
\pgfusepath{stroke,fill}%
}%
\begin{pgfscope}%
\pgfsys@transformshift{2.684788in}{0.403274in}%
\pgfsys@useobject{currentmarker}{}%
\end{pgfscope}%
\end{pgfscope}%
\begin{pgfscope}%
\pgfpathrectangle{\pgfqpoint{0.100000in}{0.100000in}}{\pgfqpoint{1.782500in}{1.232000in}}%
\pgfusepath{clip}%
\pgfsetbuttcap%
\pgfsetroundjoin%
\definecolor{currentfill}{rgb}{0.835294,0.321569,0.035294}%
\pgfsetfillcolor{currentfill}%
\pgfsetlinewidth{1.505625pt}%
\definecolor{currentstroke}{rgb}{0.835294,0.321569,0.035294}%
\pgfsetstrokecolor{currentstroke}%
\pgfsetdash{}{0pt}%
\pgfsys@defobject{currentmarker}{\pgfqpoint{-0.018373in}{-0.018373in}}{\pgfqpoint{0.018373in}{0.018373in}}{%
\pgfpathmoveto{\pgfqpoint{-0.018373in}{-0.018373in}}%
\pgfpathlineto{\pgfqpoint{0.018373in}{0.018373in}}%
\pgfpathmoveto{\pgfqpoint{-0.018373in}{0.018373in}}%
\pgfpathlineto{\pgfqpoint{0.018373in}{-0.018373in}}%
\pgfusepath{stroke,fill}%
}%
\begin{pgfscope}%
\pgfsys@transformshift{0.954799in}{0.455711in}%
\pgfsys@useobject{currentmarker}{}%
\end{pgfscope}%
\begin{pgfscope}%
\pgfsys@transformshift{2.779965in}{0.430714in}%
\pgfsys@useobject{currentmarker}{}%
\end{pgfscope}%
\end{pgfscope}%
\end{pgfpicture}%
\makeatother%
\endgroup%
}
        \caption{Iteration 2: Update interface}\label{fig:example-iter1-dumping}
    \end{subfigure}
    \begin{subfigure}[b]{.32\linewidth}
        \scalebox{0.8}{%% Creator: Matplotlib, PGF backend
%%
%% To include the figure in your LaTeX document, write
%%   \input{<filename>.pgf}
%%
%% Make sure the required packages are loaded in your preamble
%%   \usepackage{pgf}
%%
%% Also ensure that all the required font packages are loaded; for instance,
%% the lmodern package is sometimes necessary when using math font.
%%   \usepackage{lmodern}
%%
%% Figures using additional raster images can only be included by \input if
%% they are in the same directory as the main LaTeX file. For loading figures
%% from other directories you can use the `import` package
%%   \usepackage{import}
%%
%% and then include the figures with
%%   \import{<path to file>}{<filename>.pgf}
%%
%% Matplotlib used the following preamble
%%   
%%   \usepackage{fontspec}
%%   \setmainfont{DejaVuSans.ttf}[Path=\detokenize{/home/fabio/Internodes-CM/.venv/lib/python3.8/site-packages/matplotlib/mpl-data/fonts/ttf/}]
%%   \setsansfont{DejaVuSans.ttf}[Path=\detokenize{/home/fabio/Internodes-CM/.venv/lib/python3.8/site-packages/matplotlib/mpl-data/fonts/ttf/}]
%%   \setmonofont{DejaVuSansMono.ttf}[Path=\detokenize{/home/fabio/Internodes-CM/.venv/lib/python3.8/site-packages/matplotlib/mpl-data/fonts/ttf/}]
%%   \makeatletter\@ifpackageloaded{underscore}{}{\usepackage[strings]{underscore}}\makeatother
%%
\begingroup%
\makeatletter%
\begin{pgfpicture}%
\pgfpathrectangle{\pgfpointorigin}{\pgfqpoint{1.982500in}{1.432000in}}%
\pgfusepath{use as bounding box, clip}%
\begin{pgfscope}%
\pgfsetbuttcap%
\pgfsetmiterjoin%
\definecolor{currentfill}{rgb}{1.000000,1.000000,1.000000}%
\pgfsetfillcolor{currentfill}%
\pgfsetlinewidth{0.000000pt}%
\definecolor{currentstroke}{rgb}{1.000000,1.000000,1.000000}%
\pgfsetstrokecolor{currentstroke}%
\pgfsetdash{}{0pt}%
\pgfpathmoveto{\pgfqpoint{0.000000in}{0.000000in}}%
\pgfpathlineto{\pgfqpoint{1.982500in}{0.000000in}}%
\pgfpathlineto{\pgfqpoint{1.982500in}{1.432000in}}%
\pgfpathlineto{\pgfqpoint{0.000000in}{1.432000in}}%
\pgfpathlineto{\pgfqpoint{0.000000in}{0.000000in}}%
\pgfpathclose%
\pgfusepath{fill}%
\end{pgfscope}%
\begin{pgfscope}%
\pgfpathrectangle{\pgfqpoint{0.100000in}{0.100000in}}{\pgfqpoint{1.782500in}{1.232000in}}%
\pgfusepath{clip}%
\pgfsetrectcap%
\pgfsetroundjoin%
\pgfsetlinewidth{0.250937pt}%
\definecolor{currentstroke}{rgb}{0.054902,0.262745,0.486275}%
\pgfsetstrokecolor{currentstroke}%
\pgfsetdash{}{0pt}%
\pgfpathmoveto{\pgfqpoint{0.451098in}{0.100000in}}%
\pgfpathlineto{\pgfqpoint{0.181023in}{0.100000in}}%
\pgfpathmoveto{\pgfqpoint{0.721174in}{0.100000in}}%
\pgfpathlineto{\pgfqpoint{0.451098in}{0.100000in}}%
\pgfpathmoveto{\pgfqpoint{0.991250in}{0.100000in}}%
\pgfpathlineto{\pgfqpoint{0.721174in}{0.100000in}}%
\pgfpathmoveto{\pgfqpoint{1.261326in}{0.100000in}}%
\pgfpathlineto{\pgfqpoint{0.991250in}{0.100000in}}%
\pgfpathmoveto{\pgfqpoint{1.531402in}{0.100000in}}%
\pgfpathlineto{\pgfqpoint{1.801477in}{0.100000in}}%
\pgfpathmoveto{\pgfqpoint{1.531402in}{0.100000in}}%
\pgfpathlineto{\pgfqpoint{1.261326in}{0.100000in}}%
\pgfpathmoveto{\pgfqpoint{1.801477in}{0.408000in}}%
\pgfpathlineto{\pgfqpoint{1.801477in}{0.100000in}}%
\pgfpathmoveto{\pgfqpoint{1.801477in}{0.408000in}}%
\pgfpathlineto{\pgfqpoint{1.801477in}{0.716000in}}%
\pgfpathmoveto{\pgfqpoint{1.639432in}{0.716000in}}%
\pgfpathlineto{\pgfqpoint{1.801477in}{0.716000in}}%
\pgfpathmoveto{\pgfqpoint{1.639432in}{0.716000in}}%
\pgfpathlineto{\pgfqpoint{1.477386in}{0.716000in}}%
\pgfpathmoveto{\pgfqpoint{1.407938in}{0.716000in}}%
\pgfpathlineto{\pgfqpoint{1.477386in}{0.716000in}}%
\pgfpathmoveto{\pgfqpoint{1.338490in}{0.716000in}}%
\pgfpathlineto{\pgfqpoint{1.407938in}{0.716000in}}%
\pgfpathmoveto{\pgfqpoint{1.269042in}{0.716000in}}%
\pgfpathlineto{\pgfqpoint{1.338490in}{0.716000in}}%
\pgfpathmoveto{\pgfqpoint{1.199594in}{0.716000in}}%
\pgfpathlineto{\pgfqpoint{1.269042in}{0.716000in}}%
\pgfpathmoveto{\pgfqpoint{1.130146in}{0.716000in}}%
\pgfpathlineto{\pgfqpoint{1.199594in}{0.716000in}}%
\pgfpathmoveto{\pgfqpoint{1.060698in}{0.716000in}}%
\pgfpathlineto{\pgfqpoint{1.130146in}{0.716000in}}%
\pgfpathmoveto{\pgfqpoint{0.991250in}{0.716000in}}%
\pgfpathlineto{\pgfqpoint{1.060698in}{0.716000in}}%
\pgfpathmoveto{\pgfqpoint{0.921802in}{0.716000in}}%
\pgfpathlineto{\pgfqpoint{0.991250in}{0.716000in}}%
\pgfpathmoveto{\pgfqpoint{0.852354in}{0.716000in}}%
\pgfpathlineto{\pgfqpoint{0.921802in}{0.716000in}}%
\pgfpathmoveto{\pgfqpoint{0.782906in}{0.716000in}}%
\pgfpathlineto{\pgfqpoint{0.852354in}{0.716000in}}%
\pgfpathmoveto{\pgfqpoint{0.713458in}{0.716000in}}%
\pgfpathlineto{\pgfqpoint{0.782906in}{0.716000in}}%
\pgfpathmoveto{\pgfqpoint{0.644010in}{0.716000in}}%
\pgfpathlineto{\pgfqpoint{0.713458in}{0.716000in}}%
\pgfpathmoveto{\pgfqpoint{0.574562in}{0.716000in}}%
\pgfpathlineto{\pgfqpoint{0.505114in}{0.716000in}}%
\pgfpathmoveto{\pgfqpoint{0.574562in}{0.716000in}}%
\pgfpathlineto{\pgfqpoint{0.644010in}{0.716000in}}%
\pgfpathmoveto{\pgfqpoint{0.343068in}{0.716000in}}%
\pgfpathlineto{\pgfqpoint{0.505114in}{0.716000in}}%
\pgfpathmoveto{\pgfqpoint{0.343068in}{0.716000in}}%
\pgfpathlineto{\pgfqpoint{0.181023in}{0.716000in}}%
\pgfpathmoveto{\pgfqpoint{0.181023in}{0.408000in}}%
\pgfpathlineto{\pgfqpoint{0.181023in}{0.100000in}}%
\pgfpathmoveto{\pgfqpoint{0.181023in}{0.408000in}}%
\pgfpathlineto{\pgfqpoint{0.181023in}{0.716000in}}%
\pgfpathmoveto{\pgfqpoint{1.373214in}{0.396057in}}%
\pgfpathlineto{\pgfqpoint{1.261326in}{0.100000in}}%
\pgfpathmoveto{\pgfqpoint{1.082944in}{0.402689in}}%
\pgfpathlineto{\pgfqpoint{0.991250in}{0.100000in}}%
\pgfpathmoveto{\pgfqpoint{1.233147in}{0.520967in}}%
\pgfpathlineto{\pgfqpoint{1.373214in}{0.396057in}}%
\pgfpathmoveto{\pgfqpoint{1.233147in}{0.520967in}}%
\pgfpathlineto{\pgfqpoint{1.082944in}{0.402689in}}%
\pgfpathmoveto{\pgfqpoint{0.956117in}{0.522274in}}%
\pgfpathlineto{\pgfqpoint{1.082944in}{0.402689in}}%
\pgfpathmoveto{\pgfqpoint{0.956117in}{0.522274in}}%
\pgfpathlineto{\pgfqpoint{0.821141in}{0.416705in}}%
\pgfpathmoveto{\pgfqpoint{0.679173in}{0.523635in}}%
\pgfpathlineto{\pgfqpoint{0.821141in}{0.416705in}}%
\pgfpathmoveto{\pgfqpoint{0.679173in}{0.523635in}}%
\pgfpathlineto{\pgfqpoint{0.539838in}{0.401685in}}%
\pgfpathmoveto{\pgfqpoint{1.535558in}{0.497212in}}%
\pgfpathlineto{\pgfqpoint{1.801477in}{0.408000in}}%
\pgfpathmoveto{\pgfqpoint{1.535558in}{0.497212in}}%
\pgfpathlineto{\pgfqpoint{1.373214in}{0.396057in}}%
\pgfpathmoveto{\pgfqpoint{1.376927in}{0.550483in}}%
\pgfpathlineto{\pgfqpoint{1.373214in}{0.396057in}}%
\pgfpathmoveto{\pgfqpoint{1.376927in}{0.550483in}}%
\pgfpathlineto{\pgfqpoint{1.233147in}{0.520967in}}%
\pgfpathmoveto{\pgfqpoint{1.376927in}{0.550483in}}%
\pgfpathlineto{\pgfqpoint{1.535558in}{0.497212in}}%
\pgfpathmoveto{\pgfqpoint{1.093079in}{0.556190in}}%
\pgfpathlineto{\pgfqpoint{1.082944in}{0.402689in}}%
\pgfpathmoveto{\pgfqpoint{1.093079in}{0.556190in}}%
\pgfpathlineto{\pgfqpoint{1.233147in}{0.520967in}}%
\pgfpathmoveto{\pgfqpoint{1.093079in}{0.556190in}}%
\pgfpathlineto{\pgfqpoint{0.956117in}{0.522274in}}%
\pgfpathmoveto{\pgfqpoint{0.818220in}{0.558974in}}%
\pgfpathlineto{\pgfqpoint{0.821141in}{0.416705in}}%
\pgfpathmoveto{\pgfqpoint{0.818220in}{0.558974in}}%
\pgfpathlineto{\pgfqpoint{0.956117in}{0.522274in}}%
\pgfpathmoveto{\pgfqpoint{0.818220in}{0.558974in}}%
\pgfpathlineto{\pgfqpoint{0.679173in}{0.523635in}}%
\pgfpathmoveto{\pgfqpoint{0.525112in}{0.550213in}}%
\pgfpathlineto{\pgfqpoint{0.539838in}{0.401685in}}%
\pgfpathmoveto{\pgfqpoint{0.525112in}{0.550213in}}%
\pgfpathlineto{\pgfqpoint{0.679173in}{0.523635in}}%
\pgfpathmoveto{\pgfqpoint{0.356186in}{0.497584in}}%
\pgfpathlineto{\pgfqpoint{0.181023in}{0.716000in}}%
\pgfpathmoveto{\pgfqpoint{0.356186in}{0.497584in}}%
\pgfpathlineto{\pgfqpoint{0.343068in}{0.716000in}}%
\pgfpathmoveto{\pgfqpoint{0.356186in}{0.497584in}}%
\pgfpathlineto{\pgfqpoint{0.181023in}{0.408000in}}%
\pgfpathmoveto{\pgfqpoint{0.356186in}{0.497584in}}%
\pgfpathlineto{\pgfqpoint{0.539838in}{0.401685in}}%
\pgfpathmoveto{\pgfqpoint{0.356186in}{0.497584in}}%
\pgfpathlineto{\pgfqpoint{0.525112in}{0.550213in}}%
\pgfpathmoveto{\pgfqpoint{1.303766in}{0.622307in}}%
\pgfpathlineto{\pgfqpoint{1.338490in}{0.716000in}}%
\pgfpathmoveto{\pgfqpoint{1.303766in}{0.622307in}}%
\pgfpathlineto{\pgfqpoint{1.269042in}{0.716000in}}%
\pgfpathmoveto{\pgfqpoint{1.303766in}{0.622307in}}%
\pgfpathlineto{\pgfqpoint{1.233147in}{0.520967in}}%
\pgfpathmoveto{\pgfqpoint{1.303766in}{0.622307in}}%
\pgfpathlineto{\pgfqpoint{1.376927in}{0.550483in}}%
\pgfpathmoveto{\pgfqpoint{1.164870in}{0.623450in}}%
\pgfpathlineto{\pgfqpoint{1.199594in}{0.716000in}}%
\pgfpathmoveto{\pgfqpoint{1.164870in}{0.623450in}}%
\pgfpathlineto{\pgfqpoint{1.130146in}{0.716000in}}%
\pgfpathmoveto{\pgfqpoint{1.164870in}{0.623450in}}%
\pgfpathlineto{\pgfqpoint{1.233147in}{0.520967in}}%
\pgfpathmoveto{\pgfqpoint{1.164870in}{0.623450in}}%
\pgfpathlineto{\pgfqpoint{1.093079in}{0.556190in}}%
\pgfpathmoveto{\pgfqpoint{1.025974in}{0.623450in}}%
\pgfpathlineto{\pgfqpoint{1.060698in}{0.716000in}}%
\pgfpathmoveto{\pgfqpoint{1.025974in}{0.623450in}}%
\pgfpathlineto{\pgfqpoint{0.991250in}{0.716000in}}%
\pgfpathmoveto{\pgfqpoint{1.025974in}{0.623450in}}%
\pgfpathlineto{\pgfqpoint{0.956117in}{0.522274in}}%
\pgfpathmoveto{\pgfqpoint{1.025974in}{0.623450in}}%
\pgfpathlineto{\pgfqpoint{1.093079in}{0.556190in}}%
\pgfpathmoveto{\pgfqpoint{0.887078in}{0.623460in}}%
\pgfpathlineto{\pgfqpoint{0.921802in}{0.716000in}}%
\pgfpathmoveto{\pgfqpoint{0.887078in}{0.623460in}}%
\pgfpathlineto{\pgfqpoint{0.852354in}{0.716000in}}%
\pgfpathmoveto{\pgfqpoint{0.887078in}{0.623460in}}%
\pgfpathlineto{\pgfqpoint{0.956117in}{0.522274in}}%
\pgfpathmoveto{\pgfqpoint{0.887078in}{0.623460in}}%
\pgfpathlineto{\pgfqpoint{0.818220in}{0.558974in}}%
\pgfpathmoveto{\pgfqpoint{0.748182in}{0.623460in}}%
\pgfpathlineto{\pgfqpoint{0.782906in}{0.716000in}}%
\pgfpathmoveto{\pgfqpoint{0.748182in}{0.623460in}}%
\pgfpathlineto{\pgfqpoint{0.713458in}{0.716000in}}%
\pgfpathmoveto{\pgfqpoint{0.748182in}{0.623460in}}%
\pgfpathlineto{\pgfqpoint{0.679173in}{0.523635in}}%
\pgfpathmoveto{\pgfqpoint{0.748182in}{0.623460in}}%
\pgfpathlineto{\pgfqpoint{0.818220in}{0.558974in}}%
\pgfpathmoveto{\pgfqpoint{0.609286in}{0.623460in}}%
\pgfpathlineto{\pgfqpoint{0.644010in}{0.716000in}}%
\pgfpathmoveto{\pgfqpoint{0.609286in}{0.623460in}}%
\pgfpathlineto{\pgfqpoint{0.574562in}{0.716000in}}%
\pgfpathmoveto{\pgfqpoint{0.609286in}{0.623460in}}%
\pgfpathlineto{\pgfqpoint{0.679173in}{0.523635in}}%
\pgfpathmoveto{\pgfqpoint{0.609286in}{0.623460in}}%
\pgfpathlineto{\pgfqpoint{0.525112in}{0.550213in}}%
\pgfpathmoveto{\pgfqpoint{1.453902in}{0.624729in}}%
\pgfpathlineto{\pgfqpoint{1.477386in}{0.716000in}}%
\pgfpathmoveto{\pgfqpoint{1.453902in}{0.624729in}}%
\pgfpathlineto{\pgfqpoint{1.407938in}{0.716000in}}%
\pgfpathmoveto{\pgfqpoint{1.453902in}{0.624729in}}%
\pgfpathlineto{\pgfqpoint{1.535558in}{0.497212in}}%
\pgfpathmoveto{\pgfqpoint{1.453902in}{0.624729in}}%
\pgfpathlineto{\pgfqpoint{1.376927in}{0.550483in}}%
\pgfpathmoveto{\pgfqpoint{0.939717in}{0.332038in}}%
\pgfpathlineto{\pgfqpoint{0.991250in}{0.100000in}}%
\pgfpathmoveto{\pgfqpoint{0.939717in}{0.332038in}}%
\pgfpathlineto{\pgfqpoint{1.082944in}{0.402689in}}%
\pgfpathmoveto{\pgfqpoint{0.939717in}{0.332038in}}%
\pgfpathlineto{\pgfqpoint{0.821141in}{0.416705in}}%
\pgfpathmoveto{\pgfqpoint{0.939717in}{0.332038in}}%
\pgfpathlineto{\pgfqpoint{0.956117in}{0.522274in}}%
\pgfpathmoveto{\pgfqpoint{0.433515in}{0.614303in}}%
\pgfpathlineto{\pgfqpoint{0.505114in}{0.716000in}}%
\pgfpathmoveto{\pgfqpoint{0.433515in}{0.614303in}}%
\pgfpathlineto{\pgfqpoint{0.343068in}{0.716000in}}%
\pgfpathmoveto{\pgfqpoint{0.433515in}{0.614303in}}%
\pgfpathlineto{\pgfqpoint{0.525112in}{0.550213in}}%
\pgfpathmoveto{\pgfqpoint{0.433515in}{0.614303in}}%
\pgfpathlineto{\pgfqpoint{0.356186in}{0.497584in}}%
\pgfpathmoveto{\pgfqpoint{1.374740in}{0.644995in}}%
\pgfpathlineto{\pgfqpoint{1.407938in}{0.716000in}}%
\pgfpathmoveto{\pgfqpoint{1.374740in}{0.644995in}}%
\pgfpathlineto{\pgfqpoint{1.338490in}{0.716000in}}%
\pgfpathmoveto{\pgfqpoint{1.374740in}{0.644995in}}%
\pgfpathlineto{\pgfqpoint{1.376927in}{0.550483in}}%
\pgfpathmoveto{\pgfqpoint{1.374740in}{0.644995in}}%
\pgfpathlineto{\pgfqpoint{1.303766in}{0.622307in}}%
\pgfpathmoveto{\pgfqpoint{1.374740in}{0.644995in}}%
\pgfpathlineto{\pgfqpoint{1.453902in}{0.624729in}}%
\pgfpathmoveto{\pgfqpoint{1.094954in}{0.647018in}}%
\pgfpathlineto{\pgfqpoint{1.130146in}{0.716000in}}%
\pgfpathmoveto{\pgfqpoint{1.094954in}{0.647018in}}%
\pgfpathlineto{\pgfqpoint{1.060698in}{0.716000in}}%
\pgfpathmoveto{\pgfqpoint{1.094954in}{0.647018in}}%
\pgfpathlineto{\pgfqpoint{1.093079in}{0.556190in}}%
\pgfpathmoveto{\pgfqpoint{1.094954in}{0.647018in}}%
\pgfpathlineto{\pgfqpoint{1.164870in}{0.623450in}}%
\pgfpathmoveto{\pgfqpoint{1.094954in}{0.647018in}}%
\pgfpathlineto{\pgfqpoint{1.025974in}{0.623450in}}%
\pgfpathmoveto{\pgfqpoint{1.234318in}{0.644309in}}%
\pgfpathlineto{\pgfqpoint{1.269042in}{0.716000in}}%
\pgfpathmoveto{\pgfqpoint{1.234318in}{0.644309in}}%
\pgfpathlineto{\pgfqpoint{1.199594in}{0.716000in}}%
\pgfpathmoveto{\pgfqpoint{1.234318in}{0.644309in}}%
\pgfpathlineto{\pgfqpoint{1.233147in}{0.520967in}}%
\pgfpathmoveto{\pgfqpoint{1.234318in}{0.644309in}}%
\pgfpathlineto{\pgfqpoint{1.303766in}{0.622307in}}%
\pgfpathmoveto{\pgfqpoint{1.234318in}{0.644309in}}%
\pgfpathlineto{\pgfqpoint{1.164870in}{0.623450in}}%
\pgfpathmoveto{\pgfqpoint{0.956444in}{0.640237in}}%
\pgfpathlineto{\pgfqpoint{0.991250in}{0.716000in}}%
\pgfpathmoveto{\pgfqpoint{0.956444in}{0.640237in}}%
\pgfpathlineto{\pgfqpoint{0.921802in}{0.716000in}}%
\pgfpathmoveto{\pgfqpoint{0.956444in}{0.640237in}}%
\pgfpathlineto{\pgfqpoint{0.956117in}{0.522274in}}%
\pgfpathmoveto{\pgfqpoint{0.956444in}{0.640237in}}%
\pgfpathlineto{\pgfqpoint{1.025974in}{0.623450in}}%
\pgfpathmoveto{\pgfqpoint{0.956444in}{0.640237in}}%
\pgfpathlineto{\pgfqpoint{0.887078in}{0.623460in}}%
\pgfpathmoveto{\pgfqpoint{0.817690in}{0.645978in}}%
\pgfpathlineto{\pgfqpoint{0.852354in}{0.716000in}}%
\pgfpathmoveto{\pgfqpoint{0.817690in}{0.645978in}}%
\pgfpathlineto{\pgfqpoint{0.782906in}{0.716000in}}%
\pgfpathmoveto{\pgfqpoint{0.817690in}{0.645978in}}%
\pgfpathlineto{\pgfqpoint{0.818220in}{0.558974in}}%
\pgfpathmoveto{\pgfqpoint{0.817690in}{0.645978in}}%
\pgfpathlineto{\pgfqpoint{0.887078in}{0.623460in}}%
\pgfpathmoveto{\pgfqpoint{0.817690in}{0.645978in}}%
\pgfpathlineto{\pgfqpoint{0.748182in}{0.623460in}}%
\pgfpathmoveto{\pgfqpoint{1.188376in}{0.303943in}}%
\pgfpathlineto{\pgfqpoint{0.991250in}{0.100000in}}%
\pgfpathmoveto{\pgfqpoint{1.188376in}{0.303943in}}%
\pgfpathlineto{\pgfqpoint{1.261326in}{0.100000in}}%
\pgfpathmoveto{\pgfqpoint{1.188376in}{0.303943in}}%
\pgfpathlineto{\pgfqpoint{1.373214in}{0.396057in}}%
\pgfpathmoveto{\pgfqpoint{1.188376in}{0.303943in}}%
\pgfpathlineto{\pgfqpoint{1.082944in}{0.402689in}}%
\pgfpathmoveto{\pgfqpoint{1.188376in}{0.303943in}}%
\pgfpathlineto{\pgfqpoint{1.233147in}{0.520967in}}%
\pgfpathmoveto{\pgfqpoint{0.696453in}{0.313796in}}%
\pgfpathlineto{\pgfqpoint{0.721174in}{0.100000in}}%
\pgfpathmoveto{\pgfqpoint{0.696453in}{0.313796in}}%
\pgfpathlineto{\pgfqpoint{0.821141in}{0.416705in}}%
\pgfpathmoveto{\pgfqpoint{0.696453in}{0.313796in}}%
\pgfpathlineto{\pgfqpoint{0.539838in}{0.401685in}}%
\pgfpathmoveto{\pgfqpoint{0.696453in}{0.313796in}}%
\pgfpathlineto{\pgfqpoint{0.679173in}{0.523635in}}%
\pgfpathmoveto{\pgfqpoint{0.678822in}{0.640511in}}%
\pgfpathlineto{\pgfqpoint{0.713458in}{0.716000in}}%
\pgfpathmoveto{\pgfqpoint{0.678822in}{0.640511in}}%
\pgfpathlineto{\pgfqpoint{0.644010in}{0.716000in}}%
\pgfpathmoveto{\pgfqpoint{0.678822in}{0.640511in}}%
\pgfpathlineto{\pgfqpoint{0.679173in}{0.523635in}}%
\pgfpathmoveto{\pgfqpoint{0.678822in}{0.640511in}}%
\pgfpathlineto{\pgfqpoint{0.748182in}{0.623460in}}%
\pgfpathmoveto{\pgfqpoint{0.678822in}{0.640511in}}%
\pgfpathlineto{\pgfqpoint{0.609286in}{0.623460in}}%
\pgfpathmoveto{\pgfqpoint{1.556669in}{0.624742in}}%
\pgfpathlineto{\pgfqpoint{1.477386in}{0.716000in}}%
\pgfpathmoveto{\pgfqpoint{1.556669in}{0.624742in}}%
\pgfpathlineto{\pgfqpoint{1.639432in}{0.716000in}}%
\pgfpathmoveto{\pgfqpoint{1.556669in}{0.624742in}}%
\pgfpathlineto{\pgfqpoint{1.535558in}{0.497212in}}%
\pgfpathmoveto{\pgfqpoint{1.556669in}{0.624742in}}%
\pgfpathlineto{\pgfqpoint{1.453902in}{0.624729in}}%
\pgfpathmoveto{\pgfqpoint{0.539838in}{0.644311in}}%
\pgfpathlineto{\pgfqpoint{0.505114in}{0.716000in}}%
\pgfpathmoveto{\pgfqpoint{0.539838in}{0.644311in}}%
\pgfpathlineto{\pgfqpoint{0.574562in}{0.716000in}}%
\pgfpathmoveto{\pgfqpoint{0.539838in}{0.644311in}}%
\pgfpathlineto{\pgfqpoint{0.525112in}{0.550213in}}%
\pgfpathmoveto{\pgfqpoint{0.539838in}{0.644311in}}%
\pgfpathlineto{\pgfqpoint{0.609286in}{0.623460in}}%
\pgfpathmoveto{\pgfqpoint{0.539838in}{0.644311in}}%
\pgfpathlineto{\pgfqpoint{0.433515in}{0.614303in}}%
\pgfpathmoveto{\pgfqpoint{0.379905in}{0.288250in}}%
\pgfpathlineto{\pgfqpoint{0.181023in}{0.100000in}}%
\pgfpathmoveto{\pgfqpoint{0.379905in}{0.288250in}}%
\pgfpathlineto{\pgfqpoint{0.451098in}{0.100000in}}%
\pgfpathmoveto{\pgfqpoint{0.379905in}{0.288250in}}%
\pgfpathlineto{\pgfqpoint{0.181023in}{0.408000in}}%
\pgfpathmoveto{\pgfqpoint{0.379905in}{0.288250in}}%
\pgfpathlineto{\pgfqpoint{0.539838in}{0.401685in}}%
\pgfpathmoveto{\pgfqpoint{0.379905in}{0.288250in}}%
\pgfpathlineto{\pgfqpoint{0.356186in}{0.497584in}}%
\pgfpathmoveto{\pgfqpoint{1.528881in}{0.301240in}}%
\pgfpathlineto{\pgfqpoint{1.801477in}{0.100000in}}%
\pgfpathmoveto{\pgfqpoint{1.528881in}{0.301240in}}%
\pgfpathlineto{\pgfqpoint{1.261326in}{0.100000in}}%
\pgfpathmoveto{\pgfqpoint{1.528881in}{0.301240in}}%
\pgfpathlineto{\pgfqpoint{1.531402in}{0.100000in}}%
\pgfpathmoveto{\pgfqpoint{1.528881in}{0.301240in}}%
\pgfpathlineto{\pgfqpoint{1.801477in}{0.408000in}}%
\pgfpathmoveto{\pgfqpoint{1.528881in}{0.301240in}}%
\pgfpathlineto{\pgfqpoint{1.373214in}{0.396057in}}%
\pgfpathmoveto{\pgfqpoint{1.528881in}{0.301240in}}%
\pgfpathlineto{\pgfqpoint{1.535558in}{0.497212in}}%
\pgfpathmoveto{\pgfqpoint{1.666923in}{0.592391in}}%
\pgfpathlineto{\pgfqpoint{1.801477in}{0.716000in}}%
\pgfpathmoveto{\pgfqpoint{1.666923in}{0.592391in}}%
\pgfpathlineto{\pgfqpoint{1.801477in}{0.408000in}}%
\pgfpathmoveto{\pgfqpoint{1.666923in}{0.592391in}}%
\pgfpathlineto{\pgfqpoint{1.639432in}{0.716000in}}%
\pgfpathmoveto{\pgfqpoint{1.666923in}{0.592391in}}%
\pgfpathlineto{\pgfqpoint{1.535558in}{0.497212in}}%
\pgfpathmoveto{\pgfqpoint{1.666923in}{0.592391in}}%
\pgfpathlineto{\pgfqpoint{1.556669in}{0.624742in}}%
\pgfpathmoveto{\pgfqpoint{0.842327in}{0.230907in}}%
\pgfpathlineto{\pgfqpoint{0.721174in}{0.100000in}}%
\pgfpathmoveto{\pgfqpoint{0.842327in}{0.230907in}}%
\pgfpathlineto{\pgfqpoint{0.991250in}{0.100000in}}%
\pgfpathmoveto{\pgfqpoint{0.842327in}{0.230907in}}%
\pgfpathlineto{\pgfqpoint{0.821141in}{0.416705in}}%
\pgfpathmoveto{\pgfqpoint{0.842327in}{0.230907in}}%
\pgfpathlineto{\pgfqpoint{0.939717in}{0.332038in}}%
\pgfpathmoveto{\pgfqpoint{0.842327in}{0.230907in}}%
\pgfpathlineto{\pgfqpoint{0.696453in}{0.313796in}}%
\pgfpathmoveto{\pgfqpoint{0.557694in}{0.240746in}}%
\pgfpathlineto{\pgfqpoint{0.451098in}{0.100000in}}%
\pgfpathmoveto{\pgfqpoint{0.557694in}{0.240746in}}%
\pgfpathlineto{\pgfqpoint{0.721174in}{0.100000in}}%
\pgfpathmoveto{\pgfqpoint{0.557694in}{0.240746in}}%
\pgfpathlineto{\pgfqpoint{0.539838in}{0.401685in}}%
\pgfpathmoveto{\pgfqpoint{0.557694in}{0.240746in}}%
\pgfpathlineto{\pgfqpoint{0.696453in}{0.313796in}}%
\pgfpathmoveto{\pgfqpoint{0.557694in}{0.240746in}}%
\pgfpathlineto{\pgfqpoint{0.379905in}{0.288250in}}%
\pgfpathlineto{\pgfqpoint{0.379905in}{0.288250in}}%
\pgfusepath{stroke}%
\end{pgfscope}%
\begin{pgfscope}%
\pgfpathrectangle{\pgfqpoint{0.100000in}{0.100000in}}{\pgfqpoint{1.782500in}{1.232000in}}%
\pgfusepath{clip}%
\pgfsetrectcap%
\pgfsetroundjoin%
\pgfsetlinewidth{0.250937pt}%
\definecolor{currentstroke}{rgb}{0.835294,0.321569,0.035294}%
\pgfsetstrokecolor{currentstroke}%
\pgfsetdash{}{0pt}%
\pgfpathmoveto{\pgfqpoint{0.505114in}{1.001892in}}%
\pgfpathlineto{\pgfqpoint{0.451098in}{1.270400in}}%
\pgfpathmoveto{\pgfqpoint{1.531402in}{1.270400in}}%
\pgfpathlineto{\pgfqpoint{1.477386in}{1.001892in}}%
\pgfpathmoveto{\pgfqpoint{0.721174in}{1.270400in}}%
\pgfpathlineto{\pgfqpoint{0.991250in}{1.270400in}}%
\pgfpathmoveto{\pgfqpoint{0.721174in}{1.270400in}}%
\pgfpathlineto{\pgfqpoint{0.451098in}{1.270400in}}%
\pgfpathmoveto{\pgfqpoint{0.548824in}{0.917012in}}%
\pgfpathlineto{\pgfqpoint{0.505114in}{1.001892in}}%
\pgfpathmoveto{\pgfqpoint{0.603831in}{0.841156in}}%
\pgfpathlineto{\pgfqpoint{0.548824in}{0.917012in}}%
\pgfpathmoveto{\pgfqpoint{0.668731in}{0.776261in}}%
\pgfpathlineto{\pgfqpoint{0.603831in}{0.841156in}}%
\pgfpathmoveto{\pgfqpoint{0.741867in}{0.723983in}}%
\pgfpathlineto{\pgfqpoint{0.668731in}{0.776261in}}%
\pgfpathmoveto{\pgfqpoint{0.821370in}{0.685658in}}%
\pgfpathlineto{\pgfqpoint{0.741867in}{0.723983in}}%
\pgfpathmoveto{\pgfqpoint{0.905212in}{0.662265in}}%
\pgfpathlineto{\pgfqpoint{0.821370in}{0.685658in}}%
\pgfpathmoveto{\pgfqpoint{0.991250in}{0.654400in}}%
\pgfpathlineto{\pgfqpoint{0.905212in}{0.662265in}}%
\pgfpathmoveto{\pgfqpoint{1.077288in}{0.662265in}}%
\pgfpathlineto{\pgfqpoint{0.991250in}{0.654400in}}%
\pgfpathmoveto{\pgfqpoint{1.161130in}{0.685658in}}%
\pgfpathlineto{\pgfqpoint{1.077288in}{0.662265in}}%
\pgfpathmoveto{\pgfqpoint{1.240633in}{0.723983in}}%
\pgfpathlineto{\pgfqpoint{1.161130in}{0.685658in}}%
\pgfpathmoveto{\pgfqpoint{1.313769in}{0.776261in}}%
\pgfpathlineto{\pgfqpoint{1.240633in}{0.723983in}}%
\pgfpathmoveto{\pgfqpoint{1.378669in}{0.841156in}}%
\pgfpathlineto{\pgfqpoint{1.313769in}{0.776261in}}%
\pgfpathmoveto{\pgfqpoint{1.433676in}{0.917012in}}%
\pgfpathlineto{\pgfqpoint{1.477386in}{1.001892in}}%
\pgfpathmoveto{\pgfqpoint{1.433676in}{0.917012in}}%
\pgfpathlineto{\pgfqpoint{1.378669in}{0.841156in}}%
\pgfpathmoveto{\pgfqpoint{1.261326in}{1.270400in}}%
\pgfpathlineto{\pgfqpoint{0.991250in}{1.270400in}}%
\pgfpathmoveto{\pgfqpoint{1.261326in}{1.270400in}}%
\pgfpathlineto{\pgfqpoint{1.531402in}{1.270400in}}%
\pgfpathmoveto{\pgfqpoint{0.837830in}{1.028470in}}%
\pgfpathlineto{\pgfqpoint{0.991250in}{1.270400in}}%
\pgfpathmoveto{\pgfqpoint{0.837830in}{1.028470in}}%
\pgfpathlineto{\pgfqpoint{0.721174in}{1.270400in}}%
\pgfpathmoveto{\pgfqpoint{0.961801in}{0.851412in}}%
\pgfpathlineto{\pgfqpoint{1.113091in}{0.944697in}}%
\pgfpathmoveto{\pgfqpoint{0.961801in}{0.851412in}}%
\pgfpathlineto{\pgfqpoint{0.837830in}{1.028470in}}%
\pgfpathmoveto{\pgfqpoint{1.273497in}{1.004211in}}%
\pgfpathlineto{\pgfqpoint{1.113091in}{0.944697in}}%
\pgfpathmoveto{\pgfqpoint{0.674832in}{1.049989in}}%
\pgfpathlineto{\pgfqpoint{0.721174in}{1.270400in}}%
\pgfpathmoveto{\pgfqpoint{0.674832in}{1.049989in}}%
\pgfpathlineto{\pgfqpoint{0.837830in}{1.028470in}}%
\pgfpathmoveto{\pgfqpoint{1.093189in}{0.806475in}}%
\pgfpathlineto{\pgfqpoint{1.077288in}{0.662265in}}%
\pgfpathmoveto{\pgfqpoint{1.093189in}{0.806475in}}%
\pgfpathlineto{\pgfqpoint{1.161130in}{0.685658in}}%
\pgfpathmoveto{\pgfqpoint{1.093189in}{0.806475in}}%
\pgfpathlineto{\pgfqpoint{1.113091in}{0.944697in}}%
\pgfpathmoveto{\pgfqpoint{1.093189in}{0.806475in}}%
\pgfpathlineto{\pgfqpoint{0.961801in}{0.851412in}}%
\pgfpathmoveto{\pgfqpoint{1.212025in}{0.863682in}}%
\pgfpathlineto{\pgfqpoint{1.240633in}{0.723983in}}%
\pgfpathmoveto{\pgfqpoint{1.212025in}{0.863682in}}%
\pgfpathlineto{\pgfqpoint{1.313769in}{0.776261in}}%
\pgfpathmoveto{\pgfqpoint{1.212025in}{0.863682in}}%
\pgfpathlineto{\pgfqpoint{1.113091in}{0.944697in}}%
\pgfpathmoveto{\pgfqpoint{1.212025in}{0.863682in}}%
\pgfpathlineto{\pgfqpoint{1.273497in}{1.004211in}}%
\pgfpathmoveto{\pgfqpoint{1.212025in}{0.863682in}}%
\pgfpathlineto{\pgfqpoint{1.093189in}{0.806475in}}%
\pgfpathmoveto{\pgfqpoint{0.822435in}{0.830135in}}%
\pgfpathlineto{\pgfqpoint{0.741867in}{0.723983in}}%
\pgfpathmoveto{\pgfqpoint{0.822435in}{0.830135in}}%
\pgfpathlineto{\pgfqpoint{0.821370in}{0.685658in}}%
\pgfpathmoveto{\pgfqpoint{0.822435in}{0.830135in}}%
\pgfpathlineto{\pgfqpoint{0.837830in}{1.028470in}}%
\pgfpathmoveto{\pgfqpoint{0.822435in}{0.830135in}}%
\pgfpathlineto{\pgfqpoint{0.961801in}{0.851412in}}%
\pgfpathmoveto{\pgfqpoint{0.716798in}{0.910757in}}%
\pgfpathlineto{\pgfqpoint{0.603831in}{0.841156in}}%
\pgfpathmoveto{\pgfqpoint{0.716798in}{0.910757in}}%
\pgfpathlineto{\pgfqpoint{0.668731in}{0.776261in}}%
\pgfpathmoveto{\pgfqpoint{0.716798in}{0.910757in}}%
\pgfpathlineto{\pgfqpoint{0.837830in}{1.028470in}}%
\pgfpathmoveto{\pgfqpoint{0.716798in}{0.910757in}}%
\pgfpathlineto{\pgfqpoint{0.674832in}{1.049989in}}%
\pgfpathmoveto{\pgfqpoint{0.716798in}{0.910757in}}%
\pgfpathlineto{\pgfqpoint{0.822435in}{0.830135in}}%
\pgfpathmoveto{\pgfqpoint{1.362495in}{1.129252in}}%
\pgfpathlineto{\pgfqpoint{1.477386in}{1.001892in}}%
\pgfpathmoveto{\pgfqpoint{1.362495in}{1.129252in}}%
\pgfpathlineto{\pgfqpoint{1.531402in}{1.270400in}}%
\pgfpathmoveto{\pgfqpoint{1.362495in}{1.129252in}}%
\pgfpathlineto{\pgfqpoint{1.261326in}{1.270400in}}%
\pgfpathmoveto{\pgfqpoint{1.362495in}{1.129252in}}%
\pgfpathlineto{\pgfqpoint{1.273497in}{1.004211in}}%
\pgfpathmoveto{\pgfqpoint{0.953154in}{0.728383in}}%
\pgfpathlineto{\pgfqpoint{0.905212in}{0.662265in}}%
\pgfpathmoveto{\pgfqpoint{0.953154in}{0.728383in}}%
\pgfpathlineto{\pgfqpoint{0.991250in}{0.654400in}}%
\pgfpathmoveto{\pgfqpoint{0.953154in}{0.728383in}}%
\pgfpathlineto{\pgfqpoint{0.961801in}{0.851412in}}%
\pgfpathmoveto{\pgfqpoint{1.297122in}{0.872566in}}%
\pgfpathlineto{\pgfqpoint{1.313769in}{0.776261in}}%
\pgfpathmoveto{\pgfqpoint{1.297122in}{0.872566in}}%
\pgfpathlineto{\pgfqpoint{1.378669in}{0.841156in}}%
\pgfpathmoveto{\pgfqpoint{1.297122in}{0.872566in}}%
\pgfpathlineto{\pgfqpoint{1.273497in}{1.004211in}}%
\pgfpathmoveto{\pgfqpoint{1.297122in}{0.872566in}}%
\pgfpathlineto{\pgfqpoint{1.212025in}{0.863682in}}%
\pgfpathmoveto{\pgfqpoint{1.173657in}{0.778274in}}%
\pgfpathlineto{\pgfqpoint{1.161130in}{0.685658in}}%
\pgfpathmoveto{\pgfqpoint{1.173657in}{0.778274in}}%
\pgfpathlineto{\pgfqpoint{1.240633in}{0.723983in}}%
\pgfpathmoveto{\pgfqpoint{1.173657in}{0.778274in}}%
\pgfpathlineto{\pgfqpoint{1.093189in}{0.806475in}}%
\pgfpathmoveto{\pgfqpoint{1.173657in}{0.778274in}}%
\pgfpathlineto{\pgfqpoint{1.212025in}{0.863682in}}%
\pgfpathmoveto{\pgfqpoint{1.375529in}{1.013033in}}%
\pgfpathlineto{\pgfqpoint{1.477386in}{1.001892in}}%
\pgfpathmoveto{\pgfqpoint{1.375529in}{1.013033in}}%
\pgfpathlineto{\pgfqpoint{1.433676in}{0.917012in}}%
\pgfpathmoveto{\pgfqpoint{1.375529in}{1.013033in}}%
\pgfpathlineto{\pgfqpoint{1.273497in}{1.004211in}}%
\pgfpathmoveto{\pgfqpoint{1.375529in}{1.013033in}}%
\pgfpathlineto{\pgfqpoint{1.362495in}{1.129252in}}%
\pgfpathmoveto{\pgfqpoint{0.742156in}{0.817783in}}%
\pgfpathlineto{\pgfqpoint{0.668731in}{0.776261in}}%
\pgfpathmoveto{\pgfqpoint{0.742156in}{0.817783in}}%
\pgfpathlineto{\pgfqpoint{0.741867in}{0.723983in}}%
\pgfpathmoveto{\pgfqpoint{0.742156in}{0.817783in}}%
\pgfpathlineto{\pgfqpoint{0.822435in}{0.830135in}}%
\pgfpathmoveto{\pgfqpoint{0.742156in}{0.817783in}}%
\pgfpathlineto{\pgfqpoint{0.716798in}{0.910757in}}%
\pgfpathmoveto{\pgfqpoint{0.633716in}{0.933208in}}%
\pgfpathlineto{\pgfqpoint{0.548824in}{0.917012in}}%
\pgfpathmoveto{\pgfqpoint{0.633716in}{0.933208in}}%
\pgfpathlineto{\pgfqpoint{0.603831in}{0.841156in}}%
\pgfpathmoveto{\pgfqpoint{0.633716in}{0.933208in}}%
\pgfpathlineto{\pgfqpoint{0.674832in}{1.049989in}}%
\pgfpathmoveto{\pgfqpoint{0.633716in}{0.933208in}}%
\pgfpathlineto{\pgfqpoint{0.716798in}{0.910757in}}%
\pgfpathmoveto{\pgfqpoint{0.572102in}{1.136818in}}%
\pgfpathlineto{\pgfqpoint{0.451098in}{1.270400in}}%
\pgfpathmoveto{\pgfqpoint{0.572102in}{1.136818in}}%
\pgfpathlineto{\pgfqpoint{0.505114in}{1.001892in}}%
\pgfpathmoveto{\pgfqpoint{0.572102in}{1.136818in}}%
\pgfpathlineto{\pgfqpoint{0.721174in}{1.270400in}}%
\pgfpathmoveto{\pgfqpoint{0.572102in}{1.136818in}}%
\pgfpathlineto{\pgfqpoint{0.674832in}{1.049989in}}%
\pgfpathmoveto{\pgfqpoint{1.029231in}{0.730021in}}%
\pgfpathlineto{\pgfqpoint{0.991250in}{0.654400in}}%
\pgfpathmoveto{\pgfqpoint{1.029231in}{0.730021in}}%
\pgfpathlineto{\pgfqpoint{1.077288in}{0.662265in}}%
\pgfpathmoveto{\pgfqpoint{1.029231in}{0.730021in}}%
\pgfpathlineto{\pgfqpoint{0.961801in}{0.851412in}}%
\pgfpathmoveto{\pgfqpoint{1.029231in}{0.730021in}}%
\pgfpathlineto{\pgfqpoint{1.093189in}{0.806475in}}%
\pgfpathmoveto{\pgfqpoint{1.029231in}{0.730021in}}%
\pgfpathlineto{\pgfqpoint{0.953154in}{0.728383in}}%
\pgfpathmoveto{\pgfqpoint{0.878278in}{0.743820in}}%
\pgfpathlineto{\pgfqpoint{0.821370in}{0.685658in}}%
\pgfpathmoveto{\pgfqpoint{0.878278in}{0.743820in}}%
\pgfpathlineto{\pgfqpoint{0.905212in}{0.662265in}}%
\pgfpathmoveto{\pgfqpoint{0.878278in}{0.743820in}}%
\pgfpathlineto{\pgfqpoint{0.961801in}{0.851412in}}%
\pgfpathmoveto{\pgfqpoint{0.878278in}{0.743820in}}%
\pgfpathlineto{\pgfqpoint{0.822435in}{0.830135in}}%
\pgfpathmoveto{\pgfqpoint{0.878278in}{0.743820in}}%
\pgfpathlineto{\pgfqpoint{0.953154in}{0.728383in}}%
\pgfpathmoveto{\pgfqpoint{1.357947in}{0.924566in}}%
\pgfpathlineto{\pgfqpoint{1.378669in}{0.841156in}}%
\pgfpathmoveto{\pgfqpoint{1.357947in}{0.924566in}}%
\pgfpathlineto{\pgfqpoint{1.433676in}{0.917012in}}%
\pgfpathmoveto{\pgfqpoint{1.357947in}{0.924566in}}%
\pgfpathlineto{\pgfqpoint{1.273497in}{1.004211in}}%
\pgfpathmoveto{\pgfqpoint{1.357947in}{0.924566in}}%
\pgfpathlineto{\pgfqpoint{1.297122in}{0.872566in}}%
\pgfpathmoveto{\pgfqpoint{1.357947in}{0.924566in}}%
\pgfpathlineto{\pgfqpoint{1.375529in}{1.013033in}}%
\pgfpathmoveto{\pgfqpoint{1.015627in}{1.045348in}}%
\pgfpathlineto{\pgfqpoint{0.991250in}{1.270400in}}%
\pgfpathmoveto{\pgfqpoint{1.015627in}{1.045348in}}%
\pgfpathlineto{\pgfqpoint{1.113091in}{0.944697in}}%
\pgfpathmoveto{\pgfqpoint{1.015627in}{1.045348in}}%
\pgfpathlineto{\pgfqpoint{0.837830in}{1.028470in}}%
\pgfpathmoveto{\pgfqpoint{1.015627in}{1.045348in}}%
\pgfpathlineto{\pgfqpoint{0.961801in}{0.851412in}}%
\pgfpathmoveto{\pgfqpoint{1.169547in}{1.110718in}}%
\pgfpathlineto{\pgfqpoint{0.991250in}{1.270400in}}%
\pgfpathmoveto{\pgfqpoint{1.169547in}{1.110718in}}%
\pgfpathlineto{\pgfqpoint{1.261326in}{1.270400in}}%
\pgfpathmoveto{\pgfqpoint{1.169547in}{1.110718in}}%
\pgfpathlineto{\pgfqpoint{1.113091in}{0.944697in}}%
\pgfpathmoveto{\pgfqpoint{1.169547in}{1.110718in}}%
\pgfpathlineto{\pgfqpoint{1.273497in}{1.004211in}}%
\pgfpathmoveto{\pgfqpoint{1.169547in}{1.110718in}}%
\pgfpathlineto{\pgfqpoint{1.362495in}{1.129252in}}%
\pgfpathmoveto{\pgfqpoint{1.169547in}{1.110718in}}%
\pgfpathlineto{\pgfqpoint{1.015627in}{1.045348in}}%
\pgfpathmoveto{\pgfqpoint{0.586918in}{1.007784in}}%
\pgfpathlineto{\pgfqpoint{0.505114in}{1.001892in}}%
\pgfpathmoveto{\pgfqpoint{0.586918in}{1.007784in}}%
\pgfpathlineto{\pgfqpoint{0.548824in}{0.917012in}}%
\pgfpathmoveto{\pgfqpoint{0.586918in}{1.007784in}}%
\pgfpathlineto{\pgfqpoint{0.674832in}{1.049989in}}%
\pgfpathmoveto{\pgfqpoint{0.586918in}{1.007784in}}%
\pgfpathlineto{\pgfqpoint{0.633716in}{0.933208in}}%
\pgfpathmoveto{\pgfqpoint{0.586918in}{1.007784in}}%
\pgfpathlineto{\pgfqpoint{0.572102in}{1.136818in}}%
\pgfpathlineto{\pgfqpoint{0.572102in}{1.136818in}}%
\pgfusepath{stroke}%
\end{pgfscope}%
\begin{pgfscope}%
\pgfpathrectangle{\pgfqpoint{0.100000in}{0.100000in}}{\pgfqpoint{1.782500in}{1.232000in}}%
\pgfusepath{clip}%
\pgfsetbuttcap%
\pgfsetroundjoin%
\definecolor{currentfill}{rgb}{0.054902,0.262745,0.486275}%
\pgfsetfillcolor{currentfill}%
\pgfsetlinewidth{1.003750pt}%
\definecolor{currentstroke}{rgb}{0.054902,0.262745,0.486275}%
\pgfsetstrokecolor{currentstroke}%
\pgfsetdash{}{0pt}%
\pgfsys@defobject{currentmarker}{\pgfqpoint{-0.018373in}{-0.018373in}}{\pgfqpoint{0.018373in}{0.018373in}}{%
\pgfpathmoveto{\pgfqpoint{0.000000in}{-0.018373in}}%
\pgfpathcurveto{\pgfqpoint{0.004873in}{-0.018373in}}{\pgfqpoint{0.009546in}{-0.016437in}}{\pgfqpoint{0.012992in}{-0.012992in}}%
\pgfpathcurveto{\pgfqpoint{0.016437in}{-0.009546in}}{\pgfqpoint{0.018373in}{-0.004873in}}{\pgfqpoint{0.018373in}{0.000000in}}%
\pgfpathcurveto{\pgfqpoint{0.018373in}{0.004873in}}{\pgfqpoint{0.016437in}{0.009546in}}{\pgfqpoint{0.012992in}{0.012992in}}%
\pgfpathcurveto{\pgfqpoint{0.009546in}{0.016437in}}{\pgfqpoint{0.004873in}{0.018373in}}{\pgfqpoint{0.000000in}{0.018373in}}%
\pgfpathcurveto{\pgfqpoint{-0.004873in}{0.018373in}}{\pgfqpoint{-0.009546in}{0.016437in}}{\pgfqpoint{-0.012992in}{0.012992in}}%
\pgfpathcurveto{\pgfqpoint{-0.016437in}{0.009546in}}{\pgfqpoint{-0.018373in}{0.004873in}}{\pgfqpoint{-0.018373in}{0.000000in}}%
\pgfpathcurveto{\pgfqpoint{-0.018373in}{-0.004873in}}{\pgfqpoint{-0.016437in}{-0.009546in}}{\pgfqpoint{-0.012992in}{-0.012992in}}%
\pgfpathcurveto{\pgfqpoint{-0.009546in}{-0.016437in}}{\pgfqpoint{-0.004873in}{-0.018373in}}{\pgfqpoint{0.000000in}{-0.018373in}}%
\pgfpathlineto{\pgfqpoint{0.000000in}{-0.018373in}}%
\pgfpathclose%
\pgfusepath{stroke,fill}%
}%
\begin{pgfscope}%
\pgfsys@transformshift{1.338490in}{0.716000in}%
\pgfsys@useobject{currentmarker}{}%
\end{pgfscope}%
\begin{pgfscope}%
\pgfsys@transformshift{1.269042in}{0.716000in}%
\pgfsys@useobject{currentmarker}{}%
\end{pgfscope}%
\begin{pgfscope}%
\pgfsys@transformshift{1.199594in}{0.716000in}%
\pgfsys@useobject{currentmarker}{}%
\end{pgfscope}%
\begin{pgfscope}%
\pgfsys@transformshift{1.130146in}{0.716000in}%
\pgfsys@useobject{currentmarker}{}%
\end{pgfscope}%
\begin{pgfscope}%
\pgfsys@transformshift{1.060698in}{0.716000in}%
\pgfsys@useobject{currentmarker}{}%
\end{pgfscope}%
\begin{pgfscope}%
\pgfsys@transformshift{0.991250in}{0.716000in}%
\pgfsys@useobject{currentmarker}{}%
\end{pgfscope}%
\begin{pgfscope}%
\pgfsys@transformshift{0.921802in}{0.716000in}%
\pgfsys@useobject{currentmarker}{}%
\end{pgfscope}%
\begin{pgfscope}%
\pgfsys@transformshift{0.852354in}{0.716000in}%
\pgfsys@useobject{currentmarker}{}%
\end{pgfscope}%
\begin{pgfscope}%
\pgfsys@transformshift{0.782906in}{0.716000in}%
\pgfsys@useobject{currentmarker}{}%
\end{pgfscope}%
\begin{pgfscope}%
\pgfsys@transformshift{0.713458in}{0.716000in}%
\pgfsys@useobject{currentmarker}{}%
\end{pgfscope}%
\begin{pgfscope}%
\pgfsys@transformshift{0.644010in}{0.716000in}%
\pgfsys@useobject{currentmarker}{}%
\end{pgfscope}%
\end{pgfscope}%
\begin{pgfscope}%
\pgfpathrectangle{\pgfqpoint{0.100000in}{0.100000in}}{\pgfqpoint{1.782500in}{1.232000in}}%
\pgfusepath{clip}%
\pgfsetbuttcap%
\pgfsetroundjoin%
\definecolor{currentfill}{rgb}{0.835294,0.321569,0.035294}%
\pgfsetfillcolor{currentfill}%
\pgfsetlinewidth{1.003750pt}%
\definecolor{currentstroke}{rgb}{0.835294,0.321569,0.035294}%
\pgfsetstrokecolor{currentstroke}%
\pgfsetdash{}{0pt}%
\pgfsys@defobject{currentmarker}{\pgfqpoint{-0.018373in}{-0.018373in}}{\pgfqpoint{0.018373in}{0.018373in}}{%
\pgfpathmoveto{\pgfqpoint{0.000000in}{-0.018373in}}%
\pgfpathcurveto{\pgfqpoint{0.004873in}{-0.018373in}}{\pgfqpoint{0.009546in}{-0.016437in}}{\pgfqpoint{0.012992in}{-0.012992in}}%
\pgfpathcurveto{\pgfqpoint{0.016437in}{-0.009546in}}{\pgfqpoint{0.018373in}{-0.004873in}}{\pgfqpoint{0.018373in}{0.000000in}}%
\pgfpathcurveto{\pgfqpoint{0.018373in}{0.004873in}}{\pgfqpoint{0.016437in}{0.009546in}}{\pgfqpoint{0.012992in}{0.012992in}}%
\pgfpathcurveto{\pgfqpoint{0.009546in}{0.016437in}}{\pgfqpoint{0.004873in}{0.018373in}}{\pgfqpoint{0.000000in}{0.018373in}}%
\pgfpathcurveto{\pgfqpoint{-0.004873in}{0.018373in}}{\pgfqpoint{-0.009546in}{0.016437in}}{\pgfqpoint{-0.012992in}{0.012992in}}%
\pgfpathcurveto{\pgfqpoint{-0.016437in}{0.009546in}}{\pgfqpoint{-0.018373in}{0.004873in}}{\pgfqpoint{-0.018373in}{0.000000in}}%
\pgfpathcurveto{\pgfqpoint{-0.018373in}{-0.004873in}}{\pgfqpoint{-0.016437in}{-0.009546in}}{\pgfqpoint{-0.012992in}{-0.012992in}}%
\pgfpathcurveto{\pgfqpoint{-0.009546in}{-0.016437in}}{\pgfqpoint{-0.004873in}{-0.018373in}}{\pgfqpoint{0.000000in}{-0.018373in}}%
\pgfpathlineto{\pgfqpoint{0.000000in}{-0.018373in}}%
\pgfpathclose%
\pgfusepath{stroke,fill}%
}%
\begin{pgfscope}%
\pgfsys@transformshift{0.668731in}{0.776261in}%
\pgfsys@useobject{currentmarker}{}%
\end{pgfscope}%
\begin{pgfscope}%
\pgfsys@transformshift{0.741867in}{0.723983in}%
\pgfsys@useobject{currentmarker}{}%
\end{pgfscope}%
\begin{pgfscope}%
\pgfsys@transformshift{0.821370in}{0.685658in}%
\pgfsys@useobject{currentmarker}{}%
\end{pgfscope}%
\begin{pgfscope}%
\pgfsys@transformshift{0.905212in}{0.662265in}%
\pgfsys@useobject{currentmarker}{}%
\end{pgfscope}%
\begin{pgfscope}%
\pgfsys@transformshift{0.991250in}{0.654400in}%
\pgfsys@useobject{currentmarker}{}%
\end{pgfscope}%
\begin{pgfscope}%
\pgfsys@transformshift{1.077288in}{0.662265in}%
\pgfsys@useobject{currentmarker}{}%
\end{pgfscope}%
\begin{pgfscope}%
\pgfsys@transformshift{1.161130in}{0.685658in}%
\pgfsys@useobject{currentmarker}{}%
\end{pgfscope}%
\begin{pgfscope}%
\pgfsys@transformshift{1.240633in}{0.723983in}%
\pgfsys@useobject{currentmarker}{}%
\end{pgfscope}%
\begin{pgfscope}%
\pgfsys@transformshift{1.313769in}{0.776261in}%
\pgfsys@useobject{currentmarker}{}%
\end{pgfscope}%
\end{pgfscope}%
\end{pgfpicture}%
\makeatother%
\endgroup%
}
        \caption{Iteration 3: Find interface}\label{fig:example-iter2-interface}
    \end{subfigure}
    \begin{subfigure}[b]{.32\linewidth}
        \scalebox{0.8}{%% Creator: Matplotlib, PGF backend
%%
%% To include the figure in your LaTeX document, write
%%   \input{<filename>.pgf}
%%
%% Make sure the required packages are loaded in your preamble
%%   \usepackage{pgf}
%%
%% Also ensure that all the required font packages are loaded; for instance,
%% the lmodern package is sometimes necessary when using math font.
%%   \usepackage{lmodern}
%%
%% Figures using additional raster images can only be included by \input if
%% they are in the same directory as the main LaTeX file. For loading figures
%% from other directories you can use the `import` package
%%   \usepackage{import}
%%
%% and then include the figures with
%%   \import{<path to file>}{<filename>.pgf}
%%
%% Matplotlib used the following preamble
%%   
%%   \usepackage{fontspec}
%%   \setmainfont{DejaVuSans.ttf}[Path=\detokenize{/home/fabio/Internodes-CM/.venv/lib/python3.8/site-packages/matplotlib/mpl-data/fonts/ttf/}]
%%   \setsansfont{DejaVuSans.ttf}[Path=\detokenize{/home/fabio/Internodes-CM/.venv/lib/python3.8/site-packages/matplotlib/mpl-data/fonts/ttf/}]
%%   \setmonofont{DejaVuSansMono.ttf}[Path=\detokenize{/home/fabio/Internodes-CM/.venv/lib/python3.8/site-packages/matplotlib/mpl-data/fonts/ttf/}]
%%   \makeatletter\@ifpackageloaded{underscore}{}{\usepackage[strings]{underscore}}\makeatother
%%
\begingroup%
\makeatletter%
\begin{pgfpicture}%
\pgfpathrectangle{\pgfpointorigin}{\pgfqpoint{1.982500in}{1.432000in}}%
\pgfusepath{use as bounding box, clip}%
\begin{pgfscope}%
\pgfsetbuttcap%
\pgfsetmiterjoin%
\definecolor{currentfill}{rgb}{1.000000,1.000000,1.000000}%
\pgfsetfillcolor{currentfill}%
\pgfsetlinewidth{0.000000pt}%
\definecolor{currentstroke}{rgb}{1.000000,1.000000,1.000000}%
\pgfsetstrokecolor{currentstroke}%
\pgfsetdash{}{0pt}%
\pgfpathmoveto{\pgfqpoint{0.000000in}{0.000000in}}%
\pgfpathlineto{\pgfqpoint{1.982500in}{0.000000in}}%
\pgfpathlineto{\pgfqpoint{1.982500in}{1.432000in}}%
\pgfpathlineto{\pgfqpoint{0.000000in}{1.432000in}}%
\pgfpathlineto{\pgfqpoint{0.000000in}{0.000000in}}%
\pgfpathclose%
\pgfusepath{fill}%
\end{pgfscope}%
\begin{pgfscope}%
\pgfpathrectangle{\pgfqpoint{0.100000in}{0.100000in}}{\pgfqpoint{1.782500in}{1.232000in}}%
\pgfusepath{clip}%
\pgfsetrectcap%
\pgfsetroundjoin%
\pgfsetlinewidth{0.250937pt}%
\definecolor{currentstroke}{rgb}{0.054902,0.262745,0.486275}%
\pgfsetstrokecolor{currentstroke}%
\pgfsetdash{}{0pt}%
\pgfpathmoveto{\pgfqpoint{1.878506in}{0.284816in}}%
\pgfpathlineto{\pgfqpoint{1.892500in}{0.285785in}}%
\pgfpathmoveto{\pgfqpoint{1.726252in}{0.286703in}}%
\pgfpathlineto{\pgfqpoint{1.878506in}{0.284816in}}%
\pgfpathmoveto{\pgfqpoint{1.575612in}{0.301602in}}%
\pgfpathlineto{\pgfqpoint{1.726252in}{0.286703in}}%
\pgfpathmoveto{\pgfqpoint{1.443371in}{0.318705in}}%
\pgfpathlineto{\pgfqpoint{1.575612in}{0.301602in}}%
\pgfpathmoveto{\pgfqpoint{1.299293in}{0.342555in}}%
\pgfpathlineto{\pgfqpoint{1.443371in}{0.318705in}}%
\pgfpathmoveto{\pgfqpoint{1.171810in}{0.367799in}}%
\pgfpathlineto{\pgfqpoint{1.299293in}{0.342555in}}%
\pgfpathmoveto{\pgfqpoint{1.010874in}{0.404326in}}%
\pgfpathlineto{\pgfqpoint{0.847494in}{0.422933in}}%
\pgfpathmoveto{\pgfqpoint{1.010874in}{0.404326in}}%
\pgfpathlineto{\pgfqpoint{1.171810in}{0.367799in}}%
\pgfpathmoveto{\pgfqpoint{0.479331in}{0.444537in}}%
\pgfpathlineto{\pgfqpoint{0.847494in}{0.422933in}}%
\pgfpathmoveto{\pgfqpoint{0.479331in}{0.444537in}}%
\pgfpathlineto{\pgfqpoint{0.119998in}{0.458147in}}%
\pgfpathmoveto{\pgfqpoint{0.094733in}{0.090000in}}%
\pgfpathlineto{\pgfqpoint{0.119998in}{0.458147in}}%
\pgfpathmoveto{\pgfqpoint{1.182938in}{0.110001in}}%
\pgfpathlineto{\pgfqpoint{1.223230in}{0.090000in}}%
\pgfpathmoveto{\pgfqpoint{1.182938in}{0.110001in}}%
\pgfpathlineto{\pgfqpoint{1.135987in}{0.090000in}}%
\pgfpathmoveto{\pgfqpoint{1.495174in}{0.123111in}}%
\pgfpathlineto{\pgfqpoint{1.495251in}{0.090000in}}%
\pgfpathmoveto{\pgfqpoint{1.495174in}{0.123111in}}%
\pgfpathlineto{\pgfqpoint{1.679724in}{0.090000in}}%
\pgfpathmoveto{\pgfqpoint{1.495174in}{0.123111in}}%
\pgfpathlineto{\pgfqpoint{1.182938in}{0.110001in}}%
\pgfpathmoveto{\pgfqpoint{0.853768in}{0.184448in}}%
\pgfpathlineto{\pgfqpoint{0.861873in}{0.090000in}}%
\pgfpathmoveto{\pgfqpoint{0.853768in}{0.184448in}}%
\pgfpathlineto{\pgfqpoint{1.182938in}{0.110001in}}%
\pgfpathmoveto{\pgfqpoint{0.480075in}{0.132151in}}%
\pgfpathlineto{\pgfqpoint{0.119998in}{0.458147in}}%
\pgfpathmoveto{\pgfqpoint{0.480075in}{0.132151in}}%
\pgfpathlineto{\pgfqpoint{0.479331in}{0.444537in}}%
\pgfpathmoveto{\pgfqpoint{0.480075in}{0.132151in}}%
\pgfpathlineto{\pgfqpoint{0.341662in}{0.090000in}}%
\pgfpathmoveto{\pgfqpoint{0.480075in}{0.132151in}}%
\pgfpathlineto{\pgfqpoint{0.586664in}{0.090000in}}%
\pgfpathmoveto{\pgfqpoint{0.480075in}{0.132151in}}%
\pgfpathlineto{\pgfqpoint{0.853768in}{0.184448in}}%
\pgfpathmoveto{\pgfqpoint{1.892500in}{0.267463in}}%
\pgfpathlineto{\pgfqpoint{1.878506in}{0.284816in}}%
\pgfpathmoveto{\pgfqpoint{1.892500in}{0.136507in}}%
\pgfpathlineto{\pgfqpoint{1.830909in}{0.090000in}}%
\pgfpathmoveto{\pgfqpoint{1.649736in}{0.189018in}}%
\pgfpathlineto{\pgfqpoint{1.726252in}{0.286703in}}%
\pgfpathmoveto{\pgfqpoint{1.649736in}{0.189018in}}%
\pgfpathlineto{\pgfqpoint{1.575612in}{0.301602in}}%
\pgfpathmoveto{\pgfqpoint{1.649736in}{0.189018in}}%
\pgfpathlineto{\pgfqpoint{1.774310in}{0.090000in}}%
\pgfpathmoveto{\pgfqpoint{1.649736in}{0.189018in}}%
\pgfpathlineto{\pgfqpoint{1.495174in}{0.123111in}}%
\pgfpathmoveto{\pgfqpoint{1.349551in}{0.215893in}}%
\pgfpathlineto{\pgfqpoint{1.443371in}{0.318705in}}%
\pgfpathmoveto{\pgfqpoint{1.349551in}{0.215893in}}%
\pgfpathlineto{\pgfqpoint{1.299293in}{0.342555in}}%
\pgfpathmoveto{\pgfqpoint{1.349551in}{0.215893in}}%
\pgfpathlineto{\pgfqpoint{1.182938in}{0.110001in}}%
\pgfpathmoveto{\pgfqpoint{1.349551in}{0.215893in}}%
\pgfpathlineto{\pgfqpoint{1.495174in}{0.123111in}}%
\pgfpathmoveto{\pgfqpoint{1.048874in}{0.262157in}}%
\pgfpathlineto{\pgfqpoint{1.171810in}{0.367799in}}%
\pgfpathmoveto{\pgfqpoint{1.048874in}{0.262157in}}%
\pgfpathlineto{\pgfqpoint{1.010874in}{0.404326in}}%
\pgfpathmoveto{\pgfqpoint{1.048874in}{0.262157in}}%
\pgfpathlineto{\pgfqpoint{1.182938in}{0.110001in}}%
\pgfpathmoveto{\pgfqpoint{1.048874in}{0.262157in}}%
\pgfpathlineto{\pgfqpoint{0.853768in}{0.184448in}}%
\pgfpathmoveto{\pgfqpoint{0.665666in}{0.289310in}}%
\pgfpathlineto{\pgfqpoint{0.847494in}{0.422933in}}%
\pgfpathmoveto{\pgfqpoint{0.665666in}{0.289310in}}%
\pgfpathlineto{\pgfqpoint{0.479331in}{0.444537in}}%
\pgfpathmoveto{\pgfqpoint{0.665666in}{0.289310in}}%
\pgfpathlineto{\pgfqpoint{0.853768in}{0.184448in}}%
\pgfpathmoveto{\pgfqpoint{0.665666in}{0.289310in}}%
\pgfpathlineto{\pgfqpoint{0.480075in}{0.132151in}}%
\pgfpathmoveto{\pgfqpoint{1.803333in}{0.202002in}}%
\pgfpathlineto{\pgfqpoint{1.878506in}{0.284816in}}%
\pgfpathmoveto{\pgfqpoint{1.803333in}{0.202002in}}%
\pgfpathlineto{\pgfqpoint{1.726252in}{0.286703in}}%
\pgfpathmoveto{\pgfqpoint{1.803333in}{0.202002in}}%
\pgfpathlineto{\pgfqpoint{1.802121in}{0.090000in}}%
\pgfpathmoveto{\pgfqpoint{1.803333in}{0.202002in}}%
\pgfpathlineto{\pgfqpoint{1.892500in}{0.192826in}}%
\pgfpathmoveto{\pgfqpoint{1.803333in}{0.202002in}}%
\pgfpathlineto{\pgfqpoint{1.649736in}{0.189018in}}%
\pgfpathmoveto{\pgfqpoint{1.501549in}{0.225304in}}%
\pgfpathlineto{\pgfqpoint{1.575612in}{0.301602in}}%
\pgfpathmoveto{\pgfqpoint{1.501549in}{0.225304in}}%
\pgfpathlineto{\pgfqpoint{1.443371in}{0.318705in}}%
\pgfpathmoveto{\pgfqpoint{1.501549in}{0.225304in}}%
\pgfpathlineto{\pgfqpoint{1.495174in}{0.123111in}}%
\pgfpathmoveto{\pgfqpoint{1.501549in}{0.225304in}}%
\pgfpathlineto{\pgfqpoint{1.649736in}{0.189018in}}%
\pgfpathmoveto{\pgfqpoint{1.501549in}{0.225304in}}%
\pgfpathlineto{\pgfqpoint{1.349551in}{0.215893in}}%
\pgfpathmoveto{\pgfqpoint{1.185439in}{0.090000in}}%
\pgfpathlineto{\pgfqpoint{1.182938in}{0.110001in}}%
\pgfpathmoveto{\pgfqpoint{1.206294in}{0.256949in}}%
\pgfpathlineto{\pgfqpoint{1.299293in}{0.342555in}}%
\pgfpathmoveto{\pgfqpoint{1.206294in}{0.256949in}}%
\pgfpathlineto{\pgfqpoint{1.171810in}{0.367799in}}%
\pgfpathmoveto{\pgfqpoint{1.206294in}{0.256949in}}%
\pgfpathlineto{\pgfqpoint{1.182938in}{0.110001in}}%
\pgfpathmoveto{\pgfqpoint{1.206294in}{0.256949in}}%
\pgfpathlineto{\pgfqpoint{1.349551in}{0.215893in}}%
\pgfpathmoveto{\pgfqpoint{1.206294in}{0.256949in}}%
\pgfpathlineto{\pgfqpoint{1.048874in}{0.262157in}}%
\pgfpathmoveto{\pgfqpoint{0.906134in}{0.313809in}}%
\pgfpathlineto{\pgfqpoint{0.847494in}{0.422933in}}%
\pgfpathmoveto{\pgfqpoint{0.906134in}{0.313809in}}%
\pgfpathlineto{\pgfqpoint{1.010874in}{0.404326in}}%
\pgfpathmoveto{\pgfqpoint{0.906134in}{0.313809in}}%
\pgfpathlineto{\pgfqpoint{0.853768in}{0.184448in}}%
\pgfpathmoveto{\pgfqpoint{0.906134in}{0.313809in}}%
\pgfpathlineto{\pgfqpoint{1.048874in}{0.262157in}}%
\pgfpathmoveto{\pgfqpoint{0.906134in}{0.313809in}}%
\pgfpathlineto{\pgfqpoint{0.665666in}{0.289310in}}%
\pgfpathmoveto{\pgfqpoint{0.486241in}{0.090000in}}%
\pgfpathlineto{\pgfqpoint{0.480075in}{0.132151in}}%
\pgfpathlineto{\pgfqpoint{0.480075in}{0.132151in}}%
\pgfusepath{stroke}%
\end{pgfscope}%
\begin{pgfscope}%
\pgfpathrectangle{\pgfqpoint{0.100000in}{0.100000in}}{\pgfqpoint{1.782500in}{1.232000in}}%
\pgfusepath{clip}%
\pgfsetrectcap%
\pgfsetroundjoin%
\pgfsetlinewidth{0.250937pt}%
\definecolor{currentstroke}{rgb}{0.835294,0.321569,0.035294}%
\pgfsetstrokecolor{currentstroke}%
\pgfsetdash{}{0pt}%
\pgfpathmoveto{\pgfqpoint{0.781352in}{0.626027in}}%
\pgfpathlineto{\pgfqpoint{0.694167in}{0.980000in}}%
\pgfpathmoveto{\pgfqpoint{1.288333in}{0.980000in}}%
\pgfpathlineto{\pgfqpoint{1.882500in}{0.980000in}}%
\pgfpathmoveto{\pgfqpoint{1.288333in}{0.980000in}}%
\pgfpathlineto{\pgfqpoint{0.694167in}{0.980000in}}%
\pgfpathmoveto{\pgfqpoint{0.882540in}{0.514412in}}%
\pgfpathlineto{\pgfqpoint{0.781352in}{0.626027in}}%
\pgfpathmoveto{\pgfqpoint{1.011016in}{0.425896in}}%
\pgfpathlineto{\pgfqpoint{0.882540in}{0.514412in}}%
\pgfpathmoveto{\pgfqpoint{1.162956in}{0.369591in}}%
\pgfpathlineto{\pgfqpoint{1.011016in}{0.425896in}}%
\pgfpathmoveto{\pgfqpoint{1.352527in}{0.332850in}}%
\pgfpathlineto{\pgfqpoint{1.162956in}{0.369591in}}%
\pgfpathmoveto{\pgfqpoint{1.532394in}{0.307019in}}%
\pgfpathlineto{\pgfqpoint{1.352527in}{0.332850in}}%
\pgfpathmoveto{\pgfqpoint{1.706969in}{0.287629in}}%
\pgfpathlineto{\pgfqpoint{1.532394in}{0.307019in}}%
\pgfpathmoveto{\pgfqpoint{1.878573in}{0.284471in}}%
\pgfpathlineto{\pgfqpoint{1.706969in}{0.287629in}}%
\pgfpathmoveto{\pgfqpoint{1.892500in}{0.285491in}}%
\pgfpathlineto{\pgfqpoint{1.878573in}{0.284471in}}%
\pgfpathmoveto{\pgfqpoint{1.892500in}{0.980000in}}%
\pgfpathlineto{\pgfqpoint{1.882500in}{0.980000in}}%
\pgfpathmoveto{\pgfqpoint{1.527978in}{0.696263in}}%
\pgfpathlineto{\pgfqpoint{1.882500in}{0.980000in}}%
\pgfpathmoveto{\pgfqpoint{1.527978in}{0.696263in}}%
\pgfpathlineto{\pgfqpoint{1.288333in}{0.980000in}}%
\pgfpathmoveto{\pgfqpoint{1.812748in}{0.507634in}}%
\pgfpathlineto{\pgfqpoint{1.892500in}{0.530617in}}%
\pgfpathmoveto{\pgfqpoint{1.812748in}{0.507634in}}%
\pgfpathlineto{\pgfqpoint{1.527978in}{0.696263in}}%
\pgfpathmoveto{\pgfqpoint{1.162739in}{0.705979in}}%
\pgfpathlineto{\pgfqpoint{1.288333in}{0.980000in}}%
\pgfpathmoveto{\pgfqpoint{1.162739in}{0.705979in}}%
\pgfpathlineto{\pgfqpoint{1.527978in}{0.696263in}}%
\pgfpathmoveto{\pgfqpoint{1.892500in}{0.493384in}}%
\pgfpathlineto{\pgfqpoint{1.812748in}{0.507634in}}%
\pgfpathmoveto{\pgfqpoint{1.501881in}{0.470125in}}%
\pgfpathlineto{\pgfqpoint{1.352527in}{0.332850in}}%
\pgfpathmoveto{\pgfqpoint{1.501881in}{0.470125in}}%
\pgfpathlineto{\pgfqpoint{1.532394in}{0.307019in}}%
\pgfpathmoveto{\pgfqpoint{1.501881in}{0.470125in}}%
\pgfpathlineto{\pgfqpoint{1.527978in}{0.696263in}}%
\pgfpathmoveto{\pgfqpoint{1.501881in}{0.470125in}}%
\pgfpathlineto{\pgfqpoint{1.812748in}{0.507634in}}%
\pgfpathmoveto{\pgfqpoint{1.255997in}{0.543823in}}%
\pgfpathlineto{\pgfqpoint{1.011016in}{0.425896in}}%
\pgfpathmoveto{\pgfqpoint{1.255997in}{0.543823in}}%
\pgfpathlineto{\pgfqpoint{1.162956in}{0.369591in}}%
\pgfpathmoveto{\pgfqpoint{1.255997in}{0.543823in}}%
\pgfpathlineto{\pgfqpoint{1.527978in}{0.696263in}}%
\pgfpathmoveto{\pgfqpoint{1.255997in}{0.543823in}}%
\pgfpathlineto{\pgfqpoint{1.162739in}{0.705979in}}%
\pgfpathmoveto{\pgfqpoint{1.255997in}{0.543823in}}%
\pgfpathlineto{\pgfqpoint{1.501881in}{0.470125in}}%
\pgfpathmoveto{\pgfqpoint{1.798218in}{0.367589in}}%
\pgfpathlineto{\pgfqpoint{1.706969in}{0.287629in}}%
\pgfpathmoveto{\pgfqpoint{1.798218in}{0.367589in}}%
\pgfpathlineto{\pgfqpoint{1.878573in}{0.284471in}}%
\pgfpathmoveto{\pgfqpoint{1.798218in}{0.367589in}}%
\pgfpathlineto{\pgfqpoint{1.812748in}{0.507634in}}%
\pgfpathmoveto{\pgfqpoint{1.324350in}{0.439170in}}%
\pgfpathlineto{\pgfqpoint{1.162956in}{0.369591in}}%
\pgfpathmoveto{\pgfqpoint{1.324350in}{0.439170in}}%
\pgfpathlineto{\pgfqpoint{1.352527in}{0.332850in}}%
\pgfpathmoveto{\pgfqpoint{1.324350in}{0.439170in}}%
\pgfpathlineto{\pgfqpoint{1.501881in}{0.470125in}}%
\pgfpathmoveto{\pgfqpoint{1.324350in}{0.439170in}}%
\pgfpathlineto{\pgfqpoint{1.255997in}{0.543823in}}%
\pgfpathmoveto{\pgfqpoint{1.069670in}{0.552309in}}%
\pgfpathlineto{\pgfqpoint{0.882540in}{0.514412in}}%
\pgfpathmoveto{\pgfqpoint{1.069670in}{0.552309in}}%
\pgfpathlineto{\pgfqpoint{1.011016in}{0.425896in}}%
\pgfpathmoveto{\pgfqpoint{1.069670in}{0.552309in}}%
\pgfpathlineto{\pgfqpoint{1.162739in}{0.705979in}}%
\pgfpathmoveto{\pgfqpoint{1.069670in}{0.552309in}}%
\pgfpathlineto{\pgfqpoint{1.255997in}{0.543823in}}%
\pgfpathmoveto{\pgfqpoint{0.942149in}{0.807357in}}%
\pgfpathlineto{\pgfqpoint{0.694167in}{0.980000in}}%
\pgfpathmoveto{\pgfqpoint{0.942149in}{0.807357in}}%
\pgfpathlineto{\pgfqpoint{0.781352in}{0.626027in}}%
\pgfpathmoveto{\pgfqpoint{0.942149in}{0.807357in}}%
\pgfpathlineto{\pgfqpoint{1.288333in}{0.980000in}}%
\pgfpathmoveto{\pgfqpoint{0.942149in}{0.807357in}}%
\pgfpathlineto{\pgfqpoint{1.162739in}{0.705979in}}%
\pgfpathmoveto{\pgfqpoint{1.892500in}{0.300161in}}%
\pgfpathlineto{\pgfqpoint{1.878573in}{0.284471in}}%
\pgfpathmoveto{\pgfqpoint{1.892500in}{0.433533in}}%
\pgfpathlineto{\pgfqpoint{1.812748in}{0.507634in}}%
\pgfpathmoveto{\pgfqpoint{1.892500in}{0.370945in}}%
\pgfpathlineto{\pgfqpoint{1.798218in}{0.367589in}}%
\pgfpathmoveto{\pgfqpoint{1.639440in}{0.380185in}}%
\pgfpathlineto{\pgfqpoint{1.532394in}{0.307019in}}%
\pgfpathmoveto{\pgfqpoint{1.639440in}{0.380185in}}%
\pgfpathlineto{\pgfqpoint{1.706969in}{0.287629in}}%
\pgfpathmoveto{\pgfqpoint{1.639440in}{0.380185in}}%
\pgfpathlineto{\pgfqpoint{1.812748in}{0.507634in}}%
\pgfpathmoveto{\pgfqpoint{1.639440in}{0.380185in}}%
\pgfpathlineto{\pgfqpoint{1.501881in}{0.470125in}}%
\pgfpathmoveto{\pgfqpoint{1.639440in}{0.380185in}}%
\pgfpathlineto{\pgfqpoint{1.798218in}{0.367589in}}%
\pgfpathmoveto{\pgfqpoint{1.892500in}{0.931265in}}%
\pgfpathlineto{\pgfqpoint{1.882500in}{0.980000in}}%
\pgfpathmoveto{\pgfqpoint{1.892500in}{0.720254in}}%
\pgfpathlineto{\pgfqpoint{1.527978in}{0.696263in}}%
\pgfpathmoveto{\pgfqpoint{1.892500in}{0.647919in}}%
\pgfpathlineto{\pgfqpoint{1.812748in}{0.507634in}}%
\pgfpathmoveto{\pgfqpoint{1.892500in}{0.975332in}}%
\pgfpathlineto{\pgfqpoint{1.882500in}{0.980000in}}%
\pgfpathmoveto{\pgfqpoint{0.964594in}{0.641001in}}%
\pgfpathlineto{\pgfqpoint{0.781352in}{0.626027in}}%
\pgfpathmoveto{\pgfqpoint{0.964594in}{0.641001in}}%
\pgfpathlineto{\pgfqpoint{0.882540in}{0.514412in}}%
\pgfpathmoveto{\pgfqpoint{0.964594in}{0.641001in}}%
\pgfpathlineto{\pgfqpoint{1.162739in}{0.705979in}}%
\pgfpathmoveto{\pgfqpoint{0.964594in}{0.641001in}}%
\pgfpathlineto{\pgfqpoint{1.069670in}{0.552309in}}%
\pgfpathmoveto{\pgfqpoint{0.964594in}{0.641001in}}%
\pgfpathlineto{\pgfqpoint{0.942149in}{0.807357in}}%
\pgfpathlineto{\pgfqpoint{0.942149in}{0.807357in}}%
\pgfusepath{stroke}%
\end{pgfscope}%
\begin{pgfscope}%
\pgfpathrectangle{\pgfqpoint{0.100000in}{0.100000in}}{\pgfqpoint{1.782500in}{1.232000in}}%
\pgfusepath{clip}%
\pgfsetbuttcap%
\pgfsetroundjoin%
\definecolor{currentfill}{rgb}{0.054902,0.262745,0.486275}%
\pgfsetfillcolor{currentfill}%
\pgfsetlinewidth{1.003750pt}%
\definecolor{currentstroke}{rgb}{0.054902,0.262745,0.486275}%
\pgfsetstrokecolor{currentstroke}%
\pgfsetdash{}{0pt}%
\pgfsys@defobject{currentmarker}{\pgfqpoint{-0.018373in}{-0.018373in}}{\pgfqpoint{0.018373in}{0.018373in}}{%
\pgfpathmoveto{\pgfqpoint{0.000000in}{-0.018373in}}%
\pgfpathcurveto{\pgfqpoint{0.004873in}{-0.018373in}}{\pgfqpoint{0.009546in}{-0.016437in}}{\pgfqpoint{0.012992in}{-0.012992in}}%
\pgfpathcurveto{\pgfqpoint{0.016437in}{-0.009546in}}{\pgfqpoint{0.018373in}{-0.004873in}}{\pgfqpoint{0.018373in}{0.000000in}}%
\pgfpathcurveto{\pgfqpoint{0.018373in}{0.004873in}}{\pgfqpoint{0.016437in}{0.009546in}}{\pgfqpoint{0.012992in}{0.012992in}}%
\pgfpathcurveto{\pgfqpoint{0.009546in}{0.016437in}}{\pgfqpoint{0.004873in}{0.018373in}}{\pgfqpoint{0.000000in}{0.018373in}}%
\pgfpathcurveto{\pgfqpoint{-0.004873in}{0.018373in}}{\pgfqpoint{-0.009546in}{0.016437in}}{\pgfqpoint{-0.012992in}{0.012992in}}%
\pgfpathcurveto{\pgfqpoint{-0.016437in}{0.009546in}}{\pgfqpoint{-0.018373in}{0.004873in}}{\pgfqpoint{-0.018373in}{0.000000in}}%
\pgfpathcurveto{\pgfqpoint{-0.018373in}{-0.004873in}}{\pgfqpoint{-0.016437in}{-0.009546in}}{\pgfqpoint{-0.012992in}{-0.012992in}}%
\pgfpathcurveto{\pgfqpoint{-0.009546in}{-0.016437in}}{\pgfqpoint{-0.004873in}{-0.018373in}}{\pgfqpoint{0.000000in}{-0.018373in}}%
\pgfpathlineto{\pgfqpoint{0.000000in}{-0.018373in}}%
\pgfpathclose%
\pgfusepath{stroke,fill}%
}%
\begin{pgfscope}%
\pgfsys@transformshift{2.586520in}{0.372423in}%
\pgfsys@useobject{currentmarker}{}%
\end{pgfscope}%
\begin{pgfscope}%
\pgfsys@transformshift{2.459881in}{0.346407in}%
\pgfsys@useobject{currentmarker}{}%
\end{pgfscope}%
\begin{pgfscope}%
\pgfsys@transformshift{2.316297in}{0.322858in}%
\pgfsys@useobject{currentmarker}{}%
\end{pgfscope}%
\begin{pgfscope}%
\pgfsys@transformshift{2.182602in}{0.308565in}%
\pgfsys@useobject{currentmarker}{}%
\end{pgfscope}%
\begin{pgfscope}%
\pgfsys@transformshift{2.029678in}{0.295285in}%
\pgfsys@useobject{currentmarker}{}%
\end{pgfscope}%
\begin{pgfscope}%
\pgfsys@transformshift{1.878506in}{0.284816in}%
\pgfsys@useobject{currentmarker}{}%
\end{pgfscope}%
\begin{pgfscope}%
\pgfsys@transformshift{1.726252in}{0.286703in}%
\pgfsys@useobject{currentmarker}{}%
\end{pgfscope}%
\begin{pgfscope}%
\pgfsys@transformshift{1.575612in}{0.301602in}%
\pgfsys@useobject{currentmarker}{}%
\end{pgfscope}%
\begin{pgfscope}%
\pgfsys@transformshift{1.443371in}{0.318705in}%
\pgfsys@useobject{currentmarker}{}%
\end{pgfscope}%
\begin{pgfscope}%
\pgfsys@transformshift{1.299293in}{0.342555in}%
\pgfsys@useobject{currentmarker}{}%
\end{pgfscope}%
\begin{pgfscope}%
\pgfsys@transformshift{1.171810in}{0.367799in}%
\pgfsys@useobject{currentmarker}{}%
\end{pgfscope}%
\end{pgfscope}%
\begin{pgfscope}%
\pgfpathrectangle{\pgfqpoint{0.100000in}{0.100000in}}{\pgfqpoint{1.782500in}{1.232000in}}%
\pgfusepath{clip}%
\pgfsetbuttcap%
\pgfsetroundjoin%
\definecolor{currentfill}{rgb}{0.835294,0.321569,0.035294}%
\pgfsetfillcolor{currentfill}%
\pgfsetlinewidth{1.003750pt}%
\definecolor{currentstroke}{rgb}{0.835294,0.321569,0.035294}%
\pgfsetstrokecolor{currentstroke}%
\pgfsetdash{}{0pt}%
\pgfsys@defobject{currentmarker}{\pgfqpoint{-0.018373in}{-0.018373in}}{\pgfqpoint{0.018373in}{0.018373in}}{%
\pgfpathmoveto{\pgfqpoint{0.000000in}{-0.018373in}}%
\pgfpathcurveto{\pgfqpoint{0.004873in}{-0.018373in}}{\pgfqpoint{0.009546in}{-0.016437in}}{\pgfqpoint{0.012992in}{-0.012992in}}%
\pgfpathcurveto{\pgfqpoint{0.016437in}{-0.009546in}}{\pgfqpoint{0.018373in}{-0.004873in}}{\pgfqpoint{0.018373in}{0.000000in}}%
\pgfpathcurveto{\pgfqpoint{0.018373in}{0.004873in}}{\pgfqpoint{0.016437in}{0.009546in}}{\pgfqpoint{0.012992in}{0.012992in}}%
\pgfpathcurveto{\pgfqpoint{0.009546in}{0.016437in}}{\pgfqpoint{0.004873in}{0.018373in}}{\pgfqpoint{0.000000in}{0.018373in}}%
\pgfpathcurveto{\pgfqpoint{-0.004873in}{0.018373in}}{\pgfqpoint{-0.009546in}{0.016437in}}{\pgfqpoint{-0.012992in}{0.012992in}}%
\pgfpathcurveto{\pgfqpoint{-0.016437in}{0.009546in}}{\pgfqpoint{-0.018373in}{0.004873in}}{\pgfqpoint{-0.018373in}{0.000000in}}%
\pgfpathcurveto{\pgfqpoint{-0.018373in}{-0.004873in}}{\pgfqpoint{-0.016437in}{-0.009546in}}{\pgfqpoint{-0.012992in}{-0.012992in}}%
\pgfpathcurveto{\pgfqpoint{-0.009546in}{-0.016437in}}{\pgfqpoint{-0.004873in}{-0.018373in}}{\pgfqpoint{0.000000in}{-0.018373in}}%
\pgfpathlineto{\pgfqpoint{0.000000in}{-0.018373in}}%
\pgfpathclose%
\pgfusepath{stroke,fill}%
}%
\begin{pgfscope}%
\pgfsys@transformshift{1.162956in}{0.369591in}%
\pgfsys@useobject{currentmarker}{}%
\end{pgfscope}%
\begin{pgfscope}%
\pgfsys@transformshift{1.352527in}{0.332850in}%
\pgfsys@useobject{currentmarker}{}%
\end{pgfscope}%
\begin{pgfscope}%
\pgfsys@transformshift{1.532394in}{0.307019in}%
\pgfsys@useobject{currentmarker}{}%
\end{pgfscope}%
\begin{pgfscope}%
\pgfsys@transformshift{1.706969in}{0.287629in}%
\pgfsys@useobject{currentmarker}{}%
\end{pgfscope}%
\begin{pgfscope}%
\pgfsys@transformshift{1.878573in}{0.284471in}%
\pgfsys@useobject{currentmarker}{}%
\end{pgfscope}%
\begin{pgfscope}%
\pgfsys@transformshift{2.048845in}{0.296951in}%
\pgfsys@useobject{currentmarker}{}%
\end{pgfscope}%
\begin{pgfscope}%
\pgfsys@transformshift{2.226694in}{0.312675in}%
\pgfsys@useobject{currentmarker}{}%
\end{pgfscope}%
\begin{pgfscope}%
\pgfsys@transformshift{2.406928in}{0.336579in}%
\pgfsys@useobject{currentmarker}{}%
\end{pgfscope}%
\begin{pgfscope}%
\pgfsys@transformshift{2.595312in}{0.374279in}%
\pgfsys@useobject{currentmarker}{}%
\end{pgfscope}%
\end{pgfscope}%
\end{pgfpicture}%
\makeatother%
\endgroup%
}
        \caption{Iteration 3: Solve system}\label{fig:example-iter2-solution}
    \end{subfigure}
    \begin{subfigure}[b]{.32\linewidth}
        \scalebox{0.8}{%% Creator: Matplotlib, PGF backend
%%
%% To include the figure in your LaTeX document, write
%%   \input{<filename>.pgf}
%%
%% Make sure the required packages are loaded in your preamble
%%   \usepackage{pgf}
%%
%% Also ensure that all the required font packages are loaded; for instance,
%% the lmodern package is sometimes necessary when using math font.
%%   \usepackage{lmodern}
%%
%% Figures using additional raster images can only be included by \input if
%% they are in the same directory as the main LaTeX file. For loading figures
%% from other directories you can use the `import` package
%%   \usepackage{import}
%%
%% and then include the figures with
%%   \import{<path to file>}{<filename>.pgf}
%%
%% Matplotlib used the following preamble
%%   
%%   \usepackage{fontspec}
%%   \setmainfont{DejaVuSans.ttf}[Path=\detokenize{/home/fabio/Internodes-CM/.venv/lib/python3.8/site-packages/matplotlib/mpl-data/fonts/ttf/}]
%%   \setsansfont{DejaVuSans.ttf}[Path=\detokenize{/home/fabio/Internodes-CM/.venv/lib/python3.8/site-packages/matplotlib/mpl-data/fonts/ttf/}]
%%   \setmonofont{DejaVuSansMono.ttf}[Path=\detokenize{/home/fabio/Internodes-CM/.venv/lib/python3.8/site-packages/matplotlib/mpl-data/fonts/ttf/}]
%%   \makeatletter\@ifpackageloaded{underscore}{}{\usepackage[strings]{underscore}}\makeatother
%%
\begingroup%
\makeatletter%
\begin{pgfpicture}%
\pgfpathrectangle{\pgfpointorigin}{\pgfqpoint{1.982500in}{1.432000in}}%
\pgfusepath{use as bounding box, clip}%
\begin{pgfscope}%
\pgfsetbuttcap%
\pgfsetmiterjoin%
\definecolor{currentfill}{rgb}{1.000000,1.000000,1.000000}%
\pgfsetfillcolor{currentfill}%
\pgfsetlinewidth{0.000000pt}%
\definecolor{currentstroke}{rgb}{1.000000,1.000000,1.000000}%
\pgfsetstrokecolor{currentstroke}%
\pgfsetdash{}{0pt}%
\pgfpathmoveto{\pgfqpoint{0.000000in}{0.000000in}}%
\pgfpathlineto{\pgfqpoint{1.982500in}{0.000000in}}%
\pgfpathlineto{\pgfqpoint{1.982500in}{1.432000in}}%
\pgfpathlineto{\pgfqpoint{0.000000in}{1.432000in}}%
\pgfpathlineto{\pgfqpoint{0.000000in}{0.000000in}}%
\pgfpathclose%
\pgfusepath{fill}%
\end{pgfscope}%
\begin{pgfscope}%
\pgfpathrectangle{\pgfqpoint{0.100000in}{0.100000in}}{\pgfqpoint{1.782500in}{1.232000in}}%
\pgfusepath{clip}%
\pgfsetrectcap%
\pgfsetroundjoin%
\pgfsetlinewidth{0.250937pt}%
\definecolor{currentstroke}{rgb}{0.054902,0.262745,0.486275}%
\pgfsetstrokecolor{currentstroke}%
\pgfsetdash{}{0pt}%
\pgfpathmoveto{\pgfqpoint{0.454502in}{0.100000in}}%
\pgfpathlineto{\pgfqpoint{0.185788in}{0.100000in}}%
\pgfpathmoveto{\pgfqpoint{0.723215in}{0.100000in}}%
\pgfpathlineto{\pgfqpoint{0.454502in}{0.100000in}}%
\pgfpathmoveto{\pgfqpoint{0.991929in}{0.100000in}}%
\pgfpathlineto{\pgfqpoint{0.723215in}{0.100000in}}%
\pgfpathmoveto{\pgfqpoint{1.260642in}{0.100000in}}%
\pgfpathlineto{\pgfqpoint{0.991929in}{0.100000in}}%
\pgfpathmoveto{\pgfqpoint{1.529356in}{0.100000in}}%
\pgfpathlineto{\pgfqpoint{1.798069in}{0.100000in}}%
\pgfpathmoveto{\pgfqpoint{1.529356in}{0.100000in}}%
\pgfpathlineto{\pgfqpoint{1.260642in}{0.100000in}}%
\pgfpathmoveto{\pgfqpoint{1.801477in}{0.408760in}}%
\pgfpathlineto{\pgfqpoint{1.798069in}{0.100000in}}%
\pgfpathmoveto{\pgfqpoint{1.801477in}{0.408760in}}%
\pgfpathlineto{\pgfqpoint{1.787931in}{0.719684in}}%
\pgfpathmoveto{\pgfqpoint{1.625404in}{0.710683in}}%
\pgfpathlineto{\pgfqpoint{1.787931in}{0.719684in}}%
\pgfpathmoveto{\pgfqpoint{1.625404in}{0.710683in}}%
\pgfpathlineto{\pgfqpoint{1.458297in}{0.696339in}}%
\pgfpathmoveto{\pgfqpoint{1.385143in}{0.684752in}}%
\pgfpathlineto{\pgfqpoint{1.458297in}{0.696339in}}%
\pgfpathmoveto{\pgfqpoint{1.310324in}{0.660296in}}%
\pgfpathlineto{\pgfqpoint{1.385143in}{0.684752in}}%
\pgfpathmoveto{\pgfqpoint{1.253051in}{0.642085in}}%
\pgfpathlineto{\pgfqpoint{1.310324in}{0.660296in}}%
\pgfpathmoveto{\pgfqpoint{1.188115in}{0.625601in}}%
\pgfpathlineto{\pgfqpoint{1.253051in}{0.642085in}}%
\pgfpathmoveto{\pgfqpoint{1.127651in}{0.615595in}}%
\pgfpathlineto{\pgfqpoint{1.188115in}{0.625601in}}%
\pgfpathmoveto{\pgfqpoint{1.058490in}{0.606300in}}%
\pgfpathlineto{\pgfqpoint{1.127651in}{0.615595in}}%
\pgfpathmoveto{\pgfqpoint{0.990122in}{0.598971in}}%
\pgfpathlineto{\pgfqpoint{1.058490in}{0.606300in}}%
\pgfpathmoveto{\pgfqpoint{0.921265in}{0.600292in}}%
\pgfpathlineto{\pgfqpoint{0.990122in}{0.598971in}}%
\pgfpathmoveto{\pgfqpoint{0.853138in}{0.610721in}}%
\pgfpathlineto{\pgfqpoint{0.921265in}{0.600292in}}%
\pgfpathmoveto{\pgfqpoint{0.793331in}{0.622693in}}%
\pgfpathlineto{\pgfqpoint{0.853138in}{0.610721in}}%
\pgfpathmoveto{\pgfqpoint{0.728172in}{0.639388in}}%
\pgfpathlineto{\pgfqpoint{0.793331in}{0.622693in}}%
\pgfpathmoveto{\pgfqpoint{0.670517in}{0.657059in}}%
\pgfpathlineto{\pgfqpoint{0.728172in}{0.639388in}}%
\pgfpathmoveto{\pgfqpoint{0.597734in}{0.682628in}}%
\pgfpathlineto{\pgfqpoint{0.523844in}{0.695653in}}%
\pgfpathmoveto{\pgfqpoint{0.597734in}{0.682628in}}%
\pgfpathlineto{\pgfqpoint{0.670517in}{0.657059in}}%
\pgfpathmoveto{\pgfqpoint{0.357342in}{0.710776in}}%
\pgfpathlineto{\pgfqpoint{0.523844in}{0.695653in}}%
\pgfpathmoveto{\pgfqpoint{0.357342in}{0.710776in}}%
\pgfpathlineto{\pgfqpoint{0.194832in}{0.720303in}}%
\pgfpathmoveto{\pgfqpoint{0.181023in}{0.408839in}}%
\pgfpathlineto{\pgfqpoint{0.185788in}{0.100000in}}%
\pgfpathmoveto{\pgfqpoint{0.181023in}{0.408839in}}%
\pgfpathlineto{\pgfqpoint{0.194832in}{0.720303in}}%
\pgfpathmoveto{\pgfqpoint{1.380079in}{0.372824in}}%
\pgfpathlineto{\pgfqpoint{1.260642in}{0.100000in}}%
\pgfpathmoveto{\pgfqpoint{1.085607in}{0.353637in}}%
\pgfpathlineto{\pgfqpoint{0.991929in}{0.100000in}}%
\pgfpathmoveto{\pgfqpoint{1.236571in}{0.465285in}}%
\pgfpathlineto{\pgfqpoint{1.380079in}{0.372824in}}%
\pgfpathmoveto{\pgfqpoint{1.236571in}{0.465285in}}%
\pgfpathlineto{\pgfqpoint{1.085607in}{0.353637in}}%
\pgfpathmoveto{\pgfqpoint{0.955470in}{0.447258in}}%
\pgfpathlineto{\pgfqpoint{1.085607in}{0.353637in}}%
\pgfpathmoveto{\pgfqpoint{0.955470in}{0.447258in}}%
\pgfpathlineto{\pgfqpoint{0.816937in}{0.367970in}}%
\pgfpathmoveto{\pgfqpoint{0.675550in}{0.476601in}}%
\pgfpathlineto{\pgfqpoint{0.816937in}{0.367970in}}%
\pgfpathmoveto{\pgfqpoint{0.675550in}{0.476601in}}%
\pgfpathlineto{\pgfqpoint{0.534719in}{0.383741in}}%
\pgfpathmoveto{\pgfqpoint{1.535916in}{0.486155in}}%
\pgfpathlineto{\pgfqpoint{1.801477in}{0.408760in}}%
\pgfpathmoveto{\pgfqpoint{1.535916in}{0.486155in}}%
\pgfpathlineto{\pgfqpoint{1.380079in}{0.372824in}}%
\pgfpathmoveto{\pgfqpoint{1.376985in}{0.517176in}}%
\pgfpathlineto{\pgfqpoint{1.380079in}{0.372824in}}%
\pgfpathmoveto{\pgfqpoint{1.376985in}{0.517176in}}%
\pgfpathlineto{\pgfqpoint{1.236571in}{0.465285in}}%
\pgfpathmoveto{\pgfqpoint{1.376985in}{0.517176in}}%
\pgfpathlineto{\pgfqpoint{1.535916in}{0.486155in}}%
\pgfpathmoveto{\pgfqpoint{1.093931in}{0.479746in}}%
\pgfpathlineto{\pgfqpoint{1.085607in}{0.353637in}}%
\pgfpathmoveto{\pgfqpoint{1.093931in}{0.479746in}}%
\pgfpathlineto{\pgfqpoint{1.236571in}{0.465285in}}%
\pgfpathmoveto{\pgfqpoint{1.093931in}{0.479746in}}%
\pgfpathlineto{\pgfqpoint{0.955470in}{0.447258in}}%
\pgfpathmoveto{\pgfqpoint{0.816759in}{0.485778in}}%
\pgfpathlineto{\pgfqpoint{0.816937in}{0.367970in}}%
\pgfpathmoveto{\pgfqpoint{0.816759in}{0.485778in}}%
\pgfpathlineto{\pgfqpoint{0.955470in}{0.447258in}}%
\pgfpathmoveto{\pgfqpoint{0.816759in}{0.485778in}}%
\pgfpathlineto{\pgfqpoint{0.675550in}{0.476601in}}%
\pgfpathmoveto{\pgfqpoint{0.526682in}{0.528714in}}%
\pgfpathlineto{\pgfqpoint{0.534719in}{0.383741in}}%
\pgfpathmoveto{\pgfqpoint{0.526682in}{0.528714in}}%
\pgfpathlineto{\pgfqpoint{0.675550in}{0.476601in}}%
\pgfpathmoveto{\pgfqpoint{0.357678in}{0.492105in}}%
\pgfpathlineto{\pgfqpoint{0.194832in}{0.720303in}}%
\pgfpathmoveto{\pgfqpoint{0.357678in}{0.492105in}}%
\pgfpathlineto{\pgfqpoint{0.357342in}{0.710776in}}%
\pgfpathmoveto{\pgfqpoint{0.357678in}{0.492105in}}%
\pgfpathlineto{\pgfqpoint{0.181023in}{0.408839in}}%
\pgfpathmoveto{\pgfqpoint{0.357678in}{0.492105in}}%
\pgfpathlineto{\pgfqpoint{0.534719in}{0.383741in}}%
\pgfpathmoveto{\pgfqpoint{0.357678in}{0.492105in}}%
\pgfpathlineto{\pgfqpoint{0.526682in}{0.528714in}}%
\pgfpathmoveto{\pgfqpoint{1.298659in}{0.566272in}}%
\pgfpathlineto{\pgfqpoint{1.310324in}{0.660296in}}%
\pgfpathmoveto{\pgfqpoint{1.298659in}{0.566272in}}%
\pgfpathlineto{\pgfqpoint{1.253051in}{0.642085in}}%
\pgfpathmoveto{\pgfqpoint{1.298659in}{0.566272in}}%
\pgfpathlineto{\pgfqpoint{1.236571in}{0.465285in}}%
\pgfpathmoveto{\pgfqpoint{1.298659in}{0.566272in}}%
\pgfpathlineto{\pgfqpoint{1.376985in}{0.517176in}}%
\pgfpathmoveto{\pgfqpoint{1.163096in}{0.542097in}}%
\pgfpathlineto{\pgfqpoint{1.188115in}{0.625601in}}%
\pgfpathmoveto{\pgfqpoint{1.163096in}{0.542097in}}%
\pgfpathlineto{\pgfqpoint{1.127651in}{0.615595in}}%
\pgfpathmoveto{\pgfqpoint{1.163096in}{0.542097in}}%
\pgfpathlineto{\pgfqpoint{1.236571in}{0.465285in}}%
\pgfpathmoveto{\pgfqpoint{1.163096in}{0.542097in}}%
\pgfpathlineto{\pgfqpoint{1.093931in}{0.479746in}}%
\pgfpathmoveto{\pgfqpoint{1.026137in}{0.529850in}}%
\pgfpathlineto{\pgfqpoint{1.058490in}{0.606300in}}%
\pgfpathmoveto{\pgfqpoint{1.026137in}{0.529850in}}%
\pgfpathlineto{\pgfqpoint{0.990122in}{0.598971in}}%
\pgfpathmoveto{\pgfqpoint{1.026137in}{0.529850in}}%
\pgfpathlineto{\pgfqpoint{0.955470in}{0.447258in}}%
\pgfpathmoveto{\pgfqpoint{1.026137in}{0.529850in}}%
\pgfpathlineto{\pgfqpoint{1.093931in}{0.479746in}}%
\pgfpathmoveto{\pgfqpoint{0.886661in}{0.531913in}}%
\pgfpathlineto{\pgfqpoint{0.921265in}{0.600292in}}%
\pgfpathmoveto{\pgfqpoint{0.886661in}{0.531913in}}%
\pgfpathlineto{\pgfqpoint{0.853138in}{0.610721in}}%
\pgfpathmoveto{\pgfqpoint{0.886661in}{0.531913in}}%
\pgfpathlineto{\pgfqpoint{0.955470in}{0.447258in}}%
\pgfpathmoveto{\pgfqpoint{0.886661in}{0.531913in}}%
\pgfpathlineto{\pgfqpoint{0.816759in}{0.485778in}}%
\pgfpathmoveto{\pgfqpoint{0.750901in}{0.550725in}}%
\pgfpathlineto{\pgfqpoint{0.793331in}{0.622693in}}%
\pgfpathmoveto{\pgfqpoint{0.750901in}{0.550725in}}%
\pgfpathlineto{\pgfqpoint{0.728172in}{0.639388in}}%
\pgfpathmoveto{\pgfqpoint{0.750901in}{0.550725in}}%
\pgfpathlineto{\pgfqpoint{0.675550in}{0.476601in}}%
\pgfpathmoveto{\pgfqpoint{0.750901in}{0.550725in}}%
\pgfpathlineto{\pgfqpoint{0.816759in}{0.485778in}}%
\pgfpathmoveto{\pgfqpoint{0.614919in}{0.583110in}}%
\pgfpathlineto{\pgfqpoint{0.670517in}{0.657059in}}%
\pgfpathmoveto{\pgfqpoint{0.614919in}{0.583110in}}%
\pgfpathlineto{\pgfqpoint{0.597734in}{0.682628in}}%
\pgfpathmoveto{\pgfqpoint{0.614919in}{0.583110in}}%
\pgfpathlineto{\pgfqpoint{0.675550in}{0.476601in}}%
\pgfpathmoveto{\pgfqpoint{0.614919in}{0.583110in}}%
\pgfpathlineto{\pgfqpoint{0.526682in}{0.528714in}}%
\pgfpathmoveto{\pgfqpoint{1.445323in}{0.602316in}}%
\pgfpathlineto{\pgfqpoint{1.458297in}{0.696339in}}%
\pgfpathmoveto{\pgfqpoint{1.445323in}{0.602316in}}%
\pgfpathlineto{\pgfqpoint{1.385143in}{0.684752in}}%
\pgfpathmoveto{\pgfqpoint{1.445323in}{0.602316in}}%
\pgfpathlineto{\pgfqpoint{1.535916in}{0.486155in}}%
\pgfpathmoveto{\pgfqpoint{1.445323in}{0.602316in}}%
\pgfpathlineto{\pgfqpoint{1.376985in}{0.517176in}}%
\pgfpathmoveto{\pgfqpoint{0.938998in}{0.293567in}}%
\pgfpathlineto{\pgfqpoint{0.991929in}{0.100000in}}%
\pgfpathmoveto{\pgfqpoint{0.938998in}{0.293567in}}%
\pgfpathlineto{\pgfqpoint{1.085607in}{0.353637in}}%
\pgfpathmoveto{\pgfqpoint{0.938998in}{0.293567in}}%
\pgfpathlineto{\pgfqpoint{0.816937in}{0.367970in}}%
\pgfpathmoveto{\pgfqpoint{0.938998in}{0.293567in}}%
\pgfpathlineto{\pgfqpoint{0.955470in}{0.447258in}}%
\pgfpathmoveto{\pgfqpoint{0.441612in}{0.602117in}}%
\pgfpathlineto{\pgfqpoint{0.523844in}{0.695653in}}%
\pgfpathmoveto{\pgfqpoint{0.441612in}{0.602117in}}%
\pgfpathlineto{\pgfqpoint{0.357342in}{0.710776in}}%
\pgfpathmoveto{\pgfqpoint{0.441612in}{0.602117in}}%
\pgfpathlineto{\pgfqpoint{0.526682in}{0.528714in}}%
\pgfpathmoveto{\pgfqpoint{0.441612in}{0.602117in}}%
\pgfpathlineto{\pgfqpoint{0.357678in}{0.492105in}}%
\pgfpathmoveto{\pgfqpoint{1.364219in}{0.606819in}}%
\pgfpathlineto{\pgfqpoint{1.385143in}{0.684752in}}%
\pgfpathmoveto{\pgfqpoint{1.364219in}{0.606819in}}%
\pgfpathlineto{\pgfqpoint{1.310324in}{0.660296in}}%
\pgfpathmoveto{\pgfqpoint{1.364219in}{0.606819in}}%
\pgfpathlineto{\pgfqpoint{1.376985in}{0.517176in}}%
\pgfpathmoveto{\pgfqpoint{1.364219in}{0.606819in}}%
\pgfpathlineto{\pgfqpoint{1.298659in}{0.566272in}}%
\pgfpathmoveto{\pgfqpoint{1.364219in}{0.606819in}}%
\pgfpathlineto{\pgfqpoint{1.445323in}{0.602316in}}%
\pgfpathmoveto{\pgfqpoint{1.094158in}{0.554638in}}%
\pgfpathlineto{\pgfqpoint{1.127651in}{0.615595in}}%
\pgfpathmoveto{\pgfqpoint{1.094158in}{0.554638in}}%
\pgfpathlineto{\pgfqpoint{1.058490in}{0.606300in}}%
\pgfpathmoveto{\pgfqpoint{1.094158in}{0.554638in}}%
\pgfpathlineto{\pgfqpoint{1.093931in}{0.479746in}}%
\pgfpathmoveto{\pgfqpoint{1.094158in}{0.554638in}}%
\pgfpathlineto{\pgfqpoint{1.163096in}{0.542097in}}%
\pgfpathmoveto{\pgfqpoint{1.094158in}{0.554638in}}%
\pgfpathlineto{\pgfqpoint{1.026137in}{0.529850in}}%
\pgfpathmoveto{\pgfqpoint{1.228887in}{0.571665in}}%
\pgfpathlineto{\pgfqpoint{1.253051in}{0.642085in}}%
\pgfpathmoveto{\pgfqpoint{1.228887in}{0.571665in}}%
\pgfpathlineto{\pgfqpoint{1.188115in}{0.625601in}}%
\pgfpathmoveto{\pgfqpoint{1.228887in}{0.571665in}}%
\pgfpathlineto{\pgfqpoint{1.236571in}{0.465285in}}%
\pgfpathmoveto{\pgfqpoint{1.228887in}{0.571665in}}%
\pgfpathlineto{\pgfqpoint{1.298659in}{0.566272in}}%
\pgfpathmoveto{\pgfqpoint{1.228887in}{0.571665in}}%
\pgfpathlineto{\pgfqpoint{1.163096in}{0.542097in}}%
\pgfpathmoveto{\pgfqpoint{0.956125in}{0.541001in}}%
\pgfpathlineto{\pgfqpoint{0.990122in}{0.598971in}}%
\pgfpathmoveto{\pgfqpoint{0.956125in}{0.541001in}}%
\pgfpathlineto{\pgfqpoint{0.921265in}{0.600292in}}%
\pgfpathmoveto{\pgfqpoint{0.956125in}{0.541001in}}%
\pgfpathlineto{\pgfqpoint{0.955470in}{0.447258in}}%
\pgfpathmoveto{\pgfqpoint{0.956125in}{0.541001in}}%
\pgfpathlineto{\pgfqpoint{1.026137in}{0.529850in}}%
\pgfpathmoveto{\pgfqpoint{0.956125in}{0.541001in}}%
\pgfpathlineto{\pgfqpoint{0.886661in}{0.531913in}}%
\pgfpathmoveto{\pgfqpoint{0.819642in}{0.557313in}}%
\pgfpathlineto{\pgfqpoint{0.853138in}{0.610721in}}%
\pgfpathmoveto{\pgfqpoint{0.819642in}{0.557313in}}%
\pgfpathlineto{\pgfqpoint{0.793331in}{0.622693in}}%
\pgfpathmoveto{\pgfqpoint{0.819642in}{0.557313in}}%
\pgfpathlineto{\pgfqpoint{0.816759in}{0.485778in}}%
\pgfpathmoveto{\pgfqpoint{0.819642in}{0.557313in}}%
\pgfpathlineto{\pgfqpoint{0.886661in}{0.531913in}}%
\pgfpathmoveto{\pgfqpoint{0.819642in}{0.557313in}}%
\pgfpathlineto{\pgfqpoint{0.750901in}{0.550725in}}%
\pgfpathmoveto{\pgfqpoint{1.193096in}{0.276126in}}%
\pgfpathlineto{\pgfqpoint{0.991929in}{0.100000in}}%
\pgfpathmoveto{\pgfqpoint{1.193096in}{0.276126in}}%
\pgfpathlineto{\pgfqpoint{1.260642in}{0.100000in}}%
\pgfpathmoveto{\pgfqpoint{1.193096in}{0.276126in}}%
\pgfpathlineto{\pgfqpoint{1.380079in}{0.372824in}}%
\pgfpathmoveto{\pgfqpoint{1.193096in}{0.276126in}}%
\pgfpathlineto{\pgfqpoint{1.085607in}{0.353637in}}%
\pgfpathmoveto{\pgfqpoint{1.193096in}{0.276126in}}%
\pgfpathlineto{\pgfqpoint{1.236571in}{0.465285in}}%
\pgfpathmoveto{\pgfqpoint{0.690674in}{0.289421in}}%
\pgfpathlineto{\pgfqpoint{0.723215in}{0.100000in}}%
\pgfpathmoveto{\pgfqpoint{0.690674in}{0.289421in}}%
\pgfpathlineto{\pgfqpoint{0.816937in}{0.367970in}}%
\pgfpathmoveto{\pgfqpoint{0.690674in}{0.289421in}}%
\pgfpathlineto{\pgfqpoint{0.534719in}{0.383741in}}%
\pgfpathmoveto{\pgfqpoint{0.690674in}{0.289421in}}%
\pgfpathlineto{\pgfqpoint{0.675550in}{0.476601in}}%
\pgfpathmoveto{\pgfqpoint{0.686113in}{0.579464in}}%
\pgfpathlineto{\pgfqpoint{0.728172in}{0.639388in}}%
\pgfpathmoveto{\pgfqpoint{0.686113in}{0.579464in}}%
\pgfpathlineto{\pgfqpoint{0.670517in}{0.657059in}}%
\pgfpathmoveto{\pgfqpoint{0.686113in}{0.579464in}}%
\pgfpathlineto{\pgfqpoint{0.675550in}{0.476601in}}%
\pgfpathmoveto{\pgfqpoint{0.686113in}{0.579464in}}%
\pgfpathlineto{\pgfqpoint{0.750901in}{0.550725in}}%
\pgfpathmoveto{\pgfqpoint{0.686113in}{0.579464in}}%
\pgfpathlineto{\pgfqpoint{0.614919in}{0.583110in}}%
\pgfpathmoveto{\pgfqpoint{1.547819in}{0.613854in}}%
\pgfpathlineto{\pgfqpoint{1.458297in}{0.696339in}}%
\pgfpathmoveto{\pgfqpoint{1.547819in}{0.613854in}}%
\pgfpathlineto{\pgfqpoint{1.625404in}{0.710683in}}%
\pgfpathmoveto{\pgfqpoint{1.547819in}{0.613854in}}%
\pgfpathlineto{\pgfqpoint{1.535916in}{0.486155in}}%
\pgfpathmoveto{\pgfqpoint{1.547819in}{0.613854in}}%
\pgfpathlineto{\pgfqpoint{1.445323in}{0.602316in}}%
\pgfpathmoveto{\pgfqpoint{0.550365in}{0.619266in}}%
\pgfpathlineto{\pgfqpoint{0.523844in}{0.695653in}}%
\pgfpathmoveto{\pgfqpoint{0.550365in}{0.619266in}}%
\pgfpathlineto{\pgfqpoint{0.597734in}{0.682628in}}%
\pgfpathmoveto{\pgfqpoint{0.550365in}{0.619266in}}%
\pgfpathlineto{\pgfqpoint{0.526682in}{0.528714in}}%
\pgfpathmoveto{\pgfqpoint{0.550365in}{0.619266in}}%
\pgfpathlineto{\pgfqpoint{0.614919in}{0.583110in}}%
\pgfpathmoveto{\pgfqpoint{0.550365in}{0.619266in}}%
\pgfpathlineto{\pgfqpoint{0.441612in}{0.602117in}}%
\pgfpathmoveto{\pgfqpoint{0.377336in}{0.284113in}}%
\pgfpathlineto{\pgfqpoint{0.185788in}{0.100000in}}%
\pgfpathmoveto{\pgfqpoint{0.377336in}{0.284113in}}%
\pgfpathlineto{\pgfqpoint{0.454502in}{0.100000in}}%
\pgfpathmoveto{\pgfqpoint{0.377336in}{0.284113in}}%
\pgfpathlineto{\pgfqpoint{0.181023in}{0.408839in}}%
\pgfpathmoveto{\pgfqpoint{0.377336in}{0.284113in}}%
\pgfpathlineto{\pgfqpoint{0.534719in}{0.383741in}}%
\pgfpathmoveto{\pgfqpoint{0.377336in}{0.284113in}}%
\pgfpathlineto{\pgfqpoint{0.357678in}{0.492105in}}%
\pgfpathmoveto{\pgfqpoint{1.533730in}{0.293058in}}%
\pgfpathlineto{\pgfqpoint{1.798069in}{0.100000in}}%
\pgfpathmoveto{\pgfqpoint{1.533730in}{0.293058in}}%
\pgfpathlineto{\pgfqpoint{1.260642in}{0.100000in}}%
\pgfpathmoveto{\pgfqpoint{1.533730in}{0.293058in}}%
\pgfpathlineto{\pgfqpoint{1.529356in}{0.100000in}}%
\pgfpathmoveto{\pgfqpoint{1.533730in}{0.293058in}}%
\pgfpathlineto{\pgfqpoint{1.801477in}{0.408760in}}%
\pgfpathmoveto{\pgfqpoint{1.533730in}{0.293058in}}%
\pgfpathlineto{\pgfqpoint{1.380079in}{0.372824in}}%
\pgfpathmoveto{\pgfqpoint{1.533730in}{0.293058in}}%
\pgfpathlineto{\pgfqpoint{1.535916in}{0.486155in}}%
\pgfpathmoveto{\pgfqpoint{1.660037in}{0.588517in}}%
\pgfpathlineto{\pgfqpoint{1.787931in}{0.719684in}}%
\pgfpathmoveto{\pgfqpoint{1.660037in}{0.588517in}}%
\pgfpathlineto{\pgfqpoint{1.801477in}{0.408760in}}%
\pgfpathmoveto{\pgfqpoint{1.660037in}{0.588517in}}%
\pgfpathlineto{\pgfqpoint{1.625404in}{0.710683in}}%
\pgfpathmoveto{\pgfqpoint{1.660037in}{0.588517in}}%
\pgfpathlineto{\pgfqpoint{1.535916in}{0.486155in}}%
\pgfpathmoveto{\pgfqpoint{1.660037in}{0.588517in}}%
\pgfpathlineto{\pgfqpoint{1.547819in}{0.613854in}}%
\pgfpathmoveto{\pgfqpoint{0.840010in}{0.211728in}}%
\pgfpathlineto{\pgfqpoint{0.723215in}{0.100000in}}%
\pgfpathmoveto{\pgfqpoint{0.840010in}{0.211728in}}%
\pgfpathlineto{\pgfqpoint{0.991929in}{0.100000in}}%
\pgfpathmoveto{\pgfqpoint{0.840010in}{0.211728in}}%
\pgfpathlineto{\pgfqpoint{0.816937in}{0.367970in}}%
\pgfpathmoveto{\pgfqpoint{0.840010in}{0.211728in}}%
\pgfpathlineto{\pgfqpoint{0.938998in}{0.293567in}}%
\pgfpathmoveto{\pgfqpoint{0.840010in}{0.211728in}}%
\pgfpathlineto{\pgfqpoint{0.690674in}{0.289421in}}%
\pgfpathmoveto{\pgfqpoint{0.553809in}{0.230866in}}%
\pgfpathlineto{\pgfqpoint{0.454502in}{0.100000in}}%
\pgfpathmoveto{\pgfqpoint{0.553809in}{0.230866in}}%
\pgfpathlineto{\pgfqpoint{0.723215in}{0.100000in}}%
\pgfpathmoveto{\pgfqpoint{0.553809in}{0.230866in}}%
\pgfpathlineto{\pgfqpoint{0.534719in}{0.383741in}}%
\pgfpathmoveto{\pgfqpoint{0.553809in}{0.230866in}}%
\pgfpathlineto{\pgfqpoint{0.690674in}{0.289421in}}%
\pgfpathmoveto{\pgfqpoint{0.553809in}{0.230866in}}%
\pgfpathlineto{\pgfqpoint{0.377336in}{0.284113in}}%
\pgfpathlineto{\pgfqpoint{0.377336in}{0.284113in}}%
\pgfusepath{stroke}%
\end{pgfscope}%
\begin{pgfscope}%
\pgfpathrectangle{\pgfqpoint{0.100000in}{0.100000in}}{\pgfqpoint{1.782500in}{1.232000in}}%
\pgfusepath{clip}%
\pgfsetrectcap%
\pgfsetroundjoin%
\pgfsetlinewidth{0.250937pt}%
\definecolor{currentstroke}{rgb}{0.835294,0.321569,0.035294}%
\pgfsetstrokecolor{currentstroke}%
\pgfsetdash{}{0pt}%
\pgfpathmoveto{\pgfqpoint{0.493932in}{0.837819in}}%
\pgfpathlineto{\pgfqpoint{0.454502in}{1.085600in}}%
\pgfpathmoveto{\pgfqpoint{1.529356in}{1.085600in}}%
\pgfpathlineto{\pgfqpoint{1.489237in}{0.837484in}}%
\pgfpathmoveto{\pgfqpoint{0.723215in}{1.085600in}}%
\pgfpathlineto{\pgfqpoint{0.991929in}{1.085600in}}%
\pgfpathmoveto{\pgfqpoint{0.723215in}{1.085600in}}%
\pgfpathlineto{\pgfqpoint{0.454502in}{1.085600in}}%
\pgfpathmoveto{\pgfqpoint{0.539694in}{0.759689in}}%
\pgfpathlineto{\pgfqpoint{0.493932in}{0.837819in}}%
\pgfpathmoveto{\pgfqpoint{0.597797in}{0.697728in}}%
\pgfpathlineto{\pgfqpoint{0.539694in}{0.759689in}}%
\pgfpathmoveto{\pgfqpoint{0.666513in}{0.658314in}}%
\pgfpathlineto{\pgfqpoint{0.597797in}{0.697728in}}%
\pgfpathmoveto{\pgfqpoint{0.752247in}{0.632595in}}%
\pgfpathlineto{\pgfqpoint{0.666513in}{0.658314in}}%
\pgfpathmoveto{\pgfqpoint{0.833592in}{0.614513in}}%
\pgfpathlineto{\pgfqpoint{0.752247in}{0.632595in}}%
\pgfpathmoveto{\pgfqpoint{0.912544in}{0.600941in}}%
\pgfpathlineto{\pgfqpoint{0.833592in}{0.614513in}}%
\pgfpathmoveto{\pgfqpoint{0.990153in}{0.598729in}}%
\pgfpathlineto{\pgfqpoint{0.912544in}{0.600941in}}%
\pgfpathmoveto{\pgfqpoint{1.067159in}{0.607466in}}%
\pgfpathlineto{\pgfqpoint{0.990153in}{0.598729in}}%
\pgfpathmoveto{\pgfqpoint{1.147591in}{0.618472in}}%
\pgfpathlineto{\pgfqpoint{1.067159in}{0.607466in}}%
\pgfpathmoveto{\pgfqpoint{1.229103in}{0.635206in}}%
\pgfpathlineto{\pgfqpoint{1.147591in}{0.618472in}}%
\pgfpathmoveto{\pgfqpoint{1.314300in}{0.661595in}}%
\pgfpathlineto{\pgfqpoint{1.229103in}{0.635206in}}%
\pgfpathmoveto{\pgfqpoint{1.383298in}{0.696947in}}%
\pgfpathlineto{\pgfqpoint{1.314300in}{0.661595in}}%
\pgfpathmoveto{\pgfqpoint{1.443404in}{0.759511in}}%
\pgfpathlineto{\pgfqpoint{1.489237in}{0.837484in}}%
\pgfpathmoveto{\pgfqpoint{1.443404in}{0.759511in}}%
\pgfpathlineto{\pgfqpoint{1.383298in}{0.696947in}}%
\pgfpathmoveto{\pgfqpoint{1.260642in}{1.085600in}}%
\pgfpathlineto{\pgfqpoint{0.991929in}{1.085600in}}%
\pgfpathmoveto{\pgfqpoint{1.260642in}{1.085600in}}%
\pgfpathlineto{\pgfqpoint{1.529356in}{1.085600in}}%
\pgfpathmoveto{\pgfqpoint{0.831595in}{0.886984in}}%
\pgfpathlineto{\pgfqpoint{0.991929in}{1.085600in}}%
\pgfpathmoveto{\pgfqpoint{0.831595in}{0.886984in}}%
\pgfpathlineto{\pgfqpoint{0.723215in}{1.085600in}}%
\pgfpathmoveto{\pgfqpoint{0.960383in}{0.754944in}}%
\pgfpathlineto{\pgfqpoint{1.116418in}{0.824542in}}%
\pgfpathmoveto{\pgfqpoint{0.960383in}{0.754944in}}%
\pgfpathlineto{\pgfqpoint{0.831595in}{0.886984in}}%
\pgfpathmoveto{\pgfqpoint{1.283109in}{0.858031in}}%
\pgfpathlineto{\pgfqpoint{1.116418in}{0.824542in}}%
\pgfpathmoveto{\pgfqpoint{0.666415in}{0.893785in}}%
\pgfpathlineto{\pgfqpoint{0.723215in}{1.085600in}}%
\pgfpathmoveto{\pgfqpoint{0.666415in}{0.893785in}}%
\pgfpathlineto{\pgfqpoint{0.831595in}{0.886984in}}%
\pgfpathmoveto{\pgfqpoint{1.091714in}{0.718622in}}%
\pgfpathlineto{\pgfqpoint{1.067159in}{0.607466in}}%
\pgfpathmoveto{\pgfqpoint{1.091714in}{0.718622in}}%
\pgfpathlineto{\pgfqpoint{1.147591in}{0.618472in}}%
\pgfpathmoveto{\pgfqpoint{1.091714in}{0.718622in}}%
\pgfpathlineto{\pgfqpoint{1.116418in}{0.824542in}}%
\pgfpathmoveto{\pgfqpoint{1.091714in}{0.718622in}}%
\pgfpathlineto{\pgfqpoint{0.960383in}{0.754944in}}%
\pgfpathmoveto{\pgfqpoint{1.216169in}{0.750966in}}%
\pgfpathlineto{\pgfqpoint{1.229103in}{0.635206in}}%
\pgfpathmoveto{\pgfqpoint{1.216169in}{0.750966in}}%
\pgfpathlineto{\pgfqpoint{1.314300in}{0.661595in}}%
\pgfpathmoveto{\pgfqpoint{1.216169in}{0.750966in}}%
\pgfpathlineto{\pgfqpoint{1.116418in}{0.824542in}}%
\pgfpathmoveto{\pgfqpoint{1.216169in}{0.750966in}}%
\pgfpathlineto{\pgfqpoint{1.283109in}{0.858031in}}%
\pgfpathmoveto{\pgfqpoint{1.216169in}{0.750966in}}%
\pgfpathlineto{\pgfqpoint{1.091714in}{0.718622in}}%
\pgfpathmoveto{\pgfqpoint{0.819793in}{0.728687in}}%
\pgfpathlineto{\pgfqpoint{0.752247in}{0.632595in}}%
\pgfpathmoveto{\pgfqpoint{0.819793in}{0.728687in}}%
\pgfpathlineto{\pgfqpoint{0.833592in}{0.614513in}}%
\pgfpathmoveto{\pgfqpoint{0.819793in}{0.728687in}}%
\pgfpathlineto{\pgfqpoint{0.831595in}{0.886984in}}%
\pgfpathmoveto{\pgfqpoint{0.819793in}{0.728687in}}%
\pgfpathlineto{\pgfqpoint{0.960383in}{0.754944in}}%
\pgfpathmoveto{\pgfqpoint{0.708591in}{0.780276in}}%
\pgfpathlineto{\pgfqpoint{0.597797in}{0.697728in}}%
\pgfpathmoveto{\pgfqpoint{0.708591in}{0.780276in}}%
\pgfpathlineto{\pgfqpoint{0.666513in}{0.658314in}}%
\pgfpathmoveto{\pgfqpoint{0.708591in}{0.780276in}}%
\pgfpathlineto{\pgfqpoint{0.831595in}{0.886984in}}%
\pgfpathmoveto{\pgfqpoint{0.708591in}{0.780276in}}%
\pgfpathlineto{\pgfqpoint{0.666415in}{0.893785in}}%
\pgfpathmoveto{\pgfqpoint{0.708591in}{0.780276in}}%
\pgfpathlineto{\pgfqpoint{0.819793in}{0.728687in}}%
\pgfpathmoveto{\pgfqpoint{1.369911in}{0.959364in}}%
\pgfpathlineto{\pgfqpoint{1.489237in}{0.837484in}}%
\pgfpathmoveto{\pgfqpoint{1.369911in}{0.959364in}}%
\pgfpathlineto{\pgfqpoint{1.529356in}{1.085600in}}%
\pgfpathmoveto{\pgfqpoint{1.369911in}{0.959364in}}%
\pgfpathlineto{\pgfqpoint{1.260642in}{1.085600in}}%
\pgfpathmoveto{\pgfqpoint{1.369911in}{0.959364in}}%
\pgfpathlineto{\pgfqpoint{1.283109in}{0.858031in}}%
\pgfpathmoveto{\pgfqpoint{0.953812in}{0.656912in}}%
\pgfpathlineto{\pgfqpoint{0.912544in}{0.600941in}}%
\pgfpathmoveto{\pgfqpoint{0.953812in}{0.656912in}}%
\pgfpathlineto{\pgfqpoint{0.990153in}{0.598729in}}%
\pgfpathmoveto{\pgfqpoint{0.953812in}{0.656912in}}%
\pgfpathlineto{\pgfqpoint{0.960383in}{0.754944in}}%
\pgfpathmoveto{\pgfqpoint{1.303198in}{0.744504in}}%
\pgfpathlineto{\pgfqpoint{1.314300in}{0.661595in}}%
\pgfpathmoveto{\pgfqpoint{1.303198in}{0.744504in}}%
\pgfpathlineto{\pgfqpoint{1.383298in}{0.696947in}}%
\pgfpathmoveto{\pgfqpoint{1.303198in}{0.744504in}}%
\pgfpathlineto{\pgfqpoint{1.283109in}{0.858031in}}%
\pgfpathmoveto{\pgfqpoint{1.303198in}{0.744504in}}%
\pgfpathlineto{\pgfqpoint{1.216169in}{0.750966in}}%
\pgfpathmoveto{\pgfqpoint{1.169146in}{0.688366in}}%
\pgfpathlineto{\pgfqpoint{1.147591in}{0.618472in}}%
\pgfpathmoveto{\pgfqpoint{1.169146in}{0.688366in}}%
\pgfpathlineto{\pgfqpoint{1.229103in}{0.635206in}}%
\pgfpathmoveto{\pgfqpoint{1.169146in}{0.688366in}}%
\pgfpathlineto{\pgfqpoint{1.091714in}{0.718622in}}%
\pgfpathmoveto{\pgfqpoint{1.169146in}{0.688366in}}%
\pgfpathlineto{\pgfqpoint{1.216169in}{0.750966in}}%
\pgfpathmoveto{\pgfqpoint{1.386091in}{0.854492in}}%
\pgfpathlineto{\pgfqpoint{1.489237in}{0.837484in}}%
\pgfpathmoveto{\pgfqpoint{1.386091in}{0.854492in}}%
\pgfpathlineto{\pgfqpoint{1.443404in}{0.759511in}}%
\pgfpathmoveto{\pgfqpoint{1.386091in}{0.854492in}}%
\pgfpathlineto{\pgfqpoint{1.283109in}{0.858031in}}%
\pgfpathmoveto{\pgfqpoint{1.386091in}{0.854492in}}%
\pgfpathlineto{\pgfqpoint{1.369911in}{0.959364in}}%
\pgfpathmoveto{\pgfqpoint{0.739504in}{0.707019in}}%
\pgfpathlineto{\pgfqpoint{0.666513in}{0.658314in}}%
\pgfpathmoveto{\pgfqpoint{0.739504in}{0.707019in}}%
\pgfpathlineto{\pgfqpoint{0.752247in}{0.632595in}}%
\pgfpathmoveto{\pgfqpoint{0.739504in}{0.707019in}}%
\pgfpathlineto{\pgfqpoint{0.819793in}{0.728687in}}%
\pgfpathmoveto{\pgfqpoint{0.739504in}{0.707019in}}%
\pgfpathlineto{\pgfqpoint{0.708591in}{0.780276in}}%
\pgfpathmoveto{\pgfqpoint{0.624324in}{0.786217in}}%
\pgfpathlineto{\pgfqpoint{0.539694in}{0.759689in}}%
\pgfpathmoveto{\pgfqpoint{0.624324in}{0.786217in}}%
\pgfpathlineto{\pgfqpoint{0.597797in}{0.697728in}}%
\pgfpathmoveto{\pgfqpoint{0.624324in}{0.786217in}}%
\pgfpathlineto{\pgfqpoint{0.666415in}{0.893785in}}%
\pgfpathmoveto{\pgfqpoint{0.624324in}{0.786217in}}%
\pgfpathlineto{\pgfqpoint{0.708591in}{0.780276in}}%
\pgfpathmoveto{\pgfqpoint{0.566652in}{0.964750in}}%
\pgfpathlineto{\pgfqpoint{0.454502in}{1.085600in}}%
\pgfpathmoveto{\pgfqpoint{0.566652in}{0.964750in}}%
\pgfpathlineto{\pgfqpoint{0.493932in}{0.837819in}}%
\pgfpathmoveto{\pgfqpoint{0.566652in}{0.964750in}}%
\pgfpathlineto{\pgfqpoint{0.723215in}{1.085600in}}%
\pgfpathmoveto{\pgfqpoint{0.566652in}{0.964750in}}%
\pgfpathlineto{\pgfqpoint{0.666415in}{0.893785in}}%
\pgfpathmoveto{\pgfqpoint{1.025791in}{0.660878in}}%
\pgfpathlineto{\pgfqpoint{0.990153in}{0.598729in}}%
\pgfpathmoveto{\pgfqpoint{1.025791in}{0.660878in}}%
\pgfpathlineto{\pgfqpoint{1.067159in}{0.607466in}}%
\pgfpathmoveto{\pgfqpoint{1.025791in}{0.660878in}}%
\pgfpathlineto{\pgfqpoint{0.960383in}{0.754944in}}%
\pgfpathmoveto{\pgfqpoint{1.025791in}{0.660878in}}%
\pgfpathlineto{\pgfqpoint{1.091714in}{0.718622in}}%
\pgfpathmoveto{\pgfqpoint{1.025791in}{0.660878in}}%
\pgfpathlineto{\pgfqpoint{0.953812in}{0.656912in}}%
\pgfpathmoveto{\pgfqpoint{0.882004in}{0.665729in}}%
\pgfpathlineto{\pgfqpoint{0.833592in}{0.614513in}}%
\pgfpathmoveto{\pgfqpoint{0.882004in}{0.665729in}}%
\pgfpathlineto{\pgfqpoint{0.912544in}{0.600941in}}%
\pgfpathmoveto{\pgfqpoint{0.882004in}{0.665729in}}%
\pgfpathlineto{\pgfqpoint{0.960383in}{0.754944in}}%
\pgfpathmoveto{\pgfqpoint{0.882004in}{0.665729in}}%
\pgfpathlineto{\pgfqpoint{0.819793in}{0.728687in}}%
\pgfpathmoveto{\pgfqpoint{0.882004in}{0.665729in}}%
\pgfpathlineto{\pgfqpoint{0.953812in}{0.656912in}}%
\pgfpathmoveto{\pgfqpoint{1.367508in}{0.777446in}}%
\pgfpathlineto{\pgfqpoint{1.383298in}{0.696947in}}%
\pgfpathmoveto{\pgfqpoint{1.367508in}{0.777446in}}%
\pgfpathlineto{\pgfqpoint{1.443404in}{0.759511in}}%
\pgfpathmoveto{\pgfqpoint{1.367508in}{0.777446in}}%
\pgfpathlineto{\pgfqpoint{1.283109in}{0.858031in}}%
\pgfpathmoveto{\pgfqpoint{1.367508in}{0.777446in}}%
\pgfpathlineto{\pgfqpoint{1.303198in}{0.744504in}}%
\pgfpathmoveto{\pgfqpoint{1.367508in}{0.777446in}}%
\pgfpathlineto{\pgfqpoint{1.386091in}{0.854492in}}%
\pgfpathmoveto{\pgfqpoint{1.015772in}{0.905746in}}%
\pgfpathlineto{\pgfqpoint{0.991929in}{1.085600in}}%
\pgfpathmoveto{\pgfqpoint{1.015772in}{0.905746in}}%
\pgfpathlineto{\pgfqpoint{1.116418in}{0.824542in}}%
\pgfpathmoveto{\pgfqpoint{1.015772in}{0.905746in}}%
\pgfpathlineto{\pgfqpoint{0.831595in}{0.886984in}}%
\pgfpathmoveto{\pgfqpoint{1.015772in}{0.905746in}}%
\pgfpathlineto{\pgfqpoint{0.960383in}{0.754944in}}%
\pgfpathmoveto{\pgfqpoint{1.174985in}{0.953328in}}%
\pgfpathlineto{\pgfqpoint{0.991929in}{1.085600in}}%
\pgfpathmoveto{\pgfqpoint{1.174985in}{0.953328in}}%
\pgfpathlineto{\pgfqpoint{1.260642in}{1.085600in}}%
\pgfpathmoveto{\pgfqpoint{1.174985in}{0.953328in}}%
\pgfpathlineto{\pgfqpoint{1.116418in}{0.824542in}}%
\pgfpathmoveto{\pgfqpoint{1.174985in}{0.953328in}}%
\pgfpathlineto{\pgfqpoint{1.283109in}{0.858031in}}%
\pgfpathmoveto{\pgfqpoint{1.174985in}{0.953328in}}%
\pgfpathlineto{\pgfqpoint{1.369911in}{0.959364in}}%
\pgfpathmoveto{\pgfqpoint{1.174985in}{0.953328in}}%
\pgfpathlineto{\pgfqpoint{1.015772in}{0.905746in}}%
\pgfpathmoveto{\pgfqpoint{0.576803in}{0.848301in}}%
\pgfpathlineto{\pgfqpoint{0.493932in}{0.837819in}}%
\pgfpathmoveto{\pgfqpoint{0.576803in}{0.848301in}}%
\pgfpathlineto{\pgfqpoint{0.539694in}{0.759689in}}%
\pgfpathmoveto{\pgfqpoint{0.576803in}{0.848301in}}%
\pgfpathlineto{\pgfqpoint{0.666415in}{0.893785in}}%
\pgfpathmoveto{\pgfqpoint{0.576803in}{0.848301in}}%
\pgfpathlineto{\pgfqpoint{0.624324in}{0.786217in}}%
\pgfpathmoveto{\pgfqpoint{0.576803in}{0.848301in}}%
\pgfpathlineto{\pgfqpoint{0.566652in}{0.964750in}}%
\pgfpathlineto{\pgfqpoint{0.566652in}{0.964750in}}%
\pgfusepath{stroke}%
\end{pgfscope}%
\begin{pgfscope}%
\pgfpathrectangle{\pgfqpoint{0.100000in}{0.100000in}}{\pgfqpoint{1.782500in}{1.232000in}}%
\pgfusepath{clip}%
\pgfsetbuttcap%
\pgfsetroundjoin%
\definecolor{currentfill}{rgb}{0.054902,0.262745,0.486275}%
\pgfsetfillcolor{currentfill}%
\pgfsetlinewidth{1.003750pt}%
\definecolor{currentstroke}{rgb}{0.054902,0.262745,0.486275}%
\pgfsetstrokecolor{currentstroke}%
\pgfsetdash{}{0pt}%
\pgfsys@defobject{currentmarker}{\pgfqpoint{-0.018373in}{-0.018373in}}{\pgfqpoint{0.018373in}{0.018373in}}{%
\pgfpathmoveto{\pgfqpoint{0.000000in}{-0.018373in}}%
\pgfpathcurveto{\pgfqpoint{0.004873in}{-0.018373in}}{\pgfqpoint{0.009546in}{-0.016437in}}{\pgfqpoint{0.012992in}{-0.012992in}}%
\pgfpathcurveto{\pgfqpoint{0.016437in}{-0.009546in}}{\pgfqpoint{0.018373in}{-0.004873in}}{\pgfqpoint{0.018373in}{0.000000in}}%
\pgfpathcurveto{\pgfqpoint{0.018373in}{0.004873in}}{\pgfqpoint{0.016437in}{0.009546in}}{\pgfqpoint{0.012992in}{0.012992in}}%
\pgfpathcurveto{\pgfqpoint{0.009546in}{0.016437in}}{\pgfqpoint{0.004873in}{0.018373in}}{\pgfqpoint{0.000000in}{0.018373in}}%
\pgfpathcurveto{\pgfqpoint{-0.004873in}{0.018373in}}{\pgfqpoint{-0.009546in}{0.016437in}}{\pgfqpoint{-0.012992in}{0.012992in}}%
\pgfpathcurveto{\pgfqpoint{-0.016437in}{0.009546in}}{\pgfqpoint{-0.018373in}{0.004873in}}{\pgfqpoint{-0.018373in}{0.000000in}}%
\pgfpathcurveto{\pgfqpoint{-0.018373in}{-0.004873in}}{\pgfqpoint{-0.016437in}{-0.009546in}}{\pgfqpoint{-0.012992in}{-0.012992in}}%
\pgfpathcurveto{\pgfqpoint{-0.009546in}{-0.016437in}}{\pgfqpoint{-0.004873in}{-0.018373in}}{\pgfqpoint{0.000000in}{-0.018373in}}%
\pgfpathlineto{\pgfqpoint{0.000000in}{-0.018373in}}%
\pgfpathclose%
\pgfusepath{stroke,fill}%
}%
\begin{pgfscope}%
\pgfsys@transformshift{1.310324in}{0.660296in}%
\pgfsys@useobject{currentmarker}{}%
\end{pgfscope}%
\begin{pgfscope}%
\pgfsys@transformshift{1.253051in}{0.642085in}%
\pgfsys@useobject{currentmarker}{}%
\end{pgfscope}%
\begin{pgfscope}%
\pgfsys@transformshift{1.188115in}{0.625601in}%
\pgfsys@useobject{currentmarker}{}%
\end{pgfscope}%
\begin{pgfscope}%
\pgfsys@transformshift{1.127651in}{0.615595in}%
\pgfsys@useobject{currentmarker}{}%
\end{pgfscope}%
\begin{pgfscope}%
\pgfsys@transformshift{1.058490in}{0.606300in}%
\pgfsys@useobject{currentmarker}{}%
\end{pgfscope}%
\begin{pgfscope}%
\pgfsys@transformshift{0.990122in}{0.598971in}%
\pgfsys@useobject{currentmarker}{}%
\end{pgfscope}%
\begin{pgfscope}%
\pgfsys@transformshift{0.921265in}{0.600292in}%
\pgfsys@useobject{currentmarker}{}%
\end{pgfscope}%
\begin{pgfscope}%
\pgfsys@transformshift{0.853138in}{0.610721in}%
\pgfsys@useobject{currentmarker}{}%
\end{pgfscope}%
\begin{pgfscope}%
\pgfsys@transformshift{0.793331in}{0.622693in}%
\pgfsys@useobject{currentmarker}{}%
\end{pgfscope}%
\begin{pgfscope}%
\pgfsys@transformshift{0.728172in}{0.639388in}%
\pgfsys@useobject{currentmarker}{}%
\end{pgfscope}%
\begin{pgfscope}%
\pgfsys@transformshift{0.670517in}{0.657059in}%
\pgfsys@useobject{currentmarker}{}%
\end{pgfscope}%
\end{pgfscope}%
\begin{pgfscope}%
\pgfpathrectangle{\pgfqpoint{0.100000in}{0.100000in}}{\pgfqpoint{1.782500in}{1.232000in}}%
\pgfusepath{clip}%
\pgfsetbuttcap%
\pgfsetroundjoin%
\definecolor{currentfill}{rgb}{0.835294,0.321569,0.035294}%
\pgfsetfillcolor{currentfill}%
\pgfsetlinewidth{1.003750pt}%
\definecolor{currentstroke}{rgb}{0.835294,0.321569,0.035294}%
\pgfsetstrokecolor{currentstroke}%
\pgfsetdash{}{0pt}%
\pgfsys@defobject{currentmarker}{\pgfqpoint{-0.018373in}{-0.018373in}}{\pgfqpoint{0.018373in}{0.018373in}}{%
\pgfpathmoveto{\pgfqpoint{0.000000in}{-0.018373in}}%
\pgfpathcurveto{\pgfqpoint{0.004873in}{-0.018373in}}{\pgfqpoint{0.009546in}{-0.016437in}}{\pgfqpoint{0.012992in}{-0.012992in}}%
\pgfpathcurveto{\pgfqpoint{0.016437in}{-0.009546in}}{\pgfqpoint{0.018373in}{-0.004873in}}{\pgfqpoint{0.018373in}{0.000000in}}%
\pgfpathcurveto{\pgfqpoint{0.018373in}{0.004873in}}{\pgfqpoint{0.016437in}{0.009546in}}{\pgfqpoint{0.012992in}{0.012992in}}%
\pgfpathcurveto{\pgfqpoint{0.009546in}{0.016437in}}{\pgfqpoint{0.004873in}{0.018373in}}{\pgfqpoint{0.000000in}{0.018373in}}%
\pgfpathcurveto{\pgfqpoint{-0.004873in}{0.018373in}}{\pgfqpoint{-0.009546in}{0.016437in}}{\pgfqpoint{-0.012992in}{0.012992in}}%
\pgfpathcurveto{\pgfqpoint{-0.016437in}{0.009546in}}{\pgfqpoint{-0.018373in}{0.004873in}}{\pgfqpoint{-0.018373in}{0.000000in}}%
\pgfpathcurveto{\pgfqpoint{-0.018373in}{-0.004873in}}{\pgfqpoint{-0.016437in}{-0.009546in}}{\pgfqpoint{-0.012992in}{-0.012992in}}%
\pgfpathcurveto{\pgfqpoint{-0.009546in}{-0.016437in}}{\pgfqpoint{-0.004873in}{-0.018373in}}{\pgfqpoint{0.000000in}{-0.018373in}}%
\pgfpathlineto{\pgfqpoint{0.000000in}{-0.018373in}}%
\pgfpathclose%
\pgfusepath{stroke,fill}%
}%
\begin{pgfscope}%
\pgfsys@transformshift{0.666513in}{0.658314in}%
\pgfsys@useobject{currentmarker}{}%
\end{pgfscope}%
\begin{pgfscope}%
\pgfsys@transformshift{0.752247in}{0.632595in}%
\pgfsys@useobject{currentmarker}{}%
\end{pgfscope}%
\begin{pgfscope}%
\pgfsys@transformshift{0.833592in}{0.614513in}%
\pgfsys@useobject{currentmarker}{}%
\end{pgfscope}%
\begin{pgfscope}%
\pgfsys@transformshift{0.912544in}{0.600941in}%
\pgfsys@useobject{currentmarker}{}%
\end{pgfscope}%
\begin{pgfscope}%
\pgfsys@transformshift{0.990153in}{0.598729in}%
\pgfsys@useobject{currentmarker}{}%
\end{pgfscope}%
\begin{pgfscope}%
\pgfsys@transformshift{1.067159in}{0.607466in}%
\pgfsys@useobject{currentmarker}{}%
\end{pgfscope}%
\begin{pgfscope}%
\pgfsys@transformshift{1.147591in}{0.618472in}%
\pgfsys@useobject{currentmarker}{}%
\end{pgfscope}%
\begin{pgfscope}%
\pgfsys@transformshift{1.229103in}{0.635206in}%
\pgfsys@useobject{currentmarker}{}%
\end{pgfscope}%
\begin{pgfscope}%
\pgfsys@transformshift{1.314300in}{0.661595in}%
\pgfsys@useobject{currentmarker}{}%
\end{pgfscope}%
\end{pgfscope}%
\begin{pgfscope}%
\pgfpathrectangle{\pgfqpoint{0.100000in}{0.100000in}}{\pgfqpoint{1.782500in}{1.232000in}}%
\pgfusepath{clip}%
\pgfsetbuttcap%
\pgfsetroundjoin%
\pgfsetlinewidth{1.003750pt}%
\definecolor{currentstroke}{rgb}{0.054902,0.262745,0.486275}%
\pgfsetstrokecolor{currentstroke}%
\pgfsetdash{}{0pt}%
\pgfpathmoveto{\pgfqpoint{0.000000in}{-0.018373in}}%
\pgfpathcurveto{\pgfqpoint{0.004873in}{-0.018373in}}{\pgfqpoint{0.009546in}{-0.016437in}}{\pgfqpoint{0.012992in}{-0.012992in}}%
\pgfpathcurveto{\pgfqpoint{0.016437in}{-0.009546in}}{\pgfqpoint{0.018373in}{-0.004873in}}{\pgfqpoint{0.018373in}{0.000000in}}%
\pgfpathcurveto{\pgfqpoint{0.018373in}{0.004873in}}{\pgfqpoint{0.016437in}{0.009546in}}{\pgfqpoint{0.012992in}{0.012992in}}%
\pgfpathcurveto{\pgfqpoint{0.009546in}{0.016437in}}{\pgfqpoint{0.004873in}{0.018373in}}{\pgfqpoint{0.000000in}{0.018373in}}%
\pgfpathcurveto{\pgfqpoint{-0.004873in}{0.018373in}}{\pgfqpoint{-0.009546in}{0.016437in}}{\pgfqpoint{-0.012992in}{0.012992in}}%
\pgfpathcurveto{\pgfqpoint{-0.016437in}{0.009546in}}{\pgfqpoint{-0.018373in}{0.004873in}}{\pgfqpoint{-0.018373in}{0.000000in}}%
\pgfpathcurveto{\pgfqpoint{-0.018373in}{-0.004873in}}{\pgfqpoint{-0.016437in}{-0.009546in}}{\pgfqpoint{-0.012992in}{-0.012992in}}%
\pgfpathcurveto{\pgfqpoint{-0.009546in}{-0.016437in}}{\pgfqpoint{-0.004873in}{-0.018373in}}{\pgfqpoint{0.000000in}{-0.018373in}}%
\pgfusepath{stroke}%
\end{pgfscope}%
\begin{pgfscope}%
\pgfpathrectangle{\pgfqpoint{0.100000in}{0.100000in}}{\pgfqpoint{1.782500in}{1.232000in}}%
\pgfusepath{clip}%
\pgfsetbuttcap%
\pgfsetroundjoin%
\pgfsetlinewidth{1.003750pt}%
\definecolor{currentstroke}{rgb}{0.835294,0.321569,0.035294}%
\pgfsetstrokecolor{currentstroke}%
\pgfsetdash{}{0pt}%
\pgfpathmoveto{\pgfqpoint{0.000000in}{-0.018373in}}%
\pgfpathcurveto{\pgfqpoint{0.004873in}{-0.018373in}}{\pgfqpoint{0.009546in}{-0.016437in}}{\pgfqpoint{0.012992in}{-0.012992in}}%
\pgfpathcurveto{\pgfqpoint{0.016437in}{-0.009546in}}{\pgfqpoint{0.018373in}{-0.004873in}}{\pgfqpoint{0.018373in}{0.000000in}}%
\pgfpathcurveto{\pgfqpoint{0.018373in}{0.004873in}}{\pgfqpoint{0.016437in}{0.009546in}}{\pgfqpoint{0.012992in}{0.012992in}}%
\pgfpathcurveto{\pgfqpoint{0.009546in}{0.016437in}}{\pgfqpoint{0.004873in}{0.018373in}}{\pgfqpoint{0.000000in}{0.018373in}}%
\pgfpathcurveto{\pgfqpoint{-0.004873in}{0.018373in}}{\pgfqpoint{-0.009546in}{0.016437in}}{\pgfqpoint{-0.012992in}{0.012992in}}%
\pgfpathcurveto{\pgfqpoint{-0.016437in}{0.009546in}}{\pgfqpoint{-0.018373in}{0.004873in}}{\pgfqpoint{-0.018373in}{0.000000in}}%
\pgfpathcurveto{\pgfqpoint{-0.018373in}{-0.004873in}}{\pgfqpoint{-0.016437in}{-0.009546in}}{\pgfqpoint{-0.012992in}{-0.012992in}}%
\pgfpathcurveto{\pgfqpoint{-0.009546in}{-0.016437in}}{\pgfqpoint{-0.004873in}{-0.018373in}}{\pgfqpoint{0.000000in}{-0.018373in}}%
\pgfusepath{stroke}%
\end{pgfscope}%
\begin{pgfscope}%
\pgfpathrectangle{\pgfqpoint{0.100000in}{0.100000in}}{\pgfqpoint{1.782500in}{1.232000in}}%
\pgfusepath{clip}%
\pgfsetbuttcap%
\pgfsetroundjoin%
\definecolor{currentfill}{rgb}{0.054902,0.262745,0.486275}%
\pgfsetfillcolor{currentfill}%
\pgfsetlinewidth{1.505625pt}%
\definecolor{currentstroke}{rgb}{0.054902,0.262745,0.486275}%
\pgfsetstrokecolor{currentstroke}%
\pgfsetdash{}{0pt}%
\pgfsys@defobject{currentmarker}{\pgfqpoint{-0.018373in}{-0.018373in}}{\pgfqpoint{0.018373in}{0.018373in}}{%
\pgfpathmoveto{\pgfqpoint{-0.018373in}{-0.018373in}}%
\pgfpathlineto{\pgfqpoint{0.018373in}{0.018373in}}%
\pgfpathmoveto{\pgfqpoint{-0.018373in}{0.018373in}}%
\pgfpathlineto{\pgfqpoint{0.018373in}{-0.018373in}}%
\pgfusepath{stroke,fill}%
}%
\end{pgfscope}%
\begin{pgfscope}%
\pgfpathrectangle{\pgfqpoint{0.100000in}{0.100000in}}{\pgfqpoint{1.782500in}{1.232000in}}%
\pgfusepath{clip}%
\pgfsetbuttcap%
\pgfsetroundjoin%
\definecolor{currentfill}{rgb}{0.835294,0.321569,0.035294}%
\pgfsetfillcolor{currentfill}%
\pgfsetlinewidth{1.505625pt}%
\definecolor{currentstroke}{rgb}{0.835294,0.321569,0.035294}%
\pgfsetstrokecolor{currentstroke}%
\pgfsetdash{}{0pt}%
\pgfsys@defobject{currentmarker}{\pgfqpoint{-0.018373in}{-0.018373in}}{\pgfqpoint{0.018373in}{0.018373in}}{%
\pgfpathmoveto{\pgfqpoint{-0.018373in}{-0.018373in}}%
\pgfpathlineto{\pgfqpoint{0.018373in}{0.018373in}}%
\pgfpathmoveto{\pgfqpoint{-0.018373in}{0.018373in}}%
\pgfpathlineto{\pgfqpoint{0.018373in}{-0.018373in}}%
\pgfusepath{stroke,fill}%
}%
\end{pgfscope}%
\end{pgfpicture}%
\makeatother%
\endgroup%
}
        \caption{Iteration 3: Update interface}\label{fig:example-iter2-dumping}
    \end{subfigure}
    \caption{Example of the INTERNODES algorithm for contact mechanics applied to a two-dimensional problem where a circle interfaces with a rectangular structure. Convergence is obtained after three iterations.}
    \label{fig:example}
\end{figure}

\clearpage
\section{Implementation of the INTERNODES method}
\label{sec:implementation}

Our main objective with this project was to fully implement the INTERNODES method and integrate it with the Akantu finite-element library. For a few years, Akantu has had one contact mechanics model implementation, that we will usually call the \textit{penalty method}. Our main objective was to provide a viable alternative for Akantu users. As such, we had multiple goals beyond the mere implementation of the INTERNODES method:
\begin{itemize}
    \item To make future developments easier and to keep the code neatly organized, the INTERNODES and penalty implementations should follow a similar structure and share as much code as possible.
    \item To allow users to switch between the two methods easily, the Python interface for INTERNODES and penalty contact should be as close as possible.
    \item Last but not least, the INTERNODES method implementation should be robustly implemented and well-tested to be useful in a variety of scenarios.
\end{itemize}

At this point, it is important to note that our work is the continuation of a Merge Request by Moritz Waldleben \cite{moritz} adding a partially complete Akantu C++ implementation of INTERNODES. We also had access to a Python implementation by Moritz Waldleben and an earlier MATLAB implementation by Yannis Voet \cite{voet} as references.

We first cover how we finished the implementation of the INTERNODES method and how it fits within Akantu. We also briefly discuss the Python interface. Then, we discuss the tests that we wrote to validate and verify our implementation.

\subsection{Implementation discussion}

\subsubsection{Extended Python prototype}
One of the first things we did was to refactor and extend Mortz Waldleben's Python reference implementation. This allowed us to better understand the method, and made prototyping easier. The implementation is available on c4science\footnote{\url{https://c4science.ch/source/INTERNODES-CM/}}.

Thanks to this first step, we came up with validation tests that we later ported to C++ (see \refsec{subsec:experiments}).

It is also thanks to this extended Python prototype that we noticed flaws in the previous implementations, particularly around the convergence check (i.e. equations \refequ{equ:convcheck1} and \refequ{equ:convcheck1}). Once we got the convergence check working in Python, we were then able to port it to C++. In general, the extended prototype was the testing ground for various improvements that made it into the C++ version.

\subsubsection{C++ additions}
\label{subsec:cppadditions}
Let us first look at the code organization of the two contact mechanics models before our changes, which can be seen in \reffig{fig:contactdiagrams}.

% TODO: turn these into UML diagrams matching the style of the report ?
\begin{figure}[!htb]
\centering
\begin{subfigure}{.5\textwidth}
  \centering
  \includegraphics[width=0.95\linewidth]{figures/penalty_diagram.pdf}
  \caption{Penalty implementation diagram}
  \label{fig:penaltydiagram}
\end{subfigure}%
\begin{subfigure}{.5\textwidth}
  \centering
  \includegraphics[width=0.95\linewidth]{figures/internodes_diagram.pdf}
  \caption{INTERNODES implementation diagram}
  \label{fig:internodesdiagram}
\end{subfigure}
\caption{Contact mechanics architectures.}
\label{fig:contactdiagrams}
\end{figure}

The \texttt{ContactMechanicsModel} (i.e. the penalty method) was initially implemented as follows:
\begin{itemize}
    \item \texttt{ContactMechanicsModel} is the main class of that model.
    \item \texttt{ContactDetector} handles the contact detection phase, i.e. finding out which elements of the two bodies are in contact.
    \begin{itemize}
        \item To avoid searching over all pairs, the contact detector uses a \texttt{SpatialGrid} to skip checking elements that are too far.
    \end{itemize}
    \item \texttt{CouplerSolidContactTemplate} takes care of the coupling between a solid mechanics model, and the penalty contact detector model.
\end{itemize}

The \texttt{ContactMechanicsInternodesModel} (i.e. the INTERNODES method) was initially implemented as follows:
\begin{itemize}
    \item \texttt{ContactMechanicsInternodesModel} is the main class of that model.
    \item \texttt{ContactDetectorInternodes} handles the contact detection phase.
    \begin{itemize}
        \item No \texttt{SpatialGrid} is being used, so all pairs need to be checked.
    \end{itemize}
    \item The \texttt{ContactMechanicsInternodesModel} contains a \texttt{SolidMechanicsModel}. There is no separate coupler.
\end{itemize}

As a performance optimization and a first step toward shared abstraction, we changed \texttt{ContactDetectorInternodes} to use a \texttt{SpatialGrid}. With this we introduced \texttt{AbstractContactDetector}: a base class for contact detectors, containing mostly helper functions to build the spatial grids, and holding the updated positions vector used by both contact detectors. Unfortunately, we were not able to further refactor the contact detectors nor the contact models due to their largely different internal workings.

The main architectural difference between penalty and INTERNODES is that in the former a coupler model manages all the interaction between the solid and the contact mechanics models, whereas in the latter the contact model directly references the solid model. We attempted to make INTERNODES use a coupler model too, but we found it too complicated. As a compromise, we ensured that the Python interfaces were very similar, so that a user could easily swap from one to the other, as seen in \reffig{fig:pythondiff}. In \refsec{subsec:coupler}, we discuss this in more detail and propose a comprehensive solution.

\begin{figure}[!htb]
\centering
\begin{lstlisting}[language=diff]
  mesh = aka.Mesh(spatial_dimension)
  mesh.read(mesh_file)
- model = aka.CouplerSolidContact(mesh)
+ model = aka.ContactMechanicsInternodesModel(mesh)
  model.applyBC(...)
  # and so on...
\end{lstlisting}
\caption{Moving from penalty to INTERNODES contact mechanics in Python.}
\label{fig:pythondiff}
\end{figure}

Beyond architecture concerns, a crucial missing feature in the C++ implementation was the convergence check. We implemented it in C++ as we have explained in \refsec{subsec:contact-algorithm}. Even though they were loosely ported from the Python prototype, the functions \texttt{getInterfaceNormalAtNode}, \texttt{findPenetratingNodes}, and \texttt{updateAfterStep} are of particular interest.

We also extended the INTERNODES method to work with 3D problems. The code is essentially the same, except for a difference in the computation of the interface normal.

It should be noted that the number of active nodes, hence of DOFs (degrees of freedom), can change between two iterations. We were unable to resize an allocated DOFs vector with Akantu. As a workaround, we allocate enough DOFs for all possibly active nodes (i.e. the initial candidate sets). To preserve numerical stability for the system resolution, we use an identity submatrix for the inactive nodes.

While we were implementing the above, we made a few important changes to the core of Akantu. We
\begin{itemize}
    \item fixed multiplication of a non-square matrix by a vector throwing an exception;
    \item expanded \texttt{NodeGroup::applyNodeFilter} to accept lambdas and return how many elements were erased;
    \item fixed undefined behavior in \texttt{NodeGroup::applyNodeFilter} and \texttt{Array::erase};
    \item expanded \texttt{SpatialGrid} with helper methods to list values neighboring a specific cell and point;
    \item added a few missing \texttt{const} markers.
\end{itemize}

\subsection{Experiments and tests}
\label{subsec:experiments}

In order to verify the correctness and the implementation and ensure the integrity of the code during development, a set of test cases were implemented. Each is a problem which can be solved analytically using the theory of Hertzian mechanics \cite{johnson}. This allows a comparison of certain quantities of interest with the approximation obtained with the INTERNODES method. 

\subsubsection{Contact between a semispheres and a half-space}

The first test case involves a semisphere being vertically pushed into a half-space both in two and three dimensions. The experimental setup is shown in \reffig{fig:sketch-plane-sphere}.

\begin{figure}[ht]
    \centering
    \begin{tikzpicture}
    \draw[ultra thick, darkblue, fill=darkblue!10!white] (-5, -3) to (5, -3) to (5, 0) to (3, 0) to[out=180,in=0] (0, -0.75) to[out=180,in=0] (-3, 0) to  (-5, 0) -- cycle;
    \draw[ultra thick, darkblue, fill=darkblue!10!white] (3, 2.25) arc(360:180:3) -- cycle;
    %\draw[ultra thick, postaction={-,draw=mainorange,dash pattern=on 0cm off 1.4cm on 2cm}] (3, 0) to[out=180,in=0] (0, -0.75) to[out=180,in=0] (-3, 0);
    \draw[ultra thick, darkblue, dashed] (3, 3) arc(360:180:3) -- cycle;
    \draw[ultra thick, darkblue, dashed] (3, 0) -- (-3, 0);
    \draw[<->, thick] (3.25, 3) -- (3.25, 2.25) node[midway, right] {$d$};
    %\draw[->, thick] (0, 3.75) -- (0, 3.25) node[midway, right] {$F$};
    %\draw[<-, thick] (0.5, -0.8) -- (1.3, -1.4) node[right] {\Large \textcolor{mainblue}{$\Gamma_1$}};
    %\draw[<-, thick] (0.5, -0.6) -- (1.3, 0.4) node[right] {\Large \textcolor{mainblue}{$\Gamma_2$}};
    %\draw[<->, thick] (0, 2.4) -- (2.9, 2.4) node[midway, above] {$R$};
    \node at (0, -2) {\Large \textcolor{mainblue}{$\Omega_1$}};
    \node at (0, 1) {\Large \textcolor{mainblue}{$\Omega_2$}};
\end{tikzpicture}
    \caption{Sketch of the first experimental setup used for testing.}
    \label{fig:sketch-plane-sphere}
\end{figure}

Below, the analytical quantities which are used to verify the implementation are defined. The symbols listed in \reftab{tab:symbols} are used.

\begin{table}[ht]
    \caption{Explanation of symbols}
    \label{tab:symbols}
    \centering
    \renewcommand{\arraystretch}{1.2}
\begin{tabular}{@{}cl@{}}
    \toprule
    Symbol & Description \\
    \midrule
    $E$ & Young's modulus \\
    $\nu$ & Poisson's ratio \\
    $R$ & Radius of semisphere/semicircle \\
    $d$ & Applied vertical displacement \\
    $a$ & Radius of contact area \\
    $p_0$ & Contact pressure amplitude \\
    $u_z$ & Penetration depth \\
    \bottomrule
\end{tabular}
\end{table}

The radius of the contact area is given by
\begin{equation}
    a = \sqrt{Rd}
    \label{equ:contact-radius}
\end{equation}

The amplitude of the contact pressure between the two interfaces is
\begin{equation}
    p_0 = \frac{E}{\pi(1 - \nu^2)} \sqrt{\frac{d}{R}}
\end{equation}

The normal displacement (i.e. the penetration depth of the semisphere into the half-space) is
\begin{equation}
    u_z = \frac{d}{2}
    \label{equ:normal-displacement}
\end{equation}

We now run the Akantu implementation on one instance of this problem. For the two-dimensional case, \reffig{fig:solution-2d} shows the obtained solution for a single applied vertical displacement $d$. In \reffig{fig:normal-displacements} and \reffig{fig:contact-radius} we solve the same problem with a fixed mesh for multiple different displacements $d$ and plot the penetration depth $u_z$ and the radius of the contact are $a$, respectively. All quantities are normalized by the mesh size $h$ at the interface.

\begin{figure}[H]
\centering
\begin{subfigure}[b]{\textwidth}
     %% Creator: Matplotlib, PGF backend
%%
%% To include the figure in your LaTeX document, write
%%   \input{<filename>.pgf}
%%
%% Make sure the required packages are loaded in your preamble
%%   \usepackage{pgf}
%%
%% Also ensure that all the required font packages are loaded; for instance,
%% the lmodern package is sometimes necessary when using math font.
%%   \usepackage{lmodern}
%%
%% Figures using additional raster images can only be included by \input if
%% they are in the same directory as the main LaTeX file. For loading figures
%% from other directories you can use the `import` package
%%   \usepackage{import}
%%
%% and then include the figures with
%%   \import{<path to file>}{<filename>.pgf}
%%
%% Matplotlib used the following preamble
%%   
%%   \usepackage{fontspec}
%%   \setmainfont{DejaVuSans.ttf}[Path=\detokenize{/home/fabio/.local/lib/python3.8/site-packages/matplotlib/mpl-data/fonts/ttf/}]
%%   \setsansfont{DejaVuSans.ttf}[Path=\detokenize{/home/fabio/.local/lib/python3.8/site-packages/matplotlib/mpl-data/fonts/ttf/}]
%%   \setmonofont{DejaVuSansMono.ttf}[Path=\detokenize{/home/fabio/.local/lib/python3.8/site-packages/matplotlib/mpl-data/fonts/ttf/}]
%%   \makeatletter\@ifpackageloaded{underscore}{}{\usepackage[strings]{underscore}}\makeatother
%%
\begingroup%
\makeatletter%
\begin{pgfpicture}%
\pgfpathrectangle{\pgfpointorigin}{\pgfqpoint{5.463248in}{2.505353in}}%
\pgfusepath{use as bounding box, clip}%
\begin{pgfscope}%
\pgfsetbuttcap%
\pgfsetmiterjoin%
\definecolor{currentfill}{rgb}{1.000000,1.000000,1.000000}%
\pgfsetfillcolor{currentfill}%
\pgfsetlinewidth{0.000000pt}%
\definecolor{currentstroke}{rgb}{1.000000,1.000000,1.000000}%
\pgfsetstrokecolor{currentstroke}%
\pgfsetdash{}{0pt}%
\pgfpathmoveto{\pgfqpoint{0.000000in}{0.000000in}}%
\pgfpathlineto{\pgfqpoint{5.463248in}{0.000000in}}%
\pgfpathlineto{\pgfqpoint{5.463248in}{2.505353in}}%
\pgfpathlineto{\pgfqpoint{0.000000in}{2.505353in}}%
\pgfpathlineto{\pgfqpoint{0.000000in}{0.000000in}}%
\pgfpathclose%
\pgfusepath{fill}%
\end{pgfscope}%
\begin{pgfscope}%
\pgfsetbuttcap%
\pgfsetmiterjoin%
\definecolor{currentfill}{rgb}{1.000000,1.000000,1.000000}%
\pgfsetfillcolor{currentfill}%
\pgfsetlinewidth{0.000000pt}%
\definecolor{currentstroke}{rgb}{0.000000,0.000000,0.000000}%
\pgfsetstrokecolor{currentstroke}%
\pgfsetstrokeopacity{0.000000}%
\pgfsetdash{}{0pt}%
\pgfpathmoveto{\pgfqpoint{0.713248in}{0.548486in}}%
\pgfpathlineto{\pgfqpoint{5.363248in}{0.548486in}}%
\pgfpathlineto{\pgfqpoint{5.363248in}{2.405353in}}%
\pgfpathlineto{\pgfqpoint{0.713248in}{2.405353in}}%
\pgfpathlineto{\pgfqpoint{0.713248in}{0.548486in}}%
\pgfpathclose%
\pgfusepath{fill}%
\end{pgfscope}%
\begin{pgfscope}%
\pgfpathrectangle{\pgfqpoint{0.713248in}{0.548486in}}{\pgfqpoint{4.650000in}{1.856867in}}%
\pgfusepath{clip}%
\pgfsetrectcap%
\pgfsetroundjoin%
\pgfsetlinewidth{0.501875pt}%
\definecolor{currentstroke}{rgb}{0.054902,0.262745,0.486275}%
\pgfsetstrokecolor{currentstroke}%
\pgfsetdash{}{0pt}%
\pgfpathmoveto{\pgfqpoint{1.530954in}{0.632889in}}%
\pgfpathlineto{\pgfqpoint{0.928075in}{0.632889in}}%
\pgfpathmoveto{\pgfqpoint{2.133832in}{0.632889in}}%
\pgfpathlineto{\pgfqpoint{1.530954in}{0.632889in}}%
\pgfpathmoveto{\pgfqpoint{2.736711in}{0.632889in}}%
\pgfpathlineto{\pgfqpoint{2.133832in}{0.632889in}}%
\pgfpathmoveto{\pgfqpoint{3.339590in}{0.632889in}}%
\pgfpathlineto{\pgfqpoint{2.736711in}{0.632889in}}%
\pgfpathmoveto{\pgfqpoint{3.942469in}{0.632889in}}%
\pgfpathlineto{\pgfqpoint{3.339590in}{0.632889in}}%
\pgfpathmoveto{\pgfqpoint{4.545347in}{0.632889in}}%
\pgfpathlineto{\pgfqpoint{5.148226in}{0.632889in}}%
\pgfpathmoveto{\pgfqpoint{4.545347in}{0.632889in}}%
\pgfpathlineto{\pgfqpoint{3.942469in}{0.632889in}}%
\pgfpathmoveto{\pgfqpoint{5.151885in}{1.162689in}}%
\pgfpathlineto{\pgfqpoint{5.148226in}{0.632889in}}%
\pgfpathmoveto{\pgfqpoint{5.151885in}{1.162689in}}%
\pgfpathlineto{\pgfqpoint{5.144558in}{1.692655in}}%
\pgfpathmoveto{\pgfqpoint{4.507838in}{1.683975in}}%
\pgfpathlineto{\pgfqpoint{5.144558in}{1.692655in}}%
\pgfpathmoveto{\pgfqpoint{4.507838in}{1.683975in}}%
\pgfpathlineto{\pgfqpoint{3.861614in}{1.640658in}}%
\pgfpathmoveto{\pgfqpoint{3.799327in}{1.627708in}}%
\pgfpathlineto{\pgfqpoint{3.861614in}{1.640658in}}%
\pgfpathmoveto{\pgfqpoint{3.736583in}{1.613954in}}%
\pgfpathlineto{\pgfqpoint{3.799327in}{1.627708in}}%
\pgfpathmoveto{\pgfqpoint{3.674040in}{1.596375in}}%
\pgfpathlineto{\pgfqpoint{3.736583in}{1.613954in}}%
\pgfpathmoveto{\pgfqpoint{3.611695in}{1.567354in}}%
\pgfpathlineto{\pgfqpoint{3.674040in}{1.596375in}}%
\pgfpathmoveto{\pgfqpoint{3.571910in}{1.554619in}}%
\pgfpathlineto{\pgfqpoint{3.611695in}{1.567354in}}%
\pgfpathmoveto{\pgfqpoint{3.520173in}{1.540777in}}%
\pgfpathlineto{\pgfqpoint{3.571910in}{1.554619in}}%
\pgfpathmoveto{\pgfqpoint{3.466631in}{1.529168in}}%
\pgfpathlineto{\pgfqpoint{3.520173in}{1.540777in}}%
\pgfpathmoveto{\pgfqpoint{3.413805in}{1.518321in}}%
\pgfpathlineto{\pgfqpoint{3.466631in}{1.529168in}}%
\pgfpathmoveto{\pgfqpoint{3.358434in}{1.508375in}}%
\pgfpathlineto{\pgfqpoint{3.413805in}{1.518321in}}%
\pgfpathmoveto{\pgfqpoint{3.302098in}{1.500920in}}%
\pgfpathlineto{\pgfqpoint{3.358434in}{1.508375in}}%
\pgfpathmoveto{\pgfqpoint{3.244992in}{1.495350in}}%
\pgfpathlineto{\pgfqpoint{3.302098in}{1.500920in}}%
\pgfpathmoveto{\pgfqpoint{3.187184in}{1.491258in}}%
\pgfpathlineto{\pgfqpoint{3.244992in}{1.495350in}}%
\pgfpathmoveto{\pgfqpoint{3.128661in}{1.489268in}}%
\pgfpathlineto{\pgfqpoint{3.187184in}{1.491258in}}%
\pgfpathmoveto{\pgfqpoint{3.069249in}{1.488666in}}%
\pgfpathlineto{\pgfqpoint{3.128661in}{1.489268in}}%
\pgfpathmoveto{\pgfqpoint{3.009471in}{1.488021in}}%
\pgfpathlineto{\pgfqpoint{3.069249in}{1.488666in}}%
\pgfpathmoveto{\pgfqpoint{2.950249in}{1.489075in}}%
\pgfpathlineto{\pgfqpoint{3.009471in}{1.488021in}}%
\pgfpathmoveto{\pgfqpoint{2.891702in}{1.494160in}}%
\pgfpathlineto{\pgfqpoint{2.950249in}{1.489075in}}%
\pgfpathmoveto{\pgfqpoint{2.834312in}{1.498731in}}%
\pgfpathlineto{\pgfqpoint{2.891702in}{1.494160in}}%
\pgfpathmoveto{\pgfqpoint{2.776743in}{1.504940in}}%
\pgfpathlineto{\pgfqpoint{2.834312in}{1.498731in}}%
\pgfpathmoveto{\pgfqpoint{2.720995in}{1.510685in}}%
\pgfpathlineto{\pgfqpoint{2.776743in}{1.504940in}}%
\pgfpathmoveto{\pgfqpoint{2.664378in}{1.520170in}}%
\pgfpathlineto{\pgfqpoint{2.720995in}{1.510685in}}%
\pgfpathmoveto{\pgfqpoint{2.613282in}{1.532104in}}%
\pgfpathlineto{\pgfqpoint{2.664378in}{1.520170in}}%
\pgfpathmoveto{\pgfqpoint{2.559760in}{1.545331in}}%
\pgfpathlineto{\pgfqpoint{2.613282in}{1.532104in}}%
\pgfpathmoveto{\pgfqpoint{2.507966in}{1.559187in}}%
\pgfpathlineto{\pgfqpoint{2.559760in}{1.545331in}}%
\pgfpathmoveto{\pgfqpoint{2.468078in}{1.570732in}}%
\pgfpathlineto{\pgfqpoint{2.507966in}{1.559187in}}%
\pgfpathmoveto{\pgfqpoint{2.403928in}{1.597200in}}%
\pgfpathlineto{\pgfqpoint{2.468078in}{1.570732in}}%
\pgfpathmoveto{\pgfqpoint{2.340652in}{1.615657in}}%
\pgfpathlineto{\pgfqpoint{2.403928in}{1.597200in}}%
\pgfpathmoveto{\pgfqpoint{2.278119in}{1.631778in}}%
\pgfpathlineto{\pgfqpoint{2.217499in}{1.642801in}}%
\pgfpathmoveto{\pgfqpoint{2.278119in}{1.631778in}}%
\pgfpathlineto{\pgfqpoint{2.340652in}{1.615657in}}%
\pgfpathmoveto{\pgfqpoint{1.568742in}{1.685084in}}%
\pgfpathlineto{\pgfqpoint{2.217499in}{1.642801in}}%
\pgfpathmoveto{\pgfqpoint{1.568742in}{1.685084in}}%
\pgfpathlineto{\pgfqpoint{0.931733in}{1.692963in}}%
\pgfpathmoveto{\pgfqpoint{0.924612in}{1.162982in}}%
\pgfpathlineto{\pgfqpoint{0.928075in}{0.632889in}}%
\pgfpathmoveto{\pgfqpoint{0.924612in}{1.162982in}}%
\pgfpathlineto{\pgfqpoint{0.931733in}{1.692963in}}%
\pgfpathmoveto{\pgfqpoint{3.045912in}{1.028056in}}%
\pgfpathlineto{\pgfqpoint{2.736711in}{0.632889in}}%
\pgfpathmoveto{\pgfqpoint{3.045912in}{1.028056in}}%
\pgfpathlineto{\pgfqpoint{3.339590in}{0.632889in}}%
\pgfpathmoveto{\pgfqpoint{3.596286in}{1.071310in}}%
\pgfpathlineto{\pgfqpoint{3.339590in}{0.632889in}}%
\pgfpathmoveto{\pgfqpoint{3.596286in}{1.071310in}}%
\pgfpathlineto{\pgfqpoint{3.942469in}{0.632889in}}%
\pgfpathmoveto{\pgfqpoint{3.274321in}{1.249592in}}%
\pgfpathlineto{\pgfqpoint{3.045912in}{1.028056in}}%
\pgfpathmoveto{\pgfqpoint{3.515232in}{1.320497in}}%
\pgfpathlineto{\pgfqpoint{3.596286in}{1.071310in}}%
\pgfpathmoveto{\pgfqpoint{3.515232in}{1.320497in}}%
\pgfpathlineto{\pgfqpoint{3.274321in}{1.249592in}}%
\pgfpathmoveto{\pgfqpoint{3.800381in}{1.338560in}}%
\pgfpathlineto{\pgfqpoint{3.861614in}{1.640658in}}%
\pgfpathmoveto{\pgfqpoint{3.800381in}{1.338560in}}%
\pgfpathlineto{\pgfqpoint{3.596286in}{1.071310in}}%
\pgfpathmoveto{\pgfqpoint{3.800381in}{1.338560in}}%
\pgfpathlineto{\pgfqpoint{3.515232in}{1.320497in}}%
\pgfpathmoveto{\pgfqpoint{3.062572in}{1.272253in}}%
\pgfpathlineto{\pgfqpoint{3.045912in}{1.028056in}}%
\pgfpathmoveto{\pgfqpoint{3.062572in}{1.272253in}}%
\pgfpathlineto{\pgfqpoint{3.274321in}{1.249592in}}%
\pgfpathmoveto{\pgfqpoint{2.706671in}{1.278568in}}%
\pgfpathlineto{\pgfqpoint{2.457984in}{1.166425in}}%
\pgfpathmoveto{\pgfqpoint{2.503414in}{1.351810in}}%
\pgfpathlineto{\pgfqpoint{2.457984in}{1.166425in}}%
\pgfpathmoveto{\pgfqpoint{2.503414in}{1.351810in}}%
\pgfpathlineto{\pgfqpoint{2.706671in}{1.278568in}}%
\pgfpathmoveto{\pgfqpoint{2.268982in}{1.402564in}}%
\pgfpathlineto{\pgfqpoint{2.217499in}{1.642801in}}%
\pgfpathmoveto{\pgfqpoint{2.268982in}{1.402564in}}%
\pgfpathlineto{\pgfqpoint{2.278119in}{1.631778in}}%
\pgfpathmoveto{\pgfqpoint{2.268982in}{1.402564in}}%
\pgfpathlineto{\pgfqpoint{2.457984in}{1.166425in}}%
\pgfpathmoveto{\pgfqpoint{2.268982in}{1.402564in}}%
\pgfpathlineto{\pgfqpoint{2.503414in}{1.351810in}}%
\pgfpathmoveto{\pgfqpoint{2.843083in}{1.371232in}}%
\pgfpathlineto{\pgfqpoint{2.706671in}{1.278568in}}%
\pgfpathmoveto{\pgfqpoint{3.157280in}{1.374466in}}%
\pgfpathlineto{\pgfqpoint{3.274321in}{1.249592in}}%
\pgfpathmoveto{\pgfqpoint{3.157280in}{1.374466in}}%
\pgfpathlineto{\pgfqpoint{3.062572in}{1.272253in}}%
\pgfpathmoveto{\pgfqpoint{3.622615in}{1.433023in}}%
\pgfpathlineto{\pgfqpoint{3.515232in}{1.320497in}}%
\pgfpathmoveto{\pgfqpoint{3.622615in}{1.433023in}}%
\pgfpathlineto{\pgfqpoint{3.800381in}{1.338560in}}%
\pgfpathmoveto{\pgfqpoint{3.391465in}{1.392756in}}%
\pgfpathlineto{\pgfqpoint{3.274321in}{1.249592in}}%
\pgfpathmoveto{\pgfqpoint{3.391465in}{1.392756in}}%
\pgfpathlineto{\pgfqpoint{3.515232in}{1.320497in}}%
\pgfpathmoveto{\pgfqpoint{2.628264in}{1.408907in}}%
\pgfpathlineto{\pgfqpoint{2.706671in}{1.278568in}}%
\pgfpathmoveto{\pgfqpoint{2.628264in}{1.408907in}}%
\pgfpathlineto{\pgfqpoint{2.503414in}{1.351810in}}%
\pgfpathmoveto{\pgfqpoint{2.737162in}{1.404189in}}%
\pgfpathlineto{\pgfqpoint{2.706671in}{1.278568in}}%
\pgfpathmoveto{\pgfqpoint{2.737162in}{1.404189in}}%
\pgfpathlineto{\pgfqpoint{2.843083in}{1.371232in}}%
\pgfpathmoveto{\pgfqpoint{2.737162in}{1.404189in}}%
\pgfpathlineto{\pgfqpoint{2.628264in}{1.408907in}}%
\pgfpathmoveto{\pgfqpoint{2.519423in}{1.450339in}}%
\pgfpathlineto{\pgfqpoint{2.503414in}{1.351810in}}%
\pgfpathmoveto{\pgfqpoint{2.519423in}{1.450339in}}%
\pgfpathlineto{\pgfqpoint{2.628264in}{1.408907in}}%
\pgfpathmoveto{\pgfqpoint{2.393856in}{1.508447in}}%
\pgfpathlineto{\pgfqpoint{2.403928in}{1.597200in}}%
\pgfpathmoveto{\pgfqpoint{2.393856in}{1.508447in}}%
\pgfpathlineto{\pgfqpoint{2.340652in}{1.615657in}}%
\pgfpathmoveto{\pgfqpoint{2.393856in}{1.508447in}}%
\pgfpathlineto{\pgfqpoint{2.278119in}{1.631778in}}%
\pgfpathmoveto{\pgfqpoint{2.393856in}{1.508447in}}%
\pgfpathlineto{\pgfqpoint{2.503414in}{1.351810in}}%
\pgfpathmoveto{\pgfqpoint{2.393856in}{1.508447in}}%
\pgfpathlineto{\pgfqpoint{2.268982in}{1.402564in}}%
\pgfpathmoveto{\pgfqpoint{2.393856in}{1.508447in}}%
\pgfpathlineto{\pgfqpoint{2.519423in}{1.450339in}}%
\pgfpathmoveto{\pgfqpoint{3.273703in}{1.392850in}}%
\pgfpathlineto{\pgfqpoint{3.274321in}{1.249592in}}%
\pgfpathmoveto{\pgfqpoint{3.273703in}{1.392850in}}%
\pgfpathlineto{\pgfqpoint{3.157280in}{1.374466in}}%
\pgfpathmoveto{\pgfqpoint{3.273703in}{1.392850in}}%
\pgfpathlineto{\pgfqpoint{3.391465in}{1.392756in}}%
\pgfpathmoveto{\pgfqpoint{3.044666in}{1.392220in}}%
\pgfpathlineto{\pgfqpoint{3.062572in}{1.272253in}}%
\pgfpathmoveto{\pgfqpoint{3.044666in}{1.392220in}}%
\pgfpathlineto{\pgfqpoint{3.157280in}{1.374466in}}%
\pgfpathmoveto{\pgfqpoint{3.505085in}{1.430020in}}%
\pgfpathlineto{\pgfqpoint{3.515232in}{1.320497in}}%
\pgfpathmoveto{\pgfqpoint{3.505085in}{1.430020in}}%
\pgfpathlineto{\pgfqpoint{3.622615in}{1.433023in}}%
\pgfpathmoveto{\pgfqpoint{3.505085in}{1.430020in}}%
\pgfpathlineto{\pgfqpoint{3.391465in}{1.392756in}}%
\pgfpathmoveto{\pgfqpoint{3.731493in}{1.517889in}}%
\pgfpathlineto{\pgfqpoint{3.861614in}{1.640658in}}%
\pgfpathmoveto{\pgfqpoint{3.731493in}{1.517889in}}%
\pgfpathlineto{\pgfqpoint{3.799327in}{1.627708in}}%
\pgfpathmoveto{\pgfqpoint{3.731493in}{1.517889in}}%
\pgfpathlineto{\pgfqpoint{3.736583in}{1.613954in}}%
\pgfpathmoveto{\pgfqpoint{3.731493in}{1.517889in}}%
\pgfpathlineto{\pgfqpoint{3.674040in}{1.596375in}}%
\pgfpathmoveto{\pgfqpoint{3.731493in}{1.517889in}}%
\pgfpathlineto{\pgfqpoint{3.800381in}{1.338560in}}%
\pgfpathmoveto{\pgfqpoint{3.731493in}{1.517889in}}%
\pgfpathlineto{\pgfqpoint{3.622615in}{1.433023in}}%
\pgfpathmoveto{\pgfqpoint{2.921097in}{1.420627in}}%
\pgfpathlineto{\pgfqpoint{2.950249in}{1.489075in}}%
\pgfpathmoveto{\pgfqpoint{2.921097in}{1.420627in}}%
\pgfpathlineto{\pgfqpoint{2.891702in}{1.494160in}}%
\pgfpathmoveto{\pgfqpoint{2.921097in}{1.420627in}}%
\pgfpathlineto{\pgfqpoint{2.843083in}{1.371232in}}%
\pgfpathmoveto{\pgfqpoint{1.690003in}{1.133260in}}%
\pgfpathlineto{\pgfqpoint{1.530954in}{0.632889in}}%
\pgfpathmoveto{\pgfqpoint{1.690003in}{1.133260in}}%
\pgfpathlineto{\pgfqpoint{2.133832in}{0.632889in}}%
\pgfpathmoveto{\pgfqpoint{1.690003in}{1.133260in}}%
\pgfpathlineto{\pgfqpoint{1.568742in}{1.685084in}}%
\pgfpathmoveto{\pgfqpoint{4.347185in}{1.132943in}}%
\pgfpathlineto{\pgfqpoint{3.942469in}{0.632889in}}%
\pgfpathmoveto{\pgfqpoint{4.347185in}{1.132943in}}%
\pgfpathlineto{\pgfqpoint{4.545347in}{0.632889in}}%
\pgfpathmoveto{\pgfqpoint{4.347185in}{1.132943in}}%
\pgfpathlineto{\pgfqpoint{4.507838in}{1.683975in}}%
\pgfpathmoveto{\pgfqpoint{3.098254in}{1.435453in}}%
\pgfpathlineto{\pgfqpoint{3.128661in}{1.489268in}}%
\pgfpathmoveto{\pgfqpoint{3.098254in}{1.435453in}}%
\pgfpathlineto{\pgfqpoint{3.069249in}{1.488666in}}%
\pgfpathmoveto{\pgfqpoint{3.098254in}{1.435453in}}%
\pgfpathlineto{\pgfqpoint{3.157280in}{1.374466in}}%
\pgfpathmoveto{\pgfqpoint{3.098254in}{1.435453in}}%
\pgfpathlineto{\pgfqpoint{3.044666in}{1.392220in}}%
\pgfpathmoveto{\pgfqpoint{3.215607in}{1.439494in}}%
\pgfpathlineto{\pgfqpoint{3.244992in}{1.495350in}}%
\pgfpathmoveto{\pgfqpoint{3.215607in}{1.439494in}}%
\pgfpathlineto{\pgfqpoint{3.187184in}{1.491258in}}%
\pgfpathmoveto{\pgfqpoint{3.215607in}{1.439494in}}%
\pgfpathlineto{\pgfqpoint{3.157280in}{1.374466in}}%
\pgfpathmoveto{\pgfqpoint{3.215607in}{1.439494in}}%
\pgfpathlineto{\pgfqpoint{3.273703in}{1.392850in}}%
\pgfpathmoveto{\pgfqpoint{3.331265in}{1.449502in}}%
\pgfpathlineto{\pgfqpoint{3.358434in}{1.508375in}}%
\pgfpathmoveto{\pgfqpoint{3.331265in}{1.449502in}}%
\pgfpathlineto{\pgfqpoint{3.302098in}{1.500920in}}%
\pgfpathmoveto{\pgfqpoint{3.331265in}{1.449502in}}%
\pgfpathlineto{\pgfqpoint{3.391465in}{1.392756in}}%
\pgfpathmoveto{\pgfqpoint{3.331265in}{1.449502in}}%
\pgfpathlineto{\pgfqpoint{3.273703in}{1.392850in}}%
\pgfpathmoveto{\pgfqpoint{2.690937in}{1.459550in}}%
\pgfpathlineto{\pgfqpoint{2.720995in}{1.510685in}}%
\pgfpathmoveto{\pgfqpoint{2.690937in}{1.459550in}}%
\pgfpathlineto{\pgfqpoint{2.664378in}{1.520170in}}%
\pgfpathmoveto{\pgfqpoint{2.690937in}{1.459550in}}%
\pgfpathlineto{\pgfqpoint{2.628264in}{1.408907in}}%
\pgfpathmoveto{\pgfqpoint{2.690937in}{1.459550in}}%
\pgfpathlineto{\pgfqpoint{2.737162in}{1.404189in}}%
\pgfpathmoveto{\pgfqpoint{2.579475in}{1.479461in}}%
\pgfpathlineto{\pgfqpoint{2.613282in}{1.532104in}}%
\pgfpathmoveto{\pgfqpoint{2.579475in}{1.479461in}}%
\pgfpathlineto{\pgfqpoint{2.559760in}{1.545331in}}%
\pgfpathmoveto{\pgfqpoint{2.579475in}{1.479461in}}%
\pgfpathlineto{\pgfqpoint{2.628264in}{1.408907in}}%
\pgfpathmoveto{\pgfqpoint{2.579475in}{1.479461in}}%
\pgfpathlineto{\pgfqpoint{2.519423in}{1.450339in}}%
\pgfpathmoveto{\pgfqpoint{2.800571in}{1.449704in}}%
\pgfpathlineto{\pgfqpoint{2.834312in}{1.498731in}}%
\pgfpathmoveto{\pgfqpoint{2.800571in}{1.449704in}}%
\pgfpathlineto{\pgfqpoint{2.776743in}{1.504940in}}%
\pgfpathmoveto{\pgfqpoint{2.800571in}{1.449704in}}%
\pgfpathlineto{\pgfqpoint{2.843083in}{1.371232in}}%
\pgfpathmoveto{\pgfqpoint{2.800571in}{1.449704in}}%
\pgfpathlineto{\pgfqpoint{2.737162in}{1.404189in}}%
\pgfpathmoveto{\pgfqpoint{3.554663in}{1.487641in}}%
\pgfpathlineto{\pgfqpoint{3.571910in}{1.554619in}}%
\pgfpathmoveto{\pgfqpoint{3.554663in}{1.487641in}}%
\pgfpathlineto{\pgfqpoint{3.520173in}{1.540777in}}%
\pgfpathmoveto{\pgfqpoint{3.554663in}{1.487641in}}%
\pgfpathlineto{\pgfqpoint{3.622615in}{1.433023in}}%
\pgfpathmoveto{\pgfqpoint{3.554663in}{1.487641in}}%
\pgfpathlineto{\pgfqpoint{3.505085in}{1.430020in}}%
\pgfpathmoveto{\pgfqpoint{3.444488in}{1.465811in}}%
\pgfpathlineto{\pgfqpoint{3.466631in}{1.529168in}}%
\pgfpathmoveto{\pgfqpoint{3.444488in}{1.465811in}}%
\pgfpathlineto{\pgfqpoint{3.413805in}{1.518321in}}%
\pgfpathmoveto{\pgfqpoint{3.444488in}{1.465811in}}%
\pgfpathlineto{\pgfqpoint{3.391465in}{1.392756in}}%
\pgfpathmoveto{\pgfqpoint{3.444488in}{1.465811in}}%
\pgfpathlineto{\pgfqpoint{3.505085in}{1.430020in}}%
\pgfpathmoveto{\pgfqpoint{2.136587in}{1.088915in}}%
\pgfpathlineto{\pgfqpoint{2.133832in}{0.632889in}}%
\pgfpathmoveto{\pgfqpoint{2.136587in}{1.088915in}}%
\pgfpathlineto{\pgfqpoint{2.457984in}{1.166425in}}%
\pgfpathmoveto{\pgfqpoint{2.136587in}{1.088915in}}%
\pgfpathlineto{\pgfqpoint{2.268982in}{1.402564in}}%
\pgfpathmoveto{\pgfqpoint{2.136587in}{1.088915in}}%
\pgfpathlineto{\pgfqpoint{1.690003in}{1.133260in}}%
\pgfpathmoveto{\pgfqpoint{3.962068in}{1.087602in}}%
\pgfpathlineto{\pgfqpoint{3.942469in}{0.632889in}}%
\pgfpathmoveto{\pgfqpoint{3.962068in}{1.087602in}}%
\pgfpathlineto{\pgfqpoint{3.596286in}{1.071310in}}%
\pgfpathmoveto{\pgfqpoint{3.962068in}{1.087602in}}%
\pgfpathlineto{\pgfqpoint{3.800381in}{1.338560in}}%
\pgfpathmoveto{\pgfqpoint{3.962068in}{1.087602in}}%
\pgfpathlineto{\pgfqpoint{4.347185in}{1.132943in}}%
\pgfpathmoveto{\pgfqpoint{2.988117in}{1.431240in}}%
\pgfpathlineto{\pgfqpoint{3.009471in}{1.488021in}}%
\pgfpathmoveto{\pgfqpoint{2.988117in}{1.431240in}}%
\pgfpathlineto{\pgfqpoint{2.950249in}{1.489075in}}%
\pgfpathmoveto{\pgfqpoint{2.988117in}{1.431240in}}%
\pgfpathlineto{\pgfqpoint{3.044666in}{1.392220in}}%
\pgfpathmoveto{\pgfqpoint{2.988117in}{1.431240in}}%
\pgfpathlineto{\pgfqpoint{2.921097in}{1.420627in}}%
\pgfpathmoveto{\pgfqpoint{2.696400in}{1.030955in}}%
\pgfpathlineto{\pgfqpoint{2.736711in}{0.632889in}}%
\pgfpathmoveto{\pgfqpoint{2.696400in}{1.030955in}}%
\pgfpathlineto{\pgfqpoint{3.045912in}{1.028056in}}%
\pgfpathmoveto{\pgfqpoint{2.696400in}{1.030955in}}%
\pgfpathlineto{\pgfqpoint{2.457984in}{1.166425in}}%
\pgfpathmoveto{\pgfqpoint{2.696400in}{1.030955in}}%
\pgfpathlineto{\pgfqpoint{2.706671in}{1.278568in}}%
\pgfpathmoveto{\pgfqpoint{3.323419in}{1.024059in}}%
\pgfpathlineto{\pgfqpoint{3.339590in}{0.632889in}}%
\pgfpathmoveto{\pgfqpoint{3.323419in}{1.024059in}}%
\pgfpathlineto{\pgfqpoint{3.045912in}{1.028056in}}%
\pgfpathmoveto{\pgfqpoint{3.323419in}{1.024059in}}%
\pgfpathlineto{\pgfqpoint{3.596286in}{1.071310in}}%
\pgfpathmoveto{\pgfqpoint{3.323419in}{1.024059in}}%
\pgfpathlineto{\pgfqpoint{3.274321in}{1.249592in}}%
\pgfpathmoveto{\pgfqpoint{1.972803in}{1.394417in}}%
\pgfpathlineto{\pgfqpoint{2.217499in}{1.642801in}}%
\pgfpathmoveto{\pgfqpoint{1.972803in}{1.394417in}}%
\pgfpathlineto{\pgfqpoint{1.568742in}{1.685084in}}%
\pgfpathmoveto{\pgfqpoint{1.972803in}{1.394417in}}%
\pgfpathlineto{\pgfqpoint{2.268982in}{1.402564in}}%
\pgfpathmoveto{\pgfqpoint{1.972803in}{1.394417in}}%
\pgfpathlineto{\pgfqpoint{1.690003in}{1.133260in}}%
\pgfpathmoveto{\pgfqpoint{1.972803in}{1.394417in}}%
\pgfpathlineto{\pgfqpoint{2.136587in}{1.088915in}}%
\pgfpathmoveto{\pgfqpoint{2.884920in}{1.220865in}}%
\pgfpathlineto{\pgfqpoint{3.045912in}{1.028056in}}%
\pgfpathmoveto{\pgfqpoint{2.884920in}{1.220865in}}%
\pgfpathlineto{\pgfqpoint{3.062572in}{1.272253in}}%
\pgfpathmoveto{\pgfqpoint{2.884920in}{1.220865in}}%
\pgfpathlineto{\pgfqpoint{2.706671in}{1.278568in}}%
\pgfpathmoveto{\pgfqpoint{2.884920in}{1.220865in}}%
\pgfpathlineto{\pgfqpoint{2.843083in}{1.371232in}}%
\pgfpathmoveto{\pgfqpoint{2.884920in}{1.220865in}}%
\pgfpathlineto{\pgfqpoint{2.696400in}{1.030955in}}%
\pgfpathmoveto{\pgfqpoint{3.273419in}{1.455354in}}%
\pgfpathlineto{\pgfqpoint{3.302098in}{1.500920in}}%
\pgfpathmoveto{\pgfqpoint{3.273419in}{1.455354in}}%
\pgfpathlineto{\pgfqpoint{3.244992in}{1.495350in}}%
\pgfpathmoveto{\pgfqpoint{3.273419in}{1.455354in}}%
\pgfpathlineto{\pgfqpoint{3.273703in}{1.392850in}}%
\pgfpathmoveto{\pgfqpoint{3.273419in}{1.455354in}}%
\pgfpathlineto{\pgfqpoint{3.215607in}{1.439494in}}%
\pgfpathmoveto{\pgfqpoint{3.273419in}{1.455354in}}%
\pgfpathlineto{\pgfqpoint{3.331265in}{1.449502in}}%
\pgfpathmoveto{\pgfqpoint{3.157127in}{1.445847in}}%
\pgfpathlineto{\pgfqpoint{3.187184in}{1.491258in}}%
\pgfpathmoveto{\pgfqpoint{3.157127in}{1.445847in}}%
\pgfpathlineto{\pgfqpoint{3.128661in}{1.489268in}}%
\pgfpathmoveto{\pgfqpoint{3.157127in}{1.445847in}}%
\pgfpathlineto{\pgfqpoint{3.157280in}{1.374466in}}%
\pgfpathmoveto{\pgfqpoint{3.157127in}{1.445847in}}%
\pgfpathlineto{\pgfqpoint{3.098254in}{1.435453in}}%
\pgfpathmoveto{\pgfqpoint{3.157127in}{1.445847in}}%
\pgfpathlineto{\pgfqpoint{3.215607in}{1.439494in}}%
\pgfpathmoveto{\pgfqpoint{3.387742in}{1.468515in}}%
\pgfpathlineto{\pgfqpoint{3.413805in}{1.518321in}}%
\pgfpathmoveto{\pgfqpoint{3.387742in}{1.468515in}}%
\pgfpathlineto{\pgfqpoint{3.358434in}{1.508375in}}%
\pgfpathmoveto{\pgfqpoint{3.387742in}{1.468515in}}%
\pgfpathlineto{\pgfqpoint{3.391465in}{1.392756in}}%
\pgfpathmoveto{\pgfqpoint{3.387742in}{1.468515in}}%
\pgfpathlineto{\pgfqpoint{3.331265in}{1.449502in}}%
\pgfpathmoveto{\pgfqpoint{3.387742in}{1.468515in}}%
\pgfpathlineto{\pgfqpoint{3.444488in}{1.465811in}}%
\pgfpathmoveto{\pgfqpoint{2.745587in}{1.465477in}}%
\pgfpathlineto{\pgfqpoint{2.776743in}{1.504940in}}%
\pgfpathmoveto{\pgfqpoint{2.745587in}{1.465477in}}%
\pgfpathlineto{\pgfqpoint{2.720995in}{1.510685in}}%
\pgfpathmoveto{\pgfqpoint{2.745587in}{1.465477in}}%
\pgfpathlineto{\pgfqpoint{2.737162in}{1.404189in}}%
\pgfpathmoveto{\pgfqpoint{2.745587in}{1.465477in}}%
\pgfpathlineto{\pgfqpoint{2.690937in}{1.459550in}}%
\pgfpathmoveto{\pgfqpoint{2.745587in}{1.465477in}}%
\pgfpathlineto{\pgfqpoint{2.800571in}{1.449704in}}%
\pgfpathmoveto{\pgfqpoint{2.635254in}{1.479256in}}%
\pgfpathlineto{\pgfqpoint{2.664378in}{1.520170in}}%
\pgfpathmoveto{\pgfqpoint{2.635254in}{1.479256in}}%
\pgfpathlineto{\pgfqpoint{2.613282in}{1.532104in}}%
\pgfpathmoveto{\pgfqpoint{2.635254in}{1.479256in}}%
\pgfpathlineto{\pgfqpoint{2.628264in}{1.408907in}}%
\pgfpathmoveto{\pgfqpoint{2.635254in}{1.479256in}}%
\pgfpathlineto{\pgfqpoint{2.690937in}{1.459550in}}%
\pgfpathmoveto{\pgfqpoint{2.635254in}{1.479256in}}%
\pgfpathlineto{\pgfqpoint{2.579475in}{1.479461in}}%
\pgfpathmoveto{\pgfqpoint{2.527949in}{1.510578in}}%
\pgfpathlineto{\pgfqpoint{2.559760in}{1.545331in}}%
\pgfpathmoveto{\pgfqpoint{2.527949in}{1.510578in}}%
\pgfpathlineto{\pgfqpoint{2.507966in}{1.559187in}}%
\pgfpathmoveto{\pgfqpoint{2.527949in}{1.510578in}}%
\pgfpathlineto{\pgfqpoint{2.519423in}{1.450339in}}%
\pgfpathmoveto{\pgfqpoint{2.527949in}{1.510578in}}%
\pgfpathlineto{\pgfqpoint{2.579475in}{1.479461in}}%
\pgfpathmoveto{\pgfqpoint{2.474147in}{1.519645in}}%
\pgfpathlineto{\pgfqpoint{2.507966in}{1.559187in}}%
\pgfpathmoveto{\pgfqpoint{2.474147in}{1.519645in}}%
\pgfpathlineto{\pgfqpoint{2.468078in}{1.570732in}}%
\pgfpathmoveto{\pgfqpoint{2.474147in}{1.519645in}}%
\pgfpathlineto{\pgfqpoint{2.403928in}{1.597200in}}%
\pgfpathmoveto{\pgfqpoint{2.474147in}{1.519645in}}%
\pgfpathlineto{\pgfqpoint{2.519423in}{1.450339in}}%
\pgfpathmoveto{\pgfqpoint{2.474147in}{1.519645in}}%
\pgfpathlineto{\pgfqpoint{2.393856in}{1.508447in}}%
\pgfpathmoveto{\pgfqpoint{2.474147in}{1.519645in}}%
\pgfpathlineto{\pgfqpoint{2.527949in}{1.510578in}}%
\pgfpathmoveto{\pgfqpoint{2.862994in}{1.454643in}}%
\pgfpathlineto{\pgfqpoint{2.891702in}{1.494160in}}%
\pgfpathmoveto{\pgfqpoint{2.862994in}{1.454643in}}%
\pgfpathlineto{\pgfqpoint{2.834312in}{1.498731in}}%
\pgfpathmoveto{\pgfqpoint{2.862994in}{1.454643in}}%
\pgfpathlineto{\pgfqpoint{2.843083in}{1.371232in}}%
\pgfpathmoveto{\pgfqpoint{2.862994in}{1.454643in}}%
\pgfpathlineto{\pgfqpoint{2.921097in}{1.420627in}}%
\pgfpathmoveto{\pgfqpoint{2.862994in}{1.454643in}}%
\pgfpathlineto{\pgfqpoint{2.800571in}{1.449704in}}%
\pgfpathmoveto{\pgfqpoint{3.042050in}{1.447133in}}%
\pgfpathlineto{\pgfqpoint{3.069249in}{1.488666in}}%
\pgfpathmoveto{\pgfqpoint{3.042050in}{1.447133in}}%
\pgfpathlineto{\pgfqpoint{3.009471in}{1.488021in}}%
\pgfpathmoveto{\pgfqpoint{3.042050in}{1.447133in}}%
\pgfpathlineto{\pgfqpoint{3.044666in}{1.392220in}}%
\pgfpathmoveto{\pgfqpoint{3.042050in}{1.447133in}}%
\pgfpathlineto{\pgfqpoint{3.098254in}{1.435453in}}%
\pgfpathmoveto{\pgfqpoint{3.042050in}{1.447133in}}%
\pgfpathlineto{\pgfqpoint{2.988117in}{1.431240in}}%
\pgfpathmoveto{\pgfqpoint{4.098786in}{1.380487in}}%
\pgfpathlineto{\pgfqpoint{3.861614in}{1.640658in}}%
\pgfpathmoveto{\pgfqpoint{4.098786in}{1.380487in}}%
\pgfpathlineto{\pgfqpoint{4.507838in}{1.683975in}}%
\pgfpathmoveto{\pgfqpoint{4.098786in}{1.380487in}}%
\pgfpathlineto{\pgfqpoint{3.800381in}{1.338560in}}%
\pgfpathmoveto{\pgfqpoint{4.098786in}{1.380487in}}%
\pgfpathlineto{\pgfqpoint{4.347185in}{1.132943in}}%
\pgfpathmoveto{\pgfqpoint{4.098786in}{1.380487in}}%
\pgfpathlineto{\pgfqpoint{3.962068in}{1.087602in}}%
\pgfpathmoveto{\pgfqpoint{3.498249in}{1.490378in}}%
\pgfpathlineto{\pgfqpoint{3.520173in}{1.540777in}}%
\pgfpathmoveto{\pgfqpoint{3.498249in}{1.490378in}}%
\pgfpathlineto{\pgfqpoint{3.466631in}{1.529168in}}%
\pgfpathmoveto{\pgfqpoint{3.498249in}{1.490378in}}%
\pgfpathlineto{\pgfqpoint{3.505085in}{1.430020in}}%
\pgfpathmoveto{\pgfqpoint{3.498249in}{1.490378in}}%
\pgfpathlineto{\pgfqpoint{3.554663in}{1.487641in}}%
\pgfpathmoveto{\pgfqpoint{3.498249in}{1.490378in}}%
\pgfpathlineto{\pgfqpoint{3.444488in}{1.465811in}}%
\pgfpathmoveto{\pgfqpoint{3.629301in}{1.522720in}}%
\pgfpathlineto{\pgfqpoint{3.674040in}{1.596375in}}%
\pgfpathmoveto{\pgfqpoint{3.629301in}{1.522720in}}%
\pgfpathlineto{\pgfqpoint{3.611695in}{1.567354in}}%
\pgfpathmoveto{\pgfqpoint{3.629301in}{1.522720in}}%
\pgfpathlineto{\pgfqpoint{3.571910in}{1.554619in}}%
\pgfpathmoveto{\pgfqpoint{3.629301in}{1.522720in}}%
\pgfpathlineto{\pgfqpoint{3.622615in}{1.433023in}}%
\pgfpathmoveto{\pgfqpoint{3.629301in}{1.522720in}}%
\pgfpathlineto{\pgfqpoint{3.731493in}{1.517889in}}%
\pgfpathmoveto{\pgfqpoint{3.629301in}{1.522720in}}%
\pgfpathlineto{\pgfqpoint{3.554663in}{1.487641in}}%
\pgfpathmoveto{\pgfqpoint{1.316944in}{1.024950in}}%
\pgfpathlineto{\pgfqpoint{0.928075in}{0.632889in}}%
\pgfpathmoveto{\pgfqpoint{1.316944in}{1.024950in}}%
\pgfpathlineto{\pgfqpoint{1.530954in}{0.632889in}}%
\pgfpathmoveto{\pgfqpoint{1.316944in}{1.024950in}}%
\pgfpathlineto{\pgfqpoint{0.924612in}{1.162982in}}%
\pgfpathmoveto{\pgfqpoint{1.316944in}{1.024950in}}%
\pgfpathlineto{\pgfqpoint{1.690003in}{1.133260in}}%
\pgfpathmoveto{\pgfqpoint{4.773542in}{1.012203in}}%
\pgfpathlineto{\pgfqpoint{5.148226in}{0.632889in}}%
\pgfpathmoveto{\pgfqpoint{4.773542in}{1.012203in}}%
\pgfpathlineto{\pgfqpoint{4.545347in}{0.632889in}}%
\pgfpathmoveto{\pgfqpoint{4.773542in}{1.012203in}}%
\pgfpathlineto{\pgfqpoint{5.151885in}{1.162689in}}%
\pgfpathmoveto{\pgfqpoint{4.773542in}{1.012203in}}%
\pgfpathlineto{\pgfqpoint{4.347185in}{1.132943in}}%
\pgfpathmoveto{\pgfqpoint{4.799510in}{1.363558in}}%
\pgfpathlineto{\pgfqpoint{5.144558in}{1.692655in}}%
\pgfpathmoveto{\pgfqpoint{4.799510in}{1.363558in}}%
\pgfpathlineto{\pgfqpoint{5.151885in}{1.162689in}}%
\pgfpathmoveto{\pgfqpoint{4.799510in}{1.363558in}}%
\pgfpathlineto{\pgfqpoint{4.507838in}{1.683975in}}%
\pgfpathmoveto{\pgfqpoint{4.799510in}{1.363558in}}%
\pgfpathlineto{\pgfqpoint{4.347185in}{1.132943in}}%
\pgfpathmoveto{\pgfqpoint{4.799510in}{1.363558in}}%
\pgfpathlineto{\pgfqpoint{4.773542in}{1.012203in}}%
\pgfpathmoveto{\pgfqpoint{1.270728in}{1.365716in}}%
\pgfpathlineto{\pgfqpoint{0.931733in}{1.692963in}}%
\pgfpathmoveto{\pgfqpoint{1.270728in}{1.365716in}}%
\pgfpathlineto{\pgfqpoint{1.568742in}{1.685084in}}%
\pgfpathmoveto{\pgfqpoint{1.270728in}{1.365716in}}%
\pgfpathlineto{\pgfqpoint{0.924612in}{1.162982in}}%
\pgfpathmoveto{\pgfqpoint{1.270728in}{1.365716in}}%
\pgfpathlineto{\pgfqpoint{1.690003in}{1.133260in}}%
\pgfpathmoveto{\pgfqpoint{1.270728in}{1.365716in}}%
\pgfpathlineto{\pgfqpoint{1.316944in}{1.024950in}}%
\pgfpathmoveto{\pgfqpoint{2.962413in}{1.348520in}}%
\pgfpathlineto{\pgfqpoint{3.062572in}{1.272253in}}%
\pgfpathmoveto{\pgfqpoint{2.962413in}{1.348520in}}%
\pgfpathlineto{\pgfqpoint{2.843083in}{1.371232in}}%
\pgfpathmoveto{\pgfqpoint{2.962413in}{1.348520in}}%
\pgfpathlineto{\pgfqpoint{3.044666in}{1.392220in}}%
\pgfpathmoveto{\pgfqpoint{2.962413in}{1.348520in}}%
\pgfpathlineto{\pgfqpoint{2.921097in}{1.420627in}}%
\pgfpathmoveto{\pgfqpoint{2.962413in}{1.348520in}}%
\pgfpathlineto{\pgfqpoint{2.988117in}{1.431240in}}%
\pgfpathmoveto{\pgfqpoint{2.962413in}{1.348520in}}%
\pgfpathlineto{\pgfqpoint{2.884920in}{1.220865in}}%
\pgfpathmoveto{\pgfqpoint{2.406476in}{0.891397in}}%
\pgfpathlineto{\pgfqpoint{2.133832in}{0.632889in}}%
\pgfpathmoveto{\pgfqpoint{2.406476in}{0.891397in}}%
\pgfpathlineto{\pgfqpoint{2.736711in}{0.632889in}}%
\pgfpathmoveto{\pgfqpoint{2.406476in}{0.891397in}}%
\pgfpathlineto{\pgfqpoint{2.457984in}{1.166425in}}%
\pgfpathmoveto{\pgfqpoint{2.406476in}{0.891397in}}%
\pgfpathlineto{\pgfqpoint{2.136587in}{1.088915in}}%
\pgfpathmoveto{\pgfqpoint{2.406476in}{0.891397in}}%
\pgfpathlineto{\pgfqpoint{2.696400in}{1.030955in}}%
\pgfpathmoveto{\pgfqpoint{3.428270in}{1.166010in}}%
\pgfpathlineto{\pgfqpoint{3.596286in}{1.071310in}}%
\pgfpathmoveto{\pgfqpoint{3.428270in}{1.166010in}}%
\pgfpathlineto{\pgfqpoint{3.274321in}{1.249592in}}%
\pgfpathmoveto{\pgfqpoint{3.428270in}{1.166010in}}%
\pgfpathlineto{\pgfqpoint{3.515232in}{1.320497in}}%
\pgfpathmoveto{\pgfqpoint{3.428270in}{1.166010in}}%
\pgfpathlineto{\pgfqpoint{3.323419in}{1.024059in}}%
\pgfpathlineto{\pgfqpoint{3.323419in}{1.024059in}}%
\pgfusepath{stroke}%
\end{pgfscope}%
\begin{pgfscope}%
\pgfpathrectangle{\pgfqpoint{0.713248in}{0.548486in}}{\pgfqpoint{4.650000in}{1.856867in}}%
\pgfusepath{clip}%
\pgfsetrectcap%
\pgfsetroundjoin%
\pgfsetlinewidth{0.501875pt}%
\definecolor{currentstroke}{rgb}{0.835294,0.321569,0.035294}%
\pgfsetstrokecolor{currentstroke}%
\pgfsetdash{}{0pt}%
\pgfpathmoveto{\pgfqpoint{2.510632in}{2.320949in}}%
\pgfpathlineto{\pgfqpoint{3.038151in}{2.320949in}}%
\pgfpathmoveto{\pgfqpoint{2.510632in}{2.320949in}}%
\pgfpathlineto{\pgfqpoint{1.983113in}{2.320949in}}%
\pgfpathmoveto{\pgfqpoint{2.011227in}{2.012708in}}%
\pgfpathlineto{\pgfqpoint{1.983113in}{2.320949in}}%
\pgfpathmoveto{\pgfqpoint{2.011227in}{2.012708in}}%
\pgfpathlineto{\pgfqpoint{2.169123in}{1.737472in}}%
\pgfpathmoveto{\pgfqpoint{2.214594in}{1.693705in}}%
\pgfpathlineto{\pgfqpoint{2.169123in}{1.737472in}}%
\pgfpathmoveto{\pgfqpoint{2.264473in}{1.653675in}}%
\pgfpathlineto{\pgfqpoint{2.214594in}{1.693705in}}%
\pgfpathmoveto{\pgfqpoint{2.316574in}{1.620858in}}%
\pgfpathlineto{\pgfqpoint{2.264473in}{1.653675in}}%
\pgfpathmoveto{\pgfqpoint{2.368841in}{1.596779in}}%
\pgfpathlineto{\pgfqpoint{2.316574in}{1.620858in}}%
\pgfpathmoveto{\pgfqpoint{2.415768in}{1.586689in}}%
\pgfpathlineto{\pgfqpoint{2.368841in}{1.596779in}}%
\pgfpathmoveto{\pgfqpoint{2.492265in}{1.563623in}}%
\pgfpathlineto{\pgfqpoint{2.415768in}{1.586689in}}%
\pgfpathmoveto{\pgfqpoint{2.560489in}{1.545143in}}%
\pgfpathlineto{\pgfqpoint{2.492265in}{1.563623in}}%
\pgfpathmoveto{\pgfqpoint{2.627008in}{1.528820in}}%
\pgfpathlineto{\pgfqpoint{2.560489in}{1.545143in}}%
\pgfpathmoveto{\pgfqpoint{2.693933in}{1.514302in}}%
\pgfpathlineto{\pgfqpoint{2.627008in}{1.528820in}}%
\pgfpathmoveto{\pgfqpoint{2.758133in}{1.506984in}}%
\pgfpathlineto{\pgfqpoint{2.693933in}{1.514302in}}%
\pgfpathmoveto{\pgfqpoint{2.822043in}{1.499908in}}%
\pgfpathlineto{\pgfqpoint{2.758133in}{1.506984in}}%
\pgfpathmoveto{\pgfqpoint{2.884970in}{1.494696in}}%
\pgfpathlineto{\pgfqpoint{2.822043in}{1.499908in}}%
\pgfpathmoveto{\pgfqpoint{2.947263in}{1.489478in}}%
\pgfpathlineto{\pgfqpoint{2.884970in}{1.494696in}}%
\pgfpathmoveto{\pgfqpoint{3.008750in}{1.488114in}}%
\pgfpathlineto{\pgfqpoint{2.947263in}{1.489478in}}%
\pgfpathmoveto{\pgfqpoint{3.070014in}{1.488655in}}%
\pgfpathlineto{\pgfqpoint{3.008750in}{1.488114in}}%
\pgfpathmoveto{\pgfqpoint{3.131630in}{1.489406in}}%
\pgfpathlineto{\pgfqpoint{3.070014in}{1.488655in}}%
\pgfpathmoveto{\pgfqpoint{3.194003in}{1.491703in}}%
\pgfpathlineto{\pgfqpoint{3.131630in}{1.489406in}}%
\pgfpathmoveto{\pgfqpoint{3.257186in}{1.496435in}}%
\pgfpathlineto{\pgfqpoint{3.194003in}{1.491703in}}%
\pgfpathmoveto{\pgfqpoint{3.320838in}{1.503141in}}%
\pgfpathlineto{\pgfqpoint{3.257186in}{1.496435in}}%
\pgfpathmoveto{\pgfqpoint{3.385229in}{1.512800in}}%
\pgfpathlineto{\pgfqpoint{3.320838in}{1.503141in}}%
\pgfpathmoveto{\pgfqpoint{3.452312in}{1.526246in}}%
\pgfpathlineto{\pgfqpoint{3.385229in}{1.512800in}}%
\pgfpathmoveto{\pgfqpoint{3.519450in}{1.540602in}}%
\pgfpathlineto{\pgfqpoint{3.452312in}{1.526246in}}%
\pgfpathmoveto{\pgfqpoint{3.587680in}{1.559418in}}%
\pgfpathlineto{\pgfqpoint{3.519450in}{1.540602in}}%
\pgfpathmoveto{\pgfqpoint{3.662835in}{1.585622in}}%
\pgfpathlineto{\pgfqpoint{3.587680in}{1.559418in}}%
\pgfpathmoveto{\pgfqpoint{3.711013in}{1.596151in}}%
\pgfpathlineto{\pgfqpoint{3.662835in}{1.585622in}}%
\pgfpathmoveto{\pgfqpoint{3.762745in}{1.620687in}}%
\pgfpathlineto{\pgfqpoint{3.711013in}{1.596151in}}%
\pgfpathmoveto{\pgfqpoint{3.813632in}{1.654758in}}%
\pgfpathlineto{\pgfqpoint{3.762745in}{1.620687in}}%
\pgfpathmoveto{\pgfqpoint{3.863444in}{1.694459in}}%
\pgfpathlineto{\pgfqpoint{3.907429in}{1.740043in}}%
\pgfpathmoveto{\pgfqpoint{3.863444in}{1.694459in}}%
\pgfpathlineto{\pgfqpoint{3.813632in}{1.654758in}}%
\pgfpathmoveto{\pgfqpoint{4.065610in}{2.013229in}}%
\pgfpathlineto{\pgfqpoint{3.907429in}{1.740043in}}%
\pgfpathmoveto{\pgfqpoint{4.065610in}{2.013229in}}%
\pgfpathlineto{\pgfqpoint{4.093188in}{2.320949in}}%
\pgfpathmoveto{\pgfqpoint{3.565669in}{2.320949in}}%
\pgfpathlineto{\pgfqpoint{3.038151in}{2.320949in}}%
\pgfpathmoveto{\pgfqpoint{3.565669in}{2.320949in}}%
\pgfpathlineto{\pgfqpoint{4.093188in}{2.320949in}}%
\pgfpathmoveto{\pgfqpoint{3.250284in}{1.966722in}}%
\pgfpathlineto{\pgfqpoint{3.038151in}{2.320949in}}%
\pgfpathmoveto{\pgfqpoint{3.523568in}{1.892694in}}%
\pgfpathlineto{\pgfqpoint{3.250284in}{1.966722in}}%
\pgfpathmoveto{\pgfqpoint{2.508759in}{1.842769in}}%
\pgfpathlineto{\pgfqpoint{2.703826in}{1.926127in}}%
\pgfpathmoveto{\pgfqpoint{3.464831in}{1.711140in}}%
\pgfpathlineto{\pgfqpoint{3.523568in}{1.892694in}}%
\pgfpathmoveto{\pgfqpoint{2.872361in}{1.634899in}}%
\pgfpathlineto{\pgfqpoint{2.987263in}{1.749958in}}%
\pgfpathmoveto{\pgfqpoint{2.872361in}{1.634899in}}%
\pgfpathlineto{\pgfqpoint{2.734852in}{1.693845in}}%
\pgfpathmoveto{\pgfqpoint{3.039809in}{1.619208in}}%
\pgfpathlineto{\pgfqpoint{2.987263in}{1.749958in}}%
\pgfpathmoveto{\pgfqpoint{3.039809in}{1.619208in}}%
\pgfpathlineto{\pgfqpoint{2.872361in}{1.634899in}}%
\pgfpathmoveto{\pgfqpoint{3.631285in}{1.734748in}}%
\pgfpathlineto{\pgfqpoint{3.523568in}{1.892694in}}%
\pgfpathmoveto{\pgfqpoint{3.631285in}{1.734748in}}%
\pgfpathlineto{\pgfqpoint{3.464831in}{1.711140in}}%
\pgfpathmoveto{\pgfqpoint{2.585806in}{1.665219in}}%
\pgfpathlineto{\pgfqpoint{2.734852in}{1.693845in}}%
\pgfpathmoveto{\pgfqpoint{2.337860in}{1.785748in}}%
\pgfpathlineto{\pgfqpoint{2.169123in}{1.737472in}}%
\pgfpathmoveto{\pgfqpoint{2.337860in}{1.785748in}}%
\pgfpathlineto{\pgfqpoint{2.214594in}{1.693705in}}%
\pgfpathmoveto{\pgfqpoint{2.337860in}{1.785748in}}%
\pgfpathlineto{\pgfqpoint{2.508759in}{1.842769in}}%
\pgfpathmoveto{\pgfqpoint{2.738425in}{1.602825in}}%
\pgfpathlineto{\pgfqpoint{2.734852in}{1.693845in}}%
\pgfpathmoveto{\pgfqpoint{2.738425in}{1.602825in}}%
\pgfpathlineto{\pgfqpoint{2.872361in}{1.634899in}}%
\pgfpathmoveto{\pgfqpoint{3.159766in}{1.591915in}}%
\pgfpathlineto{\pgfqpoint{3.241744in}{1.696850in}}%
\pgfpathmoveto{\pgfqpoint{3.159766in}{1.591915in}}%
\pgfpathlineto{\pgfqpoint{3.039809in}{1.619208in}}%
\pgfpathmoveto{\pgfqpoint{3.277913in}{1.597638in}}%
\pgfpathlineto{\pgfqpoint{3.241744in}{1.696850in}}%
\pgfpathmoveto{\pgfqpoint{3.277913in}{1.597638in}}%
\pgfpathlineto{\pgfqpoint{3.159766in}{1.591915in}}%
\pgfpathmoveto{\pgfqpoint{3.390660in}{1.617203in}}%
\pgfpathlineto{\pgfqpoint{3.464831in}{1.711140in}}%
\pgfpathmoveto{\pgfqpoint{3.390660in}{1.617203in}}%
\pgfpathlineto{\pgfqpoint{3.277913in}{1.597638in}}%
\pgfpathmoveto{\pgfqpoint{2.478436in}{1.712894in}}%
\pgfpathlineto{\pgfqpoint{2.508759in}{1.842769in}}%
\pgfpathmoveto{\pgfqpoint{2.478436in}{1.712894in}}%
\pgfpathlineto{\pgfqpoint{2.585806in}{1.665219in}}%
\pgfpathmoveto{\pgfqpoint{2.478436in}{1.712894in}}%
\pgfpathlineto{\pgfqpoint{2.337860in}{1.785748in}}%
\pgfpathmoveto{\pgfqpoint{3.789975in}{1.834852in}}%
\pgfpathlineto{\pgfqpoint{3.907429in}{1.740043in}}%
\pgfpathmoveto{\pgfqpoint{3.789975in}{1.834852in}}%
\pgfpathlineto{\pgfqpoint{3.863444in}{1.694459in}}%
\pgfpathmoveto{\pgfqpoint{3.789975in}{1.834852in}}%
\pgfpathlineto{\pgfqpoint{4.065610in}{2.013229in}}%
\pgfpathmoveto{\pgfqpoint{3.789975in}{1.834852in}}%
\pgfpathlineto{\pgfqpoint{3.523568in}{1.892694in}}%
\pgfpathmoveto{\pgfqpoint{3.789975in}{1.834852in}}%
\pgfpathlineto{\pgfqpoint{3.631285in}{1.734748in}}%
\pgfpathmoveto{\pgfqpoint{3.565455in}{1.661015in}}%
\pgfpathlineto{\pgfqpoint{3.464831in}{1.711140in}}%
\pgfpathmoveto{\pgfqpoint{3.565455in}{1.661015in}}%
\pgfpathlineto{\pgfqpoint{3.631285in}{1.734748in}}%
\pgfpathmoveto{\pgfqpoint{2.921714in}{1.566903in}}%
\pgfpathlineto{\pgfqpoint{2.884970in}{1.494696in}}%
\pgfpathmoveto{\pgfqpoint{2.921714in}{1.566903in}}%
\pgfpathlineto{\pgfqpoint{2.947263in}{1.489478in}}%
\pgfpathmoveto{\pgfqpoint{2.921714in}{1.566903in}}%
\pgfpathlineto{\pgfqpoint{2.872361in}{1.634899in}}%
\pgfpathmoveto{\pgfqpoint{2.921714in}{1.566903in}}%
\pgfpathlineto{\pgfqpoint{3.039809in}{1.619208in}}%
\pgfpathmoveto{\pgfqpoint{2.475422in}{2.019109in}}%
\pgfpathlineto{\pgfqpoint{2.510632in}{2.320949in}}%
\pgfpathmoveto{\pgfqpoint{2.475422in}{2.019109in}}%
\pgfpathlineto{\pgfqpoint{2.703826in}{1.926127in}}%
\pgfpathmoveto{\pgfqpoint{2.475422in}{2.019109in}}%
\pgfpathlineto{\pgfqpoint{2.508759in}{1.842769in}}%
\pgfpathmoveto{\pgfqpoint{2.475422in}{2.019109in}}%
\pgfpathlineto{\pgfqpoint{2.337860in}{1.785748in}}%
\pgfpathmoveto{\pgfqpoint{3.780606in}{2.079164in}}%
\pgfpathlineto{\pgfqpoint{4.093188in}{2.320949in}}%
\pgfpathmoveto{\pgfqpoint{3.780606in}{2.079164in}}%
\pgfpathlineto{\pgfqpoint{4.065610in}{2.013229in}}%
\pgfpathmoveto{\pgfqpoint{3.780606in}{2.079164in}}%
\pgfpathlineto{\pgfqpoint{3.565669in}{2.320949in}}%
\pgfpathmoveto{\pgfqpoint{3.780606in}{2.079164in}}%
\pgfpathlineto{\pgfqpoint{3.523568in}{1.892694in}}%
\pgfpathmoveto{\pgfqpoint{3.780606in}{2.079164in}}%
\pgfpathlineto{\pgfqpoint{3.789975in}{1.834852in}}%
\pgfpathmoveto{\pgfqpoint{2.618338in}{1.600267in}}%
\pgfpathlineto{\pgfqpoint{2.560489in}{1.545143in}}%
\pgfpathmoveto{\pgfqpoint{2.618338in}{1.600267in}}%
\pgfpathlineto{\pgfqpoint{2.627008in}{1.528820in}}%
\pgfpathmoveto{\pgfqpoint{2.618338in}{1.600267in}}%
\pgfpathlineto{\pgfqpoint{2.734852in}{1.693845in}}%
\pgfpathmoveto{\pgfqpoint{2.618338in}{1.600267in}}%
\pgfpathlineto{\pgfqpoint{2.585806in}{1.665219in}}%
\pgfpathmoveto{\pgfqpoint{2.618338in}{1.600267in}}%
\pgfpathlineto{\pgfqpoint{2.738425in}{1.602825in}}%
\pgfpathmoveto{\pgfqpoint{3.761906in}{1.699336in}}%
\pgfpathlineto{\pgfqpoint{3.762745in}{1.620687in}}%
\pgfpathmoveto{\pgfqpoint{3.761906in}{1.699336in}}%
\pgfpathlineto{\pgfqpoint{3.813632in}{1.654758in}}%
\pgfpathmoveto{\pgfqpoint{3.761906in}{1.699336in}}%
\pgfpathlineto{\pgfqpoint{3.863444in}{1.694459in}}%
\pgfpathmoveto{\pgfqpoint{3.761906in}{1.699336in}}%
\pgfpathlineto{\pgfqpoint{3.631285in}{1.734748in}}%
\pgfpathmoveto{\pgfqpoint{3.761906in}{1.699336in}}%
\pgfpathlineto{\pgfqpoint{3.789975in}{1.834852in}}%
\pgfpathmoveto{\pgfqpoint{3.039580in}{1.533605in}}%
\pgfpathlineto{\pgfqpoint{3.008750in}{1.488114in}}%
\pgfpathmoveto{\pgfqpoint{3.039580in}{1.533605in}}%
\pgfpathlineto{\pgfqpoint{3.070014in}{1.488655in}}%
\pgfpathmoveto{\pgfqpoint{3.039580in}{1.533605in}}%
\pgfpathlineto{\pgfqpoint{3.039809in}{1.619208in}}%
\pgfpathmoveto{\pgfqpoint{2.982729in}{1.948869in}}%
\pgfpathlineto{\pgfqpoint{3.038151in}{2.320949in}}%
\pgfpathmoveto{\pgfqpoint{2.982729in}{1.948869in}}%
\pgfpathlineto{\pgfqpoint{3.250284in}{1.966722in}}%
\pgfpathmoveto{\pgfqpoint{2.982729in}{1.948869in}}%
\pgfpathlineto{\pgfqpoint{2.703826in}{1.926127in}}%
\pgfpathmoveto{\pgfqpoint{2.982729in}{1.948869in}}%
\pgfpathlineto{\pgfqpoint{2.987263in}{1.749958in}}%
\pgfpathmoveto{\pgfqpoint{2.796591in}{1.557608in}}%
\pgfpathlineto{\pgfqpoint{2.758133in}{1.506984in}}%
\pgfpathmoveto{\pgfqpoint{2.796591in}{1.557608in}}%
\pgfpathlineto{\pgfqpoint{2.822043in}{1.499908in}}%
\pgfpathmoveto{\pgfqpoint{2.796591in}{1.557608in}}%
\pgfpathlineto{\pgfqpoint{2.872361in}{1.634899in}}%
\pgfpathmoveto{\pgfqpoint{2.796591in}{1.557608in}}%
\pgfpathlineto{\pgfqpoint{2.738425in}{1.602825in}}%
\pgfpathmoveto{\pgfqpoint{2.238516in}{1.943840in}}%
\pgfpathlineto{\pgfqpoint{2.169123in}{1.737472in}}%
\pgfpathmoveto{\pgfqpoint{2.238516in}{1.943840in}}%
\pgfpathlineto{\pgfqpoint{2.011227in}{2.012708in}}%
\pgfpathmoveto{\pgfqpoint{2.238516in}{1.943840in}}%
\pgfpathlineto{\pgfqpoint{2.337860in}{1.785748in}}%
\pgfpathmoveto{\pgfqpoint{2.238516in}{1.943840in}}%
\pgfpathlineto{\pgfqpoint{2.475422in}{2.019109in}}%
\pgfpathmoveto{\pgfqpoint{3.499939in}{1.598374in}}%
\pgfpathlineto{\pgfqpoint{3.452312in}{1.526246in}}%
\pgfpathmoveto{\pgfqpoint{3.499939in}{1.598374in}}%
\pgfpathlineto{\pgfqpoint{3.519450in}{1.540602in}}%
\pgfpathmoveto{\pgfqpoint{3.499939in}{1.598374in}}%
\pgfpathlineto{\pgfqpoint{3.587680in}{1.559418in}}%
\pgfpathmoveto{\pgfqpoint{3.499939in}{1.598374in}}%
\pgfpathlineto{\pgfqpoint{3.464831in}{1.711140in}}%
\pgfpathmoveto{\pgfqpoint{3.499939in}{1.598374in}}%
\pgfpathlineto{\pgfqpoint{3.390660in}{1.617203in}}%
\pgfpathmoveto{\pgfqpoint{3.499939in}{1.598374in}}%
\pgfpathlineto{\pgfqpoint{3.565455in}{1.661015in}}%
\pgfpathmoveto{\pgfqpoint{3.343768in}{1.558281in}}%
\pgfpathlineto{\pgfqpoint{3.320838in}{1.503141in}}%
\pgfpathmoveto{\pgfqpoint{3.343768in}{1.558281in}}%
\pgfpathlineto{\pgfqpoint{3.385229in}{1.512800in}}%
\pgfpathmoveto{\pgfqpoint{3.343768in}{1.558281in}}%
\pgfpathlineto{\pgfqpoint{3.277913in}{1.597638in}}%
\pgfpathmoveto{\pgfqpoint{3.343768in}{1.558281in}}%
\pgfpathlineto{\pgfqpoint{3.390660in}{1.617203in}}%
\pgfpathmoveto{\pgfqpoint{3.221571in}{1.544828in}}%
\pgfpathlineto{\pgfqpoint{3.194003in}{1.491703in}}%
\pgfpathmoveto{\pgfqpoint{3.221571in}{1.544828in}}%
\pgfpathlineto{\pgfqpoint{3.257186in}{1.496435in}}%
\pgfpathmoveto{\pgfqpoint{3.221571in}{1.544828in}}%
\pgfpathlineto{\pgfqpoint{3.159766in}{1.591915in}}%
\pgfpathmoveto{\pgfqpoint{3.221571in}{1.544828in}}%
\pgfpathlineto{\pgfqpoint{3.277913in}{1.597638in}}%
\pgfpathmoveto{\pgfqpoint{2.491464in}{1.623798in}}%
\pgfpathlineto{\pgfqpoint{2.415768in}{1.586689in}}%
\pgfpathmoveto{\pgfqpoint{2.491464in}{1.623798in}}%
\pgfpathlineto{\pgfqpoint{2.492265in}{1.563623in}}%
\pgfpathmoveto{\pgfqpoint{2.491464in}{1.623798in}}%
\pgfpathlineto{\pgfqpoint{2.585806in}{1.665219in}}%
\pgfpathmoveto{\pgfqpoint{2.491464in}{1.623798in}}%
\pgfpathlineto{\pgfqpoint{2.478436in}{1.712894in}}%
\pgfpathmoveto{\pgfqpoint{2.383797in}{1.667466in}}%
\pgfpathlineto{\pgfqpoint{2.316574in}{1.620858in}}%
\pgfpathmoveto{\pgfqpoint{2.383797in}{1.667466in}}%
\pgfpathlineto{\pgfqpoint{2.368841in}{1.596779in}}%
\pgfpathmoveto{\pgfqpoint{2.383797in}{1.667466in}}%
\pgfpathlineto{\pgfqpoint{2.337860in}{1.785748in}}%
\pgfpathmoveto{\pgfqpoint{2.383797in}{1.667466in}}%
\pgfpathlineto{\pgfqpoint{2.478436in}{1.712894in}}%
\pgfpathmoveto{\pgfqpoint{3.643448in}{1.642103in}}%
\pgfpathlineto{\pgfqpoint{3.662835in}{1.585622in}}%
\pgfpathmoveto{\pgfqpoint{3.643448in}{1.642103in}}%
\pgfpathlineto{\pgfqpoint{3.711013in}{1.596151in}}%
\pgfpathmoveto{\pgfqpoint{3.643448in}{1.642103in}}%
\pgfpathlineto{\pgfqpoint{3.631285in}{1.734748in}}%
\pgfpathmoveto{\pgfqpoint{3.643448in}{1.642103in}}%
\pgfpathlineto{\pgfqpoint{3.565455in}{1.661015in}}%
\pgfpathmoveto{\pgfqpoint{2.682172in}{1.560185in}}%
\pgfpathlineto{\pgfqpoint{2.627008in}{1.528820in}}%
\pgfpathmoveto{\pgfqpoint{2.682172in}{1.560185in}}%
\pgfpathlineto{\pgfqpoint{2.693933in}{1.514302in}}%
\pgfpathmoveto{\pgfqpoint{2.682172in}{1.560185in}}%
\pgfpathlineto{\pgfqpoint{2.738425in}{1.602825in}}%
\pgfpathmoveto{\pgfqpoint{2.682172in}{1.560185in}}%
\pgfpathlineto{\pgfqpoint{2.618338in}{1.600267in}}%
\pgfpathmoveto{\pgfqpoint{2.979252in}{1.534982in}}%
\pgfpathlineto{\pgfqpoint{2.947263in}{1.489478in}}%
\pgfpathmoveto{\pgfqpoint{2.979252in}{1.534982in}}%
\pgfpathlineto{\pgfqpoint{3.008750in}{1.488114in}}%
\pgfpathmoveto{\pgfqpoint{2.979252in}{1.534982in}}%
\pgfpathlineto{\pgfqpoint{3.039809in}{1.619208in}}%
\pgfpathmoveto{\pgfqpoint{2.979252in}{1.534982in}}%
\pgfpathlineto{\pgfqpoint{2.921714in}{1.566903in}}%
\pgfpathmoveto{\pgfqpoint{2.979252in}{1.534982in}}%
\pgfpathlineto{\pgfqpoint{3.039580in}{1.533605in}}%
\pgfpathmoveto{\pgfqpoint{3.100712in}{1.543727in}}%
\pgfpathlineto{\pgfqpoint{3.070014in}{1.488655in}}%
\pgfpathmoveto{\pgfqpoint{3.100712in}{1.543727in}}%
\pgfpathlineto{\pgfqpoint{3.131630in}{1.489406in}}%
\pgfpathmoveto{\pgfqpoint{3.100712in}{1.543727in}}%
\pgfpathlineto{\pgfqpoint{3.039809in}{1.619208in}}%
\pgfpathmoveto{\pgfqpoint{3.100712in}{1.543727in}}%
\pgfpathlineto{\pgfqpoint{3.159766in}{1.591915in}}%
\pgfpathmoveto{\pgfqpoint{3.100712in}{1.543727in}}%
\pgfpathlineto{\pgfqpoint{3.039580in}{1.533605in}}%
\pgfpathmoveto{\pgfqpoint{3.161389in}{1.533074in}}%
\pgfpathlineto{\pgfqpoint{3.131630in}{1.489406in}}%
\pgfpathmoveto{\pgfqpoint{3.161389in}{1.533074in}}%
\pgfpathlineto{\pgfqpoint{3.194003in}{1.491703in}}%
\pgfpathmoveto{\pgfqpoint{3.161389in}{1.533074in}}%
\pgfpathlineto{\pgfqpoint{3.159766in}{1.591915in}}%
\pgfpathmoveto{\pgfqpoint{3.161389in}{1.533074in}}%
\pgfpathlineto{\pgfqpoint{3.221571in}{1.544828in}}%
\pgfpathmoveto{\pgfqpoint{3.161389in}{1.533074in}}%
\pgfpathlineto{\pgfqpoint{3.100712in}{1.543727in}}%
\pgfpathmoveto{\pgfqpoint{2.734785in}{1.554365in}}%
\pgfpathlineto{\pgfqpoint{2.693933in}{1.514302in}}%
\pgfpathmoveto{\pgfqpoint{2.734785in}{1.554365in}}%
\pgfpathlineto{\pgfqpoint{2.758133in}{1.506984in}}%
\pgfpathmoveto{\pgfqpoint{2.734785in}{1.554365in}}%
\pgfpathlineto{\pgfqpoint{2.738425in}{1.602825in}}%
\pgfpathmoveto{\pgfqpoint{2.734785in}{1.554365in}}%
\pgfpathlineto{\pgfqpoint{2.796591in}{1.557608in}}%
\pgfpathmoveto{\pgfqpoint{2.734785in}{1.554365in}}%
\pgfpathlineto{\pgfqpoint{2.682172in}{1.560185in}}%
\pgfpathmoveto{\pgfqpoint{2.858415in}{1.541500in}}%
\pgfpathlineto{\pgfqpoint{2.822043in}{1.499908in}}%
\pgfpathmoveto{\pgfqpoint{2.858415in}{1.541500in}}%
\pgfpathlineto{\pgfqpoint{2.884970in}{1.494696in}}%
\pgfpathmoveto{\pgfqpoint{2.858415in}{1.541500in}}%
\pgfpathlineto{\pgfqpoint{2.872361in}{1.634899in}}%
\pgfpathmoveto{\pgfqpoint{2.858415in}{1.541500in}}%
\pgfpathlineto{\pgfqpoint{2.921714in}{1.566903in}}%
\pgfpathmoveto{\pgfqpoint{2.858415in}{1.541500in}}%
\pgfpathlineto{\pgfqpoint{2.796591in}{1.557608in}}%
\pgfpathmoveto{\pgfqpoint{3.115029in}{1.693077in}}%
\pgfpathlineto{\pgfqpoint{2.987263in}{1.749958in}}%
\pgfpathmoveto{\pgfqpoint{3.115029in}{1.693077in}}%
\pgfpathlineto{\pgfqpoint{3.241744in}{1.696850in}}%
\pgfpathmoveto{\pgfqpoint{3.115029in}{1.693077in}}%
\pgfpathlineto{\pgfqpoint{3.039809in}{1.619208in}}%
\pgfpathmoveto{\pgfqpoint{3.115029in}{1.693077in}}%
\pgfpathlineto{\pgfqpoint{3.159766in}{1.591915in}}%
\pgfpathmoveto{\pgfqpoint{3.152884in}{1.811792in}}%
\pgfpathlineto{\pgfqpoint{3.250284in}{1.966722in}}%
\pgfpathmoveto{\pgfqpoint{3.152884in}{1.811792in}}%
\pgfpathlineto{\pgfqpoint{2.987263in}{1.749958in}}%
\pgfpathmoveto{\pgfqpoint{3.152884in}{1.811792in}}%
\pgfpathlineto{\pgfqpoint{3.241744in}{1.696850in}}%
\pgfpathmoveto{\pgfqpoint{3.152884in}{1.811792in}}%
\pgfpathlineto{\pgfqpoint{2.982729in}{1.948869in}}%
\pgfpathmoveto{\pgfqpoint{3.152884in}{1.811792in}}%
\pgfpathlineto{\pgfqpoint{3.115029in}{1.693077in}}%
\pgfpathmoveto{\pgfqpoint{2.547191in}{1.595880in}}%
\pgfpathlineto{\pgfqpoint{2.492265in}{1.563623in}}%
\pgfpathmoveto{\pgfqpoint{2.547191in}{1.595880in}}%
\pgfpathlineto{\pgfqpoint{2.560489in}{1.545143in}}%
\pgfpathmoveto{\pgfqpoint{2.547191in}{1.595880in}}%
\pgfpathlineto{\pgfqpoint{2.585806in}{1.665219in}}%
\pgfpathmoveto{\pgfqpoint{2.547191in}{1.595880in}}%
\pgfpathlineto{\pgfqpoint{2.618338in}{1.600267in}}%
\pgfpathmoveto{\pgfqpoint{2.547191in}{1.595880in}}%
\pgfpathlineto{\pgfqpoint{2.491464in}{1.623798in}}%
\pgfpathmoveto{\pgfqpoint{2.302473in}{1.683512in}}%
\pgfpathlineto{\pgfqpoint{2.214594in}{1.693705in}}%
\pgfpathmoveto{\pgfqpoint{2.302473in}{1.683512in}}%
\pgfpathlineto{\pgfqpoint{2.264473in}{1.653675in}}%
\pgfpathmoveto{\pgfqpoint{2.302473in}{1.683512in}}%
\pgfpathlineto{\pgfqpoint{2.316574in}{1.620858in}}%
\pgfpathmoveto{\pgfqpoint{2.302473in}{1.683512in}}%
\pgfpathlineto{\pgfqpoint{2.337860in}{1.785748in}}%
\pgfpathmoveto{\pgfqpoint{2.302473in}{1.683512in}}%
\pgfpathlineto{\pgfqpoint{2.383797in}{1.667466in}}%
\pgfpathmoveto{\pgfqpoint{3.703024in}{1.659239in}}%
\pgfpathlineto{\pgfqpoint{3.711013in}{1.596151in}}%
\pgfpathmoveto{\pgfqpoint{3.703024in}{1.659239in}}%
\pgfpathlineto{\pgfqpoint{3.762745in}{1.620687in}}%
\pgfpathmoveto{\pgfqpoint{3.703024in}{1.659239in}}%
\pgfpathlineto{\pgfqpoint{3.631285in}{1.734748in}}%
\pgfpathmoveto{\pgfqpoint{3.703024in}{1.659239in}}%
\pgfpathlineto{\pgfqpoint{3.761906in}{1.699336in}}%
\pgfpathmoveto{\pgfqpoint{3.703024in}{1.659239in}}%
\pgfpathlineto{\pgfqpoint{3.643448in}{1.642103in}}%
\pgfpathmoveto{\pgfqpoint{2.830101in}{1.819652in}}%
\pgfpathlineto{\pgfqpoint{2.703826in}{1.926127in}}%
\pgfpathmoveto{\pgfqpoint{2.830101in}{1.819652in}}%
\pgfpathlineto{\pgfqpoint{2.987263in}{1.749958in}}%
\pgfpathmoveto{\pgfqpoint{2.830101in}{1.819652in}}%
\pgfpathlineto{\pgfqpoint{2.734852in}{1.693845in}}%
\pgfpathmoveto{\pgfqpoint{2.830101in}{1.819652in}}%
\pgfpathlineto{\pgfqpoint{2.982729in}{1.948869in}}%
\pgfpathmoveto{\pgfqpoint{3.433993in}{2.117567in}}%
\pgfpathlineto{\pgfqpoint{3.038151in}{2.320949in}}%
\pgfpathmoveto{\pgfqpoint{3.433993in}{2.117567in}}%
\pgfpathlineto{\pgfqpoint{3.565669in}{2.320949in}}%
\pgfpathmoveto{\pgfqpoint{3.433993in}{2.117567in}}%
\pgfpathlineto{\pgfqpoint{3.250284in}{1.966722in}}%
\pgfpathmoveto{\pgfqpoint{3.433993in}{2.117567in}}%
\pgfpathlineto{\pgfqpoint{3.523568in}{1.892694in}}%
\pgfpathmoveto{\pgfqpoint{3.433993in}{2.117567in}}%
\pgfpathlineto{\pgfqpoint{3.780606in}{2.079164in}}%
\pgfpathmoveto{\pgfqpoint{2.638831in}{1.778312in}}%
\pgfpathlineto{\pgfqpoint{2.703826in}{1.926127in}}%
\pgfpathmoveto{\pgfqpoint{2.638831in}{1.778312in}}%
\pgfpathlineto{\pgfqpoint{2.734852in}{1.693845in}}%
\pgfpathmoveto{\pgfqpoint{2.638831in}{1.778312in}}%
\pgfpathlineto{\pgfqpoint{2.508759in}{1.842769in}}%
\pgfpathmoveto{\pgfqpoint{2.638831in}{1.778312in}}%
\pgfpathlineto{\pgfqpoint{2.585806in}{1.665219in}}%
\pgfpathmoveto{\pgfqpoint{2.638831in}{1.778312in}}%
\pgfpathlineto{\pgfqpoint{2.478436in}{1.712894in}}%
\pgfpathmoveto{\pgfqpoint{2.638831in}{1.778312in}}%
\pgfpathlineto{\pgfqpoint{2.830101in}{1.819652in}}%
\pgfpathmoveto{\pgfqpoint{3.333731in}{1.801327in}}%
\pgfpathlineto{\pgfqpoint{3.250284in}{1.966722in}}%
\pgfpathmoveto{\pgfqpoint{3.333731in}{1.801327in}}%
\pgfpathlineto{\pgfqpoint{3.523568in}{1.892694in}}%
\pgfpathmoveto{\pgfqpoint{3.333731in}{1.801327in}}%
\pgfpathlineto{\pgfqpoint{3.241744in}{1.696850in}}%
\pgfpathmoveto{\pgfqpoint{3.333731in}{1.801327in}}%
\pgfpathlineto{\pgfqpoint{3.464831in}{1.711140in}}%
\pgfpathmoveto{\pgfqpoint{3.333731in}{1.801327in}}%
\pgfpathlineto{\pgfqpoint{3.152884in}{1.811792in}}%
\pgfpathmoveto{\pgfqpoint{2.240357in}{2.121631in}}%
\pgfpathlineto{\pgfqpoint{1.983113in}{2.320949in}}%
\pgfpathmoveto{\pgfqpoint{2.240357in}{2.121631in}}%
\pgfpathlineto{\pgfqpoint{2.510632in}{2.320949in}}%
\pgfpathmoveto{\pgfqpoint{2.240357in}{2.121631in}}%
\pgfpathlineto{\pgfqpoint{2.011227in}{2.012708in}}%
\pgfpathmoveto{\pgfqpoint{2.240357in}{2.121631in}}%
\pgfpathlineto{\pgfqpoint{2.475422in}{2.019109in}}%
\pgfpathmoveto{\pgfqpoint{2.240357in}{2.121631in}}%
\pgfpathlineto{\pgfqpoint{2.238516in}{1.943840in}}%
\pgfpathmoveto{\pgfqpoint{2.860072in}{1.729865in}}%
\pgfpathlineto{\pgfqpoint{2.987263in}{1.749958in}}%
\pgfpathmoveto{\pgfqpoint{2.860072in}{1.729865in}}%
\pgfpathlineto{\pgfqpoint{2.734852in}{1.693845in}}%
\pgfpathmoveto{\pgfqpoint{2.860072in}{1.729865in}}%
\pgfpathlineto{\pgfqpoint{2.872361in}{1.634899in}}%
\pgfpathmoveto{\pgfqpoint{2.860072in}{1.729865in}}%
\pgfpathlineto{\pgfqpoint{2.830101in}{1.819652in}}%
\pgfpathmoveto{\pgfqpoint{2.429259in}{1.639202in}}%
\pgfpathlineto{\pgfqpoint{2.368841in}{1.596779in}}%
\pgfpathmoveto{\pgfqpoint{2.429259in}{1.639202in}}%
\pgfpathlineto{\pgfqpoint{2.415768in}{1.586689in}}%
\pgfpathmoveto{\pgfqpoint{2.429259in}{1.639202in}}%
\pgfpathlineto{\pgfqpoint{2.478436in}{1.712894in}}%
\pgfpathmoveto{\pgfqpoint{2.429259in}{1.639202in}}%
\pgfpathlineto{\pgfqpoint{2.491464in}{1.623798in}}%
\pgfpathmoveto{\pgfqpoint{2.429259in}{1.639202in}}%
\pgfpathlineto{\pgfqpoint{2.383797in}{1.667466in}}%
\pgfpathmoveto{\pgfqpoint{3.284551in}{1.540473in}}%
\pgfpathlineto{\pgfqpoint{3.257186in}{1.496435in}}%
\pgfpathmoveto{\pgfqpoint{3.284551in}{1.540473in}}%
\pgfpathlineto{\pgfqpoint{3.320838in}{1.503141in}}%
\pgfpathmoveto{\pgfqpoint{3.284551in}{1.540473in}}%
\pgfpathlineto{\pgfqpoint{3.277913in}{1.597638in}}%
\pgfpathmoveto{\pgfqpoint{3.284551in}{1.540473in}}%
\pgfpathlineto{\pgfqpoint{3.343768in}{1.558281in}}%
\pgfpathmoveto{\pgfqpoint{3.284551in}{1.540473in}}%
\pgfpathlineto{\pgfqpoint{3.221571in}{1.544828in}}%
\pgfpathmoveto{\pgfqpoint{2.738925in}{2.105689in}}%
\pgfpathlineto{\pgfqpoint{3.038151in}{2.320949in}}%
\pgfpathmoveto{\pgfqpoint{2.738925in}{2.105689in}}%
\pgfpathlineto{\pgfqpoint{2.510632in}{2.320949in}}%
\pgfpathmoveto{\pgfqpoint{2.738925in}{2.105689in}}%
\pgfpathlineto{\pgfqpoint{2.703826in}{1.926127in}}%
\pgfpathmoveto{\pgfqpoint{2.738925in}{2.105689in}}%
\pgfpathlineto{\pgfqpoint{2.475422in}{2.019109in}}%
\pgfpathmoveto{\pgfqpoint{2.738925in}{2.105689in}}%
\pgfpathlineto{\pgfqpoint{2.982729in}{1.948869in}}%
\pgfpathmoveto{\pgfqpoint{3.407157in}{1.561310in}}%
\pgfpathlineto{\pgfqpoint{3.385229in}{1.512800in}}%
\pgfpathmoveto{\pgfqpoint{3.407157in}{1.561310in}}%
\pgfpathlineto{\pgfqpoint{3.452312in}{1.526246in}}%
\pgfpathmoveto{\pgfqpoint{3.407157in}{1.561310in}}%
\pgfpathlineto{\pgfqpoint{3.390660in}{1.617203in}}%
\pgfpathmoveto{\pgfqpoint{3.407157in}{1.561310in}}%
\pgfpathlineto{\pgfqpoint{3.499939in}{1.598374in}}%
\pgfpathmoveto{\pgfqpoint{3.407157in}{1.561310in}}%
\pgfpathlineto{\pgfqpoint{3.343768in}{1.558281in}}%
\pgfpathmoveto{\pgfqpoint{3.357147in}{1.693578in}}%
\pgfpathlineto{\pgfqpoint{3.241744in}{1.696850in}}%
\pgfpathmoveto{\pgfqpoint{3.357147in}{1.693578in}}%
\pgfpathlineto{\pgfqpoint{3.464831in}{1.711140in}}%
\pgfpathmoveto{\pgfqpoint{3.357147in}{1.693578in}}%
\pgfpathlineto{\pgfqpoint{3.277913in}{1.597638in}}%
\pgfpathmoveto{\pgfqpoint{3.357147in}{1.693578in}}%
\pgfpathlineto{\pgfqpoint{3.390660in}{1.617203in}}%
\pgfpathmoveto{\pgfqpoint{3.357147in}{1.693578in}}%
\pgfpathlineto{\pgfqpoint{3.333731in}{1.801327in}}%
\pgfpathmoveto{\pgfqpoint{3.592671in}{1.607682in}}%
\pgfpathlineto{\pgfqpoint{3.587680in}{1.559418in}}%
\pgfpathmoveto{\pgfqpoint{3.592671in}{1.607682in}}%
\pgfpathlineto{\pgfqpoint{3.662835in}{1.585622in}}%
\pgfpathmoveto{\pgfqpoint{3.592671in}{1.607682in}}%
\pgfpathlineto{\pgfqpoint{3.565455in}{1.661015in}}%
\pgfpathmoveto{\pgfqpoint{3.592671in}{1.607682in}}%
\pgfpathlineto{\pgfqpoint{3.499939in}{1.598374in}}%
\pgfpathmoveto{\pgfqpoint{3.592671in}{1.607682in}}%
\pgfpathlineto{\pgfqpoint{3.643448in}{1.642103in}}%
\pgfpathlineto{\pgfqpoint{3.643448in}{1.642103in}}%
\pgfusepath{stroke}%
\end{pgfscope}%
\begin{pgfscope}%
\pgfsetbuttcap%
\pgfsetroundjoin%
\definecolor{currentfill}{rgb}{0.000000,0.000000,0.000000}%
\pgfsetfillcolor{currentfill}%
\pgfsetlinewidth{0.803000pt}%
\definecolor{currentstroke}{rgb}{0.000000,0.000000,0.000000}%
\pgfsetstrokecolor{currentstroke}%
\pgfsetdash{}{0pt}%
\pgfsys@defobject{currentmarker}{\pgfqpoint{0.000000in}{-0.048611in}}{\pgfqpoint{0.000000in}{0.000000in}}{%
\pgfpathmoveto{\pgfqpoint{0.000000in}{0.000000in}}%
\pgfpathlineto{\pgfqpoint{0.000000in}{-0.048611in}}%
\pgfusepath{stroke,fill}%
}%
\begin{pgfscope}%
\pgfsys@transformshift{0.928075in}{0.548486in}%
\pgfsys@useobject{currentmarker}{}%
\end{pgfscope}%
\end{pgfscope}%
\begin{pgfscope}%
\definecolor{textcolor}{rgb}{0.000000,0.000000,0.000000}%
\pgfsetstrokecolor{textcolor}%
\pgfsetfillcolor{textcolor}%
\pgftext[x=0.928075in,y=0.451264in,,top]{\color{textcolor}\rmfamily\fontsize{11.000000}{13.200000}\selectfont \(\displaystyle {\ensuremath{-}1.00}\)}%
\end{pgfscope}%
\begin{pgfscope}%
\pgfsetbuttcap%
\pgfsetroundjoin%
\definecolor{currentfill}{rgb}{0.000000,0.000000,0.000000}%
\pgfsetfillcolor{currentfill}%
\pgfsetlinewidth{0.803000pt}%
\definecolor{currentstroke}{rgb}{0.000000,0.000000,0.000000}%
\pgfsetstrokecolor{currentstroke}%
\pgfsetdash{}{0pt}%
\pgfsys@defobject{currentmarker}{\pgfqpoint{0.000000in}{-0.048611in}}{\pgfqpoint{0.000000in}{0.000000in}}{%
\pgfpathmoveto{\pgfqpoint{0.000000in}{0.000000in}}%
\pgfpathlineto{\pgfqpoint{0.000000in}{-0.048611in}}%
\pgfusepath{stroke,fill}%
}%
\begin{pgfscope}%
\pgfsys@transformshift{1.455594in}{0.548486in}%
\pgfsys@useobject{currentmarker}{}%
\end{pgfscope}%
\end{pgfscope}%
\begin{pgfscope}%
\definecolor{textcolor}{rgb}{0.000000,0.000000,0.000000}%
\pgfsetstrokecolor{textcolor}%
\pgfsetfillcolor{textcolor}%
\pgftext[x=1.455594in,y=0.451264in,,top]{\color{textcolor}\rmfamily\fontsize{11.000000}{13.200000}\selectfont \(\displaystyle {\ensuremath{-}0.75}\)}%
\end{pgfscope}%
\begin{pgfscope}%
\pgfsetbuttcap%
\pgfsetroundjoin%
\definecolor{currentfill}{rgb}{0.000000,0.000000,0.000000}%
\pgfsetfillcolor{currentfill}%
\pgfsetlinewidth{0.803000pt}%
\definecolor{currentstroke}{rgb}{0.000000,0.000000,0.000000}%
\pgfsetstrokecolor{currentstroke}%
\pgfsetdash{}{0pt}%
\pgfsys@defobject{currentmarker}{\pgfqpoint{0.000000in}{-0.048611in}}{\pgfqpoint{0.000000in}{0.000000in}}{%
\pgfpathmoveto{\pgfqpoint{0.000000in}{0.000000in}}%
\pgfpathlineto{\pgfqpoint{0.000000in}{-0.048611in}}%
\pgfusepath{stroke,fill}%
}%
\begin{pgfscope}%
\pgfsys@transformshift{1.983113in}{0.548486in}%
\pgfsys@useobject{currentmarker}{}%
\end{pgfscope}%
\end{pgfscope}%
\begin{pgfscope}%
\definecolor{textcolor}{rgb}{0.000000,0.000000,0.000000}%
\pgfsetstrokecolor{textcolor}%
\pgfsetfillcolor{textcolor}%
\pgftext[x=1.983113in,y=0.451264in,,top]{\color{textcolor}\rmfamily\fontsize{11.000000}{13.200000}\selectfont \(\displaystyle {\ensuremath{-}0.50}\)}%
\end{pgfscope}%
\begin{pgfscope}%
\pgfsetbuttcap%
\pgfsetroundjoin%
\definecolor{currentfill}{rgb}{0.000000,0.000000,0.000000}%
\pgfsetfillcolor{currentfill}%
\pgfsetlinewidth{0.803000pt}%
\definecolor{currentstroke}{rgb}{0.000000,0.000000,0.000000}%
\pgfsetstrokecolor{currentstroke}%
\pgfsetdash{}{0pt}%
\pgfsys@defobject{currentmarker}{\pgfqpoint{0.000000in}{-0.048611in}}{\pgfqpoint{0.000000in}{0.000000in}}{%
\pgfpathmoveto{\pgfqpoint{0.000000in}{0.000000in}}%
\pgfpathlineto{\pgfqpoint{0.000000in}{-0.048611in}}%
\pgfusepath{stroke,fill}%
}%
\begin{pgfscope}%
\pgfsys@transformshift{2.510632in}{0.548486in}%
\pgfsys@useobject{currentmarker}{}%
\end{pgfscope}%
\end{pgfscope}%
\begin{pgfscope}%
\definecolor{textcolor}{rgb}{0.000000,0.000000,0.000000}%
\pgfsetstrokecolor{textcolor}%
\pgfsetfillcolor{textcolor}%
\pgftext[x=2.510632in,y=0.451264in,,top]{\color{textcolor}\rmfamily\fontsize{11.000000}{13.200000}\selectfont \(\displaystyle {\ensuremath{-}0.25}\)}%
\end{pgfscope}%
\begin{pgfscope}%
\pgfsetbuttcap%
\pgfsetroundjoin%
\definecolor{currentfill}{rgb}{0.000000,0.000000,0.000000}%
\pgfsetfillcolor{currentfill}%
\pgfsetlinewidth{0.803000pt}%
\definecolor{currentstroke}{rgb}{0.000000,0.000000,0.000000}%
\pgfsetstrokecolor{currentstroke}%
\pgfsetdash{}{0pt}%
\pgfsys@defobject{currentmarker}{\pgfqpoint{0.000000in}{-0.048611in}}{\pgfqpoint{0.000000in}{0.000000in}}{%
\pgfpathmoveto{\pgfqpoint{0.000000in}{0.000000in}}%
\pgfpathlineto{\pgfqpoint{0.000000in}{-0.048611in}}%
\pgfusepath{stroke,fill}%
}%
\begin{pgfscope}%
\pgfsys@transformshift{3.038151in}{0.548486in}%
\pgfsys@useobject{currentmarker}{}%
\end{pgfscope}%
\end{pgfscope}%
\begin{pgfscope}%
\definecolor{textcolor}{rgb}{0.000000,0.000000,0.000000}%
\pgfsetstrokecolor{textcolor}%
\pgfsetfillcolor{textcolor}%
\pgftext[x=3.038151in,y=0.451264in,,top]{\color{textcolor}\rmfamily\fontsize{11.000000}{13.200000}\selectfont \(\displaystyle {0.00}\)}%
\end{pgfscope}%
\begin{pgfscope}%
\pgfsetbuttcap%
\pgfsetroundjoin%
\definecolor{currentfill}{rgb}{0.000000,0.000000,0.000000}%
\pgfsetfillcolor{currentfill}%
\pgfsetlinewidth{0.803000pt}%
\definecolor{currentstroke}{rgb}{0.000000,0.000000,0.000000}%
\pgfsetstrokecolor{currentstroke}%
\pgfsetdash{}{0pt}%
\pgfsys@defobject{currentmarker}{\pgfqpoint{0.000000in}{-0.048611in}}{\pgfqpoint{0.000000in}{0.000000in}}{%
\pgfpathmoveto{\pgfqpoint{0.000000in}{0.000000in}}%
\pgfpathlineto{\pgfqpoint{0.000000in}{-0.048611in}}%
\pgfusepath{stroke,fill}%
}%
\begin{pgfscope}%
\pgfsys@transformshift{3.565669in}{0.548486in}%
\pgfsys@useobject{currentmarker}{}%
\end{pgfscope}%
\end{pgfscope}%
\begin{pgfscope}%
\definecolor{textcolor}{rgb}{0.000000,0.000000,0.000000}%
\pgfsetstrokecolor{textcolor}%
\pgfsetfillcolor{textcolor}%
\pgftext[x=3.565669in,y=0.451264in,,top]{\color{textcolor}\rmfamily\fontsize{11.000000}{13.200000}\selectfont \(\displaystyle {0.25}\)}%
\end{pgfscope}%
\begin{pgfscope}%
\pgfsetbuttcap%
\pgfsetroundjoin%
\definecolor{currentfill}{rgb}{0.000000,0.000000,0.000000}%
\pgfsetfillcolor{currentfill}%
\pgfsetlinewidth{0.803000pt}%
\definecolor{currentstroke}{rgb}{0.000000,0.000000,0.000000}%
\pgfsetstrokecolor{currentstroke}%
\pgfsetdash{}{0pt}%
\pgfsys@defobject{currentmarker}{\pgfqpoint{0.000000in}{-0.048611in}}{\pgfqpoint{0.000000in}{0.000000in}}{%
\pgfpathmoveto{\pgfqpoint{0.000000in}{0.000000in}}%
\pgfpathlineto{\pgfqpoint{0.000000in}{-0.048611in}}%
\pgfusepath{stroke,fill}%
}%
\begin{pgfscope}%
\pgfsys@transformshift{4.093188in}{0.548486in}%
\pgfsys@useobject{currentmarker}{}%
\end{pgfscope}%
\end{pgfscope}%
\begin{pgfscope}%
\definecolor{textcolor}{rgb}{0.000000,0.000000,0.000000}%
\pgfsetstrokecolor{textcolor}%
\pgfsetfillcolor{textcolor}%
\pgftext[x=4.093188in,y=0.451264in,,top]{\color{textcolor}\rmfamily\fontsize{11.000000}{13.200000}\selectfont \(\displaystyle {0.50}\)}%
\end{pgfscope}%
\begin{pgfscope}%
\pgfsetbuttcap%
\pgfsetroundjoin%
\definecolor{currentfill}{rgb}{0.000000,0.000000,0.000000}%
\pgfsetfillcolor{currentfill}%
\pgfsetlinewidth{0.803000pt}%
\definecolor{currentstroke}{rgb}{0.000000,0.000000,0.000000}%
\pgfsetstrokecolor{currentstroke}%
\pgfsetdash{}{0pt}%
\pgfsys@defobject{currentmarker}{\pgfqpoint{0.000000in}{-0.048611in}}{\pgfqpoint{0.000000in}{0.000000in}}{%
\pgfpathmoveto{\pgfqpoint{0.000000in}{0.000000in}}%
\pgfpathlineto{\pgfqpoint{0.000000in}{-0.048611in}}%
\pgfusepath{stroke,fill}%
}%
\begin{pgfscope}%
\pgfsys@transformshift{4.620707in}{0.548486in}%
\pgfsys@useobject{currentmarker}{}%
\end{pgfscope}%
\end{pgfscope}%
\begin{pgfscope}%
\definecolor{textcolor}{rgb}{0.000000,0.000000,0.000000}%
\pgfsetstrokecolor{textcolor}%
\pgfsetfillcolor{textcolor}%
\pgftext[x=4.620707in,y=0.451264in,,top]{\color{textcolor}\rmfamily\fontsize{11.000000}{13.200000}\selectfont \(\displaystyle {0.75}\)}%
\end{pgfscope}%
\begin{pgfscope}%
\pgfsetbuttcap%
\pgfsetroundjoin%
\definecolor{currentfill}{rgb}{0.000000,0.000000,0.000000}%
\pgfsetfillcolor{currentfill}%
\pgfsetlinewidth{0.803000pt}%
\definecolor{currentstroke}{rgb}{0.000000,0.000000,0.000000}%
\pgfsetstrokecolor{currentstroke}%
\pgfsetdash{}{0pt}%
\pgfsys@defobject{currentmarker}{\pgfqpoint{0.000000in}{-0.048611in}}{\pgfqpoint{0.000000in}{0.000000in}}{%
\pgfpathmoveto{\pgfqpoint{0.000000in}{0.000000in}}%
\pgfpathlineto{\pgfqpoint{0.000000in}{-0.048611in}}%
\pgfusepath{stroke,fill}%
}%
\begin{pgfscope}%
\pgfsys@transformshift{5.148226in}{0.548486in}%
\pgfsys@useobject{currentmarker}{}%
\end{pgfscope}%
\end{pgfscope}%
\begin{pgfscope}%
\definecolor{textcolor}{rgb}{0.000000,0.000000,0.000000}%
\pgfsetstrokecolor{textcolor}%
\pgfsetfillcolor{textcolor}%
\pgftext[x=5.148226in,y=0.451264in,,top]{\color{textcolor}\rmfamily\fontsize{11.000000}{13.200000}\selectfont \(\displaystyle {1.00}\)}%
\end{pgfscope}%
\begin{pgfscope}%
\definecolor{textcolor}{rgb}{0.000000,0.000000,0.000000}%
\pgfsetstrokecolor{textcolor}%
\pgfsetfillcolor{textcolor}%
\pgftext[x=3.038248in,y=0.247854in,,top]{\color{textcolor}\rmfamily\fontsize{11.000000}{13.200000}\selectfont \(\displaystyle x\)}%
\end{pgfscope}%
\begin{pgfscope}%
\pgfsetbuttcap%
\pgfsetroundjoin%
\definecolor{currentfill}{rgb}{0.000000,0.000000,0.000000}%
\pgfsetfillcolor{currentfill}%
\pgfsetlinewidth{0.803000pt}%
\definecolor{currentstroke}{rgb}{0.000000,0.000000,0.000000}%
\pgfsetstrokecolor{currentstroke}%
\pgfsetdash{}{0pt}%
\pgfsys@defobject{currentmarker}{\pgfqpoint{-0.048611in}{0.000000in}}{\pgfqpoint{-0.000000in}{0.000000in}}{%
\pgfpathmoveto{\pgfqpoint{-0.000000in}{0.000000in}}%
\pgfpathlineto{\pgfqpoint{-0.048611in}{0.000000in}}%
\pgfusepath{stroke,fill}%
}%
\begin{pgfscope}%
\pgfsys@transformshift{0.713248in}{0.843896in}%
\pgfsys@useobject{currentmarker}{}%
\end{pgfscope}%
\end{pgfscope}%
\begin{pgfscope}%
\definecolor{textcolor}{rgb}{0.000000,0.000000,0.000000}%
\pgfsetstrokecolor{textcolor}%
\pgfsetfillcolor{textcolor}%
\pgftext[x=0.303410in, y=0.785859in, left, base]{\color{textcolor}\rmfamily\fontsize{11.000000}{13.200000}\selectfont \(\displaystyle {\ensuremath{-}0.4}\)}%
\end{pgfscope}%
\begin{pgfscope}%
\pgfsetbuttcap%
\pgfsetroundjoin%
\definecolor{currentfill}{rgb}{0.000000,0.000000,0.000000}%
\pgfsetfillcolor{currentfill}%
\pgfsetlinewidth{0.803000pt}%
\definecolor{currentstroke}{rgb}{0.000000,0.000000,0.000000}%
\pgfsetstrokecolor{currentstroke}%
\pgfsetdash{}{0pt}%
\pgfsys@defobject{currentmarker}{\pgfqpoint{-0.048611in}{0.000000in}}{\pgfqpoint{-0.000000in}{0.000000in}}{%
\pgfpathmoveto{\pgfqpoint{-0.000000in}{0.000000in}}%
\pgfpathlineto{\pgfqpoint{-0.048611in}{0.000000in}}%
\pgfusepath{stroke,fill}%
}%
\begin{pgfscope}%
\pgfsys@transformshift{0.713248in}{1.265912in}%
\pgfsys@useobject{currentmarker}{}%
\end{pgfscope}%
\end{pgfscope}%
\begin{pgfscope}%
\definecolor{textcolor}{rgb}{0.000000,0.000000,0.000000}%
\pgfsetstrokecolor{textcolor}%
\pgfsetfillcolor{textcolor}%
\pgftext[x=0.303410in, y=1.207874in, left, base]{\color{textcolor}\rmfamily\fontsize{11.000000}{13.200000}\selectfont \(\displaystyle {\ensuremath{-}0.2}\)}%
\end{pgfscope}%
\begin{pgfscope}%
\pgfsetbuttcap%
\pgfsetroundjoin%
\definecolor{currentfill}{rgb}{0.000000,0.000000,0.000000}%
\pgfsetfillcolor{currentfill}%
\pgfsetlinewidth{0.803000pt}%
\definecolor{currentstroke}{rgb}{0.000000,0.000000,0.000000}%
\pgfsetstrokecolor{currentstroke}%
\pgfsetdash{}{0pt}%
\pgfsys@defobject{currentmarker}{\pgfqpoint{-0.048611in}{0.000000in}}{\pgfqpoint{-0.000000in}{0.000000in}}{%
\pgfpathmoveto{\pgfqpoint{-0.000000in}{0.000000in}}%
\pgfpathlineto{\pgfqpoint{-0.048611in}{0.000000in}}%
\pgfusepath{stroke,fill}%
}%
\begin{pgfscope}%
\pgfsys@transformshift{0.713248in}{1.687927in}%
\pgfsys@useobject{currentmarker}{}%
\end{pgfscope}%
\end{pgfscope}%
\begin{pgfscope}%
\definecolor{textcolor}{rgb}{0.000000,0.000000,0.000000}%
\pgfsetstrokecolor{textcolor}%
\pgfsetfillcolor{textcolor}%
\pgftext[x=0.421697in, y=1.629889in, left, base]{\color{textcolor}\rmfamily\fontsize{11.000000}{13.200000}\selectfont \(\displaystyle {0.0}\)}%
\end{pgfscope}%
\begin{pgfscope}%
\pgfsetbuttcap%
\pgfsetroundjoin%
\definecolor{currentfill}{rgb}{0.000000,0.000000,0.000000}%
\pgfsetfillcolor{currentfill}%
\pgfsetlinewidth{0.803000pt}%
\definecolor{currentstroke}{rgb}{0.000000,0.000000,0.000000}%
\pgfsetstrokecolor{currentstroke}%
\pgfsetdash{}{0pt}%
\pgfsys@defobject{currentmarker}{\pgfqpoint{-0.048611in}{0.000000in}}{\pgfqpoint{-0.000000in}{0.000000in}}{%
\pgfpathmoveto{\pgfqpoint{-0.000000in}{0.000000in}}%
\pgfpathlineto{\pgfqpoint{-0.048611in}{0.000000in}}%
\pgfusepath{stroke,fill}%
}%
\begin{pgfscope}%
\pgfsys@transformshift{0.713248in}{2.109942in}%
\pgfsys@useobject{currentmarker}{}%
\end{pgfscope}%
\end{pgfscope}%
\begin{pgfscope}%
\definecolor{textcolor}{rgb}{0.000000,0.000000,0.000000}%
\pgfsetstrokecolor{textcolor}%
\pgfsetfillcolor{textcolor}%
\pgftext[x=0.421697in, y=2.051904in, left, base]{\color{textcolor}\rmfamily\fontsize{11.000000}{13.200000}\selectfont \(\displaystyle {0.2}\)}%
\end{pgfscope}%
\begin{pgfscope}%
\definecolor{textcolor}{rgb}{0.000000,0.000000,0.000000}%
\pgfsetstrokecolor{textcolor}%
\pgfsetfillcolor{textcolor}%
\pgftext[x=0.247854in,y=1.476919in,,bottom,rotate=90.000000]{\color{textcolor}\rmfamily\fontsize{11.000000}{13.200000}\selectfont \(\displaystyle y\)}%
\end{pgfscope}%
\begin{pgfscope}%
\pgfsetrectcap%
\pgfsetmiterjoin%
\pgfsetlinewidth{0.803000pt}%
\definecolor{currentstroke}{rgb}{0.000000,0.000000,0.000000}%
\pgfsetstrokecolor{currentstroke}%
\pgfsetdash{}{0pt}%
\pgfpathmoveto{\pgfqpoint{0.713248in}{0.548486in}}%
\pgfpathlineto{\pgfqpoint{0.713248in}{2.405353in}}%
\pgfusepath{stroke}%
\end{pgfscope}%
\begin{pgfscope}%
\pgfsetrectcap%
\pgfsetmiterjoin%
\pgfsetlinewidth{0.803000pt}%
\definecolor{currentstroke}{rgb}{0.000000,0.000000,0.000000}%
\pgfsetstrokecolor{currentstroke}%
\pgfsetdash{}{0pt}%
\pgfpathmoveto{\pgfqpoint{5.363248in}{0.548486in}}%
\pgfpathlineto{\pgfqpoint{5.363248in}{2.405353in}}%
\pgfusepath{stroke}%
\end{pgfscope}%
\begin{pgfscope}%
\pgfsetrectcap%
\pgfsetmiterjoin%
\pgfsetlinewidth{0.803000pt}%
\definecolor{currentstroke}{rgb}{0.000000,0.000000,0.000000}%
\pgfsetstrokecolor{currentstroke}%
\pgfsetdash{}{0pt}%
\pgfpathmoveto{\pgfqpoint{0.713248in}{0.548486in}}%
\pgfpathlineto{\pgfqpoint{5.363248in}{0.548486in}}%
\pgfusepath{stroke}%
\end{pgfscope}%
\begin{pgfscope}%
\pgfsetrectcap%
\pgfsetmiterjoin%
\pgfsetlinewidth{0.803000pt}%
\definecolor{currentstroke}{rgb}{0.000000,0.000000,0.000000}%
\pgfsetstrokecolor{currentstroke}%
\pgfsetdash{}{0pt}%
\pgfpathmoveto{\pgfqpoint{0.713248in}{2.405353in}}%
\pgfpathlineto{\pgfqpoint{5.363248in}{2.405353in}}%
\pgfusepath{stroke}%
\end{pgfscope}%
\end{pgfpicture}%
\makeatother%
\endgroup%

     \caption{The solution obtained from solving the problem. The mesh is refined around the interface in order to increase the precision.}
     \label{fig:solution-2d}
\end{subfigure}
\begin{subfigure}[t]{.49\textwidth}
    \hspace{12pt}
    \input{plots/normal_displacement_2d.pgf}
    \caption{Penetration depths $u_z$ for multiple vertical displacements $d$ scattered against the exact values obtained with \refequ{equ:normal-displacement}. The quantities are normalized by the mesh size $h$ at the interface.}
    \label{fig:normal-displacements}
\end{subfigure}
\begin{subfigure}[t]{.49\textwidth}
     %% Creator: Matplotlib, PGF backend
%%
%% To include the figure in your LaTeX document, write
%%   \input{<filename>.pgf}
%%
%% Make sure the required packages are loaded in your preamble
%%   \usepackage{pgf}
%%
%% Also ensure that all the required font packages are loaded; for instance,
%% the lmodern package is sometimes necessary when using math font.
%%   \usepackage{lmodern}
%%
%% Figures using additional raster images can only be included by \input if
%% they are in the same directory as the main LaTeX file. For loading figures
%% from other directories you can use the `import` package
%%   \usepackage{import}
%%
%% and then include the figures with
%%   \import{<path to file>}{<filename>.pgf}
%%
%% Matplotlib used the following preamble
%%   
%%   \usepackage{fontspec}
%%   \setmainfont{DejaVuSans.ttf}[Path=\detokenize{/home/fabio/.local/lib/python3.8/site-packages/matplotlib/mpl-data/fonts/ttf/}]
%%   \setsansfont{DejaVuSans.ttf}[Path=\detokenize{/home/fabio/.local/lib/python3.8/site-packages/matplotlib/mpl-data/fonts/ttf/}]
%%   \setmonofont{DejaVuSansMono.ttf}[Path=\detokenize{/home/fabio/.local/lib/python3.8/site-packages/matplotlib/mpl-data/fonts/ttf/}]
%%   \makeatletter\@ifpackageloaded{underscore}{}{\usepackage[strings]{underscore}}\makeatother
%%
\begingroup%
\makeatletter%
\begin{pgfpicture}%
\pgfpathrectangle{\pgfpointorigin}{\pgfqpoint{2.594444in}{2.577715in}}%
\pgfusepath{use as bounding box, clip}%
\begin{pgfscope}%
\pgfsetbuttcap%
\pgfsetmiterjoin%
\definecolor{currentfill}{rgb}{1.000000,1.000000,1.000000}%
\pgfsetfillcolor{currentfill}%
\pgfsetlinewidth{0.000000pt}%
\definecolor{currentstroke}{rgb}{1.000000,1.000000,1.000000}%
\pgfsetstrokecolor{currentstroke}%
\pgfsetdash{}{0pt}%
\pgfpathmoveto{\pgfqpoint{0.000000in}{0.000000in}}%
\pgfpathlineto{\pgfqpoint{2.594444in}{0.000000in}}%
\pgfpathlineto{\pgfqpoint{2.594444in}{2.577715in}}%
\pgfpathlineto{\pgfqpoint{0.000000in}{2.577715in}}%
\pgfpathlineto{\pgfqpoint{0.000000in}{0.000000in}}%
\pgfpathclose%
\pgfusepath{fill}%
\end{pgfscope}%
\begin{pgfscope}%
\pgfsetbuttcap%
\pgfsetmiterjoin%
\definecolor{currentfill}{rgb}{1.000000,1.000000,1.000000}%
\pgfsetfillcolor{currentfill}%
\pgfsetlinewidth{0.000000pt}%
\definecolor{currentstroke}{rgb}{0.000000,0.000000,0.000000}%
\pgfsetstrokecolor{currentstroke}%
\pgfsetstrokeopacity{0.000000}%
\pgfsetdash{}{0pt}%
\pgfpathmoveto{\pgfqpoint{0.556944in}{0.552715in}}%
\pgfpathlineto{\pgfqpoint{2.494444in}{0.552715in}}%
\pgfpathlineto{\pgfqpoint{2.494444in}{2.477715in}}%
\pgfpathlineto{\pgfqpoint{0.556944in}{2.477715in}}%
\pgfpathlineto{\pgfqpoint{0.556944in}{0.552715in}}%
\pgfpathclose%
\pgfusepath{fill}%
\end{pgfscope}%
\begin{pgfscope}%
\pgfpathrectangle{\pgfqpoint{0.556944in}{0.552715in}}{\pgfqpoint{1.937500in}{1.925000in}}%
\pgfusepath{clip}%
\pgfsetbuttcap%
\pgfsetroundjoin%
\definecolor{currentfill}{rgb}{0.054902,0.262745,0.486275}%
\pgfsetfillcolor{currentfill}%
\pgfsetlinewidth{1.003750pt}%
\definecolor{currentstroke}{rgb}{0.054902,0.262745,0.486275}%
\pgfsetstrokecolor{currentstroke}%
\pgfsetdash{}{0pt}%
\pgfsys@defobject{currentmarker}{\pgfqpoint{-0.041667in}{-0.041667in}}{\pgfqpoint{0.041667in}{0.041667in}}{%
\pgfpathmoveto{\pgfqpoint{0.000000in}{-0.041667in}}%
\pgfpathcurveto{\pgfqpoint{0.011050in}{-0.041667in}}{\pgfqpoint{0.021649in}{-0.037276in}}{\pgfqpoint{0.029463in}{-0.029463in}}%
\pgfpathcurveto{\pgfqpoint{0.037276in}{-0.021649in}}{\pgfqpoint{0.041667in}{-0.011050in}}{\pgfqpoint{0.041667in}{0.000000in}}%
\pgfpathcurveto{\pgfqpoint{0.041667in}{0.011050in}}{\pgfqpoint{0.037276in}{0.021649in}}{\pgfqpoint{0.029463in}{0.029463in}}%
\pgfpathcurveto{\pgfqpoint{0.021649in}{0.037276in}}{\pgfqpoint{0.011050in}{0.041667in}}{\pgfqpoint{0.000000in}{0.041667in}}%
\pgfpathcurveto{\pgfqpoint{-0.011050in}{0.041667in}}{\pgfqpoint{-0.021649in}{0.037276in}}{\pgfqpoint{-0.029463in}{0.029463in}}%
\pgfpathcurveto{\pgfqpoint{-0.037276in}{0.021649in}}{\pgfqpoint{-0.041667in}{0.011050in}}{\pgfqpoint{-0.041667in}{0.000000in}}%
\pgfpathcurveto{\pgfqpoint{-0.041667in}{-0.011050in}}{\pgfqpoint{-0.037276in}{-0.021649in}}{\pgfqpoint{-0.029463in}{-0.029463in}}%
\pgfpathcurveto{\pgfqpoint{-0.021649in}{-0.037276in}}{\pgfqpoint{-0.011050in}{-0.041667in}}{\pgfqpoint{0.000000in}{-0.041667in}}%
\pgfpathlineto{\pgfqpoint{0.000000in}{-0.041667in}}%
\pgfpathclose%
\pgfusepath{stroke,fill}%
}%
\begin{pgfscope}%
\pgfsys@transformshift{0.815278in}{0.876128in}%
\pgfsys@useobject{currentmarker}{}%
\end{pgfscope}%
\begin{pgfscope}%
\pgfsys@transformshift{0.965972in}{1.125754in}%
\pgfsys@useobject{currentmarker}{}%
\end{pgfscope}%
\begin{pgfscope}%
\pgfsys@transformshift{1.116667in}{1.124985in}%
\pgfsys@useobject{currentmarker}{}%
\end{pgfscope}%
\begin{pgfscope}%
\pgfsys@transformshift{1.267361in}{1.350786in}%
\pgfsys@useobject{currentmarker}{}%
\end{pgfscope}%
\begin{pgfscope}%
\pgfsys@transformshift{1.418056in}{1.607487in}%
\pgfsys@useobject{currentmarker}{}%
\end{pgfscope}%
\begin{pgfscope}%
\pgfsys@transformshift{1.568750in}{1.746229in}%
\pgfsys@useobject{currentmarker}{}%
\end{pgfscope}%
\begin{pgfscope}%
\pgfsys@transformshift{1.719444in}{1.863572in}%
\pgfsys@useobject{currentmarker}{}%
\end{pgfscope}%
\begin{pgfscope}%
\pgfsys@transformshift{1.870139in}{1.867085in}%
\pgfsys@useobject{currentmarker}{}%
\end{pgfscope}%
\begin{pgfscope}%
\pgfsys@transformshift{2.020833in}{2.081921in}%
\pgfsys@useobject{currentmarker}{}%
\end{pgfscope}%
\begin{pgfscope}%
\pgfsys@transformshift{2.171528in}{2.133029in}%
\pgfsys@useobject{currentmarker}{}%
\end{pgfscope}%
\end{pgfscope}%
\begin{pgfscope}%
\pgfsetbuttcap%
\pgfsetroundjoin%
\definecolor{currentfill}{rgb}{0.000000,0.000000,0.000000}%
\pgfsetfillcolor{currentfill}%
\pgfsetlinewidth{0.803000pt}%
\definecolor{currentstroke}{rgb}{0.000000,0.000000,0.000000}%
\pgfsetstrokecolor{currentstroke}%
\pgfsetdash{}{0pt}%
\pgfsys@defobject{currentmarker}{\pgfqpoint{0.000000in}{-0.048611in}}{\pgfqpoint{0.000000in}{0.000000in}}{%
\pgfpathmoveto{\pgfqpoint{0.000000in}{0.000000in}}%
\pgfpathlineto{\pgfqpoint{0.000000in}{-0.048611in}}%
\pgfusepath{stroke,fill}%
}%
\begin{pgfscope}%
\pgfsys@transformshift{0.911945in}{0.552715in}%
\pgfsys@useobject{currentmarker}{}%
\end{pgfscope}%
\end{pgfscope}%
\begin{pgfscope}%
\definecolor{textcolor}{rgb}{0.000000,0.000000,0.000000}%
\pgfsetstrokecolor{textcolor}%
\pgfsetfillcolor{textcolor}%
\pgftext[x=0.911945in,y=0.455493in,,top]{\color{textcolor}\rmfamily\fontsize{11.000000}{13.200000}\selectfont \(\displaystyle {2}\)}%
\end{pgfscope}%
\begin{pgfscope}%
\pgfsetbuttcap%
\pgfsetroundjoin%
\definecolor{currentfill}{rgb}{0.000000,0.000000,0.000000}%
\pgfsetfillcolor{currentfill}%
\pgfsetlinewidth{0.803000pt}%
\definecolor{currentstroke}{rgb}{0.000000,0.000000,0.000000}%
\pgfsetstrokecolor{currentstroke}%
\pgfsetdash{}{0pt}%
\pgfsys@defobject{currentmarker}{\pgfqpoint{0.000000in}{-0.048611in}}{\pgfqpoint{0.000000in}{0.000000in}}{%
\pgfpathmoveto{\pgfqpoint{0.000000in}{0.000000in}}%
\pgfpathlineto{\pgfqpoint{0.000000in}{-0.048611in}}%
\pgfusepath{stroke,fill}%
}%
\begin{pgfscope}%
\pgfsys@transformshift{1.525279in}{0.552715in}%
\pgfsys@useobject{currentmarker}{}%
\end{pgfscope}%
\end{pgfscope}%
\begin{pgfscope}%
\definecolor{textcolor}{rgb}{0.000000,0.000000,0.000000}%
\pgfsetstrokecolor{textcolor}%
\pgfsetfillcolor{textcolor}%
\pgftext[x=1.525279in,y=0.455493in,,top]{\color{textcolor}\rmfamily\fontsize{11.000000}{13.200000}\selectfont \(\displaystyle {4}\)}%
\end{pgfscope}%
\begin{pgfscope}%
\pgfsetbuttcap%
\pgfsetroundjoin%
\definecolor{currentfill}{rgb}{0.000000,0.000000,0.000000}%
\pgfsetfillcolor{currentfill}%
\pgfsetlinewidth{0.803000pt}%
\definecolor{currentstroke}{rgb}{0.000000,0.000000,0.000000}%
\pgfsetstrokecolor{currentstroke}%
\pgfsetdash{}{0pt}%
\pgfsys@defobject{currentmarker}{\pgfqpoint{0.000000in}{-0.048611in}}{\pgfqpoint{0.000000in}{0.000000in}}{%
\pgfpathmoveto{\pgfqpoint{0.000000in}{0.000000in}}%
\pgfpathlineto{\pgfqpoint{0.000000in}{-0.048611in}}%
\pgfusepath{stroke,fill}%
}%
\begin{pgfscope}%
\pgfsys@transformshift{2.138612in}{0.552715in}%
\pgfsys@useobject{currentmarker}{}%
\end{pgfscope}%
\end{pgfscope}%
\begin{pgfscope}%
\definecolor{textcolor}{rgb}{0.000000,0.000000,0.000000}%
\pgfsetstrokecolor{textcolor}%
\pgfsetfillcolor{textcolor}%
\pgftext[x=2.138612in,y=0.455493in,,top]{\color{textcolor}\rmfamily\fontsize{11.000000}{13.200000}\selectfont \(\displaystyle {6}\)}%
\end{pgfscope}%
\begin{pgfscope}%
\definecolor{textcolor}{rgb}{0.000000,0.000000,0.000000}%
\pgfsetstrokecolor{textcolor}%
\pgfsetfillcolor{textcolor}%
\pgftext[x=1.525694in,y=0.252083in,,top]{\color{textcolor}\rmfamily\fontsize{11.000000}{13.200000}\selectfont \(\displaystyle d/h\)}%
\end{pgfscope}%
\begin{pgfscope}%
\pgfsetbuttcap%
\pgfsetroundjoin%
\definecolor{currentfill}{rgb}{0.000000,0.000000,0.000000}%
\pgfsetfillcolor{currentfill}%
\pgfsetlinewidth{0.803000pt}%
\definecolor{currentstroke}{rgb}{0.000000,0.000000,0.000000}%
\pgfsetstrokecolor{currentstroke}%
\pgfsetdash{}{0pt}%
\pgfsys@defobject{currentmarker}{\pgfqpoint{-0.048611in}{0.000000in}}{\pgfqpoint{-0.000000in}{0.000000in}}{%
\pgfpathmoveto{\pgfqpoint{-0.000000in}{0.000000in}}%
\pgfpathlineto{\pgfqpoint{-0.048611in}{0.000000in}}%
\pgfusepath{stroke,fill}%
}%
\begin{pgfscope}%
\pgfsys@transformshift{0.556944in}{0.594240in}%
\pgfsys@useobject{currentmarker}{}%
\end{pgfscope}%
\end{pgfscope}%
\begin{pgfscope}%
\definecolor{textcolor}{rgb}{0.000000,0.000000,0.000000}%
\pgfsetstrokecolor{textcolor}%
\pgfsetfillcolor{textcolor}%
\pgftext[x=0.383681in, y=0.536203in, left, base]{\color{textcolor}\rmfamily\fontsize{11.000000}{13.200000}\selectfont \(\displaystyle {4}\)}%
\end{pgfscope}%
\begin{pgfscope}%
\pgfsetbuttcap%
\pgfsetroundjoin%
\definecolor{currentfill}{rgb}{0.000000,0.000000,0.000000}%
\pgfsetfillcolor{currentfill}%
\pgfsetlinewidth{0.803000pt}%
\definecolor{currentstroke}{rgb}{0.000000,0.000000,0.000000}%
\pgfsetstrokecolor{currentstroke}%
\pgfsetdash{}{0pt}%
\pgfsys@defobject{currentmarker}{\pgfqpoint{-0.048611in}{0.000000in}}{\pgfqpoint{-0.000000in}{0.000000in}}{%
\pgfpathmoveto{\pgfqpoint{-0.000000in}{0.000000in}}%
\pgfpathlineto{\pgfqpoint{-0.048611in}{0.000000in}}%
\pgfusepath{stroke,fill}%
}%
\begin{pgfscope}%
\pgfsys@transformshift{0.556944in}{1.028058in}%
\pgfsys@useobject{currentmarker}{}%
\end{pgfscope}%
\end{pgfscope}%
\begin{pgfscope}%
\definecolor{textcolor}{rgb}{0.000000,0.000000,0.000000}%
\pgfsetstrokecolor{textcolor}%
\pgfsetfillcolor{textcolor}%
\pgftext[x=0.383681in, y=0.970021in, left, base]{\color{textcolor}\rmfamily\fontsize{11.000000}{13.200000}\selectfont \(\displaystyle {6}\)}%
\end{pgfscope}%
\begin{pgfscope}%
\pgfsetbuttcap%
\pgfsetroundjoin%
\definecolor{currentfill}{rgb}{0.000000,0.000000,0.000000}%
\pgfsetfillcolor{currentfill}%
\pgfsetlinewidth{0.803000pt}%
\definecolor{currentstroke}{rgb}{0.000000,0.000000,0.000000}%
\pgfsetstrokecolor{currentstroke}%
\pgfsetdash{}{0pt}%
\pgfsys@defobject{currentmarker}{\pgfqpoint{-0.048611in}{0.000000in}}{\pgfqpoint{-0.000000in}{0.000000in}}{%
\pgfpathmoveto{\pgfqpoint{-0.000000in}{0.000000in}}%
\pgfpathlineto{\pgfqpoint{-0.048611in}{0.000000in}}%
\pgfusepath{stroke,fill}%
}%
\begin{pgfscope}%
\pgfsys@transformshift{0.556944in}{1.461877in}%
\pgfsys@useobject{currentmarker}{}%
\end{pgfscope}%
\end{pgfscope}%
\begin{pgfscope}%
\definecolor{textcolor}{rgb}{0.000000,0.000000,0.000000}%
\pgfsetstrokecolor{textcolor}%
\pgfsetfillcolor{textcolor}%
\pgftext[x=0.383681in, y=1.403839in, left, base]{\color{textcolor}\rmfamily\fontsize{11.000000}{13.200000}\selectfont \(\displaystyle {8}\)}%
\end{pgfscope}%
\begin{pgfscope}%
\pgfsetbuttcap%
\pgfsetroundjoin%
\definecolor{currentfill}{rgb}{0.000000,0.000000,0.000000}%
\pgfsetfillcolor{currentfill}%
\pgfsetlinewidth{0.803000pt}%
\definecolor{currentstroke}{rgb}{0.000000,0.000000,0.000000}%
\pgfsetstrokecolor{currentstroke}%
\pgfsetdash{}{0pt}%
\pgfsys@defobject{currentmarker}{\pgfqpoint{-0.048611in}{0.000000in}}{\pgfqpoint{-0.000000in}{0.000000in}}{%
\pgfpathmoveto{\pgfqpoint{-0.000000in}{0.000000in}}%
\pgfpathlineto{\pgfqpoint{-0.048611in}{0.000000in}}%
\pgfusepath{stroke,fill}%
}%
\begin{pgfscope}%
\pgfsys@transformshift{0.556944in}{1.895695in}%
\pgfsys@useobject{currentmarker}{}%
\end{pgfscope}%
\end{pgfscope}%
\begin{pgfscope}%
\definecolor{textcolor}{rgb}{0.000000,0.000000,0.000000}%
\pgfsetstrokecolor{textcolor}%
\pgfsetfillcolor{textcolor}%
\pgftext[x=0.307639in, y=1.837657in, left, base]{\color{textcolor}\rmfamily\fontsize{11.000000}{13.200000}\selectfont \(\displaystyle {10}\)}%
\end{pgfscope}%
\begin{pgfscope}%
\pgfsetbuttcap%
\pgfsetroundjoin%
\definecolor{currentfill}{rgb}{0.000000,0.000000,0.000000}%
\pgfsetfillcolor{currentfill}%
\pgfsetlinewidth{0.803000pt}%
\definecolor{currentstroke}{rgb}{0.000000,0.000000,0.000000}%
\pgfsetstrokecolor{currentstroke}%
\pgfsetdash{}{0pt}%
\pgfsys@defobject{currentmarker}{\pgfqpoint{-0.048611in}{0.000000in}}{\pgfqpoint{-0.000000in}{0.000000in}}{%
\pgfpathmoveto{\pgfqpoint{-0.000000in}{0.000000in}}%
\pgfpathlineto{\pgfqpoint{-0.048611in}{0.000000in}}%
\pgfusepath{stroke,fill}%
}%
\begin{pgfscope}%
\pgfsys@transformshift{0.556944in}{2.329513in}%
\pgfsys@useobject{currentmarker}{}%
\end{pgfscope}%
\end{pgfscope}%
\begin{pgfscope}%
\definecolor{textcolor}{rgb}{0.000000,0.000000,0.000000}%
\pgfsetstrokecolor{textcolor}%
\pgfsetfillcolor{textcolor}%
\pgftext[x=0.307639in, y=2.271475in, left, base]{\color{textcolor}\rmfamily\fontsize{11.000000}{13.200000}\selectfont \(\displaystyle {12}\)}%
\end{pgfscope}%
\begin{pgfscope}%
\definecolor{textcolor}{rgb}{0.000000,0.000000,0.000000}%
\pgfsetstrokecolor{textcolor}%
\pgfsetfillcolor{textcolor}%
\pgftext[x=0.252083in,y=1.515215in,,bottom,rotate=90.000000]{\color{textcolor}\rmfamily\fontsize{11.000000}{13.200000}\selectfont \(\displaystyle a/h\)}%
\end{pgfscope}%
\begin{pgfscope}%
\pgfpathrectangle{\pgfqpoint{0.556944in}{0.552715in}}{\pgfqpoint{1.937500in}{1.925000in}}%
\pgfusepath{clip}%
\pgfsetrectcap%
\pgfsetroundjoin%
\pgfsetlinewidth{1.505625pt}%
\definecolor{currentstroke}{rgb}{0.054902,0.262745,0.486275}%
\pgfsetstrokecolor{currentstroke}%
\pgfsetstrokeopacity{0.250000}%
\pgfsetdash{}{0pt}%
\pgfpathmoveto{\pgfqpoint{0.556944in}{0.640215in}}%
\pgfpathlineto{\pgfqpoint{0.576515in}{0.674190in}}%
\pgfpathlineto{\pgfqpoint{0.596086in}{0.706988in}}%
\pgfpathlineto{\pgfqpoint{0.615657in}{0.738724in}}%
\pgfpathlineto{\pgfqpoint{0.635227in}{0.769494in}}%
\pgfpathlineto{\pgfqpoint{0.654798in}{0.799382in}}%
\pgfpathlineto{\pgfqpoint{0.674369in}{0.828460in}}%
\pgfpathlineto{\pgfqpoint{0.693939in}{0.856790in}}%
\pgfpathlineto{\pgfqpoint{0.713510in}{0.884427in}}%
\pgfpathlineto{\pgfqpoint{0.733081in}{0.911420in}}%
\pgfpathlineto{\pgfqpoint{0.752652in}{0.937811in}}%
\pgfpathlineto{\pgfqpoint{0.772222in}{0.963639in}}%
\pgfpathlineto{\pgfqpoint{0.791793in}{0.988939in}}%
\pgfpathlineto{\pgfqpoint{0.811364in}{1.013742in}}%
\pgfpathlineto{\pgfqpoint{0.830934in}{1.038075in}}%
\pgfpathlineto{\pgfqpoint{0.850505in}{1.061966in}}%
\pgfpathlineto{\pgfqpoint{0.870076in}{1.085436in}}%
\pgfpathlineto{\pgfqpoint{0.889646in}{1.108508in}}%
\pgfpathlineto{\pgfqpoint{0.909217in}{1.131201in}}%
\pgfpathlineto{\pgfqpoint{0.928788in}{1.153533in}}%
\pgfpathlineto{\pgfqpoint{0.948359in}{1.175521in}}%
\pgfpathlineto{\pgfqpoint{0.967929in}{1.197180in}}%
\pgfpathlineto{\pgfqpoint{0.987500in}{1.218525in}}%
\pgfpathlineto{\pgfqpoint{1.007071in}{1.239568in}}%
\pgfpathlineto{\pgfqpoint{1.026641in}{1.260323in}}%
\pgfpathlineto{\pgfqpoint{1.046212in}{1.280801in}}%
\pgfpathlineto{\pgfqpoint{1.065783in}{1.301012in}}%
\pgfpathlineto{\pgfqpoint{1.085354in}{1.320968in}}%
\pgfpathlineto{\pgfqpoint{1.104924in}{1.340676in}}%
\pgfpathlineto{\pgfqpoint{1.124495in}{1.360147in}}%
\pgfpathlineto{\pgfqpoint{1.144066in}{1.379389in}}%
\pgfpathlineto{\pgfqpoint{1.163636in}{1.398409in}}%
\pgfpathlineto{\pgfqpoint{1.183207in}{1.417215in}}%
\pgfpathlineto{\pgfqpoint{1.202778in}{1.435814in}}%
\pgfpathlineto{\pgfqpoint{1.222348in}{1.454213in}}%
\pgfpathlineto{\pgfqpoint{1.241919in}{1.472418in}}%
\pgfpathlineto{\pgfqpoint{1.261490in}{1.490435in}}%
\pgfpathlineto{\pgfqpoint{1.281061in}{1.508270in}}%
\pgfpathlineto{\pgfqpoint{1.300631in}{1.525928in}}%
\pgfpathlineto{\pgfqpoint{1.320202in}{1.543414in}}%
\pgfpathlineto{\pgfqpoint{1.339773in}{1.560734in}}%
\pgfpathlineto{\pgfqpoint{1.359343in}{1.577892in}}%
\pgfpathlineto{\pgfqpoint{1.378914in}{1.594892in}}%
\pgfpathlineto{\pgfqpoint{1.398485in}{1.611739in}}%
\pgfpathlineto{\pgfqpoint{1.418056in}{1.628437in}}%
\pgfpathlineto{\pgfqpoint{1.437626in}{1.644989in}}%
\pgfpathlineto{\pgfqpoint{1.457197in}{1.661400in}}%
\pgfpathlineto{\pgfqpoint{1.476768in}{1.677673in}}%
\pgfpathlineto{\pgfqpoint{1.496338in}{1.693811in}}%
\pgfpathlineto{\pgfqpoint{1.515909in}{1.709818in}}%
\pgfpathlineto{\pgfqpoint{1.535480in}{1.725697in}}%
\pgfpathlineto{\pgfqpoint{1.555051in}{1.741450in}}%
\pgfpathlineto{\pgfqpoint{1.574621in}{1.757081in}}%
\pgfpathlineto{\pgfqpoint{1.594192in}{1.772593in}}%
\pgfpathlineto{\pgfqpoint{1.613763in}{1.787988in}}%
\pgfpathlineto{\pgfqpoint{1.633333in}{1.803270in}}%
\pgfpathlineto{\pgfqpoint{1.652904in}{1.818439in}}%
\pgfpathlineto{\pgfqpoint{1.672475in}{1.833499in}}%
\pgfpathlineto{\pgfqpoint{1.692045in}{1.848452in}}%
\pgfpathlineto{\pgfqpoint{1.711616in}{1.863301in}}%
\pgfpathlineto{\pgfqpoint{1.731187in}{1.878047in}}%
\pgfpathlineto{\pgfqpoint{1.750758in}{1.892693in}}%
\pgfpathlineto{\pgfqpoint{1.770328in}{1.907241in}}%
\pgfpathlineto{\pgfqpoint{1.789899in}{1.921692in}}%
\pgfpathlineto{\pgfqpoint{1.809470in}{1.936048in}}%
\pgfpathlineto{\pgfqpoint{1.829040in}{1.950312in}}%
\pgfpathlineto{\pgfqpoint{1.848611in}{1.964485in}}%
\pgfpathlineto{\pgfqpoint{1.868182in}{1.978569in}}%
\pgfpathlineto{\pgfqpoint{1.887753in}{1.992565in}}%
\pgfpathlineto{\pgfqpoint{1.907323in}{2.006475in}}%
\pgfpathlineto{\pgfqpoint{1.926894in}{2.020301in}}%
\pgfpathlineto{\pgfqpoint{1.946465in}{2.034044in}}%
\pgfpathlineto{\pgfqpoint{1.966035in}{2.047706in}}%
\pgfpathlineto{\pgfqpoint{1.985606in}{2.061287in}}%
\pgfpathlineto{\pgfqpoint{2.005177in}{2.074791in}}%
\pgfpathlineto{\pgfqpoint{2.024747in}{2.088217in}}%
\pgfpathlineto{\pgfqpoint{2.044318in}{2.101567in}}%
\pgfpathlineto{\pgfqpoint{2.063889in}{2.114842in}}%
\pgfpathlineto{\pgfqpoint{2.083460in}{2.128044in}}%
\pgfpathlineto{\pgfqpoint{2.103030in}{2.141174in}}%
\pgfpathlineto{\pgfqpoint{2.122601in}{2.154233in}}%
\pgfpathlineto{\pgfqpoint{2.142172in}{2.167222in}}%
\pgfpathlineto{\pgfqpoint{2.161742in}{2.180142in}}%
\pgfpathlineto{\pgfqpoint{2.181313in}{2.192995in}}%
\pgfpathlineto{\pgfqpoint{2.200884in}{2.205781in}}%
\pgfpathlineto{\pgfqpoint{2.220455in}{2.218501in}}%
\pgfpathlineto{\pgfqpoint{2.240025in}{2.231157in}}%
\pgfpathlineto{\pgfqpoint{2.259596in}{2.243749in}}%
\pgfpathlineto{\pgfqpoint{2.279167in}{2.256278in}}%
\pgfpathlineto{\pgfqpoint{2.298737in}{2.268746in}}%
\pgfpathlineto{\pgfqpoint{2.318308in}{2.281153in}}%
\pgfpathlineto{\pgfqpoint{2.337879in}{2.293500in}}%
\pgfpathlineto{\pgfqpoint{2.357449in}{2.305787in}}%
\pgfpathlineto{\pgfqpoint{2.377020in}{2.318017in}}%
\pgfpathlineto{\pgfqpoint{2.396591in}{2.330189in}}%
\pgfpathlineto{\pgfqpoint{2.416162in}{2.342304in}}%
\pgfpathlineto{\pgfqpoint{2.435732in}{2.354364in}}%
\pgfpathlineto{\pgfqpoint{2.455303in}{2.366368in}}%
\pgfpathlineto{\pgfqpoint{2.474874in}{2.378319in}}%
\pgfpathlineto{\pgfqpoint{2.494444in}{2.390215in}}%
\pgfusepath{stroke}%
\end{pgfscope}%
\begin{pgfscope}%
\pgfsetrectcap%
\pgfsetmiterjoin%
\pgfsetlinewidth{0.803000pt}%
\definecolor{currentstroke}{rgb}{0.000000,0.000000,0.000000}%
\pgfsetstrokecolor{currentstroke}%
\pgfsetdash{}{0pt}%
\pgfpathmoveto{\pgfqpoint{0.556944in}{0.552715in}}%
\pgfpathlineto{\pgfqpoint{0.556944in}{2.477715in}}%
\pgfusepath{stroke}%
\end{pgfscope}%
\begin{pgfscope}%
\pgfsetrectcap%
\pgfsetmiterjoin%
\pgfsetlinewidth{0.803000pt}%
\definecolor{currentstroke}{rgb}{0.000000,0.000000,0.000000}%
\pgfsetstrokecolor{currentstroke}%
\pgfsetdash{}{0pt}%
\pgfpathmoveto{\pgfqpoint{2.494444in}{0.552715in}}%
\pgfpathlineto{\pgfqpoint{2.494444in}{2.477715in}}%
\pgfusepath{stroke}%
\end{pgfscope}%
\begin{pgfscope}%
\pgfsetrectcap%
\pgfsetmiterjoin%
\pgfsetlinewidth{0.803000pt}%
\definecolor{currentstroke}{rgb}{0.000000,0.000000,0.000000}%
\pgfsetstrokecolor{currentstroke}%
\pgfsetdash{}{0pt}%
\pgfpathmoveto{\pgfqpoint{0.556944in}{0.552715in}}%
\pgfpathlineto{\pgfqpoint{2.494444in}{0.552715in}}%
\pgfusepath{stroke}%
\end{pgfscope}%
\begin{pgfscope}%
\pgfsetrectcap%
\pgfsetmiterjoin%
\pgfsetlinewidth{0.803000pt}%
\definecolor{currentstroke}{rgb}{0.000000,0.000000,0.000000}%
\pgfsetstrokecolor{currentstroke}%
\pgfsetdash{}{0pt}%
\pgfpathmoveto{\pgfqpoint{0.556944in}{2.477715in}}%
\pgfpathlineto{\pgfqpoint{2.494444in}{2.477715in}}%
\pgfusepath{stroke}%
\end{pgfscope}%
\begin{pgfscope}%
\pgfsetbuttcap%
\pgfsetmiterjoin%
\definecolor{currentfill}{rgb}{1.000000,1.000000,1.000000}%
\pgfsetfillcolor{currentfill}%
\pgfsetfillopacity{0.800000}%
\pgfsetlinewidth{1.003750pt}%
\definecolor{currentstroke}{rgb}{0.800000,0.800000,0.800000}%
\pgfsetstrokecolor{currentstroke}%
\pgfsetstrokeopacity{0.800000}%
\pgfsetdash{}{0pt}%
\pgfpathmoveto{\pgfqpoint{0.897574in}{0.629104in}}%
\pgfpathlineto{\pgfqpoint{2.387500in}{0.629104in}}%
\pgfpathquadraticcurveto{\pgfqpoint{2.418056in}{0.629104in}}{\pgfqpoint{2.418056in}{0.659659in}}%
\pgfpathlineto{\pgfqpoint{2.418056in}{1.092868in}}%
\pgfpathquadraticcurveto{\pgfqpoint{2.418056in}{1.123423in}}{\pgfqpoint{2.387500in}{1.123423in}}%
\pgfpathlineto{\pgfqpoint{0.897574in}{1.123423in}}%
\pgfpathquadraticcurveto{\pgfqpoint{0.867018in}{1.123423in}}{\pgfqpoint{0.867018in}{1.092868in}}%
\pgfpathlineto{\pgfqpoint{0.867018in}{0.659659in}}%
\pgfpathquadraticcurveto{\pgfqpoint{0.867018in}{0.629104in}}{\pgfqpoint{0.897574in}{0.629104in}}%
\pgfpathlineto{\pgfqpoint{0.897574in}{0.629104in}}%
\pgfpathclose%
\pgfusepath{stroke,fill}%
\end{pgfscope}%
\begin{pgfscope}%
\pgfsetbuttcap%
\pgfsetroundjoin%
\definecolor{currentfill}{rgb}{0.054902,0.262745,0.486275}%
\pgfsetfillcolor{currentfill}%
\pgfsetlinewidth{1.003750pt}%
\definecolor{currentstroke}{rgb}{0.054902,0.262745,0.486275}%
\pgfsetstrokecolor{currentstroke}%
\pgfsetdash{}{0pt}%
\pgfsys@defobject{currentmarker}{\pgfqpoint{-0.041667in}{-0.041667in}}{\pgfqpoint{0.041667in}{0.041667in}}{%
\pgfpathmoveto{\pgfqpoint{0.000000in}{-0.041667in}}%
\pgfpathcurveto{\pgfqpoint{0.011050in}{-0.041667in}}{\pgfqpoint{0.021649in}{-0.037276in}}{\pgfqpoint{0.029463in}{-0.029463in}}%
\pgfpathcurveto{\pgfqpoint{0.037276in}{-0.021649in}}{\pgfqpoint{0.041667in}{-0.011050in}}{\pgfqpoint{0.041667in}{0.000000in}}%
\pgfpathcurveto{\pgfqpoint{0.041667in}{0.011050in}}{\pgfqpoint{0.037276in}{0.021649in}}{\pgfqpoint{0.029463in}{0.029463in}}%
\pgfpathcurveto{\pgfqpoint{0.021649in}{0.037276in}}{\pgfqpoint{0.011050in}{0.041667in}}{\pgfqpoint{0.000000in}{0.041667in}}%
\pgfpathcurveto{\pgfqpoint{-0.011050in}{0.041667in}}{\pgfqpoint{-0.021649in}{0.037276in}}{\pgfqpoint{-0.029463in}{0.029463in}}%
\pgfpathcurveto{\pgfqpoint{-0.037276in}{0.021649in}}{\pgfqpoint{-0.041667in}{0.011050in}}{\pgfqpoint{-0.041667in}{0.000000in}}%
\pgfpathcurveto{\pgfqpoint{-0.041667in}{-0.011050in}}{\pgfqpoint{-0.037276in}{-0.021649in}}{\pgfqpoint{-0.029463in}{-0.029463in}}%
\pgfpathcurveto{\pgfqpoint{-0.021649in}{-0.037276in}}{\pgfqpoint{-0.011050in}{-0.041667in}}{\pgfqpoint{0.000000in}{-0.041667in}}%
\pgfpathlineto{\pgfqpoint{0.000000in}{-0.041667in}}%
\pgfpathclose%
\pgfusepath{stroke,fill}%
}%
\begin{pgfscope}%
\pgfsys@transformshift{1.080907in}{0.986341in}%
\pgfsys@useobject{currentmarker}{}%
\end{pgfscope}%
\end{pgfscope}%
\begin{pgfscope}%
\definecolor{textcolor}{rgb}{0.000000,0.000000,0.000000}%
\pgfsetstrokecolor{textcolor}%
\pgfsetfillcolor{textcolor}%
\pgftext[x=1.355907in,y=0.946237in,left,base]{\color{textcolor}\rmfamily\fontsize{11.000000}{13.200000}\selectfont INTERNODES}%
\end{pgfscope}%
\begin{pgfscope}%
\pgfsetrectcap%
\pgfsetroundjoin%
\pgfsetlinewidth{1.505625pt}%
\definecolor{currentstroke}{rgb}{0.054902,0.262745,0.486275}%
\pgfsetstrokecolor{currentstroke}%
\pgfsetstrokeopacity{0.250000}%
\pgfsetdash{}{0pt}%
\pgfpathmoveto{\pgfqpoint{0.928129in}{0.775466in}}%
\pgfpathlineto{\pgfqpoint{1.080907in}{0.775466in}}%
\pgfpathlineto{\pgfqpoint{1.233685in}{0.775466in}}%
\pgfusepath{stroke}%
\end{pgfscope}%
\begin{pgfscope}%
\definecolor{textcolor}{rgb}{0.000000,0.000000,0.000000}%
\pgfsetstrokecolor{textcolor}%
\pgfsetfillcolor{textcolor}%
\pgftext[x=1.355907in,y=0.721994in,left,base]{\color{textcolor}\rmfamily\fontsize{11.000000}{13.200000}\selectfont Theory}%
\end{pgfscope}%
\end{pgfpicture}%
\makeatother%
\endgroup%

     \caption{Radii of contact areas $a$ for multiple vertical displacements $d$ scattered against the exact values obtained with \refequ{equ:contact-radius}. The quantities are normalized by the mesh size $h$ at the interface.}
     \label{fig:contact-radius}
\end{subfigure}
\caption{Results obtained for a problem of the type sketched in \reffig{fig:sketch-plane-sphere}. For a fixed mesh of mesh size $h$ at the interface, multiple displacements $d$ are applied.}
\label{fig:contact-plane-circle}
\end{figure}

A similar analysis is now done for the three dimensional case. Here, a sphere is pushed into a half-space, analogously to \reffig{fig:sketch-plane-sphere}, and the results are visualized in \reffig{fig:contact-plane-sphere}. Notice that in three dimensions the computation of the radius of the contact area is not as robust as in two dimensions, since we have more degrees of freedom. The error for said quantity, however, is still within two mesh sizes $h$, which is to be expected.

\begin{figure}[H]
\begin{subfigure}[t]{\textwidth}
\centering
    \includegraphics[scale=0.4, trim={5cm 3cm 6cm 3cm},clip]{figures/mesh3d.png}
    \caption{Solution obtained from solving the problem. The color corresponds to the total displacement of the respective nodes. Orange is associated with a large displacement, while blue means the node was invariant.}
    \label{fig:contact3d-plane-sphere}
\end{subfigure}
\begin{subfigure}[t]{.49\textwidth}
    %% Creator: Matplotlib, PGF backend
%%
%% To include the figure in your LaTeX document, write
%%   \input{<filename>.pgf}
%%
%% Make sure the required packages are loaded in your preamble
%%   \usepackage{pgf}
%%
%% Also ensure that all the required font packages are loaded; for instance,
%% the lmodern package is sometimes necessary when using math font.
%%   \usepackage{lmodern}
%%
%% Figures using additional raster images can only be included by \input if
%% they are in the same directory as the main LaTeX file. For loading figures
%% from other directories you can use the `import` package
%%   \usepackage{import}
%%
%% and then include the figures with
%%   \import{<path to file>}{<filename>.pgf}
%%
%% Matplotlib used the following preamble
%%   
%%   \usepackage{fontspec}
%%   \setmainfont{DejaVuSans.ttf}[Path=\detokenize{/home/fabio/Internodes-CM/.venv/lib/python3.8/site-packages/matplotlib/mpl-data/fonts/ttf/}]
%%   \setsansfont{DejaVuSans.ttf}[Path=\detokenize{/home/fabio/Internodes-CM/.venv/lib/python3.8/site-packages/matplotlib/mpl-data/fonts/ttf/}]
%%   \setmonofont{DejaVuSansMono.ttf}[Path=\detokenize{/home/fabio/Internodes-CM/.venv/lib/python3.8/site-packages/matplotlib/mpl-data/fonts/ttf/}]
%%   \makeatletter\@ifpackageloaded{underscore}{}{\usepackage[strings]{underscore}}\makeatother
%%
\begingroup%
\makeatletter%
\begin{pgfpicture}%
\pgfpathrectangle{\pgfpointorigin}{\pgfqpoint{5.926790in}{3.343486in}}%
\pgfusepath{use as bounding box, clip}%
\begin{pgfscope}%
\pgfsetbuttcap%
\pgfsetmiterjoin%
\definecolor{currentfill}{rgb}{1.000000,1.000000,1.000000}%
\pgfsetfillcolor{currentfill}%
\pgfsetlinewidth{0.000000pt}%
\definecolor{currentstroke}{rgb}{1.000000,1.000000,1.000000}%
\pgfsetstrokecolor{currentstroke}%
\pgfsetdash{}{0pt}%
\pgfpathmoveto{\pgfqpoint{0.000000in}{0.000000in}}%
\pgfpathlineto{\pgfqpoint{5.926790in}{0.000000in}}%
\pgfpathlineto{\pgfqpoint{5.926790in}{3.343486in}}%
\pgfpathlineto{\pgfqpoint{0.000000in}{3.343486in}}%
\pgfpathlineto{\pgfqpoint{0.000000in}{0.000000in}}%
\pgfpathclose%
\pgfusepath{fill}%
\end{pgfscope}%
\begin{pgfscope}%
\pgfsetbuttcap%
\pgfsetmiterjoin%
\definecolor{currentfill}{rgb}{1.000000,1.000000,1.000000}%
\pgfsetfillcolor{currentfill}%
\pgfsetlinewidth{0.000000pt}%
\definecolor{currentstroke}{rgb}{0.000000,0.000000,0.000000}%
\pgfsetstrokecolor{currentstroke}%
\pgfsetstrokeopacity{0.000000}%
\pgfsetdash{}{0pt}%
\pgfpathmoveto{\pgfqpoint{0.789290in}{0.548486in}}%
\pgfpathlineto{\pgfqpoint{5.826790in}{0.548486in}}%
\pgfpathlineto{\pgfqpoint{5.826790in}{3.243486in}}%
\pgfpathlineto{\pgfqpoint{0.789290in}{3.243486in}}%
\pgfpathlineto{\pgfqpoint{0.789290in}{0.548486in}}%
\pgfpathclose%
\pgfusepath{fill}%
\end{pgfscope}%
\begin{pgfscope}%
\pgfpathrectangle{\pgfqpoint{0.789290in}{0.548486in}}{\pgfqpoint{5.037500in}{2.695000in}}%
\pgfusepath{clip}%
\pgfsetbuttcap%
\pgfsetroundjoin%
\definecolor{currentfill}{rgb}{0.054902,0.262745,0.486275}%
\pgfsetfillcolor{currentfill}%
\pgfsetlinewidth{1.003750pt}%
\definecolor{currentstroke}{rgb}{0.054902,0.262745,0.486275}%
\pgfsetstrokecolor{currentstroke}%
\pgfsetdash{}{0pt}%
\pgfsys@defobject{currentmarker}{\pgfqpoint{-0.041667in}{-0.041667in}}{\pgfqpoint{0.041667in}{0.041667in}}{%
\pgfpathmoveto{\pgfqpoint{0.000000in}{-0.041667in}}%
\pgfpathcurveto{\pgfqpoint{0.011050in}{-0.041667in}}{\pgfqpoint{0.021649in}{-0.037276in}}{\pgfqpoint{0.029463in}{-0.029463in}}%
\pgfpathcurveto{\pgfqpoint{0.037276in}{-0.021649in}}{\pgfqpoint{0.041667in}{-0.011050in}}{\pgfqpoint{0.041667in}{0.000000in}}%
\pgfpathcurveto{\pgfqpoint{0.041667in}{0.011050in}}{\pgfqpoint{0.037276in}{0.021649in}}{\pgfqpoint{0.029463in}{0.029463in}}%
\pgfpathcurveto{\pgfqpoint{0.021649in}{0.037276in}}{\pgfqpoint{0.011050in}{0.041667in}}{\pgfqpoint{0.000000in}{0.041667in}}%
\pgfpathcurveto{\pgfqpoint{-0.011050in}{0.041667in}}{\pgfqpoint{-0.021649in}{0.037276in}}{\pgfqpoint{-0.029463in}{0.029463in}}%
\pgfpathcurveto{\pgfqpoint{-0.037276in}{0.021649in}}{\pgfqpoint{-0.041667in}{0.011050in}}{\pgfqpoint{-0.041667in}{0.000000in}}%
\pgfpathcurveto{\pgfqpoint{-0.041667in}{-0.011050in}}{\pgfqpoint{-0.037276in}{-0.021649in}}{\pgfqpoint{-0.029463in}{-0.029463in}}%
\pgfpathcurveto{\pgfqpoint{-0.021649in}{-0.037276in}}{\pgfqpoint{-0.011050in}{-0.041667in}}{\pgfqpoint{0.000000in}{-0.041667in}}%
\pgfpathlineto{\pgfqpoint{0.000000in}{-0.041667in}}%
\pgfpathclose%
\pgfusepath{stroke,fill}%
}%
\begin{pgfscope}%
\pgfsys@transformshift{1.293040in}{2.955747in}%
\pgfsys@useobject{currentmarker}{}%
\end{pgfscope}%
\begin{pgfscope}%
\pgfsys@transformshift{1.740818in}{2.784307in}%
\pgfsys@useobject{currentmarker}{}%
\end{pgfscope}%
\begin{pgfscope}%
\pgfsys@transformshift{2.188595in}{2.501776in}%
\pgfsys@useobject{currentmarker}{}%
\end{pgfscope}%
\begin{pgfscope}%
\pgfsys@transformshift{2.636373in}{2.247807in}%
\pgfsys@useobject{currentmarker}{}%
\end{pgfscope}%
\begin{pgfscope}%
\pgfsys@transformshift{3.084151in}{2.032879in}%
\pgfsys@useobject{currentmarker}{}%
\end{pgfscope}%
\begin{pgfscope}%
\pgfsys@transformshift{3.531929in}{1.783552in}%
\pgfsys@useobject{currentmarker}{}%
\end{pgfscope}%
\begin{pgfscope}%
\pgfsys@transformshift{3.979707in}{1.580801in}%
\pgfsys@useobject{currentmarker}{}%
\end{pgfscope}%
\begin{pgfscope}%
\pgfsys@transformshift{4.427484in}{1.362953in}%
\pgfsys@useobject{currentmarker}{}%
\end{pgfscope}%
\begin{pgfscope}%
\pgfsys@transformshift{4.875262in}{1.132455in}%
\pgfsys@useobject{currentmarker}{}%
\end{pgfscope}%
\begin{pgfscope}%
\pgfsys@transformshift{5.323040in}{0.917032in}%
\pgfsys@useobject{currentmarker}{}%
\end{pgfscope}%
\end{pgfscope}%
\begin{pgfscope}%
\pgfsetbuttcap%
\pgfsetroundjoin%
\definecolor{currentfill}{rgb}{0.000000,0.000000,0.000000}%
\pgfsetfillcolor{currentfill}%
\pgfsetlinewidth{0.803000pt}%
\definecolor{currentstroke}{rgb}{0.000000,0.000000,0.000000}%
\pgfsetstrokecolor{currentstroke}%
\pgfsetdash{}{0pt}%
\pgfsys@defobject{currentmarker}{\pgfqpoint{0.000000in}{-0.048611in}}{\pgfqpoint{0.000000in}{0.000000in}}{%
\pgfpathmoveto{\pgfqpoint{0.000000in}{0.000000in}}%
\pgfpathlineto{\pgfqpoint{0.000000in}{-0.048611in}}%
\pgfusepath{stroke,fill}%
}%
\begin{pgfscope}%
\pgfsys@transformshift{1.293040in}{0.548486in}%
\pgfsys@useobject{currentmarker}{}%
\end{pgfscope}%
\end{pgfscope}%
\begin{pgfscope}%
\definecolor{textcolor}{rgb}{0.000000,0.000000,0.000000}%
\pgfsetstrokecolor{textcolor}%
\pgfsetfillcolor{textcolor}%
\pgftext[x=1.293040in,y=0.451264in,,top]{\color{textcolor}\rmfamily\fontsize{11.000000}{13.200000}\selectfont \(\displaystyle {0.05}\)}%
\end{pgfscope}%
\begin{pgfscope}%
\pgfsetbuttcap%
\pgfsetroundjoin%
\definecolor{currentfill}{rgb}{0.000000,0.000000,0.000000}%
\pgfsetfillcolor{currentfill}%
\pgfsetlinewidth{0.803000pt}%
\definecolor{currentstroke}{rgb}{0.000000,0.000000,0.000000}%
\pgfsetstrokecolor{currentstroke}%
\pgfsetdash{}{0pt}%
\pgfsys@defobject{currentmarker}{\pgfqpoint{0.000000in}{-0.048611in}}{\pgfqpoint{0.000000in}{0.000000in}}{%
\pgfpathmoveto{\pgfqpoint{0.000000in}{0.000000in}}%
\pgfpathlineto{\pgfqpoint{0.000000in}{-0.048611in}}%
\pgfusepath{stroke,fill}%
}%
\begin{pgfscope}%
\pgfsys@transformshift{2.300540in}{0.548486in}%
\pgfsys@useobject{currentmarker}{}%
\end{pgfscope}%
\end{pgfscope}%
\begin{pgfscope}%
\definecolor{textcolor}{rgb}{0.000000,0.000000,0.000000}%
\pgfsetstrokecolor{textcolor}%
\pgfsetfillcolor{textcolor}%
\pgftext[x=2.300540in,y=0.451264in,,top]{\color{textcolor}\rmfamily\fontsize{11.000000}{13.200000}\selectfont \(\displaystyle {0.10}\)}%
\end{pgfscope}%
\begin{pgfscope}%
\pgfsetbuttcap%
\pgfsetroundjoin%
\definecolor{currentfill}{rgb}{0.000000,0.000000,0.000000}%
\pgfsetfillcolor{currentfill}%
\pgfsetlinewidth{0.803000pt}%
\definecolor{currentstroke}{rgb}{0.000000,0.000000,0.000000}%
\pgfsetstrokecolor{currentstroke}%
\pgfsetdash{}{0pt}%
\pgfsys@defobject{currentmarker}{\pgfqpoint{0.000000in}{-0.048611in}}{\pgfqpoint{0.000000in}{0.000000in}}{%
\pgfpathmoveto{\pgfqpoint{0.000000in}{0.000000in}}%
\pgfpathlineto{\pgfqpoint{0.000000in}{-0.048611in}}%
\pgfusepath{stroke,fill}%
}%
\begin{pgfscope}%
\pgfsys@transformshift{3.308040in}{0.548486in}%
\pgfsys@useobject{currentmarker}{}%
\end{pgfscope}%
\end{pgfscope}%
\begin{pgfscope}%
\definecolor{textcolor}{rgb}{0.000000,0.000000,0.000000}%
\pgfsetstrokecolor{textcolor}%
\pgfsetfillcolor{textcolor}%
\pgftext[x=3.308040in,y=0.451264in,,top]{\color{textcolor}\rmfamily\fontsize{11.000000}{13.200000}\selectfont \(\displaystyle {0.15}\)}%
\end{pgfscope}%
\begin{pgfscope}%
\pgfsetbuttcap%
\pgfsetroundjoin%
\definecolor{currentfill}{rgb}{0.000000,0.000000,0.000000}%
\pgfsetfillcolor{currentfill}%
\pgfsetlinewidth{0.803000pt}%
\definecolor{currentstroke}{rgb}{0.000000,0.000000,0.000000}%
\pgfsetstrokecolor{currentstroke}%
\pgfsetdash{}{0pt}%
\pgfsys@defobject{currentmarker}{\pgfqpoint{0.000000in}{-0.048611in}}{\pgfqpoint{0.000000in}{0.000000in}}{%
\pgfpathmoveto{\pgfqpoint{0.000000in}{0.000000in}}%
\pgfpathlineto{\pgfqpoint{0.000000in}{-0.048611in}}%
\pgfusepath{stroke,fill}%
}%
\begin{pgfscope}%
\pgfsys@transformshift{4.315540in}{0.548486in}%
\pgfsys@useobject{currentmarker}{}%
\end{pgfscope}%
\end{pgfscope}%
\begin{pgfscope}%
\definecolor{textcolor}{rgb}{0.000000,0.000000,0.000000}%
\pgfsetstrokecolor{textcolor}%
\pgfsetfillcolor{textcolor}%
\pgftext[x=4.315540in,y=0.451264in,,top]{\color{textcolor}\rmfamily\fontsize{11.000000}{13.200000}\selectfont \(\displaystyle {0.20}\)}%
\end{pgfscope}%
\begin{pgfscope}%
\pgfsetbuttcap%
\pgfsetroundjoin%
\definecolor{currentfill}{rgb}{0.000000,0.000000,0.000000}%
\pgfsetfillcolor{currentfill}%
\pgfsetlinewidth{0.803000pt}%
\definecolor{currentstroke}{rgb}{0.000000,0.000000,0.000000}%
\pgfsetstrokecolor{currentstroke}%
\pgfsetdash{}{0pt}%
\pgfsys@defobject{currentmarker}{\pgfqpoint{0.000000in}{-0.048611in}}{\pgfqpoint{0.000000in}{0.000000in}}{%
\pgfpathmoveto{\pgfqpoint{0.000000in}{0.000000in}}%
\pgfpathlineto{\pgfqpoint{0.000000in}{-0.048611in}}%
\pgfusepath{stroke,fill}%
}%
\begin{pgfscope}%
\pgfsys@transformshift{5.323040in}{0.548486in}%
\pgfsys@useobject{currentmarker}{}%
\end{pgfscope}%
\end{pgfscope}%
\begin{pgfscope}%
\definecolor{textcolor}{rgb}{0.000000,0.000000,0.000000}%
\pgfsetstrokecolor{textcolor}%
\pgfsetfillcolor{textcolor}%
\pgftext[x=5.323040in,y=0.451264in,,top]{\color{textcolor}\rmfamily\fontsize{11.000000}{13.200000}\selectfont \(\displaystyle {0.25}\)}%
\end{pgfscope}%
\begin{pgfscope}%
\definecolor{textcolor}{rgb}{0.000000,0.000000,0.000000}%
\pgfsetstrokecolor{textcolor}%
\pgfsetfillcolor{textcolor}%
\pgftext[x=3.308040in,y=0.247854in,,top]{\color{textcolor}\rmfamily\fontsize{11.000000}{13.200000}\selectfont \(\displaystyle d\)}%
\end{pgfscope}%
\begin{pgfscope}%
\pgfsetbuttcap%
\pgfsetroundjoin%
\definecolor{currentfill}{rgb}{0.000000,0.000000,0.000000}%
\pgfsetfillcolor{currentfill}%
\pgfsetlinewidth{0.803000pt}%
\definecolor{currentstroke}{rgb}{0.000000,0.000000,0.000000}%
\pgfsetstrokecolor{currentstroke}%
\pgfsetdash{}{0pt}%
\pgfsys@defobject{currentmarker}{\pgfqpoint{-0.048611in}{0.000000in}}{\pgfqpoint{-0.000000in}{0.000000in}}{%
\pgfpathmoveto{\pgfqpoint{-0.000000in}{0.000000in}}%
\pgfpathlineto{\pgfqpoint{-0.048611in}{0.000000in}}%
\pgfusepath{stroke,fill}%
}%
\begin{pgfscope}%
\pgfsys@transformshift{0.789290in}{0.621986in}%
\pgfsys@useobject{currentmarker}{}%
\end{pgfscope}%
\end{pgfscope}%
\begin{pgfscope}%
\definecolor{textcolor}{rgb}{0.000000,0.000000,0.000000}%
\pgfsetstrokecolor{textcolor}%
\pgfsetfillcolor{textcolor}%
\pgftext[x=0.303410in, y=0.563948in, left, base]{\color{textcolor}\rmfamily\fontsize{11.000000}{13.200000}\selectfont \(\displaystyle {\ensuremath{-}0.14}\)}%
\end{pgfscope}%
\begin{pgfscope}%
\pgfsetbuttcap%
\pgfsetroundjoin%
\definecolor{currentfill}{rgb}{0.000000,0.000000,0.000000}%
\pgfsetfillcolor{currentfill}%
\pgfsetlinewidth{0.803000pt}%
\definecolor{currentstroke}{rgb}{0.000000,0.000000,0.000000}%
\pgfsetstrokecolor{currentstroke}%
\pgfsetdash{}{0pt}%
\pgfsys@defobject{currentmarker}{\pgfqpoint{-0.048611in}{0.000000in}}{\pgfqpoint{-0.000000in}{0.000000in}}{%
\pgfpathmoveto{\pgfqpoint{-0.000000in}{0.000000in}}%
\pgfpathlineto{\pgfqpoint{-0.048611in}{0.000000in}}%
\pgfusepath{stroke,fill}%
}%
\begin{pgfscope}%
\pgfsys@transformshift{0.789290in}{1.013986in}%
\pgfsys@useobject{currentmarker}{}%
\end{pgfscope}%
\end{pgfscope}%
\begin{pgfscope}%
\definecolor{textcolor}{rgb}{0.000000,0.000000,0.000000}%
\pgfsetstrokecolor{textcolor}%
\pgfsetfillcolor{textcolor}%
\pgftext[x=0.303410in, y=0.955948in, left, base]{\color{textcolor}\rmfamily\fontsize{11.000000}{13.200000}\selectfont \(\displaystyle {\ensuremath{-}0.12}\)}%
\end{pgfscope}%
\begin{pgfscope}%
\pgfsetbuttcap%
\pgfsetroundjoin%
\definecolor{currentfill}{rgb}{0.000000,0.000000,0.000000}%
\pgfsetfillcolor{currentfill}%
\pgfsetlinewidth{0.803000pt}%
\definecolor{currentstroke}{rgb}{0.000000,0.000000,0.000000}%
\pgfsetstrokecolor{currentstroke}%
\pgfsetdash{}{0pt}%
\pgfsys@defobject{currentmarker}{\pgfqpoint{-0.048611in}{0.000000in}}{\pgfqpoint{-0.000000in}{0.000000in}}{%
\pgfpathmoveto{\pgfqpoint{-0.000000in}{0.000000in}}%
\pgfpathlineto{\pgfqpoint{-0.048611in}{0.000000in}}%
\pgfusepath{stroke,fill}%
}%
\begin{pgfscope}%
\pgfsys@transformshift{0.789290in}{1.405986in}%
\pgfsys@useobject{currentmarker}{}%
\end{pgfscope}%
\end{pgfscope}%
\begin{pgfscope}%
\definecolor{textcolor}{rgb}{0.000000,0.000000,0.000000}%
\pgfsetstrokecolor{textcolor}%
\pgfsetfillcolor{textcolor}%
\pgftext[x=0.303410in, y=1.347948in, left, base]{\color{textcolor}\rmfamily\fontsize{11.000000}{13.200000}\selectfont \(\displaystyle {\ensuremath{-}0.10}\)}%
\end{pgfscope}%
\begin{pgfscope}%
\pgfsetbuttcap%
\pgfsetroundjoin%
\definecolor{currentfill}{rgb}{0.000000,0.000000,0.000000}%
\pgfsetfillcolor{currentfill}%
\pgfsetlinewidth{0.803000pt}%
\definecolor{currentstroke}{rgb}{0.000000,0.000000,0.000000}%
\pgfsetstrokecolor{currentstroke}%
\pgfsetdash{}{0pt}%
\pgfsys@defobject{currentmarker}{\pgfqpoint{-0.048611in}{0.000000in}}{\pgfqpoint{-0.000000in}{0.000000in}}{%
\pgfpathmoveto{\pgfqpoint{-0.000000in}{0.000000in}}%
\pgfpathlineto{\pgfqpoint{-0.048611in}{0.000000in}}%
\pgfusepath{stroke,fill}%
}%
\begin{pgfscope}%
\pgfsys@transformshift{0.789290in}{1.797986in}%
\pgfsys@useobject{currentmarker}{}%
\end{pgfscope}%
\end{pgfscope}%
\begin{pgfscope}%
\definecolor{textcolor}{rgb}{0.000000,0.000000,0.000000}%
\pgfsetstrokecolor{textcolor}%
\pgfsetfillcolor{textcolor}%
\pgftext[x=0.303410in, y=1.739948in, left, base]{\color{textcolor}\rmfamily\fontsize{11.000000}{13.200000}\selectfont \(\displaystyle {\ensuremath{-}0.08}\)}%
\end{pgfscope}%
\begin{pgfscope}%
\pgfsetbuttcap%
\pgfsetroundjoin%
\definecolor{currentfill}{rgb}{0.000000,0.000000,0.000000}%
\pgfsetfillcolor{currentfill}%
\pgfsetlinewidth{0.803000pt}%
\definecolor{currentstroke}{rgb}{0.000000,0.000000,0.000000}%
\pgfsetstrokecolor{currentstroke}%
\pgfsetdash{}{0pt}%
\pgfsys@defobject{currentmarker}{\pgfqpoint{-0.048611in}{0.000000in}}{\pgfqpoint{-0.000000in}{0.000000in}}{%
\pgfpathmoveto{\pgfqpoint{-0.000000in}{0.000000in}}%
\pgfpathlineto{\pgfqpoint{-0.048611in}{0.000000in}}%
\pgfusepath{stroke,fill}%
}%
\begin{pgfscope}%
\pgfsys@transformshift{0.789290in}{2.189986in}%
\pgfsys@useobject{currentmarker}{}%
\end{pgfscope}%
\end{pgfscope}%
\begin{pgfscope}%
\definecolor{textcolor}{rgb}{0.000000,0.000000,0.000000}%
\pgfsetstrokecolor{textcolor}%
\pgfsetfillcolor{textcolor}%
\pgftext[x=0.303410in, y=2.131948in, left, base]{\color{textcolor}\rmfamily\fontsize{11.000000}{13.200000}\selectfont \(\displaystyle {\ensuremath{-}0.06}\)}%
\end{pgfscope}%
\begin{pgfscope}%
\pgfsetbuttcap%
\pgfsetroundjoin%
\definecolor{currentfill}{rgb}{0.000000,0.000000,0.000000}%
\pgfsetfillcolor{currentfill}%
\pgfsetlinewidth{0.803000pt}%
\definecolor{currentstroke}{rgb}{0.000000,0.000000,0.000000}%
\pgfsetstrokecolor{currentstroke}%
\pgfsetdash{}{0pt}%
\pgfsys@defobject{currentmarker}{\pgfqpoint{-0.048611in}{0.000000in}}{\pgfqpoint{-0.000000in}{0.000000in}}{%
\pgfpathmoveto{\pgfqpoint{-0.000000in}{0.000000in}}%
\pgfpathlineto{\pgfqpoint{-0.048611in}{0.000000in}}%
\pgfusepath{stroke,fill}%
}%
\begin{pgfscope}%
\pgfsys@transformshift{0.789290in}{2.581986in}%
\pgfsys@useobject{currentmarker}{}%
\end{pgfscope}%
\end{pgfscope}%
\begin{pgfscope}%
\definecolor{textcolor}{rgb}{0.000000,0.000000,0.000000}%
\pgfsetstrokecolor{textcolor}%
\pgfsetfillcolor{textcolor}%
\pgftext[x=0.303410in, y=2.523948in, left, base]{\color{textcolor}\rmfamily\fontsize{11.000000}{13.200000}\selectfont \(\displaystyle {\ensuremath{-}0.04}\)}%
\end{pgfscope}%
\begin{pgfscope}%
\pgfsetbuttcap%
\pgfsetroundjoin%
\definecolor{currentfill}{rgb}{0.000000,0.000000,0.000000}%
\pgfsetfillcolor{currentfill}%
\pgfsetlinewidth{0.803000pt}%
\definecolor{currentstroke}{rgb}{0.000000,0.000000,0.000000}%
\pgfsetstrokecolor{currentstroke}%
\pgfsetdash{}{0pt}%
\pgfsys@defobject{currentmarker}{\pgfqpoint{-0.048611in}{0.000000in}}{\pgfqpoint{-0.000000in}{0.000000in}}{%
\pgfpathmoveto{\pgfqpoint{-0.000000in}{0.000000in}}%
\pgfpathlineto{\pgfqpoint{-0.048611in}{0.000000in}}%
\pgfusepath{stroke,fill}%
}%
\begin{pgfscope}%
\pgfsys@transformshift{0.789290in}{2.973986in}%
\pgfsys@useobject{currentmarker}{}%
\end{pgfscope}%
\end{pgfscope}%
\begin{pgfscope}%
\definecolor{textcolor}{rgb}{0.000000,0.000000,0.000000}%
\pgfsetstrokecolor{textcolor}%
\pgfsetfillcolor{textcolor}%
\pgftext[x=0.303410in, y=2.915948in, left, base]{\color{textcolor}\rmfamily\fontsize{11.000000}{13.200000}\selectfont \(\displaystyle {\ensuremath{-}0.02}\)}%
\end{pgfscope}%
\begin{pgfscope}%
\definecolor{textcolor}{rgb}{0.000000,0.000000,0.000000}%
\pgfsetstrokecolor{textcolor}%
\pgfsetfillcolor{textcolor}%
\pgftext[x=0.247854in,y=1.895986in,,bottom,rotate=90.000000]{\color{textcolor}\rmfamily\fontsize{11.000000}{13.200000}\selectfont \(\displaystyle u_z\)}%
\end{pgfscope}%
\begin{pgfscope}%
\pgfpathrectangle{\pgfqpoint{0.789290in}{0.548486in}}{\pgfqpoint{5.037500in}{2.695000in}}%
\pgfusepath{clip}%
\pgfsetrectcap%
\pgfsetroundjoin%
\pgfsetlinewidth{1.505625pt}%
\definecolor{currentstroke}{rgb}{0.054902,0.262745,0.486275}%
\pgfsetstrokecolor{currentstroke}%
\pgfsetstrokeopacity{0.250000}%
\pgfsetdash{}{0pt}%
\pgfpathmoveto{\pgfqpoint{0.789290in}{3.120986in}}%
\pgfpathlineto{\pgfqpoint{0.840174in}{3.096238in}}%
\pgfpathlineto{\pgfqpoint{0.891058in}{3.071491in}}%
\pgfpathlineto{\pgfqpoint{0.941941in}{3.046743in}}%
\pgfpathlineto{\pgfqpoint{0.992825in}{3.021996in}}%
\pgfpathlineto{\pgfqpoint{1.043709in}{2.997248in}}%
\pgfpathlineto{\pgfqpoint{1.094593in}{2.972501in}}%
\pgfpathlineto{\pgfqpoint{1.145477in}{2.947754in}}%
\pgfpathlineto{\pgfqpoint{1.196361in}{2.923006in}}%
\pgfpathlineto{\pgfqpoint{1.247244in}{2.898259in}}%
\pgfpathlineto{\pgfqpoint{1.298128in}{2.873511in}}%
\pgfpathlineto{\pgfqpoint{1.349012in}{2.848764in}}%
\pgfpathlineto{\pgfqpoint{1.399896in}{2.824016in}}%
\pgfpathlineto{\pgfqpoint{1.450780in}{2.799269in}}%
\pgfpathlineto{\pgfqpoint{1.501664in}{2.774521in}}%
\pgfpathlineto{\pgfqpoint{1.552547in}{2.749774in}}%
\pgfpathlineto{\pgfqpoint{1.603431in}{2.725026in}}%
\pgfpathlineto{\pgfqpoint{1.654315in}{2.700279in}}%
\pgfpathlineto{\pgfqpoint{1.705199in}{2.675531in}}%
\pgfpathlineto{\pgfqpoint{1.756083in}{2.650784in}}%
\pgfpathlineto{\pgfqpoint{1.806967in}{2.626036in}}%
\pgfpathlineto{\pgfqpoint{1.857850in}{2.601289in}}%
\pgfpathlineto{\pgfqpoint{1.908734in}{2.576541in}}%
\pgfpathlineto{\pgfqpoint{1.959618in}{2.551794in}}%
\pgfpathlineto{\pgfqpoint{2.010502in}{2.527046in}}%
\pgfpathlineto{\pgfqpoint{2.061386in}{2.502299in}}%
\pgfpathlineto{\pgfqpoint{2.112270in}{2.477551in}}%
\pgfpathlineto{\pgfqpoint{2.163153in}{2.452804in}}%
\pgfpathlineto{\pgfqpoint{2.214037in}{2.428057in}}%
\pgfpathlineto{\pgfqpoint{2.264921in}{2.403309in}}%
\pgfpathlineto{\pgfqpoint{2.315805in}{2.378562in}}%
\pgfpathlineto{\pgfqpoint{2.366689in}{2.353814in}}%
\pgfpathlineto{\pgfqpoint{2.417573in}{2.329067in}}%
\pgfpathlineto{\pgfqpoint{2.468457in}{2.304319in}}%
\pgfpathlineto{\pgfqpoint{2.519340in}{2.279572in}}%
\pgfpathlineto{\pgfqpoint{2.570224in}{2.254824in}}%
\pgfpathlineto{\pgfqpoint{2.621108in}{2.230077in}}%
\pgfpathlineto{\pgfqpoint{2.671992in}{2.205329in}}%
\pgfpathlineto{\pgfqpoint{2.722876in}{2.180582in}}%
\pgfpathlineto{\pgfqpoint{2.773760in}{2.155834in}}%
\pgfpathlineto{\pgfqpoint{2.824643in}{2.131087in}}%
\pgfpathlineto{\pgfqpoint{2.875527in}{2.106339in}}%
\pgfpathlineto{\pgfqpoint{2.926411in}{2.081592in}}%
\pgfpathlineto{\pgfqpoint{2.977295in}{2.056844in}}%
\pgfpathlineto{\pgfqpoint{3.028179in}{2.032097in}}%
\pgfpathlineto{\pgfqpoint{3.079063in}{2.007349in}}%
\pgfpathlineto{\pgfqpoint{3.129946in}{1.982602in}}%
\pgfpathlineto{\pgfqpoint{3.180830in}{1.957855in}}%
\pgfpathlineto{\pgfqpoint{3.231714in}{1.933107in}}%
\pgfpathlineto{\pgfqpoint{3.282598in}{1.908360in}}%
\pgfpathlineto{\pgfqpoint{3.333482in}{1.883612in}}%
\pgfpathlineto{\pgfqpoint{3.384366in}{1.858865in}}%
\pgfpathlineto{\pgfqpoint{3.435249in}{1.834117in}}%
\pgfpathlineto{\pgfqpoint{3.486133in}{1.809370in}}%
\pgfpathlineto{\pgfqpoint{3.537017in}{1.784622in}}%
\pgfpathlineto{\pgfqpoint{3.587901in}{1.759875in}}%
\pgfpathlineto{\pgfqpoint{3.638785in}{1.735127in}}%
\pgfpathlineto{\pgfqpoint{3.689669in}{1.710380in}}%
\pgfpathlineto{\pgfqpoint{3.740552in}{1.685632in}}%
\pgfpathlineto{\pgfqpoint{3.791436in}{1.660885in}}%
\pgfpathlineto{\pgfqpoint{3.842320in}{1.636137in}}%
\pgfpathlineto{\pgfqpoint{3.893204in}{1.611390in}}%
\pgfpathlineto{\pgfqpoint{3.944088in}{1.586642in}}%
\pgfpathlineto{\pgfqpoint{3.994972in}{1.561895in}}%
\pgfpathlineto{\pgfqpoint{4.045856in}{1.537147in}}%
\pgfpathlineto{\pgfqpoint{4.096739in}{1.512400in}}%
\pgfpathlineto{\pgfqpoint{4.147623in}{1.487653in}}%
\pgfpathlineto{\pgfqpoint{4.198507in}{1.462905in}}%
\pgfpathlineto{\pgfqpoint{4.249391in}{1.438158in}}%
\pgfpathlineto{\pgfqpoint{4.300275in}{1.413410in}}%
\pgfpathlineto{\pgfqpoint{4.351159in}{1.388663in}}%
\pgfpathlineto{\pgfqpoint{4.402042in}{1.363915in}}%
\pgfpathlineto{\pgfqpoint{4.452926in}{1.339168in}}%
\pgfpathlineto{\pgfqpoint{4.503810in}{1.314420in}}%
\pgfpathlineto{\pgfqpoint{4.554694in}{1.289673in}}%
\pgfpathlineto{\pgfqpoint{4.605578in}{1.264925in}}%
\pgfpathlineto{\pgfqpoint{4.656462in}{1.240178in}}%
\pgfpathlineto{\pgfqpoint{4.707345in}{1.215430in}}%
\pgfpathlineto{\pgfqpoint{4.758229in}{1.190683in}}%
\pgfpathlineto{\pgfqpoint{4.809113in}{1.165935in}}%
\pgfpathlineto{\pgfqpoint{4.859997in}{1.141188in}}%
\pgfpathlineto{\pgfqpoint{4.910881in}{1.116440in}}%
\pgfpathlineto{\pgfqpoint{4.961765in}{1.091693in}}%
\pgfpathlineto{\pgfqpoint{5.012648in}{1.066945in}}%
\pgfpathlineto{\pgfqpoint{5.063532in}{1.042198in}}%
\pgfpathlineto{\pgfqpoint{5.114416in}{1.017450in}}%
\pgfpathlineto{\pgfqpoint{5.165300in}{0.992703in}}%
\pgfpathlineto{\pgfqpoint{5.216184in}{0.967956in}}%
\pgfpathlineto{\pgfqpoint{5.267068in}{0.943208in}}%
\pgfpathlineto{\pgfqpoint{5.317951in}{0.918461in}}%
\pgfpathlineto{\pgfqpoint{5.368835in}{0.893713in}}%
\pgfpathlineto{\pgfqpoint{5.419719in}{0.868966in}}%
\pgfpathlineto{\pgfqpoint{5.470603in}{0.844218in}}%
\pgfpathlineto{\pgfqpoint{5.521487in}{0.819471in}}%
\pgfpathlineto{\pgfqpoint{5.572371in}{0.794723in}}%
\pgfpathlineto{\pgfqpoint{5.623255in}{0.769976in}}%
\pgfpathlineto{\pgfqpoint{5.674138in}{0.745228in}}%
\pgfpathlineto{\pgfqpoint{5.725022in}{0.720481in}}%
\pgfpathlineto{\pgfqpoint{5.775906in}{0.695733in}}%
\pgfpathlineto{\pgfqpoint{5.826790in}{0.670986in}}%
\pgfusepath{stroke}%
\end{pgfscope}%
\begin{pgfscope}%
\pgfsetrectcap%
\pgfsetmiterjoin%
\pgfsetlinewidth{0.803000pt}%
\definecolor{currentstroke}{rgb}{0.000000,0.000000,0.000000}%
\pgfsetstrokecolor{currentstroke}%
\pgfsetdash{}{0pt}%
\pgfpathmoveto{\pgfqpoint{0.789290in}{0.548486in}}%
\pgfpathlineto{\pgfqpoint{0.789290in}{3.243486in}}%
\pgfusepath{stroke}%
\end{pgfscope}%
\begin{pgfscope}%
\pgfsetrectcap%
\pgfsetmiterjoin%
\pgfsetlinewidth{0.803000pt}%
\definecolor{currentstroke}{rgb}{0.000000,0.000000,0.000000}%
\pgfsetstrokecolor{currentstroke}%
\pgfsetdash{}{0pt}%
\pgfpathmoveto{\pgfqpoint{5.826790in}{0.548486in}}%
\pgfpathlineto{\pgfqpoint{5.826790in}{3.243486in}}%
\pgfusepath{stroke}%
\end{pgfscope}%
\begin{pgfscope}%
\pgfsetrectcap%
\pgfsetmiterjoin%
\pgfsetlinewidth{0.803000pt}%
\definecolor{currentstroke}{rgb}{0.000000,0.000000,0.000000}%
\pgfsetstrokecolor{currentstroke}%
\pgfsetdash{}{0pt}%
\pgfpathmoveto{\pgfqpoint{0.789290in}{0.548486in}}%
\pgfpathlineto{\pgfqpoint{5.826790in}{0.548486in}}%
\pgfusepath{stroke}%
\end{pgfscope}%
\begin{pgfscope}%
\pgfsetrectcap%
\pgfsetmiterjoin%
\pgfsetlinewidth{0.803000pt}%
\definecolor{currentstroke}{rgb}{0.000000,0.000000,0.000000}%
\pgfsetstrokecolor{currentstroke}%
\pgfsetdash{}{0pt}%
\pgfpathmoveto{\pgfqpoint{0.789290in}{3.243486in}}%
\pgfpathlineto{\pgfqpoint{5.826790in}{3.243486in}}%
\pgfusepath{stroke}%
\end{pgfscope}%
\end{pgfpicture}%
\makeatother%
\endgroup%

    \caption{Penetration depths $u_z$ for multiple vertical displacements $d$ scattered against the exact values obtained with \refequ{equ:normal-displacement}. The quantities are normalized by the mesh size $h$ at the interface.}
    \label{fig:normal-displacements-3d}
\end{subfigure}
\begin{subfigure}[t]{.49\textwidth}
     %% Creator: Matplotlib, PGF backend
%%
%% To include the figure in your LaTeX document, write
%%   \input{<filename>.pgf}
%%
%% Make sure the required packages are loaded in your preamble
%%   \usepackage{pgf}
%%
%% Also ensure that all the required font packages are loaded; for instance,
%% the lmodern package is sometimes necessary when using math font.
%%   \usepackage{lmodern}
%%
%% Figures using additional raster images can only be included by \input if
%% they are in the same directory as the main LaTeX file. For loading figures
%% from other directories you can use the `import` package
%%   \usepackage{import}
%%
%% and then include the figures with
%%   \import{<path to file>}{<filename>.pgf}
%%
%% Matplotlib used the following preamble
%%   
%%   \usepackage{fontspec}
%%   \setmainfont{DejaVuSans.ttf}[Path=\detokenize{/home/fabio/Internodes-CM/.venv/lib/python3.8/site-packages/matplotlib/mpl-data/fonts/ttf/}]
%%   \setsansfont{DejaVuSans.ttf}[Path=\detokenize{/home/fabio/Internodes-CM/.venv/lib/python3.8/site-packages/matplotlib/mpl-data/fonts/ttf/}]
%%   \setmonofont{DejaVuSansMono.ttf}[Path=\detokenize{/home/fabio/Internodes-CM/.venv/lib/python3.8/site-packages/matplotlib/mpl-data/fonts/ttf/}]
%%   \makeatletter\@ifpackageloaded{underscore}{}{\usepackage[strings]{underscore}}\makeatother
%%
\begingroup%
\makeatletter%
\begin{pgfpicture}%
\pgfpathrectangle{\pgfpointorigin}{\pgfqpoint{5.808503in}{3.343486in}}%
\pgfusepath{use as bounding box, clip}%
\begin{pgfscope}%
\pgfsetbuttcap%
\pgfsetmiterjoin%
\definecolor{currentfill}{rgb}{1.000000,1.000000,1.000000}%
\pgfsetfillcolor{currentfill}%
\pgfsetlinewidth{0.000000pt}%
\definecolor{currentstroke}{rgb}{1.000000,1.000000,1.000000}%
\pgfsetstrokecolor{currentstroke}%
\pgfsetdash{}{0pt}%
\pgfpathmoveto{\pgfqpoint{0.000000in}{0.000000in}}%
\pgfpathlineto{\pgfqpoint{5.808503in}{0.000000in}}%
\pgfpathlineto{\pgfqpoint{5.808503in}{3.343486in}}%
\pgfpathlineto{\pgfqpoint{0.000000in}{3.343486in}}%
\pgfpathlineto{\pgfqpoint{0.000000in}{0.000000in}}%
\pgfpathclose%
\pgfusepath{fill}%
\end{pgfscope}%
\begin{pgfscope}%
\pgfsetbuttcap%
\pgfsetmiterjoin%
\definecolor{currentfill}{rgb}{1.000000,1.000000,1.000000}%
\pgfsetfillcolor{currentfill}%
\pgfsetlinewidth{0.000000pt}%
\definecolor{currentstroke}{rgb}{0.000000,0.000000,0.000000}%
\pgfsetstrokecolor{currentstroke}%
\pgfsetstrokeopacity{0.000000}%
\pgfsetdash{}{0pt}%
\pgfpathmoveto{\pgfqpoint{0.671003in}{0.548486in}}%
\pgfpathlineto{\pgfqpoint{5.708502in}{0.548486in}}%
\pgfpathlineto{\pgfqpoint{5.708502in}{3.243486in}}%
\pgfpathlineto{\pgfqpoint{0.671003in}{3.243486in}}%
\pgfpathlineto{\pgfqpoint{0.671003in}{0.548486in}}%
\pgfpathclose%
\pgfusepath{fill}%
\end{pgfscope}%
\begin{pgfscope}%
\pgfpathrectangle{\pgfqpoint{0.671003in}{0.548486in}}{\pgfqpoint{5.037500in}{2.695000in}}%
\pgfusepath{clip}%
\pgfsetbuttcap%
\pgfsetroundjoin%
\definecolor{currentfill}{rgb}{0.054902,0.262745,0.486275}%
\pgfsetfillcolor{currentfill}%
\pgfsetlinewidth{1.003750pt}%
\definecolor{currentstroke}{rgb}{0.054902,0.262745,0.486275}%
\pgfsetstrokecolor{currentstroke}%
\pgfsetdash{}{0pt}%
\pgfsys@defobject{currentmarker}{\pgfqpoint{-0.041667in}{-0.041667in}}{\pgfqpoint{0.041667in}{0.041667in}}{%
\pgfpathmoveto{\pgfqpoint{0.000000in}{-0.041667in}}%
\pgfpathcurveto{\pgfqpoint{0.011050in}{-0.041667in}}{\pgfqpoint{0.021649in}{-0.037276in}}{\pgfqpoint{0.029463in}{-0.029463in}}%
\pgfpathcurveto{\pgfqpoint{0.037276in}{-0.021649in}}{\pgfqpoint{0.041667in}{-0.011050in}}{\pgfqpoint{0.041667in}{0.000000in}}%
\pgfpathcurveto{\pgfqpoint{0.041667in}{0.011050in}}{\pgfqpoint{0.037276in}{0.021649in}}{\pgfqpoint{0.029463in}{0.029463in}}%
\pgfpathcurveto{\pgfqpoint{0.021649in}{0.037276in}}{\pgfqpoint{0.011050in}{0.041667in}}{\pgfqpoint{0.000000in}{0.041667in}}%
\pgfpathcurveto{\pgfqpoint{-0.011050in}{0.041667in}}{\pgfqpoint{-0.021649in}{0.037276in}}{\pgfqpoint{-0.029463in}{0.029463in}}%
\pgfpathcurveto{\pgfqpoint{-0.037276in}{0.021649in}}{\pgfqpoint{-0.041667in}{0.011050in}}{\pgfqpoint{-0.041667in}{0.000000in}}%
\pgfpathcurveto{\pgfqpoint{-0.041667in}{-0.011050in}}{\pgfqpoint{-0.037276in}{-0.021649in}}{\pgfqpoint{-0.029463in}{-0.029463in}}%
\pgfpathcurveto{\pgfqpoint{-0.021649in}{-0.037276in}}{\pgfqpoint{-0.011050in}{-0.041667in}}{\pgfqpoint{0.000000in}{-0.041667in}}%
\pgfpathlineto{\pgfqpoint{0.000000in}{-0.041667in}}%
\pgfpathclose%
\pgfusepath{stroke,fill}%
}%
\begin{pgfscope}%
\pgfsys@transformshift{1.174753in}{0.940695in}%
\pgfsys@useobject{currentmarker}{}%
\end{pgfscope}%
\begin{pgfscope}%
\pgfsys@transformshift{1.622530in}{1.438501in}%
\pgfsys@useobject{currentmarker}{}%
\end{pgfscope}%
\begin{pgfscope}%
\pgfsys@transformshift{2.070308in}{1.840724in}%
\pgfsys@useobject{currentmarker}{}%
\end{pgfscope}%
\begin{pgfscope}%
\pgfsys@transformshift{2.518086in}{2.019350in}%
\pgfsys@useobject{currentmarker}{}%
\end{pgfscope}%
\begin{pgfscope}%
\pgfsys@transformshift{2.965864in}{2.243552in}%
\pgfsys@useobject{currentmarker}{}%
\end{pgfscope}%
\begin{pgfscope}%
\pgfsys@transformshift{3.413641in}{2.623822in}%
\pgfsys@useobject{currentmarker}{}%
\end{pgfscope}%
\begin{pgfscope}%
\pgfsys@transformshift{3.861419in}{2.718059in}%
\pgfsys@useobject{currentmarker}{}%
\end{pgfscope}%
\begin{pgfscope}%
\pgfsys@transformshift{4.309197in}{2.705364in}%
\pgfsys@useobject{currentmarker}{}%
\end{pgfscope}%
\begin{pgfscope}%
\pgfsys@transformshift{4.756975in}{2.701932in}%
\pgfsys@useobject{currentmarker}{}%
\end{pgfscope}%
\begin{pgfscope}%
\pgfsys@transformshift{5.204752in}{2.696080in}%
\pgfsys@useobject{currentmarker}{}%
\end{pgfscope}%
\end{pgfscope}%
\begin{pgfscope}%
\pgfsetbuttcap%
\pgfsetroundjoin%
\definecolor{currentfill}{rgb}{0.000000,0.000000,0.000000}%
\pgfsetfillcolor{currentfill}%
\pgfsetlinewidth{0.803000pt}%
\definecolor{currentstroke}{rgb}{0.000000,0.000000,0.000000}%
\pgfsetstrokecolor{currentstroke}%
\pgfsetdash{}{0pt}%
\pgfsys@defobject{currentmarker}{\pgfqpoint{0.000000in}{-0.048611in}}{\pgfqpoint{0.000000in}{0.000000in}}{%
\pgfpathmoveto{\pgfqpoint{0.000000in}{0.000000in}}%
\pgfpathlineto{\pgfqpoint{0.000000in}{-0.048611in}}%
\pgfusepath{stroke,fill}%
}%
\begin{pgfscope}%
\pgfsys@transformshift{1.174753in}{0.548486in}%
\pgfsys@useobject{currentmarker}{}%
\end{pgfscope}%
\end{pgfscope}%
\begin{pgfscope}%
\definecolor{textcolor}{rgb}{0.000000,0.000000,0.000000}%
\pgfsetstrokecolor{textcolor}%
\pgfsetfillcolor{textcolor}%
\pgftext[x=1.174753in,y=0.451264in,,top]{\color{textcolor}\rmfamily\fontsize{11.000000}{13.200000}\selectfont \(\displaystyle {0.05}\)}%
\end{pgfscope}%
\begin{pgfscope}%
\pgfsetbuttcap%
\pgfsetroundjoin%
\definecolor{currentfill}{rgb}{0.000000,0.000000,0.000000}%
\pgfsetfillcolor{currentfill}%
\pgfsetlinewidth{0.803000pt}%
\definecolor{currentstroke}{rgb}{0.000000,0.000000,0.000000}%
\pgfsetstrokecolor{currentstroke}%
\pgfsetdash{}{0pt}%
\pgfsys@defobject{currentmarker}{\pgfqpoint{0.000000in}{-0.048611in}}{\pgfqpoint{0.000000in}{0.000000in}}{%
\pgfpathmoveto{\pgfqpoint{0.000000in}{0.000000in}}%
\pgfpathlineto{\pgfqpoint{0.000000in}{-0.048611in}}%
\pgfusepath{stroke,fill}%
}%
\begin{pgfscope}%
\pgfsys@transformshift{2.182253in}{0.548486in}%
\pgfsys@useobject{currentmarker}{}%
\end{pgfscope}%
\end{pgfscope}%
\begin{pgfscope}%
\definecolor{textcolor}{rgb}{0.000000,0.000000,0.000000}%
\pgfsetstrokecolor{textcolor}%
\pgfsetfillcolor{textcolor}%
\pgftext[x=2.182253in,y=0.451264in,,top]{\color{textcolor}\rmfamily\fontsize{11.000000}{13.200000}\selectfont \(\displaystyle {0.10}\)}%
\end{pgfscope}%
\begin{pgfscope}%
\pgfsetbuttcap%
\pgfsetroundjoin%
\definecolor{currentfill}{rgb}{0.000000,0.000000,0.000000}%
\pgfsetfillcolor{currentfill}%
\pgfsetlinewidth{0.803000pt}%
\definecolor{currentstroke}{rgb}{0.000000,0.000000,0.000000}%
\pgfsetstrokecolor{currentstroke}%
\pgfsetdash{}{0pt}%
\pgfsys@defobject{currentmarker}{\pgfqpoint{0.000000in}{-0.048611in}}{\pgfqpoint{0.000000in}{0.000000in}}{%
\pgfpathmoveto{\pgfqpoint{0.000000in}{0.000000in}}%
\pgfpathlineto{\pgfqpoint{0.000000in}{-0.048611in}}%
\pgfusepath{stroke,fill}%
}%
\begin{pgfscope}%
\pgfsys@transformshift{3.189753in}{0.548486in}%
\pgfsys@useobject{currentmarker}{}%
\end{pgfscope}%
\end{pgfscope}%
\begin{pgfscope}%
\definecolor{textcolor}{rgb}{0.000000,0.000000,0.000000}%
\pgfsetstrokecolor{textcolor}%
\pgfsetfillcolor{textcolor}%
\pgftext[x=3.189752in,y=0.451264in,,top]{\color{textcolor}\rmfamily\fontsize{11.000000}{13.200000}\selectfont \(\displaystyle {0.15}\)}%
\end{pgfscope}%
\begin{pgfscope}%
\pgfsetbuttcap%
\pgfsetroundjoin%
\definecolor{currentfill}{rgb}{0.000000,0.000000,0.000000}%
\pgfsetfillcolor{currentfill}%
\pgfsetlinewidth{0.803000pt}%
\definecolor{currentstroke}{rgb}{0.000000,0.000000,0.000000}%
\pgfsetstrokecolor{currentstroke}%
\pgfsetdash{}{0pt}%
\pgfsys@defobject{currentmarker}{\pgfqpoint{0.000000in}{-0.048611in}}{\pgfqpoint{0.000000in}{0.000000in}}{%
\pgfpathmoveto{\pgfqpoint{0.000000in}{0.000000in}}%
\pgfpathlineto{\pgfqpoint{0.000000in}{-0.048611in}}%
\pgfusepath{stroke,fill}%
}%
\begin{pgfscope}%
\pgfsys@transformshift{4.197253in}{0.548486in}%
\pgfsys@useobject{currentmarker}{}%
\end{pgfscope}%
\end{pgfscope}%
\begin{pgfscope}%
\definecolor{textcolor}{rgb}{0.000000,0.000000,0.000000}%
\pgfsetstrokecolor{textcolor}%
\pgfsetfillcolor{textcolor}%
\pgftext[x=4.197253in,y=0.451264in,,top]{\color{textcolor}\rmfamily\fontsize{11.000000}{13.200000}\selectfont \(\displaystyle {0.20}\)}%
\end{pgfscope}%
\begin{pgfscope}%
\pgfsetbuttcap%
\pgfsetroundjoin%
\definecolor{currentfill}{rgb}{0.000000,0.000000,0.000000}%
\pgfsetfillcolor{currentfill}%
\pgfsetlinewidth{0.803000pt}%
\definecolor{currentstroke}{rgb}{0.000000,0.000000,0.000000}%
\pgfsetstrokecolor{currentstroke}%
\pgfsetdash{}{0pt}%
\pgfsys@defobject{currentmarker}{\pgfqpoint{0.000000in}{-0.048611in}}{\pgfqpoint{0.000000in}{0.000000in}}{%
\pgfpathmoveto{\pgfqpoint{0.000000in}{0.000000in}}%
\pgfpathlineto{\pgfqpoint{0.000000in}{-0.048611in}}%
\pgfusepath{stroke,fill}%
}%
\begin{pgfscope}%
\pgfsys@transformshift{5.204752in}{0.548486in}%
\pgfsys@useobject{currentmarker}{}%
\end{pgfscope}%
\end{pgfscope}%
\begin{pgfscope}%
\definecolor{textcolor}{rgb}{0.000000,0.000000,0.000000}%
\pgfsetstrokecolor{textcolor}%
\pgfsetfillcolor{textcolor}%
\pgftext[x=5.204752in,y=0.451264in,,top]{\color{textcolor}\rmfamily\fontsize{11.000000}{13.200000}\selectfont \(\displaystyle {0.25}\)}%
\end{pgfscope}%
\begin{pgfscope}%
\definecolor{textcolor}{rgb}{0.000000,0.000000,0.000000}%
\pgfsetstrokecolor{textcolor}%
\pgfsetfillcolor{textcolor}%
\pgftext[x=3.189752in,y=0.247854in,,top]{\color{textcolor}\rmfamily\fontsize{11.000000}{13.200000}\selectfont \(\displaystyle d\)}%
\end{pgfscope}%
\begin{pgfscope}%
\pgfsetbuttcap%
\pgfsetroundjoin%
\definecolor{currentfill}{rgb}{0.000000,0.000000,0.000000}%
\pgfsetfillcolor{currentfill}%
\pgfsetlinewidth{0.803000pt}%
\definecolor{currentstroke}{rgb}{0.000000,0.000000,0.000000}%
\pgfsetstrokecolor{currentstroke}%
\pgfsetdash{}{0pt}%
\pgfsys@defobject{currentmarker}{\pgfqpoint{-0.048611in}{0.000000in}}{\pgfqpoint{-0.000000in}{0.000000in}}{%
\pgfpathmoveto{\pgfqpoint{-0.000000in}{0.000000in}}%
\pgfpathlineto{\pgfqpoint{-0.048611in}{0.000000in}}%
\pgfusepath{stroke,fill}%
}%
\begin{pgfscope}%
\pgfsys@transformshift{0.671003in}{0.559335in}%
\pgfsys@useobject{currentmarker}{}%
\end{pgfscope}%
\end{pgfscope}%
\begin{pgfscope}%
\definecolor{textcolor}{rgb}{0.000000,0.000000,0.000000}%
\pgfsetstrokecolor{textcolor}%
\pgfsetfillcolor{textcolor}%
\pgftext[x=0.303410in, y=0.501297in, left, base]{\color{textcolor}\rmfamily\fontsize{11.000000}{13.200000}\selectfont \(\displaystyle {0.10}\)}%
\end{pgfscope}%
\begin{pgfscope}%
\pgfsetbuttcap%
\pgfsetroundjoin%
\definecolor{currentfill}{rgb}{0.000000,0.000000,0.000000}%
\pgfsetfillcolor{currentfill}%
\pgfsetlinewidth{0.803000pt}%
\definecolor{currentstroke}{rgb}{0.000000,0.000000,0.000000}%
\pgfsetstrokecolor{currentstroke}%
\pgfsetdash{}{0pt}%
\pgfsys@defobject{currentmarker}{\pgfqpoint{-0.048611in}{0.000000in}}{\pgfqpoint{-0.000000in}{0.000000in}}{%
\pgfpathmoveto{\pgfqpoint{-0.000000in}{0.000000in}}%
\pgfpathlineto{\pgfqpoint{-0.048611in}{0.000000in}}%
\pgfusepath{stroke,fill}%
}%
\begin{pgfscope}%
\pgfsys@transformshift{0.671003in}{1.032296in}%
\pgfsys@useobject{currentmarker}{}%
\end{pgfscope}%
\end{pgfscope}%
\begin{pgfscope}%
\definecolor{textcolor}{rgb}{0.000000,0.000000,0.000000}%
\pgfsetstrokecolor{textcolor}%
\pgfsetfillcolor{textcolor}%
\pgftext[x=0.303410in, y=0.974258in, left, base]{\color{textcolor}\rmfamily\fontsize{11.000000}{13.200000}\selectfont \(\displaystyle {0.15}\)}%
\end{pgfscope}%
\begin{pgfscope}%
\pgfsetbuttcap%
\pgfsetroundjoin%
\definecolor{currentfill}{rgb}{0.000000,0.000000,0.000000}%
\pgfsetfillcolor{currentfill}%
\pgfsetlinewidth{0.803000pt}%
\definecolor{currentstroke}{rgb}{0.000000,0.000000,0.000000}%
\pgfsetstrokecolor{currentstroke}%
\pgfsetdash{}{0pt}%
\pgfsys@defobject{currentmarker}{\pgfqpoint{-0.048611in}{0.000000in}}{\pgfqpoint{-0.000000in}{0.000000in}}{%
\pgfpathmoveto{\pgfqpoint{-0.000000in}{0.000000in}}%
\pgfpathlineto{\pgfqpoint{-0.048611in}{0.000000in}}%
\pgfusepath{stroke,fill}%
}%
\begin{pgfscope}%
\pgfsys@transformshift{0.671003in}{1.505257in}%
\pgfsys@useobject{currentmarker}{}%
\end{pgfscope}%
\end{pgfscope}%
\begin{pgfscope}%
\definecolor{textcolor}{rgb}{0.000000,0.000000,0.000000}%
\pgfsetstrokecolor{textcolor}%
\pgfsetfillcolor{textcolor}%
\pgftext[x=0.303410in, y=1.447219in, left, base]{\color{textcolor}\rmfamily\fontsize{11.000000}{13.200000}\selectfont \(\displaystyle {0.20}\)}%
\end{pgfscope}%
\begin{pgfscope}%
\pgfsetbuttcap%
\pgfsetroundjoin%
\definecolor{currentfill}{rgb}{0.000000,0.000000,0.000000}%
\pgfsetfillcolor{currentfill}%
\pgfsetlinewidth{0.803000pt}%
\definecolor{currentstroke}{rgb}{0.000000,0.000000,0.000000}%
\pgfsetstrokecolor{currentstroke}%
\pgfsetdash{}{0pt}%
\pgfsys@defobject{currentmarker}{\pgfqpoint{-0.048611in}{0.000000in}}{\pgfqpoint{-0.000000in}{0.000000in}}{%
\pgfpathmoveto{\pgfqpoint{-0.000000in}{0.000000in}}%
\pgfpathlineto{\pgfqpoint{-0.048611in}{0.000000in}}%
\pgfusepath{stroke,fill}%
}%
\begin{pgfscope}%
\pgfsys@transformshift{0.671003in}{1.978218in}%
\pgfsys@useobject{currentmarker}{}%
\end{pgfscope}%
\end{pgfscope}%
\begin{pgfscope}%
\definecolor{textcolor}{rgb}{0.000000,0.000000,0.000000}%
\pgfsetstrokecolor{textcolor}%
\pgfsetfillcolor{textcolor}%
\pgftext[x=0.303410in, y=1.920180in, left, base]{\color{textcolor}\rmfamily\fontsize{11.000000}{13.200000}\selectfont \(\displaystyle {0.25}\)}%
\end{pgfscope}%
\begin{pgfscope}%
\pgfsetbuttcap%
\pgfsetroundjoin%
\definecolor{currentfill}{rgb}{0.000000,0.000000,0.000000}%
\pgfsetfillcolor{currentfill}%
\pgfsetlinewidth{0.803000pt}%
\definecolor{currentstroke}{rgb}{0.000000,0.000000,0.000000}%
\pgfsetstrokecolor{currentstroke}%
\pgfsetdash{}{0pt}%
\pgfsys@defobject{currentmarker}{\pgfqpoint{-0.048611in}{0.000000in}}{\pgfqpoint{-0.000000in}{0.000000in}}{%
\pgfpathmoveto{\pgfqpoint{-0.000000in}{0.000000in}}%
\pgfpathlineto{\pgfqpoint{-0.048611in}{0.000000in}}%
\pgfusepath{stroke,fill}%
}%
\begin{pgfscope}%
\pgfsys@transformshift{0.671003in}{2.451179in}%
\pgfsys@useobject{currentmarker}{}%
\end{pgfscope}%
\end{pgfscope}%
\begin{pgfscope}%
\definecolor{textcolor}{rgb}{0.000000,0.000000,0.000000}%
\pgfsetstrokecolor{textcolor}%
\pgfsetfillcolor{textcolor}%
\pgftext[x=0.303410in, y=2.393141in, left, base]{\color{textcolor}\rmfamily\fontsize{11.000000}{13.200000}\selectfont \(\displaystyle {0.30}\)}%
\end{pgfscope}%
\begin{pgfscope}%
\pgfsetbuttcap%
\pgfsetroundjoin%
\definecolor{currentfill}{rgb}{0.000000,0.000000,0.000000}%
\pgfsetfillcolor{currentfill}%
\pgfsetlinewidth{0.803000pt}%
\definecolor{currentstroke}{rgb}{0.000000,0.000000,0.000000}%
\pgfsetstrokecolor{currentstroke}%
\pgfsetdash{}{0pt}%
\pgfsys@defobject{currentmarker}{\pgfqpoint{-0.048611in}{0.000000in}}{\pgfqpoint{-0.000000in}{0.000000in}}{%
\pgfpathmoveto{\pgfqpoint{-0.000000in}{0.000000in}}%
\pgfpathlineto{\pgfqpoint{-0.048611in}{0.000000in}}%
\pgfusepath{stroke,fill}%
}%
\begin{pgfscope}%
\pgfsys@transformshift{0.671003in}{2.924140in}%
\pgfsys@useobject{currentmarker}{}%
\end{pgfscope}%
\end{pgfscope}%
\begin{pgfscope}%
\definecolor{textcolor}{rgb}{0.000000,0.000000,0.000000}%
\pgfsetstrokecolor{textcolor}%
\pgfsetfillcolor{textcolor}%
\pgftext[x=0.303410in, y=2.866103in, left, base]{\color{textcolor}\rmfamily\fontsize{11.000000}{13.200000}\selectfont \(\displaystyle {0.35}\)}%
\end{pgfscope}%
\begin{pgfscope}%
\definecolor{textcolor}{rgb}{0.000000,0.000000,0.000000}%
\pgfsetstrokecolor{textcolor}%
\pgfsetfillcolor{textcolor}%
\pgftext[x=0.247854in,y=1.895986in,,bottom,rotate=90.000000]{\color{textcolor}\rmfamily\fontsize{11.000000}{13.200000}\selectfont \(\displaystyle a\)}%
\end{pgfscope}%
\begin{pgfscope}%
\pgfpathrectangle{\pgfqpoint{0.671003in}{0.548486in}}{\pgfqpoint{5.037500in}{2.695000in}}%
\pgfusepath{clip}%
\pgfsetrectcap%
\pgfsetroundjoin%
\pgfsetlinewidth{1.505625pt}%
\definecolor{currentstroke}{rgb}{0.054902,0.262745,0.486275}%
\pgfsetstrokecolor{currentstroke}%
\pgfsetstrokeopacity{0.250000}%
\pgfsetdash{}{0pt}%
\pgfpathmoveto{\pgfqpoint{0.671003in}{0.670986in}}%
\pgfpathlineto{\pgfqpoint{0.721886in}{0.723114in}}%
\pgfpathlineto{\pgfqpoint{0.772770in}{0.772901in}}%
\pgfpathlineto{\pgfqpoint{0.823654in}{0.820636in}}%
\pgfpathlineto{\pgfqpoint{0.874538in}{0.866554in}}%
\pgfpathlineto{\pgfqpoint{0.925422in}{0.910849in}}%
\pgfpathlineto{\pgfqpoint{0.976306in}{0.953680in}}%
\pgfpathlineto{\pgfqpoint{1.027189in}{0.995184in}}%
\pgfpathlineto{\pgfqpoint{1.078073in}{1.035477in}}%
\pgfpathlineto{\pgfqpoint{1.128957in}{1.074660in}}%
\pgfpathlineto{\pgfqpoint{1.179841in}{1.112819in}}%
\pgfpathlineto{\pgfqpoint{1.230725in}{1.150031in}}%
\pgfpathlineto{\pgfqpoint{1.281609in}{1.186362in}}%
\pgfpathlineto{\pgfqpoint{1.332492in}{1.221874in}}%
\pgfpathlineto{\pgfqpoint{1.383376in}{1.256618in}}%
\pgfpathlineto{\pgfqpoint{1.434260in}{1.290642in}}%
\pgfpathlineto{\pgfqpoint{1.485144in}{1.323990in}}%
\pgfpathlineto{\pgfqpoint{1.536028in}{1.356700in}}%
\pgfpathlineto{\pgfqpoint{1.586912in}{1.388807in}}%
\pgfpathlineto{\pgfqpoint{1.637795in}{1.420344in}}%
\pgfpathlineto{\pgfqpoint{1.688679in}{1.451340in}}%
\pgfpathlineto{\pgfqpoint{1.739563in}{1.481822in}}%
\pgfpathlineto{\pgfqpoint{1.790447in}{1.511815in}}%
\pgfpathlineto{\pgfqpoint{1.841331in}{1.541340in}}%
\pgfpathlineto{\pgfqpoint{1.892215in}{1.570421in}}%
\pgfpathlineto{\pgfqpoint{1.943098in}{1.599076in}}%
\pgfpathlineto{\pgfqpoint{1.993982in}{1.627323in}}%
\pgfpathlineto{\pgfqpoint{2.044866in}{1.655179in}}%
\pgfpathlineto{\pgfqpoint{2.095750in}{1.682660in}}%
\pgfpathlineto{\pgfqpoint{2.146634in}{1.709781in}}%
\pgfpathlineto{\pgfqpoint{2.197518in}{1.736556in}}%
\pgfpathlineto{\pgfqpoint{2.248401in}{1.762997in}}%
\pgfpathlineto{\pgfqpoint{2.299285in}{1.789117in}}%
\pgfpathlineto{\pgfqpoint{2.350169in}{1.814927in}}%
\pgfpathlineto{\pgfqpoint{2.401053in}{1.840438in}}%
\pgfpathlineto{\pgfqpoint{2.451937in}{1.865659in}}%
\pgfpathlineto{\pgfqpoint{2.502821in}{1.890602in}}%
\pgfpathlineto{\pgfqpoint{2.553705in}{1.915274in}}%
\pgfpathlineto{\pgfqpoint{2.604588in}{1.939685in}}%
\pgfpathlineto{\pgfqpoint{2.655472in}{1.963842in}}%
\pgfpathlineto{\pgfqpoint{2.706356in}{1.987754in}}%
\pgfpathlineto{\pgfqpoint{2.757240in}{2.011427in}}%
\pgfpathlineto{\pgfqpoint{2.808124in}{2.034868in}}%
\pgfpathlineto{\pgfqpoint{2.859008in}{2.058085in}}%
\pgfpathlineto{\pgfqpoint{2.909891in}{2.081083in}}%
\pgfpathlineto{\pgfqpoint{2.960775in}{2.103869in}}%
\pgfpathlineto{\pgfqpoint{3.011659in}{2.126449in}}%
\pgfpathlineto{\pgfqpoint{3.062543in}{2.148827in}}%
\pgfpathlineto{\pgfqpoint{3.113427in}{2.171009in}}%
\pgfpathlineto{\pgfqpoint{3.164311in}{2.193001in}}%
\pgfpathlineto{\pgfqpoint{3.215194in}{2.214807in}}%
\pgfpathlineto{\pgfqpoint{3.266078in}{2.236432in}}%
\pgfpathlineto{\pgfqpoint{3.316962in}{2.257880in}}%
\pgfpathlineto{\pgfqpoint{3.367846in}{2.279155in}}%
\pgfpathlineto{\pgfqpoint{3.418730in}{2.300261in}}%
\pgfpathlineto{\pgfqpoint{3.469614in}{2.321204in}}%
\pgfpathlineto{\pgfqpoint{3.520497in}{2.341985in}}%
\pgfpathlineto{\pgfqpoint{3.571381in}{2.362610in}}%
\pgfpathlineto{\pgfqpoint{3.622265in}{2.383081in}}%
\pgfpathlineto{\pgfqpoint{3.673149in}{2.403401in}}%
\pgfpathlineto{\pgfqpoint{3.724033in}{2.423575in}}%
\pgfpathlineto{\pgfqpoint{3.774917in}{2.443605in}}%
\pgfpathlineto{\pgfqpoint{3.825800in}{2.463494in}}%
\pgfpathlineto{\pgfqpoint{3.876684in}{2.483245in}}%
\pgfpathlineto{\pgfqpoint{3.927568in}{2.502862in}}%
\pgfpathlineto{\pgfqpoint{3.978452in}{2.522346in}}%
\pgfpathlineto{\pgfqpoint{4.029336in}{2.541700in}}%
\pgfpathlineto{\pgfqpoint{4.080220in}{2.560927in}}%
\pgfpathlineto{\pgfqpoint{4.131104in}{2.580030in}}%
\pgfpathlineto{\pgfqpoint{4.181987in}{2.599010in}}%
\pgfpathlineto{\pgfqpoint{4.232871in}{2.617871in}}%
\pgfpathlineto{\pgfqpoint{4.283755in}{2.636614in}}%
\pgfpathlineto{\pgfqpoint{4.334639in}{2.655241in}}%
\pgfpathlineto{\pgfqpoint{4.385523in}{2.673755in}}%
\pgfpathlineto{\pgfqpoint{4.436407in}{2.692158in}}%
\pgfpathlineto{\pgfqpoint{4.487290in}{2.710451in}}%
\pgfpathlineto{\pgfqpoint{4.538174in}{2.728637in}}%
\pgfpathlineto{\pgfqpoint{4.589058in}{2.746718in}}%
\pgfpathlineto{\pgfqpoint{4.639942in}{2.764694in}}%
\pgfpathlineto{\pgfqpoint{4.690826in}{2.782569in}}%
\pgfpathlineto{\pgfqpoint{4.741710in}{2.800344in}}%
\pgfpathlineto{\pgfqpoint{4.792593in}{2.818019in}}%
\pgfpathlineto{\pgfqpoint{4.843477in}{2.835598in}}%
\pgfpathlineto{\pgfqpoint{4.894361in}{2.853082in}}%
\pgfpathlineto{\pgfqpoint{4.945245in}{2.870471in}}%
\pgfpathlineto{\pgfqpoint{4.996129in}{2.887769in}}%
\pgfpathlineto{\pgfqpoint{5.047013in}{2.904975in}}%
\pgfpathlineto{\pgfqpoint{5.097896in}{2.922092in}}%
\pgfpathlineto{\pgfqpoint{5.148780in}{2.939121in}}%
\pgfpathlineto{\pgfqpoint{5.199664in}{2.956063in}}%
\pgfpathlineto{\pgfqpoint{5.250548in}{2.972920in}}%
\pgfpathlineto{\pgfqpoint{5.301432in}{2.989692in}}%
\pgfpathlineto{\pgfqpoint{5.352316in}{3.006382in}}%
\pgfpathlineto{\pgfqpoint{5.403199in}{3.022990in}}%
\pgfpathlineto{\pgfqpoint{5.454083in}{3.039517in}}%
\pgfpathlineto{\pgfqpoint{5.504967in}{3.055965in}}%
\pgfpathlineto{\pgfqpoint{5.555851in}{3.072335in}}%
\pgfpathlineto{\pgfqpoint{5.606735in}{3.088627in}}%
\pgfpathlineto{\pgfqpoint{5.657619in}{3.104844in}}%
\pgfpathlineto{\pgfqpoint{5.708502in}{3.120986in}}%
\pgfusepath{stroke}%
\end{pgfscope}%
\begin{pgfscope}%
\pgfsetrectcap%
\pgfsetmiterjoin%
\pgfsetlinewidth{0.803000pt}%
\definecolor{currentstroke}{rgb}{0.000000,0.000000,0.000000}%
\pgfsetstrokecolor{currentstroke}%
\pgfsetdash{}{0pt}%
\pgfpathmoveto{\pgfqpoint{0.671003in}{0.548486in}}%
\pgfpathlineto{\pgfqpoint{0.671003in}{3.243486in}}%
\pgfusepath{stroke}%
\end{pgfscope}%
\begin{pgfscope}%
\pgfsetrectcap%
\pgfsetmiterjoin%
\pgfsetlinewidth{0.803000pt}%
\definecolor{currentstroke}{rgb}{0.000000,0.000000,0.000000}%
\pgfsetstrokecolor{currentstroke}%
\pgfsetdash{}{0pt}%
\pgfpathmoveto{\pgfqpoint{5.708502in}{0.548486in}}%
\pgfpathlineto{\pgfqpoint{5.708502in}{3.243486in}}%
\pgfusepath{stroke}%
\end{pgfscope}%
\begin{pgfscope}%
\pgfsetrectcap%
\pgfsetmiterjoin%
\pgfsetlinewidth{0.803000pt}%
\definecolor{currentstroke}{rgb}{0.000000,0.000000,0.000000}%
\pgfsetstrokecolor{currentstroke}%
\pgfsetdash{}{0pt}%
\pgfpathmoveto{\pgfqpoint{0.671003in}{0.548486in}}%
\pgfpathlineto{\pgfqpoint{5.708502in}{0.548486in}}%
\pgfusepath{stroke}%
\end{pgfscope}%
\begin{pgfscope}%
\pgfsetrectcap%
\pgfsetmiterjoin%
\pgfsetlinewidth{0.803000pt}%
\definecolor{currentstroke}{rgb}{0.000000,0.000000,0.000000}%
\pgfsetstrokecolor{currentstroke}%
\pgfsetdash{}{0pt}%
\pgfpathmoveto{\pgfqpoint{0.671003in}{3.243486in}}%
\pgfpathlineto{\pgfqpoint{5.708502in}{3.243486in}}%
\pgfusepath{stroke}%
\end{pgfscope}%
\end{pgfpicture}%
\makeatother%
\endgroup%

     \caption{Radii of contact areas $a$ for multiple vertical displacements $d$ scattered against the exact values obtained with \refequ{equ:contact-radius}. The quantities are normalized by the mesh size $h$ at the interface.}
     \label{fig:contact-radius-3d}
\end{subfigure}
\caption{Results for the three dimensional version of the problem sketched in \reffig{fig:sketch-plane-sphere}, i.e. a semisphere is pushed into a half-space. For a fixed mesh of mesh size $h$ at the interface, multiple displacements $d$ are applied.}
\label{fig:contact-plane-sphere}
\end{figure}

\clearpage
\subsubsection{Contact between two semispheres}

The second test case models the contact between two semispheres (\reffig{fig:sketch-sphere-sphere}). 

\begin{figure}[H]
    \centering
    \begin{tikzpicture}
    \draw[ultra thick, darkblue, fill=darkblue!10!white] (3, 2.75) arc(360:315:3) to[out=225, in=0, looseness=0.7] (0.75, 0) -- (-0.75, 0) to[out=180, in=315, looseness=0.7] (-2.121, 0.629) arc(225:180:3) -- cycle;
    \draw[ultra thick, darkblue, fill=darkblue!10!white] (3, -2.75) arc(0:45:3) to[out=135, in=0, looseness=0.7] (0.75, 0) -- (-0.75, 0) to[out=180, in=45, looseness=0.7] (-2.121, -0.629) arc(135:180:3) -- cycle;
    \draw[ultra thick, darkblue, dashed] (3, 3) arc(360:180:3) -- cycle;
    \draw[ultra thick, darkblue, dashed] (3, -3) arc(0:180:3) -- cycle;
    \draw[<->, thick] (3.25, 3.05) -- (3.25, 2.7) node[midway, right] {$d/2$};
    \draw[<->, thick] (3.25, -3.05) -- (3.25, -2.7) node[midway, right] {$d/2$};
    %\draw[->, thick] (0, 3.75) -- (0, 3.25) node[midway, right] {$F$};
    %\draw[->, thick] (0, -3.75) -- (0, -3.25) node[midway, right] {$F$};
\end{tikzpicture}
    \caption{Sketch of the second experimental setup where two semispheres are put into contact.}
    \label{fig:sketch-sphere-sphere}
\end{figure}

In theory, the resulting interface is planar. This fact is used to verify the implementation by testing whether the deviation of the interface nodes from the theoretical contact plane is within a reasonable upper bound.

\begin{figure}[ht]
\begin{subfigure}[t]{.49\textwidth}
    \centering
    %% Creator: Matplotlib, PGF backend
%%
%% To include the figure in your LaTeX document, write
%%   \input{<filename>.pgf}
%%
%% Make sure the required packages are loaded in your preamble
%%   \usepackage{pgf}
%%
%% Also ensure that all the required font packages are loaded; for instance,
%% the lmodern package is sometimes necessary when using math font.
%%   \usepackage{lmodern}
%%
%% Figures using additional raster images can only be included by \input if
%% they are in the same directory as the main LaTeX file. For loading figures
%% from other directories you can use the `import` package
%%   \usepackage{import}
%%
%% and then include the figures with
%%   \import{<path to file>}{<filename>.pgf}
%%
%% Matplotlib used the following preamble
%%   
%%   \usepackage{fontspec}
%%   \setmainfont{DejaVuSans.ttf}[Path=\detokenize{/home/fabio/.local/lib/python3.8/site-packages/matplotlib/mpl-data/fonts/ttf/}]
%%   \setsansfont{DejaVuSans.ttf}[Path=\detokenize{/home/fabio/.local/lib/python3.8/site-packages/matplotlib/mpl-data/fonts/ttf/}]
%%   \setmonofont{DejaVuSansMono.ttf}[Path=\detokenize{/home/fabio/.local/lib/python3.8/site-packages/matplotlib/mpl-data/fonts/ttf/}]
%%   \makeatletter\@ifpackageloaded{underscore}{}{\usepackage[strings]{underscore}}\makeatother
%%
\begingroup%
\makeatletter%
\begin{pgfpicture}%
\pgfpathrectangle{\pgfpointorigin}{\pgfqpoint{2.759844in}{2.573486in}}%
\pgfusepath{use as bounding box, clip}%
\begin{pgfscope}%
\pgfsetbuttcap%
\pgfsetmiterjoin%
\definecolor{currentfill}{rgb}{1.000000,1.000000,1.000000}%
\pgfsetfillcolor{currentfill}%
\pgfsetlinewidth{0.000000pt}%
\definecolor{currentstroke}{rgb}{1.000000,1.000000,1.000000}%
\pgfsetstrokecolor{currentstroke}%
\pgfsetdash{}{0pt}%
\pgfpathmoveto{\pgfqpoint{0.000000in}{0.000000in}}%
\pgfpathlineto{\pgfqpoint{2.759844in}{0.000000in}}%
\pgfpathlineto{\pgfqpoint{2.759844in}{2.573486in}}%
\pgfpathlineto{\pgfqpoint{0.000000in}{2.573486in}}%
\pgfpathlineto{\pgfqpoint{0.000000in}{0.000000in}}%
\pgfpathclose%
\pgfusepath{fill}%
\end{pgfscope}%
\begin{pgfscope}%
\pgfsetbuttcap%
\pgfsetmiterjoin%
\definecolor{currentfill}{rgb}{1.000000,1.000000,1.000000}%
\pgfsetfillcolor{currentfill}%
\pgfsetlinewidth{0.000000pt}%
\definecolor{currentstroke}{rgb}{0.000000,0.000000,0.000000}%
\pgfsetstrokecolor{currentstroke}%
\pgfsetstrokeopacity{0.000000}%
\pgfsetdash{}{0pt}%
\pgfpathmoveto{\pgfqpoint{0.713248in}{0.548486in}}%
\pgfpathlineto{\pgfqpoint{2.650748in}{0.548486in}}%
\pgfpathlineto{\pgfqpoint{2.650748in}{2.473486in}}%
\pgfpathlineto{\pgfqpoint{0.713248in}{2.473486in}}%
\pgfpathlineto{\pgfqpoint{0.713248in}{0.548486in}}%
\pgfpathclose%
\pgfusepath{fill}%
\end{pgfscope}%
\begin{pgfscope}%
\pgfpathrectangle{\pgfqpoint{0.713248in}{0.548486in}}{\pgfqpoint{1.937500in}{1.925000in}}%
\pgfusepath{clip}%
\pgfsetrectcap%
\pgfsetroundjoin%
\pgfsetlinewidth{0.501875pt}%
\definecolor{currentstroke}{rgb}{0.054902,0.262745,0.486275}%
\pgfsetstrokecolor{currentstroke}%
\pgfsetdash{}{0pt}%
\pgfpathmoveto{\pgfqpoint{2.122339in}{0.635986in}}%
\pgfpathlineto{\pgfqpoint{1.681998in}{0.635986in}}%
\pgfpathmoveto{\pgfqpoint{2.122339in}{0.635986in}}%
\pgfpathlineto{\pgfqpoint{2.562680in}{0.635986in}}%
\pgfpathmoveto{\pgfqpoint{2.521416in}{0.942226in}}%
\pgfpathlineto{\pgfqpoint{2.562680in}{0.635986in}}%
\pgfpathmoveto{\pgfqpoint{2.521416in}{0.942226in}}%
\pgfpathlineto{\pgfqpoint{2.386345in}{1.218237in}}%
\pgfpathmoveto{\pgfqpoint{2.348760in}{1.266085in}}%
\pgfpathlineto{\pgfqpoint{2.386345in}{1.218237in}}%
\pgfpathmoveto{\pgfqpoint{2.308288in}{1.311201in}}%
\pgfpathlineto{\pgfqpoint{2.348760in}{1.266085in}}%
\pgfpathmoveto{\pgfqpoint{2.264799in}{1.353578in}}%
\pgfpathlineto{\pgfqpoint{2.308288in}{1.311201in}}%
\pgfpathmoveto{\pgfqpoint{2.217995in}{1.392469in}}%
\pgfpathlineto{\pgfqpoint{2.264799in}{1.353578in}}%
\pgfpathmoveto{\pgfqpoint{2.167831in}{1.428076in}}%
\pgfpathlineto{\pgfqpoint{2.217995in}{1.392469in}}%
\pgfpathmoveto{\pgfqpoint{2.113663in}{1.459833in}}%
\pgfpathlineto{\pgfqpoint{2.167831in}{1.428076in}}%
\pgfpathmoveto{\pgfqpoint{2.051519in}{1.486651in}}%
\pgfpathlineto{\pgfqpoint{2.113663in}{1.459833in}}%
\pgfpathmoveto{\pgfqpoint{1.983145in}{1.503946in}}%
\pgfpathlineto{\pgfqpoint{2.051519in}{1.486651in}}%
\pgfpathmoveto{\pgfqpoint{1.952999in}{1.503972in}}%
\pgfpathlineto{\pgfqpoint{1.983145in}{1.503946in}}%
\pgfpathmoveto{\pgfqpoint{1.919038in}{1.505282in}}%
\pgfpathlineto{\pgfqpoint{1.952999in}{1.503972in}}%
\pgfpathmoveto{\pgfqpoint{1.868254in}{1.506787in}}%
\pgfpathlineto{\pgfqpoint{1.919038in}{1.505282in}}%
\pgfpathmoveto{\pgfqpoint{1.816482in}{1.509569in}}%
\pgfpathlineto{\pgfqpoint{1.868254in}{1.506787in}}%
\pgfpathmoveto{\pgfqpoint{1.761941in}{1.512217in}}%
\pgfpathlineto{\pgfqpoint{1.816482in}{1.509569in}}%
\pgfpathmoveto{\pgfqpoint{1.707484in}{1.512345in}}%
\pgfpathlineto{\pgfqpoint{1.761941in}{1.512217in}}%
\pgfpathmoveto{\pgfqpoint{1.652928in}{1.512493in}}%
\pgfpathlineto{\pgfqpoint{1.707484in}{1.512345in}}%
\pgfpathmoveto{\pgfqpoint{1.598228in}{1.511788in}}%
\pgfpathlineto{\pgfqpoint{1.652928in}{1.512493in}}%
\pgfpathmoveto{\pgfqpoint{1.543874in}{1.511040in}}%
\pgfpathlineto{\pgfqpoint{1.598228in}{1.511788in}}%
\pgfpathmoveto{\pgfqpoint{1.491725in}{1.509289in}}%
\pgfpathlineto{\pgfqpoint{1.543874in}{1.511040in}}%
\pgfpathmoveto{\pgfqpoint{1.440523in}{1.505204in}}%
\pgfpathlineto{\pgfqpoint{1.491725in}{1.509289in}}%
\pgfpathmoveto{\pgfqpoint{1.405209in}{1.503583in}}%
\pgfpathlineto{\pgfqpoint{1.440523in}{1.505204in}}%
\pgfpathmoveto{\pgfqpoint{1.376734in}{1.503884in}}%
\pgfpathlineto{\pgfqpoint{1.405209in}{1.503583in}}%
\pgfpathmoveto{\pgfqpoint{1.309162in}{1.485848in}}%
\pgfpathlineto{\pgfqpoint{1.376734in}{1.503884in}}%
\pgfpathmoveto{\pgfqpoint{1.247330in}{1.458637in}}%
\pgfpathlineto{\pgfqpoint{1.309162in}{1.485848in}}%
\pgfpathmoveto{\pgfqpoint{1.193492in}{1.426806in}}%
\pgfpathlineto{\pgfqpoint{1.247330in}{1.458637in}}%
\pgfpathmoveto{\pgfqpoint{1.143388in}{1.391241in}}%
\pgfpathlineto{\pgfqpoint{1.193492in}{1.426806in}}%
\pgfpathmoveto{\pgfqpoint{1.096650in}{1.352363in}}%
\pgfpathlineto{\pgfqpoint{1.143388in}{1.391241in}}%
\pgfpathmoveto{\pgfqpoint{1.053129in}{1.310116in}}%
\pgfpathlineto{\pgfqpoint{1.096650in}{1.352363in}}%
\pgfpathmoveto{\pgfqpoint{1.012765in}{1.264946in}}%
\pgfpathlineto{\pgfqpoint{0.975347in}{1.217128in}}%
\pgfpathmoveto{\pgfqpoint{1.012765in}{1.264946in}}%
\pgfpathlineto{\pgfqpoint{1.053129in}{1.310116in}}%
\pgfpathmoveto{\pgfqpoint{0.841667in}{0.941178in}}%
\pgfpathlineto{\pgfqpoint{0.975347in}{1.217128in}}%
\pgfpathmoveto{\pgfqpoint{0.841667in}{0.941178in}}%
\pgfpathlineto{\pgfqpoint{0.801316in}{0.635986in}}%
\pgfpathmoveto{\pgfqpoint{1.241657in}{0.635986in}}%
\pgfpathlineto{\pgfqpoint{1.681998in}{0.635986in}}%
\pgfpathmoveto{\pgfqpoint{1.241657in}{0.635986in}}%
\pgfpathlineto{\pgfqpoint{0.801316in}{0.635986in}}%
\pgfpathmoveto{\pgfqpoint{2.109121in}{1.140781in}}%
\pgfpathlineto{\pgfqpoint{1.947111in}{1.059017in}}%
\pgfpathmoveto{\pgfqpoint{1.312364in}{1.203204in}}%
\pgfpathlineto{\pgfqpoint{1.485163in}{1.092347in}}%
\pgfpathmoveto{\pgfqpoint{1.138820in}{1.189260in}}%
\pgfpathlineto{\pgfqpoint{0.975347in}{1.217128in}}%
\pgfpathmoveto{\pgfqpoint{1.138820in}{1.189260in}}%
\pgfpathlineto{\pgfqpoint{1.012765in}{1.264946in}}%
\pgfpathmoveto{\pgfqpoint{1.138820in}{1.189260in}}%
\pgfpathlineto{\pgfqpoint{1.312364in}{1.203204in}}%
\pgfpathmoveto{\pgfqpoint{1.633424in}{1.370676in}}%
\pgfpathlineto{\pgfqpoint{1.717703in}{1.244847in}}%
\pgfpathmoveto{\pgfqpoint{1.633424in}{1.370676in}}%
\pgfpathlineto{\pgfqpoint{1.518346in}{1.319445in}}%
\pgfpathmoveto{\pgfqpoint{1.772283in}{1.375785in}}%
\pgfpathlineto{\pgfqpoint{1.717703in}{1.244847in}}%
\pgfpathmoveto{\pgfqpoint{1.772283in}{1.375785in}}%
\pgfpathlineto{\pgfqpoint{1.633424in}{1.370676in}}%
\pgfpathmoveto{\pgfqpoint{1.405118in}{1.365318in}}%
\pgfpathlineto{\pgfqpoint{1.518346in}{1.319445in}}%
\pgfpathmoveto{\pgfqpoint{2.046525in}{1.339679in}}%
\pgfpathlineto{\pgfqpoint{1.930696in}{1.291141in}}%
\pgfpathmoveto{\pgfqpoint{2.248615in}{1.186532in}}%
\pgfpathlineto{\pgfqpoint{2.386345in}{1.218237in}}%
\pgfpathmoveto{\pgfqpoint{2.248615in}{1.186532in}}%
\pgfpathlineto{\pgfqpoint{2.348760in}{1.266085in}}%
\pgfpathmoveto{\pgfqpoint{2.248615in}{1.186532in}}%
\pgfpathlineto{\pgfqpoint{2.109121in}{1.140781in}}%
\pgfpathmoveto{\pgfqpoint{1.529376in}{1.414978in}}%
\pgfpathlineto{\pgfqpoint{1.518346in}{1.319445in}}%
\pgfpathmoveto{\pgfqpoint{1.529376in}{1.414978in}}%
\pgfpathlineto{\pgfqpoint{1.633424in}{1.370676in}}%
\pgfpathmoveto{\pgfqpoint{1.872186in}{1.397839in}}%
\pgfpathlineto{\pgfqpoint{1.930696in}{1.291141in}}%
\pgfpathmoveto{\pgfqpoint{1.872186in}{1.397839in}}%
\pgfpathlineto{\pgfqpoint{1.772283in}{1.375785in}}%
\pgfpathmoveto{\pgfqpoint{1.961140in}{1.388553in}}%
\pgfpathlineto{\pgfqpoint{1.930696in}{1.291141in}}%
\pgfpathmoveto{\pgfqpoint{1.961140in}{1.388553in}}%
\pgfpathlineto{\pgfqpoint{2.046525in}{1.339679in}}%
\pgfpathmoveto{\pgfqpoint{1.961140in}{1.388553in}}%
\pgfpathlineto{\pgfqpoint{1.872186in}{1.397839in}}%
\pgfpathmoveto{\pgfqpoint{1.310242in}{1.352368in}}%
\pgfpathlineto{\pgfqpoint{1.312364in}{1.203204in}}%
\pgfpathmoveto{\pgfqpoint{1.310242in}{1.352368in}}%
\pgfpathlineto{\pgfqpoint{1.405118in}{1.365318in}}%
\pgfpathmoveto{\pgfqpoint{1.211021in}{1.296806in}}%
\pgfpathlineto{\pgfqpoint{1.312364in}{1.203204in}}%
\pgfpathmoveto{\pgfqpoint{1.211021in}{1.296806in}}%
\pgfpathlineto{\pgfqpoint{1.138820in}{1.189260in}}%
\pgfpathmoveto{\pgfqpoint{1.211021in}{1.296806in}}%
\pgfpathlineto{\pgfqpoint{1.310242in}{1.352368in}}%
\pgfpathmoveto{\pgfqpoint{2.132875in}{1.281739in}}%
\pgfpathlineto{\pgfqpoint{2.109121in}{1.140781in}}%
\pgfpathmoveto{\pgfqpoint{2.132875in}{1.281739in}}%
\pgfpathlineto{\pgfqpoint{2.046525in}{1.339679in}}%
\pgfpathmoveto{\pgfqpoint{2.132875in}{1.281739in}}%
\pgfpathlineto{\pgfqpoint{2.248615in}{1.186532in}}%
\pgfpathmoveto{\pgfqpoint{1.680068in}{1.437866in}}%
\pgfpathlineto{\pgfqpoint{1.707484in}{1.512345in}}%
\pgfpathmoveto{\pgfqpoint{1.680068in}{1.437866in}}%
\pgfpathlineto{\pgfqpoint{1.652928in}{1.512493in}}%
\pgfpathmoveto{\pgfqpoint{1.680068in}{1.437866in}}%
\pgfpathlineto{\pgfqpoint{1.633424in}{1.370676in}}%
\pgfpathmoveto{\pgfqpoint{1.680068in}{1.437866in}}%
\pgfpathlineto{\pgfqpoint{1.772283in}{1.375785in}}%
\pgfpathmoveto{\pgfqpoint{2.137178in}{0.951746in}}%
\pgfpathlineto{\pgfqpoint{2.122339in}{0.635986in}}%
\pgfpathmoveto{\pgfqpoint{2.137178in}{0.951746in}}%
\pgfpathlineto{\pgfqpoint{1.947111in}{1.059017in}}%
\pgfpathmoveto{\pgfqpoint{2.137178in}{0.951746in}}%
\pgfpathlineto{\pgfqpoint{2.109121in}{1.140781in}}%
\pgfpathmoveto{\pgfqpoint{2.137178in}{0.951746in}}%
\pgfpathlineto{\pgfqpoint{2.248615in}{1.186532in}}%
\pgfpathmoveto{\pgfqpoint{1.439632in}{1.429170in}}%
\pgfpathlineto{\pgfqpoint{1.440523in}{1.505204in}}%
\pgfpathmoveto{\pgfqpoint{1.439632in}{1.429170in}}%
\pgfpathlineto{\pgfqpoint{1.405209in}{1.503583in}}%
\pgfpathmoveto{\pgfqpoint{1.439632in}{1.429170in}}%
\pgfpathlineto{\pgfqpoint{1.518346in}{1.319445in}}%
\pgfpathmoveto{\pgfqpoint{1.439632in}{1.429170in}}%
\pgfpathlineto{\pgfqpoint{1.405118in}{1.365318in}}%
\pgfpathmoveto{\pgfqpoint{1.439632in}{1.429170in}}%
\pgfpathlineto{\pgfqpoint{1.529376in}{1.414978in}}%
\pgfpathmoveto{\pgfqpoint{1.117568in}{1.279757in}}%
\pgfpathlineto{\pgfqpoint{1.096650in}{1.352363in}}%
\pgfpathmoveto{\pgfqpoint{1.117568in}{1.279757in}}%
\pgfpathlineto{\pgfqpoint{1.053129in}{1.310116in}}%
\pgfpathmoveto{\pgfqpoint{1.117568in}{1.279757in}}%
\pgfpathlineto{\pgfqpoint{1.012765in}{1.264946in}}%
\pgfpathmoveto{\pgfqpoint{1.117568in}{1.279757in}}%
\pgfpathlineto{\pgfqpoint{1.138820in}{1.189260in}}%
\pgfpathmoveto{\pgfqpoint{1.117568in}{1.279757in}}%
\pgfpathlineto{\pgfqpoint{1.211021in}{1.296806in}}%
\pgfpathmoveto{\pgfqpoint{1.782613in}{1.465162in}}%
\pgfpathlineto{\pgfqpoint{1.816482in}{1.509569in}}%
\pgfpathmoveto{\pgfqpoint{1.782613in}{1.465162in}}%
\pgfpathlineto{\pgfqpoint{1.761941in}{1.512217in}}%
\pgfpathmoveto{\pgfqpoint{1.782613in}{1.465162in}}%
\pgfpathlineto{\pgfqpoint{1.772283in}{1.375785in}}%
\pgfpathmoveto{\pgfqpoint{1.699769in}{1.040285in}}%
\pgfpathlineto{\pgfqpoint{1.485163in}{1.092347in}}%
\pgfpathmoveto{\pgfqpoint{1.699769in}{1.040285in}}%
\pgfpathlineto{\pgfqpoint{1.947111in}{1.059017in}}%
\pgfpathmoveto{\pgfqpoint{1.699769in}{1.040285in}}%
\pgfpathlineto{\pgfqpoint{1.717703in}{1.244847in}}%
\pgfpathmoveto{\pgfqpoint{1.576824in}{1.457041in}}%
\pgfpathlineto{\pgfqpoint{1.598228in}{1.511788in}}%
\pgfpathmoveto{\pgfqpoint{1.576824in}{1.457041in}}%
\pgfpathlineto{\pgfqpoint{1.543874in}{1.511040in}}%
\pgfpathmoveto{\pgfqpoint{1.576824in}{1.457041in}}%
\pgfpathlineto{\pgfqpoint{1.633424in}{1.370676in}}%
\pgfpathmoveto{\pgfqpoint{1.576824in}{1.457041in}}%
\pgfpathlineto{\pgfqpoint{1.529376in}{1.414978in}}%
\pgfpathmoveto{\pgfqpoint{2.331373in}{1.018221in}}%
\pgfpathlineto{\pgfqpoint{2.386345in}{1.218237in}}%
\pgfpathmoveto{\pgfqpoint{2.331373in}{1.018221in}}%
\pgfpathlineto{\pgfqpoint{2.521416in}{0.942226in}}%
\pgfpathmoveto{\pgfqpoint{2.331373in}{1.018221in}}%
\pgfpathlineto{\pgfqpoint{2.248615in}{1.186532in}}%
\pgfpathmoveto{\pgfqpoint{2.331373in}{1.018221in}}%
\pgfpathlineto{\pgfqpoint{2.137178in}{0.951746in}}%
\pgfpathmoveto{\pgfqpoint{1.923863in}{1.445776in}}%
\pgfpathlineto{\pgfqpoint{1.952999in}{1.503972in}}%
\pgfpathmoveto{\pgfqpoint{1.923863in}{1.445776in}}%
\pgfpathlineto{\pgfqpoint{1.919038in}{1.505282in}}%
\pgfpathmoveto{\pgfqpoint{1.923863in}{1.445776in}}%
\pgfpathlineto{\pgfqpoint{1.872186in}{1.397839in}}%
\pgfpathmoveto{\pgfqpoint{1.923863in}{1.445776in}}%
\pgfpathlineto{\pgfqpoint{1.961140in}{1.388553in}}%
\pgfpathmoveto{\pgfqpoint{2.018592in}{1.426931in}}%
\pgfpathlineto{\pgfqpoint{2.051519in}{1.486651in}}%
\pgfpathmoveto{\pgfqpoint{2.018592in}{1.426931in}}%
\pgfpathlineto{\pgfqpoint{1.983145in}{1.503946in}}%
\pgfpathmoveto{\pgfqpoint{2.018592in}{1.426931in}}%
\pgfpathlineto{\pgfqpoint{2.046525in}{1.339679in}}%
\pgfpathmoveto{\pgfqpoint{2.018592in}{1.426931in}}%
\pgfpathlineto{\pgfqpoint{1.961140in}{1.388553in}}%
\pgfpathmoveto{\pgfqpoint{1.048137in}{0.944700in}}%
\pgfpathlineto{\pgfqpoint{0.975347in}{1.217128in}}%
\pgfpathmoveto{\pgfqpoint{1.048137in}{0.944700in}}%
\pgfpathlineto{\pgfqpoint{0.801316in}{0.635986in}}%
\pgfpathmoveto{\pgfqpoint{1.048137in}{0.944700in}}%
\pgfpathlineto{\pgfqpoint{0.841667in}{0.941178in}}%
\pgfpathmoveto{\pgfqpoint{1.048137in}{0.944700in}}%
\pgfpathlineto{\pgfqpoint{1.241657in}{0.635986in}}%
\pgfpathmoveto{\pgfqpoint{1.048137in}{0.944700in}}%
\pgfpathlineto{\pgfqpoint{1.138820in}{1.189260in}}%
\pgfpathmoveto{\pgfqpoint{2.117014in}{1.385442in}}%
\pgfpathlineto{\pgfqpoint{2.167831in}{1.428076in}}%
\pgfpathmoveto{\pgfqpoint{2.117014in}{1.385442in}}%
\pgfpathlineto{\pgfqpoint{2.113663in}{1.459833in}}%
\pgfpathmoveto{\pgfqpoint{2.117014in}{1.385442in}}%
\pgfpathlineto{\pgfqpoint{2.046525in}{1.339679in}}%
\pgfpathmoveto{\pgfqpoint{2.117014in}{1.385442in}}%
\pgfpathlineto{\pgfqpoint{2.132875in}{1.281739in}}%
\pgfpathmoveto{\pgfqpoint{2.207549in}{1.321314in}}%
\pgfpathlineto{\pgfqpoint{2.264799in}{1.353578in}}%
\pgfpathmoveto{\pgfqpoint{2.207549in}{1.321314in}}%
\pgfpathlineto{\pgfqpoint{2.217995in}{1.392469in}}%
\pgfpathmoveto{\pgfqpoint{2.207549in}{1.321314in}}%
\pgfpathlineto{\pgfqpoint{2.248615in}{1.186532in}}%
\pgfpathmoveto{\pgfqpoint{2.207549in}{1.321314in}}%
\pgfpathlineto{\pgfqpoint{2.132875in}{1.281739in}}%
\pgfpathmoveto{\pgfqpoint{1.342506in}{1.425620in}}%
\pgfpathlineto{\pgfqpoint{1.376734in}{1.503884in}}%
\pgfpathmoveto{\pgfqpoint{1.342506in}{1.425620in}}%
\pgfpathlineto{\pgfqpoint{1.309162in}{1.485848in}}%
\pgfpathmoveto{\pgfqpoint{1.342506in}{1.425620in}}%
\pgfpathlineto{\pgfqpoint{1.405118in}{1.365318in}}%
\pgfpathmoveto{\pgfqpoint{1.342506in}{1.425620in}}%
\pgfpathlineto{\pgfqpoint{1.310242in}{1.352368in}}%
\pgfpathmoveto{\pgfqpoint{1.299964in}{0.984994in}}%
\pgfpathlineto{\pgfqpoint{1.241657in}{0.635986in}}%
\pgfpathmoveto{\pgfqpoint{1.299964in}{0.984994in}}%
\pgfpathlineto{\pgfqpoint{1.485163in}{1.092347in}}%
\pgfpathmoveto{\pgfqpoint{1.299964in}{0.984994in}}%
\pgfpathlineto{\pgfqpoint{1.312364in}{1.203204in}}%
\pgfpathmoveto{\pgfqpoint{1.299964in}{0.984994in}}%
\pgfpathlineto{\pgfqpoint{1.138820in}{1.189260in}}%
\pgfpathmoveto{\pgfqpoint{1.299964in}{0.984994in}}%
\pgfpathlineto{\pgfqpoint{1.048137in}{0.944700in}}%
\pgfpathmoveto{\pgfqpoint{1.244082in}{1.384740in}}%
\pgfpathlineto{\pgfqpoint{1.247330in}{1.458637in}}%
\pgfpathmoveto{\pgfqpoint{1.244082in}{1.384740in}}%
\pgfpathlineto{\pgfqpoint{1.193492in}{1.426806in}}%
\pgfpathmoveto{\pgfqpoint{1.244082in}{1.384740in}}%
\pgfpathlineto{\pgfqpoint{1.310242in}{1.352368in}}%
\pgfpathmoveto{\pgfqpoint{1.244082in}{1.384740in}}%
\pgfpathlineto{\pgfqpoint{1.211021in}{1.296806in}}%
\pgfpathmoveto{\pgfqpoint{1.485231in}{1.464971in}}%
\pgfpathlineto{\pgfqpoint{1.491725in}{1.509289in}}%
\pgfpathmoveto{\pgfqpoint{1.485231in}{1.464971in}}%
\pgfpathlineto{\pgfqpoint{1.440523in}{1.505204in}}%
\pgfpathmoveto{\pgfqpoint{1.485231in}{1.464971in}}%
\pgfpathlineto{\pgfqpoint{1.529376in}{1.414978in}}%
\pgfpathmoveto{\pgfqpoint{1.485231in}{1.464971in}}%
\pgfpathlineto{\pgfqpoint{1.439632in}{1.429170in}}%
\pgfpathmoveto{\pgfqpoint{1.731768in}{1.466439in}}%
\pgfpathlineto{\pgfqpoint{1.761941in}{1.512217in}}%
\pgfpathmoveto{\pgfqpoint{1.731768in}{1.466439in}}%
\pgfpathlineto{\pgfqpoint{1.707484in}{1.512345in}}%
\pgfpathmoveto{\pgfqpoint{1.731768in}{1.466439in}}%
\pgfpathlineto{\pgfqpoint{1.772283in}{1.375785in}}%
\pgfpathmoveto{\pgfqpoint{1.731768in}{1.466439in}}%
\pgfpathlineto{\pgfqpoint{1.680068in}{1.437866in}}%
\pgfpathmoveto{\pgfqpoint{1.731768in}{1.466439in}}%
\pgfpathlineto{\pgfqpoint{1.782613in}{1.465162in}}%
\pgfpathmoveto{\pgfqpoint{1.831844in}{1.452132in}}%
\pgfpathlineto{\pgfqpoint{1.868254in}{1.506787in}}%
\pgfpathmoveto{\pgfqpoint{1.831844in}{1.452132in}}%
\pgfpathlineto{\pgfqpoint{1.816482in}{1.509569in}}%
\pgfpathmoveto{\pgfqpoint{1.831844in}{1.452132in}}%
\pgfpathlineto{\pgfqpoint{1.772283in}{1.375785in}}%
\pgfpathmoveto{\pgfqpoint{1.831844in}{1.452132in}}%
\pgfpathlineto{\pgfqpoint{1.872186in}{1.397839in}}%
\pgfpathmoveto{\pgfqpoint{1.831844in}{1.452132in}}%
\pgfpathlineto{\pgfqpoint{1.782613in}{1.465162in}}%
\pgfpathmoveto{\pgfqpoint{1.882056in}{1.461089in}}%
\pgfpathlineto{\pgfqpoint{1.919038in}{1.505282in}}%
\pgfpathmoveto{\pgfqpoint{1.882056in}{1.461089in}}%
\pgfpathlineto{\pgfqpoint{1.868254in}{1.506787in}}%
\pgfpathmoveto{\pgfqpoint{1.882056in}{1.461089in}}%
\pgfpathlineto{\pgfqpoint{1.872186in}{1.397839in}}%
\pgfpathmoveto{\pgfqpoint{1.882056in}{1.461089in}}%
\pgfpathlineto{\pgfqpoint{1.923863in}{1.445776in}}%
\pgfpathmoveto{\pgfqpoint{1.882056in}{1.461089in}}%
\pgfpathlineto{\pgfqpoint{1.831844in}{1.452132in}}%
\pgfpathmoveto{\pgfqpoint{1.526822in}{1.465380in}}%
\pgfpathlineto{\pgfqpoint{1.543874in}{1.511040in}}%
\pgfpathmoveto{\pgfqpoint{1.526822in}{1.465380in}}%
\pgfpathlineto{\pgfqpoint{1.491725in}{1.509289in}}%
\pgfpathmoveto{\pgfqpoint{1.526822in}{1.465380in}}%
\pgfpathlineto{\pgfqpoint{1.529376in}{1.414978in}}%
\pgfpathmoveto{\pgfqpoint{1.526822in}{1.465380in}}%
\pgfpathlineto{\pgfqpoint{1.576824in}{1.457041in}}%
\pgfpathmoveto{\pgfqpoint{1.526822in}{1.465380in}}%
\pgfpathlineto{\pgfqpoint{1.485231in}{1.464971in}}%
\pgfpathmoveto{\pgfqpoint{1.628624in}{1.467848in}}%
\pgfpathlineto{\pgfqpoint{1.652928in}{1.512493in}}%
\pgfpathmoveto{\pgfqpoint{1.628624in}{1.467848in}}%
\pgfpathlineto{\pgfqpoint{1.598228in}{1.511788in}}%
\pgfpathmoveto{\pgfqpoint{1.628624in}{1.467848in}}%
\pgfpathlineto{\pgfqpoint{1.633424in}{1.370676in}}%
\pgfpathmoveto{\pgfqpoint{1.628624in}{1.467848in}}%
\pgfpathlineto{\pgfqpoint{1.680068in}{1.437866in}}%
\pgfpathmoveto{\pgfqpoint{1.628624in}{1.467848in}}%
\pgfpathlineto{\pgfqpoint{1.576824in}{1.457041in}}%
\pgfpathmoveto{\pgfqpoint{2.071998in}{1.422956in}}%
\pgfpathlineto{\pgfqpoint{2.113663in}{1.459833in}}%
\pgfpathmoveto{\pgfqpoint{2.071998in}{1.422956in}}%
\pgfpathlineto{\pgfqpoint{2.051519in}{1.486651in}}%
\pgfpathmoveto{\pgfqpoint{2.071998in}{1.422956in}}%
\pgfpathlineto{\pgfqpoint{2.046525in}{1.339679in}}%
\pgfpathmoveto{\pgfqpoint{2.071998in}{1.422956in}}%
\pgfpathlineto{\pgfqpoint{2.018592in}{1.426931in}}%
\pgfpathmoveto{\pgfqpoint{2.071998in}{1.422956in}}%
\pgfpathlineto{\pgfqpoint{2.117014in}{1.385442in}}%
\pgfpathmoveto{\pgfqpoint{1.826193in}{1.294555in}}%
\pgfpathlineto{\pgfqpoint{1.717703in}{1.244847in}}%
\pgfpathmoveto{\pgfqpoint{1.826193in}{1.294555in}}%
\pgfpathlineto{\pgfqpoint{1.930696in}{1.291141in}}%
\pgfpathmoveto{\pgfqpoint{1.826193in}{1.294555in}}%
\pgfpathlineto{\pgfqpoint{1.772283in}{1.375785in}}%
\pgfpathmoveto{\pgfqpoint{1.826193in}{1.294555in}}%
\pgfpathlineto{\pgfqpoint{1.872186in}{1.397839in}}%
\pgfpathmoveto{\pgfqpoint{1.846772in}{1.180555in}}%
\pgfpathlineto{\pgfqpoint{1.947111in}{1.059017in}}%
\pgfpathmoveto{\pgfqpoint{1.846772in}{1.180555in}}%
\pgfpathlineto{\pgfqpoint{1.717703in}{1.244847in}}%
\pgfpathmoveto{\pgfqpoint{1.846772in}{1.180555in}}%
\pgfpathlineto{\pgfqpoint{1.930696in}{1.291141in}}%
\pgfpathmoveto{\pgfqpoint{1.846772in}{1.180555in}}%
\pgfpathlineto{\pgfqpoint{1.699769in}{1.040285in}}%
\pgfpathmoveto{\pgfqpoint{1.846772in}{1.180555in}}%
\pgfpathlineto{\pgfqpoint{1.826193in}{1.294555in}}%
\pgfpathmoveto{\pgfqpoint{1.392323in}{1.445557in}}%
\pgfpathlineto{\pgfqpoint{1.405209in}{1.503583in}}%
\pgfpathmoveto{\pgfqpoint{1.392323in}{1.445557in}}%
\pgfpathlineto{\pgfqpoint{1.376734in}{1.503884in}}%
\pgfpathmoveto{\pgfqpoint{1.392323in}{1.445557in}}%
\pgfpathlineto{\pgfqpoint{1.405118in}{1.365318in}}%
\pgfpathmoveto{\pgfqpoint{1.392323in}{1.445557in}}%
\pgfpathlineto{\pgfqpoint{1.439632in}{1.429170in}}%
\pgfpathmoveto{\pgfqpoint{1.392323in}{1.445557in}}%
\pgfpathlineto{\pgfqpoint{1.342506in}{1.425620in}}%
\pgfpathmoveto{\pgfqpoint{1.582585in}{1.180140in}}%
\pgfpathlineto{\pgfqpoint{1.485163in}{1.092347in}}%
\pgfpathmoveto{\pgfqpoint{1.582585in}{1.180140in}}%
\pgfpathlineto{\pgfqpoint{1.717703in}{1.244847in}}%
\pgfpathmoveto{\pgfqpoint{1.582585in}{1.180140in}}%
\pgfpathlineto{\pgfqpoint{1.518346in}{1.319445in}}%
\pgfpathmoveto{\pgfqpoint{1.582585in}{1.180140in}}%
\pgfpathlineto{\pgfqpoint{1.699769in}{1.040285in}}%
\pgfpathmoveto{\pgfqpoint{2.275752in}{1.287830in}}%
\pgfpathlineto{\pgfqpoint{2.348760in}{1.266085in}}%
\pgfpathmoveto{\pgfqpoint{2.275752in}{1.287830in}}%
\pgfpathlineto{\pgfqpoint{2.308288in}{1.311201in}}%
\pgfpathmoveto{\pgfqpoint{2.275752in}{1.287830in}}%
\pgfpathlineto{\pgfqpoint{2.264799in}{1.353578in}}%
\pgfpathmoveto{\pgfqpoint{2.275752in}{1.287830in}}%
\pgfpathlineto{\pgfqpoint{2.248615in}{1.186532in}}%
\pgfpathmoveto{\pgfqpoint{2.275752in}{1.287830in}}%
\pgfpathlineto{\pgfqpoint{2.207549in}{1.321314in}}%
\pgfpathmoveto{\pgfqpoint{1.151928in}{1.337930in}}%
\pgfpathlineto{\pgfqpoint{1.143388in}{1.391241in}}%
\pgfpathmoveto{\pgfqpoint{1.151928in}{1.337930in}}%
\pgfpathlineto{\pgfqpoint{1.096650in}{1.352363in}}%
\pgfpathmoveto{\pgfqpoint{1.151928in}{1.337930in}}%
\pgfpathlineto{\pgfqpoint{1.211021in}{1.296806in}}%
\pgfpathmoveto{\pgfqpoint{1.151928in}{1.337930in}}%
\pgfpathlineto{\pgfqpoint{1.117568in}{1.279757in}}%
\pgfpathmoveto{\pgfqpoint{2.004293in}{1.214355in}}%
\pgfpathlineto{\pgfqpoint{1.947111in}{1.059017in}}%
\pgfpathmoveto{\pgfqpoint{2.004293in}{1.214355in}}%
\pgfpathlineto{\pgfqpoint{1.930696in}{1.291141in}}%
\pgfpathmoveto{\pgfqpoint{2.004293in}{1.214355in}}%
\pgfpathlineto{\pgfqpoint{2.109121in}{1.140781in}}%
\pgfpathmoveto{\pgfqpoint{2.004293in}{1.214355in}}%
\pgfpathlineto{\pgfqpoint{2.046525in}{1.339679in}}%
\pgfpathmoveto{\pgfqpoint{2.004293in}{1.214355in}}%
\pgfpathlineto{\pgfqpoint{2.132875in}{1.281739in}}%
\pgfpathmoveto{\pgfqpoint{2.004293in}{1.214355in}}%
\pgfpathlineto{\pgfqpoint{1.846772in}{1.180555in}}%
\pgfpathmoveto{\pgfqpoint{1.428504in}{1.250641in}}%
\pgfpathlineto{\pgfqpoint{1.485163in}{1.092347in}}%
\pgfpathmoveto{\pgfqpoint{1.428504in}{1.250641in}}%
\pgfpathlineto{\pgfqpoint{1.518346in}{1.319445in}}%
\pgfpathmoveto{\pgfqpoint{1.428504in}{1.250641in}}%
\pgfpathlineto{\pgfqpoint{1.312364in}{1.203204in}}%
\pgfpathmoveto{\pgfqpoint{1.428504in}{1.250641in}}%
\pgfpathlineto{\pgfqpoint{1.405118in}{1.365318in}}%
\pgfpathmoveto{\pgfqpoint{1.428504in}{1.250641in}}%
\pgfpathlineto{\pgfqpoint{1.310242in}{1.352368in}}%
\pgfpathmoveto{\pgfqpoint{1.428504in}{1.250641in}}%
\pgfpathlineto{\pgfqpoint{1.582585in}{1.180140in}}%
\pgfpathmoveto{\pgfqpoint{2.336300in}{0.837412in}}%
\pgfpathlineto{\pgfqpoint{2.562680in}{0.635986in}}%
\pgfpathmoveto{\pgfqpoint{2.336300in}{0.837412in}}%
\pgfpathlineto{\pgfqpoint{2.122339in}{0.635986in}}%
\pgfpathmoveto{\pgfqpoint{2.336300in}{0.837412in}}%
\pgfpathlineto{\pgfqpoint{2.521416in}{0.942226in}}%
\pgfpathmoveto{\pgfqpoint{2.336300in}{0.837412in}}%
\pgfpathlineto{\pgfqpoint{2.137178in}{0.951746in}}%
\pgfpathmoveto{\pgfqpoint{2.336300in}{0.837412in}}%
\pgfpathlineto{\pgfqpoint{2.331373in}{1.018221in}}%
\pgfpathmoveto{\pgfqpoint{1.614291in}{1.273826in}}%
\pgfpathlineto{\pgfqpoint{1.717703in}{1.244847in}}%
\pgfpathmoveto{\pgfqpoint{1.614291in}{1.273826in}}%
\pgfpathlineto{\pgfqpoint{1.518346in}{1.319445in}}%
\pgfpathmoveto{\pgfqpoint{1.614291in}{1.273826in}}%
\pgfpathlineto{\pgfqpoint{1.633424in}{1.370676in}}%
\pgfpathmoveto{\pgfqpoint{1.614291in}{1.273826in}}%
\pgfpathlineto{\pgfqpoint{1.582585in}{1.180140in}}%
\pgfpathmoveto{\pgfqpoint{1.967957in}{1.448565in}}%
\pgfpathlineto{\pgfqpoint{1.983145in}{1.503946in}}%
\pgfpathmoveto{\pgfqpoint{1.967957in}{1.448565in}}%
\pgfpathlineto{\pgfqpoint{1.952999in}{1.503972in}}%
\pgfpathmoveto{\pgfqpoint{1.967957in}{1.448565in}}%
\pgfpathlineto{\pgfqpoint{1.961140in}{1.388553in}}%
\pgfpathmoveto{\pgfqpoint{1.967957in}{1.448565in}}%
\pgfpathlineto{\pgfqpoint{1.923863in}{1.445776in}}%
\pgfpathmoveto{\pgfqpoint{1.967957in}{1.448565in}}%
\pgfpathlineto{\pgfqpoint{2.018592in}{1.426931in}}%
\pgfpathmoveto{\pgfqpoint{2.168789in}{1.362073in}}%
\pgfpathlineto{\pgfqpoint{2.217995in}{1.392469in}}%
\pgfpathmoveto{\pgfqpoint{2.168789in}{1.362073in}}%
\pgfpathlineto{\pgfqpoint{2.167831in}{1.428076in}}%
\pgfpathmoveto{\pgfqpoint{2.168789in}{1.362073in}}%
\pgfpathlineto{\pgfqpoint{2.132875in}{1.281739in}}%
\pgfpathmoveto{\pgfqpoint{2.168789in}{1.362073in}}%
\pgfpathlineto{\pgfqpoint{2.117014in}{1.385442in}}%
\pgfpathmoveto{\pgfqpoint{2.168789in}{1.362073in}}%
\pgfpathlineto{\pgfqpoint{2.207549in}{1.321314in}}%
\pgfpathmoveto{\pgfqpoint{1.289035in}{1.422535in}}%
\pgfpathlineto{\pgfqpoint{1.309162in}{1.485848in}}%
\pgfpathmoveto{\pgfqpoint{1.289035in}{1.422535in}}%
\pgfpathlineto{\pgfqpoint{1.247330in}{1.458637in}}%
\pgfpathmoveto{\pgfqpoint{1.289035in}{1.422535in}}%
\pgfpathlineto{\pgfqpoint{1.310242in}{1.352368in}}%
\pgfpathmoveto{\pgfqpoint{1.289035in}{1.422535in}}%
\pgfpathlineto{\pgfqpoint{1.342506in}{1.425620in}}%
\pgfpathmoveto{\pgfqpoint{1.289035in}{1.422535in}}%
\pgfpathlineto{\pgfqpoint{1.244082in}{1.384740in}}%
\pgfpathmoveto{\pgfqpoint{1.865642in}{0.861703in}}%
\pgfpathlineto{\pgfqpoint{1.681998in}{0.635986in}}%
\pgfpathmoveto{\pgfqpoint{1.865642in}{0.861703in}}%
\pgfpathlineto{\pgfqpoint{2.122339in}{0.635986in}}%
\pgfpathmoveto{\pgfqpoint{1.865642in}{0.861703in}}%
\pgfpathlineto{\pgfqpoint{1.947111in}{1.059017in}}%
\pgfpathmoveto{\pgfqpoint{1.865642in}{0.861703in}}%
\pgfpathlineto{\pgfqpoint{2.137178in}{0.951746in}}%
\pgfpathmoveto{\pgfqpoint{1.865642in}{0.861703in}}%
\pgfpathlineto{\pgfqpoint{1.699769in}{1.040285in}}%
\pgfpathmoveto{\pgfqpoint{1.543673in}{0.874634in}}%
\pgfpathlineto{\pgfqpoint{1.681998in}{0.635986in}}%
\pgfpathmoveto{\pgfqpoint{1.543673in}{0.874634in}}%
\pgfpathlineto{\pgfqpoint{1.241657in}{0.635986in}}%
\pgfpathmoveto{\pgfqpoint{1.543673in}{0.874634in}}%
\pgfpathlineto{\pgfqpoint{1.485163in}{1.092347in}}%
\pgfpathmoveto{\pgfqpoint{1.543673in}{0.874634in}}%
\pgfpathlineto{\pgfqpoint{1.699769in}{1.040285in}}%
\pgfpathmoveto{\pgfqpoint{1.543673in}{0.874634in}}%
\pgfpathlineto{\pgfqpoint{1.299964in}{0.984994in}}%
\pgfpathmoveto{\pgfqpoint{1.543673in}{0.874634in}}%
\pgfpathlineto{\pgfqpoint{1.865642in}{0.861703in}}%
\pgfpathmoveto{\pgfqpoint{1.190785in}{1.367540in}}%
\pgfpathlineto{\pgfqpoint{1.193492in}{1.426806in}}%
\pgfpathmoveto{\pgfqpoint{1.190785in}{1.367540in}}%
\pgfpathlineto{\pgfqpoint{1.143388in}{1.391241in}}%
\pgfpathmoveto{\pgfqpoint{1.190785in}{1.367540in}}%
\pgfpathlineto{\pgfqpoint{1.211021in}{1.296806in}}%
\pgfpathmoveto{\pgfqpoint{1.190785in}{1.367540in}}%
\pgfpathlineto{\pgfqpoint{1.244082in}{1.384740in}}%
\pgfpathmoveto{\pgfqpoint{1.190785in}{1.367540in}}%
\pgfpathlineto{\pgfqpoint{1.151928in}{1.337930in}}%
\pgfpathlineto{\pgfqpoint{1.151928in}{1.337930in}}%
\pgfusepath{stroke}%
\end{pgfscope}%
\begin{pgfscope}%
\pgfpathrectangle{\pgfqpoint{0.713248in}{0.548486in}}{\pgfqpoint{1.937500in}{1.925000in}}%
\pgfusepath{clip}%
\pgfsetrectcap%
\pgfsetroundjoin%
\pgfsetlinewidth{0.501875pt}%
\definecolor{currentstroke}{rgb}{0.835294,0.321569,0.035294}%
\pgfsetstrokecolor{currentstroke}%
\pgfsetdash{}{0pt}%
\pgfpathmoveto{\pgfqpoint{1.241657in}{2.385986in}}%
\pgfpathlineto{\pgfqpoint{1.681998in}{2.385986in}}%
\pgfpathmoveto{\pgfqpoint{1.241657in}{2.385986in}}%
\pgfpathlineto{\pgfqpoint{0.801316in}{2.385986in}}%
\pgfpathmoveto{\pgfqpoint{0.838067in}{2.086308in}}%
\pgfpathlineto{\pgfqpoint{0.801316in}{2.385986in}}%
\pgfpathmoveto{\pgfqpoint{0.838067in}{2.086308in}}%
\pgfpathlineto{\pgfqpoint{0.964625in}{1.815335in}}%
\pgfpathmoveto{\pgfqpoint{0.999217in}{1.768095in}}%
\pgfpathlineto{\pgfqpoint{0.964625in}{1.815335in}}%
\pgfpathmoveto{\pgfqpoint{1.036506in}{1.723397in}}%
\pgfpathlineto{\pgfqpoint{0.999217in}{1.768095in}}%
\pgfpathmoveto{\pgfqpoint{1.076110in}{1.681814in}}%
\pgfpathlineto{\pgfqpoint{1.036506in}{1.723397in}}%
\pgfpathmoveto{\pgfqpoint{1.117516in}{1.643378in}}%
\pgfpathlineto{\pgfqpoint{1.076110in}{1.681814in}}%
\pgfpathmoveto{\pgfqpoint{1.160025in}{1.608118in}}%
\pgfpathlineto{\pgfqpoint{1.117516in}{1.643378in}}%
\pgfpathmoveto{\pgfqpoint{1.202817in}{1.575342in}}%
\pgfpathlineto{\pgfqpoint{1.160025in}{1.608118in}}%
\pgfpathmoveto{\pgfqpoint{1.240507in}{1.538265in}}%
\pgfpathlineto{\pgfqpoint{1.202817in}{1.575342in}}%
\pgfpathmoveto{\pgfqpoint{1.300272in}{1.515004in}}%
\pgfpathlineto{\pgfqpoint{1.240507in}{1.538265in}}%
\pgfpathmoveto{\pgfqpoint{1.370290in}{1.501084in}}%
\pgfpathlineto{\pgfqpoint{1.300272in}{1.515004in}}%
\pgfpathmoveto{\pgfqpoint{1.445134in}{1.505570in}}%
\pgfpathlineto{\pgfqpoint{1.370290in}{1.501084in}}%
\pgfpathmoveto{\pgfqpoint{1.500803in}{1.509839in}}%
\pgfpathlineto{\pgfqpoint{1.445134in}{1.505570in}}%
\pgfpathmoveto{\pgfqpoint{1.554263in}{1.511195in}}%
\pgfpathlineto{\pgfqpoint{1.500803in}{1.509839in}}%
\pgfpathmoveto{\pgfqpoint{1.604583in}{1.511878in}}%
\pgfpathlineto{\pgfqpoint{1.554263in}{1.511195in}}%
\pgfpathmoveto{\pgfqpoint{1.654680in}{1.512463in}}%
\pgfpathlineto{\pgfqpoint{1.604583in}{1.511878in}}%
\pgfpathmoveto{\pgfqpoint{1.705711in}{1.512370in}}%
\pgfpathlineto{\pgfqpoint{1.654680in}{1.512463in}}%
\pgfpathmoveto{\pgfqpoint{1.755549in}{1.512311in}}%
\pgfpathlineto{\pgfqpoint{1.705711in}{1.512370in}}%
\pgfpathmoveto{\pgfqpoint{1.806182in}{1.510188in}}%
\pgfpathlineto{\pgfqpoint{1.755549in}{1.512311in}}%
\pgfpathmoveto{\pgfqpoint{1.859253in}{1.507070in}}%
\pgfpathlineto{\pgfqpoint{1.806182in}{1.510188in}}%
\pgfpathmoveto{\pgfqpoint{1.914435in}{1.505534in}}%
\pgfpathlineto{\pgfqpoint{1.859253in}{1.507070in}}%
\pgfpathmoveto{\pgfqpoint{1.989459in}{1.501094in}}%
\pgfpathlineto{\pgfqpoint{1.914435in}{1.505534in}}%
\pgfpathmoveto{\pgfqpoint{2.060913in}{1.514754in}}%
\pgfpathlineto{\pgfqpoint{1.989459in}{1.501094in}}%
\pgfpathmoveto{\pgfqpoint{2.120993in}{1.538609in}}%
\pgfpathlineto{\pgfqpoint{2.060913in}{1.514754in}}%
\pgfpathmoveto{\pgfqpoint{2.158667in}{1.575501in}}%
\pgfpathlineto{\pgfqpoint{2.120993in}{1.538609in}}%
\pgfpathmoveto{\pgfqpoint{2.201537in}{1.608116in}}%
\pgfpathlineto{\pgfqpoint{2.158667in}{1.575501in}}%
\pgfpathmoveto{\pgfqpoint{2.244160in}{1.643131in}}%
\pgfpathlineto{\pgfqpoint{2.201537in}{1.608116in}}%
\pgfpathmoveto{\pgfqpoint{2.285883in}{1.681318in}}%
\pgfpathlineto{\pgfqpoint{2.244160in}{1.643131in}}%
\pgfpathmoveto{\pgfqpoint{2.325697in}{1.722940in}}%
\pgfpathlineto{\pgfqpoint{2.285883in}{1.681318in}}%
\pgfpathmoveto{\pgfqpoint{2.363300in}{1.767500in}}%
\pgfpathlineto{\pgfqpoint{2.398187in}{1.814680in}}%
\pgfpathmoveto{\pgfqpoint{2.363300in}{1.767500in}}%
\pgfpathlineto{\pgfqpoint{2.325697in}{1.722940in}}%
\pgfpathmoveto{\pgfqpoint{2.525086in}{2.085827in}}%
\pgfpathlineto{\pgfqpoint{2.398187in}{1.814680in}}%
\pgfpathmoveto{\pgfqpoint{2.525086in}{2.085827in}}%
\pgfpathlineto{\pgfqpoint{2.562680in}{2.385986in}}%
\pgfpathmoveto{\pgfqpoint{2.122339in}{2.385986in}}%
\pgfpathlineto{\pgfqpoint{1.681998in}{2.385986in}}%
\pgfpathmoveto{\pgfqpoint{2.122339in}{2.385986in}}%
\pgfpathlineto{\pgfqpoint{2.562680in}{2.385986in}}%
\pgfpathmoveto{\pgfqpoint{1.246291in}{1.882055in}}%
\pgfpathlineto{\pgfqpoint{1.412157in}{1.957652in}}%
\pgfpathmoveto{\pgfqpoint{2.059122in}{1.815568in}}%
\pgfpathlineto{\pgfqpoint{1.881026in}{1.921938in}}%
\pgfpathmoveto{\pgfqpoint{2.233298in}{1.839015in}}%
\pgfpathlineto{\pgfqpoint{2.398187in}{1.814680in}}%
\pgfpathmoveto{\pgfqpoint{2.233298in}{1.839015in}}%
\pgfpathlineto{\pgfqpoint{2.363300in}{1.767500in}}%
\pgfpathmoveto{\pgfqpoint{2.233298in}{1.839015in}}%
\pgfpathlineto{\pgfqpoint{2.059122in}{1.815568in}}%
\pgfpathmoveto{\pgfqpoint{1.730341in}{1.651176in}}%
\pgfpathlineto{\pgfqpoint{1.641771in}{1.770883in}}%
\pgfpathmoveto{\pgfqpoint{1.730341in}{1.651176in}}%
\pgfpathlineto{\pgfqpoint{1.855876in}{1.693142in}}%
\pgfpathmoveto{\pgfqpoint{1.581753in}{1.644628in}}%
\pgfpathlineto{\pgfqpoint{1.641771in}{1.770883in}}%
\pgfpathmoveto{\pgfqpoint{1.581753in}{1.644628in}}%
\pgfpathlineto{\pgfqpoint{1.730341in}{1.651176in}}%
\pgfpathmoveto{\pgfqpoint{1.980433in}{1.640696in}}%
\pgfpathlineto{\pgfqpoint{1.855876in}{1.693142in}}%
\pgfpathmoveto{\pgfqpoint{1.294228in}{1.677431in}}%
\pgfpathlineto{\pgfqpoint{1.414920in}{1.718398in}}%
\pgfpathmoveto{\pgfqpoint{1.103979in}{1.843581in}}%
\pgfpathlineto{\pgfqpoint{0.964625in}{1.815335in}}%
\pgfpathmoveto{\pgfqpoint{1.103979in}{1.843581in}}%
\pgfpathlineto{\pgfqpoint{0.999217in}{1.768095in}}%
\pgfpathmoveto{\pgfqpoint{1.103979in}{1.843581in}}%
\pgfpathlineto{\pgfqpoint{1.246291in}{1.882055in}}%
\pgfpathmoveto{\pgfqpoint{1.841085in}{1.603299in}}%
\pgfpathlineto{\pgfqpoint{1.855876in}{1.693142in}}%
\pgfpathmoveto{\pgfqpoint{1.841085in}{1.603299in}}%
\pgfpathlineto{\pgfqpoint{1.730341in}{1.651176in}}%
\pgfpathmoveto{\pgfqpoint{1.471027in}{1.614106in}}%
\pgfpathlineto{\pgfqpoint{1.414920in}{1.718398in}}%
\pgfpathmoveto{\pgfqpoint{1.471027in}{1.614106in}}%
\pgfpathlineto{\pgfqpoint{1.581753in}{1.644628in}}%
\pgfpathmoveto{\pgfqpoint{1.372151in}{1.616253in}}%
\pgfpathlineto{\pgfqpoint{1.414920in}{1.718398in}}%
\pgfpathmoveto{\pgfqpoint{1.372151in}{1.616253in}}%
\pgfpathlineto{\pgfqpoint{1.294228in}{1.677431in}}%
\pgfpathmoveto{\pgfqpoint{1.372151in}{1.616253in}}%
\pgfpathlineto{\pgfqpoint{1.471027in}{1.614106in}}%
\pgfpathmoveto{\pgfqpoint{2.073519in}{1.664995in}}%
\pgfpathlineto{\pgfqpoint{2.059122in}{1.815568in}}%
\pgfpathmoveto{\pgfqpoint{2.073519in}{1.664995in}}%
\pgfpathlineto{\pgfqpoint{1.980433in}{1.640696in}}%
\pgfpathmoveto{\pgfqpoint{2.167361in}{1.730742in}}%
\pgfpathlineto{\pgfqpoint{2.059122in}{1.815568in}}%
\pgfpathmoveto{\pgfqpoint{2.167361in}{1.730742in}}%
\pgfpathlineto{\pgfqpoint{2.233298in}{1.839015in}}%
\pgfpathmoveto{\pgfqpoint{2.167361in}{1.730742in}}%
\pgfpathlineto{\pgfqpoint{2.073519in}{1.664995in}}%
\pgfpathmoveto{\pgfqpoint{1.214058in}{1.744875in}}%
\pgfpathlineto{\pgfqpoint{1.246291in}{1.882055in}}%
\pgfpathmoveto{\pgfqpoint{1.214058in}{1.744875in}}%
\pgfpathlineto{\pgfqpoint{1.294228in}{1.677431in}}%
\pgfpathmoveto{\pgfqpoint{1.214058in}{1.744875in}}%
\pgfpathlineto{\pgfqpoint{1.103979in}{1.843581in}}%
\pgfpathmoveto{\pgfqpoint{1.680376in}{1.587286in}}%
\pgfpathlineto{\pgfqpoint{1.654680in}{1.512463in}}%
\pgfpathmoveto{\pgfqpoint{1.680376in}{1.587286in}}%
\pgfpathlineto{\pgfqpoint{1.705711in}{1.512370in}}%
\pgfpathmoveto{\pgfqpoint{1.680376in}{1.587286in}}%
\pgfpathlineto{\pgfqpoint{1.730341in}{1.651176in}}%
\pgfpathmoveto{\pgfqpoint{1.680376in}{1.587286in}}%
\pgfpathlineto{\pgfqpoint{1.581753in}{1.644628in}}%
\pgfpathmoveto{\pgfqpoint{1.224328in}{2.070748in}}%
\pgfpathlineto{\pgfqpoint{1.241657in}{2.385986in}}%
\pgfpathmoveto{\pgfqpoint{1.224328in}{2.070748in}}%
\pgfpathlineto{\pgfqpoint{1.412157in}{1.957652in}}%
\pgfpathmoveto{\pgfqpoint{1.224328in}{2.070748in}}%
\pgfpathlineto{\pgfqpoint{1.246291in}{1.882055in}}%
\pgfpathmoveto{\pgfqpoint{1.224328in}{2.070748in}}%
\pgfpathlineto{\pgfqpoint{1.103979in}{1.843581in}}%
\pgfpathmoveto{\pgfqpoint{1.947672in}{1.576247in}}%
\pgfpathlineto{\pgfqpoint{1.914435in}{1.505534in}}%
\pgfpathmoveto{\pgfqpoint{1.947672in}{1.576247in}}%
\pgfpathlineto{\pgfqpoint{1.989459in}{1.501094in}}%
\pgfpathmoveto{\pgfqpoint{1.947672in}{1.576247in}}%
\pgfpathlineto{\pgfqpoint{1.855876in}{1.693142in}}%
\pgfpathmoveto{\pgfqpoint{1.947672in}{1.576247in}}%
\pgfpathlineto{\pgfqpoint{1.980433in}{1.640696in}}%
\pgfpathmoveto{\pgfqpoint{1.947672in}{1.576247in}}%
\pgfpathlineto{\pgfqpoint{1.841085in}{1.603299in}}%
\pgfpathmoveto{\pgfqpoint{2.259834in}{1.751542in}}%
\pgfpathlineto{\pgfqpoint{2.285883in}{1.681318in}}%
\pgfpathmoveto{\pgfqpoint{2.259834in}{1.751542in}}%
\pgfpathlineto{\pgfqpoint{2.325697in}{1.722940in}}%
\pgfpathmoveto{\pgfqpoint{2.259834in}{1.751542in}}%
\pgfpathlineto{\pgfqpoint{2.363300in}{1.767500in}}%
\pgfpathmoveto{\pgfqpoint{2.259834in}{1.751542in}}%
\pgfpathlineto{\pgfqpoint{2.233298in}{1.839015in}}%
\pgfpathmoveto{\pgfqpoint{2.259834in}{1.751542in}}%
\pgfpathlineto{\pgfqpoint{2.167361in}{1.730742in}}%
\pgfpathmoveto{\pgfqpoint{1.577929in}{1.558113in}}%
\pgfpathlineto{\pgfqpoint{1.554263in}{1.511195in}}%
\pgfpathmoveto{\pgfqpoint{1.577929in}{1.558113in}}%
\pgfpathlineto{\pgfqpoint{1.604583in}{1.511878in}}%
\pgfpathmoveto{\pgfqpoint{1.577929in}{1.558113in}}%
\pgfpathlineto{\pgfqpoint{1.581753in}{1.644628in}}%
\pgfpathmoveto{\pgfqpoint{1.662590in}{1.974278in}}%
\pgfpathlineto{\pgfqpoint{1.881026in}{1.921938in}}%
\pgfpathmoveto{\pgfqpoint{1.662590in}{1.974278in}}%
\pgfpathlineto{\pgfqpoint{1.412157in}{1.957652in}}%
\pgfpathmoveto{\pgfqpoint{1.662590in}{1.974278in}}%
\pgfpathlineto{\pgfqpoint{1.641771in}{1.770883in}}%
\pgfpathmoveto{\pgfqpoint{1.784413in}{1.566699in}}%
\pgfpathlineto{\pgfqpoint{1.755549in}{1.512311in}}%
\pgfpathmoveto{\pgfqpoint{1.784413in}{1.566699in}}%
\pgfpathlineto{\pgfqpoint{1.806182in}{1.510188in}}%
\pgfpathmoveto{\pgfqpoint{1.784413in}{1.566699in}}%
\pgfpathlineto{\pgfqpoint{1.730341in}{1.651176in}}%
\pgfpathmoveto{\pgfqpoint{1.784413in}{1.566699in}}%
\pgfpathlineto{\pgfqpoint{1.841085in}{1.603299in}}%
\pgfpathmoveto{\pgfqpoint{1.026585in}{2.009941in}}%
\pgfpathlineto{\pgfqpoint{0.964625in}{1.815335in}}%
\pgfpathmoveto{\pgfqpoint{1.026585in}{2.009941in}}%
\pgfpathlineto{\pgfqpoint{0.838067in}{2.086308in}}%
\pgfpathmoveto{\pgfqpoint{1.026585in}{2.009941in}}%
\pgfpathlineto{\pgfqpoint{1.103979in}{1.843581in}}%
\pgfpathmoveto{\pgfqpoint{1.026585in}{2.009941in}}%
\pgfpathlineto{\pgfqpoint{1.224328in}{2.070748in}}%
\pgfpathmoveto{\pgfqpoint{1.409508in}{1.559076in}}%
\pgfpathlineto{\pgfqpoint{1.370290in}{1.501084in}}%
\pgfpathmoveto{\pgfqpoint{1.409508in}{1.559076in}}%
\pgfpathlineto{\pgfqpoint{1.445134in}{1.505570in}}%
\pgfpathmoveto{\pgfqpoint{1.409508in}{1.559076in}}%
\pgfpathlineto{\pgfqpoint{1.471027in}{1.614106in}}%
\pgfpathmoveto{\pgfqpoint{1.409508in}{1.559076in}}%
\pgfpathlineto{\pgfqpoint{1.372151in}{1.616253in}}%
\pgfpathmoveto{\pgfqpoint{1.304271in}{1.586012in}}%
\pgfpathlineto{\pgfqpoint{1.240507in}{1.538265in}}%
\pgfpathmoveto{\pgfqpoint{1.304271in}{1.586012in}}%
\pgfpathlineto{\pgfqpoint{1.300272in}{1.515004in}}%
\pgfpathmoveto{\pgfqpoint{1.304271in}{1.586012in}}%
\pgfpathlineto{\pgfqpoint{1.294228in}{1.677431in}}%
\pgfpathmoveto{\pgfqpoint{1.304271in}{1.586012in}}%
\pgfpathlineto{\pgfqpoint{1.372151in}{1.616253in}}%
\pgfpathmoveto{\pgfqpoint{2.317614in}{2.080895in}}%
\pgfpathlineto{\pgfqpoint{2.398187in}{1.814680in}}%
\pgfpathmoveto{\pgfqpoint{2.317614in}{2.080895in}}%
\pgfpathlineto{\pgfqpoint{2.562680in}{2.385986in}}%
\pgfpathmoveto{\pgfqpoint{2.317614in}{2.080895in}}%
\pgfpathlineto{\pgfqpoint{2.525086in}{2.085827in}}%
\pgfpathmoveto{\pgfqpoint{2.317614in}{2.080895in}}%
\pgfpathlineto{\pgfqpoint{2.122339in}{2.385986in}}%
\pgfpathmoveto{\pgfqpoint{2.317614in}{2.080895in}}%
\pgfpathlineto{\pgfqpoint{2.233298in}{1.839015in}}%
\pgfpathmoveto{\pgfqpoint{1.216895in}{1.644251in}}%
\pgfpathlineto{\pgfqpoint{1.160025in}{1.608118in}}%
\pgfpathmoveto{\pgfqpoint{1.216895in}{1.644251in}}%
\pgfpathlineto{\pgfqpoint{1.202817in}{1.575342in}}%
\pgfpathmoveto{\pgfqpoint{1.216895in}{1.644251in}}%
\pgfpathlineto{\pgfqpoint{1.294228in}{1.677431in}}%
\pgfpathmoveto{\pgfqpoint{1.216895in}{1.644251in}}%
\pgfpathlineto{\pgfqpoint{1.214058in}{1.744875in}}%
\pgfpathmoveto{\pgfqpoint{1.135348in}{1.711658in}}%
\pgfpathlineto{\pgfqpoint{1.076110in}{1.681814in}}%
\pgfpathmoveto{\pgfqpoint{1.135348in}{1.711658in}}%
\pgfpathlineto{\pgfqpoint{1.117516in}{1.643378in}}%
\pgfpathmoveto{\pgfqpoint{1.135348in}{1.711658in}}%
\pgfpathlineto{\pgfqpoint{1.103979in}{1.843581in}}%
\pgfpathmoveto{\pgfqpoint{1.135348in}{1.711658in}}%
\pgfpathlineto{\pgfqpoint{1.214058in}{1.744875in}}%
\pgfpathmoveto{\pgfqpoint{2.056838in}{1.586465in}}%
\pgfpathlineto{\pgfqpoint{2.060913in}{1.514754in}}%
\pgfpathmoveto{\pgfqpoint{2.056838in}{1.586465in}}%
\pgfpathlineto{\pgfqpoint{2.120993in}{1.538609in}}%
\pgfpathmoveto{\pgfqpoint{2.056838in}{1.586465in}}%
\pgfpathlineto{\pgfqpoint{1.980433in}{1.640696in}}%
\pgfpathmoveto{\pgfqpoint{2.056838in}{1.586465in}}%
\pgfpathlineto{\pgfqpoint{2.073519in}{1.664995in}}%
\pgfpathmoveto{\pgfqpoint{2.064944in}{2.035177in}}%
\pgfpathlineto{\pgfqpoint{2.122339in}{2.385986in}}%
\pgfpathmoveto{\pgfqpoint{2.064944in}{2.035177in}}%
\pgfpathlineto{\pgfqpoint{1.881026in}{1.921938in}}%
\pgfpathmoveto{\pgfqpoint{2.064944in}{2.035177in}}%
\pgfpathlineto{\pgfqpoint{2.059122in}{1.815568in}}%
\pgfpathmoveto{\pgfqpoint{2.064944in}{2.035177in}}%
\pgfpathlineto{\pgfqpoint{2.233298in}{1.839015in}}%
\pgfpathmoveto{\pgfqpoint{2.064944in}{2.035177in}}%
\pgfpathlineto{\pgfqpoint{2.317614in}{2.080895in}}%
\pgfpathmoveto{\pgfqpoint{2.145015in}{1.643856in}}%
\pgfpathlineto{\pgfqpoint{2.158667in}{1.575501in}}%
\pgfpathmoveto{\pgfqpoint{2.145015in}{1.643856in}}%
\pgfpathlineto{\pgfqpoint{2.201537in}{1.608116in}}%
\pgfpathmoveto{\pgfqpoint{2.145015in}{1.643856in}}%
\pgfpathlineto{\pgfqpoint{2.073519in}{1.664995in}}%
\pgfpathmoveto{\pgfqpoint{2.145015in}{1.643856in}}%
\pgfpathlineto{\pgfqpoint{2.167361in}{1.730742in}}%
\pgfpathmoveto{\pgfqpoint{1.879947in}{1.550962in}}%
\pgfpathlineto{\pgfqpoint{1.859253in}{1.507070in}}%
\pgfpathmoveto{\pgfqpoint{1.879947in}{1.550962in}}%
\pgfpathlineto{\pgfqpoint{1.914435in}{1.505534in}}%
\pgfpathmoveto{\pgfqpoint{1.879947in}{1.550962in}}%
\pgfpathlineto{\pgfqpoint{1.841085in}{1.603299in}}%
\pgfpathmoveto{\pgfqpoint{1.879947in}{1.550962in}}%
\pgfpathlineto{\pgfqpoint{1.947672in}{1.576247in}}%
\pgfpathmoveto{\pgfqpoint{1.629639in}{1.558330in}}%
\pgfpathlineto{\pgfqpoint{1.604583in}{1.511878in}}%
\pgfpathmoveto{\pgfqpoint{1.629639in}{1.558330in}}%
\pgfpathlineto{\pgfqpoint{1.654680in}{1.512463in}}%
\pgfpathmoveto{\pgfqpoint{1.629639in}{1.558330in}}%
\pgfpathlineto{\pgfqpoint{1.581753in}{1.644628in}}%
\pgfpathmoveto{\pgfqpoint{1.629639in}{1.558330in}}%
\pgfpathlineto{\pgfqpoint{1.680376in}{1.587286in}}%
\pgfpathmoveto{\pgfqpoint{1.629639in}{1.558330in}}%
\pgfpathlineto{\pgfqpoint{1.577929in}{1.558113in}}%
\pgfpathmoveto{\pgfqpoint{1.524120in}{1.567412in}}%
\pgfpathlineto{\pgfqpoint{1.500803in}{1.509839in}}%
\pgfpathmoveto{\pgfqpoint{1.524120in}{1.567412in}}%
\pgfpathlineto{\pgfqpoint{1.554263in}{1.511195in}}%
\pgfpathmoveto{\pgfqpoint{1.524120in}{1.567412in}}%
\pgfpathlineto{\pgfqpoint{1.581753in}{1.644628in}}%
\pgfpathmoveto{\pgfqpoint{1.524120in}{1.567412in}}%
\pgfpathlineto{\pgfqpoint{1.471027in}{1.614106in}}%
\pgfpathmoveto{\pgfqpoint{1.524120in}{1.567412in}}%
\pgfpathlineto{\pgfqpoint{1.577929in}{1.558113in}}%
\pgfpathmoveto{\pgfqpoint{1.469274in}{1.554002in}}%
\pgfpathlineto{\pgfqpoint{1.445134in}{1.505570in}}%
\pgfpathmoveto{\pgfqpoint{1.469274in}{1.554002in}}%
\pgfpathlineto{\pgfqpoint{1.500803in}{1.509839in}}%
\pgfpathmoveto{\pgfqpoint{1.469274in}{1.554002in}}%
\pgfpathlineto{\pgfqpoint{1.471027in}{1.614106in}}%
\pgfpathmoveto{\pgfqpoint{1.469274in}{1.554002in}}%
\pgfpathlineto{\pgfqpoint{1.409508in}{1.559076in}}%
\pgfpathmoveto{\pgfqpoint{1.469274in}{1.554002in}}%
\pgfpathlineto{\pgfqpoint{1.524120in}{1.567412in}}%
\pgfpathmoveto{\pgfqpoint{1.834677in}{1.554197in}}%
\pgfpathlineto{\pgfqpoint{1.806182in}{1.510188in}}%
\pgfpathmoveto{\pgfqpoint{1.834677in}{1.554197in}}%
\pgfpathlineto{\pgfqpoint{1.859253in}{1.507070in}}%
\pgfpathmoveto{\pgfqpoint{1.834677in}{1.554197in}}%
\pgfpathlineto{\pgfqpoint{1.841085in}{1.603299in}}%
\pgfpathmoveto{\pgfqpoint{1.834677in}{1.554197in}}%
\pgfpathlineto{\pgfqpoint{1.784413in}{1.566699in}}%
\pgfpathmoveto{\pgfqpoint{1.834677in}{1.554197in}}%
\pgfpathlineto{\pgfqpoint{1.879947in}{1.550962in}}%
\pgfpathmoveto{\pgfqpoint{1.730866in}{1.557166in}}%
\pgfpathlineto{\pgfqpoint{1.705711in}{1.512370in}}%
\pgfpathmoveto{\pgfqpoint{1.730866in}{1.557166in}}%
\pgfpathlineto{\pgfqpoint{1.755549in}{1.512311in}}%
\pgfpathmoveto{\pgfqpoint{1.730866in}{1.557166in}}%
\pgfpathlineto{\pgfqpoint{1.730341in}{1.651176in}}%
\pgfpathmoveto{\pgfqpoint{1.730866in}{1.557166in}}%
\pgfpathlineto{\pgfqpoint{1.680376in}{1.587286in}}%
\pgfpathmoveto{\pgfqpoint{1.730866in}{1.557166in}}%
\pgfpathlineto{\pgfqpoint{1.784413in}{1.566699in}}%
\pgfpathmoveto{\pgfqpoint{1.252707in}{1.601366in}}%
\pgfpathlineto{\pgfqpoint{1.202817in}{1.575342in}}%
\pgfpathmoveto{\pgfqpoint{1.252707in}{1.601366in}}%
\pgfpathlineto{\pgfqpoint{1.240507in}{1.538265in}}%
\pgfpathmoveto{\pgfqpoint{1.252707in}{1.601366in}}%
\pgfpathlineto{\pgfqpoint{1.294228in}{1.677431in}}%
\pgfpathmoveto{\pgfqpoint{1.252707in}{1.601366in}}%
\pgfpathlineto{\pgfqpoint{1.304271in}{1.586012in}}%
\pgfpathmoveto{\pgfqpoint{1.252707in}{1.601366in}}%
\pgfpathlineto{\pgfqpoint{1.216895in}{1.644251in}}%
\pgfpathmoveto{\pgfqpoint{1.524834in}{1.719478in}}%
\pgfpathlineto{\pgfqpoint{1.641771in}{1.770883in}}%
\pgfpathmoveto{\pgfqpoint{1.524834in}{1.719478in}}%
\pgfpathlineto{\pgfqpoint{1.414920in}{1.718398in}}%
\pgfpathmoveto{\pgfqpoint{1.524834in}{1.719478in}}%
\pgfpathlineto{\pgfqpoint{1.581753in}{1.644628in}}%
\pgfpathmoveto{\pgfqpoint{1.524834in}{1.719478in}}%
\pgfpathlineto{\pgfqpoint{1.471027in}{1.614106in}}%
\pgfpathmoveto{\pgfqpoint{1.508567in}{1.832373in}}%
\pgfpathlineto{\pgfqpoint{1.412157in}{1.957652in}}%
\pgfpathmoveto{\pgfqpoint{1.508567in}{1.832373in}}%
\pgfpathlineto{\pgfqpoint{1.641771in}{1.770883in}}%
\pgfpathmoveto{\pgfqpoint{1.508567in}{1.832373in}}%
\pgfpathlineto{\pgfqpoint{1.414920in}{1.718398in}}%
\pgfpathmoveto{\pgfqpoint{1.508567in}{1.832373in}}%
\pgfpathlineto{\pgfqpoint{1.662590in}{1.974278in}}%
\pgfpathmoveto{\pgfqpoint{1.508567in}{1.832373in}}%
\pgfpathlineto{\pgfqpoint{1.524834in}{1.719478in}}%
\pgfpathmoveto{\pgfqpoint{2.010054in}{1.556817in}}%
\pgfpathlineto{\pgfqpoint{1.989459in}{1.501094in}}%
\pgfpathmoveto{\pgfqpoint{2.010054in}{1.556817in}}%
\pgfpathlineto{\pgfqpoint{2.060913in}{1.514754in}}%
\pgfpathmoveto{\pgfqpoint{2.010054in}{1.556817in}}%
\pgfpathlineto{\pgfqpoint{1.980433in}{1.640696in}}%
\pgfpathmoveto{\pgfqpoint{2.010054in}{1.556817in}}%
\pgfpathlineto{\pgfqpoint{1.947672in}{1.576247in}}%
\pgfpathmoveto{\pgfqpoint{2.010054in}{1.556817in}}%
\pgfpathlineto{\pgfqpoint{2.056838in}{1.586465in}}%
\pgfpathmoveto{\pgfqpoint{1.783457in}{1.833609in}}%
\pgfpathlineto{\pgfqpoint{1.881026in}{1.921938in}}%
\pgfpathmoveto{\pgfqpoint{1.783457in}{1.833609in}}%
\pgfpathlineto{\pgfqpoint{1.641771in}{1.770883in}}%
\pgfpathmoveto{\pgfqpoint{1.783457in}{1.833609in}}%
\pgfpathlineto{\pgfqpoint{1.855876in}{1.693142in}}%
\pgfpathmoveto{\pgfqpoint{1.783457in}{1.833609in}}%
\pgfpathlineto{\pgfqpoint{1.662590in}{1.974278in}}%
\pgfpathmoveto{\pgfqpoint{1.070178in}{1.745683in}}%
\pgfpathlineto{\pgfqpoint{0.999217in}{1.768095in}}%
\pgfpathmoveto{\pgfqpoint{1.070178in}{1.745683in}}%
\pgfpathlineto{\pgfqpoint{1.036506in}{1.723397in}}%
\pgfpathmoveto{\pgfqpoint{1.070178in}{1.745683in}}%
\pgfpathlineto{\pgfqpoint{1.076110in}{1.681814in}}%
\pgfpathmoveto{\pgfqpoint{1.070178in}{1.745683in}}%
\pgfpathlineto{\pgfqpoint{1.103979in}{1.843581in}}%
\pgfpathmoveto{\pgfqpoint{1.070178in}{1.745683in}}%
\pgfpathlineto{\pgfqpoint{1.135348in}{1.711658in}}%
\pgfpathmoveto{\pgfqpoint{2.230510in}{1.694361in}}%
\pgfpathlineto{\pgfqpoint{2.244160in}{1.643131in}}%
\pgfpathmoveto{\pgfqpoint{2.230510in}{1.694361in}}%
\pgfpathlineto{\pgfqpoint{2.285883in}{1.681318in}}%
\pgfpathmoveto{\pgfqpoint{2.230510in}{1.694361in}}%
\pgfpathlineto{\pgfqpoint{2.167361in}{1.730742in}}%
\pgfpathmoveto{\pgfqpoint{2.230510in}{1.694361in}}%
\pgfpathlineto{\pgfqpoint{2.259834in}{1.751542in}}%
\pgfpathmoveto{\pgfqpoint{1.347654in}{1.801359in}}%
\pgfpathlineto{\pgfqpoint{1.412157in}{1.957652in}}%
\pgfpathmoveto{\pgfqpoint{1.347654in}{1.801359in}}%
\pgfpathlineto{\pgfqpoint{1.414920in}{1.718398in}}%
\pgfpathmoveto{\pgfqpoint{1.347654in}{1.801359in}}%
\pgfpathlineto{\pgfqpoint{1.246291in}{1.882055in}}%
\pgfpathmoveto{\pgfqpoint{1.347654in}{1.801359in}}%
\pgfpathlineto{\pgfqpoint{1.294228in}{1.677431in}}%
\pgfpathmoveto{\pgfqpoint{1.347654in}{1.801359in}}%
\pgfpathlineto{\pgfqpoint{1.214058in}{1.744875in}}%
\pgfpathmoveto{\pgfqpoint{1.347654in}{1.801359in}}%
\pgfpathlineto{\pgfqpoint{1.508567in}{1.832373in}}%
\pgfpathmoveto{\pgfqpoint{1.945551in}{1.760362in}}%
\pgfpathlineto{\pgfqpoint{1.881026in}{1.921938in}}%
\pgfpathmoveto{\pgfqpoint{1.945551in}{1.760362in}}%
\pgfpathlineto{\pgfqpoint{1.855876in}{1.693142in}}%
\pgfpathmoveto{\pgfqpoint{1.945551in}{1.760362in}}%
\pgfpathlineto{\pgfqpoint{2.059122in}{1.815568in}}%
\pgfpathmoveto{\pgfqpoint{1.945551in}{1.760362in}}%
\pgfpathlineto{\pgfqpoint{1.980433in}{1.640696in}}%
\pgfpathmoveto{\pgfqpoint{1.945551in}{1.760362in}}%
\pgfpathlineto{\pgfqpoint{2.073519in}{1.664995in}}%
\pgfpathmoveto{\pgfqpoint{1.945551in}{1.760362in}}%
\pgfpathlineto{\pgfqpoint{1.783457in}{1.833609in}}%
\pgfpathmoveto{\pgfqpoint{1.026016in}{2.187349in}}%
\pgfpathlineto{\pgfqpoint{0.801316in}{2.385986in}}%
\pgfpathmoveto{\pgfqpoint{1.026016in}{2.187349in}}%
\pgfpathlineto{\pgfqpoint{1.241657in}{2.385986in}}%
\pgfpathmoveto{\pgfqpoint{1.026016in}{2.187349in}}%
\pgfpathlineto{\pgfqpoint{0.838067in}{2.086308in}}%
\pgfpathmoveto{\pgfqpoint{1.026016in}{2.187349in}}%
\pgfpathlineto{\pgfqpoint{1.224328in}{2.070748in}}%
\pgfpathmoveto{\pgfqpoint{1.026016in}{2.187349in}}%
\pgfpathlineto{\pgfqpoint{1.026585in}{2.009941in}}%
\pgfpathmoveto{\pgfqpoint{1.752297in}{1.742529in}}%
\pgfpathlineto{\pgfqpoint{1.641771in}{1.770883in}}%
\pgfpathmoveto{\pgfqpoint{1.752297in}{1.742529in}}%
\pgfpathlineto{\pgfqpoint{1.855876in}{1.693142in}}%
\pgfpathmoveto{\pgfqpoint{1.752297in}{1.742529in}}%
\pgfpathlineto{\pgfqpoint{1.730341in}{1.651176in}}%
\pgfpathmoveto{\pgfqpoint{1.752297in}{1.742529in}}%
\pgfpathlineto{\pgfqpoint{1.783457in}{1.833609in}}%
\pgfpathmoveto{\pgfqpoint{1.350247in}{1.554439in}}%
\pgfpathlineto{\pgfqpoint{1.300272in}{1.515004in}}%
\pgfpathmoveto{\pgfqpoint{1.350247in}{1.554439in}}%
\pgfpathlineto{\pgfqpoint{1.370290in}{1.501084in}}%
\pgfpathmoveto{\pgfqpoint{1.350247in}{1.554439in}}%
\pgfpathlineto{\pgfqpoint{1.372151in}{1.616253in}}%
\pgfpathmoveto{\pgfqpoint{1.350247in}{1.554439in}}%
\pgfpathlineto{\pgfqpoint{1.409508in}{1.559076in}}%
\pgfpathmoveto{\pgfqpoint{1.350247in}{1.554439in}}%
\pgfpathlineto{\pgfqpoint{1.304271in}{1.586012in}}%
\pgfpathmoveto{\pgfqpoint{1.168993in}{1.670609in}}%
\pgfpathlineto{\pgfqpoint{1.117516in}{1.643378in}}%
\pgfpathmoveto{\pgfqpoint{1.168993in}{1.670609in}}%
\pgfpathlineto{\pgfqpoint{1.160025in}{1.608118in}}%
\pgfpathmoveto{\pgfqpoint{1.168993in}{1.670609in}}%
\pgfpathlineto{\pgfqpoint{1.214058in}{1.744875in}}%
\pgfpathmoveto{\pgfqpoint{1.168993in}{1.670609in}}%
\pgfpathlineto{\pgfqpoint{1.216895in}{1.644251in}}%
\pgfpathmoveto{\pgfqpoint{1.168993in}{1.670609in}}%
\pgfpathlineto{\pgfqpoint{1.135348in}{1.711658in}}%
\pgfpathmoveto{\pgfqpoint{2.108936in}{1.601268in}}%
\pgfpathlineto{\pgfqpoint{2.120993in}{1.538609in}}%
\pgfpathmoveto{\pgfqpoint{2.108936in}{1.601268in}}%
\pgfpathlineto{\pgfqpoint{2.158667in}{1.575501in}}%
\pgfpathmoveto{\pgfqpoint{2.108936in}{1.601268in}}%
\pgfpathlineto{\pgfqpoint{2.073519in}{1.664995in}}%
\pgfpathmoveto{\pgfqpoint{2.108936in}{1.601268in}}%
\pgfpathlineto{\pgfqpoint{2.056838in}{1.586465in}}%
\pgfpathmoveto{\pgfqpoint{2.108936in}{1.601268in}}%
\pgfpathlineto{\pgfqpoint{2.145015in}{1.643856in}}%
\pgfpathmoveto{\pgfqpoint{1.497595in}{2.156717in}}%
\pgfpathlineto{\pgfqpoint{1.681998in}{2.385986in}}%
\pgfpathmoveto{\pgfqpoint{1.497595in}{2.156717in}}%
\pgfpathlineto{\pgfqpoint{1.241657in}{2.385986in}}%
\pgfpathmoveto{\pgfqpoint{1.497595in}{2.156717in}}%
\pgfpathlineto{\pgfqpoint{1.412157in}{1.957652in}}%
\pgfpathmoveto{\pgfqpoint{1.497595in}{2.156717in}}%
\pgfpathlineto{\pgfqpoint{1.224328in}{2.070748in}}%
\pgfpathmoveto{\pgfqpoint{1.497595in}{2.156717in}}%
\pgfpathlineto{\pgfqpoint{1.662590in}{1.974278in}}%
\pgfpathmoveto{\pgfqpoint{1.819707in}{2.143085in}}%
\pgfpathlineto{\pgfqpoint{1.681998in}{2.385986in}}%
\pgfpathmoveto{\pgfqpoint{1.819707in}{2.143085in}}%
\pgfpathlineto{\pgfqpoint{2.122339in}{2.385986in}}%
\pgfpathmoveto{\pgfqpoint{1.819707in}{2.143085in}}%
\pgfpathlineto{\pgfqpoint{1.881026in}{1.921938in}}%
\pgfpathmoveto{\pgfqpoint{1.819707in}{2.143085in}}%
\pgfpathlineto{\pgfqpoint{1.662590in}{1.974278in}}%
\pgfpathmoveto{\pgfqpoint{1.819707in}{2.143085in}}%
\pgfpathlineto{\pgfqpoint{2.064944in}{2.035177in}}%
\pgfpathmoveto{\pgfqpoint{1.819707in}{2.143085in}}%
\pgfpathlineto{\pgfqpoint{1.497595in}{2.156717in}}%
\pgfpathmoveto{\pgfqpoint{2.195373in}{1.664315in}}%
\pgfpathlineto{\pgfqpoint{2.201537in}{1.608116in}}%
\pgfpathmoveto{\pgfqpoint{2.195373in}{1.664315in}}%
\pgfpathlineto{\pgfqpoint{2.244160in}{1.643131in}}%
\pgfpathmoveto{\pgfqpoint{2.195373in}{1.664315in}}%
\pgfpathlineto{\pgfqpoint{2.167361in}{1.730742in}}%
\pgfpathmoveto{\pgfqpoint{2.195373in}{1.664315in}}%
\pgfpathlineto{\pgfqpoint{2.145015in}{1.643856in}}%
\pgfpathmoveto{\pgfqpoint{2.195373in}{1.664315in}}%
\pgfpathlineto{\pgfqpoint{2.230510in}{1.694361in}}%
\pgfpathlineto{\pgfqpoint{2.230510in}{1.694361in}}%
\pgfusepath{stroke}%
\end{pgfscope}%
\begin{pgfscope}%
\pgfsetbuttcap%
\pgfsetroundjoin%
\definecolor{currentfill}{rgb}{0.000000,0.000000,0.000000}%
\pgfsetfillcolor{currentfill}%
\pgfsetlinewidth{0.803000pt}%
\definecolor{currentstroke}{rgb}{0.000000,0.000000,0.000000}%
\pgfsetstrokecolor{currentstroke}%
\pgfsetdash{}{0pt}%
\pgfsys@defobject{currentmarker}{\pgfqpoint{0.000000in}{-0.048611in}}{\pgfqpoint{0.000000in}{0.000000in}}{%
\pgfpathmoveto{\pgfqpoint{0.000000in}{0.000000in}}%
\pgfpathlineto{\pgfqpoint{0.000000in}{-0.048611in}}%
\pgfusepath{stroke,fill}%
}%
\begin{pgfscope}%
\pgfsys@transformshift{0.801316in}{0.548486in}%
\pgfsys@useobject{currentmarker}{}%
\end{pgfscope}%
\end{pgfscope}%
\begin{pgfscope}%
\definecolor{textcolor}{rgb}{0.000000,0.000000,0.000000}%
\pgfsetstrokecolor{textcolor}%
\pgfsetfillcolor{textcolor}%
\pgftext[x=0.801316in,y=0.451264in,,top]{\color{textcolor}\rmfamily\fontsize{11.000000}{13.200000}\selectfont \(\displaystyle {\ensuremath{-}0.5}\)}%
\end{pgfscope}%
\begin{pgfscope}%
\pgfsetbuttcap%
\pgfsetroundjoin%
\definecolor{currentfill}{rgb}{0.000000,0.000000,0.000000}%
\pgfsetfillcolor{currentfill}%
\pgfsetlinewidth{0.803000pt}%
\definecolor{currentstroke}{rgb}{0.000000,0.000000,0.000000}%
\pgfsetstrokecolor{currentstroke}%
\pgfsetdash{}{0pt}%
\pgfsys@defobject{currentmarker}{\pgfqpoint{0.000000in}{-0.048611in}}{\pgfqpoint{0.000000in}{0.000000in}}{%
\pgfpathmoveto{\pgfqpoint{0.000000in}{0.000000in}}%
\pgfpathlineto{\pgfqpoint{0.000000in}{-0.048611in}}%
\pgfusepath{stroke,fill}%
}%
\begin{pgfscope}%
\pgfsys@transformshift{1.681998in}{0.548486in}%
\pgfsys@useobject{currentmarker}{}%
\end{pgfscope}%
\end{pgfscope}%
\begin{pgfscope}%
\definecolor{textcolor}{rgb}{0.000000,0.000000,0.000000}%
\pgfsetstrokecolor{textcolor}%
\pgfsetfillcolor{textcolor}%
\pgftext[x=1.681998in,y=0.451264in,,top]{\color{textcolor}\rmfamily\fontsize{11.000000}{13.200000}\selectfont \(\displaystyle {0.0}\)}%
\end{pgfscope}%
\begin{pgfscope}%
\pgfsetbuttcap%
\pgfsetroundjoin%
\definecolor{currentfill}{rgb}{0.000000,0.000000,0.000000}%
\pgfsetfillcolor{currentfill}%
\pgfsetlinewidth{0.803000pt}%
\definecolor{currentstroke}{rgb}{0.000000,0.000000,0.000000}%
\pgfsetstrokecolor{currentstroke}%
\pgfsetdash{}{0pt}%
\pgfsys@defobject{currentmarker}{\pgfqpoint{0.000000in}{-0.048611in}}{\pgfqpoint{0.000000in}{0.000000in}}{%
\pgfpathmoveto{\pgfqpoint{0.000000in}{0.000000in}}%
\pgfpathlineto{\pgfqpoint{0.000000in}{-0.048611in}}%
\pgfusepath{stroke,fill}%
}%
\begin{pgfscope}%
\pgfsys@transformshift{2.562680in}{0.548486in}%
\pgfsys@useobject{currentmarker}{}%
\end{pgfscope}%
\end{pgfscope}%
\begin{pgfscope}%
\definecolor{textcolor}{rgb}{0.000000,0.000000,0.000000}%
\pgfsetstrokecolor{textcolor}%
\pgfsetfillcolor{textcolor}%
\pgftext[x=2.562680in,y=0.451264in,,top]{\color{textcolor}\rmfamily\fontsize{11.000000}{13.200000}\selectfont \(\displaystyle {0.5}\)}%
\end{pgfscope}%
\begin{pgfscope}%
\definecolor{textcolor}{rgb}{0.000000,0.000000,0.000000}%
\pgfsetstrokecolor{textcolor}%
\pgfsetfillcolor{textcolor}%
\pgftext[x=1.681998in,y=0.247854in,,top]{\color{textcolor}\rmfamily\fontsize{11.000000}{13.200000}\selectfont \(\displaystyle x\)}%
\end{pgfscope}%
\begin{pgfscope}%
\pgfsetbuttcap%
\pgfsetroundjoin%
\definecolor{currentfill}{rgb}{0.000000,0.000000,0.000000}%
\pgfsetfillcolor{currentfill}%
\pgfsetlinewidth{0.803000pt}%
\definecolor{currentstroke}{rgb}{0.000000,0.000000,0.000000}%
\pgfsetstrokecolor{currentstroke}%
\pgfsetdash{}{0pt}%
\pgfsys@defobject{currentmarker}{\pgfqpoint{-0.048611in}{0.000000in}}{\pgfqpoint{-0.000000in}{0.000000in}}{%
\pgfpathmoveto{\pgfqpoint{-0.000000in}{0.000000in}}%
\pgfpathlineto{\pgfqpoint{-0.048611in}{0.000000in}}%
\pgfusepath{stroke,fill}%
}%
\begin{pgfscope}%
\pgfsys@transformshift{0.713248in}{0.733208in}%
\pgfsys@useobject{currentmarker}{}%
\end{pgfscope}%
\end{pgfscope}%
\begin{pgfscope}%
\definecolor{textcolor}{rgb}{0.000000,0.000000,0.000000}%
\pgfsetstrokecolor{textcolor}%
\pgfsetfillcolor{textcolor}%
\pgftext[x=0.303410in, y=0.675170in, left, base]{\color{textcolor}\rmfamily\fontsize{11.000000}{13.200000}\selectfont \(\displaystyle {\ensuremath{-}0.4}\)}%
\end{pgfscope}%
\begin{pgfscope}%
\pgfsetbuttcap%
\pgfsetroundjoin%
\definecolor{currentfill}{rgb}{0.000000,0.000000,0.000000}%
\pgfsetfillcolor{currentfill}%
\pgfsetlinewidth{0.803000pt}%
\definecolor{currentstroke}{rgb}{0.000000,0.000000,0.000000}%
\pgfsetstrokecolor{currentstroke}%
\pgfsetdash{}{0pt}%
\pgfsys@defobject{currentmarker}{\pgfqpoint{-0.048611in}{0.000000in}}{\pgfqpoint{-0.000000in}{0.000000in}}{%
\pgfpathmoveto{\pgfqpoint{-0.000000in}{0.000000in}}%
\pgfpathlineto{\pgfqpoint{-0.048611in}{0.000000in}}%
\pgfusepath{stroke,fill}%
}%
\begin{pgfscope}%
\pgfsys@transformshift{0.713248in}{1.122097in}%
\pgfsys@useobject{currentmarker}{}%
\end{pgfscope}%
\end{pgfscope}%
\begin{pgfscope}%
\definecolor{textcolor}{rgb}{0.000000,0.000000,0.000000}%
\pgfsetstrokecolor{textcolor}%
\pgfsetfillcolor{textcolor}%
\pgftext[x=0.303410in, y=1.064059in, left, base]{\color{textcolor}\rmfamily\fontsize{11.000000}{13.200000}\selectfont \(\displaystyle {\ensuremath{-}0.2}\)}%
\end{pgfscope}%
\begin{pgfscope}%
\pgfsetbuttcap%
\pgfsetroundjoin%
\definecolor{currentfill}{rgb}{0.000000,0.000000,0.000000}%
\pgfsetfillcolor{currentfill}%
\pgfsetlinewidth{0.803000pt}%
\definecolor{currentstroke}{rgb}{0.000000,0.000000,0.000000}%
\pgfsetstrokecolor{currentstroke}%
\pgfsetdash{}{0pt}%
\pgfsys@defobject{currentmarker}{\pgfqpoint{-0.048611in}{0.000000in}}{\pgfqpoint{-0.000000in}{0.000000in}}{%
\pgfpathmoveto{\pgfqpoint{-0.000000in}{0.000000in}}%
\pgfpathlineto{\pgfqpoint{-0.048611in}{0.000000in}}%
\pgfusepath{stroke,fill}%
}%
\begin{pgfscope}%
\pgfsys@transformshift{0.713248in}{1.510986in}%
\pgfsys@useobject{currentmarker}{}%
\end{pgfscope}%
\end{pgfscope}%
\begin{pgfscope}%
\definecolor{textcolor}{rgb}{0.000000,0.000000,0.000000}%
\pgfsetstrokecolor{textcolor}%
\pgfsetfillcolor{textcolor}%
\pgftext[x=0.421697in, y=1.452948in, left, base]{\color{textcolor}\rmfamily\fontsize{11.000000}{13.200000}\selectfont \(\displaystyle {0.0}\)}%
\end{pgfscope}%
\begin{pgfscope}%
\pgfsetbuttcap%
\pgfsetroundjoin%
\definecolor{currentfill}{rgb}{0.000000,0.000000,0.000000}%
\pgfsetfillcolor{currentfill}%
\pgfsetlinewidth{0.803000pt}%
\definecolor{currentstroke}{rgb}{0.000000,0.000000,0.000000}%
\pgfsetstrokecolor{currentstroke}%
\pgfsetdash{}{0pt}%
\pgfsys@defobject{currentmarker}{\pgfqpoint{-0.048611in}{0.000000in}}{\pgfqpoint{-0.000000in}{0.000000in}}{%
\pgfpathmoveto{\pgfqpoint{-0.000000in}{0.000000in}}%
\pgfpathlineto{\pgfqpoint{-0.048611in}{0.000000in}}%
\pgfusepath{stroke,fill}%
}%
\begin{pgfscope}%
\pgfsys@transformshift{0.713248in}{1.899875in}%
\pgfsys@useobject{currentmarker}{}%
\end{pgfscope}%
\end{pgfscope}%
\begin{pgfscope}%
\definecolor{textcolor}{rgb}{0.000000,0.000000,0.000000}%
\pgfsetstrokecolor{textcolor}%
\pgfsetfillcolor{textcolor}%
\pgftext[x=0.421697in, y=1.841837in, left, base]{\color{textcolor}\rmfamily\fontsize{11.000000}{13.200000}\selectfont \(\displaystyle {0.2}\)}%
\end{pgfscope}%
\begin{pgfscope}%
\pgfsetbuttcap%
\pgfsetroundjoin%
\definecolor{currentfill}{rgb}{0.000000,0.000000,0.000000}%
\pgfsetfillcolor{currentfill}%
\pgfsetlinewidth{0.803000pt}%
\definecolor{currentstroke}{rgb}{0.000000,0.000000,0.000000}%
\pgfsetstrokecolor{currentstroke}%
\pgfsetdash{}{0pt}%
\pgfsys@defobject{currentmarker}{\pgfqpoint{-0.048611in}{0.000000in}}{\pgfqpoint{-0.000000in}{0.000000in}}{%
\pgfpathmoveto{\pgfqpoint{-0.000000in}{0.000000in}}%
\pgfpathlineto{\pgfqpoint{-0.048611in}{0.000000in}}%
\pgfusepath{stroke,fill}%
}%
\begin{pgfscope}%
\pgfsys@transformshift{0.713248in}{2.288764in}%
\pgfsys@useobject{currentmarker}{}%
\end{pgfscope}%
\end{pgfscope}%
\begin{pgfscope}%
\definecolor{textcolor}{rgb}{0.000000,0.000000,0.000000}%
\pgfsetstrokecolor{textcolor}%
\pgfsetfillcolor{textcolor}%
\pgftext[x=0.421697in, y=2.230726in, left, base]{\color{textcolor}\rmfamily\fontsize{11.000000}{13.200000}\selectfont \(\displaystyle {0.4}\)}%
\end{pgfscope}%
\begin{pgfscope}%
\definecolor{textcolor}{rgb}{0.000000,0.000000,0.000000}%
\pgfsetstrokecolor{textcolor}%
\pgfsetfillcolor{textcolor}%
\pgftext[x=0.247854in,y=1.510986in,,bottom,rotate=90.000000]{\color{textcolor}\rmfamily\fontsize{11.000000}{13.200000}\selectfont \(\displaystyle y\)}%
\end{pgfscope}%
\begin{pgfscope}%
\pgfsetrectcap%
\pgfsetmiterjoin%
\pgfsetlinewidth{0.803000pt}%
\definecolor{currentstroke}{rgb}{0.000000,0.000000,0.000000}%
\pgfsetstrokecolor{currentstroke}%
\pgfsetdash{}{0pt}%
\pgfpathmoveto{\pgfqpoint{0.713248in}{0.548486in}}%
\pgfpathlineto{\pgfqpoint{0.713248in}{2.473486in}}%
\pgfusepath{stroke}%
\end{pgfscope}%
\begin{pgfscope}%
\pgfsetrectcap%
\pgfsetmiterjoin%
\pgfsetlinewidth{0.803000pt}%
\definecolor{currentstroke}{rgb}{0.000000,0.000000,0.000000}%
\pgfsetstrokecolor{currentstroke}%
\pgfsetdash{}{0pt}%
\pgfpathmoveto{\pgfqpoint{2.650748in}{0.548486in}}%
\pgfpathlineto{\pgfqpoint{2.650748in}{2.473486in}}%
\pgfusepath{stroke}%
\end{pgfscope}%
\begin{pgfscope}%
\pgfsetrectcap%
\pgfsetmiterjoin%
\pgfsetlinewidth{0.803000pt}%
\definecolor{currentstroke}{rgb}{0.000000,0.000000,0.000000}%
\pgfsetstrokecolor{currentstroke}%
\pgfsetdash{}{0pt}%
\pgfpathmoveto{\pgfqpoint{0.713248in}{0.548486in}}%
\pgfpathlineto{\pgfqpoint{2.650748in}{0.548486in}}%
\pgfusepath{stroke}%
\end{pgfscope}%
\begin{pgfscope}%
\pgfsetrectcap%
\pgfsetmiterjoin%
\pgfsetlinewidth{0.803000pt}%
\definecolor{currentstroke}{rgb}{0.000000,0.000000,0.000000}%
\pgfsetstrokecolor{currentstroke}%
\pgfsetdash{}{0pt}%
\pgfpathmoveto{\pgfqpoint{0.713248in}{2.473486in}}%
\pgfpathlineto{\pgfqpoint{2.650748in}{2.473486in}}%
\pgfusepath{stroke}%
\end{pgfscope}%
\end{pgfpicture}%
\makeatother%
\endgroup%

    \caption{Plot of the obtained solution.}
    \label{fig:solution-circle-circle}
\end{subfigure}
\begin{subfigure}[t]{.49\textwidth}
    \centering
    %% Creator: Matplotlib, PGF backend
%%
%% To include the figure in your LaTeX document, write
%%   \input{<filename>.pgf}
%%
%% Make sure the required packages are loaded in your preamble
%%   \usepackage{pgf}
%%
%% Also ensure that all the required font packages are loaded; for instance,
%% the lmodern package is sometimes necessary when using math font.
%%   \usepackage{lmodern}
%%
%% Figures using additional raster images can only be included by \input if
%% they are in the same directory as the main LaTeX file. For loading figures
%% from other directories you can use the `import` package
%%   \usepackage{import}
%%
%% and then include the figures with
%%   \import{<path to file>}{<filename>.pgf}
%%
%% Matplotlib used the following preamble
%%   
%%   \usepackage{fontspec}
%%   \setmainfont{DejaVuSans.ttf}[Path=\detokenize{/home/fabio/.local/lib/python3.8/site-packages/matplotlib/mpl-data/fonts/ttf/}]
%%   \setsansfont{DejaVuSans.ttf}[Path=\detokenize{/home/fabio/.local/lib/python3.8/site-packages/matplotlib/mpl-data/fonts/ttf/}]
%%   \setmonofont{DejaVuSansMono.ttf}[Path=\detokenize{/home/fabio/.local/lib/python3.8/site-packages/matplotlib/mpl-data/fonts/ttf/}]
%%   \makeatletter\@ifpackageloaded{underscore}{}{\usepackage[strings]{underscore}}\makeatother
%%
\begingroup%
\makeatletter%
\begin{pgfpicture}%
\pgfpathrectangle{\pgfpointorigin}{\pgfqpoint{2.754977in}{2.635753in}}%
\pgfusepath{use as bounding box, clip}%
\begin{pgfscope}%
\pgfsetbuttcap%
\pgfsetmiterjoin%
\definecolor{currentfill}{rgb}{1.000000,1.000000,1.000000}%
\pgfsetfillcolor{currentfill}%
\pgfsetlinewidth{0.000000pt}%
\definecolor{currentstroke}{rgb}{1.000000,1.000000,1.000000}%
\pgfsetstrokecolor{currentstroke}%
\pgfsetdash{}{0pt}%
\pgfpathmoveto{\pgfqpoint{0.000000in}{0.000000in}}%
\pgfpathlineto{\pgfqpoint{2.754977in}{0.000000in}}%
\pgfpathlineto{\pgfqpoint{2.754977in}{2.635753in}}%
\pgfpathlineto{\pgfqpoint{0.000000in}{2.635753in}}%
\pgfpathlineto{\pgfqpoint{0.000000in}{0.000000in}}%
\pgfpathclose%
\pgfusepath{fill}%
\end{pgfscope}%
\begin{pgfscope}%
\pgfsetbuttcap%
\pgfsetmiterjoin%
\definecolor{currentfill}{rgb}{1.000000,1.000000,1.000000}%
\pgfsetfillcolor{currentfill}%
\pgfsetlinewidth{0.000000pt}%
\definecolor{currentstroke}{rgb}{0.000000,0.000000,0.000000}%
\pgfsetstrokecolor{currentstroke}%
\pgfsetstrokeopacity{0.000000}%
\pgfsetdash{}{0pt}%
\pgfpathmoveto{\pgfqpoint{0.717477in}{0.552715in}}%
\pgfpathlineto{\pgfqpoint{2.654977in}{0.552715in}}%
\pgfpathlineto{\pgfqpoint{2.654977in}{2.477715in}}%
\pgfpathlineto{\pgfqpoint{0.717477in}{2.477715in}}%
\pgfpathlineto{\pgfqpoint{0.717477in}{0.552715in}}%
\pgfpathclose%
\pgfusepath{fill}%
\end{pgfscope}%
\begin{pgfscope}%
\pgfpathrectangle{\pgfqpoint{0.717477in}{0.552715in}}{\pgfqpoint{1.937500in}{1.925000in}}%
\pgfusepath{clip}%
\pgfsetbuttcap%
\pgfsetroundjoin%
\definecolor{currentfill}{rgb}{0.054902,0.262745,0.486275}%
\pgfsetfillcolor{currentfill}%
\pgfsetlinewidth{1.003750pt}%
\definecolor{currentstroke}{rgb}{0.054902,0.262745,0.486275}%
\pgfsetstrokecolor{currentstroke}%
\pgfsetdash{}{0pt}%
\pgfsys@defobject{currentmarker}{\pgfqpoint{-0.041667in}{-0.041667in}}{\pgfqpoint{0.041667in}{0.041667in}}{%
\pgfpathmoveto{\pgfqpoint{0.000000in}{-0.041667in}}%
\pgfpathcurveto{\pgfqpoint{0.011050in}{-0.041667in}}{\pgfqpoint{0.021649in}{-0.037276in}}{\pgfqpoint{0.029463in}{-0.029463in}}%
\pgfpathcurveto{\pgfqpoint{0.037276in}{-0.021649in}}{\pgfqpoint{0.041667in}{-0.011050in}}{\pgfqpoint{0.041667in}{0.000000in}}%
\pgfpathcurveto{\pgfqpoint{0.041667in}{0.011050in}}{\pgfqpoint{0.037276in}{0.021649in}}{\pgfqpoint{0.029463in}{0.029463in}}%
\pgfpathcurveto{\pgfqpoint{0.021649in}{0.037276in}}{\pgfqpoint{0.011050in}{0.041667in}}{\pgfqpoint{0.000000in}{0.041667in}}%
\pgfpathcurveto{\pgfqpoint{-0.011050in}{0.041667in}}{\pgfqpoint{-0.021649in}{0.037276in}}{\pgfqpoint{-0.029463in}{0.029463in}}%
\pgfpathcurveto{\pgfqpoint{-0.037276in}{0.021649in}}{\pgfqpoint{-0.041667in}{0.011050in}}{\pgfqpoint{-0.041667in}{0.000000in}}%
\pgfpathcurveto{\pgfqpoint{-0.041667in}{-0.011050in}}{\pgfqpoint{-0.037276in}{-0.021649in}}{\pgfqpoint{-0.029463in}{-0.029463in}}%
\pgfpathcurveto{\pgfqpoint{-0.021649in}{-0.037276in}}{\pgfqpoint{-0.011050in}{-0.041667in}}{\pgfqpoint{0.000000in}{-0.041667in}}%
\pgfpathlineto{\pgfqpoint{0.000000in}{-0.041667in}}%
\pgfpathclose%
\pgfusepath{stroke,fill}%
}%
\begin{pgfscope}%
\pgfsys@transformshift{2.522469in}{1.392051in}%
\pgfsys@useobject{currentmarker}{}%
\end{pgfscope}%
\begin{pgfscope}%
\pgfsys@transformshift{2.344453in}{1.415437in}%
\pgfsys@useobject{currentmarker}{}%
\end{pgfscope}%
\begin{pgfscope}%
\pgfsys@transformshift{2.203432in}{1.441755in}%
\pgfsys@useobject{currentmarker}{}%
\end{pgfscope}%
\begin{pgfscope}%
\pgfsys@transformshift{2.059670in}{1.490428in}%
\pgfsys@useobject{currentmarker}{}%
\end{pgfscope}%
\begin{pgfscope}%
\pgfsys@transformshift{1.908217in}{1.536760in}%
\pgfsys@useobject{currentmarker}{}%
\end{pgfscope}%
\begin{pgfscope}%
\pgfsys@transformshift{1.756998in}{1.539001in}%
\pgfsys@useobject{currentmarker}{}%
\end{pgfscope}%
\begin{pgfscope}%
\pgfsys@transformshift{1.605502in}{1.541590in}%
\pgfsys@useobject{currentmarker}{}%
\end{pgfscope}%
\begin{pgfscope}%
\pgfsys@transformshift{1.453610in}{1.529251in}%
\pgfsys@useobject{currentmarker}{}%
\end{pgfscope}%
\begin{pgfscope}%
\pgfsys@transformshift{1.302677in}{1.516171in}%
\pgfsys@useobject{currentmarker}{}%
\end{pgfscope}%
\begin{pgfscope}%
\pgfsys@transformshift{1.157865in}{1.485534in}%
\pgfsys@useobject{currentmarker}{}%
\end{pgfscope}%
\begin{pgfscope}%
\pgfsys@transformshift{1.015686in}{1.414070in}%
\pgfsys@useobject{currentmarker}{}%
\end{pgfscope}%
\begin{pgfscope}%
\pgfsys@transformshift{0.838553in}{1.390967in}%
\pgfsys@useobject{currentmarker}{}%
\end{pgfscope}%
\end{pgfscope}%
\begin{pgfscope}%
\pgfpathrectangle{\pgfqpoint{0.717477in}{0.552715in}}{\pgfqpoint{1.937500in}{1.925000in}}%
\pgfusepath{clip}%
\pgfsetbuttcap%
\pgfsetroundjoin%
\definecolor{currentfill}{rgb}{0.835294,0.321569,0.035294}%
\pgfsetfillcolor{currentfill}%
\pgfsetlinewidth{1.003750pt}%
\definecolor{currentstroke}{rgb}{0.835294,0.321569,0.035294}%
\pgfsetstrokecolor{currentstroke}%
\pgfsetdash{}{0pt}%
\pgfsys@defobject{currentmarker}{\pgfqpoint{-0.041667in}{-0.041667in}}{\pgfqpoint{0.041667in}{0.041667in}}{%
\pgfpathmoveto{\pgfqpoint{0.000000in}{-0.041667in}}%
\pgfpathcurveto{\pgfqpoint{0.011050in}{-0.041667in}}{\pgfqpoint{0.021649in}{-0.037276in}}{\pgfqpoint{0.029463in}{-0.029463in}}%
\pgfpathcurveto{\pgfqpoint{0.037276in}{-0.021649in}}{\pgfqpoint{0.041667in}{-0.011050in}}{\pgfqpoint{0.041667in}{0.000000in}}%
\pgfpathcurveto{\pgfqpoint{0.041667in}{0.011050in}}{\pgfqpoint{0.037276in}{0.021649in}}{\pgfqpoint{0.029463in}{0.029463in}}%
\pgfpathcurveto{\pgfqpoint{0.021649in}{0.037276in}}{\pgfqpoint{0.011050in}{0.041667in}}{\pgfqpoint{0.000000in}{0.041667in}}%
\pgfpathcurveto{\pgfqpoint{-0.011050in}{0.041667in}}{\pgfqpoint{-0.021649in}{0.037276in}}{\pgfqpoint{-0.029463in}{0.029463in}}%
\pgfpathcurveto{\pgfqpoint{-0.037276in}{0.021649in}}{\pgfqpoint{-0.041667in}{0.011050in}}{\pgfqpoint{-0.041667in}{0.000000in}}%
\pgfpathcurveto{\pgfqpoint{-0.041667in}{-0.011050in}}{\pgfqpoint{-0.037276in}{-0.021649in}}{\pgfqpoint{-0.029463in}{-0.029463in}}%
\pgfpathcurveto{\pgfqpoint{-0.021649in}{-0.037276in}}{\pgfqpoint{-0.011050in}{-0.041667in}}{\pgfqpoint{0.000000in}{-0.041667in}}%
\pgfpathlineto{\pgfqpoint{0.000000in}{-0.041667in}}%
\pgfpathclose%
\pgfusepath{stroke,fill}%
}%
\begin{pgfscope}%
\pgfsys@transformshift{1.028490in}{1.420470in}%
\pgfsys@useobject{currentmarker}{}%
\end{pgfscope}%
\begin{pgfscope}%
\pgfsys@transformshift{1.183076in}{1.495152in}%
\pgfsys@useobject{currentmarker}{}%
\end{pgfscope}%
\begin{pgfscope}%
\pgfsys@transformshift{1.331525in}{1.518870in}%
\pgfsys@useobject{currentmarker}{}%
\end{pgfscope}%
\begin{pgfscope}%
\pgfsys@transformshift{1.471256in}{1.530816in}%
\pgfsys@useobject{currentmarker}{}%
\end{pgfscope}%
\begin{pgfscope}%
\pgfsys@transformshift{1.610368in}{1.541057in}%
\pgfsys@useobject{currentmarker}{}%
\end{pgfscope}%
\begin{pgfscope}%
\pgfsys@transformshift{1.752074in}{1.539425in}%
\pgfsys@useobject{currentmarker}{}%
\end{pgfscope}%
\begin{pgfscope}%
\pgfsys@transformshift{1.890467in}{1.538405in}%
\pgfsys@useobject{currentmarker}{}%
\end{pgfscope}%
\begin{pgfscope}%
\pgfsys@transformshift{2.031069in}{1.501266in}%
\pgfsys@useobject{currentmarker}{}%
\end{pgfscope}%
\begin{pgfscope}%
\pgfsys@transformshift{2.178437in}{1.446712in}%
\pgfsys@useobject{currentmarker}{}%
\end{pgfscope}%
\begin{pgfscope}%
\pgfsys@transformshift{2.331670in}{1.419832in}%
\pgfsys@useobject{currentmarker}{}%
\end{pgfscope}%
\end{pgfscope}%
\begin{pgfscope}%
\pgfsetbuttcap%
\pgfsetroundjoin%
\definecolor{currentfill}{rgb}{0.000000,0.000000,0.000000}%
\pgfsetfillcolor{currentfill}%
\pgfsetlinewidth{0.803000pt}%
\definecolor{currentstroke}{rgb}{0.000000,0.000000,0.000000}%
\pgfsetstrokecolor{currentstroke}%
\pgfsetdash{}{0pt}%
\pgfsys@defobject{currentmarker}{\pgfqpoint{0.000000in}{-0.048611in}}{\pgfqpoint{0.000000in}{0.000000in}}{%
\pgfpathmoveto{\pgfqpoint{0.000000in}{0.000000in}}%
\pgfpathlineto{\pgfqpoint{0.000000in}{-0.048611in}}%
\pgfusepath{stroke,fill}%
}%
\begin{pgfscope}%
\pgfsys@transformshift{0.994263in}{0.552715in}%
\pgfsys@useobject{currentmarker}{}%
\end{pgfscope}%
\end{pgfscope}%
\begin{pgfscope}%
\definecolor{textcolor}{rgb}{0.000000,0.000000,0.000000}%
\pgfsetstrokecolor{textcolor}%
\pgfsetfillcolor{textcolor}%
\pgftext[x=0.994263in,y=0.455493in,,top]{\color{textcolor}\rmfamily\fontsize{11.000000}{13.200000}\selectfont \(\displaystyle {\ensuremath{-}5}\)}%
\end{pgfscope}%
\begin{pgfscope}%
\pgfsetbuttcap%
\pgfsetroundjoin%
\definecolor{currentfill}{rgb}{0.000000,0.000000,0.000000}%
\pgfsetfillcolor{currentfill}%
\pgfsetlinewidth{0.803000pt}%
\definecolor{currentstroke}{rgb}{0.000000,0.000000,0.000000}%
\pgfsetstrokecolor{currentstroke}%
\pgfsetdash{}{0pt}%
\pgfsys@defobject{currentmarker}{\pgfqpoint{0.000000in}{-0.048611in}}{\pgfqpoint{0.000000in}{0.000000in}}{%
\pgfpathmoveto{\pgfqpoint{0.000000in}{0.000000in}}%
\pgfpathlineto{\pgfqpoint{0.000000in}{-0.048611in}}%
\pgfusepath{stroke,fill}%
}%
\begin{pgfscope}%
\pgfsys@transformshift{1.686227in}{0.552715in}%
\pgfsys@useobject{currentmarker}{}%
\end{pgfscope}%
\end{pgfscope}%
\begin{pgfscope}%
\definecolor{textcolor}{rgb}{0.000000,0.000000,0.000000}%
\pgfsetstrokecolor{textcolor}%
\pgfsetfillcolor{textcolor}%
\pgftext[x=1.686227in,y=0.455493in,,top]{\color{textcolor}\rmfamily\fontsize{11.000000}{13.200000}\selectfont \(\displaystyle {0}\)}%
\end{pgfscope}%
\begin{pgfscope}%
\pgfsetbuttcap%
\pgfsetroundjoin%
\definecolor{currentfill}{rgb}{0.000000,0.000000,0.000000}%
\pgfsetfillcolor{currentfill}%
\pgfsetlinewidth{0.803000pt}%
\definecolor{currentstroke}{rgb}{0.000000,0.000000,0.000000}%
\pgfsetstrokecolor{currentstroke}%
\pgfsetdash{}{0pt}%
\pgfsys@defobject{currentmarker}{\pgfqpoint{0.000000in}{-0.048611in}}{\pgfqpoint{0.000000in}{0.000000in}}{%
\pgfpathmoveto{\pgfqpoint{0.000000in}{0.000000in}}%
\pgfpathlineto{\pgfqpoint{0.000000in}{-0.048611in}}%
\pgfusepath{stroke,fill}%
}%
\begin{pgfscope}%
\pgfsys@transformshift{2.378192in}{0.552715in}%
\pgfsys@useobject{currentmarker}{}%
\end{pgfscope}%
\end{pgfscope}%
\begin{pgfscope}%
\definecolor{textcolor}{rgb}{0.000000,0.000000,0.000000}%
\pgfsetstrokecolor{textcolor}%
\pgfsetfillcolor{textcolor}%
\pgftext[x=2.378192in,y=0.455493in,,top]{\color{textcolor}\rmfamily\fontsize{11.000000}{13.200000}\selectfont \(\displaystyle {5}\)}%
\end{pgfscope}%
\begin{pgfscope}%
\definecolor{textcolor}{rgb}{0.000000,0.000000,0.000000}%
\pgfsetstrokecolor{textcolor}%
\pgfsetfillcolor{textcolor}%
\pgftext[x=1.686227in,y=0.252083in,,top]{\color{textcolor}\rmfamily\fontsize{11.000000}{13.200000}\selectfont \(\displaystyle x/h\)}%
\end{pgfscope}%
\begin{pgfscope}%
\pgfsetbuttcap%
\pgfsetroundjoin%
\definecolor{currentfill}{rgb}{0.000000,0.000000,0.000000}%
\pgfsetfillcolor{currentfill}%
\pgfsetlinewidth{0.803000pt}%
\definecolor{currentstroke}{rgb}{0.000000,0.000000,0.000000}%
\pgfsetstrokecolor{currentstroke}%
\pgfsetdash{}{0pt}%
\pgfsys@defobject{currentmarker}{\pgfqpoint{-0.048611in}{0.000000in}}{\pgfqpoint{-0.000000in}{0.000000in}}{%
\pgfpathmoveto{\pgfqpoint{-0.000000in}{0.000000in}}%
\pgfpathlineto{\pgfqpoint{-0.048611in}{0.000000in}}%
\pgfusepath{stroke,fill}%
}%
\begin{pgfscope}%
\pgfsys@transformshift{0.717477in}{0.552715in}%
\pgfsys@useobject{currentmarker}{}%
\end{pgfscope}%
\end{pgfscope}%
\begin{pgfscope}%
\definecolor{textcolor}{rgb}{0.000000,0.000000,0.000000}%
\pgfsetstrokecolor{textcolor}%
\pgfsetfillcolor{textcolor}%
\pgftext[x=0.307639in, y=0.494677in, left, base]{\color{textcolor}\rmfamily\fontsize{11.000000}{13.200000}\selectfont \(\displaystyle {\ensuremath{-}1.0}\)}%
\end{pgfscope}%
\begin{pgfscope}%
\pgfsetbuttcap%
\pgfsetroundjoin%
\definecolor{currentfill}{rgb}{0.000000,0.000000,0.000000}%
\pgfsetfillcolor{currentfill}%
\pgfsetlinewidth{0.803000pt}%
\definecolor{currentstroke}{rgb}{0.000000,0.000000,0.000000}%
\pgfsetstrokecolor{currentstroke}%
\pgfsetdash{}{0pt}%
\pgfsys@defobject{currentmarker}{\pgfqpoint{-0.048611in}{0.000000in}}{\pgfqpoint{-0.000000in}{0.000000in}}{%
\pgfpathmoveto{\pgfqpoint{-0.000000in}{0.000000in}}%
\pgfpathlineto{\pgfqpoint{-0.048611in}{0.000000in}}%
\pgfusepath{stroke,fill}%
}%
\begin{pgfscope}%
\pgfsys@transformshift{0.717477in}{1.033965in}%
\pgfsys@useobject{currentmarker}{}%
\end{pgfscope}%
\end{pgfscope}%
\begin{pgfscope}%
\definecolor{textcolor}{rgb}{0.000000,0.000000,0.000000}%
\pgfsetstrokecolor{textcolor}%
\pgfsetfillcolor{textcolor}%
\pgftext[x=0.307639in, y=0.975927in, left, base]{\color{textcolor}\rmfamily\fontsize{11.000000}{13.200000}\selectfont \(\displaystyle {\ensuremath{-}0.5}\)}%
\end{pgfscope}%
\begin{pgfscope}%
\pgfsetbuttcap%
\pgfsetroundjoin%
\definecolor{currentfill}{rgb}{0.000000,0.000000,0.000000}%
\pgfsetfillcolor{currentfill}%
\pgfsetlinewidth{0.803000pt}%
\definecolor{currentstroke}{rgb}{0.000000,0.000000,0.000000}%
\pgfsetstrokecolor{currentstroke}%
\pgfsetdash{}{0pt}%
\pgfsys@defobject{currentmarker}{\pgfqpoint{-0.048611in}{0.000000in}}{\pgfqpoint{-0.000000in}{0.000000in}}{%
\pgfpathmoveto{\pgfqpoint{-0.000000in}{0.000000in}}%
\pgfpathlineto{\pgfqpoint{-0.048611in}{0.000000in}}%
\pgfusepath{stroke,fill}%
}%
\begin{pgfscope}%
\pgfsys@transformshift{0.717477in}{1.515215in}%
\pgfsys@useobject{currentmarker}{}%
\end{pgfscope}%
\end{pgfscope}%
\begin{pgfscope}%
\definecolor{textcolor}{rgb}{0.000000,0.000000,0.000000}%
\pgfsetstrokecolor{textcolor}%
\pgfsetfillcolor{textcolor}%
\pgftext[x=0.425926in, y=1.457177in, left, base]{\color{textcolor}\rmfamily\fontsize{11.000000}{13.200000}\selectfont \(\displaystyle {0.0}\)}%
\end{pgfscope}%
\begin{pgfscope}%
\pgfsetbuttcap%
\pgfsetroundjoin%
\definecolor{currentfill}{rgb}{0.000000,0.000000,0.000000}%
\pgfsetfillcolor{currentfill}%
\pgfsetlinewidth{0.803000pt}%
\definecolor{currentstroke}{rgb}{0.000000,0.000000,0.000000}%
\pgfsetstrokecolor{currentstroke}%
\pgfsetdash{}{0pt}%
\pgfsys@defobject{currentmarker}{\pgfqpoint{-0.048611in}{0.000000in}}{\pgfqpoint{-0.000000in}{0.000000in}}{%
\pgfpathmoveto{\pgfqpoint{-0.000000in}{0.000000in}}%
\pgfpathlineto{\pgfqpoint{-0.048611in}{0.000000in}}%
\pgfusepath{stroke,fill}%
}%
\begin{pgfscope}%
\pgfsys@transformshift{0.717477in}{1.996465in}%
\pgfsys@useobject{currentmarker}{}%
\end{pgfscope}%
\end{pgfscope}%
\begin{pgfscope}%
\definecolor{textcolor}{rgb}{0.000000,0.000000,0.000000}%
\pgfsetstrokecolor{textcolor}%
\pgfsetfillcolor{textcolor}%
\pgftext[x=0.425926in, y=1.938427in, left, base]{\color{textcolor}\rmfamily\fontsize{11.000000}{13.200000}\selectfont \(\displaystyle {0.5}\)}%
\end{pgfscope}%
\begin{pgfscope}%
\pgfsetbuttcap%
\pgfsetroundjoin%
\definecolor{currentfill}{rgb}{0.000000,0.000000,0.000000}%
\pgfsetfillcolor{currentfill}%
\pgfsetlinewidth{0.803000pt}%
\definecolor{currentstroke}{rgb}{0.000000,0.000000,0.000000}%
\pgfsetstrokecolor{currentstroke}%
\pgfsetdash{}{0pt}%
\pgfsys@defobject{currentmarker}{\pgfqpoint{-0.048611in}{0.000000in}}{\pgfqpoint{-0.000000in}{0.000000in}}{%
\pgfpathmoveto{\pgfqpoint{-0.000000in}{0.000000in}}%
\pgfpathlineto{\pgfqpoint{-0.048611in}{0.000000in}}%
\pgfusepath{stroke,fill}%
}%
\begin{pgfscope}%
\pgfsys@transformshift{0.717477in}{2.477715in}%
\pgfsys@useobject{currentmarker}{}%
\end{pgfscope}%
\end{pgfscope}%
\begin{pgfscope}%
\definecolor{textcolor}{rgb}{0.000000,0.000000,0.000000}%
\pgfsetstrokecolor{textcolor}%
\pgfsetfillcolor{textcolor}%
\pgftext[x=0.425926in, y=2.419677in, left, base]{\color{textcolor}\rmfamily\fontsize{11.000000}{13.200000}\selectfont \(\displaystyle {1.0}\)}%
\end{pgfscope}%
\begin{pgfscope}%
\definecolor{textcolor}{rgb}{0.000000,0.000000,0.000000}%
\pgfsetstrokecolor{textcolor}%
\pgfsetfillcolor{textcolor}%
\pgftext[x=0.252083in,y=1.515215in,,bottom,rotate=90.000000]{\color{textcolor}\rmfamily\fontsize{11.000000}{13.200000}\selectfont \(\displaystyle y/h\)}%
\end{pgfscope}%
\begin{pgfscope}%
\pgfpathrectangle{\pgfqpoint{0.717477in}{0.552715in}}{\pgfqpoint{1.937500in}{1.925000in}}%
\pgfusepath{clip}%
\pgfsetrectcap%
\pgfsetroundjoin%
\pgfsetlinewidth{1.505625pt}%
\definecolor{currentstroke}{rgb}{0.054902,0.262745,0.486275}%
\pgfsetstrokecolor{currentstroke}%
\pgfsetstrokeopacity{0.250000}%
\pgfsetdash{}{0pt}%
\pgfpathmoveto{\pgfqpoint{0.717477in}{1.515215in}}%
\pgfpathlineto{\pgfqpoint{2.654977in}{1.515215in}}%
\pgfusepath{stroke}%
\end{pgfscope}%
\begin{pgfscope}%
\pgfsetrectcap%
\pgfsetmiterjoin%
\pgfsetlinewidth{0.803000pt}%
\definecolor{currentstroke}{rgb}{0.000000,0.000000,0.000000}%
\pgfsetstrokecolor{currentstroke}%
\pgfsetdash{}{0pt}%
\pgfpathmoveto{\pgfqpoint{0.717477in}{0.552715in}}%
\pgfpathlineto{\pgfqpoint{0.717477in}{2.477715in}}%
\pgfusepath{stroke}%
\end{pgfscope}%
\begin{pgfscope}%
\pgfsetrectcap%
\pgfsetmiterjoin%
\pgfsetlinewidth{0.803000pt}%
\definecolor{currentstroke}{rgb}{0.000000,0.000000,0.000000}%
\pgfsetstrokecolor{currentstroke}%
\pgfsetdash{}{0pt}%
\pgfpathmoveto{\pgfqpoint{2.654977in}{0.552715in}}%
\pgfpathlineto{\pgfqpoint{2.654977in}{2.477715in}}%
\pgfusepath{stroke}%
\end{pgfscope}%
\begin{pgfscope}%
\pgfsetrectcap%
\pgfsetmiterjoin%
\pgfsetlinewidth{0.803000pt}%
\definecolor{currentstroke}{rgb}{0.000000,0.000000,0.000000}%
\pgfsetstrokecolor{currentstroke}%
\pgfsetdash{}{0pt}%
\pgfpathmoveto{\pgfqpoint{0.717477in}{0.552715in}}%
\pgfpathlineto{\pgfqpoint{2.654977in}{0.552715in}}%
\pgfusepath{stroke}%
\end{pgfscope}%
\begin{pgfscope}%
\pgfsetrectcap%
\pgfsetmiterjoin%
\pgfsetlinewidth{0.803000pt}%
\definecolor{currentstroke}{rgb}{0.000000,0.000000,0.000000}%
\pgfsetstrokecolor{currentstroke}%
\pgfsetdash{}{0pt}%
\pgfpathmoveto{\pgfqpoint{0.717477in}{2.477715in}}%
\pgfpathlineto{\pgfqpoint{2.654977in}{2.477715in}}%
\pgfusepath{stroke}%
\end{pgfscope}%
\begin{pgfscope}%
\pgfsetbuttcap%
\pgfsetmiterjoin%
\definecolor{currentfill}{rgb}{1.000000,1.000000,1.000000}%
\pgfsetfillcolor{currentfill}%
\pgfsetfillopacity{0.800000}%
\pgfsetlinewidth{1.003750pt}%
\definecolor{currentstroke}{rgb}{0.800000,0.800000,0.800000}%
\pgfsetstrokecolor{currentstroke}%
\pgfsetstrokeopacity{0.800000}%
\pgfsetdash{}{0pt}%
\pgfpathmoveto{\pgfqpoint{1.250048in}{1.682764in}}%
\pgfpathlineto{\pgfqpoint{2.548033in}{1.682764in}}%
\pgfpathquadraticcurveto{\pgfqpoint{2.578588in}{1.682764in}}{\pgfqpoint{2.578588in}{1.713320in}}%
\pgfpathlineto{\pgfqpoint{2.578588in}{2.370771in}}%
\pgfpathquadraticcurveto{\pgfqpoint{2.578588in}{2.401326in}}{\pgfqpoint{2.548033in}{2.401326in}}%
\pgfpathlineto{\pgfqpoint{1.250048in}{2.401326in}}%
\pgfpathquadraticcurveto{\pgfqpoint{1.219493in}{2.401326in}}{\pgfqpoint{1.219493in}{2.370771in}}%
\pgfpathlineto{\pgfqpoint{1.219493in}{1.713320in}}%
\pgfpathquadraticcurveto{\pgfqpoint{1.219493in}{1.682764in}}{\pgfqpoint{1.250048in}{1.682764in}}%
\pgfpathlineto{\pgfqpoint{1.250048in}{1.682764in}}%
\pgfpathclose%
\pgfusepath{stroke,fill}%
\end{pgfscope}%
\begin{pgfscope}%
\pgfsetbuttcap%
\pgfsetroundjoin%
\definecolor{currentfill}{rgb}{0.054902,0.262745,0.486275}%
\pgfsetfillcolor{currentfill}%
\pgfsetlinewidth{1.003750pt}%
\definecolor{currentstroke}{rgb}{0.054902,0.262745,0.486275}%
\pgfsetstrokecolor{currentstroke}%
\pgfsetdash{}{0pt}%
\pgfsys@defobject{currentmarker}{\pgfqpoint{-0.041667in}{-0.041667in}}{\pgfqpoint{0.041667in}{0.041667in}}{%
\pgfpathmoveto{\pgfqpoint{0.000000in}{-0.041667in}}%
\pgfpathcurveto{\pgfqpoint{0.011050in}{-0.041667in}}{\pgfqpoint{0.021649in}{-0.037276in}}{\pgfqpoint{0.029463in}{-0.029463in}}%
\pgfpathcurveto{\pgfqpoint{0.037276in}{-0.021649in}}{\pgfqpoint{0.041667in}{-0.011050in}}{\pgfqpoint{0.041667in}{0.000000in}}%
\pgfpathcurveto{\pgfqpoint{0.041667in}{0.011050in}}{\pgfqpoint{0.037276in}{0.021649in}}{\pgfqpoint{0.029463in}{0.029463in}}%
\pgfpathcurveto{\pgfqpoint{0.021649in}{0.037276in}}{\pgfqpoint{0.011050in}{0.041667in}}{\pgfqpoint{0.000000in}{0.041667in}}%
\pgfpathcurveto{\pgfqpoint{-0.011050in}{0.041667in}}{\pgfqpoint{-0.021649in}{0.037276in}}{\pgfqpoint{-0.029463in}{0.029463in}}%
\pgfpathcurveto{\pgfqpoint{-0.037276in}{0.021649in}}{\pgfqpoint{-0.041667in}{0.011050in}}{\pgfqpoint{-0.041667in}{0.000000in}}%
\pgfpathcurveto{\pgfqpoint{-0.041667in}{-0.011050in}}{\pgfqpoint{-0.037276in}{-0.021649in}}{\pgfqpoint{-0.029463in}{-0.029463in}}%
\pgfpathcurveto{\pgfqpoint{-0.021649in}{-0.037276in}}{\pgfqpoint{-0.011050in}{-0.041667in}}{\pgfqpoint{0.000000in}{-0.041667in}}%
\pgfpathlineto{\pgfqpoint{0.000000in}{-0.041667in}}%
\pgfpathclose%
\pgfusepath{stroke,fill}%
}%
\begin{pgfscope}%
\pgfsys@transformshift{1.433382in}{2.264244in}%
\pgfsys@useobject{currentmarker}{}%
\end{pgfscope}%
\end{pgfscope}%
\begin{pgfscope}%
\definecolor{textcolor}{rgb}{0.000000,0.000000,0.000000}%
\pgfsetstrokecolor{textcolor}%
\pgfsetfillcolor{textcolor}%
\pgftext[x=1.708382in,y=2.224140in,left,base]{\color{textcolor}\rmfamily\fontsize{11.000000}{13.200000}\selectfont Primary}%
\end{pgfscope}%
\begin{pgfscope}%
\pgfsetbuttcap%
\pgfsetroundjoin%
\definecolor{currentfill}{rgb}{0.835294,0.321569,0.035294}%
\pgfsetfillcolor{currentfill}%
\pgfsetlinewidth{1.003750pt}%
\definecolor{currentstroke}{rgb}{0.835294,0.321569,0.035294}%
\pgfsetstrokecolor{currentstroke}%
\pgfsetdash{}{0pt}%
\pgfsys@defobject{currentmarker}{\pgfqpoint{-0.041667in}{-0.041667in}}{\pgfqpoint{0.041667in}{0.041667in}}{%
\pgfpathmoveto{\pgfqpoint{0.000000in}{-0.041667in}}%
\pgfpathcurveto{\pgfqpoint{0.011050in}{-0.041667in}}{\pgfqpoint{0.021649in}{-0.037276in}}{\pgfqpoint{0.029463in}{-0.029463in}}%
\pgfpathcurveto{\pgfqpoint{0.037276in}{-0.021649in}}{\pgfqpoint{0.041667in}{-0.011050in}}{\pgfqpoint{0.041667in}{0.000000in}}%
\pgfpathcurveto{\pgfqpoint{0.041667in}{0.011050in}}{\pgfqpoint{0.037276in}{0.021649in}}{\pgfqpoint{0.029463in}{0.029463in}}%
\pgfpathcurveto{\pgfqpoint{0.021649in}{0.037276in}}{\pgfqpoint{0.011050in}{0.041667in}}{\pgfqpoint{0.000000in}{0.041667in}}%
\pgfpathcurveto{\pgfqpoint{-0.011050in}{0.041667in}}{\pgfqpoint{-0.021649in}{0.037276in}}{\pgfqpoint{-0.029463in}{0.029463in}}%
\pgfpathcurveto{\pgfqpoint{-0.037276in}{0.021649in}}{\pgfqpoint{-0.041667in}{0.011050in}}{\pgfqpoint{-0.041667in}{0.000000in}}%
\pgfpathcurveto{\pgfqpoint{-0.041667in}{-0.011050in}}{\pgfqpoint{-0.037276in}{-0.021649in}}{\pgfqpoint{-0.029463in}{-0.029463in}}%
\pgfpathcurveto{\pgfqpoint{-0.021649in}{-0.037276in}}{\pgfqpoint{-0.011050in}{-0.041667in}}{\pgfqpoint{0.000000in}{-0.041667in}}%
\pgfpathlineto{\pgfqpoint{0.000000in}{-0.041667in}}%
\pgfpathclose%
\pgfusepath{stroke,fill}%
}%
\begin{pgfscope}%
\pgfsys@transformshift{1.433382in}{2.040001in}%
\pgfsys@useobject{currentmarker}{}%
\end{pgfscope}%
\end{pgfscope}%
\begin{pgfscope}%
\definecolor{textcolor}{rgb}{0.000000,0.000000,0.000000}%
\pgfsetstrokecolor{textcolor}%
\pgfsetfillcolor{textcolor}%
\pgftext[x=1.708382in,y=1.999897in,left,base]{\color{textcolor}\rmfamily\fontsize{11.000000}{13.200000}\selectfont Secondary}%
\end{pgfscope}%
\begin{pgfscope}%
\pgfsetrectcap%
\pgfsetroundjoin%
\pgfsetlinewidth{1.505625pt}%
\definecolor{currentstroke}{rgb}{0.054902,0.262745,0.486275}%
\pgfsetstrokecolor{currentstroke}%
\pgfsetstrokeopacity{0.250000}%
\pgfsetdash{}{0pt}%
\pgfpathmoveto{\pgfqpoint{1.280604in}{1.829126in}}%
\pgfpathlineto{\pgfqpoint{1.433382in}{1.829126in}}%
\pgfpathlineto{\pgfqpoint{1.586160in}{1.829126in}}%
\pgfusepath{stroke}%
\end{pgfscope}%
\begin{pgfscope}%
\definecolor{textcolor}{rgb}{0.000000,0.000000,0.000000}%
\pgfsetstrokecolor{textcolor}%
\pgfsetfillcolor{textcolor}%
\pgftext[x=1.708382in,y=1.775654in,left,base]{\color{textcolor}\rmfamily\fontsize{11.000000}{13.200000}\selectfont Theory}%
\end{pgfscope}%
\end{pgfpicture}%
\makeatother%
\endgroup%

    \caption{Zoomed-in view of the interface nodes of both the primary and the secondary body. From theory we would expect them to describe a line. The axes are normalized by the mesh size $h$ at the interface.}
    \label{fig:zero-deviation}
\end{subfigure}
\caption{Results for a two dimensional problem with two semicircles in contact.}
\label{fig:contact-circle-circle}
\end{figure}

\clearpage
\section{Further work}
\label{sec:further}

\subsection{Contact mechanics refactors}
For a long time, Akantu only had a single implementation of contact mechanics, which is why its classes use generic names such as \texttt{ContactMechanicsModel} or \texttt{ContactDetector}. Now that there is a second contact mechanics implementation, we could simply rename the penalty method classes to more appropriate names, for example \texttt{ContactMechanicsPenaltyModel} or \texttt{ContactDetectorPenalty}. However, these renames should come with a larger refactor of the architecture of the contact mechanics package, that we will discuss in the next section.

\subsection{Shared interface for both contact models}
\label{subsec:coupler}

As discussed in \refsec{subsec:cppadditions}, we attempted to refactor \texttt{CouplerSolidContact} to use it for the INTERNODES method. The attempt is accessible on the Akantu repository\footnote{\url{https://gitlab.com/akantu/akantu/-/commit/87d81a453d57f97546d166989c9a4dfa27cf19f6}}. While we were not successful, this allowed us to find an issue with the current architecture, and leads us to propose a solution...

A key problem is that the \texttt{ContactMechanicsModel} (penalty method) extends \texttt{Model} even though it is not usable as a standalone model, and needs to be used through the \texttt{CouplerSolidContact}. Additionally, the \texttt{CouplerSolidContact} does not act as a standalone \texttt{Model}-\texttt{Model} coupler; it is quite tailored to the implementation details of \texttt{ContactMechanicsModel}. Hence, we should simply combine the two classes and make the resulting contact model contain the solid model.

As such, we propose the following refactor:
\begin{enumerate}
    \item Move all the penalty contact mechanics code from \texttt{ContactMechanicsModel} to \texttt{CouplerSolidContact}, under a new name. To express that it's the penalty contact model, we suggest \texttt{ContactMechanicsPenaltyModel}.
    \item Introduce a shared base class for \texttt{ContactMechanicsPenaltyModel} and \texttt{ContactMechanicsInternodesModel} that contains the solid model, and all the code that is common to all contact mechanics implementations.
    \item Once that is done, to ensure that both methods can be used in all situations, it will be useful to go through the two contact mechanics model implementations, and validate that they work correctly with all the solver types.
\end{enumerate}

\begin{figure}[!htb]
    \centering
    \includegraphics[width=1.0\linewidth]{diagrams/contact_refactor.pdf}
    \caption{Proposed contact mechanics package refactor.}
    \label{fig:contact_refactor}
\end{figure}

The final class diagram is on \reffig{fig:contact_refactor}. Here is a brief reminder of what each class is supposed to do in the final state:
\begin{enumerate}
    \item \texttt{Model}: base class for all models in Akantu, unchanged.
    \item \texttt{ContactMechanicsModel}: base class for all contact mechanics implementations. Contains a \texttt{SolidMechanicsModel} instance, and all the methods that are relevant to all the contact models.
    \item \texttt{ContactMechanicsPenaltyModel}: penalty contact mechanics model, fully handles both the contact and solid parts of the problem when \texttt{solveStep} is called.
    \item \texttt{ContactMechanicsInternodesModel}: INTERNODES contact mechanics model, fully handles the contact and solid parts of the problem when \texttt{solveStep} is called. This will be quite similar to the current state of INTERNODES, with some of the methods moved to the superclass.
\end{enumerate}

%\clearpage
%\subsection{Discussion}
%\label{sec:discussion}

%Flaws when same nodes that were dumped are added back in (no convergence) 

%Usually interpenetrating nodes won't be considered part of interface in next iteration

\clearpage
\section{Conclusion}
\label{sec:conclusion}
In conclusion, we detailed how the INTERNODES method works, identified many shortcomings of the previous implementations, and improved upon them to deliver a robust and well-tested implementation in Akantu, in the hope that its users will find it suitable for further research in numerical contact mechanics.

\clearpage
\bibliography{biblio.bib}

\clearpage
\section*{Appendix}

\subsection*{Refactor of penalty resolution classes}

While working on the contact mechanics code written for Akantu, we have identified a lot (800+ lines) of shared code in the \texttt{Resolution} family of classes. These classes we then refactored according to the diagram sketched in \reffig{fig:resolution}. In order not to pay the cost for virtual calls, the refactor is done by creating a templated \texttt{ResolutionPenalty} class. This class is now easily extendable with other penalty function beyond the linear and quadratic one. The implementation can be found on a feature branch of Akantu\footnote{\url{https://gitlab.com/akantu/akantu/-/tree/features/modularize-resolution-penalty-classes}}.

\begin{figure}[H]
    \centering
    \begin{subfigure}[b]{.49\linewidth}
        \centering
        \begin{tikzpicture}
    \draw[darkblue, ultra thick, ->] (0,-3) to (0, -1.8);
    \draw[darkblue, ultra thick, ->] (0, -1.5) to (0, -0.25);
    
    \node[rectangle, fill=darkblue] at (0, -3) {\color{white}{ResolutionPenaltyQuadratic}};
    \node[rectangle, fill=darkblue] at (0, -1.5) {\color{white}{ResolutionPenalty}};
    \node[rectangle, fill=darkblue] at (0, 0) {\color{white}{Resolution}};
  \end{tikzpicture}
        \caption{Initial class diagram of resolutions}\label{fig:resolution-initial}
    \end{subfigure}
    \begin{subfigure}[b]{.49\linewidth}
        \centering
        \begin{tikzpicture}
    \draw[darkblue, ultra thick, ->] (0, -1.5) to (0, -0.25);

    %\draw[darkblue, ultra thick] (-1.66, -1.5) rectangle (1.66, -2.5);

    \node[rectangle, fill=darkblue] at (0, -1.5) {\color{white}{ResolutionPenalty<PenaltyFunction>}};
    \node[rectangle, fill=darkblue] at (0, 0) {\color{white}{Resolution}};

  \end{tikzpicture}
        \caption{Refactored class diagram of resolutions}\label{fig:resolution-refactor}
    \end{subfigure}
    \caption{Combine linear and quadratic penalty resolution by templating the classes
    with a penalty function.}
    \label{fig:resolution}
\end{figure}

%\subsection*{Development of rigorously tested Python implementation of the INTERNODES method}

%In order to support and prototype the convergence check and the extension of the Akantu implementation to three dimensions, a Python reference implementation was developed. It was written in such a way that every function involved in the INTERNODES method is rigorously tested, and is hence suitable for prototyping. The code for this implementation is available on c4science\footnote{\url{https://c4science.ch/source/INTERNODES-CM/}}.

\end{document}
