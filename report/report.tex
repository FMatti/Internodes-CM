\documentclass[11pt, a4paper]{article}

\usepackage{stylesheet}

\institution{EPFL}
\project{Semester Project}
\title{Internodes garbage}
\author{Fabio Matti}
\supervisor{Dr Guillaume Anciaux \\ Raquel Dantas Batista}
\date{\today}

\begin{document}

\maketitle

\section{Introduction}
\label{sec:intro}

\clearpage

\section{The INTERNODES method for contact mechanics}
\label{sec:internodes}

\subsection{Radial basis interpolation}
\label{subsec:rbf-interpolation}

Interpolant of $g : \mathbb{R}^d \to \mathbb{R}$ at interpolation nodes $\boldsymbol{\xi}_1, \dots, \boldsymbol{\xi}_M$  with radius parameter $r$
\begin{equation}
    \Pi(\mathbf{x}) = \sum_{m=1}^M g_m \phi(\left\| \mathbf{x} - \boldsymbol{\xi}_m \right\|, r)
\end{equation}

\begin{figure}[ht]
    \centering
    \begin{tikzpicture}
    \node at (-2, 3) {$\mathbb{R}^d$};
    \shade[inner color=darkblue!80!white, outer color=white] (0, 0) circle (1.1);
    \fill[darkblue] (0, 0) circle (0.1) node[anchor=south] {$\boldsymbol{\xi}_{j}$};
    \shade[inner color=darkblue!80!white, outer color=white] (-2, 1) circle (0.8);
    \fill[darkblue] (-2, 1) circle (0.1) node[anchor=south] {$\boldsymbol{\xi}_{j+1}$};
    \shade[inner color=darkblue!80!white, outer color=white] (1.6, 2.1) circle (1.4);
    \fill[darkblue] (1.6, 2.1) circle (0.1) node[anchor=south] {$\boldsymbol{\xi}_{j-1}$};
    \draw[darkblue, <->] (0.1, 0) to node[midway, below] {$r_j$} (1.1, 0);
    \draw[darkblue, <->] (-1.9, 1) to node[midway, below] {$r_{j+1}$} (-1.2, 1);
    \draw[darkblue, <->] (1.7, 2.1) to node[midway, below] {$r_{j-1}$} (3, 2.1);
\end{tikzpicture}
    \caption{Radial basis functions}\label{fig:radialbasis}
\end{figure}

Wendland $C^2$ radial basis function
\begin{equation}
    \phi(\delta) = (1 - \delta)_{+}^4(1 + 4\delta) 
\end{equation}
with $\delta = \left\| \mathbf{x} - \boldsymbol{\xi}_m \right\| / r$ are well suited.

Denoting $\mathbf{g}_{\zeta} = (g(\boldsymbol{\zeta}_1), \dots, g(\boldsymbol{\zeta}_N))^T$ and $\mathbf{g}_{\xi} = (g(\boldsymbol{\xi}_1), \dots, g(\boldsymbol{\xi}_M))^T$ we can write
\begin{equation}
    \mathbf{g}_{\zeta} = \mathbf{D}_{NN}^{-1} \boldsymbol{\Phi}_{NM} \boldsymbol{\Phi}_{MM}^{-1} \mathbf{g}_{\xi}
\end{equation}
with the radial basis matrices
\begin{align}
    (\boldsymbol{\Phi}_{MM})_{ij} &= \phi(\left\| \boldsymbol{\xi}_i - \boldsymbol{\xi}_j \right\|, r_j) \hspace{10px} i, j \in \{1, \dots, M\} \\
    (\boldsymbol{\Phi}_{NM})_{ij} &= \phi(\left\| \boldsymbol{\zeta}_i - \boldsymbol{\xi}_j \right\|, r_j) \hspace{10px} i \in \{1, \dots, N\},~ j \in \{1, \dots, M\}
\end{align}

Deparis et. al. \cite{deparis} proposed two modifications:

\begin{itemize}
    \item Localized radius parameters for each node $r_{j}$, $j \in \{1, \dots, M\}$
    \item Rescaling with $\mathbf{D}_{NN}^{-1}$ to obtain exact interpolation of constant functions
\end{itemize}

\subsection{Radius parameters}
\label{subsec:radius-parameters}
Conditions \cite{voet}:

\begin{subequations}
\begin{align}
    \text{Limited number of supported interpolation nodes: } \notag \\
    \forall i: \# \{j \neq i: \left\| \boldsymbol{\xi}_i - \boldsymbol{\xi}_j \right\| < r_i \} &< 1 / \phi(c) \label{equ:support-condition} \\
    \text{Interpolation nodes not be too deep in another support: } \notag \\ 
    \exists c \in (0, 1), \forall i \neq j:~\left\| \boldsymbol{\xi}_i - \boldsymbol{\xi}_j \right\| &\geq c r_j \label{equ:c-condition} \\
    \text{All reference nodes in support of interpolation node: } \notag \\ 
    \exists C \in (c, 1), \forall i, \exists j:~\left\| \boldsymbol{\zeta}_i - \boldsymbol{\xi}_j \right\| &\leq C r_j \label{equ:C-condition}
\end{align}
\label{equ:conditions}
\end{subequations}

\begin{figure}[ht]
    \centering
    \begin{subfigure}[b]{.32\linewidth}
        \centering
        \begin{tikzpicture}[scale=0.5]
    \fill[darkblue!10!white] (0, 0) circle (0.75);
    \fill[darkblue!30!white] (0, 0) circle (0.35);

    \draw[darkblue, opacity=1] (-2, 1) to (0, 0) to (0.6, -1.1);
    \draw[darkblue, path fading=west, opacity=1] (-3.9, 1.5) to (-2, 1);
    \draw[darkblue, path fading=west, opacity=1] (-3.1, -0.5) to (-2, 1);
    \draw[darkblue, path fading=south, opacity=1] (-1.6, -0.5) to (-2, 1);
    \draw[darkblue, path fading=south, opacity=1] (0, 0) to (-1.4, -0.8);
    \draw[darkblue, path fading=south, opacity=1] (0.6, -1.1) to (1, -2.5);
    \draw[darkblue, path fading=west, opacity=1] (0.6, -1.1) to (-0.8, -1);
    \draw[darkblue, path fading=west, opacity=1] (0.6, -1.1) to (-0.2, -2.4);

    \draw[darkblue, opacity=1] (-0.5, 2) to (0.75, 0.45) to (2, -1);
    \draw[darkblue, path fading=west, opacity=1] (-0.5, 2) to (-1.2, 3);
    \draw[darkblue, path fading=north, opacity=1] (-0.5, 2) to (0, 3);
    \draw[darkblue, path fading=east, opacity=1] (-0.5, 2) to (0.8, 1.9);
    \draw[darkblue, path fading=north, opacity=1] (0.75, 0.45) to (0.9, 1.6);
    \draw[darkblue, path fading=east, opacity=1] (0.75, 0.45) to (2, 0.5);
    \draw[darkblue, path fading=north, opacity=1] (2, -1) to (2.1, 0.3);
    \draw[darkblue, path fading=east, opacity=1] (2, -1) to (3, -1.75);
    \draw[darkblue, path fading=east, opacity=1] (2, -1) to (3, -0.7);

    \fill[darkblue] (0, 0) circle (0.1) node[anchor=north east] {$\boldsymbol{\xi}_{j}$};
    \fill[darkblue] (-2, 1) circle (0.1) node[anchor=north east] {$\boldsymbol{\xi}_{j+1}$};
    \fill[darkblue] (0.6, -1.1) circle (0.1) node[anchor=north east] {$\boldsymbol{\xi}_{j-1}$};
    %\fill[white, draw=darkblue, thick] (0.75, 0.45) circle (0.08) node[anchor=south west, text=darkblue] {$\boldsymbol{\zeta}_{i}$};
    \fill[white, draw=darkblue, thick] (-0.5, 2) circle (0.08) node[anchor=south west, text=darkblue] {$\boldsymbol{\zeta}_{i+1}$};
    \fill[white, draw=darkblue, thick] (2, -1) circle (0.08) node[anchor=south west, text=darkblue] {$\boldsymbol{\zeta}_{i-1}$};

    \draw[white, draw=mainorange, thick] (0.75, 0.45) circle (0.08) node[anchor=south west, text=mainorange] {$\boldsymbol{\zeta}_{i}$};


    %\fill[darkblue!30!white] (3, -3.5) circle (0.4);
    %\fill[darkblue!10!white] (2, -3.5) circle (0.4);
    %\draw[darkblue, thick, <->] (2.0, -3.5) to (2.4, -3.5);
    %\draw[darkblue, thick, <->] (3.0, -3.5) to (3.4, -3.5);
    %\node[darkblue] at (3, -3) {$cr_j$};
    %\node[darkblue] at (2, -3) {$Cr_j$};
\end{tikzpicture}
        \caption{Too small}\label{fig:radiusparameters1}
    \end{subfigure}
    \begin{subfigure}[b]{.32\linewidth}
        \begin{tikzpicture}[scale=0.5]
    \fill[darkblue!10!white] (0, 0) circle (2.9);
    \fill[darkblue!30!white] (0, 0) circle (1.4);

    \draw[darkblue, opacity=1] (-2, 1) to (0, 0) to (0.6, -1.1);
    \draw[darkblue, path fading=west, opacity=1] (-3.9, 1.5) to (-2, 1);
    \draw[darkblue, path fading=west, opacity=1] (-3.1, -0.5) to (-2, 1);
    \draw[darkblue, path fading=south, opacity=1] (-1.6, -0.5) to (-2, 1);
    \draw[darkblue, path fading=south, opacity=1] (0, 0) to (-1.4, -0.8);
    \draw[darkblue, path fading=south, opacity=1] (0.6, -1.1) to (1, -2.5);
    \draw[darkblue, path fading=west, opacity=1] (0.6, -1.1) to (-0.8, -1);
    \draw[darkblue, path fading=west, opacity=1] (0.6, -1.1) to (-0.2, -2.4);

    \draw[darkblue, opacity=1] (-0.5, 2) to (0.75, 0.45) to (2, -1);
    \draw[darkblue, path fading=west, opacity=1] (-0.5, 2) to (-1.2, 3);
    \draw[darkblue, path fading=north, opacity=1] (-0.5, 2) to (0, 3);
    \draw[darkblue, path fading=east, opacity=1] (-0.5, 2) to (0.8, 1.9);
    \draw[darkblue, path fading=north, opacity=1] (0.75, 0.45) to (0.9, 1.6);
    \draw[darkblue, path fading=east, opacity=1] (0.75, 0.45) to (2, 0.5);
    \draw[darkblue, path fading=north, opacity=1] (2, -1) to (2.1, 0.3);
    \draw[darkblue, path fading=east, opacity=1] (2, -1) to (3, -1.75);
    \draw[darkblue, path fading=east, opacity=1] (2, -1) to (3, -0.7);

    \fill[darkblue] (0, 0) circle (0.1) node[anchor=north east] {$\boldsymbol{\xi}_{j}$};
    \fill[darkblue] (-2, 1) circle (0.1) node[anchor=north east] {$\boldsymbol{\xi}_{j+1}$};
    %\fill[darkblue] (0.6, -1.1) circle (0.1) node[anchor=north east] {$\boldsymbol{\xi}_{j-1}$};
    \fill[white, draw=darkblue, thick] (0.75, 0.45) circle (0.08) node[anchor=south west, text=darkblue] {$\boldsymbol{\zeta}_{i}$};
    \fill[white, draw=darkblue, thick] (-0.5, 2) circle (0.08) node[anchor=south west, text=darkblue] {$\boldsymbol{\zeta}_{i+1}$};
    \fill[white, draw=darkblue, thick] (2, -1) circle (0.08) node[anchor=south west, text=darkblue] {$\boldsymbol{\zeta}_{i-1}$};

    \fill[red] (0.6, -1.1, 0) circle (0.1) node[anchor=north east] {$\boldsymbol{\xi}_{j-1}$};
    
    %\fill[darkblue!30!white] (3, -3.5) circle (0.4);
    %\fill[darkblue!10!white] (2, -3.5) circle (0.4);
    %\draw[darkblue, thick, <->] (2.0, -3.5) to (2.4, -3.5);
    %\draw[darkblue, thick, <->] (3.0, -3.5) to (3.4, -3.5);
    %\node[darkblue] at (3, -3) {$cr_j$};
    %\node[darkblue] at (2, -3) {$Cr_j$};
\end{tikzpicture}
        \caption{Too big}\label{fig:radiusparameters2}
    \end{subfigure}
    \begin{subfigure}[b]{.32\linewidth}
        \begin{tikzpicture}[scale=0.5]
    \fill[darkblue!10!white] (0, 0) circle (1.7);
    \fill[darkblue!30!white] (0, 0) circle (0.8);

    \draw[darkblue, opacity=1] (-2, 1) to (0, 0) to (0.6, -1.1);
    \draw[darkblue, path fading=west, opacity=1] (-3.9, 1.5) to (-2, 1);
    \draw[darkblue, path fading=west, opacity=1] (-3.1, -0.5) to (-2, 1);
    \draw[darkblue, path fading=south, opacity=1] (-1.6, -0.5) to (-2, 1);
    \draw[darkblue, path fading=south, opacity=1] (0, 0) to (-1.4, -0.8);
    \draw[darkblue, path fading=south, opacity=1] (0.6, -1.1) to (1, -2.5);
    \draw[darkblue, path fading=west, opacity=1] (0.6, -1.1) to (-0.8, -1);
    \draw[darkblue, path fading=west, opacity=1] (0.6, -1.1) to (-0.2, -2.4);

    \draw[darkblue, opacity=1] (-0.5, 2) to (0.75, 0.45) to (2, -1);
    \draw[darkblue, path fading=west, opacity=1] (-0.5, 2) to (-1.2, 3);
    \draw[darkblue, path fading=north, opacity=1] (-0.5, 2) to (0, 3);
    \draw[darkblue, path fading=east, opacity=1] (-0.5, 2) to (0.8, 1.9);
    \draw[darkblue, path fading=north, opacity=1] (0.75, 0.45) to (0.9, 1.6);
    \draw[darkblue, path fading=east, opacity=1] (0.75, 0.45) to (2, 0.5);
    \draw[darkblue, path fading=north, opacity=1] (2, -1) to (2.1, 0.3);
    \draw[darkblue, path fading=east, opacity=1] (2, -1) to (3, -1.75);
    \draw[darkblue, path fading=east, opacity=1] (2, -1) to (3, -0.7);

    \fill[darkblue] (0, 0) circle (0.1) node[anchor=north east] {$\boldsymbol{\xi}_{j}$};
    \fill[darkblue] (-2, 1) circle (0.1) node[anchor=north east] {$\boldsymbol{\xi}_{j+1}$};
    \fill[darkblue] (0.6, -1.1) circle (0.1) node[anchor=north east] {$\boldsymbol{\xi}_{j-1}$};
    \fill[white, draw=darkblue, thick] (0.75, 0.45) circle (0.08) node[anchor=south west, text=darkblue] {$\boldsymbol{\zeta}_{i}$};
    \fill[white, draw=darkblue, thick] (-0.5, 2) circle (0.08) node[anchor=south west, text=darkblue] {$\boldsymbol{\zeta}_{i+1}$};
    \fill[white, draw=darkblue, thick] (2, -1) circle (0.08) node[anchor=south west, text=darkblue] {$\boldsymbol{\zeta}_{i-1}$};

    %\fill[darkblue!30!white] (3, -3.5) circle (0.4);
    %\fill[darkblue!10!white] (2, -3.5) circle (0.4);
    %\draw[darkblue, thick, <->] (2.0, -3.5) to (2.4, -3.5);
    %\draw[darkblue, thick, <->] (3.0, -3.5) to (3.4, -3.5);
    %\node[darkblue] at (3, -3) {$cr_j$};
    %\node[darkblue] at (2, -3) {$Cr_j$};
\end{tikzpicture}
        \caption{Just right}\label{fig:radiusparameters3}
    \end{subfigure}
    \caption{Finding radius parameters}
    \label{fig:radiusparameters}
\end{figure}

Distance matrix between nodes $\{\boldsymbol{\xi}_1, \boldsymbol{\xi}_2, \dots\}$
and $\{\boldsymbol{\zeta}_1, \boldsymbol{\zeta}_2, \dots\}$ is defined as

\begin{equation}
    \mathbf{D}^{\boldsymbol{\xi}, \boldsymbol{\zeta}}(i, j) =
        \left\| \boldsymbol{\xi}_i - \boldsymbol{\zeta}_j \right\|
    \label{equ:distance-matrix}
\end{equation}

\begin{algorithm}
    \caption{Computation of radius parameters}
    \begin{algorithmic}[1]
    \Require Positions of interpolation nodes $\{\boldsymbol{\xi}_1, \boldsymbol{\xi}_2, \dots\}$
    \Require Radial basis function $\phi: \mathbb{R} \to \mathbb{R}_{\geq 0}$
    \Require Constant $c \in (0, C)$
    \State Compute distance matrix $\mathbf{D}^{\boldsymbol{\xi}, \boldsymbol{\xi}}$ defined in \refequ{equ:distance-matrix}
    \State For each node $i$, compute distance to closest neighbor $d_i = \min_{j \neq i} \mathbf{D}^{\boldsymbol{\xi}, \boldsymbol{\xi}}(i, j)$
    \While{$c \leq C$}
        \State For each node $i$, let $r_i \gets d_i / c$ \Comment{\refcon{con:c-condition}}
        \State For each node $i$, count $n_i = \# \{j \neq i: \mathbf{D}^{\boldsymbol{\xi}, \boldsymbol{\xi}}(i, j) < r_i \}$
        \If{For all nodes $i$, $n_i < 1/\phi(c)$} \Comment{\refcon{con:support-condition}}
            \State \textbf{break}
        \EndIf
        \State Increase $c$
    \EndWhile
    \State \textbf{return} Radius parameters $\{r_1, r_2, \dots \}$
\end{algorithmic}
    \label{alg:radiusparameters}
\end{algorithm}

This algorithm has three main benefits over the previous implementation:

\begin{itemize}
    \item For uniform meshes (constant mesh size) all radial basis parameters will be the exact same
    \item The computation of the distance matrix is done outside the \texttt{while}-loops
    \item The algorithm will find interface nodes for a wider class of examples
\end{itemize}

\begin{algorithm}
    \caption{Search for interpolation nodes}
    \begin{algorithmic}[1]
    \Require Positions of primary nodes $\{\boldsymbol{\xi}_1, \boldsymbol{\xi}_2, \dots\}$
    \Require Positions of secondary nodes $\{\boldsymbol{\zeta}_1, \boldsymbol{\zeta}_2, \dots\}$
    \Require Radial basis function $\phi: \mathbb{R}^d \to \mathbb{R}_{\geq 0}$
    \Require Constant $C \in (0, 1)$
    \State Let $\mathcal{I} = \{1, 2, \dots \}$ and $\mathcal{J} = \{1, 2, \dots \}$ denote an index set of active nodes
    \State Compute distance matrix $\mathbf{D}^{\boldsymbol{\xi}, \boldsymbol{\zeta}}$ defined in \refequ{equ:distance-matrix}
    \While{$\mathcal{I}$ or $\mathcal{J}$ were modified in the previous iteration}
        \State Obtain radial basis parameters $r^{\boldsymbol{\xi}}_i$, $i \in \mathcal{I}$ and
        $r^{\boldsymbol{\zeta}}_j$, $j \in \mathcal{J}$ using \refalg{alg:radiusparameters}
        \State Remove isolated nodes $i \in \mathcal{I}$ with $\min_{j \in \mathcal{J}} \mathbf{D}^{\boldsymbol{\xi}, \boldsymbol{\zeta}}(i, j) \geq Cr^{\boldsymbol{\zeta}}_j$ \Comment{Condition \refequ{equ:C-condition}}
        \State Remove isolated nodes $j \in \mathcal{J}$ with $\min_{i \in \mathcal{I}} \mathbf{D}^{\boldsymbol{\xi}, \boldsymbol{\zeta}}(i, j) \geq Cr^{\boldsymbol{\xi}}_i$ \Comment{Condition \refequ{equ:C-condition}}
    \EndWhile
    \State \textbf{return} Sets of active nodes $\mathcal{I}$ and $\mathcal{J}$ with radius parameters $r^{\boldsymbol{\xi}}_i, i \in \mathcal{I}$ and $r^{\boldsymbol{\zeta}}_j, j \in \mathcal{J}$
\end{algorithmic}
    \label{alg:nodesearch}
\end{algorithm}


\begin{figure}[ht]
    \centering
    \begin{subfigure}[b]{\linewidth}
        \centering
        %% Creator: Matplotlib, PGF backend
%%
%% To include the figure in your LaTeX document, write
%%   \input{<filename>.pgf}
%%
%% Make sure the required packages are loaded in your preamble
%%   \usepackage{pgf}
%%
%% Also ensure that all the required font packages are loaded; for instance,
%% the lmodern package is sometimes necessary when using math font.
%%   \usepackage{lmodern}
%%
%% Figures using additional raster images can only be included by \input if
%% they are in the same directory as the main LaTeX file. For loading figures
%% from other directories you can use the `import` package
%%   \usepackage{import}
%%
%% and then include the figures with
%%   \import{<path to file>}{<filename>.pgf}
%%
%% Matplotlib used the following preamble
%%   
%%   \usepackage{fontspec}
%%   \setmainfont{DejaVuSans.ttf}[Path=\detokenize{/home/fabio/Internodes-CM/.venv/lib/python3.8/site-packages/matplotlib/mpl-data/fonts/ttf/}]
%%   \setsansfont{DejaVuSans.ttf}[Path=\detokenize{/home/fabio/Internodes-CM/.venv/lib/python3.8/site-packages/matplotlib/mpl-data/fonts/ttf/}]
%%   \setmonofont{DejaVuSansMono.ttf}[Path=\detokenize{/home/fabio/Internodes-CM/.venv/lib/python3.8/site-packages/matplotlib/mpl-data/fonts/ttf/}]
%%   \makeatletter\@ifpackageloaded{underscore}{}{\usepackage[strings]{underscore}}\makeatother
%%
\begingroup%
\makeatletter%
\begin{pgfpicture}%
\pgfpathrectangle{\pgfpointorigin}{\pgfqpoint{1.982500in}{1.432000in}}%
\pgfusepath{use as bounding box, clip}%
\begin{pgfscope}%
\pgfsetbuttcap%
\pgfsetmiterjoin%
\definecolor{currentfill}{rgb}{1.000000,1.000000,1.000000}%
\pgfsetfillcolor{currentfill}%
\pgfsetlinewidth{0.000000pt}%
\definecolor{currentstroke}{rgb}{1.000000,1.000000,1.000000}%
\pgfsetstrokecolor{currentstroke}%
\pgfsetdash{}{0pt}%
\pgfpathmoveto{\pgfqpoint{0.000000in}{0.000000in}}%
\pgfpathlineto{\pgfqpoint{1.982500in}{0.000000in}}%
\pgfpathlineto{\pgfqpoint{1.982500in}{1.432000in}}%
\pgfpathlineto{\pgfqpoint{0.000000in}{1.432000in}}%
\pgfpathlineto{\pgfqpoint{0.000000in}{0.000000in}}%
\pgfpathclose%
\pgfusepath{fill}%
\end{pgfscope}%
\begin{pgfscope}%
\pgfpathrectangle{\pgfqpoint{0.100000in}{0.100000in}}{\pgfqpoint{1.782500in}{1.232000in}}%
\pgfusepath{clip}%
\pgfsetrectcap%
\pgfsetroundjoin%
\pgfsetlinewidth{0.250937pt}%
\definecolor{currentstroke}{rgb}{0.054902,0.262745,0.486275}%
\pgfsetstrokecolor{currentstroke}%
\pgfsetdash{}{0pt}%
\pgfpathmoveto{\pgfqpoint{0.451098in}{0.100000in}}%
\pgfpathlineto{\pgfqpoint{0.181023in}{0.100000in}}%
\pgfpathmoveto{\pgfqpoint{0.721174in}{0.100000in}}%
\pgfpathlineto{\pgfqpoint{0.451098in}{0.100000in}}%
\pgfpathmoveto{\pgfqpoint{0.991250in}{0.100000in}}%
\pgfpathlineto{\pgfqpoint{0.721174in}{0.100000in}}%
\pgfpathmoveto{\pgfqpoint{1.261326in}{0.100000in}}%
\pgfpathlineto{\pgfqpoint{0.991250in}{0.100000in}}%
\pgfpathmoveto{\pgfqpoint{1.531402in}{0.100000in}}%
\pgfpathlineto{\pgfqpoint{1.801477in}{0.100000in}}%
\pgfpathmoveto{\pgfqpoint{1.531402in}{0.100000in}}%
\pgfpathlineto{\pgfqpoint{1.261326in}{0.100000in}}%
\pgfpathmoveto{\pgfqpoint{1.801477in}{0.408000in}}%
\pgfpathlineto{\pgfqpoint{1.801477in}{0.100000in}}%
\pgfpathmoveto{\pgfqpoint{1.801477in}{0.408000in}}%
\pgfpathlineto{\pgfqpoint{1.801477in}{0.716000in}}%
\pgfpathmoveto{\pgfqpoint{1.639432in}{0.716000in}}%
\pgfpathlineto{\pgfqpoint{1.801477in}{0.716000in}}%
\pgfpathmoveto{\pgfqpoint{1.639432in}{0.716000in}}%
\pgfpathlineto{\pgfqpoint{1.477386in}{0.716000in}}%
\pgfpathmoveto{\pgfqpoint{1.407938in}{0.716000in}}%
\pgfpathlineto{\pgfqpoint{1.477386in}{0.716000in}}%
\pgfpathmoveto{\pgfqpoint{1.338490in}{0.716000in}}%
\pgfpathlineto{\pgfqpoint{1.407938in}{0.716000in}}%
\pgfpathmoveto{\pgfqpoint{1.269042in}{0.716000in}}%
\pgfpathlineto{\pgfqpoint{1.338490in}{0.716000in}}%
\pgfpathmoveto{\pgfqpoint{1.199594in}{0.716000in}}%
\pgfpathlineto{\pgfqpoint{1.269042in}{0.716000in}}%
\pgfpathmoveto{\pgfqpoint{1.130146in}{0.716000in}}%
\pgfpathlineto{\pgfqpoint{1.199594in}{0.716000in}}%
\pgfpathmoveto{\pgfqpoint{1.060698in}{0.716000in}}%
\pgfpathlineto{\pgfqpoint{1.130146in}{0.716000in}}%
\pgfpathmoveto{\pgfqpoint{0.991250in}{0.716000in}}%
\pgfpathlineto{\pgfqpoint{1.060698in}{0.716000in}}%
\pgfpathmoveto{\pgfqpoint{0.921802in}{0.716000in}}%
\pgfpathlineto{\pgfqpoint{0.991250in}{0.716000in}}%
\pgfpathmoveto{\pgfqpoint{0.852354in}{0.716000in}}%
\pgfpathlineto{\pgfqpoint{0.921802in}{0.716000in}}%
\pgfpathmoveto{\pgfqpoint{0.782906in}{0.716000in}}%
\pgfpathlineto{\pgfqpoint{0.852354in}{0.716000in}}%
\pgfpathmoveto{\pgfqpoint{0.713458in}{0.716000in}}%
\pgfpathlineto{\pgfqpoint{0.782906in}{0.716000in}}%
\pgfpathmoveto{\pgfqpoint{0.644010in}{0.716000in}}%
\pgfpathlineto{\pgfqpoint{0.713458in}{0.716000in}}%
\pgfpathmoveto{\pgfqpoint{0.574562in}{0.716000in}}%
\pgfpathlineto{\pgfqpoint{0.505114in}{0.716000in}}%
\pgfpathmoveto{\pgfqpoint{0.574562in}{0.716000in}}%
\pgfpathlineto{\pgfqpoint{0.644010in}{0.716000in}}%
\pgfpathmoveto{\pgfqpoint{0.343068in}{0.716000in}}%
\pgfpathlineto{\pgfqpoint{0.505114in}{0.716000in}}%
\pgfpathmoveto{\pgfqpoint{0.343068in}{0.716000in}}%
\pgfpathlineto{\pgfqpoint{0.181023in}{0.716000in}}%
\pgfpathmoveto{\pgfqpoint{0.181023in}{0.408000in}}%
\pgfpathlineto{\pgfqpoint{0.181023in}{0.100000in}}%
\pgfpathmoveto{\pgfqpoint{0.181023in}{0.408000in}}%
\pgfpathlineto{\pgfqpoint{0.181023in}{0.716000in}}%
\pgfpathmoveto{\pgfqpoint{1.373214in}{0.396057in}}%
\pgfpathlineto{\pgfqpoint{1.261326in}{0.100000in}}%
\pgfpathmoveto{\pgfqpoint{1.082944in}{0.402689in}}%
\pgfpathlineto{\pgfqpoint{0.991250in}{0.100000in}}%
\pgfpathmoveto{\pgfqpoint{1.233147in}{0.520967in}}%
\pgfpathlineto{\pgfqpoint{1.373214in}{0.396057in}}%
\pgfpathmoveto{\pgfqpoint{1.233147in}{0.520967in}}%
\pgfpathlineto{\pgfqpoint{1.082944in}{0.402689in}}%
\pgfpathmoveto{\pgfqpoint{0.956117in}{0.522274in}}%
\pgfpathlineto{\pgfqpoint{1.082944in}{0.402689in}}%
\pgfpathmoveto{\pgfqpoint{0.956117in}{0.522274in}}%
\pgfpathlineto{\pgfqpoint{0.821141in}{0.416705in}}%
\pgfpathmoveto{\pgfqpoint{0.679173in}{0.523635in}}%
\pgfpathlineto{\pgfqpoint{0.821141in}{0.416705in}}%
\pgfpathmoveto{\pgfqpoint{0.679173in}{0.523635in}}%
\pgfpathlineto{\pgfqpoint{0.539838in}{0.401685in}}%
\pgfpathmoveto{\pgfqpoint{1.535558in}{0.497212in}}%
\pgfpathlineto{\pgfqpoint{1.801477in}{0.408000in}}%
\pgfpathmoveto{\pgfqpoint{1.535558in}{0.497212in}}%
\pgfpathlineto{\pgfqpoint{1.373214in}{0.396057in}}%
\pgfpathmoveto{\pgfqpoint{1.376927in}{0.550483in}}%
\pgfpathlineto{\pgfqpoint{1.373214in}{0.396057in}}%
\pgfpathmoveto{\pgfqpoint{1.376927in}{0.550483in}}%
\pgfpathlineto{\pgfqpoint{1.233147in}{0.520967in}}%
\pgfpathmoveto{\pgfqpoint{1.376927in}{0.550483in}}%
\pgfpathlineto{\pgfqpoint{1.535558in}{0.497212in}}%
\pgfpathmoveto{\pgfqpoint{1.093079in}{0.556190in}}%
\pgfpathlineto{\pgfqpoint{1.082944in}{0.402689in}}%
\pgfpathmoveto{\pgfqpoint{1.093079in}{0.556190in}}%
\pgfpathlineto{\pgfqpoint{1.233147in}{0.520967in}}%
\pgfpathmoveto{\pgfqpoint{1.093079in}{0.556190in}}%
\pgfpathlineto{\pgfqpoint{0.956117in}{0.522274in}}%
\pgfpathmoveto{\pgfqpoint{0.818220in}{0.558974in}}%
\pgfpathlineto{\pgfqpoint{0.821141in}{0.416705in}}%
\pgfpathmoveto{\pgfqpoint{0.818220in}{0.558974in}}%
\pgfpathlineto{\pgfqpoint{0.956117in}{0.522274in}}%
\pgfpathmoveto{\pgfqpoint{0.818220in}{0.558974in}}%
\pgfpathlineto{\pgfqpoint{0.679173in}{0.523635in}}%
\pgfpathmoveto{\pgfqpoint{0.525112in}{0.550213in}}%
\pgfpathlineto{\pgfqpoint{0.539838in}{0.401685in}}%
\pgfpathmoveto{\pgfqpoint{0.525112in}{0.550213in}}%
\pgfpathlineto{\pgfqpoint{0.679173in}{0.523635in}}%
\pgfpathmoveto{\pgfqpoint{0.356186in}{0.497584in}}%
\pgfpathlineto{\pgfqpoint{0.181023in}{0.716000in}}%
\pgfpathmoveto{\pgfqpoint{0.356186in}{0.497584in}}%
\pgfpathlineto{\pgfqpoint{0.343068in}{0.716000in}}%
\pgfpathmoveto{\pgfqpoint{0.356186in}{0.497584in}}%
\pgfpathlineto{\pgfqpoint{0.181023in}{0.408000in}}%
\pgfpathmoveto{\pgfqpoint{0.356186in}{0.497584in}}%
\pgfpathlineto{\pgfqpoint{0.539838in}{0.401685in}}%
\pgfpathmoveto{\pgfqpoint{0.356186in}{0.497584in}}%
\pgfpathlineto{\pgfqpoint{0.525112in}{0.550213in}}%
\pgfpathmoveto{\pgfqpoint{1.303766in}{0.622307in}}%
\pgfpathlineto{\pgfqpoint{1.338490in}{0.716000in}}%
\pgfpathmoveto{\pgfqpoint{1.303766in}{0.622307in}}%
\pgfpathlineto{\pgfqpoint{1.269042in}{0.716000in}}%
\pgfpathmoveto{\pgfqpoint{1.303766in}{0.622307in}}%
\pgfpathlineto{\pgfqpoint{1.233147in}{0.520967in}}%
\pgfpathmoveto{\pgfqpoint{1.303766in}{0.622307in}}%
\pgfpathlineto{\pgfqpoint{1.376927in}{0.550483in}}%
\pgfpathmoveto{\pgfqpoint{1.164870in}{0.623450in}}%
\pgfpathlineto{\pgfqpoint{1.199594in}{0.716000in}}%
\pgfpathmoveto{\pgfqpoint{1.164870in}{0.623450in}}%
\pgfpathlineto{\pgfqpoint{1.130146in}{0.716000in}}%
\pgfpathmoveto{\pgfqpoint{1.164870in}{0.623450in}}%
\pgfpathlineto{\pgfqpoint{1.233147in}{0.520967in}}%
\pgfpathmoveto{\pgfqpoint{1.164870in}{0.623450in}}%
\pgfpathlineto{\pgfqpoint{1.093079in}{0.556190in}}%
\pgfpathmoveto{\pgfqpoint{1.025974in}{0.623450in}}%
\pgfpathlineto{\pgfqpoint{1.060698in}{0.716000in}}%
\pgfpathmoveto{\pgfqpoint{1.025974in}{0.623450in}}%
\pgfpathlineto{\pgfqpoint{0.991250in}{0.716000in}}%
\pgfpathmoveto{\pgfqpoint{1.025974in}{0.623450in}}%
\pgfpathlineto{\pgfqpoint{0.956117in}{0.522274in}}%
\pgfpathmoveto{\pgfqpoint{1.025974in}{0.623450in}}%
\pgfpathlineto{\pgfqpoint{1.093079in}{0.556190in}}%
\pgfpathmoveto{\pgfqpoint{0.887078in}{0.623460in}}%
\pgfpathlineto{\pgfqpoint{0.921802in}{0.716000in}}%
\pgfpathmoveto{\pgfqpoint{0.887078in}{0.623460in}}%
\pgfpathlineto{\pgfqpoint{0.852354in}{0.716000in}}%
\pgfpathmoveto{\pgfqpoint{0.887078in}{0.623460in}}%
\pgfpathlineto{\pgfqpoint{0.956117in}{0.522274in}}%
\pgfpathmoveto{\pgfqpoint{0.887078in}{0.623460in}}%
\pgfpathlineto{\pgfqpoint{0.818220in}{0.558974in}}%
\pgfpathmoveto{\pgfqpoint{0.748182in}{0.623460in}}%
\pgfpathlineto{\pgfqpoint{0.782906in}{0.716000in}}%
\pgfpathmoveto{\pgfqpoint{0.748182in}{0.623460in}}%
\pgfpathlineto{\pgfqpoint{0.713458in}{0.716000in}}%
\pgfpathmoveto{\pgfqpoint{0.748182in}{0.623460in}}%
\pgfpathlineto{\pgfqpoint{0.679173in}{0.523635in}}%
\pgfpathmoveto{\pgfqpoint{0.748182in}{0.623460in}}%
\pgfpathlineto{\pgfqpoint{0.818220in}{0.558974in}}%
\pgfpathmoveto{\pgfqpoint{0.609286in}{0.623460in}}%
\pgfpathlineto{\pgfqpoint{0.644010in}{0.716000in}}%
\pgfpathmoveto{\pgfqpoint{0.609286in}{0.623460in}}%
\pgfpathlineto{\pgfqpoint{0.574562in}{0.716000in}}%
\pgfpathmoveto{\pgfqpoint{0.609286in}{0.623460in}}%
\pgfpathlineto{\pgfqpoint{0.679173in}{0.523635in}}%
\pgfpathmoveto{\pgfqpoint{0.609286in}{0.623460in}}%
\pgfpathlineto{\pgfqpoint{0.525112in}{0.550213in}}%
\pgfpathmoveto{\pgfqpoint{1.453902in}{0.624729in}}%
\pgfpathlineto{\pgfqpoint{1.477386in}{0.716000in}}%
\pgfpathmoveto{\pgfqpoint{1.453902in}{0.624729in}}%
\pgfpathlineto{\pgfqpoint{1.407938in}{0.716000in}}%
\pgfpathmoveto{\pgfqpoint{1.453902in}{0.624729in}}%
\pgfpathlineto{\pgfqpoint{1.535558in}{0.497212in}}%
\pgfpathmoveto{\pgfqpoint{1.453902in}{0.624729in}}%
\pgfpathlineto{\pgfqpoint{1.376927in}{0.550483in}}%
\pgfpathmoveto{\pgfqpoint{0.939717in}{0.332038in}}%
\pgfpathlineto{\pgfqpoint{0.991250in}{0.100000in}}%
\pgfpathmoveto{\pgfqpoint{0.939717in}{0.332038in}}%
\pgfpathlineto{\pgfqpoint{1.082944in}{0.402689in}}%
\pgfpathmoveto{\pgfqpoint{0.939717in}{0.332038in}}%
\pgfpathlineto{\pgfqpoint{0.821141in}{0.416705in}}%
\pgfpathmoveto{\pgfqpoint{0.939717in}{0.332038in}}%
\pgfpathlineto{\pgfqpoint{0.956117in}{0.522274in}}%
\pgfpathmoveto{\pgfqpoint{0.433515in}{0.614303in}}%
\pgfpathlineto{\pgfqpoint{0.505114in}{0.716000in}}%
\pgfpathmoveto{\pgfqpoint{0.433515in}{0.614303in}}%
\pgfpathlineto{\pgfqpoint{0.343068in}{0.716000in}}%
\pgfpathmoveto{\pgfqpoint{0.433515in}{0.614303in}}%
\pgfpathlineto{\pgfqpoint{0.525112in}{0.550213in}}%
\pgfpathmoveto{\pgfqpoint{0.433515in}{0.614303in}}%
\pgfpathlineto{\pgfqpoint{0.356186in}{0.497584in}}%
\pgfpathmoveto{\pgfqpoint{1.374740in}{0.644995in}}%
\pgfpathlineto{\pgfqpoint{1.407938in}{0.716000in}}%
\pgfpathmoveto{\pgfqpoint{1.374740in}{0.644995in}}%
\pgfpathlineto{\pgfqpoint{1.338490in}{0.716000in}}%
\pgfpathmoveto{\pgfqpoint{1.374740in}{0.644995in}}%
\pgfpathlineto{\pgfqpoint{1.376927in}{0.550483in}}%
\pgfpathmoveto{\pgfqpoint{1.374740in}{0.644995in}}%
\pgfpathlineto{\pgfqpoint{1.303766in}{0.622307in}}%
\pgfpathmoveto{\pgfqpoint{1.374740in}{0.644995in}}%
\pgfpathlineto{\pgfqpoint{1.453902in}{0.624729in}}%
\pgfpathmoveto{\pgfqpoint{1.094954in}{0.647018in}}%
\pgfpathlineto{\pgfqpoint{1.130146in}{0.716000in}}%
\pgfpathmoveto{\pgfqpoint{1.094954in}{0.647018in}}%
\pgfpathlineto{\pgfqpoint{1.060698in}{0.716000in}}%
\pgfpathmoveto{\pgfqpoint{1.094954in}{0.647018in}}%
\pgfpathlineto{\pgfqpoint{1.093079in}{0.556190in}}%
\pgfpathmoveto{\pgfqpoint{1.094954in}{0.647018in}}%
\pgfpathlineto{\pgfqpoint{1.164870in}{0.623450in}}%
\pgfpathmoveto{\pgfqpoint{1.094954in}{0.647018in}}%
\pgfpathlineto{\pgfqpoint{1.025974in}{0.623450in}}%
\pgfpathmoveto{\pgfqpoint{1.234318in}{0.644309in}}%
\pgfpathlineto{\pgfqpoint{1.269042in}{0.716000in}}%
\pgfpathmoveto{\pgfqpoint{1.234318in}{0.644309in}}%
\pgfpathlineto{\pgfqpoint{1.199594in}{0.716000in}}%
\pgfpathmoveto{\pgfqpoint{1.234318in}{0.644309in}}%
\pgfpathlineto{\pgfqpoint{1.233147in}{0.520967in}}%
\pgfpathmoveto{\pgfqpoint{1.234318in}{0.644309in}}%
\pgfpathlineto{\pgfqpoint{1.303766in}{0.622307in}}%
\pgfpathmoveto{\pgfqpoint{1.234318in}{0.644309in}}%
\pgfpathlineto{\pgfqpoint{1.164870in}{0.623450in}}%
\pgfpathmoveto{\pgfqpoint{0.956444in}{0.640237in}}%
\pgfpathlineto{\pgfqpoint{0.991250in}{0.716000in}}%
\pgfpathmoveto{\pgfqpoint{0.956444in}{0.640237in}}%
\pgfpathlineto{\pgfqpoint{0.921802in}{0.716000in}}%
\pgfpathmoveto{\pgfqpoint{0.956444in}{0.640237in}}%
\pgfpathlineto{\pgfqpoint{0.956117in}{0.522274in}}%
\pgfpathmoveto{\pgfqpoint{0.956444in}{0.640237in}}%
\pgfpathlineto{\pgfqpoint{1.025974in}{0.623450in}}%
\pgfpathmoveto{\pgfqpoint{0.956444in}{0.640237in}}%
\pgfpathlineto{\pgfqpoint{0.887078in}{0.623460in}}%
\pgfpathmoveto{\pgfqpoint{0.817690in}{0.645978in}}%
\pgfpathlineto{\pgfqpoint{0.852354in}{0.716000in}}%
\pgfpathmoveto{\pgfqpoint{0.817690in}{0.645978in}}%
\pgfpathlineto{\pgfqpoint{0.782906in}{0.716000in}}%
\pgfpathmoveto{\pgfqpoint{0.817690in}{0.645978in}}%
\pgfpathlineto{\pgfqpoint{0.818220in}{0.558974in}}%
\pgfpathmoveto{\pgfqpoint{0.817690in}{0.645978in}}%
\pgfpathlineto{\pgfqpoint{0.887078in}{0.623460in}}%
\pgfpathmoveto{\pgfqpoint{0.817690in}{0.645978in}}%
\pgfpathlineto{\pgfqpoint{0.748182in}{0.623460in}}%
\pgfpathmoveto{\pgfqpoint{1.188376in}{0.303943in}}%
\pgfpathlineto{\pgfqpoint{0.991250in}{0.100000in}}%
\pgfpathmoveto{\pgfqpoint{1.188376in}{0.303943in}}%
\pgfpathlineto{\pgfqpoint{1.261326in}{0.100000in}}%
\pgfpathmoveto{\pgfqpoint{1.188376in}{0.303943in}}%
\pgfpathlineto{\pgfqpoint{1.373214in}{0.396057in}}%
\pgfpathmoveto{\pgfqpoint{1.188376in}{0.303943in}}%
\pgfpathlineto{\pgfqpoint{1.082944in}{0.402689in}}%
\pgfpathmoveto{\pgfqpoint{1.188376in}{0.303943in}}%
\pgfpathlineto{\pgfqpoint{1.233147in}{0.520967in}}%
\pgfpathmoveto{\pgfqpoint{0.696453in}{0.313796in}}%
\pgfpathlineto{\pgfqpoint{0.721174in}{0.100000in}}%
\pgfpathmoveto{\pgfqpoint{0.696453in}{0.313796in}}%
\pgfpathlineto{\pgfqpoint{0.821141in}{0.416705in}}%
\pgfpathmoveto{\pgfqpoint{0.696453in}{0.313796in}}%
\pgfpathlineto{\pgfqpoint{0.539838in}{0.401685in}}%
\pgfpathmoveto{\pgfqpoint{0.696453in}{0.313796in}}%
\pgfpathlineto{\pgfqpoint{0.679173in}{0.523635in}}%
\pgfpathmoveto{\pgfqpoint{0.678822in}{0.640511in}}%
\pgfpathlineto{\pgfqpoint{0.713458in}{0.716000in}}%
\pgfpathmoveto{\pgfqpoint{0.678822in}{0.640511in}}%
\pgfpathlineto{\pgfqpoint{0.644010in}{0.716000in}}%
\pgfpathmoveto{\pgfqpoint{0.678822in}{0.640511in}}%
\pgfpathlineto{\pgfqpoint{0.679173in}{0.523635in}}%
\pgfpathmoveto{\pgfqpoint{0.678822in}{0.640511in}}%
\pgfpathlineto{\pgfqpoint{0.748182in}{0.623460in}}%
\pgfpathmoveto{\pgfqpoint{0.678822in}{0.640511in}}%
\pgfpathlineto{\pgfqpoint{0.609286in}{0.623460in}}%
\pgfpathmoveto{\pgfqpoint{1.556669in}{0.624742in}}%
\pgfpathlineto{\pgfqpoint{1.477386in}{0.716000in}}%
\pgfpathmoveto{\pgfqpoint{1.556669in}{0.624742in}}%
\pgfpathlineto{\pgfqpoint{1.639432in}{0.716000in}}%
\pgfpathmoveto{\pgfqpoint{1.556669in}{0.624742in}}%
\pgfpathlineto{\pgfqpoint{1.535558in}{0.497212in}}%
\pgfpathmoveto{\pgfqpoint{1.556669in}{0.624742in}}%
\pgfpathlineto{\pgfqpoint{1.453902in}{0.624729in}}%
\pgfpathmoveto{\pgfqpoint{0.539838in}{0.644311in}}%
\pgfpathlineto{\pgfqpoint{0.505114in}{0.716000in}}%
\pgfpathmoveto{\pgfqpoint{0.539838in}{0.644311in}}%
\pgfpathlineto{\pgfqpoint{0.574562in}{0.716000in}}%
\pgfpathmoveto{\pgfqpoint{0.539838in}{0.644311in}}%
\pgfpathlineto{\pgfqpoint{0.525112in}{0.550213in}}%
\pgfpathmoveto{\pgfqpoint{0.539838in}{0.644311in}}%
\pgfpathlineto{\pgfqpoint{0.609286in}{0.623460in}}%
\pgfpathmoveto{\pgfqpoint{0.539838in}{0.644311in}}%
\pgfpathlineto{\pgfqpoint{0.433515in}{0.614303in}}%
\pgfpathmoveto{\pgfqpoint{0.379905in}{0.288250in}}%
\pgfpathlineto{\pgfqpoint{0.181023in}{0.100000in}}%
\pgfpathmoveto{\pgfqpoint{0.379905in}{0.288250in}}%
\pgfpathlineto{\pgfqpoint{0.451098in}{0.100000in}}%
\pgfpathmoveto{\pgfqpoint{0.379905in}{0.288250in}}%
\pgfpathlineto{\pgfqpoint{0.181023in}{0.408000in}}%
\pgfpathmoveto{\pgfqpoint{0.379905in}{0.288250in}}%
\pgfpathlineto{\pgfqpoint{0.539838in}{0.401685in}}%
\pgfpathmoveto{\pgfqpoint{0.379905in}{0.288250in}}%
\pgfpathlineto{\pgfqpoint{0.356186in}{0.497584in}}%
\pgfpathmoveto{\pgfqpoint{1.528881in}{0.301240in}}%
\pgfpathlineto{\pgfqpoint{1.801477in}{0.100000in}}%
\pgfpathmoveto{\pgfqpoint{1.528881in}{0.301240in}}%
\pgfpathlineto{\pgfqpoint{1.261326in}{0.100000in}}%
\pgfpathmoveto{\pgfqpoint{1.528881in}{0.301240in}}%
\pgfpathlineto{\pgfqpoint{1.531402in}{0.100000in}}%
\pgfpathmoveto{\pgfqpoint{1.528881in}{0.301240in}}%
\pgfpathlineto{\pgfqpoint{1.801477in}{0.408000in}}%
\pgfpathmoveto{\pgfqpoint{1.528881in}{0.301240in}}%
\pgfpathlineto{\pgfqpoint{1.373214in}{0.396057in}}%
\pgfpathmoveto{\pgfqpoint{1.528881in}{0.301240in}}%
\pgfpathlineto{\pgfqpoint{1.535558in}{0.497212in}}%
\pgfpathmoveto{\pgfqpoint{1.666923in}{0.592391in}}%
\pgfpathlineto{\pgfqpoint{1.801477in}{0.716000in}}%
\pgfpathmoveto{\pgfqpoint{1.666923in}{0.592391in}}%
\pgfpathlineto{\pgfqpoint{1.801477in}{0.408000in}}%
\pgfpathmoveto{\pgfqpoint{1.666923in}{0.592391in}}%
\pgfpathlineto{\pgfqpoint{1.639432in}{0.716000in}}%
\pgfpathmoveto{\pgfqpoint{1.666923in}{0.592391in}}%
\pgfpathlineto{\pgfqpoint{1.535558in}{0.497212in}}%
\pgfpathmoveto{\pgfqpoint{1.666923in}{0.592391in}}%
\pgfpathlineto{\pgfqpoint{1.556669in}{0.624742in}}%
\pgfpathmoveto{\pgfqpoint{0.842327in}{0.230907in}}%
\pgfpathlineto{\pgfqpoint{0.721174in}{0.100000in}}%
\pgfpathmoveto{\pgfqpoint{0.842327in}{0.230907in}}%
\pgfpathlineto{\pgfqpoint{0.991250in}{0.100000in}}%
\pgfpathmoveto{\pgfqpoint{0.842327in}{0.230907in}}%
\pgfpathlineto{\pgfqpoint{0.821141in}{0.416705in}}%
\pgfpathmoveto{\pgfqpoint{0.842327in}{0.230907in}}%
\pgfpathlineto{\pgfqpoint{0.939717in}{0.332038in}}%
\pgfpathmoveto{\pgfqpoint{0.842327in}{0.230907in}}%
\pgfpathlineto{\pgfqpoint{0.696453in}{0.313796in}}%
\pgfpathmoveto{\pgfqpoint{0.557694in}{0.240746in}}%
\pgfpathlineto{\pgfqpoint{0.451098in}{0.100000in}}%
\pgfpathmoveto{\pgfqpoint{0.557694in}{0.240746in}}%
\pgfpathlineto{\pgfqpoint{0.721174in}{0.100000in}}%
\pgfpathmoveto{\pgfqpoint{0.557694in}{0.240746in}}%
\pgfpathlineto{\pgfqpoint{0.539838in}{0.401685in}}%
\pgfpathmoveto{\pgfqpoint{0.557694in}{0.240746in}}%
\pgfpathlineto{\pgfqpoint{0.696453in}{0.313796in}}%
\pgfpathmoveto{\pgfqpoint{0.557694in}{0.240746in}}%
\pgfpathlineto{\pgfqpoint{0.379905in}{0.288250in}}%
\pgfpathlineto{\pgfqpoint{0.379905in}{0.288250in}}%
\pgfusepath{stroke}%
\end{pgfscope}%
\begin{pgfscope}%
\pgfpathrectangle{\pgfqpoint{0.100000in}{0.100000in}}{\pgfqpoint{1.782500in}{1.232000in}}%
\pgfusepath{clip}%
\pgfsetrectcap%
\pgfsetroundjoin%
\pgfsetlinewidth{0.250937pt}%
\definecolor{currentstroke}{rgb}{0.835294,0.321569,0.035294}%
\pgfsetstrokecolor{currentstroke}%
\pgfsetdash{}{0pt}%
\pgfpathmoveto{\pgfqpoint{0.505114in}{1.001892in}}%
\pgfpathlineto{\pgfqpoint{0.451098in}{1.270400in}}%
\pgfpathmoveto{\pgfqpoint{1.531402in}{1.270400in}}%
\pgfpathlineto{\pgfqpoint{1.477386in}{1.001892in}}%
\pgfpathmoveto{\pgfqpoint{0.721174in}{1.270400in}}%
\pgfpathlineto{\pgfqpoint{0.991250in}{1.270400in}}%
\pgfpathmoveto{\pgfqpoint{0.721174in}{1.270400in}}%
\pgfpathlineto{\pgfqpoint{0.451098in}{1.270400in}}%
\pgfpathmoveto{\pgfqpoint{0.548824in}{0.917012in}}%
\pgfpathlineto{\pgfqpoint{0.505114in}{1.001892in}}%
\pgfpathmoveto{\pgfqpoint{0.603831in}{0.841156in}}%
\pgfpathlineto{\pgfqpoint{0.548824in}{0.917012in}}%
\pgfpathmoveto{\pgfqpoint{0.668731in}{0.776261in}}%
\pgfpathlineto{\pgfqpoint{0.603831in}{0.841156in}}%
\pgfpathmoveto{\pgfqpoint{0.741867in}{0.723983in}}%
\pgfpathlineto{\pgfqpoint{0.668731in}{0.776261in}}%
\pgfpathmoveto{\pgfqpoint{0.821370in}{0.685658in}}%
\pgfpathlineto{\pgfqpoint{0.741867in}{0.723983in}}%
\pgfpathmoveto{\pgfqpoint{0.905212in}{0.662265in}}%
\pgfpathlineto{\pgfqpoint{0.821370in}{0.685658in}}%
\pgfpathmoveto{\pgfqpoint{0.991250in}{0.654400in}}%
\pgfpathlineto{\pgfqpoint{0.905212in}{0.662265in}}%
\pgfpathmoveto{\pgfqpoint{1.077288in}{0.662265in}}%
\pgfpathlineto{\pgfqpoint{0.991250in}{0.654400in}}%
\pgfpathmoveto{\pgfqpoint{1.161130in}{0.685658in}}%
\pgfpathlineto{\pgfqpoint{1.077288in}{0.662265in}}%
\pgfpathmoveto{\pgfqpoint{1.240633in}{0.723983in}}%
\pgfpathlineto{\pgfqpoint{1.161130in}{0.685658in}}%
\pgfpathmoveto{\pgfqpoint{1.313769in}{0.776261in}}%
\pgfpathlineto{\pgfqpoint{1.240633in}{0.723983in}}%
\pgfpathmoveto{\pgfqpoint{1.378669in}{0.841156in}}%
\pgfpathlineto{\pgfqpoint{1.313769in}{0.776261in}}%
\pgfpathmoveto{\pgfqpoint{1.433676in}{0.917012in}}%
\pgfpathlineto{\pgfqpoint{1.477386in}{1.001892in}}%
\pgfpathmoveto{\pgfqpoint{1.433676in}{0.917012in}}%
\pgfpathlineto{\pgfqpoint{1.378669in}{0.841156in}}%
\pgfpathmoveto{\pgfqpoint{1.261326in}{1.270400in}}%
\pgfpathlineto{\pgfqpoint{0.991250in}{1.270400in}}%
\pgfpathmoveto{\pgfqpoint{1.261326in}{1.270400in}}%
\pgfpathlineto{\pgfqpoint{1.531402in}{1.270400in}}%
\pgfpathmoveto{\pgfqpoint{0.837830in}{1.028470in}}%
\pgfpathlineto{\pgfqpoint{0.991250in}{1.270400in}}%
\pgfpathmoveto{\pgfqpoint{0.837830in}{1.028470in}}%
\pgfpathlineto{\pgfqpoint{0.721174in}{1.270400in}}%
\pgfpathmoveto{\pgfqpoint{0.961801in}{0.851412in}}%
\pgfpathlineto{\pgfqpoint{1.113091in}{0.944697in}}%
\pgfpathmoveto{\pgfqpoint{0.961801in}{0.851412in}}%
\pgfpathlineto{\pgfqpoint{0.837830in}{1.028470in}}%
\pgfpathmoveto{\pgfqpoint{1.273497in}{1.004211in}}%
\pgfpathlineto{\pgfqpoint{1.113091in}{0.944697in}}%
\pgfpathmoveto{\pgfqpoint{0.674832in}{1.049989in}}%
\pgfpathlineto{\pgfqpoint{0.721174in}{1.270400in}}%
\pgfpathmoveto{\pgfqpoint{0.674832in}{1.049989in}}%
\pgfpathlineto{\pgfqpoint{0.837830in}{1.028470in}}%
\pgfpathmoveto{\pgfqpoint{1.093189in}{0.806475in}}%
\pgfpathlineto{\pgfqpoint{1.077288in}{0.662265in}}%
\pgfpathmoveto{\pgfqpoint{1.093189in}{0.806475in}}%
\pgfpathlineto{\pgfqpoint{1.161130in}{0.685658in}}%
\pgfpathmoveto{\pgfqpoint{1.093189in}{0.806475in}}%
\pgfpathlineto{\pgfqpoint{1.113091in}{0.944697in}}%
\pgfpathmoveto{\pgfqpoint{1.093189in}{0.806475in}}%
\pgfpathlineto{\pgfqpoint{0.961801in}{0.851412in}}%
\pgfpathmoveto{\pgfqpoint{1.212025in}{0.863682in}}%
\pgfpathlineto{\pgfqpoint{1.240633in}{0.723983in}}%
\pgfpathmoveto{\pgfqpoint{1.212025in}{0.863682in}}%
\pgfpathlineto{\pgfqpoint{1.313769in}{0.776261in}}%
\pgfpathmoveto{\pgfqpoint{1.212025in}{0.863682in}}%
\pgfpathlineto{\pgfqpoint{1.113091in}{0.944697in}}%
\pgfpathmoveto{\pgfqpoint{1.212025in}{0.863682in}}%
\pgfpathlineto{\pgfqpoint{1.273497in}{1.004211in}}%
\pgfpathmoveto{\pgfqpoint{1.212025in}{0.863682in}}%
\pgfpathlineto{\pgfqpoint{1.093189in}{0.806475in}}%
\pgfpathmoveto{\pgfqpoint{0.822435in}{0.830135in}}%
\pgfpathlineto{\pgfqpoint{0.741867in}{0.723983in}}%
\pgfpathmoveto{\pgfqpoint{0.822435in}{0.830135in}}%
\pgfpathlineto{\pgfqpoint{0.821370in}{0.685658in}}%
\pgfpathmoveto{\pgfqpoint{0.822435in}{0.830135in}}%
\pgfpathlineto{\pgfqpoint{0.837830in}{1.028470in}}%
\pgfpathmoveto{\pgfqpoint{0.822435in}{0.830135in}}%
\pgfpathlineto{\pgfqpoint{0.961801in}{0.851412in}}%
\pgfpathmoveto{\pgfqpoint{0.716798in}{0.910757in}}%
\pgfpathlineto{\pgfqpoint{0.603831in}{0.841156in}}%
\pgfpathmoveto{\pgfqpoint{0.716798in}{0.910757in}}%
\pgfpathlineto{\pgfqpoint{0.668731in}{0.776261in}}%
\pgfpathmoveto{\pgfqpoint{0.716798in}{0.910757in}}%
\pgfpathlineto{\pgfqpoint{0.837830in}{1.028470in}}%
\pgfpathmoveto{\pgfqpoint{0.716798in}{0.910757in}}%
\pgfpathlineto{\pgfqpoint{0.674832in}{1.049989in}}%
\pgfpathmoveto{\pgfqpoint{0.716798in}{0.910757in}}%
\pgfpathlineto{\pgfqpoint{0.822435in}{0.830135in}}%
\pgfpathmoveto{\pgfqpoint{1.362495in}{1.129252in}}%
\pgfpathlineto{\pgfqpoint{1.477386in}{1.001892in}}%
\pgfpathmoveto{\pgfqpoint{1.362495in}{1.129252in}}%
\pgfpathlineto{\pgfqpoint{1.531402in}{1.270400in}}%
\pgfpathmoveto{\pgfqpoint{1.362495in}{1.129252in}}%
\pgfpathlineto{\pgfqpoint{1.261326in}{1.270400in}}%
\pgfpathmoveto{\pgfqpoint{1.362495in}{1.129252in}}%
\pgfpathlineto{\pgfqpoint{1.273497in}{1.004211in}}%
\pgfpathmoveto{\pgfqpoint{0.953154in}{0.728383in}}%
\pgfpathlineto{\pgfqpoint{0.905212in}{0.662265in}}%
\pgfpathmoveto{\pgfqpoint{0.953154in}{0.728383in}}%
\pgfpathlineto{\pgfqpoint{0.991250in}{0.654400in}}%
\pgfpathmoveto{\pgfqpoint{0.953154in}{0.728383in}}%
\pgfpathlineto{\pgfqpoint{0.961801in}{0.851412in}}%
\pgfpathmoveto{\pgfqpoint{1.297122in}{0.872566in}}%
\pgfpathlineto{\pgfqpoint{1.313769in}{0.776261in}}%
\pgfpathmoveto{\pgfqpoint{1.297122in}{0.872566in}}%
\pgfpathlineto{\pgfqpoint{1.378669in}{0.841156in}}%
\pgfpathmoveto{\pgfqpoint{1.297122in}{0.872566in}}%
\pgfpathlineto{\pgfqpoint{1.273497in}{1.004211in}}%
\pgfpathmoveto{\pgfqpoint{1.297122in}{0.872566in}}%
\pgfpathlineto{\pgfqpoint{1.212025in}{0.863682in}}%
\pgfpathmoveto{\pgfqpoint{1.173657in}{0.778274in}}%
\pgfpathlineto{\pgfqpoint{1.161130in}{0.685658in}}%
\pgfpathmoveto{\pgfqpoint{1.173657in}{0.778274in}}%
\pgfpathlineto{\pgfqpoint{1.240633in}{0.723983in}}%
\pgfpathmoveto{\pgfqpoint{1.173657in}{0.778274in}}%
\pgfpathlineto{\pgfqpoint{1.093189in}{0.806475in}}%
\pgfpathmoveto{\pgfqpoint{1.173657in}{0.778274in}}%
\pgfpathlineto{\pgfqpoint{1.212025in}{0.863682in}}%
\pgfpathmoveto{\pgfqpoint{1.375529in}{1.013033in}}%
\pgfpathlineto{\pgfqpoint{1.477386in}{1.001892in}}%
\pgfpathmoveto{\pgfqpoint{1.375529in}{1.013033in}}%
\pgfpathlineto{\pgfqpoint{1.433676in}{0.917012in}}%
\pgfpathmoveto{\pgfqpoint{1.375529in}{1.013033in}}%
\pgfpathlineto{\pgfqpoint{1.273497in}{1.004211in}}%
\pgfpathmoveto{\pgfqpoint{1.375529in}{1.013033in}}%
\pgfpathlineto{\pgfqpoint{1.362495in}{1.129252in}}%
\pgfpathmoveto{\pgfqpoint{0.742156in}{0.817783in}}%
\pgfpathlineto{\pgfqpoint{0.668731in}{0.776261in}}%
\pgfpathmoveto{\pgfqpoint{0.742156in}{0.817783in}}%
\pgfpathlineto{\pgfqpoint{0.741867in}{0.723983in}}%
\pgfpathmoveto{\pgfqpoint{0.742156in}{0.817783in}}%
\pgfpathlineto{\pgfqpoint{0.822435in}{0.830135in}}%
\pgfpathmoveto{\pgfqpoint{0.742156in}{0.817783in}}%
\pgfpathlineto{\pgfqpoint{0.716798in}{0.910757in}}%
\pgfpathmoveto{\pgfqpoint{0.633716in}{0.933208in}}%
\pgfpathlineto{\pgfqpoint{0.548824in}{0.917012in}}%
\pgfpathmoveto{\pgfqpoint{0.633716in}{0.933208in}}%
\pgfpathlineto{\pgfqpoint{0.603831in}{0.841156in}}%
\pgfpathmoveto{\pgfqpoint{0.633716in}{0.933208in}}%
\pgfpathlineto{\pgfqpoint{0.674832in}{1.049989in}}%
\pgfpathmoveto{\pgfqpoint{0.633716in}{0.933208in}}%
\pgfpathlineto{\pgfqpoint{0.716798in}{0.910757in}}%
\pgfpathmoveto{\pgfqpoint{0.572102in}{1.136818in}}%
\pgfpathlineto{\pgfqpoint{0.451098in}{1.270400in}}%
\pgfpathmoveto{\pgfqpoint{0.572102in}{1.136818in}}%
\pgfpathlineto{\pgfqpoint{0.505114in}{1.001892in}}%
\pgfpathmoveto{\pgfqpoint{0.572102in}{1.136818in}}%
\pgfpathlineto{\pgfqpoint{0.721174in}{1.270400in}}%
\pgfpathmoveto{\pgfqpoint{0.572102in}{1.136818in}}%
\pgfpathlineto{\pgfqpoint{0.674832in}{1.049989in}}%
\pgfpathmoveto{\pgfqpoint{1.029231in}{0.730021in}}%
\pgfpathlineto{\pgfqpoint{0.991250in}{0.654400in}}%
\pgfpathmoveto{\pgfqpoint{1.029231in}{0.730021in}}%
\pgfpathlineto{\pgfqpoint{1.077288in}{0.662265in}}%
\pgfpathmoveto{\pgfqpoint{1.029231in}{0.730021in}}%
\pgfpathlineto{\pgfqpoint{0.961801in}{0.851412in}}%
\pgfpathmoveto{\pgfqpoint{1.029231in}{0.730021in}}%
\pgfpathlineto{\pgfqpoint{1.093189in}{0.806475in}}%
\pgfpathmoveto{\pgfqpoint{1.029231in}{0.730021in}}%
\pgfpathlineto{\pgfqpoint{0.953154in}{0.728383in}}%
\pgfpathmoveto{\pgfqpoint{0.878278in}{0.743820in}}%
\pgfpathlineto{\pgfqpoint{0.821370in}{0.685658in}}%
\pgfpathmoveto{\pgfqpoint{0.878278in}{0.743820in}}%
\pgfpathlineto{\pgfqpoint{0.905212in}{0.662265in}}%
\pgfpathmoveto{\pgfqpoint{0.878278in}{0.743820in}}%
\pgfpathlineto{\pgfqpoint{0.961801in}{0.851412in}}%
\pgfpathmoveto{\pgfqpoint{0.878278in}{0.743820in}}%
\pgfpathlineto{\pgfqpoint{0.822435in}{0.830135in}}%
\pgfpathmoveto{\pgfqpoint{0.878278in}{0.743820in}}%
\pgfpathlineto{\pgfqpoint{0.953154in}{0.728383in}}%
\pgfpathmoveto{\pgfqpoint{1.357947in}{0.924566in}}%
\pgfpathlineto{\pgfqpoint{1.378669in}{0.841156in}}%
\pgfpathmoveto{\pgfqpoint{1.357947in}{0.924566in}}%
\pgfpathlineto{\pgfqpoint{1.433676in}{0.917012in}}%
\pgfpathmoveto{\pgfqpoint{1.357947in}{0.924566in}}%
\pgfpathlineto{\pgfqpoint{1.273497in}{1.004211in}}%
\pgfpathmoveto{\pgfqpoint{1.357947in}{0.924566in}}%
\pgfpathlineto{\pgfqpoint{1.297122in}{0.872566in}}%
\pgfpathmoveto{\pgfqpoint{1.357947in}{0.924566in}}%
\pgfpathlineto{\pgfqpoint{1.375529in}{1.013033in}}%
\pgfpathmoveto{\pgfqpoint{1.015627in}{1.045348in}}%
\pgfpathlineto{\pgfqpoint{0.991250in}{1.270400in}}%
\pgfpathmoveto{\pgfqpoint{1.015627in}{1.045348in}}%
\pgfpathlineto{\pgfqpoint{1.113091in}{0.944697in}}%
\pgfpathmoveto{\pgfqpoint{1.015627in}{1.045348in}}%
\pgfpathlineto{\pgfqpoint{0.837830in}{1.028470in}}%
\pgfpathmoveto{\pgfqpoint{1.015627in}{1.045348in}}%
\pgfpathlineto{\pgfqpoint{0.961801in}{0.851412in}}%
\pgfpathmoveto{\pgfqpoint{1.169547in}{1.110718in}}%
\pgfpathlineto{\pgfqpoint{0.991250in}{1.270400in}}%
\pgfpathmoveto{\pgfqpoint{1.169547in}{1.110718in}}%
\pgfpathlineto{\pgfqpoint{1.261326in}{1.270400in}}%
\pgfpathmoveto{\pgfqpoint{1.169547in}{1.110718in}}%
\pgfpathlineto{\pgfqpoint{1.113091in}{0.944697in}}%
\pgfpathmoveto{\pgfqpoint{1.169547in}{1.110718in}}%
\pgfpathlineto{\pgfqpoint{1.273497in}{1.004211in}}%
\pgfpathmoveto{\pgfqpoint{1.169547in}{1.110718in}}%
\pgfpathlineto{\pgfqpoint{1.362495in}{1.129252in}}%
\pgfpathmoveto{\pgfqpoint{1.169547in}{1.110718in}}%
\pgfpathlineto{\pgfqpoint{1.015627in}{1.045348in}}%
\pgfpathmoveto{\pgfqpoint{0.586918in}{1.007784in}}%
\pgfpathlineto{\pgfqpoint{0.505114in}{1.001892in}}%
\pgfpathmoveto{\pgfqpoint{0.586918in}{1.007784in}}%
\pgfpathlineto{\pgfqpoint{0.548824in}{0.917012in}}%
\pgfpathmoveto{\pgfqpoint{0.586918in}{1.007784in}}%
\pgfpathlineto{\pgfqpoint{0.674832in}{1.049989in}}%
\pgfpathmoveto{\pgfqpoint{0.586918in}{1.007784in}}%
\pgfpathlineto{\pgfqpoint{0.633716in}{0.933208in}}%
\pgfpathmoveto{\pgfqpoint{0.586918in}{1.007784in}}%
\pgfpathlineto{\pgfqpoint{0.572102in}{1.136818in}}%
\pgfpathlineto{\pgfqpoint{0.572102in}{1.136818in}}%
\pgfusepath{stroke}%
\end{pgfscope}%
\begin{pgfscope}%
\pgfpathrectangle{\pgfqpoint{0.100000in}{0.100000in}}{\pgfqpoint{1.782500in}{1.232000in}}%
\pgfusepath{clip}%
\pgfsetbuttcap%
\pgfsetroundjoin%
\pgfsetlinewidth{0.501875pt}%
\definecolor{currentstroke}{rgb}{0.054902,0.262745,0.486275}%
\pgfsetstrokecolor{currentstroke}%
\pgfsetdash{}{0pt}%
\pgfpathmoveto{\pgfqpoint{1.477386in}{0.697627in}}%
\pgfpathcurveto{\pgfqpoint{1.482259in}{0.697627in}}{\pgfqpoint{1.486933in}{0.699563in}}{\pgfqpoint{1.490378in}{0.703008in}}%
\pgfpathcurveto{\pgfqpoint{1.493824in}{0.706454in}}{\pgfqpoint{1.495760in}{0.711127in}}{\pgfqpoint{1.495760in}{0.716000in}}%
\pgfpathcurveto{\pgfqpoint{1.495760in}{0.720873in}}{\pgfqpoint{1.493824in}{0.725546in}}{\pgfqpoint{1.490378in}{0.728992in}}%
\pgfpathcurveto{\pgfqpoint{1.486933in}{0.732437in}}{\pgfqpoint{1.482259in}{0.734373in}}{\pgfqpoint{1.477386in}{0.734373in}}%
\pgfpathcurveto{\pgfqpoint{1.472514in}{0.734373in}}{\pgfqpoint{1.467840in}{0.732437in}}{\pgfqpoint{1.464394in}{0.728992in}}%
\pgfpathcurveto{\pgfqpoint{1.460949in}{0.725546in}}{\pgfqpoint{1.459013in}{0.720873in}}{\pgfqpoint{1.459013in}{0.716000in}}%
\pgfpathcurveto{\pgfqpoint{1.459013in}{0.711127in}}{\pgfqpoint{1.460949in}{0.706454in}}{\pgfqpoint{1.464394in}{0.703008in}}%
\pgfpathcurveto{\pgfqpoint{1.467840in}{0.699563in}}{\pgfqpoint{1.472514in}{0.697627in}}{\pgfqpoint{1.477386in}{0.697627in}}%
\pgfpathlineto{\pgfqpoint{1.477386in}{0.697627in}}%
\pgfpathclose%
\pgfusepath{stroke}%
\end{pgfscope}%
\begin{pgfscope}%
\pgfpathrectangle{\pgfqpoint{0.100000in}{0.100000in}}{\pgfqpoint{1.782500in}{1.232000in}}%
\pgfusepath{clip}%
\pgfsetbuttcap%
\pgfsetroundjoin%
\pgfsetlinewidth{0.501875pt}%
\definecolor{currentstroke}{rgb}{0.054902,0.262745,0.486275}%
\pgfsetstrokecolor{currentstroke}%
\pgfsetdash{}{0pt}%
\pgfpathmoveto{\pgfqpoint{0.505114in}{0.697627in}}%
\pgfpathcurveto{\pgfqpoint{0.509986in}{0.697627in}}{\pgfqpoint{0.514660in}{0.699563in}}{\pgfqpoint{0.518106in}{0.703008in}}%
\pgfpathcurveto{\pgfqpoint{0.521551in}{0.706454in}}{\pgfqpoint{0.523487in}{0.711127in}}{\pgfqpoint{0.523487in}{0.716000in}}%
\pgfpathcurveto{\pgfqpoint{0.523487in}{0.720873in}}{\pgfqpoint{0.521551in}{0.725546in}}{\pgfqpoint{0.518106in}{0.728992in}}%
\pgfpathcurveto{\pgfqpoint{0.514660in}{0.732437in}}{\pgfqpoint{0.509986in}{0.734373in}}{\pgfqpoint{0.505114in}{0.734373in}}%
\pgfpathcurveto{\pgfqpoint{0.500241in}{0.734373in}}{\pgfqpoint{0.495567in}{0.732437in}}{\pgfqpoint{0.492122in}{0.728992in}}%
\pgfpathcurveto{\pgfqpoint{0.488676in}{0.725546in}}{\pgfqpoint{0.486740in}{0.720873in}}{\pgfqpoint{0.486740in}{0.716000in}}%
\pgfpathcurveto{\pgfqpoint{0.486740in}{0.711127in}}{\pgfqpoint{0.488676in}{0.706454in}}{\pgfqpoint{0.492122in}{0.703008in}}%
\pgfpathcurveto{\pgfqpoint{0.495567in}{0.699563in}}{\pgfqpoint{0.500241in}{0.697627in}}{\pgfqpoint{0.505114in}{0.697627in}}%
\pgfpathlineto{\pgfqpoint{0.505114in}{0.697627in}}%
\pgfpathclose%
\pgfusepath{stroke}%
\end{pgfscope}%
\begin{pgfscope}%
\pgfpathrectangle{\pgfqpoint{0.100000in}{0.100000in}}{\pgfqpoint{1.782500in}{1.232000in}}%
\pgfusepath{clip}%
\pgfsetbuttcap%
\pgfsetroundjoin%
\pgfsetlinewidth{0.501875pt}%
\definecolor{currentstroke}{rgb}{0.054902,0.262745,0.486275}%
\pgfsetstrokecolor{currentstroke}%
\pgfsetdash{}{0pt}%
\pgfpathmoveto{\pgfqpoint{1.407938in}{0.697627in}}%
\pgfpathcurveto{\pgfqpoint{1.412811in}{0.697627in}}{\pgfqpoint{1.417485in}{0.699563in}}{\pgfqpoint{1.420930in}{0.703008in}}%
\pgfpathcurveto{\pgfqpoint{1.424376in}{0.706454in}}{\pgfqpoint{1.426312in}{0.711127in}}{\pgfqpoint{1.426312in}{0.716000in}}%
\pgfpathcurveto{\pgfqpoint{1.426312in}{0.720873in}}{\pgfqpoint{1.424376in}{0.725546in}}{\pgfqpoint{1.420930in}{0.728992in}}%
\pgfpathcurveto{\pgfqpoint{1.417485in}{0.732437in}}{\pgfqpoint{1.412811in}{0.734373in}}{\pgfqpoint{1.407938in}{0.734373in}}%
\pgfpathcurveto{\pgfqpoint{1.403066in}{0.734373in}}{\pgfqpoint{1.398392in}{0.732437in}}{\pgfqpoint{1.394946in}{0.728992in}}%
\pgfpathcurveto{\pgfqpoint{1.391501in}{0.725546in}}{\pgfqpoint{1.389565in}{0.720873in}}{\pgfqpoint{1.389565in}{0.716000in}}%
\pgfpathcurveto{\pgfqpoint{1.389565in}{0.711127in}}{\pgfqpoint{1.391501in}{0.706454in}}{\pgfqpoint{1.394946in}{0.703008in}}%
\pgfpathcurveto{\pgfqpoint{1.398392in}{0.699563in}}{\pgfqpoint{1.403066in}{0.697627in}}{\pgfqpoint{1.407938in}{0.697627in}}%
\pgfpathlineto{\pgfqpoint{1.407938in}{0.697627in}}%
\pgfpathclose%
\pgfusepath{stroke}%
\end{pgfscope}%
\begin{pgfscope}%
\pgfpathrectangle{\pgfqpoint{0.100000in}{0.100000in}}{\pgfqpoint{1.782500in}{1.232000in}}%
\pgfusepath{clip}%
\pgfsetbuttcap%
\pgfsetroundjoin%
\pgfsetlinewidth{0.501875pt}%
\definecolor{currentstroke}{rgb}{0.054902,0.262745,0.486275}%
\pgfsetstrokecolor{currentstroke}%
\pgfsetdash{}{0pt}%
\pgfpathmoveto{\pgfqpoint{1.338490in}{0.697627in}}%
\pgfpathcurveto{\pgfqpoint{1.343363in}{0.697627in}}{\pgfqpoint{1.348037in}{0.699563in}}{\pgfqpoint{1.351482in}{0.703008in}}%
\pgfpathcurveto{\pgfqpoint{1.354928in}{0.706454in}}{\pgfqpoint{1.356864in}{0.711127in}}{\pgfqpoint{1.356864in}{0.716000in}}%
\pgfpathcurveto{\pgfqpoint{1.356864in}{0.720873in}}{\pgfqpoint{1.354928in}{0.725546in}}{\pgfqpoint{1.351482in}{0.728992in}}%
\pgfpathcurveto{\pgfqpoint{1.348037in}{0.732437in}}{\pgfqpoint{1.343363in}{0.734373in}}{\pgfqpoint{1.338490in}{0.734373in}}%
\pgfpathcurveto{\pgfqpoint{1.333618in}{0.734373in}}{\pgfqpoint{1.328944in}{0.732437in}}{\pgfqpoint{1.325498in}{0.728992in}}%
\pgfpathcurveto{\pgfqpoint{1.322053in}{0.725546in}}{\pgfqpoint{1.320117in}{0.720873in}}{\pgfqpoint{1.320117in}{0.716000in}}%
\pgfpathcurveto{\pgfqpoint{1.320117in}{0.711127in}}{\pgfqpoint{1.322053in}{0.706454in}}{\pgfqpoint{1.325498in}{0.703008in}}%
\pgfpathcurveto{\pgfqpoint{1.328944in}{0.699563in}}{\pgfqpoint{1.333618in}{0.697627in}}{\pgfqpoint{1.338490in}{0.697627in}}%
\pgfpathlineto{\pgfqpoint{1.338490in}{0.697627in}}%
\pgfpathclose%
\pgfusepath{stroke}%
\end{pgfscope}%
\begin{pgfscope}%
\pgfpathrectangle{\pgfqpoint{0.100000in}{0.100000in}}{\pgfqpoint{1.782500in}{1.232000in}}%
\pgfusepath{clip}%
\pgfsetbuttcap%
\pgfsetroundjoin%
\pgfsetlinewidth{0.501875pt}%
\definecolor{currentstroke}{rgb}{0.054902,0.262745,0.486275}%
\pgfsetstrokecolor{currentstroke}%
\pgfsetdash{}{0pt}%
\pgfpathmoveto{\pgfqpoint{1.269042in}{0.697627in}}%
\pgfpathcurveto{\pgfqpoint{1.273915in}{0.697627in}}{\pgfqpoint{1.278589in}{0.699563in}}{\pgfqpoint{1.282034in}{0.703008in}}%
\pgfpathcurveto{\pgfqpoint{1.285480in}{0.706454in}}{\pgfqpoint{1.287415in}{0.711127in}}{\pgfqpoint{1.287415in}{0.716000in}}%
\pgfpathcurveto{\pgfqpoint{1.287415in}{0.720873in}}{\pgfqpoint{1.285480in}{0.725546in}}{\pgfqpoint{1.282034in}{0.728992in}}%
\pgfpathcurveto{\pgfqpoint{1.278589in}{0.732437in}}{\pgfqpoint{1.273915in}{0.734373in}}{\pgfqpoint{1.269042in}{0.734373in}}%
\pgfpathcurveto{\pgfqpoint{1.264170in}{0.734373in}}{\pgfqpoint{1.259496in}{0.732437in}}{\pgfqpoint{1.256050in}{0.728992in}}%
\pgfpathcurveto{\pgfqpoint{1.252605in}{0.725546in}}{\pgfqpoint{1.250669in}{0.720873in}}{\pgfqpoint{1.250669in}{0.716000in}}%
\pgfpathcurveto{\pgfqpoint{1.250669in}{0.711127in}}{\pgfqpoint{1.252605in}{0.706454in}}{\pgfqpoint{1.256050in}{0.703008in}}%
\pgfpathcurveto{\pgfqpoint{1.259496in}{0.699563in}}{\pgfqpoint{1.264170in}{0.697627in}}{\pgfqpoint{1.269042in}{0.697627in}}%
\pgfpathlineto{\pgfqpoint{1.269042in}{0.697627in}}%
\pgfpathclose%
\pgfusepath{stroke}%
\end{pgfscope}%
\begin{pgfscope}%
\pgfpathrectangle{\pgfqpoint{0.100000in}{0.100000in}}{\pgfqpoint{1.782500in}{1.232000in}}%
\pgfusepath{clip}%
\pgfsetbuttcap%
\pgfsetroundjoin%
\pgfsetlinewidth{0.501875pt}%
\definecolor{currentstroke}{rgb}{0.054902,0.262745,0.486275}%
\pgfsetstrokecolor{currentstroke}%
\pgfsetdash{}{0pt}%
\pgfpathmoveto{\pgfqpoint{1.199594in}{0.697627in}}%
\pgfpathcurveto{\pgfqpoint{1.204467in}{0.697627in}}{\pgfqpoint{1.209141in}{0.699563in}}{\pgfqpoint{1.212586in}{0.703008in}}%
\pgfpathcurveto{\pgfqpoint{1.216032in}{0.706454in}}{\pgfqpoint{1.217967in}{0.711127in}}{\pgfqpoint{1.217967in}{0.716000in}}%
\pgfpathcurveto{\pgfqpoint{1.217967in}{0.720873in}}{\pgfqpoint{1.216032in}{0.725546in}}{\pgfqpoint{1.212586in}{0.728992in}}%
\pgfpathcurveto{\pgfqpoint{1.209141in}{0.732437in}}{\pgfqpoint{1.204467in}{0.734373in}}{\pgfqpoint{1.199594in}{0.734373in}}%
\pgfpathcurveto{\pgfqpoint{1.194722in}{0.734373in}}{\pgfqpoint{1.190048in}{0.732437in}}{\pgfqpoint{1.186602in}{0.728992in}}%
\pgfpathcurveto{\pgfqpoint{1.183157in}{0.725546in}}{\pgfqpoint{1.181221in}{0.720873in}}{\pgfqpoint{1.181221in}{0.716000in}}%
\pgfpathcurveto{\pgfqpoint{1.181221in}{0.711127in}}{\pgfqpoint{1.183157in}{0.706454in}}{\pgfqpoint{1.186602in}{0.703008in}}%
\pgfpathcurveto{\pgfqpoint{1.190048in}{0.699563in}}{\pgfqpoint{1.194722in}{0.697627in}}{\pgfqpoint{1.199594in}{0.697627in}}%
\pgfpathlineto{\pgfqpoint{1.199594in}{0.697627in}}%
\pgfpathclose%
\pgfusepath{stroke}%
\end{pgfscope}%
\begin{pgfscope}%
\pgfpathrectangle{\pgfqpoint{0.100000in}{0.100000in}}{\pgfqpoint{1.782500in}{1.232000in}}%
\pgfusepath{clip}%
\pgfsetbuttcap%
\pgfsetroundjoin%
\pgfsetlinewidth{0.501875pt}%
\definecolor{currentstroke}{rgb}{0.054902,0.262745,0.486275}%
\pgfsetstrokecolor{currentstroke}%
\pgfsetdash{}{0pt}%
\pgfpathmoveto{\pgfqpoint{1.130146in}{0.697627in}}%
\pgfpathcurveto{\pgfqpoint{1.135019in}{0.697627in}}{\pgfqpoint{1.139692in}{0.699563in}}{\pgfqpoint{1.143138in}{0.703008in}}%
\pgfpathcurveto{\pgfqpoint{1.146583in}{0.706454in}}{\pgfqpoint{1.148519in}{0.711127in}}{\pgfqpoint{1.148519in}{0.716000in}}%
\pgfpathcurveto{\pgfqpoint{1.148519in}{0.720873in}}{\pgfqpoint{1.146583in}{0.725546in}}{\pgfqpoint{1.143138in}{0.728992in}}%
\pgfpathcurveto{\pgfqpoint{1.139692in}{0.732437in}}{\pgfqpoint{1.135019in}{0.734373in}}{\pgfqpoint{1.130146in}{0.734373in}}%
\pgfpathcurveto{\pgfqpoint{1.125273in}{0.734373in}}{\pgfqpoint{1.120600in}{0.732437in}}{\pgfqpoint{1.117154in}{0.728992in}}%
\pgfpathcurveto{\pgfqpoint{1.113709in}{0.725546in}}{\pgfqpoint{1.111773in}{0.720873in}}{\pgfqpoint{1.111773in}{0.716000in}}%
\pgfpathcurveto{\pgfqpoint{1.111773in}{0.711127in}}{\pgfqpoint{1.113709in}{0.706454in}}{\pgfqpoint{1.117154in}{0.703008in}}%
\pgfpathcurveto{\pgfqpoint{1.120600in}{0.699563in}}{\pgfqpoint{1.125273in}{0.697627in}}{\pgfqpoint{1.130146in}{0.697627in}}%
\pgfpathlineto{\pgfqpoint{1.130146in}{0.697627in}}%
\pgfpathclose%
\pgfusepath{stroke}%
\end{pgfscope}%
\begin{pgfscope}%
\pgfpathrectangle{\pgfqpoint{0.100000in}{0.100000in}}{\pgfqpoint{1.782500in}{1.232000in}}%
\pgfusepath{clip}%
\pgfsetbuttcap%
\pgfsetroundjoin%
\pgfsetlinewidth{0.501875pt}%
\definecolor{currentstroke}{rgb}{0.054902,0.262745,0.486275}%
\pgfsetstrokecolor{currentstroke}%
\pgfsetdash{}{0pt}%
\pgfpathmoveto{\pgfqpoint{1.060698in}{0.697627in}}%
\pgfpathcurveto{\pgfqpoint{1.065571in}{0.697627in}}{\pgfqpoint{1.070244in}{0.699563in}}{\pgfqpoint{1.073690in}{0.703008in}}%
\pgfpathcurveto{\pgfqpoint{1.077135in}{0.706454in}}{\pgfqpoint{1.079071in}{0.711127in}}{\pgfqpoint{1.079071in}{0.716000in}}%
\pgfpathcurveto{\pgfqpoint{1.079071in}{0.720873in}}{\pgfqpoint{1.077135in}{0.725546in}}{\pgfqpoint{1.073690in}{0.728992in}}%
\pgfpathcurveto{\pgfqpoint{1.070244in}{0.732437in}}{\pgfqpoint{1.065571in}{0.734373in}}{\pgfqpoint{1.060698in}{0.734373in}}%
\pgfpathcurveto{\pgfqpoint{1.055825in}{0.734373in}}{\pgfqpoint{1.051152in}{0.732437in}}{\pgfqpoint{1.047706in}{0.728992in}}%
\pgfpathcurveto{\pgfqpoint{1.044261in}{0.725546in}}{\pgfqpoint{1.042325in}{0.720873in}}{\pgfqpoint{1.042325in}{0.716000in}}%
\pgfpathcurveto{\pgfqpoint{1.042325in}{0.711127in}}{\pgfqpoint{1.044261in}{0.706454in}}{\pgfqpoint{1.047706in}{0.703008in}}%
\pgfpathcurveto{\pgfqpoint{1.051152in}{0.699563in}}{\pgfqpoint{1.055825in}{0.697627in}}{\pgfqpoint{1.060698in}{0.697627in}}%
\pgfpathlineto{\pgfqpoint{1.060698in}{0.697627in}}%
\pgfpathclose%
\pgfusepath{stroke}%
\end{pgfscope}%
\begin{pgfscope}%
\pgfpathrectangle{\pgfqpoint{0.100000in}{0.100000in}}{\pgfqpoint{1.782500in}{1.232000in}}%
\pgfusepath{clip}%
\pgfsetbuttcap%
\pgfsetroundjoin%
\pgfsetlinewidth{0.501875pt}%
\definecolor{currentstroke}{rgb}{0.054902,0.262745,0.486275}%
\pgfsetstrokecolor{currentstroke}%
\pgfsetdash{}{0pt}%
\pgfpathmoveto{\pgfqpoint{0.991250in}{0.697627in}}%
\pgfpathcurveto{\pgfqpoint{0.996123in}{0.697627in}}{\pgfqpoint{1.000796in}{0.699563in}}{\pgfqpoint{1.004242in}{0.703008in}}%
\pgfpathcurveto{\pgfqpoint{1.007687in}{0.706454in}}{\pgfqpoint{1.009623in}{0.711127in}}{\pgfqpoint{1.009623in}{0.716000in}}%
\pgfpathcurveto{\pgfqpoint{1.009623in}{0.720873in}}{\pgfqpoint{1.007687in}{0.725546in}}{\pgfqpoint{1.004242in}{0.728992in}}%
\pgfpathcurveto{\pgfqpoint{1.000796in}{0.732437in}}{\pgfqpoint{0.996123in}{0.734373in}}{\pgfqpoint{0.991250in}{0.734373in}}%
\pgfpathcurveto{\pgfqpoint{0.986377in}{0.734373in}}{\pgfqpoint{0.981704in}{0.732437in}}{\pgfqpoint{0.978258in}{0.728992in}}%
\pgfpathcurveto{\pgfqpoint{0.974813in}{0.725546in}}{\pgfqpoint{0.972877in}{0.720873in}}{\pgfqpoint{0.972877in}{0.716000in}}%
\pgfpathcurveto{\pgfqpoint{0.972877in}{0.711127in}}{\pgfqpoint{0.974813in}{0.706454in}}{\pgfqpoint{0.978258in}{0.703008in}}%
\pgfpathcurveto{\pgfqpoint{0.981704in}{0.699563in}}{\pgfqpoint{0.986377in}{0.697627in}}{\pgfqpoint{0.991250in}{0.697627in}}%
\pgfpathlineto{\pgfqpoint{0.991250in}{0.697627in}}%
\pgfpathclose%
\pgfusepath{stroke}%
\end{pgfscope}%
\begin{pgfscope}%
\pgfpathrectangle{\pgfqpoint{0.100000in}{0.100000in}}{\pgfqpoint{1.782500in}{1.232000in}}%
\pgfusepath{clip}%
\pgfsetbuttcap%
\pgfsetroundjoin%
\pgfsetlinewidth{0.501875pt}%
\definecolor{currentstroke}{rgb}{0.054902,0.262745,0.486275}%
\pgfsetstrokecolor{currentstroke}%
\pgfsetdash{}{0pt}%
\pgfpathmoveto{\pgfqpoint{0.921802in}{0.697627in}}%
\pgfpathcurveto{\pgfqpoint{0.926675in}{0.697627in}}{\pgfqpoint{0.931348in}{0.699563in}}{\pgfqpoint{0.934794in}{0.703008in}}%
\pgfpathcurveto{\pgfqpoint{0.938239in}{0.706454in}}{\pgfqpoint{0.940175in}{0.711127in}}{\pgfqpoint{0.940175in}{0.716000in}}%
\pgfpathcurveto{\pgfqpoint{0.940175in}{0.720873in}}{\pgfqpoint{0.938239in}{0.725546in}}{\pgfqpoint{0.934794in}{0.728992in}}%
\pgfpathcurveto{\pgfqpoint{0.931348in}{0.732437in}}{\pgfqpoint{0.926675in}{0.734373in}}{\pgfqpoint{0.921802in}{0.734373in}}%
\pgfpathcurveto{\pgfqpoint{0.916929in}{0.734373in}}{\pgfqpoint{0.912256in}{0.732437in}}{\pgfqpoint{0.908810in}{0.728992in}}%
\pgfpathcurveto{\pgfqpoint{0.905365in}{0.725546in}}{\pgfqpoint{0.903429in}{0.720873in}}{\pgfqpoint{0.903429in}{0.716000in}}%
\pgfpathcurveto{\pgfqpoint{0.903429in}{0.711127in}}{\pgfqpoint{0.905365in}{0.706454in}}{\pgfqpoint{0.908810in}{0.703008in}}%
\pgfpathcurveto{\pgfqpoint{0.912256in}{0.699563in}}{\pgfqpoint{0.916929in}{0.697627in}}{\pgfqpoint{0.921802in}{0.697627in}}%
\pgfpathlineto{\pgfqpoint{0.921802in}{0.697627in}}%
\pgfpathclose%
\pgfusepath{stroke}%
\end{pgfscope}%
\begin{pgfscope}%
\pgfpathrectangle{\pgfqpoint{0.100000in}{0.100000in}}{\pgfqpoint{1.782500in}{1.232000in}}%
\pgfusepath{clip}%
\pgfsetbuttcap%
\pgfsetroundjoin%
\pgfsetlinewidth{0.501875pt}%
\definecolor{currentstroke}{rgb}{0.054902,0.262745,0.486275}%
\pgfsetstrokecolor{currentstroke}%
\pgfsetdash{}{0pt}%
\pgfpathmoveto{\pgfqpoint{0.852354in}{0.697627in}}%
\pgfpathcurveto{\pgfqpoint{0.857227in}{0.697627in}}{\pgfqpoint{0.861900in}{0.699563in}}{\pgfqpoint{0.865346in}{0.703008in}}%
\pgfpathcurveto{\pgfqpoint{0.868791in}{0.706454in}}{\pgfqpoint{0.870727in}{0.711127in}}{\pgfqpoint{0.870727in}{0.716000in}}%
\pgfpathcurveto{\pgfqpoint{0.870727in}{0.720873in}}{\pgfqpoint{0.868791in}{0.725546in}}{\pgfqpoint{0.865346in}{0.728992in}}%
\pgfpathcurveto{\pgfqpoint{0.861900in}{0.732437in}}{\pgfqpoint{0.857227in}{0.734373in}}{\pgfqpoint{0.852354in}{0.734373in}}%
\pgfpathcurveto{\pgfqpoint{0.847481in}{0.734373in}}{\pgfqpoint{0.842808in}{0.732437in}}{\pgfqpoint{0.839362in}{0.728992in}}%
\pgfpathcurveto{\pgfqpoint{0.835917in}{0.725546in}}{\pgfqpoint{0.833981in}{0.720873in}}{\pgfqpoint{0.833981in}{0.716000in}}%
\pgfpathcurveto{\pgfqpoint{0.833981in}{0.711127in}}{\pgfqpoint{0.835917in}{0.706454in}}{\pgfqpoint{0.839362in}{0.703008in}}%
\pgfpathcurveto{\pgfqpoint{0.842808in}{0.699563in}}{\pgfqpoint{0.847481in}{0.697627in}}{\pgfqpoint{0.852354in}{0.697627in}}%
\pgfpathlineto{\pgfqpoint{0.852354in}{0.697627in}}%
\pgfpathclose%
\pgfusepath{stroke}%
\end{pgfscope}%
\begin{pgfscope}%
\pgfpathrectangle{\pgfqpoint{0.100000in}{0.100000in}}{\pgfqpoint{1.782500in}{1.232000in}}%
\pgfusepath{clip}%
\pgfsetbuttcap%
\pgfsetroundjoin%
\pgfsetlinewidth{0.501875pt}%
\definecolor{currentstroke}{rgb}{0.054902,0.262745,0.486275}%
\pgfsetstrokecolor{currentstroke}%
\pgfsetdash{}{0pt}%
\pgfpathmoveto{\pgfqpoint{0.782906in}{0.697627in}}%
\pgfpathcurveto{\pgfqpoint{0.787778in}{0.697627in}}{\pgfqpoint{0.792452in}{0.699563in}}{\pgfqpoint{0.795898in}{0.703008in}}%
\pgfpathcurveto{\pgfqpoint{0.799343in}{0.706454in}}{\pgfqpoint{0.801279in}{0.711127in}}{\pgfqpoint{0.801279in}{0.716000in}}%
\pgfpathcurveto{\pgfqpoint{0.801279in}{0.720873in}}{\pgfqpoint{0.799343in}{0.725546in}}{\pgfqpoint{0.795898in}{0.728992in}}%
\pgfpathcurveto{\pgfqpoint{0.792452in}{0.732437in}}{\pgfqpoint{0.787778in}{0.734373in}}{\pgfqpoint{0.782906in}{0.734373in}}%
\pgfpathcurveto{\pgfqpoint{0.778033in}{0.734373in}}{\pgfqpoint{0.773359in}{0.732437in}}{\pgfqpoint{0.769914in}{0.728992in}}%
\pgfpathcurveto{\pgfqpoint{0.766468in}{0.725546in}}{\pgfqpoint{0.764533in}{0.720873in}}{\pgfqpoint{0.764533in}{0.716000in}}%
\pgfpathcurveto{\pgfqpoint{0.764533in}{0.711127in}}{\pgfqpoint{0.766468in}{0.706454in}}{\pgfqpoint{0.769914in}{0.703008in}}%
\pgfpathcurveto{\pgfqpoint{0.773359in}{0.699563in}}{\pgfqpoint{0.778033in}{0.697627in}}{\pgfqpoint{0.782906in}{0.697627in}}%
\pgfpathlineto{\pgfqpoint{0.782906in}{0.697627in}}%
\pgfpathclose%
\pgfusepath{stroke}%
\end{pgfscope}%
\begin{pgfscope}%
\pgfpathrectangle{\pgfqpoint{0.100000in}{0.100000in}}{\pgfqpoint{1.782500in}{1.232000in}}%
\pgfusepath{clip}%
\pgfsetbuttcap%
\pgfsetroundjoin%
\pgfsetlinewidth{0.501875pt}%
\definecolor{currentstroke}{rgb}{0.054902,0.262745,0.486275}%
\pgfsetstrokecolor{currentstroke}%
\pgfsetdash{}{0pt}%
\pgfpathmoveto{\pgfqpoint{0.713458in}{0.697627in}}%
\pgfpathcurveto{\pgfqpoint{0.718330in}{0.697627in}}{\pgfqpoint{0.723004in}{0.699563in}}{\pgfqpoint{0.726450in}{0.703008in}}%
\pgfpathcurveto{\pgfqpoint{0.729895in}{0.706454in}}{\pgfqpoint{0.731831in}{0.711127in}}{\pgfqpoint{0.731831in}{0.716000in}}%
\pgfpathcurveto{\pgfqpoint{0.731831in}{0.720873in}}{\pgfqpoint{0.729895in}{0.725546in}}{\pgfqpoint{0.726450in}{0.728992in}}%
\pgfpathcurveto{\pgfqpoint{0.723004in}{0.732437in}}{\pgfqpoint{0.718330in}{0.734373in}}{\pgfqpoint{0.713458in}{0.734373in}}%
\pgfpathcurveto{\pgfqpoint{0.708585in}{0.734373in}}{\pgfqpoint{0.703911in}{0.732437in}}{\pgfqpoint{0.700466in}{0.728992in}}%
\pgfpathcurveto{\pgfqpoint{0.697020in}{0.725546in}}{\pgfqpoint{0.695085in}{0.720873in}}{\pgfqpoint{0.695085in}{0.716000in}}%
\pgfpathcurveto{\pgfqpoint{0.695085in}{0.711127in}}{\pgfqpoint{0.697020in}{0.706454in}}{\pgfqpoint{0.700466in}{0.703008in}}%
\pgfpathcurveto{\pgfqpoint{0.703911in}{0.699563in}}{\pgfqpoint{0.708585in}{0.697627in}}{\pgfqpoint{0.713458in}{0.697627in}}%
\pgfpathlineto{\pgfqpoint{0.713458in}{0.697627in}}%
\pgfpathclose%
\pgfusepath{stroke}%
\end{pgfscope}%
\begin{pgfscope}%
\pgfpathrectangle{\pgfqpoint{0.100000in}{0.100000in}}{\pgfqpoint{1.782500in}{1.232000in}}%
\pgfusepath{clip}%
\pgfsetbuttcap%
\pgfsetroundjoin%
\pgfsetlinewidth{0.501875pt}%
\definecolor{currentstroke}{rgb}{0.054902,0.262745,0.486275}%
\pgfsetstrokecolor{currentstroke}%
\pgfsetdash{}{0pt}%
\pgfpathmoveto{\pgfqpoint{0.644010in}{0.697627in}}%
\pgfpathcurveto{\pgfqpoint{0.648882in}{0.697627in}}{\pgfqpoint{0.653556in}{0.699563in}}{\pgfqpoint{0.657002in}{0.703008in}}%
\pgfpathcurveto{\pgfqpoint{0.660447in}{0.706454in}}{\pgfqpoint{0.662383in}{0.711127in}}{\pgfqpoint{0.662383in}{0.716000in}}%
\pgfpathcurveto{\pgfqpoint{0.662383in}{0.720873in}}{\pgfqpoint{0.660447in}{0.725546in}}{\pgfqpoint{0.657002in}{0.728992in}}%
\pgfpathcurveto{\pgfqpoint{0.653556in}{0.732437in}}{\pgfqpoint{0.648882in}{0.734373in}}{\pgfqpoint{0.644010in}{0.734373in}}%
\pgfpathcurveto{\pgfqpoint{0.639137in}{0.734373in}}{\pgfqpoint{0.634463in}{0.732437in}}{\pgfqpoint{0.631018in}{0.728992in}}%
\pgfpathcurveto{\pgfqpoint{0.627572in}{0.725546in}}{\pgfqpoint{0.625636in}{0.720873in}}{\pgfqpoint{0.625636in}{0.716000in}}%
\pgfpathcurveto{\pgfqpoint{0.625636in}{0.711127in}}{\pgfqpoint{0.627572in}{0.706454in}}{\pgfqpoint{0.631018in}{0.703008in}}%
\pgfpathcurveto{\pgfqpoint{0.634463in}{0.699563in}}{\pgfqpoint{0.639137in}{0.697627in}}{\pgfqpoint{0.644010in}{0.697627in}}%
\pgfpathlineto{\pgfqpoint{0.644010in}{0.697627in}}%
\pgfpathclose%
\pgfusepath{stroke}%
\end{pgfscope}%
\begin{pgfscope}%
\pgfpathrectangle{\pgfqpoint{0.100000in}{0.100000in}}{\pgfqpoint{1.782500in}{1.232000in}}%
\pgfusepath{clip}%
\pgfsetbuttcap%
\pgfsetroundjoin%
\pgfsetlinewidth{0.501875pt}%
\definecolor{currentstroke}{rgb}{0.054902,0.262745,0.486275}%
\pgfsetstrokecolor{currentstroke}%
\pgfsetdash{}{0pt}%
\pgfpathmoveto{\pgfqpoint{0.574562in}{0.697627in}}%
\pgfpathcurveto{\pgfqpoint{0.579434in}{0.697627in}}{\pgfqpoint{0.584108in}{0.699563in}}{\pgfqpoint{0.587554in}{0.703008in}}%
\pgfpathcurveto{\pgfqpoint{0.590999in}{0.706454in}}{\pgfqpoint{0.592935in}{0.711127in}}{\pgfqpoint{0.592935in}{0.716000in}}%
\pgfpathcurveto{\pgfqpoint{0.592935in}{0.720873in}}{\pgfqpoint{0.590999in}{0.725546in}}{\pgfqpoint{0.587554in}{0.728992in}}%
\pgfpathcurveto{\pgfqpoint{0.584108in}{0.732437in}}{\pgfqpoint{0.579434in}{0.734373in}}{\pgfqpoint{0.574562in}{0.734373in}}%
\pgfpathcurveto{\pgfqpoint{0.569689in}{0.734373in}}{\pgfqpoint{0.565015in}{0.732437in}}{\pgfqpoint{0.561570in}{0.728992in}}%
\pgfpathcurveto{\pgfqpoint{0.558124in}{0.725546in}}{\pgfqpoint{0.556188in}{0.720873in}}{\pgfqpoint{0.556188in}{0.716000in}}%
\pgfpathcurveto{\pgfqpoint{0.556188in}{0.711127in}}{\pgfqpoint{0.558124in}{0.706454in}}{\pgfqpoint{0.561570in}{0.703008in}}%
\pgfpathcurveto{\pgfqpoint{0.565015in}{0.699563in}}{\pgfqpoint{0.569689in}{0.697627in}}{\pgfqpoint{0.574562in}{0.697627in}}%
\pgfpathlineto{\pgfqpoint{0.574562in}{0.697627in}}%
\pgfpathclose%
\pgfusepath{stroke}%
\end{pgfscope}%
\begin{pgfscope}%
\pgfpathrectangle{\pgfqpoint{0.100000in}{0.100000in}}{\pgfqpoint{1.782500in}{1.232000in}}%
\pgfusepath{clip}%
\pgfsetbuttcap%
\pgfsetroundjoin%
\pgfsetlinewidth{0.501875pt}%
\definecolor{currentstroke}{rgb}{0.835294,0.321569,0.035294}%
\pgfsetstrokecolor{currentstroke}%
\pgfsetdash{}{0pt}%
\pgfpathmoveto{\pgfqpoint{0.505114in}{0.983519in}}%
\pgfpathcurveto{\pgfqpoint{0.509986in}{0.983519in}}{\pgfqpoint{0.514660in}{0.985454in}}{\pgfqpoint{0.518106in}{0.988900in}}%
\pgfpathcurveto{\pgfqpoint{0.521551in}{0.992345in}}{\pgfqpoint{0.523487in}{0.997019in}}{\pgfqpoint{0.523487in}{1.001892in}}%
\pgfpathcurveto{\pgfqpoint{0.523487in}{1.006764in}}{\pgfqpoint{0.521551in}{1.011438in}}{\pgfqpoint{0.518106in}{1.014884in}}%
\pgfpathcurveto{\pgfqpoint{0.514660in}{1.018329in}}{\pgfqpoint{0.509986in}{1.020265in}}{\pgfqpoint{0.505114in}{1.020265in}}%
\pgfpathcurveto{\pgfqpoint{0.500241in}{1.020265in}}{\pgfqpoint{0.495567in}{1.018329in}}{\pgfqpoint{0.492122in}{1.014884in}}%
\pgfpathcurveto{\pgfqpoint{0.488676in}{1.011438in}}{\pgfqpoint{0.486740in}{1.006764in}}{\pgfqpoint{0.486740in}{1.001892in}}%
\pgfpathcurveto{\pgfqpoint{0.486740in}{0.997019in}}{\pgfqpoint{0.488676in}{0.992345in}}{\pgfqpoint{0.492122in}{0.988900in}}%
\pgfpathcurveto{\pgfqpoint{0.495567in}{0.985454in}}{\pgfqpoint{0.500241in}{0.983519in}}{\pgfqpoint{0.505114in}{0.983519in}}%
\pgfpathlineto{\pgfqpoint{0.505114in}{0.983519in}}%
\pgfpathclose%
\pgfusepath{stroke}%
\end{pgfscope}%
\begin{pgfscope}%
\pgfpathrectangle{\pgfqpoint{0.100000in}{0.100000in}}{\pgfqpoint{1.782500in}{1.232000in}}%
\pgfusepath{clip}%
\pgfsetbuttcap%
\pgfsetroundjoin%
\pgfsetlinewidth{0.501875pt}%
\definecolor{currentstroke}{rgb}{0.835294,0.321569,0.035294}%
\pgfsetstrokecolor{currentstroke}%
\pgfsetdash{}{0pt}%
\pgfpathmoveto{\pgfqpoint{1.477386in}{0.983519in}}%
\pgfpathcurveto{\pgfqpoint{1.482259in}{0.983519in}}{\pgfqpoint{1.486933in}{0.985454in}}{\pgfqpoint{1.490378in}{0.988900in}}%
\pgfpathcurveto{\pgfqpoint{1.493824in}{0.992345in}}{\pgfqpoint{1.495760in}{0.997019in}}{\pgfqpoint{1.495760in}{1.001892in}}%
\pgfpathcurveto{\pgfqpoint{1.495760in}{1.006764in}}{\pgfqpoint{1.493824in}{1.011438in}}{\pgfqpoint{1.490378in}{1.014884in}}%
\pgfpathcurveto{\pgfqpoint{1.486933in}{1.018329in}}{\pgfqpoint{1.482259in}{1.020265in}}{\pgfqpoint{1.477386in}{1.020265in}}%
\pgfpathcurveto{\pgfqpoint{1.472514in}{1.020265in}}{\pgfqpoint{1.467840in}{1.018329in}}{\pgfqpoint{1.464394in}{1.014884in}}%
\pgfpathcurveto{\pgfqpoint{1.460949in}{1.011438in}}{\pgfqpoint{1.459013in}{1.006764in}}{\pgfqpoint{1.459013in}{1.001892in}}%
\pgfpathcurveto{\pgfqpoint{1.459013in}{0.997019in}}{\pgfqpoint{1.460949in}{0.992345in}}{\pgfqpoint{1.464394in}{0.988900in}}%
\pgfpathcurveto{\pgfqpoint{1.467840in}{0.985454in}}{\pgfqpoint{1.472514in}{0.983519in}}{\pgfqpoint{1.477386in}{0.983519in}}%
\pgfpathlineto{\pgfqpoint{1.477386in}{0.983519in}}%
\pgfpathclose%
\pgfusepath{stroke}%
\end{pgfscope}%
\begin{pgfscope}%
\pgfpathrectangle{\pgfqpoint{0.100000in}{0.100000in}}{\pgfqpoint{1.782500in}{1.232000in}}%
\pgfusepath{clip}%
\pgfsetbuttcap%
\pgfsetroundjoin%
\pgfsetlinewidth{0.501875pt}%
\definecolor{currentstroke}{rgb}{0.835294,0.321569,0.035294}%
\pgfsetstrokecolor{currentstroke}%
\pgfsetdash{}{0pt}%
\pgfpathmoveto{\pgfqpoint{0.548824in}{0.898639in}}%
\pgfpathcurveto{\pgfqpoint{0.553696in}{0.898639in}}{\pgfqpoint{0.558370in}{0.900575in}}{\pgfqpoint{0.561816in}{0.904020in}}%
\pgfpathcurveto{\pgfqpoint{0.565261in}{0.907466in}}{\pgfqpoint{0.567197in}{0.912139in}}{\pgfqpoint{0.567197in}{0.917012in}}%
\pgfpathcurveto{\pgfqpoint{0.567197in}{0.921885in}}{\pgfqpoint{0.565261in}{0.926558in}}{\pgfqpoint{0.561816in}{0.930004in}}%
\pgfpathcurveto{\pgfqpoint{0.558370in}{0.933449in}}{\pgfqpoint{0.553696in}{0.935385in}}{\pgfqpoint{0.548824in}{0.935385in}}%
\pgfpathcurveto{\pgfqpoint{0.543951in}{0.935385in}}{\pgfqpoint{0.539277in}{0.933449in}}{\pgfqpoint{0.535832in}{0.930004in}}%
\pgfpathcurveto{\pgfqpoint{0.532386in}{0.926558in}}{\pgfqpoint{0.530450in}{0.921885in}}{\pgfqpoint{0.530450in}{0.917012in}}%
\pgfpathcurveto{\pgfqpoint{0.530450in}{0.912139in}}{\pgfqpoint{0.532386in}{0.907466in}}{\pgfqpoint{0.535832in}{0.904020in}}%
\pgfpathcurveto{\pgfqpoint{0.539277in}{0.900575in}}{\pgfqpoint{0.543951in}{0.898639in}}{\pgfqpoint{0.548824in}{0.898639in}}%
\pgfpathlineto{\pgfqpoint{0.548824in}{0.898639in}}%
\pgfpathclose%
\pgfusepath{stroke}%
\end{pgfscope}%
\begin{pgfscope}%
\pgfpathrectangle{\pgfqpoint{0.100000in}{0.100000in}}{\pgfqpoint{1.782500in}{1.232000in}}%
\pgfusepath{clip}%
\pgfsetbuttcap%
\pgfsetroundjoin%
\pgfsetlinewidth{0.501875pt}%
\definecolor{currentstroke}{rgb}{0.835294,0.321569,0.035294}%
\pgfsetstrokecolor{currentstroke}%
\pgfsetdash{}{0pt}%
\pgfpathmoveto{\pgfqpoint{0.603831in}{0.822783in}}%
\pgfpathcurveto{\pgfqpoint{0.608704in}{0.822783in}}{\pgfqpoint{0.613377in}{0.824719in}}{\pgfqpoint{0.616823in}{0.828164in}}%
\pgfpathcurveto{\pgfqpoint{0.620268in}{0.831610in}}{\pgfqpoint{0.622204in}{0.836283in}}{\pgfqpoint{0.622204in}{0.841156in}}%
\pgfpathcurveto{\pgfqpoint{0.622204in}{0.846029in}}{\pgfqpoint{0.620268in}{0.850702in}}{\pgfqpoint{0.616823in}{0.854148in}}%
\pgfpathcurveto{\pgfqpoint{0.613377in}{0.857593in}}{\pgfqpoint{0.608704in}{0.859529in}}{\pgfqpoint{0.603831in}{0.859529in}}%
\pgfpathcurveto{\pgfqpoint{0.598958in}{0.859529in}}{\pgfqpoint{0.594285in}{0.857593in}}{\pgfqpoint{0.590839in}{0.854148in}}%
\pgfpathcurveto{\pgfqpoint{0.587394in}{0.850702in}}{\pgfqpoint{0.585458in}{0.846029in}}{\pgfqpoint{0.585458in}{0.841156in}}%
\pgfpathcurveto{\pgfqpoint{0.585458in}{0.836283in}}{\pgfqpoint{0.587394in}{0.831610in}}{\pgfqpoint{0.590839in}{0.828164in}}%
\pgfpathcurveto{\pgfqpoint{0.594285in}{0.824719in}}{\pgfqpoint{0.598958in}{0.822783in}}{\pgfqpoint{0.603831in}{0.822783in}}%
\pgfpathlineto{\pgfqpoint{0.603831in}{0.822783in}}%
\pgfpathclose%
\pgfusepath{stroke}%
\end{pgfscope}%
\begin{pgfscope}%
\pgfpathrectangle{\pgfqpoint{0.100000in}{0.100000in}}{\pgfqpoint{1.782500in}{1.232000in}}%
\pgfusepath{clip}%
\pgfsetbuttcap%
\pgfsetroundjoin%
\pgfsetlinewidth{0.501875pt}%
\definecolor{currentstroke}{rgb}{0.835294,0.321569,0.035294}%
\pgfsetstrokecolor{currentstroke}%
\pgfsetdash{}{0pt}%
\pgfpathmoveto{\pgfqpoint{0.668731in}{0.757887in}}%
\pgfpathcurveto{\pgfqpoint{0.673604in}{0.757887in}}{\pgfqpoint{0.678277in}{0.759823in}}{\pgfqpoint{0.681723in}{0.763269in}}%
\pgfpathcurveto{\pgfqpoint{0.685168in}{0.766714in}}{\pgfqpoint{0.687104in}{0.771388in}}{\pgfqpoint{0.687104in}{0.776261in}}%
\pgfpathcurveto{\pgfqpoint{0.687104in}{0.781133in}}{\pgfqpoint{0.685168in}{0.785807in}}{\pgfqpoint{0.681723in}{0.789252in}}%
\pgfpathcurveto{\pgfqpoint{0.678277in}{0.792698in}}{\pgfqpoint{0.673604in}{0.794634in}}{\pgfqpoint{0.668731in}{0.794634in}}%
\pgfpathcurveto{\pgfqpoint{0.663858in}{0.794634in}}{\pgfqpoint{0.659185in}{0.792698in}}{\pgfqpoint{0.655739in}{0.789252in}}%
\pgfpathcurveto{\pgfqpoint{0.652294in}{0.785807in}}{\pgfqpoint{0.650358in}{0.781133in}}{\pgfqpoint{0.650358in}{0.776261in}}%
\pgfpathcurveto{\pgfqpoint{0.650358in}{0.771388in}}{\pgfqpoint{0.652294in}{0.766714in}}{\pgfqpoint{0.655739in}{0.763269in}}%
\pgfpathcurveto{\pgfqpoint{0.659185in}{0.759823in}}{\pgfqpoint{0.663858in}{0.757887in}}{\pgfqpoint{0.668731in}{0.757887in}}%
\pgfpathlineto{\pgfqpoint{0.668731in}{0.757887in}}%
\pgfpathclose%
\pgfusepath{stroke}%
\end{pgfscope}%
\begin{pgfscope}%
\pgfpathrectangle{\pgfqpoint{0.100000in}{0.100000in}}{\pgfqpoint{1.782500in}{1.232000in}}%
\pgfusepath{clip}%
\pgfsetbuttcap%
\pgfsetroundjoin%
\pgfsetlinewidth{0.501875pt}%
\definecolor{currentstroke}{rgb}{0.835294,0.321569,0.035294}%
\pgfsetstrokecolor{currentstroke}%
\pgfsetdash{}{0pt}%
\pgfpathmoveto{\pgfqpoint{0.741867in}{0.705610in}}%
\pgfpathcurveto{\pgfqpoint{0.746739in}{0.705610in}}{\pgfqpoint{0.751413in}{0.707546in}}{\pgfqpoint{0.754858in}{0.710991in}}%
\pgfpathcurveto{\pgfqpoint{0.758304in}{0.714437in}}{\pgfqpoint{0.760240in}{0.719110in}}{\pgfqpoint{0.760240in}{0.723983in}}%
\pgfpathcurveto{\pgfqpoint{0.760240in}{0.728856in}}{\pgfqpoint{0.758304in}{0.733529in}}{\pgfqpoint{0.754858in}{0.736975in}}%
\pgfpathcurveto{\pgfqpoint{0.751413in}{0.740420in}}{\pgfqpoint{0.746739in}{0.742356in}}{\pgfqpoint{0.741867in}{0.742356in}}%
\pgfpathcurveto{\pgfqpoint{0.736994in}{0.742356in}}{\pgfqpoint{0.732320in}{0.740420in}}{\pgfqpoint{0.728875in}{0.736975in}}%
\pgfpathcurveto{\pgfqpoint{0.725429in}{0.733529in}}{\pgfqpoint{0.723493in}{0.728856in}}{\pgfqpoint{0.723493in}{0.723983in}}%
\pgfpathcurveto{\pgfqpoint{0.723493in}{0.719110in}}{\pgfqpoint{0.725429in}{0.714437in}}{\pgfqpoint{0.728875in}{0.710991in}}%
\pgfpathcurveto{\pgfqpoint{0.732320in}{0.707546in}}{\pgfqpoint{0.736994in}{0.705610in}}{\pgfqpoint{0.741867in}{0.705610in}}%
\pgfpathlineto{\pgfqpoint{0.741867in}{0.705610in}}%
\pgfpathclose%
\pgfusepath{stroke}%
\end{pgfscope}%
\begin{pgfscope}%
\pgfpathrectangle{\pgfqpoint{0.100000in}{0.100000in}}{\pgfqpoint{1.782500in}{1.232000in}}%
\pgfusepath{clip}%
\pgfsetbuttcap%
\pgfsetroundjoin%
\pgfsetlinewidth{0.501875pt}%
\definecolor{currentstroke}{rgb}{0.835294,0.321569,0.035294}%
\pgfsetstrokecolor{currentstroke}%
\pgfsetdash{}{0pt}%
\pgfpathmoveto{\pgfqpoint{0.821370in}{0.667285in}}%
\pgfpathcurveto{\pgfqpoint{0.826243in}{0.667285in}}{\pgfqpoint{0.830917in}{0.669221in}}{\pgfqpoint{0.834362in}{0.672666in}}%
\pgfpathcurveto{\pgfqpoint{0.837808in}{0.676112in}}{\pgfqpoint{0.839743in}{0.680786in}}{\pgfqpoint{0.839743in}{0.685658in}}%
\pgfpathcurveto{\pgfqpoint{0.839743in}{0.690531in}}{\pgfqpoint{0.837808in}{0.695205in}}{\pgfqpoint{0.834362in}{0.698650in}}%
\pgfpathcurveto{\pgfqpoint{0.830917in}{0.702096in}}{\pgfqpoint{0.826243in}{0.704031in}}{\pgfqpoint{0.821370in}{0.704031in}}%
\pgfpathcurveto{\pgfqpoint{0.816498in}{0.704031in}}{\pgfqpoint{0.811824in}{0.702096in}}{\pgfqpoint{0.808378in}{0.698650in}}%
\pgfpathcurveto{\pgfqpoint{0.804933in}{0.695205in}}{\pgfqpoint{0.802997in}{0.690531in}}{\pgfqpoint{0.802997in}{0.685658in}}%
\pgfpathcurveto{\pgfqpoint{0.802997in}{0.680786in}}{\pgfqpoint{0.804933in}{0.676112in}}{\pgfqpoint{0.808378in}{0.672666in}}%
\pgfpathcurveto{\pgfqpoint{0.811824in}{0.669221in}}{\pgfqpoint{0.816498in}{0.667285in}}{\pgfqpoint{0.821370in}{0.667285in}}%
\pgfpathlineto{\pgfqpoint{0.821370in}{0.667285in}}%
\pgfpathclose%
\pgfusepath{stroke}%
\end{pgfscope}%
\begin{pgfscope}%
\pgfpathrectangle{\pgfqpoint{0.100000in}{0.100000in}}{\pgfqpoint{1.782500in}{1.232000in}}%
\pgfusepath{clip}%
\pgfsetbuttcap%
\pgfsetroundjoin%
\pgfsetlinewidth{0.501875pt}%
\definecolor{currentstroke}{rgb}{0.835294,0.321569,0.035294}%
\pgfsetstrokecolor{currentstroke}%
\pgfsetdash{}{0pt}%
\pgfpathmoveto{\pgfqpoint{0.905212in}{0.643891in}}%
\pgfpathcurveto{\pgfqpoint{0.910084in}{0.643891in}}{\pgfqpoint{0.914758in}{0.645827in}}{\pgfqpoint{0.918203in}{0.649273in}}%
\pgfpathcurveto{\pgfqpoint{0.921649in}{0.652718in}}{\pgfqpoint{0.923585in}{0.657392in}}{\pgfqpoint{0.923585in}{0.662265in}}%
\pgfpathcurveto{\pgfqpoint{0.923585in}{0.667137in}}{\pgfqpoint{0.921649in}{0.671811in}}{\pgfqpoint{0.918203in}{0.675257in}}%
\pgfpathcurveto{\pgfqpoint{0.914758in}{0.678702in}}{\pgfqpoint{0.910084in}{0.680638in}}{\pgfqpoint{0.905212in}{0.680638in}}%
\pgfpathcurveto{\pgfqpoint{0.900339in}{0.680638in}}{\pgfqpoint{0.895665in}{0.678702in}}{\pgfqpoint{0.892220in}{0.675257in}}%
\pgfpathcurveto{\pgfqpoint{0.888774in}{0.671811in}}{\pgfqpoint{0.886838in}{0.667137in}}{\pgfqpoint{0.886838in}{0.662265in}}%
\pgfpathcurveto{\pgfqpoint{0.886838in}{0.657392in}}{\pgfqpoint{0.888774in}{0.652718in}}{\pgfqpoint{0.892220in}{0.649273in}}%
\pgfpathcurveto{\pgfqpoint{0.895665in}{0.645827in}}{\pgfqpoint{0.900339in}{0.643891in}}{\pgfqpoint{0.905212in}{0.643891in}}%
\pgfpathlineto{\pgfqpoint{0.905212in}{0.643891in}}%
\pgfpathclose%
\pgfusepath{stroke}%
\end{pgfscope}%
\begin{pgfscope}%
\pgfpathrectangle{\pgfqpoint{0.100000in}{0.100000in}}{\pgfqpoint{1.782500in}{1.232000in}}%
\pgfusepath{clip}%
\pgfsetbuttcap%
\pgfsetroundjoin%
\pgfsetlinewidth{0.501875pt}%
\definecolor{currentstroke}{rgb}{0.835294,0.321569,0.035294}%
\pgfsetstrokecolor{currentstroke}%
\pgfsetdash{}{0pt}%
\pgfpathmoveto{\pgfqpoint{0.991250in}{0.636027in}}%
\pgfpathcurveto{\pgfqpoint{0.996123in}{0.636027in}}{\pgfqpoint{1.000796in}{0.637963in}}{\pgfqpoint{1.004242in}{0.641408in}}%
\pgfpathcurveto{\pgfqpoint{1.007687in}{0.644854in}}{\pgfqpoint{1.009623in}{0.649527in}}{\pgfqpoint{1.009623in}{0.654400in}}%
\pgfpathcurveto{\pgfqpoint{1.009623in}{0.659273in}}{\pgfqpoint{1.007687in}{0.663946in}}{\pgfqpoint{1.004242in}{0.667392in}}%
\pgfpathcurveto{\pgfqpoint{1.000796in}{0.670837in}}{\pgfqpoint{0.996123in}{0.672773in}}{\pgfqpoint{0.991250in}{0.672773in}}%
\pgfpathcurveto{\pgfqpoint{0.986377in}{0.672773in}}{\pgfqpoint{0.981704in}{0.670837in}}{\pgfqpoint{0.978258in}{0.667392in}}%
\pgfpathcurveto{\pgfqpoint{0.974813in}{0.663946in}}{\pgfqpoint{0.972877in}{0.659273in}}{\pgfqpoint{0.972877in}{0.654400in}}%
\pgfpathcurveto{\pgfqpoint{0.972877in}{0.649527in}}{\pgfqpoint{0.974813in}{0.644854in}}{\pgfqpoint{0.978258in}{0.641408in}}%
\pgfpathcurveto{\pgfqpoint{0.981704in}{0.637963in}}{\pgfqpoint{0.986377in}{0.636027in}}{\pgfqpoint{0.991250in}{0.636027in}}%
\pgfpathlineto{\pgfqpoint{0.991250in}{0.636027in}}%
\pgfpathclose%
\pgfusepath{stroke}%
\end{pgfscope}%
\begin{pgfscope}%
\pgfpathrectangle{\pgfqpoint{0.100000in}{0.100000in}}{\pgfqpoint{1.782500in}{1.232000in}}%
\pgfusepath{clip}%
\pgfsetbuttcap%
\pgfsetroundjoin%
\pgfsetlinewidth{0.501875pt}%
\definecolor{currentstroke}{rgb}{0.835294,0.321569,0.035294}%
\pgfsetstrokecolor{currentstroke}%
\pgfsetdash{}{0pt}%
\pgfpathmoveto{\pgfqpoint{1.077288in}{0.643891in}}%
\pgfpathcurveto{\pgfqpoint{1.082161in}{0.643891in}}{\pgfqpoint{1.086835in}{0.645827in}}{\pgfqpoint{1.090280in}{0.649273in}}%
\pgfpathcurveto{\pgfqpoint{1.093726in}{0.652718in}}{\pgfqpoint{1.095662in}{0.657392in}}{\pgfqpoint{1.095662in}{0.662265in}}%
\pgfpathcurveto{\pgfqpoint{1.095662in}{0.667137in}}{\pgfqpoint{1.093726in}{0.671811in}}{\pgfqpoint{1.090280in}{0.675257in}}%
\pgfpathcurveto{\pgfqpoint{1.086835in}{0.678702in}}{\pgfqpoint{1.082161in}{0.680638in}}{\pgfqpoint{1.077288in}{0.680638in}}%
\pgfpathcurveto{\pgfqpoint{1.072416in}{0.680638in}}{\pgfqpoint{1.067742in}{0.678702in}}{\pgfqpoint{1.064297in}{0.675257in}}%
\pgfpathcurveto{\pgfqpoint{1.060851in}{0.671811in}}{\pgfqpoint{1.058915in}{0.667137in}}{\pgfqpoint{1.058915in}{0.662265in}}%
\pgfpathcurveto{\pgfqpoint{1.058915in}{0.657392in}}{\pgfqpoint{1.060851in}{0.652718in}}{\pgfqpoint{1.064297in}{0.649273in}}%
\pgfpathcurveto{\pgfqpoint{1.067742in}{0.645827in}}{\pgfqpoint{1.072416in}{0.643891in}}{\pgfqpoint{1.077288in}{0.643891in}}%
\pgfpathlineto{\pgfqpoint{1.077288in}{0.643891in}}%
\pgfpathclose%
\pgfusepath{stroke}%
\end{pgfscope}%
\begin{pgfscope}%
\pgfpathrectangle{\pgfqpoint{0.100000in}{0.100000in}}{\pgfqpoint{1.782500in}{1.232000in}}%
\pgfusepath{clip}%
\pgfsetbuttcap%
\pgfsetroundjoin%
\pgfsetlinewidth{0.501875pt}%
\definecolor{currentstroke}{rgb}{0.835294,0.321569,0.035294}%
\pgfsetstrokecolor{currentstroke}%
\pgfsetdash{}{0pt}%
\pgfpathmoveto{\pgfqpoint{1.161130in}{0.667285in}}%
\pgfpathcurveto{\pgfqpoint{1.166002in}{0.667285in}}{\pgfqpoint{1.170676in}{0.669221in}}{\pgfqpoint{1.174122in}{0.672666in}}%
\pgfpathcurveto{\pgfqpoint{1.177567in}{0.676112in}}{\pgfqpoint{1.179503in}{0.680786in}}{\pgfqpoint{1.179503in}{0.685658in}}%
\pgfpathcurveto{\pgfqpoint{1.179503in}{0.690531in}}{\pgfqpoint{1.177567in}{0.695205in}}{\pgfqpoint{1.174122in}{0.698650in}}%
\pgfpathcurveto{\pgfqpoint{1.170676in}{0.702096in}}{\pgfqpoint{1.166002in}{0.704031in}}{\pgfqpoint{1.161130in}{0.704031in}}%
\pgfpathcurveto{\pgfqpoint{1.156257in}{0.704031in}}{\pgfqpoint{1.151583in}{0.702096in}}{\pgfqpoint{1.148138in}{0.698650in}}%
\pgfpathcurveto{\pgfqpoint{1.144692in}{0.695205in}}{\pgfqpoint{1.142757in}{0.690531in}}{\pgfqpoint{1.142757in}{0.685658in}}%
\pgfpathcurveto{\pgfqpoint{1.142757in}{0.680786in}}{\pgfqpoint{1.144692in}{0.676112in}}{\pgfqpoint{1.148138in}{0.672666in}}%
\pgfpathcurveto{\pgfqpoint{1.151583in}{0.669221in}}{\pgfqpoint{1.156257in}{0.667285in}}{\pgfqpoint{1.161130in}{0.667285in}}%
\pgfpathlineto{\pgfqpoint{1.161130in}{0.667285in}}%
\pgfpathclose%
\pgfusepath{stroke}%
\end{pgfscope}%
\begin{pgfscope}%
\pgfpathrectangle{\pgfqpoint{0.100000in}{0.100000in}}{\pgfqpoint{1.782500in}{1.232000in}}%
\pgfusepath{clip}%
\pgfsetbuttcap%
\pgfsetroundjoin%
\pgfsetlinewidth{0.501875pt}%
\definecolor{currentstroke}{rgb}{0.835294,0.321569,0.035294}%
\pgfsetstrokecolor{currentstroke}%
\pgfsetdash{}{0pt}%
\pgfpathmoveto{\pgfqpoint{1.240633in}{0.705610in}}%
\pgfpathcurveto{\pgfqpoint{1.245506in}{0.705610in}}{\pgfqpoint{1.250180in}{0.707546in}}{\pgfqpoint{1.253625in}{0.710991in}}%
\pgfpathcurveto{\pgfqpoint{1.257071in}{0.714437in}}{\pgfqpoint{1.259007in}{0.719110in}}{\pgfqpoint{1.259007in}{0.723983in}}%
\pgfpathcurveto{\pgfqpoint{1.259007in}{0.728856in}}{\pgfqpoint{1.257071in}{0.733529in}}{\pgfqpoint{1.253625in}{0.736975in}}%
\pgfpathcurveto{\pgfqpoint{1.250180in}{0.740420in}}{\pgfqpoint{1.245506in}{0.742356in}}{\pgfqpoint{1.240633in}{0.742356in}}%
\pgfpathcurveto{\pgfqpoint{1.235761in}{0.742356in}}{\pgfqpoint{1.231087in}{0.740420in}}{\pgfqpoint{1.227642in}{0.736975in}}%
\pgfpathcurveto{\pgfqpoint{1.224196in}{0.733529in}}{\pgfqpoint{1.222260in}{0.728856in}}{\pgfqpoint{1.222260in}{0.723983in}}%
\pgfpathcurveto{\pgfqpoint{1.222260in}{0.719110in}}{\pgfqpoint{1.224196in}{0.714437in}}{\pgfqpoint{1.227642in}{0.710991in}}%
\pgfpathcurveto{\pgfqpoint{1.231087in}{0.707546in}}{\pgfqpoint{1.235761in}{0.705610in}}{\pgfqpoint{1.240633in}{0.705610in}}%
\pgfpathlineto{\pgfqpoint{1.240633in}{0.705610in}}%
\pgfpathclose%
\pgfusepath{stroke}%
\end{pgfscope}%
\begin{pgfscope}%
\pgfpathrectangle{\pgfqpoint{0.100000in}{0.100000in}}{\pgfqpoint{1.782500in}{1.232000in}}%
\pgfusepath{clip}%
\pgfsetbuttcap%
\pgfsetroundjoin%
\pgfsetlinewidth{0.501875pt}%
\definecolor{currentstroke}{rgb}{0.835294,0.321569,0.035294}%
\pgfsetstrokecolor{currentstroke}%
\pgfsetdash{}{0pt}%
\pgfpathmoveto{\pgfqpoint{1.313769in}{0.757887in}}%
\pgfpathcurveto{\pgfqpoint{1.318642in}{0.757887in}}{\pgfqpoint{1.323315in}{0.759823in}}{\pgfqpoint{1.326761in}{0.763269in}}%
\pgfpathcurveto{\pgfqpoint{1.330206in}{0.766714in}}{\pgfqpoint{1.332142in}{0.771388in}}{\pgfqpoint{1.332142in}{0.776261in}}%
\pgfpathcurveto{\pgfqpoint{1.332142in}{0.781133in}}{\pgfqpoint{1.330206in}{0.785807in}}{\pgfqpoint{1.326761in}{0.789252in}}%
\pgfpathcurveto{\pgfqpoint{1.323315in}{0.792698in}}{\pgfqpoint{1.318642in}{0.794634in}}{\pgfqpoint{1.313769in}{0.794634in}}%
\pgfpathcurveto{\pgfqpoint{1.308896in}{0.794634in}}{\pgfqpoint{1.304223in}{0.792698in}}{\pgfqpoint{1.300777in}{0.789252in}}%
\pgfpathcurveto{\pgfqpoint{1.297332in}{0.785807in}}{\pgfqpoint{1.295396in}{0.781133in}}{\pgfqpoint{1.295396in}{0.776261in}}%
\pgfpathcurveto{\pgfqpoint{1.295396in}{0.771388in}}{\pgfqpoint{1.297332in}{0.766714in}}{\pgfqpoint{1.300777in}{0.763269in}}%
\pgfpathcurveto{\pgfqpoint{1.304223in}{0.759823in}}{\pgfqpoint{1.308896in}{0.757887in}}{\pgfqpoint{1.313769in}{0.757887in}}%
\pgfpathlineto{\pgfqpoint{1.313769in}{0.757887in}}%
\pgfpathclose%
\pgfusepath{stroke}%
\end{pgfscope}%
\begin{pgfscope}%
\pgfpathrectangle{\pgfqpoint{0.100000in}{0.100000in}}{\pgfqpoint{1.782500in}{1.232000in}}%
\pgfusepath{clip}%
\pgfsetbuttcap%
\pgfsetroundjoin%
\pgfsetlinewidth{0.501875pt}%
\definecolor{currentstroke}{rgb}{0.835294,0.321569,0.035294}%
\pgfsetstrokecolor{currentstroke}%
\pgfsetdash{}{0pt}%
\pgfpathmoveto{\pgfqpoint{1.378669in}{0.822783in}}%
\pgfpathcurveto{\pgfqpoint{1.383542in}{0.822783in}}{\pgfqpoint{1.388215in}{0.824719in}}{\pgfqpoint{1.391661in}{0.828164in}}%
\pgfpathcurveto{\pgfqpoint{1.395106in}{0.831610in}}{\pgfqpoint{1.397042in}{0.836283in}}{\pgfqpoint{1.397042in}{0.841156in}}%
\pgfpathcurveto{\pgfqpoint{1.397042in}{0.846029in}}{\pgfqpoint{1.395106in}{0.850702in}}{\pgfqpoint{1.391661in}{0.854148in}}%
\pgfpathcurveto{\pgfqpoint{1.388215in}{0.857593in}}{\pgfqpoint{1.383542in}{0.859529in}}{\pgfqpoint{1.378669in}{0.859529in}}%
\pgfpathcurveto{\pgfqpoint{1.373796in}{0.859529in}}{\pgfqpoint{1.369123in}{0.857593in}}{\pgfqpoint{1.365677in}{0.854148in}}%
\pgfpathcurveto{\pgfqpoint{1.362232in}{0.850702in}}{\pgfqpoint{1.360296in}{0.846029in}}{\pgfqpoint{1.360296in}{0.841156in}}%
\pgfpathcurveto{\pgfqpoint{1.360296in}{0.836283in}}{\pgfqpoint{1.362232in}{0.831610in}}{\pgfqpoint{1.365677in}{0.828164in}}%
\pgfpathcurveto{\pgfqpoint{1.369123in}{0.824719in}}{\pgfqpoint{1.373796in}{0.822783in}}{\pgfqpoint{1.378669in}{0.822783in}}%
\pgfpathlineto{\pgfqpoint{1.378669in}{0.822783in}}%
\pgfpathclose%
\pgfusepath{stroke}%
\end{pgfscope}%
\begin{pgfscope}%
\pgfpathrectangle{\pgfqpoint{0.100000in}{0.100000in}}{\pgfqpoint{1.782500in}{1.232000in}}%
\pgfusepath{clip}%
\pgfsetbuttcap%
\pgfsetroundjoin%
\pgfsetlinewidth{0.501875pt}%
\definecolor{currentstroke}{rgb}{0.835294,0.321569,0.035294}%
\pgfsetstrokecolor{currentstroke}%
\pgfsetdash{}{0pt}%
\pgfpathmoveto{\pgfqpoint{1.433676in}{0.898639in}}%
\pgfpathcurveto{\pgfqpoint{1.438549in}{0.898639in}}{\pgfqpoint{1.443223in}{0.900575in}}{\pgfqpoint{1.446668in}{0.904020in}}%
\pgfpathcurveto{\pgfqpoint{1.450114in}{0.907466in}}{\pgfqpoint{1.452050in}{0.912139in}}{\pgfqpoint{1.452050in}{0.917012in}}%
\pgfpathcurveto{\pgfqpoint{1.452050in}{0.921885in}}{\pgfqpoint{1.450114in}{0.926558in}}{\pgfqpoint{1.446668in}{0.930004in}}%
\pgfpathcurveto{\pgfqpoint{1.443223in}{0.933449in}}{\pgfqpoint{1.438549in}{0.935385in}}{\pgfqpoint{1.433676in}{0.935385in}}%
\pgfpathcurveto{\pgfqpoint{1.428804in}{0.935385in}}{\pgfqpoint{1.424130in}{0.933449in}}{\pgfqpoint{1.420684in}{0.930004in}}%
\pgfpathcurveto{\pgfqpoint{1.417239in}{0.926558in}}{\pgfqpoint{1.415303in}{0.921885in}}{\pgfqpoint{1.415303in}{0.917012in}}%
\pgfpathcurveto{\pgfqpoint{1.415303in}{0.912139in}}{\pgfqpoint{1.417239in}{0.907466in}}{\pgfqpoint{1.420684in}{0.904020in}}%
\pgfpathcurveto{\pgfqpoint{1.424130in}{0.900575in}}{\pgfqpoint{1.428804in}{0.898639in}}{\pgfqpoint{1.433676in}{0.898639in}}%
\pgfpathlineto{\pgfqpoint{1.433676in}{0.898639in}}%
\pgfpathclose%
\pgfusepath{stroke}%
\end{pgfscope}%
\end{pgfpicture}%
\makeatother%
\endgroup%

        \caption{Initial configuration}\label{fig:example-initial}
    \end{subfigure}
    \begin{subfigure}[b]{.32\linewidth}
        %% Creator: Matplotlib, PGF backend
%%
%% To include the figure in your LaTeX document, write
%%   \input{<filename>.pgf}
%%
%% Make sure the required packages are loaded in your preamble
%%   \usepackage{pgf}
%%
%% Also ensure that all the required font packages are loaded; for instance,
%% the lmodern package is sometimes necessary when using math font.
%%   \usepackage{lmodern}
%%
%% Figures using additional raster images can only be included by \input if
%% they are in the same directory as the main LaTeX file. For loading figures
%% from other directories you can use the `import` package
%%   \usepackage{import}
%%
%% and then include the figures with
%%   \import{<path to file>}{<filename>.pgf}
%%
%% Matplotlib used the following preamble
%%   
%%   \usepackage{fontspec}
%%   \setmainfont{DejaVuSans.ttf}[Path=\detokenize{/home/fabio/Internodes-CM/.venv/lib/python3.8/site-packages/matplotlib/mpl-data/fonts/ttf/}]
%%   \setsansfont{DejaVuSans.ttf}[Path=\detokenize{/home/fabio/Internodes-CM/.venv/lib/python3.8/site-packages/matplotlib/mpl-data/fonts/ttf/}]
%%   \setmonofont{DejaVuSansMono.ttf}[Path=\detokenize{/home/fabio/Internodes-CM/.venv/lib/python3.8/site-packages/matplotlib/mpl-data/fonts/ttf/}]
%%   \makeatletter\@ifpackageloaded{underscore}{}{\usepackage[strings]{underscore}}\makeatother
%%
\begingroup%
\makeatletter%
\begin{pgfpicture}%
\pgfpathrectangle{\pgfpointorigin}{\pgfqpoint{1.982500in}{1.432000in}}%
\pgfusepath{use as bounding box, clip}%
\begin{pgfscope}%
\pgfsetbuttcap%
\pgfsetmiterjoin%
\definecolor{currentfill}{rgb}{1.000000,1.000000,1.000000}%
\pgfsetfillcolor{currentfill}%
\pgfsetlinewidth{0.000000pt}%
\definecolor{currentstroke}{rgb}{1.000000,1.000000,1.000000}%
\pgfsetstrokecolor{currentstroke}%
\pgfsetdash{}{0pt}%
\pgfpathmoveto{\pgfqpoint{0.000000in}{0.000000in}}%
\pgfpathlineto{\pgfqpoint{1.982500in}{0.000000in}}%
\pgfpathlineto{\pgfqpoint{1.982500in}{1.432000in}}%
\pgfpathlineto{\pgfqpoint{0.000000in}{1.432000in}}%
\pgfpathlineto{\pgfqpoint{0.000000in}{0.000000in}}%
\pgfpathclose%
\pgfusepath{fill}%
\end{pgfscope}%
\begin{pgfscope}%
\pgfpathrectangle{\pgfqpoint{0.100000in}{0.100000in}}{\pgfqpoint{1.782500in}{1.232000in}}%
\pgfusepath{clip}%
\pgfsetrectcap%
\pgfsetroundjoin%
\pgfsetlinewidth{0.250937pt}%
\definecolor{currentstroke}{rgb}{0.054902,0.262745,0.486275}%
\pgfsetstrokecolor{currentstroke}%
\pgfsetdash{}{0pt}%
\pgfpathmoveto{\pgfqpoint{0.451098in}{0.100000in}}%
\pgfpathlineto{\pgfqpoint{0.181023in}{0.100000in}}%
\pgfpathmoveto{\pgfqpoint{0.721174in}{0.100000in}}%
\pgfpathlineto{\pgfqpoint{0.451098in}{0.100000in}}%
\pgfpathmoveto{\pgfqpoint{0.991250in}{0.100000in}}%
\pgfpathlineto{\pgfqpoint{0.721174in}{0.100000in}}%
\pgfpathmoveto{\pgfqpoint{1.261326in}{0.100000in}}%
\pgfpathlineto{\pgfqpoint{0.991250in}{0.100000in}}%
\pgfpathmoveto{\pgfqpoint{1.531402in}{0.100000in}}%
\pgfpathlineto{\pgfqpoint{1.801477in}{0.100000in}}%
\pgfpathmoveto{\pgfqpoint{1.531402in}{0.100000in}}%
\pgfpathlineto{\pgfqpoint{1.261326in}{0.100000in}}%
\pgfpathmoveto{\pgfqpoint{1.801477in}{0.408000in}}%
\pgfpathlineto{\pgfqpoint{1.801477in}{0.100000in}}%
\pgfpathmoveto{\pgfqpoint{1.801477in}{0.408000in}}%
\pgfpathlineto{\pgfqpoint{1.801477in}{0.716000in}}%
\pgfpathmoveto{\pgfqpoint{1.639432in}{0.716000in}}%
\pgfpathlineto{\pgfqpoint{1.801477in}{0.716000in}}%
\pgfpathmoveto{\pgfqpoint{1.639432in}{0.716000in}}%
\pgfpathlineto{\pgfqpoint{1.477386in}{0.716000in}}%
\pgfpathmoveto{\pgfqpoint{1.407938in}{0.716000in}}%
\pgfpathlineto{\pgfqpoint{1.477386in}{0.716000in}}%
\pgfpathmoveto{\pgfqpoint{1.338490in}{0.716000in}}%
\pgfpathlineto{\pgfqpoint{1.407938in}{0.716000in}}%
\pgfpathmoveto{\pgfqpoint{1.269042in}{0.716000in}}%
\pgfpathlineto{\pgfqpoint{1.338490in}{0.716000in}}%
\pgfpathmoveto{\pgfqpoint{1.199594in}{0.716000in}}%
\pgfpathlineto{\pgfqpoint{1.269042in}{0.716000in}}%
\pgfpathmoveto{\pgfqpoint{1.130146in}{0.716000in}}%
\pgfpathlineto{\pgfqpoint{1.199594in}{0.716000in}}%
\pgfpathmoveto{\pgfqpoint{1.060698in}{0.716000in}}%
\pgfpathlineto{\pgfqpoint{1.130146in}{0.716000in}}%
\pgfpathmoveto{\pgfqpoint{0.991250in}{0.716000in}}%
\pgfpathlineto{\pgfqpoint{1.060698in}{0.716000in}}%
\pgfpathmoveto{\pgfqpoint{0.921802in}{0.716000in}}%
\pgfpathlineto{\pgfqpoint{0.991250in}{0.716000in}}%
\pgfpathmoveto{\pgfqpoint{0.852354in}{0.716000in}}%
\pgfpathlineto{\pgfqpoint{0.921802in}{0.716000in}}%
\pgfpathmoveto{\pgfqpoint{0.782906in}{0.716000in}}%
\pgfpathlineto{\pgfqpoint{0.852354in}{0.716000in}}%
\pgfpathmoveto{\pgfqpoint{0.713458in}{0.716000in}}%
\pgfpathlineto{\pgfqpoint{0.782906in}{0.716000in}}%
\pgfpathmoveto{\pgfqpoint{0.644010in}{0.716000in}}%
\pgfpathlineto{\pgfqpoint{0.713458in}{0.716000in}}%
\pgfpathmoveto{\pgfqpoint{0.574562in}{0.716000in}}%
\pgfpathlineto{\pgfqpoint{0.505114in}{0.716000in}}%
\pgfpathmoveto{\pgfqpoint{0.574562in}{0.716000in}}%
\pgfpathlineto{\pgfqpoint{0.644010in}{0.716000in}}%
\pgfpathmoveto{\pgfqpoint{0.343068in}{0.716000in}}%
\pgfpathlineto{\pgfqpoint{0.505114in}{0.716000in}}%
\pgfpathmoveto{\pgfqpoint{0.343068in}{0.716000in}}%
\pgfpathlineto{\pgfqpoint{0.181023in}{0.716000in}}%
\pgfpathmoveto{\pgfqpoint{0.181023in}{0.408000in}}%
\pgfpathlineto{\pgfqpoint{0.181023in}{0.100000in}}%
\pgfpathmoveto{\pgfqpoint{0.181023in}{0.408000in}}%
\pgfpathlineto{\pgfqpoint{0.181023in}{0.716000in}}%
\pgfpathmoveto{\pgfqpoint{1.373214in}{0.396057in}}%
\pgfpathlineto{\pgfqpoint{1.261326in}{0.100000in}}%
\pgfpathmoveto{\pgfqpoint{1.082944in}{0.402689in}}%
\pgfpathlineto{\pgfqpoint{0.991250in}{0.100000in}}%
\pgfpathmoveto{\pgfqpoint{1.233147in}{0.520967in}}%
\pgfpathlineto{\pgfqpoint{1.373214in}{0.396057in}}%
\pgfpathmoveto{\pgfqpoint{1.233147in}{0.520967in}}%
\pgfpathlineto{\pgfqpoint{1.082944in}{0.402689in}}%
\pgfpathmoveto{\pgfqpoint{0.956117in}{0.522274in}}%
\pgfpathlineto{\pgfqpoint{1.082944in}{0.402689in}}%
\pgfpathmoveto{\pgfqpoint{0.956117in}{0.522274in}}%
\pgfpathlineto{\pgfqpoint{0.821141in}{0.416705in}}%
\pgfpathmoveto{\pgfqpoint{0.679173in}{0.523635in}}%
\pgfpathlineto{\pgfqpoint{0.821141in}{0.416705in}}%
\pgfpathmoveto{\pgfqpoint{0.679173in}{0.523635in}}%
\pgfpathlineto{\pgfqpoint{0.539838in}{0.401685in}}%
\pgfpathmoveto{\pgfqpoint{1.535558in}{0.497212in}}%
\pgfpathlineto{\pgfqpoint{1.801477in}{0.408000in}}%
\pgfpathmoveto{\pgfqpoint{1.535558in}{0.497212in}}%
\pgfpathlineto{\pgfqpoint{1.373214in}{0.396057in}}%
\pgfpathmoveto{\pgfqpoint{1.376927in}{0.550483in}}%
\pgfpathlineto{\pgfqpoint{1.373214in}{0.396057in}}%
\pgfpathmoveto{\pgfqpoint{1.376927in}{0.550483in}}%
\pgfpathlineto{\pgfqpoint{1.233147in}{0.520967in}}%
\pgfpathmoveto{\pgfqpoint{1.376927in}{0.550483in}}%
\pgfpathlineto{\pgfqpoint{1.535558in}{0.497212in}}%
\pgfpathmoveto{\pgfqpoint{1.093079in}{0.556190in}}%
\pgfpathlineto{\pgfqpoint{1.082944in}{0.402689in}}%
\pgfpathmoveto{\pgfqpoint{1.093079in}{0.556190in}}%
\pgfpathlineto{\pgfqpoint{1.233147in}{0.520967in}}%
\pgfpathmoveto{\pgfqpoint{1.093079in}{0.556190in}}%
\pgfpathlineto{\pgfqpoint{0.956117in}{0.522274in}}%
\pgfpathmoveto{\pgfqpoint{0.818220in}{0.558974in}}%
\pgfpathlineto{\pgfqpoint{0.821141in}{0.416705in}}%
\pgfpathmoveto{\pgfqpoint{0.818220in}{0.558974in}}%
\pgfpathlineto{\pgfqpoint{0.956117in}{0.522274in}}%
\pgfpathmoveto{\pgfqpoint{0.818220in}{0.558974in}}%
\pgfpathlineto{\pgfqpoint{0.679173in}{0.523635in}}%
\pgfpathmoveto{\pgfqpoint{0.525112in}{0.550213in}}%
\pgfpathlineto{\pgfqpoint{0.539838in}{0.401685in}}%
\pgfpathmoveto{\pgfqpoint{0.525112in}{0.550213in}}%
\pgfpathlineto{\pgfqpoint{0.679173in}{0.523635in}}%
\pgfpathmoveto{\pgfqpoint{0.356186in}{0.497584in}}%
\pgfpathlineto{\pgfqpoint{0.181023in}{0.716000in}}%
\pgfpathmoveto{\pgfqpoint{0.356186in}{0.497584in}}%
\pgfpathlineto{\pgfqpoint{0.343068in}{0.716000in}}%
\pgfpathmoveto{\pgfqpoint{0.356186in}{0.497584in}}%
\pgfpathlineto{\pgfqpoint{0.181023in}{0.408000in}}%
\pgfpathmoveto{\pgfqpoint{0.356186in}{0.497584in}}%
\pgfpathlineto{\pgfqpoint{0.539838in}{0.401685in}}%
\pgfpathmoveto{\pgfqpoint{0.356186in}{0.497584in}}%
\pgfpathlineto{\pgfqpoint{0.525112in}{0.550213in}}%
\pgfpathmoveto{\pgfqpoint{1.303766in}{0.622307in}}%
\pgfpathlineto{\pgfqpoint{1.338490in}{0.716000in}}%
\pgfpathmoveto{\pgfqpoint{1.303766in}{0.622307in}}%
\pgfpathlineto{\pgfqpoint{1.269042in}{0.716000in}}%
\pgfpathmoveto{\pgfqpoint{1.303766in}{0.622307in}}%
\pgfpathlineto{\pgfqpoint{1.233147in}{0.520967in}}%
\pgfpathmoveto{\pgfqpoint{1.303766in}{0.622307in}}%
\pgfpathlineto{\pgfqpoint{1.376927in}{0.550483in}}%
\pgfpathmoveto{\pgfqpoint{1.164870in}{0.623450in}}%
\pgfpathlineto{\pgfqpoint{1.199594in}{0.716000in}}%
\pgfpathmoveto{\pgfqpoint{1.164870in}{0.623450in}}%
\pgfpathlineto{\pgfqpoint{1.130146in}{0.716000in}}%
\pgfpathmoveto{\pgfqpoint{1.164870in}{0.623450in}}%
\pgfpathlineto{\pgfqpoint{1.233147in}{0.520967in}}%
\pgfpathmoveto{\pgfqpoint{1.164870in}{0.623450in}}%
\pgfpathlineto{\pgfqpoint{1.093079in}{0.556190in}}%
\pgfpathmoveto{\pgfqpoint{1.025974in}{0.623450in}}%
\pgfpathlineto{\pgfqpoint{1.060698in}{0.716000in}}%
\pgfpathmoveto{\pgfqpoint{1.025974in}{0.623450in}}%
\pgfpathlineto{\pgfqpoint{0.991250in}{0.716000in}}%
\pgfpathmoveto{\pgfqpoint{1.025974in}{0.623450in}}%
\pgfpathlineto{\pgfqpoint{0.956117in}{0.522274in}}%
\pgfpathmoveto{\pgfqpoint{1.025974in}{0.623450in}}%
\pgfpathlineto{\pgfqpoint{1.093079in}{0.556190in}}%
\pgfpathmoveto{\pgfqpoint{0.887078in}{0.623460in}}%
\pgfpathlineto{\pgfqpoint{0.921802in}{0.716000in}}%
\pgfpathmoveto{\pgfqpoint{0.887078in}{0.623460in}}%
\pgfpathlineto{\pgfqpoint{0.852354in}{0.716000in}}%
\pgfpathmoveto{\pgfqpoint{0.887078in}{0.623460in}}%
\pgfpathlineto{\pgfqpoint{0.956117in}{0.522274in}}%
\pgfpathmoveto{\pgfqpoint{0.887078in}{0.623460in}}%
\pgfpathlineto{\pgfqpoint{0.818220in}{0.558974in}}%
\pgfpathmoveto{\pgfqpoint{0.748182in}{0.623460in}}%
\pgfpathlineto{\pgfqpoint{0.782906in}{0.716000in}}%
\pgfpathmoveto{\pgfqpoint{0.748182in}{0.623460in}}%
\pgfpathlineto{\pgfqpoint{0.713458in}{0.716000in}}%
\pgfpathmoveto{\pgfqpoint{0.748182in}{0.623460in}}%
\pgfpathlineto{\pgfqpoint{0.679173in}{0.523635in}}%
\pgfpathmoveto{\pgfqpoint{0.748182in}{0.623460in}}%
\pgfpathlineto{\pgfqpoint{0.818220in}{0.558974in}}%
\pgfpathmoveto{\pgfqpoint{0.609286in}{0.623460in}}%
\pgfpathlineto{\pgfqpoint{0.644010in}{0.716000in}}%
\pgfpathmoveto{\pgfqpoint{0.609286in}{0.623460in}}%
\pgfpathlineto{\pgfqpoint{0.574562in}{0.716000in}}%
\pgfpathmoveto{\pgfqpoint{0.609286in}{0.623460in}}%
\pgfpathlineto{\pgfqpoint{0.679173in}{0.523635in}}%
\pgfpathmoveto{\pgfqpoint{0.609286in}{0.623460in}}%
\pgfpathlineto{\pgfqpoint{0.525112in}{0.550213in}}%
\pgfpathmoveto{\pgfqpoint{1.453902in}{0.624729in}}%
\pgfpathlineto{\pgfqpoint{1.477386in}{0.716000in}}%
\pgfpathmoveto{\pgfqpoint{1.453902in}{0.624729in}}%
\pgfpathlineto{\pgfqpoint{1.407938in}{0.716000in}}%
\pgfpathmoveto{\pgfqpoint{1.453902in}{0.624729in}}%
\pgfpathlineto{\pgfqpoint{1.535558in}{0.497212in}}%
\pgfpathmoveto{\pgfqpoint{1.453902in}{0.624729in}}%
\pgfpathlineto{\pgfqpoint{1.376927in}{0.550483in}}%
\pgfpathmoveto{\pgfqpoint{0.939717in}{0.332038in}}%
\pgfpathlineto{\pgfqpoint{0.991250in}{0.100000in}}%
\pgfpathmoveto{\pgfqpoint{0.939717in}{0.332038in}}%
\pgfpathlineto{\pgfqpoint{1.082944in}{0.402689in}}%
\pgfpathmoveto{\pgfqpoint{0.939717in}{0.332038in}}%
\pgfpathlineto{\pgfqpoint{0.821141in}{0.416705in}}%
\pgfpathmoveto{\pgfqpoint{0.939717in}{0.332038in}}%
\pgfpathlineto{\pgfqpoint{0.956117in}{0.522274in}}%
\pgfpathmoveto{\pgfqpoint{0.433515in}{0.614303in}}%
\pgfpathlineto{\pgfqpoint{0.505114in}{0.716000in}}%
\pgfpathmoveto{\pgfqpoint{0.433515in}{0.614303in}}%
\pgfpathlineto{\pgfqpoint{0.343068in}{0.716000in}}%
\pgfpathmoveto{\pgfqpoint{0.433515in}{0.614303in}}%
\pgfpathlineto{\pgfqpoint{0.525112in}{0.550213in}}%
\pgfpathmoveto{\pgfqpoint{0.433515in}{0.614303in}}%
\pgfpathlineto{\pgfqpoint{0.356186in}{0.497584in}}%
\pgfpathmoveto{\pgfqpoint{1.374740in}{0.644995in}}%
\pgfpathlineto{\pgfqpoint{1.407938in}{0.716000in}}%
\pgfpathmoveto{\pgfqpoint{1.374740in}{0.644995in}}%
\pgfpathlineto{\pgfqpoint{1.338490in}{0.716000in}}%
\pgfpathmoveto{\pgfqpoint{1.374740in}{0.644995in}}%
\pgfpathlineto{\pgfqpoint{1.376927in}{0.550483in}}%
\pgfpathmoveto{\pgfqpoint{1.374740in}{0.644995in}}%
\pgfpathlineto{\pgfqpoint{1.303766in}{0.622307in}}%
\pgfpathmoveto{\pgfqpoint{1.374740in}{0.644995in}}%
\pgfpathlineto{\pgfqpoint{1.453902in}{0.624729in}}%
\pgfpathmoveto{\pgfqpoint{1.094954in}{0.647018in}}%
\pgfpathlineto{\pgfqpoint{1.130146in}{0.716000in}}%
\pgfpathmoveto{\pgfqpoint{1.094954in}{0.647018in}}%
\pgfpathlineto{\pgfqpoint{1.060698in}{0.716000in}}%
\pgfpathmoveto{\pgfqpoint{1.094954in}{0.647018in}}%
\pgfpathlineto{\pgfqpoint{1.093079in}{0.556190in}}%
\pgfpathmoveto{\pgfqpoint{1.094954in}{0.647018in}}%
\pgfpathlineto{\pgfqpoint{1.164870in}{0.623450in}}%
\pgfpathmoveto{\pgfqpoint{1.094954in}{0.647018in}}%
\pgfpathlineto{\pgfqpoint{1.025974in}{0.623450in}}%
\pgfpathmoveto{\pgfqpoint{1.234318in}{0.644309in}}%
\pgfpathlineto{\pgfqpoint{1.269042in}{0.716000in}}%
\pgfpathmoveto{\pgfqpoint{1.234318in}{0.644309in}}%
\pgfpathlineto{\pgfqpoint{1.199594in}{0.716000in}}%
\pgfpathmoveto{\pgfqpoint{1.234318in}{0.644309in}}%
\pgfpathlineto{\pgfqpoint{1.233147in}{0.520967in}}%
\pgfpathmoveto{\pgfqpoint{1.234318in}{0.644309in}}%
\pgfpathlineto{\pgfqpoint{1.303766in}{0.622307in}}%
\pgfpathmoveto{\pgfqpoint{1.234318in}{0.644309in}}%
\pgfpathlineto{\pgfqpoint{1.164870in}{0.623450in}}%
\pgfpathmoveto{\pgfqpoint{0.956444in}{0.640237in}}%
\pgfpathlineto{\pgfqpoint{0.991250in}{0.716000in}}%
\pgfpathmoveto{\pgfqpoint{0.956444in}{0.640237in}}%
\pgfpathlineto{\pgfqpoint{0.921802in}{0.716000in}}%
\pgfpathmoveto{\pgfqpoint{0.956444in}{0.640237in}}%
\pgfpathlineto{\pgfqpoint{0.956117in}{0.522274in}}%
\pgfpathmoveto{\pgfqpoint{0.956444in}{0.640237in}}%
\pgfpathlineto{\pgfqpoint{1.025974in}{0.623450in}}%
\pgfpathmoveto{\pgfqpoint{0.956444in}{0.640237in}}%
\pgfpathlineto{\pgfqpoint{0.887078in}{0.623460in}}%
\pgfpathmoveto{\pgfqpoint{0.817690in}{0.645978in}}%
\pgfpathlineto{\pgfqpoint{0.852354in}{0.716000in}}%
\pgfpathmoveto{\pgfqpoint{0.817690in}{0.645978in}}%
\pgfpathlineto{\pgfqpoint{0.782906in}{0.716000in}}%
\pgfpathmoveto{\pgfqpoint{0.817690in}{0.645978in}}%
\pgfpathlineto{\pgfqpoint{0.818220in}{0.558974in}}%
\pgfpathmoveto{\pgfqpoint{0.817690in}{0.645978in}}%
\pgfpathlineto{\pgfqpoint{0.887078in}{0.623460in}}%
\pgfpathmoveto{\pgfqpoint{0.817690in}{0.645978in}}%
\pgfpathlineto{\pgfqpoint{0.748182in}{0.623460in}}%
\pgfpathmoveto{\pgfqpoint{1.188376in}{0.303943in}}%
\pgfpathlineto{\pgfqpoint{0.991250in}{0.100000in}}%
\pgfpathmoveto{\pgfqpoint{1.188376in}{0.303943in}}%
\pgfpathlineto{\pgfqpoint{1.261326in}{0.100000in}}%
\pgfpathmoveto{\pgfqpoint{1.188376in}{0.303943in}}%
\pgfpathlineto{\pgfqpoint{1.373214in}{0.396057in}}%
\pgfpathmoveto{\pgfqpoint{1.188376in}{0.303943in}}%
\pgfpathlineto{\pgfqpoint{1.082944in}{0.402689in}}%
\pgfpathmoveto{\pgfqpoint{1.188376in}{0.303943in}}%
\pgfpathlineto{\pgfqpoint{1.233147in}{0.520967in}}%
\pgfpathmoveto{\pgfqpoint{0.696453in}{0.313796in}}%
\pgfpathlineto{\pgfqpoint{0.721174in}{0.100000in}}%
\pgfpathmoveto{\pgfqpoint{0.696453in}{0.313796in}}%
\pgfpathlineto{\pgfqpoint{0.821141in}{0.416705in}}%
\pgfpathmoveto{\pgfqpoint{0.696453in}{0.313796in}}%
\pgfpathlineto{\pgfqpoint{0.539838in}{0.401685in}}%
\pgfpathmoveto{\pgfqpoint{0.696453in}{0.313796in}}%
\pgfpathlineto{\pgfqpoint{0.679173in}{0.523635in}}%
\pgfpathmoveto{\pgfqpoint{0.678822in}{0.640511in}}%
\pgfpathlineto{\pgfqpoint{0.713458in}{0.716000in}}%
\pgfpathmoveto{\pgfqpoint{0.678822in}{0.640511in}}%
\pgfpathlineto{\pgfqpoint{0.644010in}{0.716000in}}%
\pgfpathmoveto{\pgfqpoint{0.678822in}{0.640511in}}%
\pgfpathlineto{\pgfqpoint{0.679173in}{0.523635in}}%
\pgfpathmoveto{\pgfqpoint{0.678822in}{0.640511in}}%
\pgfpathlineto{\pgfqpoint{0.748182in}{0.623460in}}%
\pgfpathmoveto{\pgfqpoint{0.678822in}{0.640511in}}%
\pgfpathlineto{\pgfqpoint{0.609286in}{0.623460in}}%
\pgfpathmoveto{\pgfqpoint{1.556669in}{0.624742in}}%
\pgfpathlineto{\pgfqpoint{1.477386in}{0.716000in}}%
\pgfpathmoveto{\pgfqpoint{1.556669in}{0.624742in}}%
\pgfpathlineto{\pgfqpoint{1.639432in}{0.716000in}}%
\pgfpathmoveto{\pgfqpoint{1.556669in}{0.624742in}}%
\pgfpathlineto{\pgfqpoint{1.535558in}{0.497212in}}%
\pgfpathmoveto{\pgfqpoint{1.556669in}{0.624742in}}%
\pgfpathlineto{\pgfqpoint{1.453902in}{0.624729in}}%
\pgfpathmoveto{\pgfqpoint{0.539838in}{0.644311in}}%
\pgfpathlineto{\pgfqpoint{0.505114in}{0.716000in}}%
\pgfpathmoveto{\pgfqpoint{0.539838in}{0.644311in}}%
\pgfpathlineto{\pgfqpoint{0.574562in}{0.716000in}}%
\pgfpathmoveto{\pgfqpoint{0.539838in}{0.644311in}}%
\pgfpathlineto{\pgfqpoint{0.525112in}{0.550213in}}%
\pgfpathmoveto{\pgfqpoint{0.539838in}{0.644311in}}%
\pgfpathlineto{\pgfqpoint{0.609286in}{0.623460in}}%
\pgfpathmoveto{\pgfqpoint{0.539838in}{0.644311in}}%
\pgfpathlineto{\pgfqpoint{0.433515in}{0.614303in}}%
\pgfpathmoveto{\pgfqpoint{0.379905in}{0.288250in}}%
\pgfpathlineto{\pgfqpoint{0.181023in}{0.100000in}}%
\pgfpathmoveto{\pgfqpoint{0.379905in}{0.288250in}}%
\pgfpathlineto{\pgfqpoint{0.451098in}{0.100000in}}%
\pgfpathmoveto{\pgfqpoint{0.379905in}{0.288250in}}%
\pgfpathlineto{\pgfqpoint{0.181023in}{0.408000in}}%
\pgfpathmoveto{\pgfqpoint{0.379905in}{0.288250in}}%
\pgfpathlineto{\pgfqpoint{0.539838in}{0.401685in}}%
\pgfpathmoveto{\pgfqpoint{0.379905in}{0.288250in}}%
\pgfpathlineto{\pgfqpoint{0.356186in}{0.497584in}}%
\pgfpathmoveto{\pgfqpoint{1.528881in}{0.301240in}}%
\pgfpathlineto{\pgfqpoint{1.801477in}{0.100000in}}%
\pgfpathmoveto{\pgfqpoint{1.528881in}{0.301240in}}%
\pgfpathlineto{\pgfqpoint{1.261326in}{0.100000in}}%
\pgfpathmoveto{\pgfqpoint{1.528881in}{0.301240in}}%
\pgfpathlineto{\pgfqpoint{1.531402in}{0.100000in}}%
\pgfpathmoveto{\pgfqpoint{1.528881in}{0.301240in}}%
\pgfpathlineto{\pgfqpoint{1.801477in}{0.408000in}}%
\pgfpathmoveto{\pgfqpoint{1.528881in}{0.301240in}}%
\pgfpathlineto{\pgfqpoint{1.373214in}{0.396057in}}%
\pgfpathmoveto{\pgfqpoint{1.528881in}{0.301240in}}%
\pgfpathlineto{\pgfqpoint{1.535558in}{0.497212in}}%
\pgfpathmoveto{\pgfqpoint{1.666923in}{0.592391in}}%
\pgfpathlineto{\pgfqpoint{1.801477in}{0.716000in}}%
\pgfpathmoveto{\pgfqpoint{1.666923in}{0.592391in}}%
\pgfpathlineto{\pgfqpoint{1.801477in}{0.408000in}}%
\pgfpathmoveto{\pgfqpoint{1.666923in}{0.592391in}}%
\pgfpathlineto{\pgfqpoint{1.639432in}{0.716000in}}%
\pgfpathmoveto{\pgfqpoint{1.666923in}{0.592391in}}%
\pgfpathlineto{\pgfqpoint{1.535558in}{0.497212in}}%
\pgfpathmoveto{\pgfqpoint{1.666923in}{0.592391in}}%
\pgfpathlineto{\pgfqpoint{1.556669in}{0.624742in}}%
\pgfpathmoveto{\pgfqpoint{0.842327in}{0.230907in}}%
\pgfpathlineto{\pgfqpoint{0.721174in}{0.100000in}}%
\pgfpathmoveto{\pgfqpoint{0.842327in}{0.230907in}}%
\pgfpathlineto{\pgfqpoint{0.991250in}{0.100000in}}%
\pgfpathmoveto{\pgfqpoint{0.842327in}{0.230907in}}%
\pgfpathlineto{\pgfqpoint{0.821141in}{0.416705in}}%
\pgfpathmoveto{\pgfqpoint{0.842327in}{0.230907in}}%
\pgfpathlineto{\pgfqpoint{0.939717in}{0.332038in}}%
\pgfpathmoveto{\pgfqpoint{0.842327in}{0.230907in}}%
\pgfpathlineto{\pgfqpoint{0.696453in}{0.313796in}}%
\pgfpathmoveto{\pgfqpoint{0.557694in}{0.240746in}}%
\pgfpathlineto{\pgfqpoint{0.451098in}{0.100000in}}%
\pgfpathmoveto{\pgfqpoint{0.557694in}{0.240746in}}%
\pgfpathlineto{\pgfqpoint{0.721174in}{0.100000in}}%
\pgfpathmoveto{\pgfqpoint{0.557694in}{0.240746in}}%
\pgfpathlineto{\pgfqpoint{0.539838in}{0.401685in}}%
\pgfpathmoveto{\pgfqpoint{0.557694in}{0.240746in}}%
\pgfpathlineto{\pgfqpoint{0.696453in}{0.313796in}}%
\pgfpathmoveto{\pgfqpoint{0.557694in}{0.240746in}}%
\pgfpathlineto{\pgfqpoint{0.379905in}{0.288250in}}%
\pgfpathlineto{\pgfqpoint{0.379905in}{0.288250in}}%
\pgfusepath{stroke}%
\end{pgfscope}%
\begin{pgfscope}%
\pgfpathrectangle{\pgfqpoint{0.100000in}{0.100000in}}{\pgfqpoint{1.782500in}{1.232000in}}%
\pgfusepath{clip}%
\pgfsetrectcap%
\pgfsetroundjoin%
\pgfsetlinewidth{0.250937pt}%
\definecolor{currentstroke}{rgb}{0.835294,0.321569,0.035294}%
\pgfsetstrokecolor{currentstroke}%
\pgfsetdash{}{0pt}%
\pgfpathmoveto{\pgfqpoint{0.505114in}{1.001892in}}%
\pgfpathlineto{\pgfqpoint{0.451098in}{1.270400in}}%
\pgfpathmoveto{\pgfqpoint{1.531402in}{1.270400in}}%
\pgfpathlineto{\pgfqpoint{1.477386in}{1.001892in}}%
\pgfpathmoveto{\pgfqpoint{0.721174in}{1.270400in}}%
\pgfpathlineto{\pgfqpoint{0.991250in}{1.270400in}}%
\pgfpathmoveto{\pgfqpoint{0.721174in}{1.270400in}}%
\pgfpathlineto{\pgfqpoint{0.451098in}{1.270400in}}%
\pgfpathmoveto{\pgfqpoint{0.548824in}{0.917012in}}%
\pgfpathlineto{\pgfqpoint{0.505114in}{1.001892in}}%
\pgfpathmoveto{\pgfqpoint{0.603831in}{0.841156in}}%
\pgfpathlineto{\pgfqpoint{0.548824in}{0.917012in}}%
\pgfpathmoveto{\pgfqpoint{0.668731in}{0.776261in}}%
\pgfpathlineto{\pgfqpoint{0.603831in}{0.841156in}}%
\pgfpathmoveto{\pgfqpoint{0.741867in}{0.723983in}}%
\pgfpathlineto{\pgfqpoint{0.668731in}{0.776261in}}%
\pgfpathmoveto{\pgfqpoint{0.821370in}{0.685658in}}%
\pgfpathlineto{\pgfqpoint{0.741867in}{0.723983in}}%
\pgfpathmoveto{\pgfqpoint{0.905212in}{0.662265in}}%
\pgfpathlineto{\pgfqpoint{0.821370in}{0.685658in}}%
\pgfpathmoveto{\pgfqpoint{0.991250in}{0.654400in}}%
\pgfpathlineto{\pgfqpoint{0.905212in}{0.662265in}}%
\pgfpathmoveto{\pgfqpoint{1.077288in}{0.662265in}}%
\pgfpathlineto{\pgfqpoint{0.991250in}{0.654400in}}%
\pgfpathmoveto{\pgfqpoint{1.161130in}{0.685658in}}%
\pgfpathlineto{\pgfqpoint{1.077288in}{0.662265in}}%
\pgfpathmoveto{\pgfqpoint{1.240633in}{0.723983in}}%
\pgfpathlineto{\pgfqpoint{1.161130in}{0.685658in}}%
\pgfpathmoveto{\pgfqpoint{1.313769in}{0.776261in}}%
\pgfpathlineto{\pgfqpoint{1.240633in}{0.723983in}}%
\pgfpathmoveto{\pgfqpoint{1.378669in}{0.841156in}}%
\pgfpathlineto{\pgfqpoint{1.313769in}{0.776261in}}%
\pgfpathmoveto{\pgfqpoint{1.433676in}{0.917012in}}%
\pgfpathlineto{\pgfqpoint{1.477386in}{1.001892in}}%
\pgfpathmoveto{\pgfqpoint{1.433676in}{0.917012in}}%
\pgfpathlineto{\pgfqpoint{1.378669in}{0.841156in}}%
\pgfpathmoveto{\pgfqpoint{1.261326in}{1.270400in}}%
\pgfpathlineto{\pgfqpoint{0.991250in}{1.270400in}}%
\pgfpathmoveto{\pgfqpoint{1.261326in}{1.270400in}}%
\pgfpathlineto{\pgfqpoint{1.531402in}{1.270400in}}%
\pgfpathmoveto{\pgfqpoint{0.837830in}{1.028470in}}%
\pgfpathlineto{\pgfqpoint{0.991250in}{1.270400in}}%
\pgfpathmoveto{\pgfqpoint{0.837830in}{1.028470in}}%
\pgfpathlineto{\pgfqpoint{0.721174in}{1.270400in}}%
\pgfpathmoveto{\pgfqpoint{0.961801in}{0.851412in}}%
\pgfpathlineto{\pgfqpoint{1.113091in}{0.944697in}}%
\pgfpathmoveto{\pgfqpoint{0.961801in}{0.851412in}}%
\pgfpathlineto{\pgfqpoint{0.837830in}{1.028470in}}%
\pgfpathmoveto{\pgfqpoint{1.273497in}{1.004211in}}%
\pgfpathlineto{\pgfqpoint{1.113091in}{0.944697in}}%
\pgfpathmoveto{\pgfqpoint{0.674832in}{1.049989in}}%
\pgfpathlineto{\pgfqpoint{0.721174in}{1.270400in}}%
\pgfpathmoveto{\pgfqpoint{0.674832in}{1.049989in}}%
\pgfpathlineto{\pgfqpoint{0.837830in}{1.028470in}}%
\pgfpathmoveto{\pgfqpoint{1.093189in}{0.806475in}}%
\pgfpathlineto{\pgfqpoint{1.077288in}{0.662265in}}%
\pgfpathmoveto{\pgfqpoint{1.093189in}{0.806475in}}%
\pgfpathlineto{\pgfqpoint{1.161130in}{0.685658in}}%
\pgfpathmoveto{\pgfqpoint{1.093189in}{0.806475in}}%
\pgfpathlineto{\pgfqpoint{1.113091in}{0.944697in}}%
\pgfpathmoveto{\pgfqpoint{1.093189in}{0.806475in}}%
\pgfpathlineto{\pgfqpoint{0.961801in}{0.851412in}}%
\pgfpathmoveto{\pgfqpoint{1.212025in}{0.863682in}}%
\pgfpathlineto{\pgfqpoint{1.240633in}{0.723983in}}%
\pgfpathmoveto{\pgfqpoint{1.212025in}{0.863682in}}%
\pgfpathlineto{\pgfqpoint{1.313769in}{0.776261in}}%
\pgfpathmoveto{\pgfqpoint{1.212025in}{0.863682in}}%
\pgfpathlineto{\pgfqpoint{1.113091in}{0.944697in}}%
\pgfpathmoveto{\pgfqpoint{1.212025in}{0.863682in}}%
\pgfpathlineto{\pgfqpoint{1.273497in}{1.004211in}}%
\pgfpathmoveto{\pgfqpoint{1.212025in}{0.863682in}}%
\pgfpathlineto{\pgfqpoint{1.093189in}{0.806475in}}%
\pgfpathmoveto{\pgfqpoint{0.822435in}{0.830135in}}%
\pgfpathlineto{\pgfqpoint{0.741867in}{0.723983in}}%
\pgfpathmoveto{\pgfqpoint{0.822435in}{0.830135in}}%
\pgfpathlineto{\pgfqpoint{0.821370in}{0.685658in}}%
\pgfpathmoveto{\pgfqpoint{0.822435in}{0.830135in}}%
\pgfpathlineto{\pgfqpoint{0.837830in}{1.028470in}}%
\pgfpathmoveto{\pgfqpoint{0.822435in}{0.830135in}}%
\pgfpathlineto{\pgfqpoint{0.961801in}{0.851412in}}%
\pgfpathmoveto{\pgfqpoint{0.716798in}{0.910757in}}%
\pgfpathlineto{\pgfqpoint{0.603831in}{0.841156in}}%
\pgfpathmoveto{\pgfqpoint{0.716798in}{0.910757in}}%
\pgfpathlineto{\pgfqpoint{0.668731in}{0.776261in}}%
\pgfpathmoveto{\pgfqpoint{0.716798in}{0.910757in}}%
\pgfpathlineto{\pgfqpoint{0.837830in}{1.028470in}}%
\pgfpathmoveto{\pgfqpoint{0.716798in}{0.910757in}}%
\pgfpathlineto{\pgfqpoint{0.674832in}{1.049989in}}%
\pgfpathmoveto{\pgfqpoint{0.716798in}{0.910757in}}%
\pgfpathlineto{\pgfqpoint{0.822435in}{0.830135in}}%
\pgfpathmoveto{\pgfqpoint{1.362495in}{1.129252in}}%
\pgfpathlineto{\pgfqpoint{1.477386in}{1.001892in}}%
\pgfpathmoveto{\pgfqpoint{1.362495in}{1.129252in}}%
\pgfpathlineto{\pgfqpoint{1.531402in}{1.270400in}}%
\pgfpathmoveto{\pgfqpoint{1.362495in}{1.129252in}}%
\pgfpathlineto{\pgfqpoint{1.261326in}{1.270400in}}%
\pgfpathmoveto{\pgfqpoint{1.362495in}{1.129252in}}%
\pgfpathlineto{\pgfqpoint{1.273497in}{1.004211in}}%
\pgfpathmoveto{\pgfqpoint{0.953154in}{0.728383in}}%
\pgfpathlineto{\pgfqpoint{0.905212in}{0.662265in}}%
\pgfpathmoveto{\pgfqpoint{0.953154in}{0.728383in}}%
\pgfpathlineto{\pgfqpoint{0.991250in}{0.654400in}}%
\pgfpathmoveto{\pgfqpoint{0.953154in}{0.728383in}}%
\pgfpathlineto{\pgfqpoint{0.961801in}{0.851412in}}%
\pgfpathmoveto{\pgfqpoint{1.297122in}{0.872566in}}%
\pgfpathlineto{\pgfqpoint{1.313769in}{0.776261in}}%
\pgfpathmoveto{\pgfqpoint{1.297122in}{0.872566in}}%
\pgfpathlineto{\pgfqpoint{1.378669in}{0.841156in}}%
\pgfpathmoveto{\pgfqpoint{1.297122in}{0.872566in}}%
\pgfpathlineto{\pgfqpoint{1.273497in}{1.004211in}}%
\pgfpathmoveto{\pgfqpoint{1.297122in}{0.872566in}}%
\pgfpathlineto{\pgfqpoint{1.212025in}{0.863682in}}%
\pgfpathmoveto{\pgfqpoint{1.173657in}{0.778274in}}%
\pgfpathlineto{\pgfqpoint{1.161130in}{0.685658in}}%
\pgfpathmoveto{\pgfqpoint{1.173657in}{0.778274in}}%
\pgfpathlineto{\pgfqpoint{1.240633in}{0.723983in}}%
\pgfpathmoveto{\pgfqpoint{1.173657in}{0.778274in}}%
\pgfpathlineto{\pgfqpoint{1.093189in}{0.806475in}}%
\pgfpathmoveto{\pgfqpoint{1.173657in}{0.778274in}}%
\pgfpathlineto{\pgfqpoint{1.212025in}{0.863682in}}%
\pgfpathmoveto{\pgfqpoint{1.375529in}{1.013033in}}%
\pgfpathlineto{\pgfqpoint{1.477386in}{1.001892in}}%
\pgfpathmoveto{\pgfqpoint{1.375529in}{1.013033in}}%
\pgfpathlineto{\pgfqpoint{1.433676in}{0.917012in}}%
\pgfpathmoveto{\pgfqpoint{1.375529in}{1.013033in}}%
\pgfpathlineto{\pgfqpoint{1.273497in}{1.004211in}}%
\pgfpathmoveto{\pgfqpoint{1.375529in}{1.013033in}}%
\pgfpathlineto{\pgfqpoint{1.362495in}{1.129252in}}%
\pgfpathmoveto{\pgfqpoint{0.742156in}{0.817783in}}%
\pgfpathlineto{\pgfqpoint{0.668731in}{0.776261in}}%
\pgfpathmoveto{\pgfqpoint{0.742156in}{0.817783in}}%
\pgfpathlineto{\pgfqpoint{0.741867in}{0.723983in}}%
\pgfpathmoveto{\pgfqpoint{0.742156in}{0.817783in}}%
\pgfpathlineto{\pgfqpoint{0.822435in}{0.830135in}}%
\pgfpathmoveto{\pgfqpoint{0.742156in}{0.817783in}}%
\pgfpathlineto{\pgfqpoint{0.716798in}{0.910757in}}%
\pgfpathmoveto{\pgfqpoint{0.633716in}{0.933208in}}%
\pgfpathlineto{\pgfqpoint{0.548824in}{0.917012in}}%
\pgfpathmoveto{\pgfqpoint{0.633716in}{0.933208in}}%
\pgfpathlineto{\pgfqpoint{0.603831in}{0.841156in}}%
\pgfpathmoveto{\pgfqpoint{0.633716in}{0.933208in}}%
\pgfpathlineto{\pgfqpoint{0.674832in}{1.049989in}}%
\pgfpathmoveto{\pgfqpoint{0.633716in}{0.933208in}}%
\pgfpathlineto{\pgfqpoint{0.716798in}{0.910757in}}%
\pgfpathmoveto{\pgfqpoint{0.572102in}{1.136818in}}%
\pgfpathlineto{\pgfqpoint{0.451098in}{1.270400in}}%
\pgfpathmoveto{\pgfqpoint{0.572102in}{1.136818in}}%
\pgfpathlineto{\pgfqpoint{0.505114in}{1.001892in}}%
\pgfpathmoveto{\pgfqpoint{0.572102in}{1.136818in}}%
\pgfpathlineto{\pgfqpoint{0.721174in}{1.270400in}}%
\pgfpathmoveto{\pgfqpoint{0.572102in}{1.136818in}}%
\pgfpathlineto{\pgfqpoint{0.674832in}{1.049989in}}%
\pgfpathmoveto{\pgfqpoint{1.029231in}{0.730021in}}%
\pgfpathlineto{\pgfqpoint{0.991250in}{0.654400in}}%
\pgfpathmoveto{\pgfqpoint{1.029231in}{0.730021in}}%
\pgfpathlineto{\pgfqpoint{1.077288in}{0.662265in}}%
\pgfpathmoveto{\pgfqpoint{1.029231in}{0.730021in}}%
\pgfpathlineto{\pgfqpoint{0.961801in}{0.851412in}}%
\pgfpathmoveto{\pgfqpoint{1.029231in}{0.730021in}}%
\pgfpathlineto{\pgfqpoint{1.093189in}{0.806475in}}%
\pgfpathmoveto{\pgfqpoint{1.029231in}{0.730021in}}%
\pgfpathlineto{\pgfqpoint{0.953154in}{0.728383in}}%
\pgfpathmoveto{\pgfqpoint{0.878278in}{0.743820in}}%
\pgfpathlineto{\pgfqpoint{0.821370in}{0.685658in}}%
\pgfpathmoveto{\pgfqpoint{0.878278in}{0.743820in}}%
\pgfpathlineto{\pgfqpoint{0.905212in}{0.662265in}}%
\pgfpathmoveto{\pgfqpoint{0.878278in}{0.743820in}}%
\pgfpathlineto{\pgfqpoint{0.961801in}{0.851412in}}%
\pgfpathmoveto{\pgfqpoint{0.878278in}{0.743820in}}%
\pgfpathlineto{\pgfqpoint{0.822435in}{0.830135in}}%
\pgfpathmoveto{\pgfqpoint{0.878278in}{0.743820in}}%
\pgfpathlineto{\pgfqpoint{0.953154in}{0.728383in}}%
\pgfpathmoveto{\pgfqpoint{1.357947in}{0.924566in}}%
\pgfpathlineto{\pgfqpoint{1.378669in}{0.841156in}}%
\pgfpathmoveto{\pgfqpoint{1.357947in}{0.924566in}}%
\pgfpathlineto{\pgfqpoint{1.433676in}{0.917012in}}%
\pgfpathmoveto{\pgfqpoint{1.357947in}{0.924566in}}%
\pgfpathlineto{\pgfqpoint{1.273497in}{1.004211in}}%
\pgfpathmoveto{\pgfqpoint{1.357947in}{0.924566in}}%
\pgfpathlineto{\pgfqpoint{1.297122in}{0.872566in}}%
\pgfpathmoveto{\pgfqpoint{1.357947in}{0.924566in}}%
\pgfpathlineto{\pgfqpoint{1.375529in}{1.013033in}}%
\pgfpathmoveto{\pgfqpoint{1.015627in}{1.045348in}}%
\pgfpathlineto{\pgfqpoint{0.991250in}{1.270400in}}%
\pgfpathmoveto{\pgfqpoint{1.015627in}{1.045348in}}%
\pgfpathlineto{\pgfqpoint{1.113091in}{0.944697in}}%
\pgfpathmoveto{\pgfqpoint{1.015627in}{1.045348in}}%
\pgfpathlineto{\pgfqpoint{0.837830in}{1.028470in}}%
\pgfpathmoveto{\pgfqpoint{1.015627in}{1.045348in}}%
\pgfpathlineto{\pgfqpoint{0.961801in}{0.851412in}}%
\pgfpathmoveto{\pgfqpoint{1.169547in}{1.110718in}}%
\pgfpathlineto{\pgfqpoint{0.991250in}{1.270400in}}%
\pgfpathmoveto{\pgfqpoint{1.169547in}{1.110718in}}%
\pgfpathlineto{\pgfqpoint{1.261326in}{1.270400in}}%
\pgfpathmoveto{\pgfqpoint{1.169547in}{1.110718in}}%
\pgfpathlineto{\pgfqpoint{1.113091in}{0.944697in}}%
\pgfpathmoveto{\pgfqpoint{1.169547in}{1.110718in}}%
\pgfpathlineto{\pgfqpoint{1.273497in}{1.004211in}}%
\pgfpathmoveto{\pgfqpoint{1.169547in}{1.110718in}}%
\pgfpathlineto{\pgfqpoint{1.362495in}{1.129252in}}%
\pgfpathmoveto{\pgfqpoint{1.169547in}{1.110718in}}%
\pgfpathlineto{\pgfqpoint{1.015627in}{1.045348in}}%
\pgfpathmoveto{\pgfqpoint{0.586918in}{1.007784in}}%
\pgfpathlineto{\pgfqpoint{0.505114in}{1.001892in}}%
\pgfpathmoveto{\pgfqpoint{0.586918in}{1.007784in}}%
\pgfpathlineto{\pgfqpoint{0.548824in}{0.917012in}}%
\pgfpathmoveto{\pgfqpoint{0.586918in}{1.007784in}}%
\pgfpathlineto{\pgfqpoint{0.674832in}{1.049989in}}%
\pgfpathmoveto{\pgfqpoint{0.586918in}{1.007784in}}%
\pgfpathlineto{\pgfqpoint{0.633716in}{0.933208in}}%
\pgfpathmoveto{\pgfqpoint{0.586918in}{1.007784in}}%
\pgfpathlineto{\pgfqpoint{0.572102in}{1.136818in}}%
\pgfpathlineto{\pgfqpoint{0.572102in}{1.136818in}}%
\pgfusepath{stroke}%
\end{pgfscope}%
\begin{pgfscope}%
\pgfpathrectangle{\pgfqpoint{0.100000in}{0.100000in}}{\pgfqpoint{1.782500in}{1.232000in}}%
\pgfusepath{clip}%
\pgfsetbuttcap%
\pgfsetroundjoin%
\pgfsetlinewidth{0.501875pt}%
\definecolor{currentstroke}{rgb}{0.054902,0.262745,0.486275}%
\pgfsetstrokecolor{currentstroke}%
\pgfsetdash{}{0pt}%
\pgfpathmoveto{\pgfqpoint{1.477386in}{0.697627in}}%
\pgfpathcurveto{\pgfqpoint{1.482259in}{0.697627in}}{\pgfqpoint{1.486933in}{0.699563in}}{\pgfqpoint{1.490378in}{0.703008in}}%
\pgfpathcurveto{\pgfqpoint{1.493824in}{0.706454in}}{\pgfqpoint{1.495760in}{0.711127in}}{\pgfqpoint{1.495760in}{0.716000in}}%
\pgfpathcurveto{\pgfqpoint{1.495760in}{0.720873in}}{\pgfqpoint{1.493824in}{0.725546in}}{\pgfqpoint{1.490378in}{0.728992in}}%
\pgfpathcurveto{\pgfqpoint{1.486933in}{0.732437in}}{\pgfqpoint{1.482259in}{0.734373in}}{\pgfqpoint{1.477386in}{0.734373in}}%
\pgfpathcurveto{\pgfqpoint{1.472514in}{0.734373in}}{\pgfqpoint{1.467840in}{0.732437in}}{\pgfqpoint{1.464394in}{0.728992in}}%
\pgfpathcurveto{\pgfqpoint{1.460949in}{0.725546in}}{\pgfqpoint{1.459013in}{0.720873in}}{\pgfqpoint{1.459013in}{0.716000in}}%
\pgfpathcurveto{\pgfqpoint{1.459013in}{0.711127in}}{\pgfqpoint{1.460949in}{0.706454in}}{\pgfqpoint{1.464394in}{0.703008in}}%
\pgfpathcurveto{\pgfqpoint{1.467840in}{0.699563in}}{\pgfqpoint{1.472514in}{0.697627in}}{\pgfqpoint{1.477386in}{0.697627in}}%
\pgfpathlineto{\pgfqpoint{1.477386in}{0.697627in}}%
\pgfpathclose%
\pgfusepath{stroke}%
\end{pgfscope}%
\begin{pgfscope}%
\pgfpathrectangle{\pgfqpoint{0.100000in}{0.100000in}}{\pgfqpoint{1.782500in}{1.232000in}}%
\pgfusepath{clip}%
\pgfsetbuttcap%
\pgfsetroundjoin%
\pgfsetlinewidth{0.501875pt}%
\definecolor{currentstroke}{rgb}{0.054902,0.262745,0.486275}%
\pgfsetstrokecolor{currentstroke}%
\pgfsetdash{}{0pt}%
\pgfpathmoveto{\pgfqpoint{0.505114in}{0.697627in}}%
\pgfpathcurveto{\pgfqpoint{0.509986in}{0.697627in}}{\pgfqpoint{0.514660in}{0.699563in}}{\pgfqpoint{0.518106in}{0.703008in}}%
\pgfpathcurveto{\pgfqpoint{0.521551in}{0.706454in}}{\pgfqpoint{0.523487in}{0.711127in}}{\pgfqpoint{0.523487in}{0.716000in}}%
\pgfpathcurveto{\pgfqpoint{0.523487in}{0.720873in}}{\pgfqpoint{0.521551in}{0.725546in}}{\pgfqpoint{0.518106in}{0.728992in}}%
\pgfpathcurveto{\pgfqpoint{0.514660in}{0.732437in}}{\pgfqpoint{0.509986in}{0.734373in}}{\pgfqpoint{0.505114in}{0.734373in}}%
\pgfpathcurveto{\pgfqpoint{0.500241in}{0.734373in}}{\pgfqpoint{0.495567in}{0.732437in}}{\pgfqpoint{0.492122in}{0.728992in}}%
\pgfpathcurveto{\pgfqpoint{0.488676in}{0.725546in}}{\pgfqpoint{0.486740in}{0.720873in}}{\pgfqpoint{0.486740in}{0.716000in}}%
\pgfpathcurveto{\pgfqpoint{0.486740in}{0.711127in}}{\pgfqpoint{0.488676in}{0.706454in}}{\pgfqpoint{0.492122in}{0.703008in}}%
\pgfpathcurveto{\pgfqpoint{0.495567in}{0.699563in}}{\pgfqpoint{0.500241in}{0.697627in}}{\pgfqpoint{0.505114in}{0.697627in}}%
\pgfpathlineto{\pgfqpoint{0.505114in}{0.697627in}}%
\pgfpathclose%
\pgfusepath{stroke}%
\end{pgfscope}%
\begin{pgfscope}%
\pgfpathrectangle{\pgfqpoint{0.100000in}{0.100000in}}{\pgfqpoint{1.782500in}{1.232000in}}%
\pgfusepath{clip}%
\pgfsetbuttcap%
\pgfsetroundjoin%
\pgfsetlinewidth{0.501875pt}%
\definecolor{currentstroke}{rgb}{0.054902,0.262745,0.486275}%
\pgfsetstrokecolor{currentstroke}%
\pgfsetdash{}{0pt}%
\pgfpathmoveto{\pgfqpoint{1.407938in}{0.697627in}}%
\pgfpathcurveto{\pgfqpoint{1.412811in}{0.697627in}}{\pgfqpoint{1.417485in}{0.699563in}}{\pgfqpoint{1.420930in}{0.703008in}}%
\pgfpathcurveto{\pgfqpoint{1.424376in}{0.706454in}}{\pgfqpoint{1.426312in}{0.711127in}}{\pgfqpoint{1.426312in}{0.716000in}}%
\pgfpathcurveto{\pgfqpoint{1.426312in}{0.720873in}}{\pgfqpoint{1.424376in}{0.725546in}}{\pgfqpoint{1.420930in}{0.728992in}}%
\pgfpathcurveto{\pgfqpoint{1.417485in}{0.732437in}}{\pgfqpoint{1.412811in}{0.734373in}}{\pgfqpoint{1.407938in}{0.734373in}}%
\pgfpathcurveto{\pgfqpoint{1.403066in}{0.734373in}}{\pgfqpoint{1.398392in}{0.732437in}}{\pgfqpoint{1.394946in}{0.728992in}}%
\pgfpathcurveto{\pgfqpoint{1.391501in}{0.725546in}}{\pgfqpoint{1.389565in}{0.720873in}}{\pgfqpoint{1.389565in}{0.716000in}}%
\pgfpathcurveto{\pgfqpoint{1.389565in}{0.711127in}}{\pgfqpoint{1.391501in}{0.706454in}}{\pgfqpoint{1.394946in}{0.703008in}}%
\pgfpathcurveto{\pgfqpoint{1.398392in}{0.699563in}}{\pgfqpoint{1.403066in}{0.697627in}}{\pgfqpoint{1.407938in}{0.697627in}}%
\pgfpathlineto{\pgfqpoint{1.407938in}{0.697627in}}%
\pgfpathclose%
\pgfusepath{stroke}%
\end{pgfscope}%
\begin{pgfscope}%
\pgfpathrectangle{\pgfqpoint{0.100000in}{0.100000in}}{\pgfqpoint{1.782500in}{1.232000in}}%
\pgfusepath{clip}%
\pgfsetbuttcap%
\pgfsetroundjoin%
\pgfsetlinewidth{0.501875pt}%
\definecolor{currentstroke}{rgb}{0.054902,0.262745,0.486275}%
\pgfsetstrokecolor{currentstroke}%
\pgfsetdash{}{0pt}%
\pgfpathmoveto{\pgfqpoint{1.338490in}{0.697627in}}%
\pgfpathcurveto{\pgfqpoint{1.343363in}{0.697627in}}{\pgfqpoint{1.348037in}{0.699563in}}{\pgfqpoint{1.351482in}{0.703008in}}%
\pgfpathcurveto{\pgfqpoint{1.354928in}{0.706454in}}{\pgfqpoint{1.356864in}{0.711127in}}{\pgfqpoint{1.356864in}{0.716000in}}%
\pgfpathcurveto{\pgfqpoint{1.356864in}{0.720873in}}{\pgfqpoint{1.354928in}{0.725546in}}{\pgfqpoint{1.351482in}{0.728992in}}%
\pgfpathcurveto{\pgfqpoint{1.348037in}{0.732437in}}{\pgfqpoint{1.343363in}{0.734373in}}{\pgfqpoint{1.338490in}{0.734373in}}%
\pgfpathcurveto{\pgfqpoint{1.333618in}{0.734373in}}{\pgfqpoint{1.328944in}{0.732437in}}{\pgfqpoint{1.325498in}{0.728992in}}%
\pgfpathcurveto{\pgfqpoint{1.322053in}{0.725546in}}{\pgfqpoint{1.320117in}{0.720873in}}{\pgfqpoint{1.320117in}{0.716000in}}%
\pgfpathcurveto{\pgfqpoint{1.320117in}{0.711127in}}{\pgfqpoint{1.322053in}{0.706454in}}{\pgfqpoint{1.325498in}{0.703008in}}%
\pgfpathcurveto{\pgfqpoint{1.328944in}{0.699563in}}{\pgfqpoint{1.333618in}{0.697627in}}{\pgfqpoint{1.338490in}{0.697627in}}%
\pgfpathlineto{\pgfqpoint{1.338490in}{0.697627in}}%
\pgfpathclose%
\pgfusepath{stroke}%
\end{pgfscope}%
\begin{pgfscope}%
\pgfpathrectangle{\pgfqpoint{0.100000in}{0.100000in}}{\pgfqpoint{1.782500in}{1.232000in}}%
\pgfusepath{clip}%
\pgfsetbuttcap%
\pgfsetroundjoin%
\pgfsetlinewidth{0.501875pt}%
\definecolor{currentstroke}{rgb}{0.054902,0.262745,0.486275}%
\pgfsetstrokecolor{currentstroke}%
\pgfsetdash{}{0pt}%
\pgfpathmoveto{\pgfqpoint{1.269042in}{0.697627in}}%
\pgfpathcurveto{\pgfqpoint{1.273915in}{0.697627in}}{\pgfqpoint{1.278589in}{0.699563in}}{\pgfqpoint{1.282034in}{0.703008in}}%
\pgfpathcurveto{\pgfqpoint{1.285480in}{0.706454in}}{\pgfqpoint{1.287415in}{0.711127in}}{\pgfqpoint{1.287415in}{0.716000in}}%
\pgfpathcurveto{\pgfqpoint{1.287415in}{0.720873in}}{\pgfqpoint{1.285480in}{0.725546in}}{\pgfqpoint{1.282034in}{0.728992in}}%
\pgfpathcurveto{\pgfqpoint{1.278589in}{0.732437in}}{\pgfqpoint{1.273915in}{0.734373in}}{\pgfqpoint{1.269042in}{0.734373in}}%
\pgfpathcurveto{\pgfqpoint{1.264170in}{0.734373in}}{\pgfqpoint{1.259496in}{0.732437in}}{\pgfqpoint{1.256050in}{0.728992in}}%
\pgfpathcurveto{\pgfqpoint{1.252605in}{0.725546in}}{\pgfqpoint{1.250669in}{0.720873in}}{\pgfqpoint{1.250669in}{0.716000in}}%
\pgfpathcurveto{\pgfqpoint{1.250669in}{0.711127in}}{\pgfqpoint{1.252605in}{0.706454in}}{\pgfqpoint{1.256050in}{0.703008in}}%
\pgfpathcurveto{\pgfqpoint{1.259496in}{0.699563in}}{\pgfqpoint{1.264170in}{0.697627in}}{\pgfqpoint{1.269042in}{0.697627in}}%
\pgfpathlineto{\pgfqpoint{1.269042in}{0.697627in}}%
\pgfpathclose%
\pgfusepath{stroke}%
\end{pgfscope}%
\begin{pgfscope}%
\pgfpathrectangle{\pgfqpoint{0.100000in}{0.100000in}}{\pgfqpoint{1.782500in}{1.232000in}}%
\pgfusepath{clip}%
\pgfsetbuttcap%
\pgfsetroundjoin%
\pgfsetlinewidth{0.501875pt}%
\definecolor{currentstroke}{rgb}{0.054902,0.262745,0.486275}%
\pgfsetstrokecolor{currentstroke}%
\pgfsetdash{}{0pt}%
\pgfpathmoveto{\pgfqpoint{1.199594in}{0.697627in}}%
\pgfpathcurveto{\pgfqpoint{1.204467in}{0.697627in}}{\pgfqpoint{1.209141in}{0.699563in}}{\pgfqpoint{1.212586in}{0.703008in}}%
\pgfpathcurveto{\pgfqpoint{1.216032in}{0.706454in}}{\pgfqpoint{1.217967in}{0.711127in}}{\pgfqpoint{1.217967in}{0.716000in}}%
\pgfpathcurveto{\pgfqpoint{1.217967in}{0.720873in}}{\pgfqpoint{1.216032in}{0.725546in}}{\pgfqpoint{1.212586in}{0.728992in}}%
\pgfpathcurveto{\pgfqpoint{1.209141in}{0.732437in}}{\pgfqpoint{1.204467in}{0.734373in}}{\pgfqpoint{1.199594in}{0.734373in}}%
\pgfpathcurveto{\pgfqpoint{1.194722in}{0.734373in}}{\pgfqpoint{1.190048in}{0.732437in}}{\pgfqpoint{1.186602in}{0.728992in}}%
\pgfpathcurveto{\pgfqpoint{1.183157in}{0.725546in}}{\pgfqpoint{1.181221in}{0.720873in}}{\pgfqpoint{1.181221in}{0.716000in}}%
\pgfpathcurveto{\pgfqpoint{1.181221in}{0.711127in}}{\pgfqpoint{1.183157in}{0.706454in}}{\pgfqpoint{1.186602in}{0.703008in}}%
\pgfpathcurveto{\pgfqpoint{1.190048in}{0.699563in}}{\pgfqpoint{1.194722in}{0.697627in}}{\pgfqpoint{1.199594in}{0.697627in}}%
\pgfpathlineto{\pgfqpoint{1.199594in}{0.697627in}}%
\pgfpathclose%
\pgfusepath{stroke}%
\end{pgfscope}%
\begin{pgfscope}%
\pgfpathrectangle{\pgfqpoint{0.100000in}{0.100000in}}{\pgfqpoint{1.782500in}{1.232000in}}%
\pgfusepath{clip}%
\pgfsetbuttcap%
\pgfsetroundjoin%
\pgfsetlinewidth{0.501875pt}%
\definecolor{currentstroke}{rgb}{0.054902,0.262745,0.486275}%
\pgfsetstrokecolor{currentstroke}%
\pgfsetdash{}{0pt}%
\pgfpathmoveto{\pgfqpoint{1.130146in}{0.697627in}}%
\pgfpathcurveto{\pgfqpoint{1.135019in}{0.697627in}}{\pgfqpoint{1.139692in}{0.699563in}}{\pgfqpoint{1.143138in}{0.703008in}}%
\pgfpathcurveto{\pgfqpoint{1.146583in}{0.706454in}}{\pgfqpoint{1.148519in}{0.711127in}}{\pgfqpoint{1.148519in}{0.716000in}}%
\pgfpathcurveto{\pgfqpoint{1.148519in}{0.720873in}}{\pgfqpoint{1.146583in}{0.725546in}}{\pgfqpoint{1.143138in}{0.728992in}}%
\pgfpathcurveto{\pgfqpoint{1.139692in}{0.732437in}}{\pgfqpoint{1.135019in}{0.734373in}}{\pgfqpoint{1.130146in}{0.734373in}}%
\pgfpathcurveto{\pgfqpoint{1.125273in}{0.734373in}}{\pgfqpoint{1.120600in}{0.732437in}}{\pgfqpoint{1.117154in}{0.728992in}}%
\pgfpathcurveto{\pgfqpoint{1.113709in}{0.725546in}}{\pgfqpoint{1.111773in}{0.720873in}}{\pgfqpoint{1.111773in}{0.716000in}}%
\pgfpathcurveto{\pgfqpoint{1.111773in}{0.711127in}}{\pgfqpoint{1.113709in}{0.706454in}}{\pgfqpoint{1.117154in}{0.703008in}}%
\pgfpathcurveto{\pgfqpoint{1.120600in}{0.699563in}}{\pgfqpoint{1.125273in}{0.697627in}}{\pgfqpoint{1.130146in}{0.697627in}}%
\pgfpathlineto{\pgfqpoint{1.130146in}{0.697627in}}%
\pgfpathclose%
\pgfusepath{stroke}%
\end{pgfscope}%
\begin{pgfscope}%
\pgfpathrectangle{\pgfqpoint{0.100000in}{0.100000in}}{\pgfqpoint{1.782500in}{1.232000in}}%
\pgfusepath{clip}%
\pgfsetbuttcap%
\pgfsetroundjoin%
\pgfsetlinewidth{0.501875pt}%
\definecolor{currentstroke}{rgb}{0.054902,0.262745,0.486275}%
\pgfsetstrokecolor{currentstroke}%
\pgfsetdash{}{0pt}%
\pgfpathmoveto{\pgfqpoint{1.060698in}{0.697627in}}%
\pgfpathcurveto{\pgfqpoint{1.065571in}{0.697627in}}{\pgfqpoint{1.070244in}{0.699563in}}{\pgfqpoint{1.073690in}{0.703008in}}%
\pgfpathcurveto{\pgfqpoint{1.077135in}{0.706454in}}{\pgfqpoint{1.079071in}{0.711127in}}{\pgfqpoint{1.079071in}{0.716000in}}%
\pgfpathcurveto{\pgfqpoint{1.079071in}{0.720873in}}{\pgfqpoint{1.077135in}{0.725546in}}{\pgfqpoint{1.073690in}{0.728992in}}%
\pgfpathcurveto{\pgfqpoint{1.070244in}{0.732437in}}{\pgfqpoint{1.065571in}{0.734373in}}{\pgfqpoint{1.060698in}{0.734373in}}%
\pgfpathcurveto{\pgfqpoint{1.055825in}{0.734373in}}{\pgfqpoint{1.051152in}{0.732437in}}{\pgfqpoint{1.047706in}{0.728992in}}%
\pgfpathcurveto{\pgfqpoint{1.044261in}{0.725546in}}{\pgfqpoint{1.042325in}{0.720873in}}{\pgfqpoint{1.042325in}{0.716000in}}%
\pgfpathcurveto{\pgfqpoint{1.042325in}{0.711127in}}{\pgfqpoint{1.044261in}{0.706454in}}{\pgfqpoint{1.047706in}{0.703008in}}%
\pgfpathcurveto{\pgfqpoint{1.051152in}{0.699563in}}{\pgfqpoint{1.055825in}{0.697627in}}{\pgfqpoint{1.060698in}{0.697627in}}%
\pgfpathlineto{\pgfqpoint{1.060698in}{0.697627in}}%
\pgfpathclose%
\pgfusepath{stroke}%
\end{pgfscope}%
\begin{pgfscope}%
\pgfpathrectangle{\pgfqpoint{0.100000in}{0.100000in}}{\pgfqpoint{1.782500in}{1.232000in}}%
\pgfusepath{clip}%
\pgfsetbuttcap%
\pgfsetroundjoin%
\pgfsetlinewidth{0.501875pt}%
\definecolor{currentstroke}{rgb}{0.054902,0.262745,0.486275}%
\pgfsetstrokecolor{currentstroke}%
\pgfsetdash{}{0pt}%
\pgfpathmoveto{\pgfqpoint{0.991250in}{0.697627in}}%
\pgfpathcurveto{\pgfqpoint{0.996123in}{0.697627in}}{\pgfqpoint{1.000796in}{0.699563in}}{\pgfqpoint{1.004242in}{0.703008in}}%
\pgfpathcurveto{\pgfqpoint{1.007687in}{0.706454in}}{\pgfqpoint{1.009623in}{0.711127in}}{\pgfqpoint{1.009623in}{0.716000in}}%
\pgfpathcurveto{\pgfqpoint{1.009623in}{0.720873in}}{\pgfqpoint{1.007687in}{0.725546in}}{\pgfqpoint{1.004242in}{0.728992in}}%
\pgfpathcurveto{\pgfqpoint{1.000796in}{0.732437in}}{\pgfqpoint{0.996123in}{0.734373in}}{\pgfqpoint{0.991250in}{0.734373in}}%
\pgfpathcurveto{\pgfqpoint{0.986377in}{0.734373in}}{\pgfqpoint{0.981704in}{0.732437in}}{\pgfqpoint{0.978258in}{0.728992in}}%
\pgfpathcurveto{\pgfqpoint{0.974813in}{0.725546in}}{\pgfqpoint{0.972877in}{0.720873in}}{\pgfqpoint{0.972877in}{0.716000in}}%
\pgfpathcurveto{\pgfqpoint{0.972877in}{0.711127in}}{\pgfqpoint{0.974813in}{0.706454in}}{\pgfqpoint{0.978258in}{0.703008in}}%
\pgfpathcurveto{\pgfqpoint{0.981704in}{0.699563in}}{\pgfqpoint{0.986377in}{0.697627in}}{\pgfqpoint{0.991250in}{0.697627in}}%
\pgfpathlineto{\pgfqpoint{0.991250in}{0.697627in}}%
\pgfpathclose%
\pgfusepath{stroke}%
\end{pgfscope}%
\begin{pgfscope}%
\pgfpathrectangle{\pgfqpoint{0.100000in}{0.100000in}}{\pgfqpoint{1.782500in}{1.232000in}}%
\pgfusepath{clip}%
\pgfsetbuttcap%
\pgfsetroundjoin%
\pgfsetlinewidth{0.501875pt}%
\definecolor{currentstroke}{rgb}{0.054902,0.262745,0.486275}%
\pgfsetstrokecolor{currentstroke}%
\pgfsetdash{}{0pt}%
\pgfpathmoveto{\pgfqpoint{0.921802in}{0.697627in}}%
\pgfpathcurveto{\pgfqpoint{0.926675in}{0.697627in}}{\pgfqpoint{0.931348in}{0.699563in}}{\pgfqpoint{0.934794in}{0.703008in}}%
\pgfpathcurveto{\pgfqpoint{0.938239in}{0.706454in}}{\pgfqpoint{0.940175in}{0.711127in}}{\pgfqpoint{0.940175in}{0.716000in}}%
\pgfpathcurveto{\pgfqpoint{0.940175in}{0.720873in}}{\pgfqpoint{0.938239in}{0.725546in}}{\pgfqpoint{0.934794in}{0.728992in}}%
\pgfpathcurveto{\pgfqpoint{0.931348in}{0.732437in}}{\pgfqpoint{0.926675in}{0.734373in}}{\pgfqpoint{0.921802in}{0.734373in}}%
\pgfpathcurveto{\pgfqpoint{0.916929in}{0.734373in}}{\pgfqpoint{0.912256in}{0.732437in}}{\pgfqpoint{0.908810in}{0.728992in}}%
\pgfpathcurveto{\pgfqpoint{0.905365in}{0.725546in}}{\pgfqpoint{0.903429in}{0.720873in}}{\pgfqpoint{0.903429in}{0.716000in}}%
\pgfpathcurveto{\pgfqpoint{0.903429in}{0.711127in}}{\pgfqpoint{0.905365in}{0.706454in}}{\pgfqpoint{0.908810in}{0.703008in}}%
\pgfpathcurveto{\pgfqpoint{0.912256in}{0.699563in}}{\pgfqpoint{0.916929in}{0.697627in}}{\pgfqpoint{0.921802in}{0.697627in}}%
\pgfpathlineto{\pgfqpoint{0.921802in}{0.697627in}}%
\pgfpathclose%
\pgfusepath{stroke}%
\end{pgfscope}%
\begin{pgfscope}%
\pgfpathrectangle{\pgfqpoint{0.100000in}{0.100000in}}{\pgfqpoint{1.782500in}{1.232000in}}%
\pgfusepath{clip}%
\pgfsetbuttcap%
\pgfsetroundjoin%
\pgfsetlinewidth{0.501875pt}%
\definecolor{currentstroke}{rgb}{0.054902,0.262745,0.486275}%
\pgfsetstrokecolor{currentstroke}%
\pgfsetdash{}{0pt}%
\pgfpathmoveto{\pgfqpoint{0.852354in}{0.697627in}}%
\pgfpathcurveto{\pgfqpoint{0.857227in}{0.697627in}}{\pgfqpoint{0.861900in}{0.699563in}}{\pgfqpoint{0.865346in}{0.703008in}}%
\pgfpathcurveto{\pgfqpoint{0.868791in}{0.706454in}}{\pgfqpoint{0.870727in}{0.711127in}}{\pgfqpoint{0.870727in}{0.716000in}}%
\pgfpathcurveto{\pgfqpoint{0.870727in}{0.720873in}}{\pgfqpoint{0.868791in}{0.725546in}}{\pgfqpoint{0.865346in}{0.728992in}}%
\pgfpathcurveto{\pgfqpoint{0.861900in}{0.732437in}}{\pgfqpoint{0.857227in}{0.734373in}}{\pgfqpoint{0.852354in}{0.734373in}}%
\pgfpathcurveto{\pgfqpoint{0.847481in}{0.734373in}}{\pgfqpoint{0.842808in}{0.732437in}}{\pgfqpoint{0.839362in}{0.728992in}}%
\pgfpathcurveto{\pgfqpoint{0.835917in}{0.725546in}}{\pgfqpoint{0.833981in}{0.720873in}}{\pgfqpoint{0.833981in}{0.716000in}}%
\pgfpathcurveto{\pgfqpoint{0.833981in}{0.711127in}}{\pgfqpoint{0.835917in}{0.706454in}}{\pgfqpoint{0.839362in}{0.703008in}}%
\pgfpathcurveto{\pgfqpoint{0.842808in}{0.699563in}}{\pgfqpoint{0.847481in}{0.697627in}}{\pgfqpoint{0.852354in}{0.697627in}}%
\pgfpathlineto{\pgfqpoint{0.852354in}{0.697627in}}%
\pgfpathclose%
\pgfusepath{stroke}%
\end{pgfscope}%
\begin{pgfscope}%
\pgfpathrectangle{\pgfqpoint{0.100000in}{0.100000in}}{\pgfqpoint{1.782500in}{1.232000in}}%
\pgfusepath{clip}%
\pgfsetbuttcap%
\pgfsetroundjoin%
\pgfsetlinewidth{0.501875pt}%
\definecolor{currentstroke}{rgb}{0.054902,0.262745,0.486275}%
\pgfsetstrokecolor{currentstroke}%
\pgfsetdash{}{0pt}%
\pgfpathmoveto{\pgfqpoint{0.782906in}{0.697627in}}%
\pgfpathcurveto{\pgfqpoint{0.787778in}{0.697627in}}{\pgfqpoint{0.792452in}{0.699563in}}{\pgfqpoint{0.795898in}{0.703008in}}%
\pgfpathcurveto{\pgfqpoint{0.799343in}{0.706454in}}{\pgfqpoint{0.801279in}{0.711127in}}{\pgfqpoint{0.801279in}{0.716000in}}%
\pgfpathcurveto{\pgfqpoint{0.801279in}{0.720873in}}{\pgfqpoint{0.799343in}{0.725546in}}{\pgfqpoint{0.795898in}{0.728992in}}%
\pgfpathcurveto{\pgfqpoint{0.792452in}{0.732437in}}{\pgfqpoint{0.787778in}{0.734373in}}{\pgfqpoint{0.782906in}{0.734373in}}%
\pgfpathcurveto{\pgfqpoint{0.778033in}{0.734373in}}{\pgfqpoint{0.773359in}{0.732437in}}{\pgfqpoint{0.769914in}{0.728992in}}%
\pgfpathcurveto{\pgfqpoint{0.766468in}{0.725546in}}{\pgfqpoint{0.764533in}{0.720873in}}{\pgfqpoint{0.764533in}{0.716000in}}%
\pgfpathcurveto{\pgfqpoint{0.764533in}{0.711127in}}{\pgfqpoint{0.766468in}{0.706454in}}{\pgfqpoint{0.769914in}{0.703008in}}%
\pgfpathcurveto{\pgfqpoint{0.773359in}{0.699563in}}{\pgfqpoint{0.778033in}{0.697627in}}{\pgfqpoint{0.782906in}{0.697627in}}%
\pgfpathlineto{\pgfqpoint{0.782906in}{0.697627in}}%
\pgfpathclose%
\pgfusepath{stroke}%
\end{pgfscope}%
\begin{pgfscope}%
\pgfpathrectangle{\pgfqpoint{0.100000in}{0.100000in}}{\pgfqpoint{1.782500in}{1.232000in}}%
\pgfusepath{clip}%
\pgfsetbuttcap%
\pgfsetroundjoin%
\pgfsetlinewidth{0.501875pt}%
\definecolor{currentstroke}{rgb}{0.054902,0.262745,0.486275}%
\pgfsetstrokecolor{currentstroke}%
\pgfsetdash{}{0pt}%
\pgfpathmoveto{\pgfqpoint{0.713458in}{0.697627in}}%
\pgfpathcurveto{\pgfqpoint{0.718330in}{0.697627in}}{\pgfqpoint{0.723004in}{0.699563in}}{\pgfqpoint{0.726450in}{0.703008in}}%
\pgfpathcurveto{\pgfqpoint{0.729895in}{0.706454in}}{\pgfqpoint{0.731831in}{0.711127in}}{\pgfqpoint{0.731831in}{0.716000in}}%
\pgfpathcurveto{\pgfqpoint{0.731831in}{0.720873in}}{\pgfqpoint{0.729895in}{0.725546in}}{\pgfqpoint{0.726450in}{0.728992in}}%
\pgfpathcurveto{\pgfqpoint{0.723004in}{0.732437in}}{\pgfqpoint{0.718330in}{0.734373in}}{\pgfqpoint{0.713458in}{0.734373in}}%
\pgfpathcurveto{\pgfqpoint{0.708585in}{0.734373in}}{\pgfqpoint{0.703911in}{0.732437in}}{\pgfqpoint{0.700466in}{0.728992in}}%
\pgfpathcurveto{\pgfqpoint{0.697020in}{0.725546in}}{\pgfqpoint{0.695085in}{0.720873in}}{\pgfqpoint{0.695085in}{0.716000in}}%
\pgfpathcurveto{\pgfqpoint{0.695085in}{0.711127in}}{\pgfqpoint{0.697020in}{0.706454in}}{\pgfqpoint{0.700466in}{0.703008in}}%
\pgfpathcurveto{\pgfqpoint{0.703911in}{0.699563in}}{\pgfqpoint{0.708585in}{0.697627in}}{\pgfqpoint{0.713458in}{0.697627in}}%
\pgfpathlineto{\pgfqpoint{0.713458in}{0.697627in}}%
\pgfpathclose%
\pgfusepath{stroke}%
\end{pgfscope}%
\begin{pgfscope}%
\pgfpathrectangle{\pgfqpoint{0.100000in}{0.100000in}}{\pgfqpoint{1.782500in}{1.232000in}}%
\pgfusepath{clip}%
\pgfsetbuttcap%
\pgfsetroundjoin%
\pgfsetlinewidth{0.501875pt}%
\definecolor{currentstroke}{rgb}{0.054902,0.262745,0.486275}%
\pgfsetstrokecolor{currentstroke}%
\pgfsetdash{}{0pt}%
\pgfpathmoveto{\pgfqpoint{0.644010in}{0.697627in}}%
\pgfpathcurveto{\pgfqpoint{0.648882in}{0.697627in}}{\pgfqpoint{0.653556in}{0.699563in}}{\pgfqpoint{0.657002in}{0.703008in}}%
\pgfpathcurveto{\pgfqpoint{0.660447in}{0.706454in}}{\pgfqpoint{0.662383in}{0.711127in}}{\pgfqpoint{0.662383in}{0.716000in}}%
\pgfpathcurveto{\pgfqpoint{0.662383in}{0.720873in}}{\pgfqpoint{0.660447in}{0.725546in}}{\pgfqpoint{0.657002in}{0.728992in}}%
\pgfpathcurveto{\pgfqpoint{0.653556in}{0.732437in}}{\pgfqpoint{0.648882in}{0.734373in}}{\pgfqpoint{0.644010in}{0.734373in}}%
\pgfpathcurveto{\pgfqpoint{0.639137in}{0.734373in}}{\pgfqpoint{0.634463in}{0.732437in}}{\pgfqpoint{0.631018in}{0.728992in}}%
\pgfpathcurveto{\pgfqpoint{0.627572in}{0.725546in}}{\pgfqpoint{0.625636in}{0.720873in}}{\pgfqpoint{0.625636in}{0.716000in}}%
\pgfpathcurveto{\pgfqpoint{0.625636in}{0.711127in}}{\pgfqpoint{0.627572in}{0.706454in}}{\pgfqpoint{0.631018in}{0.703008in}}%
\pgfpathcurveto{\pgfqpoint{0.634463in}{0.699563in}}{\pgfqpoint{0.639137in}{0.697627in}}{\pgfqpoint{0.644010in}{0.697627in}}%
\pgfpathlineto{\pgfqpoint{0.644010in}{0.697627in}}%
\pgfpathclose%
\pgfusepath{stroke}%
\end{pgfscope}%
\begin{pgfscope}%
\pgfpathrectangle{\pgfqpoint{0.100000in}{0.100000in}}{\pgfqpoint{1.782500in}{1.232000in}}%
\pgfusepath{clip}%
\pgfsetbuttcap%
\pgfsetroundjoin%
\pgfsetlinewidth{0.501875pt}%
\definecolor{currentstroke}{rgb}{0.054902,0.262745,0.486275}%
\pgfsetstrokecolor{currentstroke}%
\pgfsetdash{}{0pt}%
\pgfpathmoveto{\pgfqpoint{0.574562in}{0.697627in}}%
\pgfpathcurveto{\pgfqpoint{0.579434in}{0.697627in}}{\pgfqpoint{0.584108in}{0.699563in}}{\pgfqpoint{0.587554in}{0.703008in}}%
\pgfpathcurveto{\pgfqpoint{0.590999in}{0.706454in}}{\pgfqpoint{0.592935in}{0.711127in}}{\pgfqpoint{0.592935in}{0.716000in}}%
\pgfpathcurveto{\pgfqpoint{0.592935in}{0.720873in}}{\pgfqpoint{0.590999in}{0.725546in}}{\pgfqpoint{0.587554in}{0.728992in}}%
\pgfpathcurveto{\pgfqpoint{0.584108in}{0.732437in}}{\pgfqpoint{0.579434in}{0.734373in}}{\pgfqpoint{0.574562in}{0.734373in}}%
\pgfpathcurveto{\pgfqpoint{0.569689in}{0.734373in}}{\pgfqpoint{0.565015in}{0.732437in}}{\pgfqpoint{0.561570in}{0.728992in}}%
\pgfpathcurveto{\pgfqpoint{0.558124in}{0.725546in}}{\pgfqpoint{0.556188in}{0.720873in}}{\pgfqpoint{0.556188in}{0.716000in}}%
\pgfpathcurveto{\pgfqpoint{0.556188in}{0.711127in}}{\pgfqpoint{0.558124in}{0.706454in}}{\pgfqpoint{0.561570in}{0.703008in}}%
\pgfpathcurveto{\pgfqpoint{0.565015in}{0.699563in}}{\pgfqpoint{0.569689in}{0.697627in}}{\pgfqpoint{0.574562in}{0.697627in}}%
\pgfpathlineto{\pgfqpoint{0.574562in}{0.697627in}}%
\pgfpathclose%
\pgfusepath{stroke}%
\end{pgfscope}%
\begin{pgfscope}%
\pgfpathrectangle{\pgfqpoint{0.100000in}{0.100000in}}{\pgfqpoint{1.782500in}{1.232000in}}%
\pgfusepath{clip}%
\pgfsetbuttcap%
\pgfsetroundjoin%
\pgfsetlinewidth{0.501875pt}%
\definecolor{currentstroke}{rgb}{0.835294,0.321569,0.035294}%
\pgfsetstrokecolor{currentstroke}%
\pgfsetdash{}{0pt}%
\pgfpathmoveto{\pgfqpoint{0.505114in}{0.983519in}}%
\pgfpathcurveto{\pgfqpoint{0.509986in}{0.983519in}}{\pgfqpoint{0.514660in}{0.985454in}}{\pgfqpoint{0.518106in}{0.988900in}}%
\pgfpathcurveto{\pgfqpoint{0.521551in}{0.992345in}}{\pgfqpoint{0.523487in}{0.997019in}}{\pgfqpoint{0.523487in}{1.001892in}}%
\pgfpathcurveto{\pgfqpoint{0.523487in}{1.006764in}}{\pgfqpoint{0.521551in}{1.011438in}}{\pgfqpoint{0.518106in}{1.014884in}}%
\pgfpathcurveto{\pgfqpoint{0.514660in}{1.018329in}}{\pgfqpoint{0.509986in}{1.020265in}}{\pgfqpoint{0.505114in}{1.020265in}}%
\pgfpathcurveto{\pgfqpoint{0.500241in}{1.020265in}}{\pgfqpoint{0.495567in}{1.018329in}}{\pgfqpoint{0.492122in}{1.014884in}}%
\pgfpathcurveto{\pgfqpoint{0.488676in}{1.011438in}}{\pgfqpoint{0.486740in}{1.006764in}}{\pgfqpoint{0.486740in}{1.001892in}}%
\pgfpathcurveto{\pgfqpoint{0.486740in}{0.997019in}}{\pgfqpoint{0.488676in}{0.992345in}}{\pgfqpoint{0.492122in}{0.988900in}}%
\pgfpathcurveto{\pgfqpoint{0.495567in}{0.985454in}}{\pgfqpoint{0.500241in}{0.983519in}}{\pgfqpoint{0.505114in}{0.983519in}}%
\pgfpathlineto{\pgfqpoint{0.505114in}{0.983519in}}%
\pgfpathclose%
\pgfusepath{stroke}%
\end{pgfscope}%
\begin{pgfscope}%
\pgfpathrectangle{\pgfqpoint{0.100000in}{0.100000in}}{\pgfqpoint{1.782500in}{1.232000in}}%
\pgfusepath{clip}%
\pgfsetbuttcap%
\pgfsetroundjoin%
\pgfsetlinewidth{0.501875pt}%
\definecolor{currentstroke}{rgb}{0.835294,0.321569,0.035294}%
\pgfsetstrokecolor{currentstroke}%
\pgfsetdash{}{0pt}%
\pgfpathmoveto{\pgfqpoint{1.477386in}{0.983519in}}%
\pgfpathcurveto{\pgfqpoint{1.482259in}{0.983519in}}{\pgfqpoint{1.486933in}{0.985454in}}{\pgfqpoint{1.490378in}{0.988900in}}%
\pgfpathcurveto{\pgfqpoint{1.493824in}{0.992345in}}{\pgfqpoint{1.495760in}{0.997019in}}{\pgfqpoint{1.495760in}{1.001892in}}%
\pgfpathcurveto{\pgfqpoint{1.495760in}{1.006764in}}{\pgfqpoint{1.493824in}{1.011438in}}{\pgfqpoint{1.490378in}{1.014884in}}%
\pgfpathcurveto{\pgfqpoint{1.486933in}{1.018329in}}{\pgfqpoint{1.482259in}{1.020265in}}{\pgfqpoint{1.477386in}{1.020265in}}%
\pgfpathcurveto{\pgfqpoint{1.472514in}{1.020265in}}{\pgfqpoint{1.467840in}{1.018329in}}{\pgfqpoint{1.464394in}{1.014884in}}%
\pgfpathcurveto{\pgfqpoint{1.460949in}{1.011438in}}{\pgfqpoint{1.459013in}{1.006764in}}{\pgfqpoint{1.459013in}{1.001892in}}%
\pgfpathcurveto{\pgfqpoint{1.459013in}{0.997019in}}{\pgfqpoint{1.460949in}{0.992345in}}{\pgfqpoint{1.464394in}{0.988900in}}%
\pgfpathcurveto{\pgfqpoint{1.467840in}{0.985454in}}{\pgfqpoint{1.472514in}{0.983519in}}{\pgfqpoint{1.477386in}{0.983519in}}%
\pgfpathlineto{\pgfqpoint{1.477386in}{0.983519in}}%
\pgfpathclose%
\pgfusepath{stroke}%
\end{pgfscope}%
\begin{pgfscope}%
\pgfpathrectangle{\pgfqpoint{0.100000in}{0.100000in}}{\pgfqpoint{1.782500in}{1.232000in}}%
\pgfusepath{clip}%
\pgfsetbuttcap%
\pgfsetroundjoin%
\pgfsetlinewidth{0.501875pt}%
\definecolor{currentstroke}{rgb}{0.835294,0.321569,0.035294}%
\pgfsetstrokecolor{currentstroke}%
\pgfsetdash{}{0pt}%
\pgfpathmoveto{\pgfqpoint{0.548824in}{0.898639in}}%
\pgfpathcurveto{\pgfqpoint{0.553696in}{0.898639in}}{\pgfqpoint{0.558370in}{0.900575in}}{\pgfqpoint{0.561816in}{0.904020in}}%
\pgfpathcurveto{\pgfqpoint{0.565261in}{0.907466in}}{\pgfqpoint{0.567197in}{0.912139in}}{\pgfqpoint{0.567197in}{0.917012in}}%
\pgfpathcurveto{\pgfqpoint{0.567197in}{0.921885in}}{\pgfqpoint{0.565261in}{0.926558in}}{\pgfqpoint{0.561816in}{0.930004in}}%
\pgfpathcurveto{\pgfqpoint{0.558370in}{0.933449in}}{\pgfqpoint{0.553696in}{0.935385in}}{\pgfqpoint{0.548824in}{0.935385in}}%
\pgfpathcurveto{\pgfqpoint{0.543951in}{0.935385in}}{\pgfqpoint{0.539277in}{0.933449in}}{\pgfqpoint{0.535832in}{0.930004in}}%
\pgfpathcurveto{\pgfqpoint{0.532386in}{0.926558in}}{\pgfqpoint{0.530450in}{0.921885in}}{\pgfqpoint{0.530450in}{0.917012in}}%
\pgfpathcurveto{\pgfqpoint{0.530450in}{0.912139in}}{\pgfqpoint{0.532386in}{0.907466in}}{\pgfqpoint{0.535832in}{0.904020in}}%
\pgfpathcurveto{\pgfqpoint{0.539277in}{0.900575in}}{\pgfqpoint{0.543951in}{0.898639in}}{\pgfqpoint{0.548824in}{0.898639in}}%
\pgfpathlineto{\pgfqpoint{0.548824in}{0.898639in}}%
\pgfpathclose%
\pgfusepath{stroke}%
\end{pgfscope}%
\begin{pgfscope}%
\pgfpathrectangle{\pgfqpoint{0.100000in}{0.100000in}}{\pgfqpoint{1.782500in}{1.232000in}}%
\pgfusepath{clip}%
\pgfsetbuttcap%
\pgfsetroundjoin%
\pgfsetlinewidth{0.501875pt}%
\definecolor{currentstroke}{rgb}{0.835294,0.321569,0.035294}%
\pgfsetstrokecolor{currentstroke}%
\pgfsetdash{}{0pt}%
\pgfpathmoveto{\pgfqpoint{0.603831in}{0.822783in}}%
\pgfpathcurveto{\pgfqpoint{0.608704in}{0.822783in}}{\pgfqpoint{0.613377in}{0.824719in}}{\pgfqpoint{0.616823in}{0.828164in}}%
\pgfpathcurveto{\pgfqpoint{0.620268in}{0.831610in}}{\pgfqpoint{0.622204in}{0.836283in}}{\pgfqpoint{0.622204in}{0.841156in}}%
\pgfpathcurveto{\pgfqpoint{0.622204in}{0.846029in}}{\pgfqpoint{0.620268in}{0.850702in}}{\pgfqpoint{0.616823in}{0.854148in}}%
\pgfpathcurveto{\pgfqpoint{0.613377in}{0.857593in}}{\pgfqpoint{0.608704in}{0.859529in}}{\pgfqpoint{0.603831in}{0.859529in}}%
\pgfpathcurveto{\pgfqpoint{0.598958in}{0.859529in}}{\pgfqpoint{0.594285in}{0.857593in}}{\pgfqpoint{0.590839in}{0.854148in}}%
\pgfpathcurveto{\pgfqpoint{0.587394in}{0.850702in}}{\pgfqpoint{0.585458in}{0.846029in}}{\pgfqpoint{0.585458in}{0.841156in}}%
\pgfpathcurveto{\pgfqpoint{0.585458in}{0.836283in}}{\pgfqpoint{0.587394in}{0.831610in}}{\pgfqpoint{0.590839in}{0.828164in}}%
\pgfpathcurveto{\pgfqpoint{0.594285in}{0.824719in}}{\pgfqpoint{0.598958in}{0.822783in}}{\pgfqpoint{0.603831in}{0.822783in}}%
\pgfpathlineto{\pgfqpoint{0.603831in}{0.822783in}}%
\pgfpathclose%
\pgfusepath{stroke}%
\end{pgfscope}%
\begin{pgfscope}%
\pgfpathrectangle{\pgfqpoint{0.100000in}{0.100000in}}{\pgfqpoint{1.782500in}{1.232000in}}%
\pgfusepath{clip}%
\pgfsetbuttcap%
\pgfsetroundjoin%
\pgfsetlinewidth{0.501875pt}%
\definecolor{currentstroke}{rgb}{0.835294,0.321569,0.035294}%
\pgfsetstrokecolor{currentstroke}%
\pgfsetdash{}{0pt}%
\pgfpathmoveto{\pgfqpoint{0.668731in}{0.757887in}}%
\pgfpathcurveto{\pgfqpoint{0.673604in}{0.757887in}}{\pgfqpoint{0.678277in}{0.759823in}}{\pgfqpoint{0.681723in}{0.763269in}}%
\pgfpathcurveto{\pgfqpoint{0.685168in}{0.766714in}}{\pgfqpoint{0.687104in}{0.771388in}}{\pgfqpoint{0.687104in}{0.776261in}}%
\pgfpathcurveto{\pgfqpoint{0.687104in}{0.781133in}}{\pgfqpoint{0.685168in}{0.785807in}}{\pgfqpoint{0.681723in}{0.789252in}}%
\pgfpathcurveto{\pgfqpoint{0.678277in}{0.792698in}}{\pgfqpoint{0.673604in}{0.794634in}}{\pgfqpoint{0.668731in}{0.794634in}}%
\pgfpathcurveto{\pgfqpoint{0.663858in}{0.794634in}}{\pgfqpoint{0.659185in}{0.792698in}}{\pgfqpoint{0.655739in}{0.789252in}}%
\pgfpathcurveto{\pgfqpoint{0.652294in}{0.785807in}}{\pgfqpoint{0.650358in}{0.781133in}}{\pgfqpoint{0.650358in}{0.776261in}}%
\pgfpathcurveto{\pgfqpoint{0.650358in}{0.771388in}}{\pgfqpoint{0.652294in}{0.766714in}}{\pgfqpoint{0.655739in}{0.763269in}}%
\pgfpathcurveto{\pgfqpoint{0.659185in}{0.759823in}}{\pgfqpoint{0.663858in}{0.757887in}}{\pgfqpoint{0.668731in}{0.757887in}}%
\pgfpathlineto{\pgfqpoint{0.668731in}{0.757887in}}%
\pgfpathclose%
\pgfusepath{stroke}%
\end{pgfscope}%
\begin{pgfscope}%
\pgfpathrectangle{\pgfqpoint{0.100000in}{0.100000in}}{\pgfqpoint{1.782500in}{1.232000in}}%
\pgfusepath{clip}%
\pgfsetbuttcap%
\pgfsetroundjoin%
\pgfsetlinewidth{0.501875pt}%
\definecolor{currentstroke}{rgb}{0.835294,0.321569,0.035294}%
\pgfsetstrokecolor{currentstroke}%
\pgfsetdash{}{0pt}%
\pgfpathmoveto{\pgfqpoint{0.741867in}{0.705610in}}%
\pgfpathcurveto{\pgfqpoint{0.746739in}{0.705610in}}{\pgfqpoint{0.751413in}{0.707546in}}{\pgfqpoint{0.754858in}{0.710991in}}%
\pgfpathcurveto{\pgfqpoint{0.758304in}{0.714437in}}{\pgfqpoint{0.760240in}{0.719110in}}{\pgfqpoint{0.760240in}{0.723983in}}%
\pgfpathcurveto{\pgfqpoint{0.760240in}{0.728856in}}{\pgfqpoint{0.758304in}{0.733529in}}{\pgfqpoint{0.754858in}{0.736975in}}%
\pgfpathcurveto{\pgfqpoint{0.751413in}{0.740420in}}{\pgfqpoint{0.746739in}{0.742356in}}{\pgfqpoint{0.741867in}{0.742356in}}%
\pgfpathcurveto{\pgfqpoint{0.736994in}{0.742356in}}{\pgfqpoint{0.732320in}{0.740420in}}{\pgfqpoint{0.728875in}{0.736975in}}%
\pgfpathcurveto{\pgfqpoint{0.725429in}{0.733529in}}{\pgfqpoint{0.723493in}{0.728856in}}{\pgfqpoint{0.723493in}{0.723983in}}%
\pgfpathcurveto{\pgfqpoint{0.723493in}{0.719110in}}{\pgfqpoint{0.725429in}{0.714437in}}{\pgfqpoint{0.728875in}{0.710991in}}%
\pgfpathcurveto{\pgfqpoint{0.732320in}{0.707546in}}{\pgfqpoint{0.736994in}{0.705610in}}{\pgfqpoint{0.741867in}{0.705610in}}%
\pgfpathlineto{\pgfqpoint{0.741867in}{0.705610in}}%
\pgfpathclose%
\pgfusepath{stroke}%
\end{pgfscope}%
\begin{pgfscope}%
\pgfpathrectangle{\pgfqpoint{0.100000in}{0.100000in}}{\pgfqpoint{1.782500in}{1.232000in}}%
\pgfusepath{clip}%
\pgfsetbuttcap%
\pgfsetroundjoin%
\pgfsetlinewidth{0.501875pt}%
\definecolor{currentstroke}{rgb}{0.835294,0.321569,0.035294}%
\pgfsetstrokecolor{currentstroke}%
\pgfsetdash{}{0pt}%
\pgfpathmoveto{\pgfqpoint{0.821370in}{0.667285in}}%
\pgfpathcurveto{\pgfqpoint{0.826243in}{0.667285in}}{\pgfqpoint{0.830917in}{0.669221in}}{\pgfqpoint{0.834362in}{0.672666in}}%
\pgfpathcurveto{\pgfqpoint{0.837808in}{0.676112in}}{\pgfqpoint{0.839743in}{0.680786in}}{\pgfqpoint{0.839743in}{0.685658in}}%
\pgfpathcurveto{\pgfqpoint{0.839743in}{0.690531in}}{\pgfqpoint{0.837808in}{0.695205in}}{\pgfqpoint{0.834362in}{0.698650in}}%
\pgfpathcurveto{\pgfqpoint{0.830917in}{0.702096in}}{\pgfqpoint{0.826243in}{0.704031in}}{\pgfqpoint{0.821370in}{0.704031in}}%
\pgfpathcurveto{\pgfqpoint{0.816498in}{0.704031in}}{\pgfqpoint{0.811824in}{0.702096in}}{\pgfqpoint{0.808378in}{0.698650in}}%
\pgfpathcurveto{\pgfqpoint{0.804933in}{0.695205in}}{\pgfqpoint{0.802997in}{0.690531in}}{\pgfqpoint{0.802997in}{0.685658in}}%
\pgfpathcurveto{\pgfqpoint{0.802997in}{0.680786in}}{\pgfqpoint{0.804933in}{0.676112in}}{\pgfqpoint{0.808378in}{0.672666in}}%
\pgfpathcurveto{\pgfqpoint{0.811824in}{0.669221in}}{\pgfqpoint{0.816498in}{0.667285in}}{\pgfqpoint{0.821370in}{0.667285in}}%
\pgfpathlineto{\pgfqpoint{0.821370in}{0.667285in}}%
\pgfpathclose%
\pgfusepath{stroke}%
\end{pgfscope}%
\begin{pgfscope}%
\pgfpathrectangle{\pgfqpoint{0.100000in}{0.100000in}}{\pgfqpoint{1.782500in}{1.232000in}}%
\pgfusepath{clip}%
\pgfsetbuttcap%
\pgfsetroundjoin%
\pgfsetlinewidth{0.501875pt}%
\definecolor{currentstroke}{rgb}{0.835294,0.321569,0.035294}%
\pgfsetstrokecolor{currentstroke}%
\pgfsetdash{}{0pt}%
\pgfpathmoveto{\pgfqpoint{0.905212in}{0.643891in}}%
\pgfpathcurveto{\pgfqpoint{0.910084in}{0.643891in}}{\pgfqpoint{0.914758in}{0.645827in}}{\pgfqpoint{0.918203in}{0.649273in}}%
\pgfpathcurveto{\pgfqpoint{0.921649in}{0.652718in}}{\pgfqpoint{0.923585in}{0.657392in}}{\pgfqpoint{0.923585in}{0.662265in}}%
\pgfpathcurveto{\pgfqpoint{0.923585in}{0.667137in}}{\pgfqpoint{0.921649in}{0.671811in}}{\pgfqpoint{0.918203in}{0.675257in}}%
\pgfpathcurveto{\pgfqpoint{0.914758in}{0.678702in}}{\pgfqpoint{0.910084in}{0.680638in}}{\pgfqpoint{0.905212in}{0.680638in}}%
\pgfpathcurveto{\pgfqpoint{0.900339in}{0.680638in}}{\pgfqpoint{0.895665in}{0.678702in}}{\pgfqpoint{0.892220in}{0.675257in}}%
\pgfpathcurveto{\pgfqpoint{0.888774in}{0.671811in}}{\pgfqpoint{0.886838in}{0.667137in}}{\pgfqpoint{0.886838in}{0.662265in}}%
\pgfpathcurveto{\pgfqpoint{0.886838in}{0.657392in}}{\pgfqpoint{0.888774in}{0.652718in}}{\pgfqpoint{0.892220in}{0.649273in}}%
\pgfpathcurveto{\pgfqpoint{0.895665in}{0.645827in}}{\pgfqpoint{0.900339in}{0.643891in}}{\pgfqpoint{0.905212in}{0.643891in}}%
\pgfpathlineto{\pgfqpoint{0.905212in}{0.643891in}}%
\pgfpathclose%
\pgfusepath{stroke}%
\end{pgfscope}%
\begin{pgfscope}%
\pgfpathrectangle{\pgfqpoint{0.100000in}{0.100000in}}{\pgfqpoint{1.782500in}{1.232000in}}%
\pgfusepath{clip}%
\pgfsetbuttcap%
\pgfsetroundjoin%
\pgfsetlinewidth{0.501875pt}%
\definecolor{currentstroke}{rgb}{0.835294,0.321569,0.035294}%
\pgfsetstrokecolor{currentstroke}%
\pgfsetdash{}{0pt}%
\pgfpathmoveto{\pgfqpoint{0.991250in}{0.636027in}}%
\pgfpathcurveto{\pgfqpoint{0.996123in}{0.636027in}}{\pgfqpoint{1.000796in}{0.637963in}}{\pgfqpoint{1.004242in}{0.641408in}}%
\pgfpathcurveto{\pgfqpoint{1.007687in}{0.644854in}}{\pgfqpoint{1.009623in}{0.649527in}}{\pgfqpoint{1.009623in}{0.654400in}}%
\pgfpathcurveto{\pgfqpoint{1.009623in}{0.659273in}}{\pgfqpoint{1.007687in}{0.663946in}}{\pgfqpoint{1.004242in}{0.667392in}}%
\pgfpathcurveto{\pgfqpoint{1.000796in}{0.670837in}}{\pgfqpoint{0.996123in}{0.672773in}}{\pgfqpoint{0.991250in}{0.672773in}}%
\pgfpathcurveto{\pgfqpoint{0.986377in}{0.672773in}}{\pgfqpoint{0.981704in}{0.670837in}}{\pgfqpoint{0.978258in}{0.667392in}}%
\pgfpathcurveto{\pgfqpoint{0.974813in}{0.663946in}}{\pgfqpoint{0.972877in}{0.659273in}}{\pgfqpoint{0.972877in}{0.654400in}}%
\pgfpathcurveto{\pgfqpoint{0.972877in}{0.649527in}}{\pgfqpoint{0.974813in}{0.644854in}}{\pgfqpoint{0.978258in}{0.641408in}}%
\pgfpathcurveto{\pgfqpoint{0.981704in}{0.637963in}}{\pgfqpoint{0.986377in}{0.636027in}}{\pgfqpoint{0.991250in}{0.636027in}}%
\pgfpathlineto{\pgfqpoint{0.991250in}{0.636027in}}%
\pgfpathclose%
\pgfusepath{stroke}%
\end{pgfscope}%
\begin{pgfscope}%
\pgfpathrectangle{\pgfqpoint{0.100000in}{0.100000in}}{\pgfqpoint{1.782500in}{1.232000in}}%
\pgfusepath{clip}%
\pgfsetbuttcap%
\pgfsetroundjoin%
\pgfsetlinewidth{0.501875pt}%
\definecolor{currentstroke}{rgb}{0.835294,0.321569,0.035294}%
\pgfsetstrokecolor{currentstroke}%
\pgfsetdash{}{0pt}%
\pgfpathmoveto{\pgfqpoint{1.077288in}{0.643891in}}%
\pgfpathcurveto{\pgfqpoint{1.082161in}{0.643891in}}{\pgfqpoint{1.086835in}{0.645827in}}{\pgfqpoint{1.090280in}{0.649273in}}%
\pgfpathcurveto{\pgfqpoint{1.093726in}{0.652718in}}{\pgfqpoint{1.095662in}{0.657392in}}{\pgfqpoint{1.095662in}{0.662265in}}%
\pgfpathcurveto{\pgfqpoint{1.095662in}{0.667137in}}{\pgfqpoint{1.093726in}{0.671811in}}{\pgfqpoint{1.090280in}{0.675257in}}%
\pgfpathcurveto{\pgfqpoint{1.086835in}{0.678702in}}{\pgfqpoint{1.082161in}{0.680638in}}{\pgfqpoint{1.077288in}{0.680638in}}%
\pgfpathcurveto{\pgfqpoint{1.072416in}{0.680638in}}{\pgfqpoint{1.067742in}{0.678702in}}{\pgfqpoint{1.064297in}{0.675257in}}%
\pgfpathcurveto{\pgfqpoint{1.060851in}{0.671811in}}{\pgfqpoint{1.058915in}{0.667137in}}{\pgfqpoint{1.058915in}{0.662265in}}%
\pgfpathcurveto{\pgfqpoint{1.058915in}{0.657392in}}{\pgfqpoint{1.060851in}{0.652718in}}{\pgfqpoint{1.064297in}{0.649273in}}%
\pgfpathcurveto{\pgfqpoint{1.067742in}{0.645827in}}{\pgfqpoint{1.072416in}{0.643891in}}{\pgfqpoint{1.077288in}{0.643891in}}%
\pgfpathlineto{\pgfqpoint{1.077288in}{0.643891in}}%
\pgfpathclose%
\pgfusepath{stroke}%
\end{pgfscope}%
\begin{pgfscope}%
\pgfpathrectangle{\pgfqpoint{0.100000in}{0.100000in}}{\pgfqpoint{1.782500in}{1.232000in}}%
\pgfusepath{clip}%
\pgfsetbuttcap%
\pgfsetroundjoin%
\pgfsetlinewidth{0.501875pt}%
\definecolor{currentstroke}{rgb}{0.835294,0.321569,0.035294}%
\pgfsetstrokecolor{currentstroke}%
\pgfsetdash{}{0pt}%
\pgfpathmoveto{\pgfqpoint{1.161130in}{0.667285in}}%
\pgfpathcurveto{\pgfqpoint{1.166002in}{0.667285in}}{\pgfqpoint{1.170676in}{0.669221in}}{\pgfqpoint{1.174122in}{0.672666in}}%
\pgfpathcurveto{\pgfqpoint{1.177567in}{0.676112in}}{\pgfqpoint{1.179503in}{0.680786in}}{\pgfqpoint{1.179503in}{0.685658in}}%
\pgfpathcurveto{\pgfqpoint{1.179503in}{0.690531in}}{\pgfqpoint{1.177567in}{0.695205in}}{\pgfqpoint{1.174122in}{0.698650in}}%
\pgfpathcurveto{\pgfqpoint{1.170676in}{0.702096in}}{\pgfqpoint{1.166002in}{0.704031in}}{\pgfqpoint{1.161130in}{0.704031in}}%
\pgfpathcurveto{\pgfqpoint{1.156257in}{0.704031in}}{\pgfqpoint{1.151583in}{0.702096in}}{\pgfqpoint{1.148138in}{0.698650in}}%
\pgfpathcurveto{\pgfqpoint{1.144692in}{0.695205in}}{\pgfqpoint{1.142757in}{0.690531in}}{\pgfqpoint{1.142757in}{0.685658in}}%
\pgfpathcurveto{\pgfqpoint{1.142757in}{0.680786in}}{\pgfqpoint{1.144692in}{0.676112in}}{\pgfqpoint{1.148138in}{0.672666in}}%
\pgfpathcurveto{\pgfqpoint{1.151583in}{0.669221in}}{\pgfqpoint{1.156257in}{0.667285in}}{\pgfqpoint{1.161130in}{0.667285in}}%
\pgfpathlineto{\pgfqpoint{1.161130in}{0.667285in}}%
\pgfpathclose%
\pgfusepath{stroke}%
\end{pgfscope}%
\begin{pgfscope}%
\pgfpathrectangle{\pgfqpoint{0.100000in}{0.100000in}}{\pgfqpoint{1.782500in}{1.232000in}}%
\pgfusepath{clip}%
\pgfsetbuttcap%
\pgfsetroundjoin%
\pgfsetlinewidth{0.501875pt}%
\definecolor{currentstroke}{rgb}{0.835294,0.321569,0.035294}%
\pgfsetstrokecolor{currentstroke}%
\pgfsetdash{}{0pt}%
\pgfpathmoveto{\pgfqpoint{1.240633in}{0.705610in}}%
\pgfpathcurveto{\pgfqpoint{1.245506in}{0.705610in}}{\pgfqpoint{1.250180in}{0.707546in}}{\pgfqpoint{1.253625in}{0.710991in}}%
\pgfpathcurveto{\pgfqpoint{1.257071in}{0.714437in}}{\pgfqpoint{1.259007in}{0.719110in}}{\pgfqpoint{1.259007in}{0.723983in}}%
\pgfpathcurveto{\pgfqpoint{1.259007in}{0.728856in}}{\pgfqpoint{1.257071in}{0.733529in}}{\pgfqpoint{1.253625in}{0.736975in}}%
\pgfpathcurveto{\pgfqpoint{1.250180in}{0.740420in}}{\pgfqpoint{1.245506in}{0.742356in}}{\pgfqpoint{1.240633in}{0.742356in}}%
\pgfpathcurveto{\pgfqpoint{1.235761in}{0.742356in}}{\pgfqpoint{1.231087in}{0.740420in}}{\pgfqpoint{1.227642in}{0.736975in}}%
\pgfpathcurveto{\pgfqpoint{1.224196in}{0.733529in}}{\pgfqpoint{1.222260in}{0.728856in}}{\pgfqpoint{1.222260in}{0.723983in}}%
\pgfpathcurveto{\pgfqpoint{1.222260in}{0.719110in}}{\pgfqpoint{1.224196in}{0.714437in}}{\pgfqpoint{1.227642in}{0.710991in}}%
\pgfpathcurveto{\pgfqpoint{1.231087in}{0.707546in}}{\pgfqpoint{1.235761in}{0.705610in}}{\pgfqpoint{1.240633in}{0.705610in}}%
\pgfpathlineto{\pgfqpoint{1.240633in}{0.705610in}}%
\pgfpathclose%
\pgfusepath{stroke}%
\end{pgfscope}%
\begin{pgfscope}%
\pgfpathrectangle{\pgfqpoint{0.100000in}{0.100000in}}{\pgfqpoint{1.782500in}{1.232000in}}%
\pgfusepath{clip}%
\pgfsetbuttcap%
\pgfsetroundjoin%
\pgfsetlinewidth{0.501875pt}%
\definecolor{currentstroke}{rgb}{0.835294,0.321569,0.035294}%
\pgfsetstrokecolor{currentstroke}%
\pgfsetdash{}{0pt}%
\pgfpathmoveto{\pgfqpoint{1.313769in}{0.757887in}}%
\pgfpathcurveto{\pgfqpoint{1.318642in}{0.757887in}}{\pgfqpoint{1.323315in}{0.759823in}}{\pgfqpoint{1.326761in}{0.763269in}}%
\pgfpathcurveto{\pgfqpoint{1.330206in}{0.766714in}}{\pgfqpoint{1.332142in}{0.771388in}}{\pgfqpoint{1.332142in}{0.776261in}}%
\pgfpathcurveto{\pgfqpoint{1.332142in}{0.781133in}}{\pgfqpoint{1.330206in}{0.785807in}}{\pgfqpoint{1.326761in}{0.789252in}}%
\pgfpathcurveto{\pgfqpoint{1.323315in}{0.792698in}}{\pgfqpoint{1.318642in}{0.794634in}}{\pgfqpoint{1.313769in}{0.794634in}}%
\pgfpathcurveto{\pgfqpoint{1.308896in}{0.794634in}}{\pgfqpoint{1.304223in}{0.792698in}}{\pgfqpoint{1.300777in}{0.789252in}}%
\pgfpathcurveto{\pgfqpoint{1.297332in}{0.785807in}}{\pgfqpoint{1.295396in}{0.781133in}}{\pgfqpoint{1.295396in}{0.776261in}}%
\pgfpathcurveto{\pgfqpoint{1.295396in}{0.771388in}}{\pgfqpoint{1.297332in}{0.766714in}}{\pgfqpoint{1.300777in}{0.763269in}}%
\pgfpathcurveto{\pgfqpoint{1.304223in}{0.759823in}}{\pgfqpoint{1.308896in}{0.757887in}}{\pgfqpoint{1.313769in}{0.757887in}}%
\pgfpathlineto{\pgfqpoint{1.313769in}{0.757887in}}%
\pgfpathclose%
\pgfusepath{stroke}%
\end{pgfscope}%
\begin{pgfscope}%
\pgfpathrectangle{\pgfqpoint{0.100000in}{0.100000in}}{\pgfqpoint{1.782500in}{1.232000in}}%
\pgfusepath{clip}%
\pgfsetbuttcap%
\pgfsetroundjoin%
\pgfsetlinewidth{0.501875pt}%
\definecolor{currentstroke}{rgb}{0.835294,0.321569,0.035294}%
\pgfsetstrokecolor{currentstroke}%
\pgfsetdash{}{0pt}%
\pgfpathmoveto{\pgfqpoint{1.378669in}{0.822783in}}%
\pgfpathcurveto{\pgfqpoint{1.383542in}{0.822783in}}{\pgfqpoint{1.388215in}{0.824719in}}{\pgfqpoint{1.391661in}{0.828164in}}%
\pgfpathcurveto{\pgfqpoint{1.395106in}{0.831610in}}{\pgfqpoint{1.397042in}{0.836283in}}{\pgfqpoint{1.397042in}{0.841156in}}%
\pgfpathcurveto{\pgfqpoint{1.397042in}{0.846029in}}{\pgfqpoint{1.395106in}{0.850702in}}{\pgfqpoint{1.391661in}{0.854148in}}%
\pgfpathcurveto{\pgfqpoint{1.388215in}{0.857593in}}{\pgfqpoint{1.383542in}{0.859529in}}{\pgfqpoint{1.378669in}{0.859529in}}%
\pgfpathcurveto{\pgfqpoint{1.373796in}{0.859529in}}{\pgfqpoint{1.369123in}{0.857593in}}{\pgfqpoint{1.365677in}{0.854148in}}%
\pgfpathcurveto{\pgfqpoint{1.362232in}{0.850702in}}{\pgfqpoint{1.360296in}{0.846029in}}{\pgfqpoint{1.360296in}{0.841156in}}%
\pgfpathcurveto{\pgfqpoint{1.360296in}{0.836283in}}{\pgfqpoint{1.362232in}{0.831610in}}{\pgfqpoint{1.365677in}{0.828164in}}%
\pgfpathcurveto{\pgfqpoint{1.369123in}{0.824719in}}{\pgfqpoint{1.373796in}{0.822783in}}{\pgfqpoint{1.378669in}{0.822783in}}%
\pgfpathlineto{\pgfqpoint{1.378669in}{0.822783in}}%
\pgfpathclose%
\pgfusepath{stroke}%
\end{pgfscope}%
\begin{pgfscope}%
\pgfpathrectangle{\pgfqpoint{0.100000in}{0.100000in}}{\pgfqpoint{1.782500in}{1.232000in}}%
\pgfusepath{clip}%
\pgfsetbuttcap%
\pgfsetroundjoin%
\pgfsetlinewidth{0.501875pt}%
\definecolor{currentstroke}{rgb}{0.835294,0.321569,0.035294}%
\pgfsetstrokecolor{currentstroke}%
\pgfsetdash{}{0pt}%
\pgfpathmoveto{\pgfqpoint{1.433676in}{0.898639in}}%
\pgfpathcurveto{\pgfqpoint{1.438549in}{0.898639in}}{\pgfqpoint{1.443223in}{0.900575in}}{\pgfqpoint{1.446668in}{0.904020in}}%
\pgfpathcurveto{\pgfqpoint{1.450114in}{0.907466in}}{\pgfqpoint{1.452050in}{0.912139in}}{\pgfqpoint{1.452050in}{0.917012in}}%
\pgfpathcurveto{\pgfqpoint{1.452050in}{0.921885in}}{\pgfqpoint{1.450114in}{0.926558in}}{\pgfqpoint{1.446668in}{0.930004in}}%
\pgfpathcurveto{\pgfqpoint{1.443223in}{0.933449in}}{\pgfqpoint{1.438549in}{0.935385in}}{\pgfqpoint{1.433676in}{0.935385in}}%
\pgfpathcurveto{\pgfqpoint{1.428804in}{0.935385in}}{\pgfqpoint{1.424130in}{0.933449in}}{\pgfqpoint{1.420684in}{0.930004in}}%
\pgfpathcurveto{\pgfqpoint{1.417239in}{0.926558in}}{\pgfqpoint{1.415303in}{0.921885in}}{\pgfqpoint{1.415303in}{0.917012in}}%
\pgfpathcurveto{\pgfqpoint{1.415303in}{0.912139in}}{\pgfqpoint{1.417239in}{0.907466in}}{\pgfqpoint{1.420684in}{0.904020in}}%
\pgfpathcurveto{\pgfqpoint{1.424130in}{0.900575in}}{\pgfqpoint{1.428804in}{0.898639in}}{\pgfqpoint{1.433676in}{0.898639in}}%
\pgfpathlineto{\pgfqpoint{1.433676in}{0.898639in}}%
\pgfpathclose%
\pgfusepath{stroke}%
\end{pgfscope}%
\begin{pgfscope}%
\pgfpathrectangle{\pgfqpoint{0.100000in}{0.100000in}}{\pgfqpoint{1.782500in}{1.232000in}}%
\pgfusepath{clip}%
\pgfsetbuttcap%
\pgfsetroundjoin%
\definecolor{currentfill}{rgb}{0.054902,0.262745,0.486275}%
\pgfsetfillcolor{currentfill}%
\pgfsetlinewidth{1.003750pt}%
\definecolor{currentstroke}{rgb}{0.054902,0.262745,0.486275}%
\pgfsetstrokecolor{currentstroke}%
\pgfsetdash{}{0pt}%
\pgfsys@defobject{currentmarker}{\pgfqpoint{-0.018373in}{-0.018373in}}{\pgfqpoint{0.018373in}{0.018373in}}{%
\pgfpathmoveto{\pgfqpoint{0.000000in}{-0.018373in}}%
\pgfpathcurveto{\pgfqpoint{0.004873in}{-0.018373in}}{\pgfqpoint{0.009546in}{-0.016437in}}{\pgfqpoint{0.012992in}{-0.012992in}}%
\pgfpathcurveto{\pgfqpoint{0.016437in}{-0.009546in}}{\pgfqpoint{0.018373in}{-0.004873in}}{\pgfqpoint{0.018373in}{0.000000in}}%
\pgfpathcurveto{\pgfqpoint{0.018373in}{0.004873in}}{\pgfqpoint{0.016437in}{0.009546in}}{\pgfqpoint{0.012992in}{0.012992in}}%
\pgfpathcurveto{\pgfqpoint{0.009546in}{0.016437in}}{\pgfqpoint{0.004873in}{0.018373in}}{\pgfqpoint{0.000000in}{0.018373in}}%
\pgfpathcurveto{\pgfqpoint{-0.004873in}{0.018373in}}{\pgfqpoint{-0.009546in}{0.016437in}}{\pgfqpoint{-0.012992in}{0.012992in}}%
\pgfpathcurveto{\pgfqpoint{-0.016437in}{0.009546in}}{\pgfqpoint{-0.018373in}{0.004873in}}{\pgfqpoint{-0.018373in}{0.000000in}}%
\pgfpathcurveto{\pgfqpoint{-0.018373in}{-0.004873in}}{\pgfqpoint{-0.016437in}{-0.009546in}}{\pgfqpoint{-0.012992in}{-0.012992in}}%
\pgfpathcurveto{\pgfqpoint{-0.009546in}{-0.016437in}}{\pgfqpoint{-0.004873in}{-0.018373in}}{\pgfqpoint{0.000000in}{-0.018373in}}%
\pgfpathlineto{\pgfqpoint{0.000000in}{-0.018373in}}%
\pgfpathclose%
\pgfusepath{stroke,fill}%
}%
\begin{pgfscope}%
\pgfsys@transformshift{1.477386in}{0.716000in}%
\pgfsys@useobject{currentmarker}{}%
\end{pgfscope}%
\begin{pgfscope}%
\pgfsys@transformshift{0.505114in}{0.716000in}%
\pgfsys@useobject{currentmarker}{}%
\end{pgfscope}%
\begin{pgfscope}%
\pgfsys@transformshift{1.407938in}{0.716000in}%
\pgfsys@useobject{currentmarker}{}%
\end{pgfscope}%
\begin{pgfscope}%
\pgfsys@transformshift{1.338490in}{0.716000in}%
\pgfsys@useobject{currentmarker}{}%
\end{pgfscope}%
\begin{pgfscope}%
\pgfsys@transformshift{1.269042in}{0.716000in}%
\pgfsys@useobject{currentmarker}{}%
\end{pgfscope}%
\begin{pgfscope}%
\pgfsys@transformshift{1.199594in}{0.716000in}%
\pgfsys@useobject{currentmarker}{}%
\end{pgfscope}%
\begin{pgfscope}%
\pgfsys@transformshift{1.130146in}{0.716000in}%
\pgfsys@useobject{currentmarker}{}%
\end{pgfscope}%
\begin{pgfscope}%
\pgfsys@transformshift{1.060698in}{0.716000in}%
\pgfsys@useobject{currentmarker}{}%
\end{pgfscope}%
\begin{pgfscope}%
\pgfsys@transformshift{0.991250in}{0.716000in}%
\pgfsys@useobject{currentmarker}{}%
\end{pgfscope}%
\begin{pgfscope}%
\pgfsys@transformshift{0.921802in}{0.716000in}%
\pgfsys@useobject{currentmarker}{}%
\end{pgfscope}%
\begin{pgfscope}%
\pgfsys@transformshift{0.852354in}{0.716000in}%
\pgfsys@useobject{currentmarker}{}%
\end{pgfscope}%
\begin{pgfscope}%
\pgfsys@transformshift{0.782906in}{0.716000in}%
\pgfsys@useobject{currentmarker}{}%
\end{pgfscope}%
\begin{pgfscope}%
\pgfsys@transformshift{0.713458in}{0.716000in}%
\pgfsys@useobject{currentmarker}{}%
\end{pgfscope}%
\begin{pgfscope}%
\pgfsys@transformshift{0.644010in}{0.716000in}%
\pgfsys@useobject{currentmarker}{}%
\end{pgfscope}%
\begin{pgfscope}%
\pgfsys@transformshift{0.574562in}{0.716000in}%
\pgfsys@useobject{currentmarker}{}%
\end{pgfscope}%
\end{pgfscope}%
\begin{pgfscope}%
\pgfpathrectangle{\pgfqpoint{0.100000in}{0.100000in}}{\pgfqpoint{1.782500in}{1.232000in}}%
\pgfusepath{clip}%
\pgfsetbuttcap%
\pgfsetroundjoin%
\definecolor{currentfill}{rgb}{0.835294,0.321569,0.035294}%
\pgfsetfillcolor{currentfill}%
\pgfsetlinewidth{1.003750pt}%
\definecolor{currentstroke}{rgb}{0.835294,0.321569,0.035294}%
\pgfsetstrokecolor{currentstroke}%
\pgfsetdash{}{0pt}%
\pgfsys@defobject{currentmarker}{\pgfqpoint{-0.018373in}{-0.018373in}}{\pgfqpoint{0.018373in}{0.018373in}}{%
\pgfpathmoveto{\pgfqpoint{0.000000in}{-0.018373in}}%
\pgfpathcurveto{\pgfqpoint{0.004873in}{-0.018373in}}{\pgfqpoint{0.009546in}{-0.016437in}}{\pgfqpoint{0.012992in}{-0.012992in}}%
\pgfpathcurveto{\pgfqpoint{0.016437in}{-0.009546in}}{\pgfqpoint{0.018373in}{-0.004873in}}{\pgfqpoint{0.018373in}{0.000000in}}%
\pgfpathcurveto{\pgfqpoint{0.018373in}{0.004873in}}{\pgfqpoint{0.016437in}{0.009546in}}{\pgfqpoint{0.012992in}{0.012992in}}%
\pgfpathcurveto{\pgfqpoint{0.009546in}{0.016437in}}{\pgfqpoint{0.004873in}{0.018373in}}{\pgfqpoint{0.000000in}{0.018373in}}%
\pgfpathcurveto{\pgfqpoint{-0.004873in}{0.018373in}}{\pgfqpoint{-0.009546in}{0.016437in}}{\pgfqpoint{-0.012992in}{0.012992in}}%
\pgfpathcurveto{\pgfqpoint{-0.016437in}{0.009546in}}{\pgfqpoint{-0.018373in}{0.004873in}}{\pgfqpoint{-0.018373in}{0.000000in}}%
\pgfpathcurveto{\pgfqpoint{-0.018373in}{-0.004873in}}{\pgfqpoint{-0.016437in}{-0.009546in}}{\pgfqpoint{-0.012992in}{-0.012992in}}%
\pgfpathcurveto{\pgfqpoint{-0.009546in}{-0.016437in}}{\pgfqpoint{-0.004873in}{-0.018373in}}{\pgfqpoint{0.000000in}{-0.018373in}}%
\pgfpathlineto{\pgfqpoint{0.000000in}{-0.018373in}}%
\pgfpathclose%
\pgfusepath{stroke,fill}%
}%
\begin{pgfscope}%
\pgfsys@transformshift{0.603831in}{0.841156in}%
\pgfsys@useobject{currentmarker}{}%
\end{pgfscope}%
\begin{pgfscope}%
\pgfsys@transformshift{0.668731in}{0.776261in}%
\pgfsys@useobject{currentmarker}{}%
\end{pgfscope}%
\begin{pgfscope}%
\pgfsys@transformshift{0.741867in}{0.723983in}%
\pgfsys@useobject{currentmarker}{}%
\end{pgfscope}%
\begin{pgfscope}%
\pgfsys@transformshift{0.821370in}{0.685658in}%
\pgfsys@useobject{currentmarker}{}%
\end{pgfscope}%
\begin{pgfscope}%
\pgfsys@transformshift{0.905212in}{0.662265in}%
\pgfsys@useobject{currentmarker}{}%
\end{pgfscope}%
\begin{pgfscope}%
\pgfsys@transformshift{0.991250in}{0.654400in}%
\pgfsys@useobject{currentmarker}{}%
\end{pgfscope}%
\begin{pgfscope}%
\pgfsys@transformshift{1.077288in}{0.662265in}%
\pgfsys@useobject{currentmarker}{}%
\end{pgfscope}%
\begin{pgfscope}%
\pgfsys@transformshift{1.161130in}{0.685658in}%
\pgfsys@useobject{currentmarker}{}%
\end{pgfscope}%
\begin{pgfscope}%
\pgfsys@transformshift{1.240633in}{0.723983in}%
\pgfsys@useobject{currentmarker}{}%
\end{pgfscope}%
\begin{pgfscope}%
\pgfsys@transformshift{1.313769in}{0.776261in}%
\pgfsys@useobject{currentmarker}{}%
\end{pgfscope}%
\begin{pgfscope}%
\pgfsys@transformshift{1.378669in}{0.841156in}%
\pgfsys@useobject{currentmarker}{}%
\end{pgfscope}%
\end{pgfscope}%
\end{pgfpicture}%
\makeatother%
\endgroup%

        \caption{Iteration 1: Find interface}\label{fig:example-iter0-interface}
    \end{subfigure}
    \begin{subfigure}[b]{.32\linewidth}
        %% Creator: Matplotlib, PGF backend
%%
%% To include the figure in your LaTeX document, write
%%   \input{<filename>.pgf}
%%
%% Make sure the required packages are loaded in your preamble
%%   \usepackage{pgf}
%%
%% Also ensure that all the required font packages are loaded; for instance,
%% the lmodern package is sometimes necessary when using math font.
%%   \usepackage{lmodern}
%%
%% Figures using additional raster images can only be included by \input if
%% they are in the same directory as the main LaTeX file. For loading figures
%% from other directories you can use the `import` package
%%   \usepackage{import}
%%
%% and then include the figures with
%%   \import{<path to file>}{<filename>.pgf}
%%
%% Matplotlib used the following preamble
%%   
%%   \usepackage{fontspec}
%%   \setmainfont{DejaVuSans.ttf}[Path=\detokenize{/home/fabio/Internodes-CM/.venv/lib/python3.8/site-packages/matplotlib/mpl-data/fonts/ttf/}]
%%   \setsansfont{DejaVuSans.ttf}[Path=\detokenize{/home/fabio/Internodes-CM/.venv/lib/python3.8/site-packages/matplotlib/mpl-data/fonts/ttf/}]
%%   \setmonofont{DejaVuSansMono.ttf}[Path=\detokenize{/home/fabio/Internodes-CM/.venv/lib/python3.8/site-packages/matplotlib/mpl-data/fonts/ttf/}]
%%   \makeatletter\@ifpackageloaded{underscore}{}{\usepackage[strings]{underscore}}\makeatother
%%
\begingroup%
\makeatletter%
\begin{pgfpicture}%
\pgfpathrectangle{\pgfpointorigin}{\pgfqpoint{1.982500in}{1.432000in}}%
\pgfusepath{use as bounding box, clip}%
\begin{pgfscope}%
\pgfsetbuttcap%
\pgfsetmiterjoin%
\definecolor{currentfill}{rgb}{1.000000,1.000000,1.000000}%
\pgfsetfillcolor{currentfill}%
\pgfsetlinewidth{0.000000pt}%
\definecolor{currentstroke}{rgb}{1.000000,1.000000,1.000000}%
\pgfsetstrokecolor{currentstroke}%
\pgfsetdash{}{0pt}%
\pgfpathmoveto{\pgfqpoint{0.000000in}{0.000000in}}%
\pgfpathlineto{\pgfqpoint{1.982500in}{0.000000in}}%
\pgfpathlineto{\pgfqpoint{1.982500in}{1.432000in}}%
\pgfpathlineto{\pgfqpoint{0.000000in}{1.432000in}}%
\pgfpathlineto{\pgfqpoint{0.000000in}{0.000000in}}%
\pgfpathclose%
\pgfusepath{fill}%
\end{pgfscope}%
\begin{pgfscope}%
\pgfpathrectangle{\pgfqpoint{0.100000in}{0.100000in}}{\pgfqpoint{1.782500in}{1.232000in}}%
\pgfusepath{clip}%
\pgfsetrectcap%
\pgfsetroundjoin%
\pgfsetlinewidth{0.250937pt}%
\definecolor{currentstroke}{rgb}{0.054902,0.262745,0.486275}%
\pgfsetstrokecolor{currentstroke}%
\pgfsetdash{}{0pt}%
\pgfpathmoveto{\pgfqpoint{0.451098in}{0.100000in}}%
\pgfpathlineto{\pgfqpoint{0.181023in}{0.100000in}}%
\pgfpathmoveto{\pgfqpoint{0.721174in}{0.100000in}}%
\pgfpathlineto{\pgfqpoint{0.451098in}{0.100000in}}%
\pgfpathmoveto{\pgfqpoint{0.991250in}{0.100000in}}%
\pgfpathlineto{\pgfqpoint{0.721174in}{0.100000in}}%
\pgfpathmoveto{\pgfqpoint{1.261326in}{0.100000in}}%
\pgfpathlineto{\pgfqpoint{0.991250in}{0.100000in}}%
\pgfpathmoveto{\pgfqpoint{1.531402in}{0.100000in}}%
\pgfpathlineto{\pgfqpoint{1.801477in}{0.100000in}}%
\pgfpathmoveto{\pgfqpoint{1.531402in}{0.100000in}}%
\pgfpathlineto{\pgfqpoint{1.261326in}{0.100000in}}%
\pgfpathmoveto{\pgfqpoint{1.784788in}{0.431960in}}%
\pgfpathlineto{\pgfqpoint{1.801477in}{0.100000in}}%
\pgfpathmoveto{\pgfqpoint{1.784788in}{0.431960in}}%
\pgfpathlineto{\pgfqpoint{1.740685in}{0.754365in}}%
\pgfpathmoveto{\pgfqpoint{1.575146in}{0.731227in}}%
\pgfpathlineto{\pgfqpoint{1.740685in}{0.754365in}}%
\pgfpathmoveto{\pgfqpoint{1.575146in}{0.731227in}}%
\pgfpathlineto{\pgfqpoint{1.377693in}{0.703727in}}%
\pgfpathmoveto{\pgfqpoint{1.333507in}{0.678982in}}%
\pgfpathlineto{\pgfqpoint{1.377693in}{0.703727in}}%
\pgfpathmoveto{\pgfqpoint{1.286136in}{0.655839in}}%
\pgfpathlineto{\pgfqpoint{1.333507in}{0.678982in}}%
\pgfpathmoveto{\pgfqpoint{1.234900in}{0.638023in}}%
\pgfpathlineto{\pgfqpoint{1.286136in}{0.655839in}}%
\pgfpathmoveto{\pgfqpoint{1.180840in}{0.624920in}}%
\pgfpathlineto{\pgfqpoint{1.234900in}{0.638023in}}%
\pgfpathmoveto{\pgfqpoint{1.119285in}{0.614311in}}%
\pgfpathlineto{\pgfqpoint{1.180840in}{0.624920in}}%
\pgfpathmoveto{\pgfqpoint{1.054943in}{0.605752in}}%
\pgfpathlineto{\pgfqpoint{1.119285in}{0.614311in}}%
\pgfpathmoveto{\pgfqpoint{0.990043in}{0.598875in}}%
\pgfpathlineto{\pgfqpoint{1.054943in}{0.605752in}}%
\pgfpathmoveto{\pgfqpoint{0.924598in}{0.600413in}}%
\pgfpathlineto{\pgfqpoint{0.990043in}{0.598875in}}%
\pgfpathmoveto{\pgfqpoint{0.861703in}{0.610513in}}%
\pgfpathlineto{\pgfqpoint{0.924598in}{0.600413in}}%
\pgfpathmoveto{\pgfqpoint{0.800345in}{0.623839in}}%
\pgfpathlineto{\pgfqpoint{0.861703in}{0.610513in}}%
\pgfpathmoveto{\pgfqpoint{0.746146in}{0.638036in}}%
\pgfpathlineto{\pgfqpoint{0.800345in}{0.623839in}}%
\pgfpathmoveto{\pgfqpoint{0.695822in}{0.656509in}}%
\pgfpathlineto{\pgfqpoint{0.746146in}{0.638036in}}%
\pgfpathmoveto{\pgfqpoint{0.647969in}{0.682601in}}%
\pgfpathlineto{\pgfqpoint{0.602566in}{0.711351in}}%
\pgfpathmoveto{\pgfqpoint{0.647969in}{0.682601in}}%
\pgfpathlineto{\pgfqpoint{0.695822in}{0.656509in}}%
\pgfpathmoveto{\pgfqpoint{0.407330in}{0.734379in}}%
\pgfpathlineto{\pgfqpoint{0.602566in}{0.711351in}}%
\pgfpathmoveto{\pgfqpoint{0.407330in}{0.734379in}}%
\pgfpathlineto{\pgfqpoint{0.241845in}{0.758819in}}%
\pgfpathmoveto{\pgfqpoint{0.197059in}{0.435954in}}%
\pgfpathlineto{\pgfqpoint{0.181023in}{0.100000in}}%
\pgfpathmoveto{\pgfqpoint{0.197059in}{0.435954in}}%
\pgfpathlineto{\pgfqpoint{0.241845in}{0.758819in}}%
\pgfpathmoveto{\pgfqpoint{1.370566in}{0.368055in}}%
\pgfpathlineto{\pgfqpoint{1.261326in}{0.100000in}}%
\pgfpathmoveto{\pgfqpoint{1.081725in}{0.349008in}}%
\pgfpathlineto{\pgfqpoint{0.991250in}{0.100000in}}%
\pgfpathmoveto{\pgfqpoint{1.223541in}{0.458479in}}%
\pgfpathlineto{\pgfqpoint{1.370566in}{0.368055in}}%
\pgfpathmoveto{\pgfqpoint{1.223541in}{0.458479in}}%
\pgfpathlineto{\pgfqpoint{1.081725in}{0.349008in}}%
\pgfpathmoveto{\pgfqpoint{0.956373in}{0.443934in}}%
\pgfpathlineto{\pgfqpoint{1.081725in}{0.349008in}}%
\pgfpathmoveto{\pgfqpoint{0.956373in}{0.443934in}}%
\pgfpathlineto{\pgfqpoint{0.821549in}{0.363441in}}%
\pgfpathmoveto{\pgfqpoint{0.690710in}{0.470697in}}%
\pgfpathlineto{\pgfqpoint{0.821549in}{0.363441in}}%
\pgfpathmoveto{\pgfqpoint{0.690710in}{0.470697in}}%
\pgfpathlineto{\pgfqpoint{0.544697in}{0.384052in}}%
\pgfpathmoveto{\pgfqpoint{1.514560in}{0.493546in}}%
\pgfpathlineto{\pgfqpoint{1.784788in}{0.431960in}}%
\pgfpathmoveto{\pgfqpoint{1.514560in}{0.493546in}}%
\pgfpathlineto{\pgfqpoint{1.370566in}{0.368055in}}%
\pgfpathmoveto{\pgfqpoint{1.354379in}{0.509631in}}%
\pgfpathlineto{\pgfqpoint{1.370566in}{0.368055in}}%
\pgfpathmoveto{\pgfqpoint{1.354379in}{0.509631in}}%
\pgfpathlineto{\pgfqpoint{1.223541in}{0.458479in}}%
\pgfpathmoveto{\pgfqpoint{1.354379in}{0.509631in}}%
\pgfpathlineto{\pgfqpoint{1.514560in}{0.493546in}}%
\pgfpathmoveto{\pgfqpoint{1.088111in}{0.476529in}}%
\pgfpathlineto{\pgfqpoint{1.081725in}{0.349008in}}%
\pgfpathmoveto{\pgfqpoint{1.088111in}{0.476529in}}%
\pgfpathlineto{\pgfqpoint{1.223541in}{0.458479in}}%
\pgfpathmoveto{\pgfqpoint{1.088111in}{0.476529in}}%
\pgfpathlineto{\pgfqpoint{0.956373in}{0.443934in}}%
\pgfpathmoveto{\pgfqpoint{0.825301in}{0.482755in}}%
\pgfpathlineto{\pgfqpoint{0.821549in}{0.363441in}}%
\pgfpathmoveto{\pgfqpoint{0.825301in}{0.482755in}}%
\pgfpathlineto{\pgfqpoint{0.956373in}{0.443934in}}%
\pgfpathmoveto{\pgfqpoint{0.825301in}{0.482755in}}%
\pgfpathlineto{\pgfqpoint{0.690710in}{0.470697in}}%
\pgfpathmoveto{\pgfqpoint{0.551323in}{0.531974in}}%
\pgfpathlineto{\pgfqpoint{0.544697in}{0.384052in}}%
\pgfpathmoveto{\pgfqpoint{0.551323in}{0.531974in}}%
\pgfpathlineto{\pgfqpoint{0.690710in}{0.470697in}}%
\pgfpathmoveto{\pgfqpoint{0.381486in}{0.510355in}}%
\pgfpathlineto{\pgfqpoint{0.241845in}{0.758819in}}%
\pgfpathmoveto{\pgfqpoint{0.381486in}{0.510355in}}%
\pgfpathlineto{\pgfqpoint{0.407330in}{0.734379in}}%
\pgfpathmoveto{\pgfqpoint{0.381486in}{0.510355in}}%
\pgfpathlineto{\pgfqpoint{0.197059in}{0.435954in}}%
\pgfpathmoveto{\pgfqpoint{0.381486in}{0.510355in}}%
\pgfpathlineto{\pgfqpoint{0.544697in}{0.384052in}}%
\pgfpathmoveto{\pgfqpoint{0.381486in}{0.510355in}}%
\pgfpathlineto{\pgfqpoint{0.551323in}{0.531974in}}%
\pgfpathmoveto{\pgfqpoint{1.276566in}{0.559152in}}%
\pgfpathlineto{\pgfqpoint{1.286136in}{0.655839in}}%
\pgfpathmoveto{\pgfqpoint{1.276566in}{0.559152in}}%
\pgfpathlineto{\pgfqpoint{1.234900in}{0.638023in}}%
\pgfpathmoveto{\pgfqpoint{1.276566in}{0.559152in}}%
\pgfpathlineto{\pgfqpoint{1.223541in}{0.458479in}}%
\pgfpathmoveto{\pgfqpoint{1.276566in}{0.559152in}}%
\pgfpathlineto{\pgfqpoint{1.354379in}{0.509631in}}%
\pgfpathmoveto{\pgfqpoint{1.152718in}{0.539270in}}%
\pgfpathlineto{\pgfqpoint{1.180840in}{0.624920in}}%
\pgfpathmoveto{\pgfqpoint{1.152718in}{0.539270in}}%
\pgfpathlineto{\pgfqpoint{1.119285in}{0.614311in}}%
\pgfpathmoveto{\pgfqpoint{1.152718in}{0.539270in}}%
\pgfpathlineto{\pgfqpoint{1.223541in}{0.458479in}}%
\pgfpathmoveto{\pgfqpoint{1.152718in}{0.539270in}}%
\pgfpathlineto{\pgfqpoint{1.088111in}{0.476529in}}%
\pgfpathmoveto{\pgfqpoint{1.023657in}{0.527880in}}%
\pgfpathlineto{\pgfqpoint{1.054943in}{0.605752in}}%
\pgfpathmoveto{\pgfqpoint{1.023657in}{0.527880in}}%
\pgfpathlineto{\pgfqpoint{0.990043in}{0.598875in}}%
\pgfpathmoveto{\pgfqpoint{1.023657in}{0.527880in}}%
\pgfpathlineto{\pgfqpoint{0.956373in}{0.443934in}}%
\pgfpathmoveto{\pgfqpoint{1.023657in}{0.527880in}}%
\pgfpathlineto{\pgfqpoint{1.088111in}{0.476529in}}%
\pgfpathmoveto{\pgfqpoint{0.891682in}{0.530129in}}%
\pgfpathlineto{\pgfqpoint{0.924598in}{0.600413in}}%
\pgfpathmoveto{\pgfqpoint{0.891682in}{0.530129in}}%
\pgfpathlineto{\pgfqpoint{0.861703in}{0.610513in}}%
\pgfpathmoveto{\pgfqpoint{0.891682in}{0.530129in}}%
\pgfpathlineto{\pgfqpoint{0.956373in}{0.443934in}}%
\pgfpathmoveto{\pgfqpoint{0.891682in}{0.530129in}}%
\pgfpathlineto{\pgfqpoint{0.825301in}{0.482755in}}%
\pgfpathmoveto{\pgfqpoint{0.765740in}{0.547058in}}%
\pgfpathlineto{\pgfqpoint{0.800345in}{0.623839in}}%
\pgfpathmoveto{\pgfqpoint{0.765740in}{0.547058in}}%
\pgfpathlineto{\pgfqpoint{0.746146in}{0.638036in}}%
\pgfpathmoveto{\pgfqpoint{0.765740in}{0.547058in}}%
\pgfpathlineto{\pgfqpoint{0.690710in}{0.470697in}}%
\pgfpathmoveto{\pgfqpoint{0.765740in}{0.547058in}}%
\pgfpathlineto{\pgfqpoint{0.825301in}{0.482755in}}%
\pgfpathmoveto{\pgfqpoint{0.644521in}{0.578018in}}%
\pgfpathlineto{\pgfqpoint{0.695822in}{0.656509in}}%
\pgfpathmoveto{\pgfqpoint{0.644521in}{0.578018in}}%
\pgfpathlineto{\pgfqpoint{0.647969in}{0.682601in}}%
\pgfpathmoveto{\pgfqpoint{0.644521in}{0.578018in}}%
\pgfpathlineto{\pgfqpoint{0.690710in}{0.470697in}}%
\pgfpathmoveto{\pgfqpoint{0.644521in}{0.578018in}}%
\pgfpathlineto{\pgfqpoint{0.551323in}{0.531974in}}%
\pgfpathmoveto{\pgfqpoint{1.407587in}{0.599878in}}%
\pgfpathlineto{\pgfqpoint{1.377693in}{0.703727in}}%
\pgfpathmoveto{\pgfqpoint{1.407587in}{0.599878in}}%
\pgfpathlineto{\pgfqpoint{1.333507in}{0.678982in}}%
\pgfpathmoveto{\pgfqpoint{1.407587in}{0.599878in}}%
\pgfpathlineto{\pgfqpoint{1.514560in}{0.493546in}}%
\pgfpathmoveto{\pgfqpoint{1.407587in}{0.599878in}}%
\pgfpathlineto{\pgfqpoint{1.354379in}{0.509631in}}%
\pgfpathmoveto{\pgfqpoint{0.938577in}{0.290022in}}%
\pgfpathlineto{\pgfqpoint{0.991250in}{0.100000in}}%
\pgfpathmoveto{\pgfqpoint{0.938577in}{0.290022in}}%
\pgfpathlineto{\pgfqpoint{1.081725in}{0.349008in}}%
\pgfpathmoveto{\pgfqpoint{0.938577in}{0.290022in}}%
\pgfpathlineto{\pgfqpoint{0.821549in}{0.363441in}}%
\pgfpathmoveto{\pgfqpoint{0.938577in}{0.290022in}}%
\pgfpathlineto{\pgfqpoint{0.956373in}{0.443934in}}%
\pgfpathmoveto{\pgfqpoint{0.481120in}{0.621443in}}%
\pgfpathlineto{\pgfqpoint{0.602566in}{0.711351in}}%
\pgfpathmoveto{\pgfqpoint{0.481120in}{0.621443in}}%
\pgfpathlineto{\pgfqpoint{0.407330in}{0.734379in}}%
\pgfpathmoveto{\pgfqpoint{0.481120in}{0.621443in}}%
\pgfpathlineto{\pgfqpoint{0.551323in}{0.531974in}}%
\pgfpathmoveto{\pgfqpoint{0.481120in}{0.621443in}}%
\pgfpathlineto{\pgfqpoint{0.381486in}{0.510355in}}%
\pgfpathmoveto{\pgfqpoint{1.332349in}{0.597525in}}%
\pgfpathlineto{\pgfqpoint{1.333507in}{0.678982in}}%
\pgfpathmoveto{\pgfqpoint{1.332349in}{0.597525in}}%
\pgfpathlineto{\pgfqpoint{1.286136in}{0.655839in}}%
\pgfpathmoveto{\pgfqpoint{1.332349in}{0.597525in}}%
\pgfpathlineto{\pgfqpoint{1.354379in}{0.509631in}}%
\pgfpathmoveto{\pgfqpoint{1.332349in}{0.597525in}}%
\pgfpathlineto{\pgfqpoint{1.276566in}{0.559152in}}%
\pgfpathmoveto{\pgfqpoint{1.332349in}{0.597525in}}%
\pgfpathlineto{\pgfqpoint{1.407587in}{0.599878in}}%
\pgfpathmoveto{\pgfqpoint{1.087875in}{0.552554in}}%
\pgfpathlineto{\pgfqpoint{1.119285in}{0.614311in}}%
\pgfpathmoveto{\pgfqpoint{1.087875in}{0.552554in}}%
\pgfpathlineto{\pgfqpoint{1.054943in}{0.605752in}}%
\pgfpathmoveto{\pgfqpoint{1.087875in}{0.552554in}}%
\pgfpathlineto{\pgfqpoint{1.088111in}{0.476529in}}%
\pgfpathmoveto{\pgfqpoint{1.087875in}{0.552554in}}%
\pgfpathlineto{\pgfqpoint{1.152718in}{0.539270in}}%
\pgfpathmoveto{\pgfqpoint{1.087875in}{0.552554in}}%
\pgfpathlineto{\pgfqpoint{1.023657in}{0.527880in}}%
\pgfpathmoveto{\pgfqpoint{1.213366in}{0.566671in}}%
\pgfpathlineto{\pgfqpoint{1.234900in}{0.638023in}}%
\pgfpathmoveto{\pgfqpoint{1.213366in}{0.566671in}}%
\pgfpathlineto{\pgfqpoint{1.180840in}{0.624920in}}%
\pgfpathmoveto{\pgfqpoint{1.213366in}{0.566671in}}%
\pgfpathlineto{\pgfqpoint{1.223541in}{0.458479in}}%
\pgfpathmoveto{\pgfqpoint{1.213366in}{0.566671in}}%
\pgfpathlineto{\pgfqpoint{1.276566in}{0.559152in}}%
\pgfpathmoveto{\pgfqpoint{1.213366in}{0.566671in}}%
\pgfpathlineto{\pgfqpoint{1.152718in}{0.539270in}}%
\pgfpathmoveto{\pgfqpoint{0.957452in}{0.539685in}}%
\pgfpathlineto{\pgfqpoint{0.990043in}{0.598875in}}%
\pgfpathmoveto{\pgfqpoint{0.957452in}{0.539685in}}%
\pgfpathlineto{\pgfqpoint{0.924598in}{0.600413in}}%
\pgfpathmoveto{\pgfqpoint{0.957452in}{0.539685in}}%
\pgfpathlineto{\pgfqpoint{0.956373in}{0.443934in}}%
\pgfpathmoveto{\pgfqpoint{0.957452in}{0.539685in}}%
\pgfpathlineto{\pgfqpoint{1.023657in}{0.527880in}}%
\pgfpathmoveto{\pgfqpoint{0.957452in}{0.539685in}}%
\pgfpathlineto{\pgfqpoint{0.891682in}{0.530129in}}%
\pgfpathmoveto{\pgfqpoint{0.828623in}{0.557099in}}%
\pgfpathlineto{\pgfqpoint{0.861703in}{0.610513in}}%
\pgfpathmoveto{\pgfqpoint{0.828623in}{0.557099in}}%
\pgfpathlineto{\pgfqpoint{0.800345in}{0.623839in}}%
\pgfpathmoveto{\pgfqpoint{0.828623in}{0.557099in}}%
\pgfpathlineto{\pgfqpoint{0.825301in}{0.482755in}}%
\pgfpathmoveto{\pgfqpoint{0.828623in}{0.557099in}}%
\pgfpathlineto{\pgfqpoint{0.891682in}{0.530129in}}%
\pgfpathmoveto{\pgfqpoint{0.828623in}{0.557099in}}%
\pgfpathlineto{\pgfqpoint{0.765740in}{0.547058in}}%
\pgfpathmoveto{\pgfqpoint{1.188848in}{0.271931in}}%
\pgfpathlineto{\pgfqpoint{0.991250in}{0.100000in}}%
\pgfpathmoveto{\pgfqpoint{1.188848in}{0.271931in}}%
\pgfpathlineto{\pgfqpoint{1.261326in}{0.100000in}}%
\pgfpathmoveto{\pgfqpoint{1.188848in}{0.271931in}}%
\pgfpathlineto{\pgfqpoint{1.370566in}{0.368055in}}%
\pgfpathmoveto{\pgfqpoint{1.188848in}{0.271931in}}%
\pgfpathlineto{\pgfqpoint{1.081725in}{0.349008in}}%
\pgfpathmoveto{\pgfqpoint{1.188848in}{0.271931in}}%
\pgfpathlineto{\pgfqpoint{1.223541in}{0.458479in}}%
\pgfpathmoveto{\pgfqpoint{0.693813in}{0.285882in}}%
\pgfpathlineto{\pgfqpoint{0.721174in}{0.100000in}}%
\pgfpathmoveto{\pgfqpoint{0.693813in}{0.285882in}}%
\pgfpathlineto{\pgfqpoint{0.821549in}{0.363441in}}%
\pgfpathmoveto{\pgfqpoint{0.693813in}{0.285882in}}%
\pgfpathlineto{\pgfqpoint{0.544697in}{0.384052in}}%
\pgfpathmoveto{\pgfqpoint{0.693813in}{0.285882in}}%
\pgfpathlineto{\pgfqpoint{0.690710in}{0.470697in}}%
\pgfpathmoveto{\pgfqpoint{0.707873in}{0.576304in}}%
\pgfpathlineto{\pgfqpoint{0.746146in}{0.638036in}}%
\pgfpathmoveto{\pgfqpoint{0.707873in}{0.576304in}}%
\pgfpathlineto{\pgfqpoint{0.695822in}{0.656509in}}%
\pgfpathmoveto{\pgfqpoint{0.707873in}{0.576304in}}%
\pgfpathlineto{\pgfqpoint{0.690710in}{0.470697in}}%
\pgfpathmoveto{\pgfqpoint{0.707873in}{0.576304in}}%
\pgfpathlineto{\pgfqpoint{0.765740in}{0.547058in}}%
\pgfpathmoveto{\pgfqpoint{0.707873in}{0.576304in}}%
\pgfpathlineto{\pgfqpoint{0.644521in}{0.578018in}}%
\pgfpathmoveto{\pgfqpoint{1.505766in}{0.629724in}}%
\pgfpathlineto{\pgfqpoint{1.377693in}{0.703727in}}%
\pgfpathmoveto{\pgfqpoint{1.505766in}{0.629724in}}%
\pgfpathlineto{\pgfqpoint{1.575146in}{0.731227in}}%
\pgfpathmoveto{\pgfqpoint{1.505766in}{0.629724in}}%
\pgfpathlineto{\pgfqpoint{1.514560in}{0.493546in}}%
\pgfpathmoveto{\pgfqpoint{1.505766in}{0.629724in}}%
\pgfpathlineto{\pgfqpoint{1.407587in}{0.599878in}}%
\pgfpathmoveto{\pgfqpoint{0.592313in}{0.620955in}}%
\pgfpathlineto{\pgfqpoint{0.602566in}{0.711351in}}%
\pgfpathmoveto{\pgfqpoint{0.592313in}{0.620955in}}%
\pgfpathlineto{\pgfqpoint{0.647969in}{0.682601in}}%
\pgfpathmoveto{\pgfqpoint{0.592313in}{0.620955in}}%
\pgfpathlineto{\pgfqpoint{0.551323in}{0.531974in}}%
\pgfpathmoveto{\pgfqpoint{0.592313in}{0.620955in}}%
\pgfpathlineto{\pgfqpoint{0.644521in}{0.578018in}}%
\pgfpathmoveto{\pgfqpoint{0.592313in}{0.620955in}}%
\pgfpathlineto{\pgfqpoint{0.481120in}{0.621443in}}%
\pgfpathmoveto{\pgfqpoint{0.380766in}{0.291207in}}%
\pgfpathlineto{\pgfqpoint{0.181023in}{0.100000in}}%
\pgfpathmoveto{\pgfqpoint{0.380766in}{0.291207in}}%
\pgfpathlineto{\pgfqpoint{0.451098in}{0.100000in}}%
\pgfpathmoveto{\pgfqpoint{0.380766in}{0.291207in}}%
\pgfpathlineto{\pgfqpoint{0.197059in}{0.435954in}}%
\pgfpathmoveto{\pgfqpoint{0.380766in}{0.291207in}}%
\pgfpathlineto{\pgfqpoint{0.544697in}{0.384052in}}%
\pgfpathmoveto{\pgfqpoint{0.380766in}{0.291207in}}%
\pgfpathlineto{\pgfqpoint{0.381486in}{0.510355in}}%
\pgfpathmoveto{\pgfqpoint{1.529279in}{0.294989in}}%
\pgfpathlineto{\pgfqpoint{1.801477in}{0.100000in}}%
\pgfpathmoveto{\pgfqpoint{1.529279in}{0.294989in}}%
\pgfpathlineto{\pgfqpoint{1.261326in}{0.100000in}}%
\pgfpathmoveto{\pgfqpoint{1.529279in}{0.294989in}}%
\pgfpathlineto{\pgfqpoint{1.531402in}{0.100000in}}%
\pgfpathmoveto{\pgfqpoint{1.529279in}{0.294989in}}%
\pgfpathlineto{\pgfqpoint{1.784788in}{0.431960in}}%
\pgfpathmoveto{\pgfqpoint{1.529279in}{0.294989in}}%
\pgfpathlineto{\pgfqpoint{1.370566in}{0.368055in}}%
\pgfpathmoveto{\pgfqpoint{1.529279in}{0.294989in}}%
\pgfpathlineto{\pgfqpoint{1.514560in}{0.493546in}}%
\pgfpathmoveto{\pgfqpoint{1.624417in}{0.609671in}}%
\pgfpathlineto{\pgfqpoint{1.740685in}{0.754365in}}%
\pgfpathmoveto{\pgfqpoint{1.624417in}{0.609671in}}%
\pgfpathlineto{\pgfqpoint{1.784788in}{0.431960in}}%
\pgfpathmoveto{\pgfqpoint{1.624417in}{0.609671in}}%
\pgfpathlineto{\pgfqpoint{1.575146in}{0.731227in}}%
\pgfpathmoveto{\pgfqpoint{1.624417in}{0.609671in}}%
\pgfpathlineto{\pgfqpoint{1.514560in}{0.493546in}}%
\pgfpathmoveto{\pgfqpoint{1.624417in}{0.609671in}}%
\pgfpathlineto{\pgfqpoint{1.505766in}{0.629724in}}%
\pgfpathmoveto{\pgfqpoint{0.840111in}{0.209031in}}%
\pgfpathlineto{\pgfqpoint{0.721174in}{0.100000in}}%
\pgfpathmoveto{\pgfqpoint{0.840111in}{0.209031in}}%
\pgfpathlineto{\pgfqpoint{0.991250in}{0.100000in}}%
\pgfpathmoveto{\pgfqpoint{0.840111in}{0.209031in}}%
\pgfpathlineto{\pgfqpoint{0.821549in}{0.363441in}}%
\pgfpathmoveto{\pgfqpoint{0.840111in}{0.209031in}}%
\pgfpathlineto{\pgfqpoint{0.938577in}{0.290022in}}%
\pgfpathmoveto{\pgfqpoint{0.840111in}{0.209031in}}%
\pgfpathlineto{\pgfqpoint{0.693813in}{0.285882in}}%
\pgfpathmoveto{\pgfqpoint{0.554609in}{0.230276in}}%
\pgfpathlineto{\pgfqpoint{0.451098in}{0.100000in}}%
\pgfpathmoveto{\pgfqpoint{0.554609in}{0.230276in}}%
\pgfpathlineto{\pgfqpoint{0.721174in}{0.100000in}}%
\pgfpathmoveto{\pgfqpoint{0.554609in}{0.230276in}}%
\pgfpathlineto{\pgfqpoint{0.544697in}{0.384052in}}%
\pgfpathmoveto{\pgfqpoint{0.554609in}{0.230276in}}%
\pgfpathlineto{\pgfqpoint{0.693813in}{0.285882in}}%
\pgfpathmoveto{\pgfqpoint{0.554609in}{0.230276in}}%
\pgfpathlineto{\pgfqpoint{0.380766in}{0.291207in}}%
\pgfpathlineto{\pgfqpoint{0.380766in}{0.291207in}}%
\pgfusepath{stroke}%
\end{pgfscope}%
\begin{pgfscope}%
\pgfpathrectangle{\pgfqpoint{0.100000in}{0.100000in}}{\pgfqpoint{1.782500in}{1.232000in}}%
\pgfusepath{clip}%
\pgfsetrectcap%
\pgfsetroundjoin%
\pgfsetlinewidth{0.250937pt}%
\definecolor{currentstroke}{rgb}{0.835294,0.321569,0.035294}%
\pgfsetstrokecolor{currentstroke}%
\pgfsetdash{}{0pt}%
\pgfpathmoveto{\pgfqpoint{0.496062in}{0.835205in}}%
\pgfpathlineto{\pgfqpoint{0.451098in}{1.085600in}}%
\pgfpathmoveto{\pgfqpoint{1.531402in}{1.085600in}}%
\pgfpathlineto{\pgfqpoint{1.486187in}{0.833488in}}%
\pgfpathmoveto{\pgfqpoint{0.721174in}{1.085600in}}%
\pgfpathlineto{\pgfqpoint{0.991250in}{1.085600in}}%
\pgfpathmoveto{\pgfqpoint{0.721174in}{1.085600in}}%
\pgfpathlineto{\pgfqpoint{0.451098in}{1.085600in}}%
\pgfpathmoveto{\pgfqpoint{0.546727in}{0.758322in}}%
\pgfpathlineto{\pgfqpoint{0.496062in}{0.835205in}}%
\pgfpathmoveto{\pgfqpoint{0.613861in}{0.704199in}}%
\pgfpathlineto{\pgfqpoint{0.546727in}{0.758322in}}%
\pgfpathmoveto{\pgfqpoint{0.684747in}{0.662330in}}%
\pgfpathlineto{\pgfqpoint{0.613861in}{0.704199in}}%
\pgfpathmoveto{\pgfqpoint{0.763773in}{0.632922in}}%
\pgfpathlineto{\pgfqpoint{0.684747in}{0.662330in}}%
\pgfpathmoveto{\pgfqpoint{0.842615in}{0.614483in}}%
\pgfpathlineto{\pgfqpoint{0.763773in}{0.632922in}}%
\pgfpathmoveto{\pgfqpoint{0.916422in}{0.601060in}}%
\pgfpathlineto{\pgfqpoint{0.842615in}{0.614483in}}%
\pgfpathmoveto{\pgfqpoint{0.990080in}{0.598631in}}%
\pgfpathlineto{\pgfqpoint{0.916422in}{0.601060in}}%
\pgfpathmoveto{\pgfqpoint{1.063069in}{0.606844in}}%
\pgfpathlineto{\pgfqpoint{0.990080in}{0.598631in}}%
\pgfpathmoveto{\pgfqpoint{1.138803in}{0.617203in}}%
\pgfpathlineto{\pgfqpoint{1.063069in}{0.606844in}}%
\pgfpathmoveto{\pgfqpoint{1.217115in}{0.633115in}}%
\pgfpathlineto{\pgfqpoint{1.138803in}{0.617203in}}%
\pgfpathmoveto{\pgfqpoint{1.297142in}{0.661043in}}%
\pgfpathlineto{\pgfqpoint{1.217115in}{0.633115in}}%
\pgfpathmoveto{\pgfqpoint{1.366701in}{0.697571in}}%
\pgfpathlineto{\pgfqpoint{1.297142in}{0.661043in}}%
\pgfpathmoveto{\pgfqpoint{1.435831in}{0.755419in}}%
\pgfpathlineto{\pgfqpoint{1.486187in}{0.833488in}}%
\pgfpathmoveto{\pgfqpoint{1.435831in}{0.755419in}}%
\pgfpathlineto{\pgfqpoint{1.366701in}{0.697571in}}%
\pgfpathmoveto{\pgfqpoint{1.261326in}{1.085600in}}%
\pgfpathlineto{\pgfqpoint{0.991250in}{1.085600in}}%
\pgfpathmoveto{\pgfqpoint{1.261326in}{1.085600in}}%
\pgfpathlineto{\pgfqpoint{1.531402in}{1.085600in}}%
\pgfpathmoveto{\pgfqpoint{0.833368in}{0.890343in}}%
\pgfpathlineto{\pgfqpoint{0.991250in}{1.085600in}}%
\pgfpathmoveto{\pgfqpoint{0.833368in}{0.890343in}}%
\pgfpathlineto{\pgfqpoint{0.721174in}{1.085600in}}%
\pgfpathmoveto{\pgfqpoint{0.961724in}{0.757684in}}%
\pgfpathlineto{\pgfqpoint{1.113897in}{0.827208in}}%
\pgfpathmoveto{\pgfqpoint{0.961724in}{0.757684in}}%
\pgfpathlineto{\pgfqpoint{0.833368in}{0.890343in}}%
\pgfpathmoveto{\pgfqpoint{1.279672in}{0.859587in}}%
\pgfpathlineto{\pgfqpoint{1.113897in}{0.827208in}}%
\pgfpathmoveto{\pgfqpoint{0.668050in}{0.896214in}}%
\pgfpathlineto{\pgfqpoint{0.721174in}{1.085600in}}%
\pgfpathmoveto{\pgfqpoint{0.668050in}{0.896214in}}%
\pgfpathlineto{\pgfqpoint{0.833368in}{0.890343in}}%
\pgfpathmoveto{\pgfqpoint{1.087918in}{0.720099in}}%
\pgfpathlineto{\pgfqpoint{1.063069in}{0.606844in}}%
\pgfpathmoveto{\pgfqpoint{1.087918in}{0.720099in}}%
\pgfpathlineto{\pgfqpoint{1.138803in}{0.617203in}}%
\pgfpathmoveto{\pgfqpoint{1.087918in}{0.720099in}}%
\pgfpathlineto{\pgfqpoint{1.113897in}{0.827208in}}%
\pgfpathmoveto{\pgfqpoint{1.087918in}{0.720099in}}%
\pgfpathlineto{\pgfqpoint{0.961724in}{0.757684in}}%
\pgfpathmoveto{\pgfqpoint{1.208497in}{0.752670in}}%
\pgfpathlineto{\pgfqpoint{1.217115in}{0.633115in}}%
\pgfpathmoveto{\pgfqpoint{1.208497in}{0.752670in}}%
\pgfpathlineto{\pgfqpoint{1.297142in}{0.661043in}}%
\pgfpathmoveto{\pgfqpoint{1.208497in}{0.752670in}}%
\pgfpathlineto{\pgfqpoint{1.113897in}{0.827208in}}%
\pgfpathmoveto{\pgfqpoint{1.208497in}{0.752670in}}%
\pgfpathlineto{\pgfqpoint{1.279672in}{0.859587in}}%
\pgfpathmoveto{\pgfqpoint{1.208497in}{0.752670in}}%
\pgfpathlineto{\pgfqpoint{1.087918in}{0.720099in}}%
\pgfpathmoveto{\pgfqpoint{0.826322in}{0.731329in}}%
\pgfpathlineto{\pgfqpoint{0.763773in}{0.632922in}}%
\pgfpathmoveto{\pgfqpoint{0.826322in}{0.731329in}}%
\pgfpathlineto{\pgfqpoint{0.842615in}{0.614483in}}%
\pgfpathmoveto{\pgfqpoint{0.826322in}{0.731329in}}%
\pgfpathlineto{\pgfqpoint{0.833368in}{0.890343in}}%
\pgfpathmoveto{\pgfqpoint{0.826322in}{0.731329in}}%
\pgfpathlineto{\pgfqpoint{0.961724in}{0.757684in}}%
\pgfpathmoveto{\pgfqpoint{0.715986in}{0.784595in}}%
\pgfpathlineto{\pgfqpoint{0.613861in}{0.704199in}}%
\pgfpathmoveto{\pgfqpoint{0.715986in}{0.784595in}}%
\pgfpathlineto{\pgfqpoint{0.684747in}{0.662330in}}%
\pgfpathmoveto{\pgfqpoint{0.715986in}{0.784595in}}%
\pgfpathlineto{\pgfqpoint{0.833368in}{0.890343in}}%
\pgfpathmoveto{\pgfqpoint{0.715986in}{0.784595in}}%
\pgfpathlineto{\pgfqpoint{0.668050in}{0.896214in}}%
\pgfpathmoveto{\pgfqpoint{0.715986in}{0.784595in}}%
\pgfpathlineto{\pgfqpoint{0.826322in}{0.731329in}}%
\pgfpathmoveto{\pgfqpoint{1.369718in}{0.959100in}}%
\pgfpathlineto{\pgfqpoint{1.486187in}{0.833488in}}%
\pgfpathmoveto{\pgfqpoint{1.369718in}{0.959100in}}%
\pgfpathlineto{\pgfqpoint{1.531402in}{1.085600in}}%
\pgfpathmoveto{\pgfqpoint{1.369718in}{0.959100in}}%
\pgfpathlineto{\pgfqpoint{1.261326in}{1.085600in}}%
\pgfpathmoveto{\pgfqpoint{1.369718in}{0.959100in}}%
\pgfpathlineto{\pgfqpoint{1.279672in}{0.859587in}}%
\pgfpathmoveto{\pgfqpoint{0.955632in}{0.657941in}}%
\pgfpathlineto{\pgfqpoint{0.916422in}{0.601060in}}%
\pgfpathmoveto{\pgfqpoint{0.955632in}{0.657941in}}%
\pgfpathlineto{\pgfqpoint{0.990080in}{0.598631in}}%
\pgfpathmoveto{\pgfqpoint{0.955632in}{0.657941in}}%
\pgfpathlineto{\pgfqpoint{0.961724in}{0.757684in}}%
\pgfpathmoveto{\pgfqpoint{1.292856in}{0.745563in}}%
\pgfpathlineto{\pgfqpoint{1.297142in}{0.661043in}}%
\pgfpathmoveto{\pgfqpoint{1.292856in}{0.745563in}}%
\pgfpathlineto{\pgfqpoint{1.366701in}{0.697571in}}%
\pgfpathmoveto{\pgfqpoint{1.292856in}{0.745563in}}%
\pgfpathlineto{\pgfqpoint{1.279672in}{0.859587in}}%
\pgfpathmoveto{\pgfqpoint{1.292856in}{0.745563in}}%
\pgfpathlineto{\pgfqpoint{1.208497in}{0.752670in}}%
\pgfpathmoveto{\pgfqpoint{1.161401in}{0.688685in}}%
\pgfpathlineto{\pgfqpoint{1.138803in}{0.617203in}}%
\pgfpathmoveto{\pgfqpoint{1.161401in}{0.688685in}}%
\pgfpathlineto{\pgfqpoint{1.217115in}{0.633115in}}%
\pgfpathmoveto{\pgfqpoint{1.161401in}{0.688685in}}%
\pgfpathlineto{\pgfqpoint{1.087918in}{0.720099in}}%
\pgfpathmoveto{\pgfqpoint{1.161401in}{0.688685in}}%
\pgfpathlineto{\pgfqpoint{1.208497in}{0.752670in}}%
\pgfpathmoveto{\pgfqpoint{1.382877in}{0.853640in}}%
\pgfpathlineto{\pgfqpoint{1.486187in}{0.833488in}}%
\pgfpathmoveto{\pgfqpoint{1.382877in}{0.853640in}}%
\pgfpathlineto{\pgfqpoint{1.435831in}{0.755419in}}%
\pgfpathmoveto{\pgfqpoint{1.382877in}{0.853640in}}%
\pgfpathlineto{\pgfqpoint{1.279672in}{0.859587in}}%
\pgfpathmoveto{\pgfqpoint{1.382877in}{0.853640in}}%
\pgfpathlineto{\pgfqpoint{1.369718in}{0.959100in}}%
\pgfpathmoveto{\pgfqpoint{0.750725in}{0.710054in}}%
\pgfpathlineto{\pgfqpoint{0.684747in}{0.662330in}}%
\pgfpathmoveto{\pgfqpoint{0.750725in}{0.710054in}}%
\pgfpathlineto{\pgfqpoint{0.763773in}{0.632922in}}%
\pgfpathmoveto{\pgfqpoint{0.750725in}{0.710054in}}%
\pgfpathlineto{\pgfqpoint{0.826322in}{0.731329in}}%
\pgfpathmoveto{\pgfqpoint{0.750725in}{0.710054in}}%
\pgfpathlineto{\pgfqpoint{0.715986in}{0.784595in}}%
\pgfpathmoveto{\pgfqpoint{0.631249in}{0.790338in}}%
\pgfpathlineto{\pgfqpoint{0.546727in}{0.758322in}}%
\pgfpathmoveto{\pgfqpoint{0.631249in}{0.790338in}}%
\pgfpathlineto{\pgfqpoint{0.613861in}{0.704199in}}%
\pgfpathmoveto{\pgfqpoint{0.631249in}{0.790338in}}%
\pgfpathlineto{\pgfqpoint{0.668050in}{0.896214in}}%
\pgfpathmoveto{\pgfqpoint{0.631249in}{0.790338in}}%
\pgfpathlineto{\pgfqpoint{0.715986in}{0.784595in}}%
\pgfpathmoveto{\pgfqpoint{0.565758in}{0.964836in}}%
\pgfpathlineto{\pgfqpoint{0.451098in}{1.085600in}}%
\pgfpathmoveto{\pgfqpoint{0.565758in}{0.964836in}}%
\pgfpathlineto{\pgfqpoint{0.496062in}{0.835205in}}%
\pgfpathmoveto{\pgfqpoint{0.565758in}{0.964836in}}%
\pgfpathlineto{\pgfqpoint{0.721174in}{1.085600in}}%
\pgfpathmoveto{\pgfqpoint{0.565758in}{0.964836in}}%
\pgfpathlineto{\pgfqpoint{0.668050in}{0.896214in}}%
\pgfpathmoveto{\pgfqpoint{1.024300in}{0.661701in}}%
\pgfpathlineto{\pgfqpoint{0.990080in}{0.598631in}}%
\pgfpathmoveto{\pgfqpoint{1.024300in}{0.661701in}}%
\pgfpathlineto{\pgfqpoint{1.063069in}{0.606844in}}%
\pgfpathmoveto{\pgfqpoint{1.024300in}{0.661701in}}%
\pgfpathlineto{\pgfqpoint{0.961724in}{0.757684in}}%
\pgfpathmoveto{\pgfqpoint{1.024300in}{0.661701in}}%
\pgfpathlineto{\pgfqpoint{1.087918in}{0.720099in}}%
\pgfpathmoveto{\pgfqpoint{1.024300in}{0.661701in}}%
\pgfpathlineto{\pgfqpoint{0.955632in}{0.657941in}}%
\pgfpathmoveto{\pgfqpoint{0.887232in}{0.667312in}}%
\pgfpathlineto{\pgfqpoint{0.842615in}{0.614483in}}%
\pgfpathmoveto{\pgfqpoint{0.887232in}{0.667312in}}%
\pgfpathlineto{\pgfqpoint{0.916422in}{0.601060in}}%
\pgfpathmoveto{\pgfqpoint{0.887232in}{0.667312in}}%
\pgfpathlineto{\pgfqpoint{0.961724in}{0.757684in}}%
\pgfpathmoveto{\pgfqpoint{0.887232in}{0.667312in}}%
\pgfpathlineto{\pgfqpoint{0.826322in}{0.731329in}}%
\pgfpathmoveto{\pgfqpoint{0.887232in}{0.667312in}}%
\pgfpathlineto{\pgfqpoint{0.955632in}{0.657941in}}%
\pgfpathmoveto{\pgfqpoint{1.359582in}{0.777756in}}%
\pgfpathlineto{\pgfqpoint{1.366701in}{0.697571in}}%
\pgfpathmoveto{\pgfqpoint{1.359582in}{0.777756in}}%
\pgfpathlineto{\pgfqpoint{1.435831in}{0.755419in}}%
\pgfpathmoveto{\pgfqpoint{1.359582in}{0.777756in}}%
\pgfpathlineto{\pgfqpoint{1.279672in}{0.859587in}}%
\pgfpathmoveto{\pgfqpoint{1.359582in}{0.777756in}}%
\pgfpathlineto{\pgfqpoint{1.292856in}{0.745563in}}%
\pgfpathmoveto{\pgfqpoint{1.359582in}{0.777756in}}%
\pgfpathlineto{\pgfqpoint{1.382877in}{0.853640in}}%
\pgfpathmoveto{\pgfqpoint{1.015543in}{0.908082in}}%
\pgfpathlineto{\pgfqpoint{0.991250in}{1.085600in}}%
\pgfpathmoveto{\pgfqpoint{1.015543in}{0.908082in}}%
\pgfpathlineto{\pgfqpoint{1.113897in}{0.827208in}}%
\pgfpathmoveto{\pgfqpoint{1.015543in}{0.908082in}}%
\pgfpathlineto{\pgfqpoint{0.833368in}{0.890343in}}%
\pgfpathmoveto{\pgfqpoint{1.015543in}{0.908082in}}%
\pgfpathlineto{\pgfqpoint{0.961724in}{0.757684in}}%
\pgfpathmoveto{\pgfqpoint{1.174253in}{0.955154in}}%
\pgfpathlineto{\pgfqpoint{0.991250in}{1.085600in}}%
\pgfpathmoveto{\pgfqpoint{1.174253in}{0.955154in}}%
\pgfpathlineto{\pgfqpoint{1.261326in}{1.085600in}}%
\pgfpathmoveto{\pgfqpoint{1.174253in}{0.955154in}}%
\pgfpathlineto{\pgfqpoint{1.113897in}{0.827208in}}%
\pgfpathmoveto{\pgfqpoint{1.174253in}{0.955154in}}%
\pgfpathlineto{\pgfqpoint{1.279672in}{0.859587in}}%
\pgfpathmoveto{\pgfqpoint{1.174253in}{0.955154in}}%
\pgfpathlineto{\pgfqpoint{1.369718in}{0.959100in}}%
\pgfpathmoveto{\pgfqpoint{1.174253in}{0.955154in}}%
\pgfpathlineto{\pgfqpoint{1.015543in}{0.908082in}}%
\pgfpathmoveto{\pgfqpoint{0.579353in}{0.848924in}}%
\pgfpathlineto{\pgfqpoint{0.496062in}{0.835205in}}%
\pgfpathmoveto{\pgfqpoint{0.579353in}{0.848924in}}%
\pgfpathlineto{\pgfqpoint{0.546727in}{0.758322in}}%
\pgfpathmoveto{\pgfqpoint{0.579353in}{0.848924in}}%
\pgfpathlineto{\pgfqpoint{0.668050in}{0.896214in}}%
\pgfpathmoveto{\pgfqpoint{0.579353in}{0.848924in}}%
\pgfpathlineto{\pgfqpoint{0.631249in}{0.790338in}}%
\pgfpathmoveto{\pgfqpoint{0.579353in}{0.848924in}}%
\pgfpathlineto{\pgfqpoint{0.565758in}{0.964836in}}%
\pgfpathlineto{\pgfqpoint{0.565758in}{0.964836in}}%
\pgfusepath{stroke}%
\end{pgfscope}%
\begin{pgfscope}%
\pgfpathrectangle{\pgfqpoint{0.100000in}{0.100000in}}{\pgfqpoint{1.782500in}{1.232000in}}%
\pgfusepath{clip}%
\pgfsetbuttcap%
\pgfsetroundjoin%
\definecolor{currentfill}{rgb}{0.054902,0.262745,0.486275}%
\pgfsetfillcolor{currentfill}%
\pgfsetlinewidth{1.003750pt}%
\definecolor{currentstroke}{rgb}{0.054902,0.262745,0.486275}%
\pgfsetstrokecolor{currentstroke}%
\pgfsetdash{}{0pt}%
\pgfsys@defobject{currentmarker}{\pgfqpoint{-0.018373in}{-0.018373in}}{\pgfqpoint{0.018373in}{0.018373in}}{%
\pgfpathmoveto{\pgfqpoint{0.000000in}{-0.018373in}}%
\pgfpathcurveto{\pgfqpoint{0.004873in}{-0.018373in}}{\pgfqpoint{0.009546in}{-0.016437in}}{\pgfqpoint{0.012992in}{-0.012992in}}%
\pgfpathcurveto{\pgfqpoint{0.016437in}{-0.009546in}}{\pgfqpoint{0.018373in}{-0.004873in}}{\pgfqpoint{0.018373in}{0.000000in}}%
\pgfpathcurveto{\pgfqpoint{0.018373in}{0.004873in}}{\pgfqpoint{0.016437in}{0.009546in}}{\pgfqpoint{0.012992in}{0.012992in}}%
\pgfpathcurveto{\pgfqpoint{0.009546in}{0.016437in}}{\pgfqpoint{0.004873in}{0.018373in}}{\pgfqpoint{0.000000in}{0.018373in}}%
\pgfpathcurveto{\pgfqpoint{-0.004873in}{0.018373in}}{\pgfqpoint{-0.009546in}{0.016437in}}{\pgfqpoint{-0.012992in}{0.012992in}}%
\pgfpathcurveto{\pgfqpoint{-0.016437in}{0.009546in}}{\pgfqpoint{-0.018373in}{0.004873in}}{\pgfqpoint{-0.018373in}{0.000000in}}%
\pgfpathcurveto{\pgfqpoint{-0.018373in}{-0.004873in}}{\pgfqpoint{-0.016437in}{-0.009546in}}{\pgfqpoint{-0.012992in}{-0.012992in}}%
\pgfpathcurveto{\pgfqpoint{-0.009546in}{-0.016437in}}{\pgfqpoint{-0.004873in}{-0.018373in}}{\pgfqpoint{0.000000in}{-0.018373in}}%
\pgfpathlineto{\pgfqpoint{0.000000in}{-0.018373in}}%
\pgfpathclose%
\pgfusepath{stroke,fill}%
}%
\begin{pgfscope}%
\pgfsys@transformshift{1.377693in}{0.703727in}%
\pgfsys@useobject{currentmarker}{}%
\end{pgfscope}%
\begin{pgfscope}%
\pgfsys@transformshift{0.602566in}{0.711351in}%
\pgfsys@useobject{currentmarker}{}%
\end{pgfscope}%
\begin{pgfscope}%
\pgfsys@transformshift{1.333507in}{0.678982in}%
\pgfsys@useobject{currentmarker}{}%
\end{pgfscope}%
\begin{pgfscope}%
\pgfsys@transformshift{1.286136in}{0.655839in}%
\pgfsys@useobject{currentmarker}{}%
\end{pgfscope}%
\begin{pgfscope}%
\pgfsys@transformshift{1.234900in}{0.638023in}%
\pgfsys@useobject{currentmarker}{}%
\end{pgfscope}%
\begin{pgfscope}%
\pgfsys@transformshift{1.180840in}{0.624920in}%
\pgfsys@useobject{currentmarker}{}%
\end{pgfscope}%
\begin{pgfscope}%
\pgfsys@transformshift{1.119285in}{0.614311in}%
\pgfsys@useobject{currentmarker}{}%
\end{pgfscope}%
\begin{pgfscope}%
\pgfsys@transformshift{1.054943in}{0.605752in}%
\pgfsys@useobject{currentmarker}{}%
\end{pgfscope}%
\begin{pgfscope}%
\pgfsys@transformshift{0.990043in}{0.598875in}%
\pgfsys@useobject{currentmarker}{}%
\end{pgfscope}%
\begin{pgfscope}%
\pgfsys@transformshift{0.924598in}{0.600413in}%
\pgfsys@useobject{currentmarker}{}%
\end{pgfscope}%
\begin{pgfscope}%
\pgfsys@transformshift{0.861703in}{0.610513in}%
\pgfsys@useobject{currentmarker}{}%
\end{pgfscope}%
\begin{pgfscope}%
\pgfsys@transformshift{0.800345in}{0.623839in}%
\pgfsys@useobject{currentmarker}{}%
\end{pgfscope}%
\begin{pgfscope}%
\pgfsys@transformshift{0.746146in}{0.638036in}%
\pgfsys@useobject{currentmarker}{}%
\end{pgfscope}%
\begin{pgfscope}%
\pgfsys@transformshift{0.695822in}{0.656509in}%
\pgfsys@useobject{currentmarker}{}%
\end{pgfscope}%
\begin{pgfscope}%
\pgfsys@transformshift{0.647969in}{0.682601in}%
\pgfsys@useobject{currentmarker}{}%
\end{pgfscope}%
\end{pgfscope}%
\begin{pgfscope}%
\pgfpathrectangle{\pgfqpoint{0.100000in}{0.100000in}}{\pgfqpoint{1.782500in}{1.232000in}}%
\pgfusepath{clip}%
\pgfsetbuttcap%
\pgfsetroundjoin%
\definecolor{currentfill}{rgb}{0.835294,0.321569,0.035294}%
\pgfsetfillcolor{currentfill}%
\pgfsetlinewidth{1.003750pt}%
\definecolor{currentstroke}{rgb}{0.835294,0.321569,0.035294}%
\pgfsetstrokecolor{currentstroke}%
\pgfsetdash{}{0pt}%
\pgfsys@defobject{currentmarker}{\pgfqpoint{-0.018373in}{-0.018373in}}{\pgfqpoint{0.018373in}{0.018373in}}{%
\pgfpathmoveto{\pgfqpoint{0.000000in}{-0.018373in}}%
\pgfpathcurveto{\pgfqpoint{0.004873in}{-0.018373in}}{\pgfqpoint{0.009546in}{-0.016437in}}{\pgfqpoint{0.012992in}{-0.012992in}}%
\pgfpathcurveto{\pgfqpoint{0.016437in}{-0.009546in}}{\pgfqpoint{0.018373in}{-0.004873in}}{\pgfqpoint{0.018373in}{0.000000in}}%
\pgfpathcurveto{\pgfqpoint{0.018373in}{0.004873in}}{\pgfqpoint{0.016437in}{0.009546in}}{\pgfqpoint{0.012992in}{0.012992in}}%
\pgfpathcurveto{\pgfqpoint{0.009546in}{0.016437in}}{\pgfqpoint{0.004873in}{0.018373in}}{\pgfqpoint{0.000000in}{0.018373in}}%
\pgfpathcurveto{\pgfqpoint{-0.004873in}{0.018373in}}{\pgfqpoint{-0.009546in}{0.016437in}}{\pgfqpoint{-0.012992in}{0.012992in}}%
\pgfpathcurveto{\pgfqpoint{-0.016437in}{0.009546in}}{\pgfqpoint{-0.018373in}{0.004873in}}{\pgfqpoint{-0.018373in}{0.000000in}}%
\pgfpathcurveto{\pgfqpoint{-0.018373in}{-0.004873in}}{\pgfqpoint{-0.016437in}{-0.009546in}}{\pgfqpoint{-0.012992in}{-0.012992in}}%
\pgfpathcurveto{\pgfqpoint{-0.009546in}{-0.016437in}}{\pgfqpoint{-0.004873in}{-0.018373in}}{\pgfqpoint{0.000000in}{-0.018373in}}%
\pgfpathlineto{\pgfqpoint{0.000000in}{-0.018373in}}%
\pgfpathclose%
\pgfusepath{stroke,fill}%
}%
\begin{pgfscope}%
\pgfsys@transformshift{0.613861in}{0.704199in}%
\pgfsys@useobject{currentmarker}{}%
\end{pgfscope}%
\begin{pgfscope}%
\pgfsys@transformshift{0.684747in}{0.662330in}%
\pgfsys@useobject{currentmarker}{}%
\end{pgfscope}%
\begin{pgfscope}%
\pgfsys@transformshift{0.763773in}{0.632922in}%
\pgfsys@useobject{currentmarker}{}%
\end{pgfscope}%
\begin{pgfscope}%
\pgfsys@transformshift{0.842615in}{0.614483in}%
\pgfsys@useobject{currentmarker}{}%
\end{pgfscope}%
\begin{pgfscope}%
\pgfsys@transformshift{0.916422in}{0.601060in}%
\pgfsys@useobject{currentmarker}{}%
\end{pgfscope}%
\begin{pgfscope}%
\pgfsys@transformshift{0.990080in}{0.598631in}%
\pgfsys@useobject{currentmarker}{}%
\end{pgfscope}%
\begin{pgfscope}%
\pgfsys@transformshift{1.063069in}{0.606844in}%
\pgfsys@useobject{currentmarker}{}%
\end{pgfscope}%
\begin{pgfscope}%
\pgfsys@transformshift{1.138803in}{0.617203in}%
\pgfsys@useobject{currentmarker}{}%
\end{pgfscope}%
\begin{pgfscope}%
\pgfsys@transformshift{1.217115in}{0.633115in}%
\pgfsys@useobject{currentmarker}{}%
\end{pgfscope}%
\begin{pgfscope}%
\pgfsys@transformshift{1.297142in}{0.661043in}%
\pgfsys@useobject{currentmarker}{}%
\end{pgfscope}%
\begin{pgfscope}%
\pgfsys@transformshift{1.366701in}{0.697571in}%
\pgfsys@useobject{currentmarker}{}%
\end{pgfscope}%
\end{pgfscope}%
\end{pgfpicture}%
\makeatother%
\endgroup%

        \caption{Iteration 1: Solve system}\label{fig:example-iter0-solution}
    \end{subfigure}
    \begin{subfigure}[b]{.32\linewidth}
        %% Creator: Matplotlib, PGF backend
%%
%% To include the figure in your LaTeX document, write
%%   \input{<filename>.pgf}
%%
%% Make sure the required packages are loaded in your preamble
%%   \usepackage{pgf}
%%
%% Also ensure that all the required font packages are loaded; for instance,
%% the lmodern package is sometimes necessary when using math font.
%%   \usepackage{lmodern}
%%
%% Figures using additional raster images can only be included by \input if
%% they are in the same directory as the main LaTeX file. For loading figures
%% from other directories you can use the `import` package
%%   \usepackage{import}
%%
%% and then include the figures with
%%   \import{<path to file>}{<filename>.pgf}
%%
%% Matplotlib used the following preamble
%%   
%%   \usepackage{fontspec}
%%   \setmainfont{DejaVuSans.ttf}[Path=\detokenize{/home/fabio/Internodes-CM/.venv/lib/python3.8/site-packages/matplotlib/mpl-data/fonts/ttf/}]
%%   \setsansfont{DejaVuSans.ttf}[Path=\detokenize{/home/fabio/Internodes-CM/.venv/lib/python3.8/site-packages/matplotlib/mpl-data/fonts/ttf/}]
%%   \setmonofont{DejaVuSansMono.ttf}[Path=\detokenize{/home/fabio/Internodes-CM/.venv/lib/python3.8/site-packages/matplotlib/mpl-data/fonts/ttf/}]
%%   \makeatletter\@ifpackageloaded{underscore}{}{\usepackage[strings]{underscore}}\makeatother
%%
\begingroup%
\makeatletter%
\begin{pgfpicture}%
\pgfpathrectangle{\pgfpointorigin}{\pgfqpoint{1.982500in}{1.432000in}}%
\pgfusepath{use as bounding box, clip}%
\begin{pgfscope}%
\pgfsetbuttcap%
\pgfsetmiterjoin%
\definecolor{currentfill}{rgb}{1.000000,1.000000,1.000000}%
\pgfsetfillcolor{currentfill}%
\pgfsetlinewidth{0.000000pt}%
\definecolor{currentstroke}{rgb}{1.000000,1.000000,1.000000}%
\pgfsetstrokecolor{currentstroke}%
\pgfsetdash{}{0pt}%
\pgfpathmoveto{\pgfqpoint{0.000000in}{0.000000in}}%
\pgfpathlineto{\pgfqpoint{1.982500in}{0.000000in}}%
\pgfpathlineto{\pgfqpoint{1.982500in}{1.432000in}}%
\pgfpathlineto{\pgfqpoint{0.000000in}{1.432000in}}%
\pgfpathlineto{\pgfqpoint{0.000000in}{0.000000in}}%
\pgfpathclose%
\pgfusepath{fill}%
\end{pgfscope}%
\begin{pgfscope}%
\pgfpathrectangle{\pgfqpoint{0.100000in}{0.100000in}}{\pgfqpoint{1.782500in}{1.232000in}}%
\pgfusepath{clip}%
\pgfsetrectcap%
\pgfsetroundjoin%
\pgfsetlinewidth{0.250937pt}%
\definecolor{currentstroke}{rgb}{0.054902,0.262745,0.486275}%
\pgfsetstrokecolor{currentstroke}%
\pgfsetdash{}{0pt}%
\pgfpathmoveto{\pgfqpoint{0.451098in}{0.100000in}}%
\pgfpathlineto{\pgfqpoint{0.181023in}{0.100000in}}%
\pgfpathmoveto{\pgfqpoint{0.721174in}{0.100000in}}%
\pgfpathlineto{\pgfqpoint{0.451098in}{0.100000in}}%
\pgfpathmoveto{\pgfqpoint{0.991250in}{0.100000in}}%
\pgfpathlineto{\pgfqpoint{0.721174in}{0.100000in}}%
\pgfpathmoveto{\pgfqpoint{1.261326in}{0.100000in}}%
\pgfpathlineto{\pgfqpoint{0.991250in}{0.100000in}}%
\pgfpathmoveto{\pgfqpoint{1.531402in}{0.100000in}}%
\pgfpathlineto{\pgfqpoint{1.801477in}{0.100000in}}%
\pgfpathmoveto{\pgfqpoint{1.531402in}{0.100000in}}%
\pgfpathlineto{\pgfqpoint{1.261326in}{0.100000in}}%
\pgfpathmoveto{\pgfqpoint{1.784788in}{0.431960in}}%
\pgfpathlineto{\pgfqpoint{1.801477in}{0.100000in}}%
\pgfpathmoveto{\pgfqpoint{1.784788in}{0.431960in}}%
\pgfpathlineto{\pgfqpoint{1.740685in}{0.754365in}}%
\pgfpathmoveto{\pgfqpoint{1.575146in}{0.731227in}}%
\pgfpathlineto{\pgfqpoint{1.740685in}{0.754365in}}%
\pgfpathmoveto{\pgfqpoint{1.575146in}{0.731227in}}%
\pgfpathlineto{\pgfqpoint{1.377693in}{0.703727in}}%
\pgfpathmoveto{\pgfqpoint{1.333507in}{0.678982in}}%
\pgfpathlineto{\pgfqpoint{1.377693in}{0.703727in}}%
\pgfpathmoveto{\pgfqpoint{1.286136in}{0.655839in}}%
\pgfpathlineto{\pgfqpoint{1.333507in}{0.678982in}}%
\pgfpathmoveto{\pgfqpoint{1.234900in}{0.638023in}}%
\pgfpathlineto{\pgfqpoint{1.286136in}{0.655839in}}%
\pgfpathmoveto{\pgfqpoint{1.180840in}{0.624920in}}%
\pgfpathlineto{\pgfqpoint{1.234900in}{0.638023in}}%
\pgfpathmoveto{\pgfqpoint{1.119285in}{0.614311in}}%
\pgfpathlineto{\pgfqpoint{1.180840in}{0.624920in}}%
\pgfpathmoveto{\pgfqpoint{1.054943in}{0.605752in}}%
\pgfpathlineto{\pgfqpoint{1.119285in}{0.614311in}}%
\pgfpathmoveto{\pgfqpoint{0.990043in}{0.598875in}}%
\pgfpathlineto{\pgfqpoint{1.054943in}{0.605752in}}%
\pgfpathmoveto{\pgfqpoint{0.924598in}{0.600413in}}%
\pgfpathlineto{\pgfqpoint{0.990043in}{0.598875in}}%
\pgfpathmoveto{\pgfqpoint{0.861703in}{0.610513in}}%
\pgfpathlineto{\pgfqpoint{0.924598in}{0.600413in}}%
\pgfpathmoveto{\pgfqpoint{0.800345in}{0.623839in}}%
\pgfpathlineto{\pgfqpoint{0.861703in}{0.610513in}}%
\pgfpathmoveto{\pgfqpoint{0.746146in}{0.638036in}}%
\pgfpathlineto{\pgfqpoint{0.800345in}{0.623839in}}%
\pgfpathmoveto{\pgfqpoint{0.695822in}{0.656509in}}%
\pgfpathlineto{\pgfqpoint{0.746146in}{0.638036in}}%
\pgfpathmoveto{\pgfqpoint{0.647969in}{0.682601in}}%
\pgfpathlineto{\pgfqpoint{0.602566in}{0.711351in}}%
\pgfpathmoveto{\pgfqpoint{0.647969in}{0.682601in}}%
\pgfpathlineto{\pgfqpoint{0.695822in}{0.656509in}}%
\pgfpathmoveto{\pgfqpoint{0.407330in}{0.734379in}}%
\pgfpathlineto{\pgfqpoint{0.602566in}{0.711351in}}%
\pgfpathmoveto{\pgfqpoint{0.407330in}{0.734379in}}%
\pgfpathlineto{\pgfqpoint{0.241845in}{0.758819in}}%
\pgfpathmoveto{\pgfqpoint{0.197059in}{0.435954in}}%
\pgfpathlineto{\pgfqpoint{0.181023in}{0.100000in}}%
\pgfpathmoveto{\pgfqpoint{0.197059in}{0.435954in}}%
\pgfpathlineto{\pgfqpoint{0.241845in}{0.758819in}}%
\pgfpathmoveto{\pgfqpoint{1.370566in}{0.368055in}}%
\pgfpathlineto{\pgfqpoint{1.261326in}{0.100000in}}%
\pgfpathmoveto{\pgfqpoint{1.081725in}{0.349008in}}%
\pgfpathlineto{\pgfqpoint{0.991250in}{0.100000in}}%
\pgfpathmoveto{\pgfqpoint{1.223541in}{0.458479in}}%
\pgfpathlineto{\pgfqpoint{1.370566in}{0.368055in}}%
\pgfpathmoveto{\pgfqpoint{1.223541in}{0.458479in}}%
\pgfpathlineto{\pgfqpoint{1.081725in}{0.349008in}}%
\pgfpathmoveto{\pgfqpoint{0.956373in}{0.443934in}}%
\pgfpathlineto{\pgfqpoint{1.081725in}{0.349008in}}%
\pgfpathmoveto{\pgfqpoint{0.956373in}{0.443934in}}%
\pgfpathlineto{\pgfqpoint{0.821549in}{0.363441in}}%
\pgfpathmoveto{\pgfqpoint{0.690710in}{0.470697in}}%
\pgfpathlineto{\pgfqpoint{0.821549in}{0.363441in}}%
\pgfpathmoveto{\pgfqpoint{0.690710in}{0.470697in}}%
\pgfpathlineto{\pgfqpoint{0.544697in}{0.384052in}}%
\pgfpathmoveto{\pgfqpoint{1.514560in}{0.493546in}}%
\pgfpathlineto{\pgfqpoint{1.784788in}{0.431960in}}%
\pgfpathmoveto{\pgfqpoint{1.514560in}{0.493546in}}%
\pgfpathlineto{\pgfqpoint{1.370566in}{0.368055in}}%
\pgfpathmoveto{\pgfqpoint{1.354379in}{0.509631in}}%
\pgfpathlineto{\pgfqpoint{1.370566in}{0.368055in}}%
\pgfpathmoveto{\pgfqpoint{1.354379in}{0.509631in}}%
\pgfpathlineto{\pgfqpoint{1.223541in}{0.458479in}}%
\pgfpathmoveto{\pgfqpoint{1.354379in}{0.509631in}}%
\pgfpathlineto{\pgfqpoint{1.514560in}{0.493546in}}%
\pgfpathmoveto{\pgfqpoint{1.088111in}{0.476529in}}%
\pgfpathlineto{\pgfqpoint{1.081725in}{0.349008in}}%
\pgfpathmoveto{\pgfqpoint{1.088111in}{0.476529in}}%
\pgfpathlineto{\pgfqpoint{1.223541in}{0.458479in}}%
\pgfpathmoveto{\pgfqpoint{1.088111in}{0.476529in}}%
\pgfpathlineto{\pgfqpoint{0.956373in}{0.443934in}}%
\pgfpathmoveto{\pgfqpoint{0.825301in}{0.482755in}}%
\pgfpathlineto{\pgfqpoint{0.821549in}{0.363441in}}%
\pgfpathmoveto{\pgfqpoint{0.825301in}{0.482755in}}%
\pgfpathlineto{\pgfqpoint{0.956373in}{0.443934in}}%
\pgfpathmoveto{\pgfqpoint{0.825301in}{0.482755in}}%
\pgfpathlineto{\pgfqpoint{0.690710in}{0.470697in}}%
\pgfpathmoveto{\pgfqpoint{0.551323in}{0.531974in}}%
\pgfpathlineto{\pgfqpoint{0.544697in}{0.384052in}}%
\pgfpathmoveto{\pgfqpoint{0.551323in}{0.531974in}}%
\pgfpathlineto{\pgfqpoint{0.690710in}{0.470697in}}%
\pgfpathmoveto{\pgfqpoint{0.381486in}{0.510355in}}%
\pgfpathlineto{\pgfqpoint{0.241845in}{0.758819in}}%
\pgfpathmoveto{\pgfqpoint{0.381486in}{0.510355in}}%
\pgfpathlineto{\pgfqpoint{0.407330in}{0.734379in}}%
\pgfpathmoveto{\pgfqpoint{0.381486in}{0.510355in}}%
\pgfpathlineto{\pgfqpoint{0.197059in}{0.435954in}}%
\pgfpathmoveto{\pgfqpoint{0.381486in}{0.510355in}}%
\pgfpathlineto{\pgfqpoint{0.544697in}{0.384052in}}%
\pgfpathmoveto{\pgfqpoint{0.381486in}{0.510355in}}%
\pgfpathlineto{\pgfqpoint{0.551323in}{0.531974in}}%
\pgfpathmoveto{\pgfqpoint{1.276566in}{0.559152in}}%
\pgfpathlineto{\pgfqpoint{1.286136in}{0.655839in}}%
\pgfpathmoveto{\pgfqpoint{1.276566in}{0.559152in}}%
\pgfpathlineto{\pgfqpoint{1.234900in}{0.638023in}}%
\pgfpathmoveto{\pgfqpoint{1.276566in}{0.559152in}}%
\pgfpathlineto{\pgfqpoint{1.223541in}{0.458479in}}%
\pgfpathmoveto{\pgfqpoint{1.276566in}{0.559152in}}%
\pgfpathlineto{\pgfqpoint{1.354379in}{0.509631in}}%
\pgfpathmoveto{\pgfqpoint{1.152718in}{0.539270in}}%
\pgfpathlineto{\pgfqpoint{1.180840in}{0.624920in}}%
\pgfpathmoveto{\pgfqpoint{1.152718in}{0.539270in}}%
\pgfpathlineto{\pgfqpoint{1.119285in}{0.614311in}}%
\pgfpathmoveto{\pgfqpoint{1.152718in}{0.539270in}}%
\pgfpathlineto{\pgfqpoint{1.223541in}{0.458479in}}%
\pgfpathmoveto{\pgfqpoint{1.152718in}{0.539270in}}%
\pgfpathlineto{\pgfqpoint{1.088111in}{0.476529in}}%
\pgfpathmoveto{\pgfqpoint{1.023657in}{0.527880in}}%
\pgfpathlineto{\pgfqpoint{1.054943in}{0.605752in}}%
\pgfpathmoveto{\pgfqpoint{1.023657in}{0.527880in}}%
\pgfpathlineto{\pgfqpoint{0.990043in}{0.598875in}}%
\pgfpathmoveto{\pgfqpoint{1.023657in}{0.527880in}}%
\pgfpathlineto{\pgfqpoint{0.956373in}{0.443934in}}%
\pgfpathmoveto{\pgfqpoint{1.023657in}{0.527880in}}%
\pgfpathlineto{\pgfqpoint{1.088111in}{0.476529in}}%
\pgfpathmoveto{\pgfqpoint{0.891682in}{0.530129in}}%
\pgfpathlineto{\pgfqpoint{0.924598in}{0.600413in}}%
\pgfpathmoveto{\pgfqpoint{0.891682in}{0.530129in}}%
\pgfpathlineto{\pgfqpoint{0.861703in}{0.610513in}}%
\pgfpathmoveto{\pgfqpoint{0.891682in}{0.530129in}}%
\pgfpathlineto{\pgfqpoint{0.956373in}{0.443934in}}%
\pgfpathmoveto{\pgfqpoint{0.891682in}{0.530129in}}%
\pgfpathlineto{\pgfqpoint{0.825301in}{0.482755in}}%
\pgfpathmoveto{\pgfqpoint{0.765740in}{0.547058in}}%
\pgfpathlineto{\pgfqpoint{0.800345in}{0.623839in}}%
\pgfpathmoveto{\pgfqpoint{0.765740in}{0.547058in}}%
\pgfpathlineto{\pgfqpoint{0.746146in}{0.638036in}}%
\pgfpathmoveto{\pgfqpoint{0.765740in}{0.547058in}}%
\pgfpathlineto{\pgfqpoint{0.690710in}{0.470697in}}%
\pgfpathmoveto{\pgfqpoint{0.765740in}{0.547058in}}%
\pgfpathlineto{\pgfqpoint{0.825301in}{0.482755in}}%
\pgfpathmoveto{\pgfqpoint{0.644521in}{0.578018in}}%
\pgfpathlineto{\pgfqpoint{0.695822in}{0.656509in}}%
\pgfpathmoveto{\pgfqpoint{0.644521in}{0.578018in}}%
\pgfpathlineto{\pgfqpoint{0.647969in}{0.682601in}}%
\pgfpathmoveto{\pgfqpoint{0.644521in}{0.578018in}}%
\pgfpathlineto{\pgfqpoint{0.690710in}{0.470697in}}%
\pgfpathmoveto{\pgfqpoint{0.644521in}{0.578018in}}%
\pgfpathlineto{\pgfqpoint{0.551323in}{0.531974in}}%
\pgfpathmoveto{\pgfqpoint{1.407587in}{0.599878in}}%
\pgfpathlineto{\pgfqpoint{1.377693in}{0.703727in}}%
\pgfpathmoveto{\pgfqpoint{1.407587in}{0.599878in}}%
\pgfpathlineto{\pgfqpoint{1.333507in}{0.678982in}}%
\pgfpathmoveto{\pgfqpoint{1.407587in}{0.599878in}}%
\pgfpathlineto{\pgfqpoint{1.514560in}{0.493546in}}%
\pgfpathmoveto{\pgfqpoint{1.407587in}{0.599878in}}%
\pgfpathlineto{\pgfqpoint{1.354379in}{0.509631in}}%
\pgfpathmoveto{\pgfqpoint{0.938577in}{0.290022in}}%
\pgfpathlineto{\pgfqpoint{0.991250in}{0.100000in}}%
\pgfpathmoveto{\pgfqpoint{0.938577in}{0.290022in}}%
\pgfpathlineto{\pgfqpoint{1.081725in}{0.349008in}}%
\pgfpathmoveto{\pgfqpoint{0.938577in}{0.290022in}}%
\pgfpathlineto{\pgfqpoint{0.821549in}{0.363441in}}%
\pgfpathmoveto{\pgfqpoint{0.938577in}{0.290022in}}%
\pgfpathlineto{\pgfqpoint{0.956373in}{0.443934in}}%
\pgfpathmoveto{\pgfqpoint{0.481120in}{0.621443in}}%
\pgfpathlineto{\pgfqpoint{0.602566in}{0.711351in}}%
\pgfpathmoveto{\pgfqpoint{0.481120in}{0.621443in}}%
\pgfpathlineto{\pgfqpoint{0.407330in}{0.734379in}}%
\pgfpathmoveto{\pgfqpoint{0.481120in}{0.621443in}}%
\pgfpathlineto{\pgfqpoint{0.551323in}{0.531974in}}%
\pgfpathmoveto{\pgfqpoint{0.481120in}{0.621443in}}%
\pgfpathlineto{\pgfqpoint{0.381486in}{0.510355in}}%
\pgfpathmoveto{\pgfqpoint{1.332349in}{0.597525in}}%
\pgfpathlineto{\pgfqpoint{1.333507in}{0.678982in}}%
\pgfpathmoveto{\pgfqpoint{1.332349in}{0.597525in}}%
\pgfpathlineto{\pgfqpoint{1.286136in}{0.655839in}}%
\pgfpathmoveto{\pgfqpoint{1.332349in}{0.597525in}}%
\pgfpathlineto{\pgfqpoint{1.354379in}{0.509631in}}%
\pgfpathmoveto{\pgfqpoint{1.332349in}{0.597525in}}%
\pgfpathlineto{\pgfqpoint{1.276566in}{0.559152in}}%
\pgfpathmoveto{\pgfqpoint{1.332349in}{0.597525in}}%
\pgfpathlineto{\pgfqpoint{1.407587in}{0.599878in}}%
\pgfpathmoveto{\pgfqpoint{1.087875in}{0.552554in}}%
\pgfpathlineto{\pgfqpoint{1.119285in}{0.614311in}}%
\pgfpathmoveto{\pgfqpoint{1.087875in}{0.552554in}}%
\pgfpathlineto{\pgfqpoint{1.054943in}{0.605752in}}%
\pgfpathmoveto{\pgfqpoint{1.087875in}{0.552554in}}%
\pgfpathlineto{\pgfqpoint{1.088111in}{0.476529in}}%
\pgfpathmoveto{\pgfqpoint{1.087875in}{0.552554in}}%
\pgfpathlineto{\pgfqpoint{1.152718in}{0.539270in}}%
\pgfpathmoveto{\pgfqpoint{1.087875in}{0.552554in}}%
\pgfpathlineto{\pgfqpoint{1.023657in}{0.527880in}}%
\pgfpathmoveto{\pgfqpoint{1.213366in}{0.566671in}}%
\pgfpathlineto{\pgfqpoint{1.234900in}{0.638023in}}%
\pgfpathmoveto{\pgfqpoint{1.213366in}{0.566671in}}%
\pgfpathlineto{\pgfqpoint{1.180840in}{0.624920in}}%
\pgfpathmoveto{\pgfqpoint{1.213366in}{0.566671in}}%
\pgfpathlineto{\pgfqpoint{1.223541in}{0.458479in}}%
\pgfpathmoveto{\pgfqpoint{1.213366in}{0.566671in}}%
\pgfpathlineto{\pgfqpoint{1.276566in}{0.559152in}}%
\pgfpathmoveto{\pgfqpoint{1.213366in}{0.566671in}}%
\pgfpathlineto{\pgfqpoint{1.152718in}{0.539270in}}%
\pgfpathmoveto{\pgfqpoint{0.957452in}{0.539685in}}%
\pgfpathlineto{\pgfqpoint{0.990043in}{0.598875in}}%
\pgfpathmoveto{\pgfqpoint{0.957452in}{0.539685in}}%
\pgfpathlineto{\pgfqpoint{0.924598in}{0.600413in}}%
\pgfpathmoveto{\pgfqpoint{0.957452in}{0.539685in}}%
\pgfpathlineto{\pgfqpoint{0.956373in}{0.443934in}}%
\pgfpathmoveto{\pgfqpoint{0.957452in}{0.539685in}}%
\pgfpathlineto{\pgfqpoint{1.023657in}{0.527880in}}%
\pgfpathmoveto{\pgfqpoint{0.957452in}{0.539685in}}%
\pgfpathlineto{\pgfqpoint{0.891682in}{0.530129in}}%
\pgfpathmoveto{\pgfqpoint{0.828623in}{0.557099in}}%
\pgfpathlineto{\pgfqpoint{0.861703in}{0.610513in}}%
\pgfpathmoveto{\pgfqpoint{0.828623in}{0.557099in}}%
\pgfpathlineto{\pgfqpoint{0.800345in}{0.623839in}}%
\pgfpathmoveto{\pgfqpoint{0.828623in}{0.557099in}}%
\pgfpathlineto{\pgfqpoint{0.825301in}{0.482755in}}%
\pgfpathmoveto{\pgfqpoint{0.828623in}{0.557099in}}%
\pgfpathlineto{\pgfqpoint{0.891682in}{0.530129in}}%
\pgfpathmoveto{\pgfqpoint{0.828623in}{0.557099in}}%
\pgfpathlineto{\pgfqpoint{0.765740in}{0.547058in}}%
\pgfpathmoveto{\pgfqpoint{1.188848in}{0.271931in}}%
\pgfpathlineto{\pgfqpoint{0.991250in}{0.100000in}}%
\pgfpathmoveto{\pgfqpoint{1.188848in}{0.271931in}}%
\pgfpathlineto{\pgfqpoint{1.261326in}{0.100000in}}%
\pgfpathmoveto{\pgfqpoint{1.188848in}{0.271931in}}%
\pgfpathlineto{\pgfqpoint{1.370566in}{0.368055in}}%
\pgfpathmoveto{\pgfqpoint{1.188848in}{0.271931in}}%
\pgfpathlineto{\pgfqpoint{1.081725in}{0.349008in}}%
\pgfpathmoveto{\pgfqpoint{1.188848in}{0.271931in}}%
\pgfpathlineto{\pgfqpoint{1.223541in}{0.458479in}}%
\pgfpathmoveto{\pgfqpoint{0.693813in}{0.285882in}}%
\pgfpathlineto{\pgfqpoint{0.721174in}{0.100000in}}%
\pgfpathmoveto{\pgfqpoint{0.693813in}{0.285882in}}%
\pgfpathlineto{\pgfqpoint{0.821549in}{0.363441in}}%
\pgfpathmoveto{\pgfqpoint{0.693813in}{0.285882in}}%
\pgfpathlineto{\pgfqpoint{0.544697in}{0.384052in}}%
\pgfpathmoveto{\pgfqpoint{0.693813in}{0.285882in}}%
\pgfpathlineto{\pgfqpoint{0.690710in}{0.470697in}}%
\pgfpathmoveto{\pgfqpoint{0.707873in}{0.576304in}}%
\pgfpathlineto{\pgfqpoint{0.746146in}{0.638036in}}%
\pgfpathmoveto{\pgfqpoint{0.707873in}{0.576304in}}%
\pgfpathlineto{\pgfqpoint{0.695822in}{0.656509in}}%
\pgfpathmoveto{\pgfqpoint{0.707873in}{0.576304in}}%
\pgfpathlineto{\pgfqpoint{0.690710in}{0.470697in}}%
\pgfpathmoveto{\pgfqpoint{0.707873in}{0.576304in}}%
\pgfpathlineto{\pgfqpoint{0.765740in}{0.547058in}}%
\pgfpathmoveto{\pgfqpoint{0.707873in}{0.576304in}}%
\pgfpathlineto{\pgfqpoint{0.644521in}{0.578018in}}%
\pgfpathmoveto{\pgfqpoint{1.505766in}{0.629724in}}%
\pgfpathlineto{\pgfqpoint{1.377693in}{0.703727in}}%
\pgfpathmoveto{\pgfqpoint{1.505766in}{0.629724in}}%
\pgfpathlineto{\pgfqpoint{1.575146in}{0.731227in}}%
\pgfpathmoveto{\pgfqpoint{1.505766in}{0.629724in}}%
\pgfpathlineto{\pgfqpoint{1.514560in}{0.493546in}}%
\pgfpathmoveto{\pgfqpoint{1.505766in}{0.629724in}}%
\pgfpathlineto{\pgfqpoint{1.407587in}{0.599878in}}%
\pgfpathmoveto{\pgfqpoint{0.592313in}{0.620955in}}%
\pgfpathlineto{\pgfqpoint{0.602566in}{0.711351in}}%
\pgfpathmoveto{\pgfqpoint{0.592313in}{0.620955in}}%
\pgfpathlineto{\pgfqpoint{0.647969in}{0.682601in}}%
\pgfpathmoveto{\pgfqpoint{0.592313in}{0.620955in}}%
\pgfpathlineto{\pgfqpoint{0.551323in}{0.531974in}}%
\pgfpathmoveto{\pgfqpoint{0.592313in}{0.620955in}}%
\pgfpathlineto{\pgfqpoint{0.644521in}{0.578018in}}%
\pgfpathmoveto{\pgfqpoint{0.592313in}{0.620955in}}%
\pgfpathlineto{\pgfqpoint{0.481120in}{0.621443in}}%
\pgfpathmoveto{\pgfqpoint{0.380766in}{0.291207in}}%
\pgfpathlineto{\pgfqpoint{0.181023in}{0.100000in}}%
\pgfpathmoveto{\pgfqpoint{0.380766in}{0.291207in}}%
\pgfpathlineto{\pgfqpoint{0.451098in}{0.100000in}}%
\pgfpathmoveto{\pgfqpoint{0.380766in}{0.291207in}}%
\pgfpathlineto{\pgfqpoint{0.197059in}{0.435954in}}%
\pgfpathmoveto{\pgfqpoint{0.380766in}{0.291207in}}%
\pgfpathlineto{\pgfqpoint{0.544697in}{0.384052in}}%
\pgfpathmoveto{\pgfqpoint{0.380766in}{0.291207in}}%
\pgfpathlineto{\pgfqpoint{0.381486in}{0.510355in}}%
\pgfpathmoveto{\pgfqpoint{1.529279in}{0.294989in}}%
\pgfpathlineto{\pgfqpoint{1.801477in}{0.100000in}}%
\pgfpathmoveto{\pgfqpoint{1.529279in}{0.294989in}}%
\pgfpathlineto{\pgfqpoint{1.261326in}{0.100000in}}%
\pgfpathmoveto{\pgfqpoint{1.529279in}{0.294989in}}%
\pgfpathlineto{\pgfqpoint{1.531402in}{0.100000in}}%
\pgfpathmoveto{\pgfqpoint{1.529279in}{0.294989in}}%
\pgfpathlineto{\pgfqpoint{1.784788in}{0.431960in}}%
\pgfpathmoveto{\pgfqpoint{1.529279in}{0.294989in}}%
\pgfpathlineto{\pgfqpoint{1.370566in}{0.368055in}}%
\pgfpathmoveto{\pgfqpoint{1.529279in}{0.294989in}}%
\pgfpathlineto{\pgfqpoint{1.514560in}{0.493546in}}%
\pgfpathmoveto{\pgfqpoint{1.624417in}{0.609671in}}%
\pgfpathlineto{\pgfqpoint{1.740685in}{0.754365in}}%
\pgfpathmoveto{\pgfqpoint{1.624417in}{0.609671in}}%
\pgfpathlineto{\pgfqpoint{1.784788in}{0.431960in}}%
\pgfpathmoveto{\pgfqpoint{1.624417in}{0.609671in}}%
\pgfpathlineto{\pgfqpoint{1.575146in}{0.731227in}}%
\pgfpathmoveto{\pgfqpoint{1.624417in}{0.609671in}}%
\pgfpathlineto{\pgfqpoint{1.514560in}{0.493546in}}%
\pgfpathmoveto{\pgfqpoint{1.624417in}{0.609671in}}%
\pgfpathlineto{\pgfqpoint{1.505766in}{0.629724in}}%
\pgfpathmoveto{\pgfqpoint{0.840111in}{0.209031in}}%
\pgfpathlineto{\pgfqpoint{0.721174in}{0.100000in}}%
\pgfpathmoveto{\pgfqpoint{0.840111in}{0.209031in}}%
\pgfpathlineto{\pgfqpoint{0.991250in}{0.100000in}}%
\pgfpathmoveto{\pgfqpoint{0.840111in}{0.209031in}}%
\pgfpathlineto{\pgfqpoint{0.821549in}{0.363441in}}%
\pgfpathmoveto{\pgfqpoint{0.840111in}{0.209031in}}%
\pgfpathlineto{\pgfqpoint{0.938577in}{0.290022in}}%
\pgfpathmoveto{\pgfqpoint{0.840111in}{0.209031in}}%
\pgfpathlineto{\pgfqpoint{0.693813in}{0.285882in}}%
\pgfpathmoveto{\pgfqpoint{0.554609in}{0.230276in}}%
\pgfpathlineto{\pgfqpoint{0.451098in}{0.100000in}}%
\pgfpathmoveto{\pgfqpoint{0.554609in}{0.230276in}}%
\pgfpathlineto{\pgfqpoint{0.721174in}{0.100000in}}%
\pgfpathmoveto{\pgfqpoint{0.554609in}{0.230276in}}%
\pgfpathlineto{\pgfqpoint{0.544697in}{0.384052in}}%
\pgfpathmoveto{\pgfqpoint{0.554609in}{0.230276in}}%
\pgfpathlineto{\pgfqpoint{0.693813in}{0.285882in}}%
\pgfpathmoveto{\pgfqpoint{0.554609in}{0.230276in}}%
\pgfpathlineto{\pgfqpoint{0.380766in}{0.291207in}}%
\pgfpathlineto{\pgfqpoint{0.380766in}{0.291207in}}%
\pgfusepath{stroke}%
\end{pgfscope}%
\begin{pgfscope}%
\pgfpathrectangle{\pgfqpoint{0.100000in}{0.100000in}}{\pgfqpoint{1.782500in}{1.232000in}}%
\pgfusepath{clip}%
\pgfsetrectcap%
\pgfsetroundjoin%
\pgfsetlinewidth{0.250937pt}%
\definecolor{currentstroke}{rgb}{0.835294,0.321569,0.035294}%
\pgfsetstrokecolor{currentstroke}%
\pgfsetdash{}{0pt}%
\pgfpathmoveto{\pgfqpoint{0.496062in}{0.835205in}}%
\pgfpathlineto{\pgfqpoint{0.451098in}{1.085600in}}%
\pgfpathmoveto{\pgfqpoint{1.531402in}{1.085600in}}%
\pgfpathlineto{\pgfqpoint{1.486187in}{0.833488in}}%
\pgfpathmoveto{\pgfqpoint{0.721174in}{1.085600in}}%
\pgfpathlineto{\pgfqpoint{0.991250in}{1.085600in}}%
\pgfpathmoveto{\pgfqpoint{0.721174in}{1.085600in}}%
\pgfpathlineto{\pgfqpoint{0.451098in}{1.085600in}}%
\pgfpathmoveto{\pgfqpoint{0.546727in}{0.758322in}}%
\pgfpathlineto{\pgfqpoint{0.496062in}{0.835205in}}%
\pgfpathmoveto{\pgfqpoint{0.613861in}{0.704199in}}%
\pgfpathlineto{\pgfqpoint{0.546727in}{0.758322in}}%
\pgfpathmoveto{\pgfqpoint{0.684747in}{0.662330in}}%
\pgfpathlineto{\pgfqpoint{0.613861in}{0.704199in}}%
\pgfpathmoveto{\pgfqpoint{0.763773in}{0.632922in}}%
\pgfpathlineto{\pgfqpoint{0.684747in}{0.662330in}}%
\pgfpathmoveto{\pgfqpoint{0.842615in}{0.614483in}}%
\pgfpathlineto{\pgfqpoint{0.763773in}{0.632922in}}%
\pgfpathmoveto{\pgfqpoint{0.916422in}{0.601060in}}%
\pgfpathlineto{\pgfqpoint{0.842615in}{0.614483in}}%
\pgfpathmoveto{\pgfqpoint{0.990080in}{0.598631in}}%
\pgfpathlineto{\pgfqpoint{0.916422in}{0.601060in}}%
\pgfpathmoveto{\pgfqpoint{1.063069in}{0.606844in}}%
\pgfpathlineto{\pgfqpoint{0.990080in}{0.598631in}}%
\pgfpathmoveto{\pgfqpoint{1.138803in}{0.617203in}}%
\pgfpathlineto{\pgfqpoint{1.063069in}{0.606844in}}%
\pgfpathmoveto{\pgfqpoint{1.217115in}{0.633115in}}%
\pgfpathlineto{\pgfqpoint{1.138803in}{0.617203in}}%
\pgfpathmoveto{\pgfqpoint{1.297142in}{0.661043in}}%
\pgfpathlineto{\pgfqpoint{1.217115in}{0.633115in}}%
\pgfpathmoveto{\pgfqpoint{1.366701in}{0.697571in}}%
\pgfpathlineto{\pgfqpoint{1.297142in}{0.661043in}}%
\pgfpathmoveto{\pgfqpoint{1.435831in}{0.755419in}}%
\pgfpathlineto{\pgfqpoint{1.486187in}{0.833488in}}%
\pgfpathmoveto{\pgfqpoint{1.435831in}{0.755419in}}%
\pgfpathlineto{\pgfqpoint{1.366701in}{0.697571in}}%
\pgfpathmoveto{\pgfqpoint{1.261326in}{1.085600in}}%
\pgfpathlineto{\pgfqpoint{0.991250in}{1.085600in}}%
\pgfpathmoveto{\pgfqpoint{1.261326in}{1.085600in}}%
\pgfpathlineto{\pgfqpoint{1.531402in}{1.085600in}}%
\pgfpathmoveto{\pgfqpoint{0.833368in}{0.890343in}}%
\pgfpathlineto{\pgfqpoint{0.991250in}{1.085600in}}%
\pgfpathmoveto{\pgfqpoint{0.833368in}{0.890343in}}%
\pgfpathlineto{\pgfqpoint{0.721174in}{1.085600in}}%
\pgfpathmoveto{\pgfqpoint{0.961724in}{0.757684in}}%
\pgfpathlineto{\pgfqpoint{1.113897in}{0.827208in}}%
\pgfpathmoveto{\pgfqpoint{0.961724in}{0.757684in}}%
\pgfpathlineto{\pgfqpoint{0.833368in}{0.890343in}}%
\pgfpathmoveto{\pgfqpoint{1.279672in}{0.859587in}}%
\pgfpathlineto{\pgfqpoint{1.113897in}{0.827208in}}%
\pgfpathmoveto{\pgfqpoint{0.668050in}{0.896214in}}%
\pgfpathlineto{\pgfqpoint{0.721174in}{1.085600in}}%
\pgfpathmoveto{\pgfqpoint{0.668050in}{0.896214in}}%
\pgfpathlineto{\pgfqpoint{0.833368in}{0.890343in}}%
\pgfpathmoveto{\pgfqpoint{1.087918in}{0.720099in}}%
\pgfpathlineto{\pgfqpoint{1.063069in}{0.606844in}}%
\pgfpathmoveto{\pgfqpoint{1.087918in}{0.720099in}}%
\pgfpathlineto{\pgfqpoint{1.138803in}{0.617203in}}%
\pgfpathmoveto{\pgfqpoint{1.087918in}{0.720099in}}%
\pgfpathlineto{\pgfqpoint{1.113897in}{0.827208in}}%
\pgfpathmoveto{\pgfqpoint{1.087918in}{0.720099in}}%
\pgfpathlineto{\pgfqpoint{0.961724in}{0.757684in}}%
\pgfpathmoveto{\pgfqpoint{1.208497in}{0.752670in}}%
\pgfpathlineto{\pgfqpoint{1.217115in}{0.633115in}}%
\pgfpathmoveto{\pgfqpoint{1.208497in}{0.752670in}}%
\pgfpathlineto{\pgfqpoint{1.297142in}{0.661043in}}%
\pgfpathmoveto{\pgfqpoint{1.208497in}{0.752670in}}%
\pgfpathlineto{\pgfqpoint{1.113897in}{0.827208in}}%
\pgfpathmoveto{\pgfqpoint{1.208497in}{0.752670in}}%
\pgfpathlineto{\pgfqpoint{1.279672in}{0.859587in}}%
\pgfpathmoveto{\pgfqpoint{1.208497in}{0.752670in}}%
\pgfpathlineto{\pgfqpoint{1.087918in}{0.720099in}}%
\pgfpathmoveto{\pgfqpoint{0.826322in}{0.731329in}}%
\pgfpathlineto{\pgfqpoint{0.763773in}{0.632922in}}%
\pgfpathmoveto{\pgfqpoint{0.826322in}{0.731329in}}%
\pgfpathlineto{\pgfqpoint{0.842615in}{0.614483in}}%
\pgfpathmoveto{\pgfqpoint{0.826322in}{0.731329in}}%
\pgfpathlineto{\pgfqpoint{0.833368in}{0.890343in}}%
\pgfpathmoveto{\pgfqpoint{0.826322in}{0.731329in}}%
\pgfpathlineto{\pgfqpoint{0.961724in}{0.757684in}}%
\pgfpathmoveto{\pgfqpoint{0.715986in}{0.784595in}}%
\pgfpathlineto{\pgfqpoint{0.613861in}{0.704199in}}%
\pgfpathmoveto{\pgfqpoint{0.715986in}{0.784595in}}%
\pgfpathlineto{\pgfqpoint{0.684747in}{0.662330in}}%
\pgfpathmoveto{\pgfqpoint{0.715986in}{0.784595in}}%
\pgfpathlineto{\pgfqpoint{0.833368in}{0.890343in}}%
\pgfpathmoveto{\pgfqpoint{0.715986in}{0.784595in}}%
\pgfpathlineto{\pgfqpoint{0.668050in}{0.896214in}}%
\pgfpathmoveto{\pgfqpoint{0.715986in}{0.784595in}}%
\pgfpathlineto{\pgfqpoint{0.826322in}{0.731329in}}%
\pgfpathmoveto{\pgfqpoint{1.369718in}{0.959100in}}%
\pgfpathlineto{\pgfqpoint{1.486187in}{0.833488in}}%
\pgfpathmoveto{\pgfqpoint{1.369718in}{0.959100in}}%
\pgfpathlineto{\pgfqpoint{1.531402in}{1.085600in}}%
\pgfpathmoveto{\pgfqpoint{1.369718in}{0.959100in}}%
\pgfpathlineto{\pgfqpoint{1.261326in}{1.085600in}}%
\pgfpathmoveto{\pgfqpoint{1.369718in}{0.959100in}}%
\pgfpathlineto{\pgfqpoint{1.279672in}{0.859587in}}%
\pgfpathmoveto{\pgfqpoint{0.955632in}{0.657941in}}%
\pgfpathlineto{\pgfqpoint{0.916422in}{0.601060in}}%
\pgfpathmoveto{\pgfqpoint{0.955632in}{0.657941in}}%
\pgfpathlineto{\pgfqpoint{0.990080in}{0.598631in}}%
\pgfpathmoveto{\pgfqpoint{0.955632in}{0.657941in}}%
\pgfpathlineto{\pgfqpoint{0.961724in}{0.757684in}}%
\pgfpathmoveto{\pgfqpoint{1.292856in}{0.745563in}}%
\pgfpathlineto{\pgfqpoint{1.297142in}{0.661043in}}%
\pgfpathmoveto{\pgfqpoint{1.292856in}{0.745563in}}%
\pgfpathlineto{\pgfqpoint{1.366701in}{0.697571in}}%
\pgfpathmoveto{\pgfqpoint{1.292856in}{0.745563in}}%
\pgfpathlineto{\pgfqpoint{1.279672in}{0.859587in}}%
\pgfpathmoveto{\pgfqpoint{1.292856in}{0.745563in}}%
\pgfpathlineto{\pgfqpoint{1.208497in}{0.752670in}}%
\pgfpathmoveto{\pgfqpoint{1.161401in}{0.688685in}}%
\pgfpathlineto{\pgfqpoint{1.138803in}{0.617203in}}%
\pgfpathmoveto{\pgfqpoint{1.161401in}{0.688685in}}%
\pgfpathlineto{\pgfqpoint{1.217115in}{0.633115in}}%
\pgfpathmoveto{\pgfqpoint{1.161401in}{0.688685in}}%
\pgfpathlineto{\pgfqpoint{1.087918in}{0.720099in}}%
\pgfpathmoveto{\pgfqpoint{1.161401in}{0.688685in}}%
\pgfpathlineto{\pgfqpoint{1.208497in}{0.752670in}}%
\pgfpathmoveto{\pgfqpoint{1.382877in}{0.853640in}}%
\pgfpathlineto{\pgfqpoint{1.486187in}{0.833488in}}%
\pgfpathmoveto{\pgfqpoint{1.382877in}{0.853640in}}%
\pgfpathlineto{\pgfqpoint{1.435831in}{0.755419in}}%
\pgfpathmoveto{\pgfqpoint{1.382877in}{0.853640in}}%
\pgfpathlineto{\pgfqpoint{1.279672in}{0.859587in}}%
\pgfpathmoveto{\pgfqpoint{1.382877in}{0.853640in}}%
\pgfpathlineto{\pgfqpoint{1.369718in}{0.959100in}}%
\pgfpathmoveto{\pgfqpoint{0.750725in}{0.710054in}}%
\pgfpathlineto{\pgfqpoint{0.684747in}{0.662330in}}%
\pgfpathmoveto{\pgfqpoint{0.750725in}{0.710054in}}%
\pgfpathlineto{\pgfqpoint{0.763773in}{0.632922in}}%
\pgfpathmoveto{\pgfqpoint{0.750725in}{0.710054in}}%
\pgfpathlineto{\pgfqpoint{0.826322in}{0.731329in}}%
\pgfpathmoveto{\pgfqpoint{0.750725in}{0.710054in}}%
\pgfpathlineto{\pgfqpoint{0.715986in}{0.784595in}}%
\pgfpathmoveto{\pgfqpoint{0.631249in}{0.790338in}}%
\pgfpathlineto{\pgfqpoint{0.546727in}{0.758322in}}%
\pgfpathmoveto{\pgfqpoint{0.631249in}{0.790338in}}%
\pgfpathlineto{\pgfqpoint{0.613861in}{0.704199in}}%
\pgfpathmoveto{\pgfqpoint{0.631249in}{0.790338in}}%
\pgfpathlineto{\pgfqpoint{0.668050in}{0.896214in}}%
\pgfpathmoveto{\pgfqpoint{0.631249in}{0.790338in}}%
\pgfpathlineto{\pgfqpoint{0.715986in}{0.784595in}}%
\pgfpathmoveto{\pgfqpoint{0.565758in}{0.964836in}}%
\pgfpathlineto{\pgfqpoint{0.451098in}{1.085600in}}%
\pgfpathmoveto{\pgfqpoint{0.565758in}{0.964836in}}%
\pgfpathlineto{\pgfqpoint{0.496062in}{0.835205in}}%
\pgfpathmoveto{\pgfqpoint{0.565758in}{0.964836in}}%
\pgfpathlineto{\pgfqpoint{0.721174in}{1.085600in}}%
\pgfpathmoveto{\pgfqpoint{0.565758in}{0.964836in}}%
\pgfpathlineto{\pgfqpoint{0.668050in}{0.896214in}}%
\pgfpathmoveto{\pgfqpoint{1.024300in}{0.661701in}}%
\pgfpathlineto{\pgfqpoint{0.990080in}{0.598631in}}%
\pgfpathmoveto{\pgfqpoint{1.024300in}{0.661701in}}%
\pgfpathlineto{\pgfqpoint{1.063069in}{0.606844in}}%
\pgfpathmoveto{\pgfqpoint{1.024300in}{0.661701in}}%
\pgfpathlineto{\pgfqpoint{0.961724in}{0.757684in}}%
\pgfpathmoveto{\pgfqpoint{1.024300in}{0.661701in}}%
\pgfpathlineto{\pgfqpoint{1.087918in}{0.720099in}}%
\pgfpathmoveto{\pgfqpoint{1.024300in}{0.661701in}}%
\pgfpathlineto{\pgfqpoint{0.955632in}{0.657941in}}%
\pgfpathmoveto{\pgfqpoint{0.887232in}{0.667312in}}%
\pgfpathlineto{\pgfqpoint{0.842615in}{0.614483in}}%
\pgfpathmoveto{\pgfqpoint{0.887232in}{0.667312in}}%
\pgfpathlineto{\pgfqpoint{0.916422in}{0.601060in}}%
\pgfpathmoveto{\pgfqpoint{0.887232in}{0.667312in}}%
\pgfpathlineto{\pgfqpoint{0.961724in}{0.757684in}}%
\pgfpathmoveto{\pgfqpoint{0.887232in}{0.667312in}}%
\pgfpathlineto{\pgfqpoint{0.826322in}{0.731329in}}%
\pgfpathmoveto{\pgfqpoint{0.887232in}{0.667312in}}%
\pgfpathlineto{\pgfqpoint{0.955632in}{0.657941in}}%
\pgfpathmoveto{\pgfqpoint{1.359582in}{0.777756in}}%
\pgfpathlineto{\pgfqpoint{1.366701in}{0.697571in}}%
\pgfpathmoveto{\pgfqpoint{1.359582in}{0.777756in}}%
\pgfpathlineto{\pgfqpoint{1.435831in}{0.755419in}}%
\pgfpathmoveto{\pgfqpoint{1.359582in}{0.777756in}}%
\pgfpathlineto{\pgfqpoint{1.279672in}{0.859587in}}%
\pgfpathmoveto{\pgfqpoint{1.359582in}{0.777756in}}%
\pgfpathlineto{\pgfqpoint{1.292856in}{0.745563in}}%
\pgfpathmoveto{\pgfqpoint{1.359582in}{0.777756in}}%
\pgfpathlineto{\pgfqpoint{1.382877in}{0.853640in}}%
\pgfpathmoveto{\pgfqpoint{1.015543in}{0.908082in}}%
\pgfpathlineto{\pgfqpoint{0.991250in}{1.085600in}}%
\pgfpathmoveto{\pgfqpoint{1.015543in}{0.908082in}}%
\pgfpathlineto{\pgfqpoint{1.113897in}{0.827208in}}%
\pgfpathmoveto{\pgfqpoint{1.015543in}{0.908082in}}%
\pgfpathlineto{\pgfqpoint{0.833368in}{0.890343in}}%
\pgfpathmoveto{\pgfqpoint{1.015543in}{0.908082in}}%
\pgfpathlineto{\pgfqpoint{0.961724in}{0.757684in}}%
\pgfpathmoveto{\pgfqpoint{1.174253in}{0.955154in}}%
\pgfpathlineto{\pgfqpoint{0.991250in}{1.085600in}}%
\pgfpathmoveto{\pgfqpoint{1.174253in}{0.955154in}}%
\pgfpathlineto{\pgfqpoint{1.261326in}{1.085600in}}%
\pgfpathmoveto{\pgfqpoint{1.174253in}{0.955154in}}%
\pgfpathlineto{\pgfqpoint{1.113897in}{0.827208in}}%
\pgfpathmoveto{\pgfqpoint{1.174253in}{0.955154in}}%
\pgfpathlineto{\pgfqpoint{1.279672in}{0.859587in}}%
\pgfpathmoveto{\pgfqpoint{1.174253in}{0.955154in}}%
\pgfpathlineto{\pgfqpoint{1.369718in}{0.959100in}}%
\pgfpathmoveto{\pgfqpoint{1.174253in}{0.955154in}}%
\pgfpathlineto{\pgfqpoint{1.015543in}{0.908082in}}%
\pgfpathmoveto{\pgfqpoint{0.579353in}{0.848924in}}%
\pgfpathlineto{\pgfqpoint{0.496062in}{0.835205in}}%
\pgfpathmoveto{\pgfqpoint{0.579353in}{0.848924in}}%
\pgfpathlineto{\pgfqpoint{0.546727in}{0.758322in}}%
\pgfpathmoveto{\pgfqpoint{0.579353in}{0.848924in}}%
\pgfpathlineto{\pgfqpoint{0.668050in}{0.896214in}}%
\pgfpathmoveto{\pgfqpoint{0.579353in}{0.848924in}}%
\pgfpathlineto{\pgfqpoint{0.631249in}{0.790338in}}%
\pgfpathmoveto{\pgfqpoint{0.579353in}{0.848924in}}%
\pgfpathlineto{\pgfqpoint{0.565758in}{0.964836in}}%
\pgfpathlineto{\pgfqpoint{0.565758in}{0.964836in}}%
\pgfusepath{stroke}%
\end{pgfscope}%
\begin{pgfscope}%
\pgfpathrectangle{\pgfqpoint{0.100000in}{0.100000in}}{\pgfqpoint{1.782500in}{1.232000in}}%
\pgfusepath{clip}%
\pgfsetbuttcap%
\pgfsetroundjoin%
\definecolor{currentfill}{rgb}{0.054902,0.262745,0.486275}%
\pgfsetfillcolor{currentfill}%
\pgfsetlinewidth{1.003750pt}%
\definecolor{currentstroke}{rgb}{0.054902,0.262745,0.486275}%
\pgfsetstrokecolor{currentstroke}%
\pgfsetdash{}{0pt}%
\pgfsys@defobject{currentmarker}{\pgfqpoint{-0.018373in}{-0.018373in}}{\pgfqpoint{0.018373in}{0.018373in}}{%
\pgfpathmoveto{\pgfqpoint{0.000000in}{-0.018373in}}%
\pgfpathcurveto{\pgfqpoint{0.004873in}{-0.018373in}}{\pgfqpoint{0.009546in}{-0.016437in}}{\pgfqpoint{0.012992in}{-0.012992in}}%
\pgfpathcurveto{\pgfqpoint{0.016437in}{-0.009546in}}{\pgfqpoint{0.018373in}{-0.004873in}}{\pgfqpoint{0.018373in}{0.000000in}}%
\pgfpathcurveto{\pgfqpoint{0.018373in}{0.004873in}}{\pgfqpoint{0.016437in}{0.009546in}}{\pgfqpoint{0.012992in}{0.012992in}}%
\pgfpathcurveto{\pgfqpoint{0.009546in}{0.016437in}}{\pgfqpoint{0.004873in}{0.018373in}}{\pgfqpoint{0.000000in}{0.018373in}}%
\pgfpathcurveto{\pgfqpoint{-0.004873in}{0.018373in}}{\pgfqpoint{-0.009546in}{0.016437in}}{\pgfqpoint{-0.012992in}{0.012992in}}%
\pgfpathcurveto{\pgfqpoint{-0.016437in}{0.009546in}}{\pgfqpoint{-0.018373in}{0.004873in}}{\pgfqpoint{-0.018373in}{0.000000in}}%
\pgfpathcurveto{\pgfqpoint{-0.018373in}{-0.004873in}}{\pgfqpoint{-0.016437in}{-0.009546in}}{\pgfqpoint{-0.012992in}{-0.012992in}}%
\pgfpathcurveto{\pgfqpoint{-0.009546in}{-0.016437in}}{\pgfqpoint{-0.004873in}{-0.018373in}}{\pgfqpoint{0.000000in}{-0.018373in}}%
\pgfpathlineto{\pgfqpoint{0.000000in}{-0.018373in}}%
\pgfpathclose%
\pgfusepath{stroke,fill}%
}%
\begin{pgfscope}%
\pgfsys@transformshift{1.333507in}{0.678982in}%
\pgfsys@useobject{currentmarker}{}%
\end{pgfscope}%
\begin{pgfscope}%
\pgfsys@transformshift{1.286136in}{0.655839in}%
\pgfsys@useobject{currentmarker}{}%
\end{pgfscope}%
\begin{pgfscope}%
\pgfsys@transformshift{1.234900in}{0.638023in}%
\pgfsys@useobject{currentmarker}{}%
\end{pgfscope}%
\begin{pgfscope}%
\pgfsys@transformshift{1.180840in}{0.624920in}%
\pgfsys@useobject{currentmarker}{}%
\end{pgfscope}%
\begin{pgfscope}%
\pgfsys@transformshift{1.119285in}{0.614311in}%
\pgfsys@useobject{currentmarker}{}%
\end{pgfscope}%
\begin{pgfscope}%
\pgfsys@transformshift{1.054943in}{0.605752in}%
\pgfsys@useobject{currentmarker}{}%
\end{pgfscope}%
\begin{pgfscope}%
\pgfsys@transformshift{0.990043in}{0.598875in}%
\pgfsys@useobject{currentmarker}{}%
\end{pgfscope}%
\begin{pgfscope}%
\pgfsys@transformshift{0.924598in}{0.600413in}%
\pgfsys@useobject{currentmarker}{}%
\end{pgfscope}%
\begin{pgfscope}%
\pgfsys@transformshift{0.861703in}{0.610513in}%
\pgfsys@useobject{currentmarker}{}%
\end{pgfscope}%
\begin{pgfscope}%
\pgfsys@transformshift{0.800345in}{0.623839in}%
\pgfsys@useobject{currentmarker}{}%
\end{pgfscope}%
\begin{pgfscope}%
\pgfsys@transformshift{0.746146in}{0.638036in}%
\pgfsys@useobject{currentmarker}{}%
\end{pgfscope}%
\begin{pgfscope}%
\pgfsys@transformshift{0.695822in}{0.656509in}%
\pgfsys@useobject{currentmarker}{}%
\end{pgfscope}%
\end{pgfscope}%
\begin{pgfscope}%
\pgfpathrectangle{\pgfqpoint{0.100000in}{0.100000in}}{\pgfqpoint{1.782500in}{1.232000in}}%
\pgfusepath{clip}%
\pgfsetbuttcap%
\pgfsetroundjoin%
\definecolor{currentfill}{rgb}{0.835294,0.321569,0.035294}%
\pgfsetfillcolor{currentfill}%
\pgfsetlinewidth{1.003750pt}%
\definecolor{currentstroke}{rgb}{0.835294,0.321569,0.035294}%
\pgfsetstrokecolor{currentstroke}%
\pgfsetdash{}{0pt}%
\pgfsys@defobject{currentmarker}{\pgfqpoint{-0.018373in}{-0.018373in}}{\pgfqpoint{0.018373in}{0.018373in}}{%
\pgfpathmoveto{\pgfqpoint{0.000000in}{-0.018373in}}%
\pgfpathcurveto{\pgfqpoint{0.004873in}{-0.018373in}}{\pgfqpoint{0.009546in}{-0.016437in}}{\pgfqpoint{0.012992in}{-0.012992in}}%
\pgfpathcurveto{\pgfqpoint{0.016437in}{-0.009546in}}{\pgfqpoint{0.018373in}{-0.004873in}}{\pgfqpoint{0.018373in}{0.000000in}}%
\pgfpathcurveto{\pgfqpoint{0.018373in}{0.004873in}}{\pgfqpoint{0.016437in}{0.009546in}}{\pgfqpoint{0.012992in}{0.012992in}}%
\pgfpathcurveto{\pgfqpoint{0.009546in}{0.016437in}}{\pgfqpoint{0.004873in}{0.018373in}}{\pgfqpoint{0.000000in}{0.018373in}}%
\pgfpathcurveto{\pgfqpoint{-0.004873in}{0.018373in}}{\pgfqpoint{-0.009546in}{0.016437in}}{\pgfqpoint{-0.012992in}{0.012992in}}%
\pgfpathcurveto{\pgfqpoint{-0.016437in}{0.009546in}}{\pgfqpoint{-0.018373in}{0.004873in}}{\pgfqpoint{-0.018373in}{0.000000in}}%
\pgfpathcurveto{\pgfqpoint{-0.018373in}{-0.004873in}}{\pgfqpoint{-0.016437in}{-0.009546in}}{\pgfqpoint{-0.012992in}{-0.012992in}}%
\pgfpathcurveto{\pgfqpoint{-0.009546in}{-0.016437in}}{\pgfqpoint{-0.004873in}{-0.018373in}}{\pgfqpoint{0.000000in}{-0.018373in}}%
\pgfpathlineto{\pgfqpoint{0.000000in}{-0.018373in}}%
\pgfpathclose%
\pgfusepath{stroke,fill}%
}%
\begin{pgfscope}%
\pgfsys@transformshift{0.613861in}{0.704199in}%
\pgfsys@useobject{currentmarker}{}%
\end{pgfscope}%
\begin{pgfscope}%
\pgfsys@transformshift{0.684747in}{0.662330in}%
\pgfsys@useobject{currentmarker}{}%
\end{pgfscope}%
\begin{pgfscope}%
\pgfsys@transformshift{0.763773in}{0.632922in}%
\pgfsys@useobject{currentmarker}{}%
\end{pgfscope}%
\begin{pgfscope}%
\pgfsys@transformshift{0.842615in}{0.614483in}%
\pgfsys@useobject{currentmarker}{}%
\end{pgfscope}%
\begin{pgfscope}%
\pgfsys@transformshift{0.916422in}{0.601060in}%
\pgfsys@useobject{currentmarker}{}%
\end{pgfscope}%
\begin{pgfscope}%
\pgfsys@transformshift{0.990080in}{0.598631in}%
\pgfsys@useobject{currentmarker}{}%
\end{pgfscope}%
\begin{pgfscope}%
\pgfsys@transformshift{1.063069in}{0.606844in}%
\pgfsys@useobject{currentmarker}{}%
\end{pgfscope}%
\begin{pgfscope}%
\pgfsys@transformshift{1.138803in}{0.617203in}%
\pgfsys@useobject{currentmarker}{}%
\end{pgfscope}%
\begin{pgfscope}%
\pgfsys@transformshift{1.217115in}{0.633115in}%
\pgfsys@useobject{currentmarker}{}%
\end{pgfscope}%
\begin{pgfscope}%
\pgfsys@transformshift{1.297142in}{0.661043in}%
\pgfsys@useobject{currentmarker}{}%
\end{pgfscope}%
\begin{pgfscope}%
\pgfsys@transformshift{1.366701in}{0.697571in}%
\pgfsys@useobject{currentmarker}{}%
\end{pgfscope}%
\end{pgfscope}%
\begin{pgfscope}%
\pgfpathrectangle{\pgfqpoint{0.100000in}{0.100000in}}{\pgfqpoint{1.782500in}{1.232000in}}%
\pgfusepath{clip}%
\pgfsetbuttcap%
\pgfsetroundjoin%
\pgfsetlinewidth{1.003750pt}%
\definecolor{currentstroke}{rgb}{0.054902,0.262745,0.486275}%
\pgfsetstrokecolor{currentstroke}%
\pgfsetdash{}{0pt}%
\pgfpathmoveto{\pgfqpoint{0.000000in}{-0.018373in}}%
\pgfpathcurveto{\pgfqpoint{0.004873in}{-0.018373in}}{\pgfqpoint{0.009546in}{-0.016437in}}{\pgfqpoint{0.012992in}{-0.012992in}}%
\pgfpathcurveto{\pgfqpoint{0.016437in}{-0.009546in}}{\pgfqpoint{0.018373in}{-0.004873in}}{\pgfqpoint{0.018373in}{0.000000in}}%
\pgfpathcurveto{\pgfqpoint{0.018373in}{0.004873in}}{\pgfqpoint{0.016437in}{0.009546in}}{\pgfqpoint{0.012992in}{0.012992in}}%
\pgfpathcurveto{\pgfqpoint{0.009546in}{0.016437in}}{\pgfqpoint{0.004873in}{0.018373in}}{\pgfqpoint{0.000000in}{0.018373in}}%
\pgfpathcurveto{\pgfqpoint{-0.004873in}{0.018373in}}{\pgfqpoint{-0.009546in}{0.016437in}}{\pgfqpoint{-0.012992in}{0.012992in}}%
\pgfpathcurveto{\pgfqpoint{-0.016437in}{0.009546in}}{\pgfqpoint{-0.018373in}{0.004873in}}{\pgfqpoint{-0.018373in}{0.000000in}}%
\pgfpathcurveto{\pgfqpoint{-0.018373in}{-0.004873in}}{\pgfqpoint{-0.016437in}{-0.009546in}}{\pgfqpoint{-0.012992in}{-0.012992in}}%
\pgfpathcurveto{\pgfqpoint{-0.009546in}{-0.016437in}}{\pgfqpoint{-0.004873in}{-0.018373in}}{\pgfqpoint{0.000000in}{-0.018373in}}%
\pgfusepath{stroke}%
\end{pgfscope}%
\begin{pgfscope}%
\pgfpathrectangle{\pgfqpoint{0.100000in}{0.100000in}}{\pgfqpoint{1.782500in}{1.232000in}}%
\pgfusepath{clip}%
\pgfsetbuttcap%
\pgfsetroundjoin%
\pgfsetlinewidth{1.003750pt}%
\definecolor{currentstroke}{rgb}{0.835294,0.321569,0.035294}%
\pgfsetstrokecolor{currentstroke}%
\pgfsetdash{}{0pt}%
\pgfpathmoveto{\pgfqpoint{0.000000in}{-0.018373in}}%
\pgfpathcurveto{\pgfqpoint{0.004873in}{-0.018373in}}{\pgfqpoint{0.009546in}{-0.016437in}}{\pgfqpoint{0.012992in}{-0.012992in}}%
\pgfpathcurveto{\pgfqpoint{0.016437in}{-0.009546in}}{\pgfqpoint{0.018373in}{-0.004873in}}{\pgfqpoint{0.018373in}{0.000000in}}%
\pgfpathcurveto{\pgfqpoint{0.018373in}{0.004873in}}{\pgfqpoint{0.016437in}{0.009546in}}{\pgfqpoint{0.012992in}{0.012992in}}%
\pgfpathcurveto{\pgfqpoint{0.009546in}{0.016437in}}{\pgfqpoint{0.004873in}{0.018373in}}{\pgfqpoint{0.000000in}{0.018373in}}%
\pgfpathcurveto{\pgfqpoint{-0.004873in}{0.018373in}}{\pgfqpoint{-0.009546in}{0.016437in}}{\pgfqpoint{-0.012992in}{0.012992in}}%
\pgfpathcurveto{\pgfqpoint{-0.016437in}{0.009546in}}{\pgfqpoint{-0.018373in}{0.004873in}}{\pgfqpoint{-0.018373in}{0.000000in}}%
\pgfpathcurveto{\pgfqpoint{-0.018373in}{-0.004873in}}{\pgfqpoint{-0.016437in}{-0.009546in}}{\pgfqpoint{-0.012992in}{-0.012992in}}%
\pgfpathcurveto{\pgfqpoint{-0.009546in}{-0.016437in}}{\pgfqpoint{-0.004873in}{-0.018373in}}{\pgfqpoint{0.000000in}{-0.018373in}}%
\pgfusepath{stroke}%
\end{pgfscope}%
\begin{pgfscope}%
\pgfpathrectangle{\pgfqpoint{0.100000in}{0.100000in}}{\pgfqpoint{1.782500in}{1.232000in}}%
\pgfusepath{clip}%
\pgfsetbuttcap%
\pgfsetroundjoin%
\definecolor{currentfill}{rgb}{0.054902,0.262745,0.486275}%
\pgfsetfillcolor{currentfill}%
\pgfsetlinewidth{1.505625pt}%
\definecolor{currentstroke}{rgb}{0.054902,0.262745,0.486275}%
\pgfsetstrokecolor{currentstroke}%
\pgfsetdash{}{0pt}%
\pgfsys@defobject{currentmarker}{\pgfqpoint{-0.018373in}{-0.018373in}}{\pgfqpoint{0.018373in}{0.018373in}}{%
\pgfpathmoveto{\pgfqpoint{-0.018373in}{-0.018373in}}%
\pgfpathlineto{\pgfqpoint{0.018373in}{0.018373in}}%
\pgfpathmoveto{\pgfqpoint{-0.018373in}{0.018373in}}%
\pgfpathlineto{\pgfqpoint{0.018373in}{-0.018373in}}%
\pgfusepath{stroke,fill}%
}%
\begin{pgfscope}%
\pgfsys@transformshift{1.377693in}{0.703727in}%
\pgfsys@useobject{currentmarker}{}%
\end{pgfscope}%
\begin{pgfscope}%
\pgfsys@transformshift{0.602566in}{0.711351in}%
\pgfsys@useobject{currentmarker}{}%
\end{pgfscope}%
\begin{pgfscope}%
\pgfsys@transformshift{0.647969in}{0.682601in}%
\pgfsys@useobject{currentmarker}{}%
\end{pgfscope}%
\end{pgfscope}%
\begin{pgfscope}%
\pgfpathrectangle{\pgfqpoint{0.100000in}{0.100000in}}{\pgfqpoint{1.782500in}{1.232000in}}%
\pgfusepath{clip}%
\pgfsetbuttcap%
\pgfsetroundjoin%
\definecolor{currentfill}{rgb}{0.835294,0.321569,0.035294}%
\pgfsetfillcolor{currentfill}%
\pgfsetlinewidth{1.505625pt}%
\definecolor{currentstroke}{rgb}{0.835294,0.321569,0.035294}%
\pgfsetstrokecolor{currentstroke}%
\pgfsetdash{}{0pt}%
\pgfsys@defobject{currentmarker}{\pgfqpoint{-0.018373in}{-0.018373in}}{\pgfqpoint{0.018373in}{0.018373in}}{%
\pgfpathmoveto{\pgfqpoint{-0.018373in}{-0.018373in}}%
\pgfpathlineto{\pgfqpoint{0.018373in}{0.018373in}}%
\pgfpathmoveto{\pgfqpoint{-0.018373in}{0.018373in}}%
\pgfpathlineto{\pgfqpoint{0.018373in}{-0.018373in}}%
\pgfusepath{stroke,fill}%
}%
\end{pgfscope}%
\end{pgfpicture}%
\makeatother%
\endgroup%

        \caption{Iteration 1: Update interface}\label{fig:example-iter0-dumping}
    \end{subfigure}
    \begin{subfigure}[b]{.32\linewidth}
        %% Creator: Matplotlib, PGF backend
%%
%% To include the figure in your LaTeX document, write
%%   \input{<filename>.pgf}
%%
%% Make sure the required packages are loaded in your preamble
%%   \usepackage{pgf}
%%
%% Also ensure that all the required font packages are loaded; for instance,
%% the lmodern package is sometimes necessary when using math font.
%%   \usepackage{lmodern}
%%
%% Figures using additional raster images can only be included by \input if
%% they are in the same directory as the main LaTeX file. For loading figures
%% from other directories you can use the `import` package
%%   \usepackage{import}
%%
%% and then include the figures with
%%   \import{<path to file>}{<filename>.pgf}
%%
%% Matplotlib used the following preamble
%%   
%%   \usepackage{fontspec}
%%   \setmainfont{DejaVuSans.ttf}[Path=\detokenize{/home/fabio/Internodes-CM/.venv/lib/python3.8/site-packages/matplotlib/mpl-data/fonts/ttf/}]
%%   \setsansfont{DejaVuSans.ttf}[Path=\detokenize{/home/fabio/Internodes-CM/.venv/lib/python3.8/site-packages/matplotlib/mpl-data/fonts/ttf/}]
%%   \setmonofont{DejaVuSansMono.ttf}[Path=\detokenize{/home/fabio/Internodes-CM/.venv/lib/python3.8/site-packages/matplotlib/mpl-data/fonts/ttf/}]
%%   \makeatletter\@ifpackageloaded{underscore}{}{\usepackage[strings]{underscore}}\makeatother
%%
\begingroup%
\makeatletter%
\begin{pgfpicture}%
\pgfpathrectangle{\pgfpointorigin}{\pgfqpoint{1.982500in}{1.432000in}}%
\pgfusepath{use as bounding box, clip}%
\begin{pgfscope}%
\pgfsetbuttcap%
\pgfsetmiterjoin%
\definecolor{currentfill}{rgb}{1.000000,1.000000,1.000000}%
\pgfsetfillcolor{currentfill}%
\pgfsetlinewidth{0.000000pt}%
\definecolor{currentstroke}{rgb}{1.000000,1.000000,1.000000}%
\pgfsetstrokecolor{currentstroke}%
\pgfsetdash{}{0pt}%
\pgfpathmoveto{\pgfqpoint{0.000000in}{0.000000in}}%
\pgfpathlineto{\pgfqpoint{1.982500in}{0.000000in}}%
\pgfpathlineto{\pgfqpoint{1.982500in}{1.432000in}}%
\pgfpathlineto{\pgfqpoint{0.000000in}{1.432000in}}%
\pgfpathlineto{\pgfqpoint{0.000000in}{0.000000in}}%
\pgfpathclose%
\pgfusepath{fill}%
\end{pgfscope}%
\begin{pgfscope}%
\pgfpathrectangle{\pgfqpoint{0.100000in}{0.100000in}}{\pgfqpoint{1.782500in}{1.232000in}}%
\pgfusepath{clip}%
\pgfsetrectcap%
\pgfsetroundjoin%
\pgfsetlinewidth{0.250937pt}%
\definecolor{currentstroke}{rgb}{0.054902,0.262745,0.486275}%
\pgfsetstrokecolor{currentstroke}%
\pgfsetdash{}{0pt}%
\pgfpathmoveto{\pgfqpoint{1.882500in}{0.452000in}}%
\pgfpathlineto{\pgfqpoint{1.892500in}{0.452000in}}%
\pgfpathmoveto{\pgfqpoint{1.729714in}{0.452000in}}%
\pgfpathlineto{\pgfqpoint{1.882500in}{0.452000in}}%
\pgfpathmoveto{\pgfqpoint{1.576929in}{0.452000in}}%
\pgfpathlineto{\pgfqpoint{1.729714in}{0.452000in}}%
\pgfpathmoveto{\pgfqpoint{1.424143in}{0.452000in}}%
\pgfpathlineto{\pgfqpoint{1.576929in}{0.452000in}}%
\pgfpathmoveto{\pgfqpoint{1.271357in}{0.452000in}}%
\pgfpathlineto{\pgfqpoint{1.424143in}{0.452000in}}%
\pgfpathmoveto{\pgfqpoint{1.118571in}{0.452000in}}%
\pgfpathlineto{\pgfqpoint{1.271357in}{0.452000in}}%
\pgfpathmoveto{\pgfqpoint{0.965786in}{0.452000in}}%
\pgfpathlineto{\pgfqpoint{0.813000in}{0.452000in}}%
\pgfpathmoveto{\pgfqpoint{0.965786in}{0.452000in}}%
\pgfpathlineto{\pgfqpoint{1.118571in}{0.452000in}}%
\pgfpathmoveto{\pgfqpoint{0.456500in}{0.452000in}}%
\pgfpathlineto{\pgfqpoint{0.813000in}{0.452000in}}%
\pgfpathmoveto{\pgfqpoint{0.456500in}{0.452000in}}%
\pgfpathlineto{\pgfqpoint{0.100000in}{0.452000in}}%
\pgfpathmoveto{\pgfqpoint{0.100000in}{0.090000in}}%
\pgfpathlineto{\pgfqpoint{0.100000in}{0.452000in}}%
\pgfpathmoveto{\pgfqpoint{1.805208in}{0.175249in}}%
\pgfpathlineto{\pgfqpoint{1.892500in}{0.121802in}}%
\pgfpathmoveto{\pgfqpoint{1.805208in}{0.175249in}}%
\pgfpathlineto{\pgfqpoint{1.637355in}{0.090000in}}%
\pgfpathmoveto{\pgfqpoint{1.195930in}{0.177192in}}%
\pgfpathlineto{\pgfqpoint{1.374206in}{0.090000in}}%
\pgfpathmoveto{\pgfqpoint{1.195930in}{0.177192in}}%
\pgfpathlineto{\pgfqpoint{1.042511in}{0.090000in}}%
\pgfpathmoveto{\pgfqpoint{1.892500in}{0.189285in}}%
\pgfpathlineto{\pgfqpoint{1.805208in}{0.175249in}}%
\pgfpathmoveto{\pgfqpoint{1.501834in}{0.227677in}}%
\pgfpathlineto{\pgfqpoint{1.506188in}{0.090000in}}%
\pgfpathmoveto{\pgfqpoint{1.501834in}{0.227677in}}%
\pgfpathlineto{\pgfqpoint{1.805208in}{0.175249in}}%
\pgfpathmoveto{\pgfqpoint{1.501834in}{0.227677in}}%
\pgfpathlineto{\pgfqpoint{1.195930in}{0.177192in}}%
\pgfpathmoveto{\pgfqpoint{0.856997in}{0.215161in}}%
\pgfpathlineto{\pgfqpoint{0.876107in}{0.090000in}}%
\pgfpathmoveto{\pgfqpoint{0.856997in}{0.215161in}}%
\pgfpathlineto{\pgfqpoint{1.195930in}{0.177192in}}%
\pgfpathmoveto{\pgfqpoint{0.485360in}{0.139977in}}%
\pgfpathlineto{\pgfqpoint{0.100000in}{0.452000in}}%
\pgfpathmoveto{\pgfqpoint{0.485360in}{0.139977in}}%
\pgfpathlineto{\pgfqpoint{0.456500in}{0.452000in}}%
\pgfpathmoveto{\pgfqpoint{0.485360in}{0.139977in}}%
\pgfpathlineto{\pgfqpoint{0.334871in}{0.090000in}}%
\pgfpathmoveto{\pgfqpoint{0.485360in}{0.139977in}}%
\pgfpathlineto{\pgfqpoint{0.632751in}{0.090000in}}%
\pgfpathmoveto{\pgfqpoint{0.485360in}{0.139977in}}%
\pgfpathlineto{\pgfqpoint{0.856997in}{0.215161in}}%
\pgfpathmoveto{\pgfqpoint{1.892500in}{0.434693in}}%
\pgfpathlineto{\pgfqpoint{1.882500in}{0.452000in}}%
\pgfpathmoveto{\pgfqpoint{1.892500in}{0.257344in}}%
\pgfpathlineto{\pgfqpoint{1.805208in}{0.175249in}}%
\pgfpathmoveto{\pgfqpoint{1.653321in}{0.319800in}}%
\pgfpathlineto{\pgfqpoint{1.729714in}{0.452000in}}%
\pgfpathmoveto{\pgfqpoint{1.653321in}{0.319800in}}%
\pgfpathlineto{\pgfqpoint{1.576929in}{0.452000in}}%
\pgfpathmoveto{\pgfqpoint{1.653321in}{0.319800in}}%
\pgfpathlineto{\pgfqpoint{1.805208in}{0.175249in}}%
\pgfpathmoveto{\pgfqpoint{1.653321in}{0.319800in}}%
\pgfpathlineto{\pgfqpoint{1.501834in}{0.227677in}}%
\pgfpathmoveto{\pgfqpoint{1.347750in}{0.319800in}}%
\pgfpathlineto{\pgfqpoint{1.424143in}{0.452000in}}%
\pgfpathmoveto{\pgfqpoint{1.347750in}{0.319800in}}%
\pgfpathlineto{\pgfqpoint{1.271357in}{0.452000in}}%
\pgfpathmoveto{\pgfqpoint{1.347750in}{0.319800in}}%
\pgfpathlineto{\pgfqpoint{1.195930in}{0.177192in}}%
\pgfpathmoveto{\pgfqpoint{1.347750in}{0.319800in}}%
\pgfpathlineto{\pgfqpoint{1.501834in}{0.227677in}}%
\pgfpathmoveto{\pgfqpoint{1.042179in}{0.319800in}}%
\pgfpathlineto{\pgfqpoint{1.118571in}{0.452000in}}%
\pgfpathmoveto{\pgfqpoint{1.042179in}{0.319800in}}%
\pgfpathlineto{\pgfqpoint{0.965786in}{0.452000in}}%
\pgfpathmoveto{\pgfqpoint{1.042179in}{0.319800in}}%
\pgfpathlineto{\pgfqpoint{1.195930in}{0.177192in}}%
\pgfpathmoveto{\pgfqpoint{1.042179in}{0.319800in}}%
\pgfpathlineto{\pgfqpoint{0.856997in}{0.215161in}}%
\pgfpathmoveto{\pgfqpoint{1.793890in}{0.090000in}}%
\pgfpathlineto{\pgfqpoint{1.805208in}{0.175249in}}%
\pgfpathmoveto{\pgfqpoint{0.655483in}{0.306719in}}%
\pgfpathlineto{\pgfqpoint{0.813000in}{0.452000in}}%
\pgfpathmoveto{\pgfqpoint{0.655483in}{0.306719in}}%
\pgfpathlineto{\pgfqpoint{0.456500in}{0.452000in}}%
\pgfpathmoveto{\pgfqpoint{0.655483in}{0.306719in}}%
\pgfpathlineto{\pgfqpoint{0.856997in}{0.215161in}}%
\pgfpathmoveto{\pgfqpoint{0.655483in}{0.306719in}}%
\pgfpathlineto{\pgfqpoint{0.485360in}{0.139977in}}%
\pgfpathmoveto{\pgfqpoint{1.805927in}{0.343767in}}%
\pgfpathlineto{\pgfqpoint{1.882500in}{0.452000in}}%
\pgfpathmoveto{\pgfqpoint{1.805927in}{0.343767in}}%
\pgfpathlineto{\pgfqpoint{1.729714in}{0.452000in}}%
\pgfpathmoveto{\pgfqpoint{1.805927in}{0.343767in}}%
\pgfpathlineto{\pgfqpoint{1.805208in}{0.175249in}}%
\pgfpathmoveto{\pgfqpoint{1.805927in}{0.343767in}}%
\pgfpathlineto{\pgfqpoint{1.892500in}{0.330194in}}%
\pgfpathmoveto{\pgfqpoint{1.805927in}{0.343767in}}%
\pgfpathlineto{\pgfqpoint{1.653321in}{0.319800in}}%
\pgfpathmoveto{\pgfqpoint{1.500668in}{0.351969in}}%
\pgfpathlineto{\pgfqpoint{1.576929in}{0.452000in}}%
\pgfpathmoveto{\pgfqpoint{1.500668in}{0.351969in}}%
\pgfpathlineto{\pgfqpoint{1.424143in}{0.452000in}}%
\pgfpathmoveto{\pgfqpoint{1.500668in}{0.351969in}}%
\pgfpathlineto{\pgfqpoint{1.501834in}{0.227677in}}%
\pgfpathmoveto{\pgfqpoint{1.500668in}{0.351969in}}%
\pgfpathlineto{\pgfqpoint{1.653321in}{0.319800in}}%
\pgfpathmoveto{\pgfqpoint{1.500668in}{0.351969in}}%
\pgfpathlineto{\pgfqpoint{1.347750in}{0.319800in}}%
\pgfpathmoveto{\pgfqpoint{1.206988in}{0.090000in}}%
\pgfpathlineto{\pgfqpoint{1.195930in}{0.177192in}}%
\pgfpathmoveto{\pgfqpoint{1.195157in}{0.344158in}}%
\pgfpathlineto{\pgfqpoint{1.271357in}{0.452000in}}%
\pgfpathmoveto{\pgfqpoint{1.195157in}{0.344158in}}%
\pgfpathlineto{\pgfqpoint{1.118571in}{0.452000in}}%
\pgfpathmoveto{\pgfqpoint{1.195157in}{0.344158in}}%
\pgfpathlineto{\pgfqpoint{1.195930in}{0.177192in}}%
\pgfpathmoveto{\pgfqpoint{1.195157in}{0.344158in}}%
\pgfpathlineto{\pgfqpoint{1.347750in}{0.319800in}}%
\pgfpathmoveto{\pgfqpoint{1.195157in}{0.344158in}}%
\pgfpathlineto{\pgfqpoint{1.042179in}{0.319800in}}%
\pgfpathmoveto{\pgfqpoint{0.889393in}{0.349588in}}%
\pgfpathlineto{\pgfqpoint{0.813000in}{0.452000in}}%
\pgfpathmoveto{\pgfqpoint{0.889393in}{0.349588in}}%
\pgfpathlineto{\pgfqpoint{0.965786in}{0.452000in}}%
\pgfpathmoveto{\pgfqpoint{0.889393in}{0.349588in}}%
\pgfpathlineto{\pgfqpoint{0.856997in}{0.215161in}}%
\pgfpathmoveto{\pgfqpoint{0.889393in}{0.349588in}}%
\pgfpathlineto{\pgfqpoint{1.042179in}{0.319800in}}%
\pgfpathmoveto{\pgfqpoint{0.889393in}{0.349588in}}%
\pgfpathlineto{\pgfqpoint{0.655483in}{0.306719in}}%
\pgfpathmoveto{\pgfqpoint{0.494080in}{0.090000in}}%
\pgfpathlineto{\pgfqpoint{0.485360in}{0.139977in}}%
\pgfpathlineto{\pgfqpoint{0.485360in}{0.139977in}}%
\pgfusepath{stroke}%
\end{pgfscope}%
\begin{pgfscope}%
\pgfpathrectangle{\pgfqpoint{0.100000in}{0.100000in}}{\pgfqpoint{1.782500in}{1.232000in}}%
\pgfusepath{clip}%
\pgfsetrectcap%
\pgfsetroundjoin%
\pgfsetlinewidth{0.250937pt}%
\definecolor{currentstroke}{rgb}{0.835294,0.321569,0.035294}%
\pgfsetstrokecolor{currentstroke}%
\pgfsetdash{}{0pt}%
\pgfpathmoveto{\pgfqpoint{0.813000in}{0.860417in}}%
\pgfpathlineto{\pgfqpoint{0.694167in}{1.244000in}}%
\pgfpathmoveto{\pgfqpoint{1.288333in}{1.244000in}}%
\pgfpathlineto{\pgfqpoint{1.882500in}{1.244000in}}%
\pgfpathmoveto{\pgfqpoint{1.288333in}{1.244000in}}%
\pgfpathlineto{\pgfqpoint{0.694167in}{1.244000in}}%
\pgfpathmoveto{\pgfqpoint{0.909162in}{0.739160in}}%
\pgfpathlineto{\pgfqpoint{0.813000in}{0.860417in}}%
\pgfpathmoveto{\pgfqpoint{1.030178in}{0.630794in}}%
\pgfpathlineto{\pgfqpoint{0.909162in}{0.739160in}}%
\pgfpathmoveto{\pgfqpoint{1.172958in}{0.538087in}}%
\pgfpathlineto{\pgfqpoint{1.030178in}{0.630794in}}%
\pgfpathmoveto{\pgfqpoint{1.333857in}{0.463404in}}%
\pgfpathlineto{\pgfqpoint{1.172958in}{0.538087in}}%
\pgfpathmoveto{\pgfqpoint{1.508764in}{0.408655in}}%
\pgfpathlineto{\pgfqpoint{1.333857in}{0.463404in}}%
\pgfpathmoveto{\pgfqpoint{1.693215in}{0.375235in}}%
\pgfpathlineto{\pgfqpoint{1.508764in}{0.408655in}}%
\pgfpathmoveto{\pgfqpoint{1.882500in}{0.364000in}}%
\pgfpathlineto{\pgfqpoint{1.693215in}{0.375235in}}%
\pgfpathmoveto{\pgfqpoint{1.892500in}{0.364594in}}%
\pgfpathlineto{\pgfqpoint{1.882500in}{0.364000in}}%
\pgfpathmoveto{\pgfqpoint{1.892500in}{1.244000in}}%
\pgfpathlineto{\pgfqpoint{1.882500in}{1.244000in}}%
\pgfpathmoveto{\pgfqpoint{1.544975in}{0.898385in}}%
\pgfpathlineto{\pgfqpoint{1.882500in}{1.244000in}}%
\pgfpathmoveto{\pgfqpoint{1.544975in}{0.898385in}}%
\pgfpathlineto{\pgfqpoint{1.288333in}{1.244000in}}%
\pgfpathmoveto{\pgfqpoint{1.817713in}{0.645445in}}%
\pgfpathlineto{\pgfqpoint{1.892500in}{0.675390in}}%
\pgfpathmoveto{\pgfqpoint{1.817713in}{0.645445in}}%
\pgfpathlineto{\pgfqpoint{1.544975in}{0.898385in}}%
\pgfpathmoveto{\pgfqpoint{1.186381in}{0.929127in}}%
\pgfpathlineto{\pgfqpoint{1.288333in}{1.244000in}}%
\pgfpathmoveto{\pgfqpoint{1.186381in}{0.929127in}}%
\pgfpathlineto{\pgfqpoint{1.544975in}{0.898385in}}%
\pgfpathmoveto{\pgfqpoint{1.892500in}{0.628836in}}%
\pgfpathlineto{\pgfqpoint{1.817713in}{0.645445in}}%
\pgfpathmoveto{\pgfqpoint{1.511108in}{0.615050in}}%
\pgfpathlineto{\pgfqpoint{1.333857in}{0.463404in}}%
\pgfpathmoveto{\pgfqpoint{1.511108in}{0.615050in}}%
\pgfpathlineto{\pgfqpoint{1.508764in}{0.408655in}}%
\pgfpathmoveto{\pgfqpoint{1.511108in}{0.615050in}}%
\pgfpathlineto{\pgfqpoint{1.544975in}{0.898385in}}%
\pgfpathmoveto{\pgfqpoint{1.511108in}{0.615050in}}%
\pgfpathlineto{\pgfqpoint{1.817713in}{0.645445in}}%
\pgfpathmoveto{\pgfqpoint{1.278705in}{0.730225in}}%
\pgfpathlineto{\pgfqpoint{1.030178in}{0.630794in}}%
\pgfpathmoveto{\pgfqpoint{1.278705in}{0.730225in}}%
\pgfpathlineto{\pgfqpoint{1.172958in}{0.538087in}}%
\pgfpathmoveto{\pgfqpoint{1.278705in}{0.730225in}}%
\pgfpathlineto{\pgfqpoint{1.544975in}{0.898385in}}%
\pgfpathmoveto{\pgfqpoint{1.278705in}{0.730225in}}%
\pgfpathlineto{\pgfqpoint{1.186381in}{0.929127in}}%
\pgfpathmoveto{\pgfqpoint{1.278705in}{0.730225in}}%
\pgfpathlineto{\pgfqpoint{1.511108in}{0.615050in}}%
\pgfpathmoveto{\pgfqpoint{1.798690in}{0.469691in}}%
\pgfpathlineto{\pgfqpoint{1.693215in}{0.375235in}}%
\pgfpathmoveto{\pgfqpoint{1.798690in}{0.469691in}}%
\pgfpathlineto{\pgfqpoint{1.882500in}{0.364000in}}%
\pgfpathmoveto{\pgfqpoint{1.798690in}{0.469691in}}%
\pgfpathlineto{\pgfqpoint{1.817713in}{0.645445in}}%
\pgfpathmoveto{\pgfqpoint{1.334493in}{0.597404in}}%
\pgfpathlineto{\pgfqpoint{1.172958in}{0.538087in}}%
\pgfpathmoveto{\pgfqpoint{1.334493in}{0.597404in}}%
\pgfpathlineto{\pgfqpoint{1.333857in}{0.463404in}}%
\pgfpathmoveto{\pgfqpoint{1.334493in}{0.597404in}}%
\pgfpathlineto{\pgfqpoint{1.511108in}{0.615050in}}%
\pgfpathmoveto{\pgfqpoint{1.334493in}{0.597404in}}%
\pgfpathlineto{\pgfqpoint{1.278705in}{0.730225in}}%
\pgfpathmoveto{\pgfqpoint{1.095926in}{0.762297in}}%
\pgfpathlineto{\pgfqpoint{0.909162in}{0.739160in}}%
\pgfpathmoveto{\pgfqpoint{1.095926in}{0.762297in}}%
\pgfpathlineto{\pgfqpoint{1.030178in}{0.630794in}}%
\pgfpathmoveto{\pgfqpoint{1.095926in}{0.762297in}}%
\pgfpathlineto{\pgfqpoint{1.186381in}{0.929127in}}%
\pgfpathmoveto{\pgfqpoint{1.095926in}{0.762297in}}%
\pgfpathlineto{\pgfqpoint{1.278705in}{0.730225in}}%
\pgfpathmoveto{\pgfqpoint{0.960375in}{1.053168in}}%
\pgfpathlineto{\pgfqpoint{0.694167in}{1.244000in}}%
\pgfpathmoveto{\pgfqpoint{0.960375in}{1.053168in}}%
\pgfpathlineto{\pgfqpoint{0.813000in}{0.860417in}}%
\pgfpathmoveto{\pgfqpoint{0.960375in}{1.053168in}}%
\pgfpathlineto{\pgfqpoint{1.288333in}{1.244000in}}%
\pgfpathmoveto{\pgfqpoint{0.960375in}{1.053168in}}%
\pgfpathlineto{\pgfqpoint{1.186381in}{0.929127in}}%
\pgfpathmoveto{\pgfqpoint{1.892500in}{0.376929in}}%
\pgfpathlineto{\pgfqpoint{1.882500in}{0.364000in}}%
\pgfpathmoveto{\pgfqpoint{1.892500in}{0.558019in}}%
\pgfpathlineto{\pgfqpoint{1.817713in}{0.645445in}}%
\pgfpathmoveto{\pgfqpoint{1.892500in}{0.471002in}}%
\pgfpathlineto{\pgfqpoint{1.798690in}{0.469691in}}%
\pgfpathmoveto{\pgfqpoint{1.633962in}{0.491742in}}%
\pgfpathlineto{\pgfqpoint{1.508764in}{0.408655in}}%
\pgfpathmoveto{\pgfqpoint{1.633962in}{0.491742in}}%
\pgfpathlineto{\pgfqpoint{1.693215in}{0.375235in}}%
\pgfpathmoveto{\pgfqpoint{1.633962in}{0.491742in}}%
\pgfpathlineto{\pgfqpoint{1.817713in}{0.645445in}}%
\pgfpathmoveto{\pgfqpoint{1.633962in}{0.491742in}}%
\pgfpathlineto{\pgfqpoint{1.511108in}{0.615050in}}%
\pgfpathmoveto{\pgfqpoint{1.633962in}{0.491742in}}%
\pgfpathlineto{\pgfqpoint{1.798690in}{0.469691in}}%
\pgfpathmoveto{\pgfqpoint{1.892500in}{1.184050in}}%
\pgfpathlineto{\pgfqpoint{1.882500in}{1.244000in}}%
\pgfpathmoveto{\pgfqpoint{1.892500in}{0.919808in}}%
\pgfpathlineto{\pgfqpoint{1.544975in}{0.898385in}}%
\pgfpathmoveto{\pgfqpoint{1.892500in}{0.820421in}}%
\pgfpathlineto{\pgfqpoint{1.817713in}{0.645445in}}%
\pgfpathmoveto{\pgfqpoint{1.892500in}{1.238184in}}%
\pgfpathlineto{\pgfqpoint{1.882500in}{1.244000in}}%
\pgfpathmoveto{\pgfqpoint{0.992969in}{0.868834in}}%
\pgfpathlineto{\pgfqpoint{0.813000in}{0.860417in}}%
\pgfpathmoveto{\pgfqpoint{0.992969in}{0.868834in}}%
\pgfpathlineto{\pgfqpoint{0.909162in}{0.739160in}}%
\pgfpathmoveto{\pgfqpoint{0.992969in}{0.868834in}}%
\pgfpathlineto{\pgfqpoint{1.186381in}{0.929127in}}%
\pgfpathmoveto{\pgfqpoint{0.992969in}{0.868834in}}%
\pgfpathlineto{\pgfqpoint{1.095926in}{0.762297in}}%
\pgfpathmoveto{\pgfqpoint{0.992969in}{0.868834in}}%
\pgfpathlineto{\pgfqpoint{0.960375in}{1.053168in}}%
\pgfpathlineto{\pgfqpoint{0.960375in}{1.053168in}}%
\pgfusepath{stroke}%
\end{pgfscope}%
\begin{pgfscope}%
\pgfpathrectangle{\pgfqpoint{0.100000in}{0.100000in}}{\pgfqpoint{1.782500in}{1.232000in}}%
\pgfusepath{clip}%
\pgfsetbuttcap%
\pgfsetroundjoin%
\definecolor{currentfill}{rgb}{0.054902,0.262745,0.486275}%
\pgfsetfillcolor{currentfill}%
\pgfsetlinewidth{1.003750pt}%
\definecolor{currentstroke}{rgb}{0.054902,0.262745,0.486275}%
\pgfsetstrokecolor{currentstroke}%
\pgfsetdash{}{0pt}%
\pgfsys@defobject{currentmarker}{\pgfqpoint{-0.018373in}{-0.018373in}}{\pgfqpoint{0.018373in}{0.018373in}}{%
\pgfpathmoveto{\pgfqpoint{0.000000in}{-0.018373in}}%
\pgfpathcurveto{\pgfqpoint{0.004873in}{-0.018373in}}{\pgfqpoint{0.009546in}{-0.016437in}}{\pgfqpoint{0.012992in}{-0.012992in}}%
\pgfpathcurveto{\pgfqpoint{0.016437in}{-0.009546in}}{\pgfqpoint{0.018373in}{-0.004873in}}{\pgfqpoint{0.018373in}{0.000000in}}%
\pgfpathcurveto{\pgfqpoint{0.018373in}{0.004873in}}{\pgfqpoint{0.016437in}{0.009546in}}{\pgfqpoint{0.012992in}{0.012992in}}%
\pgfpathcurveto{\pgfqpoint{0.009546in}{0.016437in}}{\pgfqpoint{0.004873in}{0.018373in}}{\pgfqpoint{0.000000in}{0.018373in}}%
\pgfpathcurveto{\pgfqpoint{-0.004873in}{0.018373in}}{\pgfqpoint{-0.009546in}{0.016437in}}{\pgfqpoint{-0.012992in}{0.012992in}}%
\pgfpathcurveto{\pgfqpoint{-0.016437in}{0.009546in}}{\pgfqpoint{-0.018373in}{0.004873in}}{\pgfqpoint{-0.018373in}{0.000000in}}%
\pgfpathcurveto{\pgfqpoint{-0.018373in}{-0.004873in}}{\pgfqpoint{-0.016437in}{-0.009546in}}{\pgfqpoint{-0.012992in}{-0.012992in}}%
\pgfpathcurveto{\pgfqpoint{-0.009546in}{-0.016437in}}{\pgfqpoint{-0.004873in}{-0.018373in}}{\pgfqpoint{0.000000in}{-0.018373in}}%
\pgfpathlineto{\pgfqpoint{0.000000in}{-0.018373in}}%
\pgfpathclose%
\pgfusepath{stroke,fill}%
}%
\begin{pgfscope}%
\pgfsys@transformshift{2.799214in}{0.452000in}%
\pgfsys@useobject{currentmarker}{}%
\end{pgfscope}%
\begin{pgfscope}%
\pgfsys@transformshift{2.646429in}{0.452000in}%
\pgfsys@useobject{currentmarker}{}%
\end{pgfscope}%
\begin{pgfscope}%
\pgfsys@transformshift{2.493643in}{0.452000in}%
\pgfsys@useobject{currentmarker}{}%
\end{pgfscope}%
\begin{pgfscope}%
\pgfsys@transformshift{2.340857in}{0.452000in}%
\pgfsys@useobject{currentmarker}{}%
\end{pgfscope}%
\begin{pgfscope}%
\pgfsys@transformshift{2.188071in}{0.452000in}%
\pgfsys@useobject{currentmarker}{}%
\end{pgfscope}%
\begin{pgfscope}%
\pgfsys@transformshift{2.035286in}{0.452000in}%
\pgfsys@useobject{currentmarker}{}%
\end{pgfscope}%
\begin{pgfscope}%
\pgfsys@transformshift{1.882500in}{0.452000in}%
\pgfsys@useobject{currentmarker}{}%
\end{pgfscope}%
\begin{pgfscope}%
\pgfsys@transformshift{1.729714in}{0.452000in}%
\pgfsys@useobject{currentmarker}{}%
\end{pgfscope}%
\begin{pgfscope}%
\pgfsys@transformshift{1.576929in}{0.452000in}%
\pgfsys@useobject{currentmarker}{}%
\end{pgfscope}%
\begin{pgfscope}%
\pgfsys@transformshift{1.424143in}{0.452000in}%
\pgfsys@useobject{currentmarker}{}%
\end{pgfscope}%
\begin{pgfscope}%
\pgfsys@transformshift{1.271357in}{0.452000in}%
\pgfsys@useobject{currentmarker}{}%
\end{pgfscope}%
\begin{pgfscope}%
\pgfsys@transformshift{1.118571in}{0.452000in}%
\pgfsys@useobject{currentmarker}{}%
\end{pgfscope}%
\end{pgfscope}%
\begin{pgfscope}%
\pgfpathrectangle{\pgfqpoint{0.100000in}{0.100000in}}{\pgfqpoint{1.782500in}{1.232000in}}%
\pgfusepath{clip}%
\pgfsetbuttcap%
\pgfsetroundjoin%
\definecolor{currentfill}{rgb}{0.835294,0.321569,0.035294}%
\pgfsetfillcolor{currentfill}%
\pgfsetlinewidth{1.003750pt}%
\definecolor{currentstroke}{rgb}{0.835294,0.321569,0.035294}%
\pgfsetstrokecolor{currentstroke}%
\pgfsetdash{}{0pt}%
\pgfsys@defobject{currentmarker}{\pgfqpoint{-0.018373in}{-0.018373in}}{\pgfqpoint{0.018373in}{0.018373in}}{%
\pgfpathmoveto{\pgfqpoint{0.000000in}{-0.018373in}}%
\pgfpathcurveto{\pgfqpoint{0.004873in}{-0.018373in}}{\pgfqpoint{0.009546in}{-0.016437in}}{\pgfqpoint{0.012992in}{-0.012992in}}%
\pgfpathcurveto{\pgfqpoint{0.016437in}{-0.009546in}}{\pgfqpoint{0.018373in}{-0.004873in}}{\pgfqpoint{0.018373in}{0.000000in}}%
\pgfpathcurveto{\pgfqpoint{0.018373in}{0.004873in}}{\pgfqpoint{0.016437in}{0.009546in}}{\pgfqpoint{0.012992in}{0.012992in}}%
\pgfpathcurveto{\pgfqpoint{0.009546in}{0.016437in}}{\pgfqpoint{0.004873in}{0.018373in}}{\pgfqpoint{0.000000in}{0.018373in}}%
\pgfpathcurveto{\pgfqpoint{-0.004873in}{0.018373in}}{\pgfqpoint{-0.009546in}{0.016437in}}{\pgfqpoint{-0.012992in}{0.012992in}}%
\pgfpathcurveto{\pgfqpoint{-0.016437in}{0.009546in}}{\pgfqpoint{-0.018373in}{0.004873in}}{\pgfqpoint{-0.018373in}{0.000000in}}%
\pgfpathcurveto{\pgfqpoint{-0.018373in}{-0.004873in}}{\pgfqpoint{-0.016437in}{-0.009546in}}{\pgfqpoint{-0.012992in}{-0.012992in}}%
\pgfpathcurveto{\pgfqpoint{-0.009546in}{-0.016437in}}{\pgfqpoint{-0.004873in}{-0.018373in}}{\pgfqpoint{0.000000in}{-0.018373in}}%
\pgfpathlineto{\pgfqpoint{0.000000in}{-0.018373in}}%
\pgfpathclose%
\pgfusepath{stroke,fill}%
}%
\begin{pgfscope}%
\pgfsys@transformshift{1.030178in}{0.630794in}%
\pgfsys@useobject{currentmarker}{}%
\end{pgfscope}%
\begin{pgfscope}%
\pgfsys@transformshift{1.172958in}{0.538087in}%
\pgfsys@useobject{currentmarker}{}%
\end{pgfscope}%
\begin{pgfscope}%
\pgfsys@transformshift{1.333857in}{0.463404in}%
\pgfsys@useobject{currentmarker}{}%
\end{pgfscope}%
\begin{pgfscope}%
\pgfsys@transformshift{1.508764in}{0.408655in}%
\pgfsys@useobject{currentmarker}{}%
\end{pgfscope}%
\begin{pgfscope}%
\pgfsys@transformshift{1.693215in}{0.375235in}%
\pgfsys@useobject{currentmarker}{}%
\end{pgfscope}%
\begin{pgfscope}%
\pgfsys@transformshift{1.882500in}{0.364000in}%
\pgfsys@useobject{currentmarker}{}%
\end{pgfscope}%
\begin{pgfscope}%
\pgfsys@transformshift{2.071785in}{0.375235in}%
\pgfsys@useobject{currentmarker}{}%
\end{pgfscope}%
\begin{pgfscope}%
\pgfsys@transformshift{2.256236in}{0.408655in}%
\pgfsys@useobject{currentmarker}{}%
\end{pgfscope}%
\begin{pgfscope}%
\pgfsys@transformshift{2.431143in}{0.463404in}%
\pgfsys@useobject{currentmarker}{}%
\end{pgfscope}%
\begin{pgfscope}%
\pgfsys@transformshift{2.592042in}{0.538087in}%
\pgfsys@useobject{currentmarker}{}%
\end{pgfscope}%
\begin{pgfscope}%
\pgfsys@transformshift{2.734822in}{0.630794in}%
\pgfsys@useobject{currentmarker}{}%
\end{pgfscope}%
\end{pgfscope}%
\end{pgfpicture}%
\makeatother%
\endgroup%

        \caption{Iteration 2: Find interface}\label{fig:example-iter1-interface}
    \end{subfigure}
    \begin{subfigure}[b]{.32\linewidth}
        %% Creator: Matplotlib, PGF backend
%%
%% To include the figure in your LaTeX document, write
%%   \input{<filename>.pgf}
%%
%% Make sure the required packages are loaded in your preamble
%%   \usepackage{pgf}
%%
%% Also ensure that all the required font packages are loaded; for instance,
%% the lmodern package is sometimes necessary when using math font.
%%   \usepackage{lmodern}
%%
%% Figures using additional raster images can only be included by \input if
%% they are in the same directory as the main LaTeX file. For loading figures
%% from other directories you can use the `import` package
%%   \usepackage{import}
%%
%% and then include the figures with
%%   \import{<path to file>}{<filename>.pgf}
%%
%% Matplotlib used the following preamble
%%   
%%   \usepackage{fontspec}
%%   \setmainfont{DejaVuSans.ttf}[Path=\detokenize{/home/fabio/Internodes-CM/.venv/lib/python3.8/site-packages/matplotlib/mpl-data/fonts/ttf/}]
%%   \setsansfont{DejaVuSans.ttf}[Path=\detokenize{/home/fabio/Internodes-CM/.venv/lib/python3.8/site-packages/matplotlib/mpl-data/fonts/ttf/}]
%%   \setmonofont{DejaVuSansMono.ttf}[Path=\detokenize{/home/fabio/Internodes-CM/.venv/lib/python3.8/site-packages/matplotlib/mpl-data/fonts/ttf/}]
%%   \makeatletter\@ifpackageloaded{underscore}{}{\usepackage[strings]{underscore}}\makeatother
%%
\begingroup%
\makeatletter%
\begin{pgfpicture}%
\pgfpathrectangle{\pgfpointorigin}{\pgfqpoint{1.982500in}{1.432000in}}%
\pgfusepath{use as bounding box, clip}%
\begin{pgfscope}%
\pgfsetbuttcap%
\pgfsetmiterjoin%
\definecolor{currentfill}{rgb}{1.000000,1.000000,1.000000}%
\pgfsetfillcolor{currentfill}%
\pgfsetlinewidth{0.000000pt}%
\definecolor{currentstroke}{rgb}{1.000000,1.000000,1.000000}%
\pgfsetstrokecolor{currentstroke}%
\pgfsetdash{}{0pt}%
\pgfpathmoveto{\pgfqpoint{0.000000in}{0.000000in}}%
\pgfpathlineto{\pgfqpoint{1.982500in}{0.000000in}}%
\pgfpathlineto{\pgfqpoint{1.982500in}{1.432000in}}%
\pgfpathlineto{\pgfqpoint{0.000000in}{1.432000in}}%
\pgfpathlineto{\pgfqpoint{0.000000in}{0.000000in}}%
\pgfpathclose%
\pgfusepath{fill}%
\end{pgfscope}%
\begin{pgfscope}%
\pgfpathrectangle{\pgfqpoint{0.100000in}{0.100000in}}{\pgfqpoint{1.782500in}{1.232000in}}%
\pgfusepath{clip}%
\pgfsetrectcap%
\pgfsetroundjoin%
\pgfsetlinewidth{0.250937pt}%
\definecolor{currentstroke}{rgb}{0.054902,0.262745,0.486275}%
\pgfsetstrokecolor{currentstroke}%
\pgfsetdash{}{0pt}%
\pgfpathmoveto{\pgfqpoint{0.458635in}{0.100000in}}%
\pgfpathlineto{\pgfqpoint{0.190067in}{0.100000in}}%
\pgfpathmoveto{\pgfqpoint{0.727204in}{0.100000in}}%
\pgfpathlineto{\pgfqpoint{0.458635in}{0.100000in}}%
\pgfpathmoveto{\pgfqpoint{0.995772in}{0.100000in}}%
\pgfpathlineto{\pgfqpoint{0.727204in}{0.100000in}}%
\pgfpathmoveto{\pgfqpoint{1.264340in}{0.100000in}}%
\pgfpathlineto{\pgfqpoint{0.995772in}{0.100000in}}%
\pgfpathmoveto{\pgfqpoint{1.532909in}{0.100000in}}%
\pgfpathlineto{\pgfqpoint{1.801477in}{0.100000in}}%
\pgfpathmoveto{\pgfqpoint{1.532909in}{0.100000in}}%
\pgfpathlineto{\pgfqpoint{1.264340in}{0.100000in}}%
\pgfpathmoveto{\pgfqpoint{1.796841in}{0.415884in}}%
\pgfpathlineto{\pgfqpoint{1.801477in}{0.100000in}}%
\pgfpathmoveto{\pgfqpoint{1.796841in}{0.415884in}}%
\pgfpathlineto{\pgfqpoint{1.774230in}{0.729694in}}%
\pgfpathmoveto{\pgfqpoint{1.611368in}{0.716211in}}%
\pgfpathlineto{\pgfqpoint{1.774230in}{0.729694in}}%
\pgfpathmoveto{\pgfqpoint{1.611368in}{0.716211in}}%
\pgfpathlineto{\pgfqpoint{1.438592in}{0.697903in}}%
\pgfpathmoveto{\pgfqpoint{1.358413in}{0.681892in}}%
\pgfpathlineto{\pgfqpoint{1.438592in}{0.697903in}}%
\pgfpathmoveto{\pgfqpoint{1.300610in}{0.658103in}}%
\pgfpathlineto{\pgfqpoint{1.358413in}{0.681892in}}%
\pgfpathmoveto{\pgfqpoint{1.245113in}{0.640065in}}%
\pgfpathlineto{\pgfqpoint{1.300610in}{0.658103in}}%
\pgfpathmoveto{\pgfqpoint{1.187563in}{0.626883in}}%
\pgfpathlineto{\pgfqpoint{1.245113in}{0.640065in}}%
\pgfpathmoveto{\pgfqpoint{1.121983in}{0.616327in}}%
\pgfpathlineto{\pgfqpoint{1.187563in}{0.626883in}}%
\pgfpathmoveto{\pgfqpoint{1.054136in}{0.608017in}}%
\pgfpathlineto{\pgfqpoint{1.121983in}{0.616327in}}%
\pgfpathmoveto{\pgfqpoint{0.985700in}{0.601411in}}%
\pgfpathlineto{\pgfqpoint{1.054136in}{0.608017in}}%
\pgfpathmoveto{\pgfqpoint{0.916417in}{0.603252in}}%
\pgfpathlineto{\pgfqpoint{0.985700in}{0.601411in}}%
\pgfpathmoveto{\pgfqpoint{0.848834in}{0.613810in}}%
\pgfpathlineto{\pgfqpoint{0.916417in}{0.603252in}}%
\pgfpathmoveto{\pgfqpoint{0.783038in}{0.628109in}}%
\pgfpathlineto{\pgfqpoint{0.848834in}{0.613810in}}%
\pgfpathmoveto{\pgfqpoint{0.721820in}{0.643258in}}%
\pgfpathlineto{\pgfqpoint{0.783038in}{0.628109in}}%
\pgfpathmoveto{\pgfqpoint{0.661196in}{0.663049in}}%
\pgfpathlineto{\pgfqpoint{0.721820in}{0.643258in}}%
\pgfpathmoveto{\pgfqpoint{0.590312in}{0.684620in}}%
\pgfpathlineto{\pgfqpoint{0.517884in}{0.696169in}}%
\pgfpathmoveto{\pgfqpoint{0.590312in}{0.684620in}}%
\pgfpathlineto{\pgfqpoint{0.661196in}{0.663049in}}%
\pgfpathmoveto{\pgfqpoint{0.352544in}{0.709066in}}%
\pgfpathlineto{\pgfqpoint{0.517884in}{0.696169in}}%
\pgfpathmoveto{\pgfqpoint{0.352544in}{0.709066in}}%
\pgfpathlineto{\pgfqpoint{0.190332in}{0.716109in}}%
\pgfpathmoveto{\pgfqpoint{0.181023in}{0.405438in}}%
\pgfpathlineto{\pgfqpoint{0.190067in}{0.100000in}}%
\pgfpathmoveto{\pgfqpoint{0.181023in}{0.405438in}}%
\pgfpathlineto{\pgfqpoint{0.190332in}{0.716109in}}%
\pgfpathmoveto{\pgfqpoint{1.377207in}{0.371708in}}%
\pgfpathlineto{\pgfqpoint{1.264340in}{0.100000in}}%
\pgfpathmoveto{\pgfqpoint{1.083002in}{0.353907in}}%
\pgfpathlineto{\pgfqpoint{0.995772in}{0.100000in}}%
\pgfpathmoveto{\pgfqpoint{1.230784in}{0.463627in}}%
\pgfpathlineto{\pgfqpoint{1.377207in}{0.371708in}}%
\pgfpathmoveto{\pgfqpoint{1.230784in}{0.463627in}}%
\pgfpathlineto{\pgfqpoint{1.083002in}{0.353907in}}%
\pgfpathmoveto{\pgfqpoint{0.951674in}{0.449240in}}%
\pgfpathlineto{\pgfqpoint{1.083002in}{0.353907in}}%
\pgfpathmoveto{\pgfqpoint{0.951674in}{0.449240in}}%
\pgfpathlineto{\pgfqpoint{0.815316in}{0.370020in}}%
\pgfpathmoveto{\pgfqpoint{0.672687in}{0.479220in}}%
\pgfpathlineto{\pgfqpoint{0.815316in}{0.370020in}}%
\pgfpathmoveto{\pgfqpoint{0.672687in}{0.479220in}}%
\pgfpathlineto{\pgfqpoint{0.534657in}{0.384219in}}%
\pgfpathmoveto{\pgfqpoint{1.529457in}{0.488685in}}%
\pgfpathlineto{\pgfqpoint{1.796841in}{0.415884in}}%
\pgfpathmoveto{\pgfqpoint{1.529457in}{0.488685in}}%
\pgfpathlineto{\pgfqpoint{1.377207in}{0.371708in}}%
\pgfpathmoveto{\pgfqpoint{1.369377in}{0.514759in}}%
\pgfpathlineto{\pgfqpoint{1.377207in}{0.371708in}}%
\pgfpathmoveto{\pgfqpoint{1.369377in}{0.514759in}}%
\pgfpathlineto{\pgfqpoint{1.230784in}{0.463627in}}%
\pgfpathmoveto{\pgfqpoint{1.369377in}{0.514759in}}%
\pgfpathlineto{\pgfqpoint{1.529457in}{0.488685in}}%
\pgfpathmoveto{\pgfqpoint{1.089290in}{0.480619in}}%
\pgfpathlineto{\pgfqpoint{1.083002in}{0.353907in}}%
\pgfpathmoveto{\pgfqpoint{1.089290in}{0.480619in}}%
\pgfpathlineto{\pgfqpoint{1.230784in}{0.463627in}}%
\pgfpathmoveto{\pgfqpoint{1.089290in}{0.480619in}}%
\pgfpathlineto{\pgfqpoint{0.951674in}{0.449240in}}%
\pgfpathmoveto{\pgfqpoint{0.812716in}{0.489158in}}%
\pgfpathlineto{\pgfqpoint{0.815316in}{0.370020in}}%
\pgfpathmoveto{\pgfqpoint{0.812716in}{0.489158in}}%
\pgfpathlineto{\pgfqpoint{0.951674in}{0.449240in}}%
\pgfpathmoveto{\pgfqpoint{0.812716in}{0.489158in}}%
\pgfpathlineto{\pgfqpoint{0.672687in}{0.479220in}}%
\pgfpathmoveto{\pgfqpoint{0.523931in}{0.529297in}}%
\pgfpathlineto{\pgfqpoint{0.534657in}{0.384219in}}%
\pgfpathmoveto{\pgfqpoint{0.523931in}{0.529297in}}%
\pgfpathlineto{\pgfqpoint{0.672687in}{0.479220in}}%
\pgfpathmoveto{\pgfqpoint{0.356202in}{0.490685in}}%
\pgfpathlineto{\pgfqpoint{0.190332in}{0.716109in}}%
\pgfpathmoveto{\pgfqpoint{0.356202in}{0.490685in}}%
\pgfpathlineto{\pgfqpoint{0.352544in}{0.709066in}}%
\pgfpathmoveto{\pgfqpoint{0.356202in}{0.490685in}}%
\pgfpathlineto{\pgfqpoint{0.181023in}{0.405438in}}%
\pgfpathmoveto{\pgfqpoint{0.356202in}{0.490685in}}%
\pgfpathlineto{\pgfqpoint{0.534657in}{0.384219in}}%
\pgfpathmoveto{\pgfqpoint{0.356202in}{0.490685in}}%
\pgfpathlineto{\pgfqpoint{0.523931in}{0.529297in}}%
\pgfpathmoveto{\pgfqpoint{1.289734in}{0.563347in}}%
\pgfpathlineto{\pgfqpoint{1.300610in}{0.658103in}}%
\pgfpathmoveto{\pgfqpoint{1.289734in}{0.563347in}}%
\pgfpathlineto{\pgfqpoint{1.245113in}{0.640065in}}%
\pgfpathmoveto{\pgfqpoint{1.289734in}{0.563347in}}%
\pgfpathlineto{\pgfqpoint{1.230784in}{0.463627in}}%
\pgfpathmoveto{\pgfqpoint{1.289734in}{0.563347in}}%
\pgfpathlineto{\pgfqpoint{1.369377in}{0.514759in}}%
\pgfpathmoveto{\pgfqpoint{1.157523in}{0.542874in}}%
\pgfpathlineto{\pgfqpoint{1.187563in}{0.626883in}}%
\pgfpathmoveto{\pgfqpoint{1.157523in}{0.542874in}}%
\pgfpathlineto{\pgfqpoint{1.121983in}{0.616327in}}%
\pgfpathmoveto{\pgfqpoint{1.157523in}{0.542874in}}%
\pgfpathlineto{\pgfqpoint{1.230784in}{0.463627in}}%
\pgfpathmoveto{\pgfqpoint{1.157523in}{0.542874in}}%
\pgfpathlineto{\pgfqpoint{1.089290in}{0.480619in}}%
\pgfpathmoveto{\pgfqpoint{1.021299in}{0.531634in}}%
\pgfpathlineto{\pgfqpoint{1.054136in}{0.608017in}}%
\pgfpathmoveto{\pgfqpoint{1.021299in}{0.531634in}}%
\pgfpathlineto{\pgfqpoint{0.985700in}{0.601411in}}%
\pgfpathmoveto{\pgfqpoint{1.021299in}{0.531634in}}%
\pgfpathlineto{\pgfqpoint{0.951674in}{0.449240in}}%
\pgfpathmoveto{\pgfqpoint{1.021299in}{0.531634in}}%
\pgfpathlineto{\pgfqpoint{1.089290in}{0.480619in}}%
\pgfpathmoveto{\pgfqpoint{0.881689in}{0.534959in}}%
\pgfpathlineto{\pgfqpoint{0.916417in}{0.603252in}}%
\pgfpathmoveto{\pgfqpoint{0.881689in}{0.534959in}}%
\pgfpathlineto{\pgfqpoint{0.848834in}{0.613810in}}%
\pgfpathmoveto{\pgfqpoint{0.881689in}{0.534959in}}%
\pgfpathlineto{\pgfqpoint{0.951674in}{0.449240in}}%
\pgfpathmoveto{\pgfqpoint{0.881689in}{0.534959in}}%
\pgfpathlineto{\pgfqpoint{0.812716in}{0.489158in}}%
\pgfpathmoveto{\pgfqpoint{0.745570in}{0.554174in}}%
\pgfpathlineto{\pgfqpoint{0.783038in}{0.628109in}}%
\pgfpathmoveto{\pgfqpoint{0.745570in}{0.554174in}}%
\pgfpathlineto{\pgfqpoint{0.721820in}{0.643258in}}%
\pgfpathmoveto{\pgfqpoint{0.745570in}{0.554174in}}%
\pgfpathlineto{\pgfqpoint{0.672687in}{0.479220in}}%
\pgfpathmoveto{\pgfqpoint{0.745570in}{0.554174in}}%
\pgfpathlineto{\pgfqpoint{0.812716in}{0.489158in}}%
\pgfpathmoveto{\pgfqpoint{0.610324in}{0.585562in}}%
\pgfpathlineto{\pgfqpoint{0.661196in}{0.663049in}}%
\pgfpathmoveto{\pgfqpoint{0.610324in}{0.585562in}}%
\pgfpathlineto{\pgfqpoint{0.590312in}{0.684620in}}%
\pgfpathmoveto{\pgfqpoint{0.610324in}{0.585562in}}%
\pgfpathlineto{\pgfqpoint{0.672687in}{0.479220in}}%
\pgfpathmoveto{\pgfqpoint{0.610324in}{0.585562in}}%
\pgfpathlineto{\pgfqpoint{0.523931in}{0.529297in}}%
\pgfpathmoveto{\pgfqpoint{1.433326in}{0.603291in}}%
\pgfpathlineto{\pgfqpoint{1.438592in}{0.697903in}}%
\pgfpathmoveto{\pgfqpoint{1.433326in}{0.603291in}}%
\pgfpathlineto{\pgfqpoint{1.358413in}{0.681892in}}%
\pgfpathmoveto{\pgfqpoint{1.433326in}{0.603291in}}%
\pgfpathlineto{\pgfqpoint{1.529457in}{0.488685in}}%
\pgfpathmoveto{\pgfqpoint{1.433326in}{0.603291in}}%
\pgfpathlineto{\pgfqpoint{1.369377in}{0.514759in}}%
\pgfpathmoveto{\pgfqpoint{0.938271in}{0.294479in}}%
\pgfpathlineto{\pgfqpoint{0.995772in}{0.100000in}}%
\pgfpathmoveto{\pgfqpoint{0.938271in}{0.294479in}}%
\pgfpathlineto{\pgfqpoint{1.083002in}{0.353907in}}%
\pgfpathmoveto{\pgfqpoint{0.938271in}{0.294479in}}%
\pgfpathlineto{\pgfqpoint{0.815316in}{0.370020in}}%
\pgfpathmoveto{\pgfqpoint{0.938271in}{0.294479in}}%
\pgfpathlineto{\pgfqpoint{0.951674in}{0.449240in}}%
\pgfpathmoveto{\pgfqpoint{0.437987in}{0.601518in}}%
\pgfpathlineto{\pgfqpoint{0.517884in}{0.696169in}}%
\pgfpathmoveto{\pgfqpoint{0.437987in}{0.601518in}}%
\pgfpathlineto{\pgfqpoint{0.352544in}{0.709066in}}%
\pgfpathmoveto{\pgfqpoint{0.437987in}{0.601518in}}%
\pgfpathlineto{\pgfqpoint{0.523931in}{0.529297in}}%
\pgfpathmoveto{\pgfqpoint{0.437987in}{0.601518in}}%
\pgfpathlineto{\pgfqpoint{0.356202in}{0.490685in}}%
\pgfpathmoveto{\pgfqpoint{1.351528in}{0.601963in}}%
\pgfpathlineto{\pgfqpoint{1.358413in}{0.681892in}}%
\pgfpathmoveto{\pgfqpoint{1.351528in}{0.601963in}}%
\pgfpathlineto{\pgfqpoint{1.300610in}{0.658103in}}%
\pgfpathmoveto{\pgfqpoint{1.351528in}{0.601963in}}%
\pgfpathlineto{\pgfqpoint{1.369377in}{0.514759in}}%
\pgfpathmoveto{\pgfqpoint{1.351528in}{0.601963in}}%
\pgfpathlineto{\pgfqpoint{1.289734in}{0.563347in}}%
\pgfpathmoveto{\pgfqpoint{1.351528in}{0.601963in}}%
\pgfpathlineto{\pgfqpoint{1.433326in}{0.603291in}}%
\pgfpathmoveto{\pgfqpoint{1.088974in}{0.555599in}}%
\pgfpathlineto{\pgfqpoint{1.121983in}{0.616327in}}%
\pgfpathmoveto{\pgfqpoint{1.088974in}{0.555599in}}%
\pgfpathlineto{\pgfqpoint{1.054136in}{0.608017in}}%
\pgfpathmoveto{\pgfqpoint{1.088974in}{0.555599in}}%
\pgfpathlineto{\pgfqpoint{1.089290in}{0.480619in}}%
\pgfpathmoveto{\pgfqpoint{1.088974in}{0.555599in}}%
\pgfpathlineto{\pgfqpoint{1.157523in}{0.542874in}}%
\pgfpathmoveto{\pgfqpoint{1.088974in}{0.555599in}}%
\pgfpathlineto{\pgfqpoint{1.021299in}{0.531634in}}%
\pgfpathmoveto{\pgfqpoint{1.222194in}{0.569805in}}%
\pgfpathlineto{\pgfqpoint{1.245113in}{0.640065in}}%
\pgfpathmoveto{\pgfqpoint{1.222194in}{0.569805in}}%
\pgfpathlineto{\pgfqpoint{1.187563in}{0.626883in}}%
\pgfpathmoveto{\pgfqpoint{1.222194in}{0.569805in}}%
\pgfpathlineto{\pgfqpoint{1.230784in}{0.463627in}}%
\pgfpathmoveto{\pgfqpoint{1.222194in}{0.569805in}}%
\pgfpathlineto{\pgfqpoint{1.289734in}{0.563347in}}%
\pgfpathmoveto{\pgfqpoint{1.222194in}{0.569805in}}%
\pgfpathlineto{\pgfqpoint{1.157523in}{0.542874in}}%
\pgfpathmoveto{\pgfqpoint{0.951314in}{0.543497in}}%
\pgfpathlineto{\pgfqpoint{0.985700in}{0.601411in}}%
\pgfpathmoveto{\pgfqpoint{0.951314in}{0.543497in}}%
\pgfpathlineto{\pgfqpoint{0.916417in}{0.603252in}}%
\pgfpathmoveto{\pgfqpoint{0.951314in}{0.543497in}}%
\pgfpathlineto{\pgfqpoint{0.951674in}{0.449240in}}%
\pgfpathmoveto{\pgfqpoint{0.951314in}{0.543497in}}%
\pgfpathlineto{\pgfqpoint{1.021299in}{0.531634in}}%
\pgfpathmoveto{\pgfqpoint{0.951314in}{0.543497in}}%
\pgfpathlineto{\pgfqpoint{0.881689in}{0.534959in}}%
\pgfpathmoveto{\pgfqpoint{0.813693in}{0.562058in}}%
\pgfpathlineto{\pgfqpoint{0.848834in}{0.613810in}}%
\pgfpathmoveto{\pgfqpoint{0.813693in}{0.562058in}}%
\pgfpathlineto{\pgfqpoint{0.783038in}{0.628109in}}%
\pgfpathmoveto{\pgfqpoint{0.813693in}{0.562058in}}%
\pgfpathlineto{\pgfqpoint{0.812716in}{0.489158in}}%
\pgfpathmoveto{\pgfqpoint{0.813693in}{0.562058in}}%
\pgfpathlineto{\pgfqpoint{0.881689in}{0.534959in}}%
\pgfpathmoveto{\pgfqpoint{0.813693in}{0.562058in}}%
\pgfpathlineto{\pgfqpoint{0.745570in}{0.554174in}}%
\pgfpathmoveto{\pgfqpoint{1.192290in}{0.275687in}}%
\pgfpathlineto{\pgfqpoint{0.995772in}{0.100000in}}%
\pgfpathmoveto{\pgfqpoint{1.192290in}{0.275687in}}%
\pgfpathlineto{\pgfqpoint{1.264340in}{0.100000in}}%
\pgfpathmoveto{\pgfqpoint{1.192290in}{0.275687in}}%
\pgfpathlineto{\pgfqpoint{1.377207in}{0.371708in}}%
\pgfpathmoveto{\pgfqpoint{1.192290in}{0.275687in}}%
\pgfpathlineto{\pgfqpoint{1.083002in}{0.353907in}}%
\pgfpathmoveto{\pgfqpoint{1.192290in}{0.275687in}}%
\pgfpathlineto{\pgfqpoint{1.230784in}{0.463627in}}%
\pgfpathmoveto{\pgfqpoint{0.691426in}{0.290497in}}%
\pgfpathlineto{\pgfqpoint{0.727204in}{0.100000in}}%
\pgfpathmoveto{\pgfqpoint{0.691426in}{0.290497in}}%
\pgfpathlineto{\pgfqpoint{0.815316in}{0.370020in}}%
\pgfpathmoveto{\pgfqpoint{0.691426in}{0.290497in}}%
\pgfpathlineto{\pgfqpoint{0.534657in}{0.384219in}}%
\pgfpathmoveto{\pgfqpoint{0.691426in}{0.290497in}}%
\pgfpathlineto{\pgfqpoint{0.672687in}{0.479220in}}%
\pgfpathmoveto{\pgfqpoint{0.680324in}{0.583817in}}%
\pgfpathlineto{\pgfqpoint{0.721820in}{0.643258in}}%
\pgfpathmoveto{\pgfqpoint{0.680324in}{0.583817in}}%
\pgfpathlineto{\pgfqpoint{0.661196in}{0.663049in}}%
\pgfpathmoveto{\pgfqpoint{0.680324in}{0.583817in}}%
\pgfpathlineto{\pgfqpoint{0.672687in}{0.479220in}}%
\pgfpathmoveto{\pgfqpoint{0.680324in}{0.583817in}}%
\pgfpathlineto{\pgfqpoint{0.745570in}{0.554174in}}%
\pgfpathmoveto{\pgfqpoint{0.680324in}{0.583817in}}%
\pgfpathlineto{\pgfqpoint{0.610324in}{0.585562in}}%
\pgfpathmoveto{\pgfqpoint{1.535895in}{0.617892in}}%
\pgfpathlineto{\pgfqpoint{1.438592in}{0.697903in}}%
\pgfpathmoveto{\pgfqpoint{1.535895in}{0.617892in}}%
\pgfpathlineto{\pgfqpoint{1.611368in}{0.716211in}}%
\pgfpathmoveto{\pgfqpoint{1.535895in}{0.617892in}}%
\pgfpathlineto{\pgfqpoint{1.529457in}{0.488685in}}%
\pgfpathmoveto{\pgfqpoint{1.535895in}{0.617892in}}%
\pgfpathlineto{\pgfqpoint{1.433326in}{0.603291in}}%
\pgfpathmoveto{\pgfqpoint{0.545528in}{0.620176in}}%
\pgfpathlineto{\pgfqpoint{0.517884in}{0.696169in}}%
\pgfpathmoveto{\pgfqpoint{0.545528in}{0.620176in}}%
\pgfpathlineto{\pgfqpoint{0.590312in}{0.684620in}}%
\pgfpathmoveto{\pgfqpoint{0.545528in}{0.620176in}}%
\pgfpathlineto{\pgfqpoint{0.523931in}{0.529297in}}%
\pgfpathmoveto{\pgfqpoint{0.545528in}{0.620176in}}%
\pgfpathlineto{\pgfqpoint{0.610324in}{0.585562in}}%
\pgfpathmoveto{\pgfqpoint{0.545528in}{0.620176in}}%
\pgfpathlineto{\pgfqpoint{0.437987in}{0.601518in}}%
\pgfpathmoveto{\pgfqpoint{0.379348in}{0.283472in}}%
\pgfpathlineto{\pgfqpoint{0.190067in}{0.100000in}}%
\pgfpathmoveto{\pgfqpoint{0.379348in}{0.283472in}}%
\pgfpathlineto{\pgfqpoint{0.458635in}{0.100000in}}%
\pgfpathmoveto{\pgfqpoint{0.379348in}{0.283472in}}%
\pgfpathlineto{\pgfqpoint{0.181023in}{0.405438in}}%
\pgfpathmoveto{\pgfqpoint{0.379348in}{0.283472in}}%
\pgfpathlineto{\pgfqpoint{0.534657in}{0.384219in}}%
\pgfpathmoveto{\pgfqpoint{0.379348in}{0.283472in}}%
\pgfpathlineto{\pgfqpoint{0.356202in}{0.490685in}}%
\pgfpathmoveto{\pgfqpoint{1.533495in}{0.293969in}}%
\pgfpathlineto{\pgfqpoint{1.801477in}{0.100000in}}%
\pgfpathmoveto{\pgfqpoint{1.533495in}{0.293969in}}%
\pgfpathlineto{\pgfqpoint{1.264340in}{0.100000in}}%
\pgfpathmoveto{\pgfqpoint{1.533495in}{0.293969in}}%
\pgfpathlineto{\pgfqpoint{1.532909in}{0.100000in}}%
\pgfpathmoveto{\pgfqpoint{1.533495in}{0.293969in}}%
\pgfpathlineto{\pgfqpoint{1.796841in}{0.415884in}}%
\pgfpathmoveto{\pgfqpoint{1.533495in}{0.293969in}}%
\pgfpathlineto{\pgfqpoint{1.377207in}{0.371708in}}%
\pgfpathmoveto{\pgfqpoint{1.533495in}{0.293969in}}%
\pgfpathlineto{\pgfqpoint{1.529457in}{0.488685in}}%
\pgfpathmoveto{\pgfqpoint{1.649933in}{0.594471in}}%
\pgfpathlineto{\pgfqpoint{1.774230in}{0.729694in}}%
\pgfpathmoveto{\pgfqpoint{1.649933in}{0.594471in}}%
\pgfpathlineto{\pgfqpoint{1.796841in}{0.415884in}}%
\pgfpathmoveto{\pgfqpoint{1.649933in}{0.594471in}}%
\pgfpathlineto{\pgfqpoint{1.611368in}{0.716211in}}%
\pgfpathmoveto{\pgfqpoint{1.649933in}{0.594471in}}%
\pgfpathlineto{\pgfqpoint{1.529457in}{0.488685in}}%
\pgfpathmoveto{\pgfqpoint{1.649933in}{0.594471in}}%
\pgfpathlineto{\pgfqpoint{1.535895in}{0.617892in}}%
\pgfpathmoveto{\pgfqpoint{0.841552in}{0.212299in}}%
\pgfpathlineto{\pgfqpoint{0.727204in}{0.100000in}}%
\pgfpathmoveto{\pgfqpoint{0.841552in}{0.212299in}}%
\pgfpathlineto{\pgfqpoint{0.995772in}{0.100000in}}%
\pgfpathmoveto{\pgfqpoint{0.841552in}{0.212299in}}%
\pgfpathlineto{\pgfqpoint{0.815316in}{0.370020in}}%
\pgfpathmoveto{\pgfqpoint{0.841552in}{0.212299in}}%
\pgfpathlineto{\pgfqpoint{0.938271in}{0.294479in}}%
\pgfpathmoveto{\pgfqpoint{0.841552in}{0.212299in}}%
\pgfpathlineto{\pgfqpoint{0.691426in}{0.290497in}}%
\pgfpathmoveto{\pgfqpoint{0.556128in}{0.231085in}}%
\pgfpathlineto{\pgfqpoint{0.458635in}{0.100000in}}%
\pgfpathmoveto{\pgfqpoint{0.556128in}{0.231085in}}%
\pgfpathlineto{\pgfqpoint{0.727204in}{0.100000in}}%
\pgfpathmoveto{\pgfqpoint{0.556128in}{0.231085in}}%
\pgfpathlineto{\pgfqpoint{0.534657in}{0.384219in}}%
\pgfpathmoveto{\pgfqpoint{0.556128in}{0.231085in}}%
\pgfpathlineto{\pgfqpoint{0.691426in}{0.290497in}}%
\pgfpathmoveto{\pgfqpoint{0.556128in}{0.231085in}}%
\pgfpathlineto{\pgfqpoint{0.379348in}{0.283472in}}%
\pgfpathlineto{\pgfqpoint{0.379348in}{0.283472in}}%
\pgfusepath{stroke}%
\end{pgfscope}%
\begin{pgfscope}%
\pgfpathrectangle{\pgfqpoint{0.100000in}{0.100000in}}{\pgfqpoint{1.782500in}{1.232000in}}%
\pgfusepath{clip}%
\pgfsetrectcap%
\pgfsetroundjoin%
\pgfsetlinewidth{0.250937pt}%
\definecolor{currentstroke}{rgb}{0.835294,0.321569,0.035294}%
\pgfsetstrokecolor{currentstroke}%
\pgfsetdash{}{0pt}%
\pgfpathmoveto{\pgfqpoint{0.488920in}{0.850184in}}%
\pgfpathlineto{\pgfqpoint{0.458635in}{1.085600in}}%
\pgfpathmoveto{\pgfqpoint{1.532909in}{1.085600in}}%
\pgfpathlineto{\pgfqpoint{1.494399in}{0.840973in}}%
\pgfpathmoveto{\pgfqpoint{0.727204in}{1.085600in}}%
\pgfpathlineto{\pgfqpoint{0.995772in}{1.085600in}}%
\pgfpathmoveto{\pgfqpoint{0.727204in}{1.085600in}}%
\pgfpathlineto{\pgfqpoint{0.458635in}{1.085600in}}%
\pgfpathmoveto{\pgfqpoint{0.529202in}{0.776277in}}%
\pgfpathlineto{\pgfqpoint{0.488920in}{0.850184in}}%
\pgfpathmoveto{\pgfqpoint{0.576443in}{0.718597in}}%
\pgfpathlineto{\pgfqpoint{0.529202in}{0.776277in}}%
\pgfpathmoveto{\pgfqpoint{0.649179in}{0.669718in}}%
\pgfpathlineto{\pgfqpoint{0.576443in}{0.718597in}}%
\pgfpathmoveto{\pgfqpoint{0.742562in}{0.637869in}}%
\pgfpathlineto{\pgfqpoint{0.649179in}{0.669718in}}%
\pgfpathmoveto{\pgfqpoint{0.828308in}{0.618003in}}%
\pgfpathlineto{\pgfqpoint{0.742562in}{0.637869in}}%
\pgfpathmoveto{\pgfqpoint{0.907725in}{0.603940in}}%
\pgfpathlineto{\pgfqpoint{0.828308in}{0.618003in}}%
\pgfpathmoveto{\pgfqpoint{0.985741in}{0.601166in}}%
\pgfpathlineto{\pgfqpoint{0.907725in}{0.603940in}}%
\pgfpathmoveto{\pgfqpoint{1.062700in}{0.609075in}}%
\pgfpathlineto{\pgfqpoint{0.985741in}{0.601166in}}%
\pgfpathmoveto{\pgfqpoint{1.142662in}{0.619169in}}%
\pgfpathlineto{\pgfqpoint{1.062700in}{0.609075in}}%
\pgfpathmoveto{\pgfqpoint{1.226275in}{0.635125in}}%
\pgfpathlineto{\pgfqpoint{1.142662in}{0.619169in}}%
\pgfpathmoveto{\pgfqpoint{1.313856in}{0.663445in}}%
\pgfpathlineto{\pgfqpoint{1.226275in}{0.635125in}}%
\pgfpathmoveto{\pgfqpoint{1.401434in}{0.701100in}}%
\pgfpathlineto{\pgfqpoint{1.313856in}{0.663445in}}%
\pgfpathmoveto{\pgfqpoint{1.451289in}{0.764757in}}%
\pgfpathlineto{\pgfqpoint{1.494399in}{0.840973in}}%
\pgfpathmoveto{\pgfqpoint{1.451289in}{0.764757in}}%
\pgfpathlineto{\pgfqpoint{1.401434in}{0.701100in}}%
\pgfpathmoveto{\pgfqpoint{1.264340in}{1.085600in}}%
\pgfpathlineto{\pgfqpoint{0.995772in}{1.085600in}}%
\pgfpathmoveto{\pgfqpoint{1.264340in}{1.085600in}}%
\pgfpathlineto{\pgfqpoint{1.532909in}{1.085600in}}%
\pgfpathmoveto{\pgfqpoint{0.830943in}{0.887574in}}%
\pgfpathlineto{\pgfqpoint{0.995772in}{1.085600in}}%
\pgfpathmoveto{\pgfqpoint{0.830943in}{0.887574in}}%
\pgfpathlineto{\pgfqpoint{0.727204in}{1.085600in}}%
\pgfpathmoveto{\pgfqpoint{0.957322in}{0.756420in}}%
\pgfpathlineto{\pgfqpoint{1.117688in}{0.824660in}}%
\pgfpathmoveto{\pgfqpoint{0.957322in}{0.756420in}}%
\pgfpathlineto{\pgfqpoint{0.830943in}{0.887574in}}%
\pgfpathmoveto{\pgfqpoint{1.287084in}{0.857316in}}%
\pgfpathlineto{\pgfqpoint{1.117688in}{0.824660in}}%
\pgfpathmoveto{\pgfqpoint{0.665556in}{0.898299in}}%
\pgfpathlineto{\pgfqpoint{0.727204in}{1.085600in}}%
\pgfpathmoveto{\pgfqpoint{0.665556in}{0.898299in}}%
\pgfpathlineto{\pgfqpoint{0.830943in}{0.887574in}}%
\pgfpathmoveto{\pgfqpoint{1.090371in}{0.720244in}}%
\pgfpathlineto{\pgfqpoint{1.062700in}{0.609075in}}%
\pgfpathmoveto{\pgfqpoint{1.090371in}{0.720244in}}%
\pgfpathlineto{\pgfqpoint{1.142662in}{0.619169in}}%
\pgfpathmoveto{\pgfqpoint{1.090371in}{0.720244in}}%
\pgfpathlineto{\pgfqpoint{1.117688in}{0.824660in}}%
\pgfpathmoveto{\pgfqpoint{1.090371in}{0.720244in}}%
\pgfpathlineto{\pgfqpoint{0.957322in}{0.756420in}}%
\pgfpathmoveto{\pgfqpoint{1.219108in}{0.751293in}}%
\pgfpathlineto{\pgfqpoint{1.226275in}{0.635125in}}%
\pgfpathmoveto{\pgfqpoint{1.219108in}{0.751293in}}%
\pgfpathlineto{\pgfqpoint{1.313856in}{0.663445in}}%
\pgfpathmoveto{\pgfqpoint{1.219108in}{0.751293in}}%
\pgfpathlineto{\pgfqpoint{1.117688in}{0.824660in}}%
\pgfpathmoveto{\pgfqpoint{1.219108in}{0.751293in}}%
\pgfpathlineto{\pgfqpoint{1.287084in}{0.857316in}}%
\pgfpathmoveto{\pgfqpoint{1.219108in}{0.751293in}}%
\pgfpathlineto{\pgfqpoint{1.090371in}{0.720244in}}%
\pgfpathmoveto{\pgfqpoint{0.812724in}{0.731232in}}%
\pgfpathlineto{\pgfqpoint{0.742562in}{0.637869in}}%
\pgfpathmoveto{\pgfqpoint{0.812724in}{0.731232in}}%
\pgfpathlineto{\pgfqpoint{0.828308in}{0.618003in}}%
\pgfpathmoveto{\pgfqpoint{0.812724in}{0.731232in}}%
\pgfpathlineto{\pgfqpoint{0.830943in}{0.887574in}}%
\pgfpathmoveto{\pgfqpoint{0.812724in}{0.731232in}}%
\pgfpathlineto{\pgfqpoint{0.957322in}{0.756420in}}%
\pgfpathmoveto{\pgfqpoint{0.701489in}{0.784036in}}%
\pgfpathlineto{\pgfqpoint{0.576443in}{0.718597in}}%
\pgfpathmoveto{\pgfqpoint{0.701489in}{0.784036in}}%
\pgfpathlineto{\pgfqpoint{0.649179in}{0.669718in}}%
\pgfpathmoveto{\pgfqpoint{0.701489in}{0.784036in}}%
\pgfpathlineto{\pgfqpoint{0.830943in}{0.887574in}}%
\pgfpathmoveto{\pgfqpoint{0.701489in}{0.784036in}}%
\pgfpathlineto{\pgfqpoint{0.665556in}{0.898299in}}%
\pgfpathmoveto{\pgfqpoint{0.701489in}{0.784036in}}%
\pgfpathlineto{\pgfqpoint{0.812724in}{0.731232in}}%
\pgfpathmoveto{\pgfqpoint{1.373352in}{0.959810in}}%
\pgfpathlineto{\pgfqpoint{1.494399in}{0.840973in}}%
\pgfpathmoveto{\pgfqpoint{1.373352in}{0.959810in}}%
\pgfpathlineto{\pgfqpoint{1.532909in}{1.085600in}}%
\pgfpathmoveto{\pgfqpoint{1.373352in}{0.959810in}}%
\pgfpathlineto{\pgfqpoint{1.264340in}{1.085600in}}%
\pgfpathmoveto{\pgfqpoint{1.373352in}{0.959810in}}%
\pgfpathlineto{\pgfqpoint{1.287084in}{0.857316in}}%
\pgfpathmoveto{\pgfqpoint{0.949513in}{0.659395in}}%
\pgfpathlineto{\pgfqpoint{0.907725in}{0.603940in}}%
\pgfpathmoveto{\pgfqpoint{0.949513in}{0.659395in}}%
\pgfpathlineto{\pgfqpoint{0.985741in}{0.601166in}}%
\pgfpathmoveto{\pgfqpoint{0.949513in}{0.659395in}}%
\pgfpathlineto{\pgfqpoint{0.957322in}{0.756420in}}%
\pgfpathmoveto{\pgfqpoint{1.310269in}{0.744148in}}%
\pgfpathlineto{\pgfqpoint{1.313856in}{0.663445in}}%
\pgfpathmoveto{\pgfqpoint{1.310269in}{0.744148in}}%
\pgfpathlineto{\pgfqpoint{1.401434in}{0.701100in}}%
\pgfpathmoveto{\pgfqpoint{1.310269in}{0.744148in}}%
\pgfpathlineto{\pgfqpoint{1.287084in}{0.857316in}}%
\pgfpathmoveto{\pgfqpoint{1.310269in}{0.744148in}}%
\pgfpathlineto{\pgfqpoint{1.219108in}{0.751293in}}%
\pgfpathmoveto{\pgfqpoint{1.167827in}{0.689125in}}%
\pgfpathlineto{\pgfqpoint{1.142662in}{0.619169in}}%
\pgfpathmoveto{\pgfqpoint{1.167827in}{0.689125in}}%
\pgfpathlineto{\pgfqpoint{1.226275in}{0.635125in}}%
\pgfpathmoveto{\pgfqpoint{1.167827in}{0.689125in}}%
\pgfpathlineto{\pgfqpoint{1.090371in}{0.720244in}}%
\pgfpathmoveto{\pgfqpoint{1.167827in}{0.689125in}}%
\pgfpathlineto{\pgfqpoint{1.219108in}{0.751293in}}%
\pgfpathmoveto{\pgfqpoint{1.390634in}{0.855971in}}%
\pgfpathlineto{\pgfqpoint{1.494399in}{0.840973in}}%
\pgfpathmoveto{\pgfqpoint{1.390634in}{0.855971in}}%
\pgfpathlineto{\pgfqpoint{1.451289in}{0.764757in}}%
\pgfpathmoveto{\pgfqpoint{1.390634in}{0.855971in}}%
\pgfpathlineto{\pgfqpoint{1.287084in}{0.857316in}}%
\pgfpathmoveto{\pgfqpoint{1.390634in}{0.855971in}}%
\pgfpathlineto{\pgfqpoint{1.373352in}{0.959810in}}%
\pgfpathmoveto{\pgfqpoint{0.728960in}{0.711312in}}%
\pgfpathlineto{\pgfqpoint{0.649179in}{0.669718in}}%
\pgfpathmoveto{\pgfqpoint{0.728960in}{0.711312in}}%
\pgfpathlineto{\pgfqpoint{0.742562in}{0.637869in}}%
\pgfpathmoveto{\pgfqpoint{0.728960in}{0.711312in}}%
\pgfpathlineto{\pgfqpoint{0.812724in}{0.731232in}}%
\pgfpathmoveto{\pgfqpoint{0.728960in}{0.711312in}}%
\pgfpathlineto{\pgfqpoint{0.701489in}{0.784036in}}%
\pgfpathmoveto{\pgfqpoint{0.616584in}{0.797742in}}%
\pgfpathlineto{\pgfqpoint{0.529202in}{0.776277in}}%
\pgfpathmoveto{\pgfqpoint{0.616584in}{0.797742in}}%
\pgfpathlineto{\pgfqpoint{0.576443in}{0.718597in}}%
\pgfpathmoveto{\pgfqpoint{0.616584in}{0.797742in}}%
\pgfpathlineto{\pgfqpoint{0.665556in}{0.898299in}}%
\pgfpathmoveto{\pgfqpoint{0.616584in}{0.797742in}}%
\pgfpathlineto{\pgfqpoint{0.701489in}{0.784036in}}%
\pgfpathmoveto{\pgfqpoint{0.567574in}{0.969290in}}%
\pgfpathlineto{\pgfqpoint{0.458635in}{1.085600in}}%
\pgfpathmoveto{\pgfqpoint{0.567574in}{0.969290in}}%
\pgfpathlineto{\pgfqpoint{0.488920in}{0.850184in}}%
\pgfpathmoveto{\pgfqpoint{0.567574in}{0.969290in}}%
\pgfpathlineto{\pgfqpoint{0.727204in}{1.085600in}}%
\pgfpathmoveto{\pgfqpoint{0.567574in}{0.969290in}}%
\pgfpathlineto{\pgfqpoint{0.665556in}{0.898299in}}%
\pgfpathmoveto{\pgfqpoint{1.022401in}{0.663058in}}%
\pgfpathlineto{\pgfqpoint{0.985741in}{0.601166in}}%
\pgfpathmoveto{\pgfqpoint{1.022401in}{0.663058in}}%
\pgfpathlineto{\pgfqpoint{1.062700in}{0.609075in}}%
\pgfpathmoveto{\pgfqpoint{1.022401in}{0.663058in}}%
\pgfpathlineto{\pgfqpoint{0.957322in}{0.756420in}}%
\pgfpathmoveto{\pgfqpoint{1.022401in}{0.663058in}}%
\pgfpathlineto{\pgfqpoint{1.090371in}{0.720244in}}%
\pgfpathmoveto{\pgfqpoint{1.022401in}{0.663058in}}%
\pgfpathlineto{\pgfqpoint{0.949513in}{0.659395in}}%
\pgfpathmoveto{\pgfqpoint{0.876764in}{0.668875in}}%
\pgfpathlineto{\pgfqpoint{0.828308in}{0.618003in}}%
\pgfpathmoveto{\pgfqpoint{0.876764in}{0.668875in}}%
\pgfpathlineto{\pgfqpoint{0.907725in}{0.603940in}}%
\pgfpathmoveto{\pgfqpoint{0.876764in}{0.668875in}}%
\pgfpathlineto{\pgfqpoint{0.957322in}{0.756420in}}%
\pgfpathmoveto{\pgfqpoint{0.876764in}{0.668875in}}%
\pgfpathlineto{\pgfqpoint{0.812724in}{0.731232in}}%
\pgfpathmoveto{\pgfqpoint{0.876764in}{0.668875in}}%
\pgfpathlineto{\pgfqpoint{0.949513in}{0.659395in}}%
\pgfpathmoveto{\pgfqpoint{1.374566in}{0.778570in}}%
\pgfpathlineto{\pgfqpoint{1.401434in}{0.701100in}}%
\pgfpathmoveto{\pgfqpoint{1.374566in}{0.778570in}}%
\pgfpathlineto{\pgfqpoint{1.451289in}{0.764757in}}%
\pgfpathmoveto{\pgfqpoint{1.374566in}{0.778570in}}%
\pgfpathlineto{\pgfqpoint{1.287084in}{0.857316in}}%
\pgfpathmoveto{\pgfqpoint{1.374566in}{0.778570in}}%
\pgfpathlineto{\pgfqpoint{1.310269in}{0.744148in}}%
\pgfpathmoveto{\pgfqpoint{1.374566in}{0.778570in}}%
\pgfpathlineto{\pgfqpoint{1.390634in}{0.855971in}}%
\pgfpathmoveto{\pgfqpoint{1.016954in}{0.905905in}}%
\pgfpathlineto{\pgfqpoint{0.995772in}{1.085600in}}%
\pgfpathmoveto{\pgfqpoint{1.016954in}{0.905905in}}%
\pgfpathlineto{\pgfqpoint{1.117688in}{0.824660in}}%
\pgfpathmoveto{\pgfqpoint{1.016954in}{0.905905in}}%
\pgfpathlineto{\pgfqpoint{0.830943in}{0.887574in}}%
\pgfpathmoveto{\pgfqpoint{1.016954in}{0.905905in}}%
\pgfpathlineto{\pgfqpoint{0.957322in}{0.756420in}}%
\pgfpathmoveto{\pgfqpoint{1.178001in}{0.952692in}}%
\pgfpathlineto{\pgfqpoint{0.995772in}{1.085600in}}%
\pgfpathmoveto{\pgfqpoint{1.178001in}{0.952692in}}%
\pgfpathlineto{\pgfqpoint{1.264340in}{1.085600in}}%
\pgfpathmoveto{\pgfqpoint{1.178001in}{0.952692in}}%
\pgfpathlineto{\pgfqpoint{1.117688in}{0.824660in}}%
\pgfpathmoveto{\pgfqpoint{1.178001in}{0.952692in}}%
\pgfpathlineto{\pgfqpoint{1.287084in}{0.857316in}}%
\pgfpathmoveto{\pgfqpoint{1.178001in}{0.952692in}}%
\pgfpathlineto{\pgfqpoint{1.373352in}{0.959810in}}%
\pgfpathmoveto{\pgfqpoint{1.178001in}{0.952692in}}%
\pgfpathlineto{\pgfqpoint{1.016954in}{0.905905in}}%
\pgfpathmoveto{\pgfqpoint{0.572757in}{0.858321in}}%
\pgfpathlineto{\pgfqpoint{0.488920in}{0.850184in}}%
\pgfpathmoveto{\pgfqpoint{0.572757in}{0.858321in}}%
\pgfpathlineto{\pgfqpoint{0.529202in}{0.776277in}}%
\pgfpathmoveto{\pgfqpoint{0.572757in}{0.858321in}}%
\pgfpathlineto{\pgfqpoint{0.665556in}{0.898299in}}%
\pgfpathmoveto{\pgfqpoint{0.572757in}{0.858321in}}%
\pgfpathlineto{\pgfqpoint{0.616584in}{0.797742in}}%
\pgfpathmoveto{\pgfqpoint{0.572757in}{0.858321in}}%
\pgfpathlineto{\pgfqpoint{0.567574in}{0.969290in}}%
\pgfpathlineto{\pgfqpoint{0.567574in}{0.969290in}}%
\pgfusepath{stroke}%
\end{pgfscope}%
\begin{pgfscope}%
\pgfpathrectangle{\pgfqpoint{0.100000in}{0.100000in}}{\pgfqpoint{1.782500in}{1.232000in}}%
\pgfusepath{clip}%
\pgfsetbuttcap%
\pgfsetroundjoin%
\definecolor{currentfill}{rgb}{0.054902,0.262745,0.486275}%
\pgfsetfillcolor{currentfill}%
\pgfsetlinewidth{1.003750pt}%
\definecolor{currentstroke}{rgb}{0.054902,0.262745,0.486275}%
\pgfsetstrokecolor{currentstroke}%
\pgfsetdash{}{0pt}%
\pgfsys@defobject{currentmarker}{\pgfqpoint{-0.018373in}{-0.018373in}}{\pgfqpoint{0.018373in}{0.018373in}}{%
\pgfpathmoveto{\pgfqpoint{0.000000in}{-0.018373in}}%
\pgfpathcurveto{\pgfqpoint{0.004873in}{-0.018373in}}{\pgfqpoint{0.009546in}{-0.016437in}}{\pgfqpoint{0.012992in}{-0.012992in}}%
\pgfpathcurveto{\pgfqpoint{0.016437in}{-0.009546in}}{\pgfqpoint{0.018373in}{-0.004873in}}{\pgfqpoint{0.018373in}{0.000000in}}%
\pgfpathcurveto{\pgfqpoint{0.018373in}{0.004873in}}{\pgfqpoint{0.016437in}{0.009546in}}{\pgfqpoint{0.012992in}{0.012992in}}%
\pgfpathcurveto{\pgfqpoint{0.009546in}{0.016437in}}{\pgfqpoint{0.004873in}{0.018373in}}{\pgfqpoint{0.000000in}{0.018373in}}%
\pgfpathcurveto{\pgfqpoint{-0.004873in}{0.018373in}}{\pgfqpoint{-0.009546in}{0.016437in}}{\pgfqpoint{-0.012992in}{0.012992in}}%
\pgfpathcurveto{\pgfqpoint{-0.016437in}{0.009546in}}{\pgfqpoint{-0.018373in}{0.004873in}}{\pgfqpoint{-0.018373in}{0.000000in}}%
\pgfpathcurveto{\pgfqpoint{-0.018373in}{-0.004873in}}{\pgfqpoint{-0.016437in}{-0.009546in}}{\pgfqpoint{-0.012992in}{-0.012992in}}%
\pgfpathcurveto{\pgfqpoint{-0.009546in}{-0.016437in}}{\pgfqpoint{-0.004873in}{-0.018373in}}{\pgfqpoint{0.000000in}{-0.018373in}}%
\pgfpathlineto{\pgfqpoint{0.000000in}{-0.018373in}}%
\pgfpathclose%
\pgfusepath{stroke,fill}%
}%
\begin{pgfscope}%
\pgfsys@transformshift{1.358413in}{0.681892in}%
\pgfsys@useobject{currentmarker}{}%
\end{pgfscope}%
\begin{pgfscope}%
\pgfsys@transformshift{1.300610in}{0.658103in}%
\pgfsys@useobject{currentmarker}{}%
\end{pgfscope}%
\begin{pgfscope}%
\pgfsys@transformshift{1.245113in}{0.640065in}%
\pgfsys@useobject{currentmarker}{}%
\end{pgfscope}%
\begin{pgfscope}%
\pgfsys@transformshift{1.187563in}{0.626883in}%
\pgfsys@useobject{currentmarker}{}%
\end{pgfscope}%
\begin{pgfscope}%
\pgfsys@transformshift{1.121983in}{0.616327in}%
\pgfsys@useobject{currentmarker}{}%
\end{pgfscope}%
\begin{pgfscope}%
\pgfsys@transformshift{1.054136in}{0.608017in}%
\pgfsys@useobject{currentmarker}{}%
\end{pgfscope}%
\begin{pgfscope}%
\pgfsys@transformshift{0.985700in}{0.601411in}%
\pgfsys@useobject{currentmarker}{}%
\end{pgfscope}%
\begin{pgfscope}%
\pgfsys@transformshift{0.916417in}{0.603252in}%
\pgfsys@useobject{currentmarker}{}%
\end{pgfscope}%
\begin{pgfscope}%
\pgfsys@transformshift{0.848834in}{0.613810in}%
\pgfsys@useobject{currentmarker}{}%
\end{pgfscope}%
\begin{pgfscope}%
\pgfsys@transformshift{0.783038in}{0.628109in}%
\pgfsys@useobject{currentmarker}{}%
\end{pgfscope}%
\begin{pgfscope}%
\pgfsys@transformshift{0.721820in}{0.643258in}%
\pgfsys@useobject{currentmarker}{}%
\end{pgfscope}%
\begin{pgfscope}%
\pgfsys@transformshift{0.661196in}{0.663049in}%
\pgfsys@useobject{currentmarker}{}%
\end{pgfscope}%
\end{pgfscope}%
\begin{pgfscope}%
\pgfpathrectangle{\pgfqpoint{0.100000in}{0.100000in}}{\pgfqpoint{1.782500in}{1.232000in}}%
\pgfusepath{clip}%
\pgfsetbuttcap%
\pgfsetroundjoin%
\definecolor{currentfill}{rgb}{0.835294,0.321569,0.035294}%
\pgfsetfillcolor{currentfill}%
\pgfsetlinewidth{1.003750pt}%
\definecolor{currentstroke}{rgb}{0.835294,0.321569,0.035294}%
\pgfsetstrokecolor{currentstroke}%
\pgfsetdash{}{0pt}%
\pgfsys@defobject{currentmarker}{\pgfqpoint{-0.018373in}{-0.018373in}}{\pgfqpoint{0.018373in}{0.018373in}}{%
\pgfpathmoveto{\pgfqpoint{0.000000in}{-0.018373in}}%
\pgfpathcurveto{\pgfqpoint{0.004873in}{-0.018373in}}{\pgfqpoint{0.009546in}{-0.016437in}}{\pgfqpoint{0.012992in}{-0.012992in}}%
\pgfpathcurveto{\pgfqpoint{0.016437in}{-0.009546in}}{\pgfqpoint{0.018373in}{-0.004873in}}{\pgfqpoint{0.018373in}{0.000000in}}%
\pgfpathcurveto{\pgfqpoint{0.018373in}{0.004873in}}{\pgfqpoint{0.016437in}{0.009546in}}{\pgfqpoint{0.012992in}{0.012992in}}%
\pgfpathcurveto{\pgfqpoint{0.009546in}{0.016437in}}{\pgfqpoint{0.004873in}{0.018373in}}{\pgfqpoint{0.000000in}{0.018373in}}%
\pgfpathcurveto{\pgfqpoint{-0.004873in}{0.018373in}}{\pgfqpoint{-0.009546in}{0.016437in}}{\pgfqpoint{-0.012992in}{0.012992in}}%
\pgfpathcurveto{\pgfqpoint{-0.016437in}{0.009546in}}{\pgfqpoint{-0.018373in}{0.004873in}}{\pgfqpoint{-0.018373in}{0.000000in}}%
\pgfpathcurveto{\pgfqpoint{-0.018373in}{-0.004873in}}{\pgfqpoint{-0.016437in}{-0.009546in}}{\pgfqpoint{-0.012992in}{-0.012992in}}%
\pgfpathcurveto{\pgfqpoint{-0.009546in}{-0.016437in}}{\pgfqpoint{-0.004873in}{-0.018373in}}{\pgfqpoint{0.000000in}{-0.018373in}}%
\pgfpathlineto{\pgfqpoint{0.000000in}{-0.018373in}}%
\pgfpathclose%
\pgfusepath{stroke,fill}%
}%
\begin{pgfscope}%
\pgfsys@transformshift{0.576443in}{0.718597in}%
\pgfsys@useobject{currentmarker}{}%
\end{pgfscope}%
\begin{pgfscope}%
\pgfsys@transformshift{0.649179in}{0.669718in}%
\pgfsys@useobject{currentmarker}{}%
\end{pgfscope}%
\begin{pgfscope}%
\pgfsys@transformshift{0.742562in}{0.637869in}%
\pgfsys@useobject{currentmarker}{}%
\end{pgfscope}%
\begin{pgfscope}%
\pgfsys@transformshift{0.828308in}{0.618003in}%
\pgfsys@useobject{currentmarker}{}%
\end{pgfscope}%
\begin{pgfscope}%
\pgfsys@transformshift{0.907725in}{0.603940in}%
\pgfsys@useobject{currentmarker}{}%
\end{pgfscope}%
\begin{pgfscope}%
\pgfsys@transformshift{0.985741in}{0.601166in}%
\pgfsys@useobject{currentmarker}{}%
\end{pgfscope}%
\begin{pgfscope}%
\pgfsys@transformshift{1.062700in}{0.609075in}%
\pgfsys@useobject{currentmarker}{}%
\end{pgfscope}%
\begin{pgfscope}%
\pgfsys@transformshift{1.142662in}{0.619169in}%
\pgfsys@useobject{currentmarker}{}%
\end{pgfscope}%
\begin{pgfscope}%
\pgfsys@transformshift{1.226275in}{0.635125in}%
\pgfsys@useobject{currentmarker}{}%
\end{pgfscope}%
\begin{pgfscope}%
\pgfsys@transformshift{1.313856in}{0.663445in}%
\pgfsys@useobject{currentmarker}{}%
\end{pgfscope}%
\begin{pgfscope}%
\pgfsys@transformshift{1.401434in}{0.701100in}%
\pgfsys@useobject{currentmarker}{}%
\end{pgfscope}%
\end{pgfscope}%
\end{pgfpicture}%
\makeatother%
\endgroup%

        \caption{Iteration 2: Solve system}\label{fig:example-iter1-solution}
    \end{subfigure}
    \begin{subfigure}[b]{.32\linewidth}
        %% Creator: Matplotlib, PGF backend
%%
%% To include the figure in your LaTeX document, write
%%   \input{<filename>.pgf}
%%
%% Make sure the required packages are loaded in your preamble
%%   \usepackage{pgf}
%%
%% Also ensure that all the required font packages are loaded; for instance,
%% the lmodern package is sometimes necessary when using math font.
%%   \usepackage{lmodern}
%%
%% Figures using additional raster images can only be included by \input if
%% they are in the same directory as the main LaTeX file. For loading figures
%% from other directories you can use the `import` package
%%   \usepackage{import}
%%
%% and then include the figures with
%%   \import{<path to file>}{<filename>.pgf}
%%
%% Matplotlib used the following preamble
%%   
%%   \usepackage{fontspec}
%%   \setmainfont{DejaVuSans.ttf}[Path=\detokenize{/home/fabio/Internodes-CM/.venv/lib/python3.8/site-packages/matplotlib/mpl-data/fonts/ttf/}]
%%   \setsansfont{DejaVuSans.ttf}[Path=\detokenize{/home/fabio/Internodes-CM/.venv/lib/python3.8/site-packages/matplotlib/mpl-data/fonts/ttf/}]
%%   \setmonofont{DejaVuSansMono.ttf}[Path=\detokenize{/home/fabio/Internodes-CM/.venv/lib/python3.8/site-packages/matplotlib/mpl-data/fonts/ttf/}]
%%   \makeatletter\@ifpackageloaded{underscore}{}{\usepackage[strings]{underscore}}\makeatother
%%
\begingroup%
\makeatletter%
\begin{pgfpicture}%
\pgfpathrectangle{\pgfpointorigin}{\pgfqpoint{1.982500in}{1.432000in}}%
\pgfusepath{use as bounding box, clip}%
\begin{pgfscope}%
\pgfsetbuttcap%
\pgfsetmiterjoin%
\definecolor{currentfill}{rgb}{1.000000,1.000000,1.000000}%
\pgfsetfillcolor{currentfill}%
\pgfsetlinewidth{0.000000pt}%
\definecolor{currentstroke}{rgb}{1.000000,1.000000,1.000000}%
\pgfsetstrokecolor{currentstroke}%
\pgfsetdash{}{0pt}%
\pgfpathmoveto{\pgfqpoint{0.000000in}{0.000000in}}%
\pgfpathlineto{\pgfqpoint{1.982500in}{0.000000in}}%
\pgfpathlineto{\pgfqpoint{1.982500in}{1.432000in}}%
\pgfpathlineto{\pgfqpoint{0.000000in}{1.432000in}}%
\pgfpathlineto{\pgfqpoint{0.000000in}{0.000000in}}%
\pgfpathclose%
\pgfusepath{fill}%
\end{pgfscope}%
\begin{pgfscope}%
\pgfpathrectangle{\pgfqpoint{0.100000in}{0.100000in}}{\pgfqpoint{1.782500in}{1.232000in}}%
\pgfusepath{clip}%
\pgfsetrectcap%
\pgfsetroundjoin%
\pgfsetlinewidth{0.250937pt}%
\definecolor{currentstroke}{rgb}{0.054902,0.262745,0.486275}%
\pgfsetstrokecolor{currentstroke}%
\pgfsetdash{}{0pt}%
\pgfpathmoveto{\pgfqpoint{0.458635in}{0.100000in}}%
\pgfpathlineto{\pgfqpoint{0.190067in}{0.100000in}}%
\pgfpathmoveto{\pgfqpoint{0.727204in}{0.100000in}}%
\pgfpathlineto{\pgfqpoint{0.458635in}{0.100000in}}%
\pgfpathmoveto{\pgfqpoint{0.995772in}{0.100000in}}%
\pgfpathlineto{\pgfqpoint{0.727204in}{0.100000in}}%
\pgfpathmoveto{\pgfqpoint{1.264340in}{0.100000in}}%
\pgfpathlineto{\pgfqpoint{0.995772in}{0.100000in}}%
\pgfpathmoveto{\pgfqpoint{1.532909in}{0.100000in}}%
\pgfpathlineto{\pgfqpoint{1.801477in}{0.100000in}}%
\pgfpathmoveto{\pgfqpoint{1.532909in}{0.100000in}}%
\pgfpathlineto{\pgfqpoint{1.264340in}{0.100000in}}%
\pgfpathmoveto{\pgfqpoint{1.796841in}{0.415884in}}%
\pgfpathlineto{\pgfqpoint{1.801477in}{0.100000in}}%
\pgfpathmoveto{\pgfqpoint{1.796841in}{0.415884in}}%
\pgfpathlineto{\pgfqpoint{1.774230in}{0.729694in}}%
\pgfpathmoveto{\pgfqpoint{1.611368in}{0.716211in}}%
\pgfpathlineto{\pgfqpoint{1.774230in}{0.729694in}}%
\pgfpathmoveto{\pgfqpoint{1.611368in}{0.716211in}}%
\pgfpathlineto{\pgfqpoint{1.438592in}{0.697903in}}%
\pgfpathmoveto{\pgfqpoint{1.358413in}{0.681892in}}%
\pgfpathlineto{\pgfqpoint{1.438592in}{0.697903in}}%
\pgfpathmoveto{\pgfqpoint{1.300610in}{0.658103in}}%
\pgfpathlineto{\pgfqpoint{1.358413in}{0.681892in}}%
\pgfpathmoveto{\pgfqpoint{1.245113in}{0.640065in}}%
\pgfpathlineto{\pgfqpoint{1.300610in}{0.658103in}}%
\pgfpathmoveto{\pgfqpoint{1.187563in}{0.626883in}}%
\pgfpathlineto{\pgfqpoint{1.245113in}{0.640065in}}%
\pgfpathmoveto{\pgfqpoint{1.121983in}{0.616327in}}%
\pgfpathlineto{\pgfqpoint{1.187563in}{0.626883in}}%
\pgfpathmoveto{\pgfqpoint{1.054136in}{0.608017in}}%
\pgfpathlineto{\pgfqpoint{1.121983in}{0.616327in}}%
\pgfpathmoveto{\pgfqpoint{0.985700in}{0.601411in}}%
\pgfpathlineto{\pgfqpoint{1.054136in}{0.608017in}}%
\pgfpathmoveto{\pgfqpoint{0.916417in}{0.603252in}}%
\pgfpathlineto{\pgfqpoint{0.985700in}{0.601411in}}%
\pgfpathmoveto{\pgfqpoint{0.848834in}{0.613810in}}%
\pgfpathlineto{\pgfqpoint{0.916417in}{0.603252in}}%
\pgfpathmoveto{\pgfqpoint{0.783038in}{0.628109in}}%
\pgfpathlineto{\pgfqpoint{0.848834in}{0.613810in}}%
\pgfpathmoveto{\pgfqpoint{0.721820in}{0.643258in}}%
\pgfpathlineto{\pgfqpoint{0.783038in}{0.628109in}}%
\pgfpathmoveto{\pgfqpoint{0.661196in}{0.663049in}}%
\pgfpathlineto{\pgfqpoint{0.721820in}{0.643258in}}%
\pgfpathmoveto{\pgfqpoint{0.590312in}{0.684620in}}%
\pgfpathlineto{\pgfqpoint{0.517884in}{0.696169in}}%
\pgfpathmoveto{\pgfqpoint{0.590312in}{0.684620in}}%
\pgfpathlineto{\pgfqpoint{0.661196in}{0.663049in}}%
\pgfpathmoveto{\pgfqpoint{0.352544in}{0.709066in}}%
\pgfpathlineto{\pgfqpoint{0.517884in}{0.696169in}}%
\pgfpathmoveto{\pgfqpoint{0.352544in}{0.709066in}}%
\pgfpathlineto{\pgfqpoint{0.190332in}{0.716109in}}%
\pgfpathmoveto{\pgfqpoint{0.181023in}{0.405438in}}%
\pgfpathlineto{\pgfqpoint{0.190067in}{0.100000in}}%
\pgfpathmoveto{\pgfqpoint{0.181023in}{0.405438in}}%
\pgfpathlineto{\pgfqpoint{0.190332in}{0.716109in}}%
\pgfpathmoveto{\pgfqpoint{1.377207in}{0.371708in}}%
\pgfpathlineto{\pgfqpoint{1.264340in}{0.100000in}}%
\pgfpathmoveto{\pgfqpoint{1.083002in}{0.353907in}}%
\pgfpathlineto{\pgfqpoint{0.995772in}{0.100000in}}%
\pgfpathmoveto{\pgfqpoint{1.230784in}{0.463627in}}%
\pgfpathlineto{\pgfqpoint{1.377207in}{0.371708in}}%
\pgfpathmoveto{\pgfqpoint{1.230784in}{0.463627in}}%
\pgfpathlineto{\pgfqpoint{1.083002in}{0.353907in}}%
\pgfpathmoveto{\pgfqpoint{0.951674in}{0.449240in}}%
\pgfpathlineto{\pgfqpoint{1.083002in}{0.353907in}}%
\pgfpathmoveto{\pgfqpoint{0.951674in}{0.449240in}}%
\pgfpathlineto{\pgfqpoint{0.815316in}{0.370020in}}%
\pgfpathmoveto{\pgfqpoint{0.672687in}{0.479220in}}%
\pgfpathlineto{\pgfqpoint{0.815316in}{0.370020in}}%
\pgfpathmoveto{\pgfqpoint{0.672687in}{0.479220in}}%
\pgfpathlineto{\pgfqpoint{0.534657in}{0.384219in}}%
\pgfpathmoveto{\pgfqpoint{1.529457in}{0.488685in}}%
\pgfpathlineto{\pgfqpoint{1.796841in}{0.415884in}}%
\pgfpathmoveto{\pgfqpoint{1.529457in}{0.488685in}}%
\pgfpathlineto{\pgfqpoint{1.377207in}{0.371708in}}%
\pgfpathmoveto{\pgfqpoint{1.369377in}{0.514759in}}%
\pgfpathlineto{\pgfqpoint{1.377207in}{0.371708in}}%
\pgfpathmoveto{\pgfqpoint{1.369377in}{0.514759in}}%
\pgfpathlineto{\pgfqpoint{1.230784in}{0.463627in}}%
\pgfpathmoveto{\pgfqpoint{1.369377in}{0.514759in}}%
\pgfpathlineto{\pgfqpoint{1.529457in}{0.488685in}}%
\pgfpathmoveto{\pgfqpoint{1.089290in}{0.480619in}}%
\pgfpathlineto{\pgfqpoint{1.083002in}{0.353907in}}%
\pgfpathmoveto{\pgfqpoint{1.089290in}{0.480619in}}%
\pgfpathlineto{\pgfqpoint{1.230784in}{0.463627in}}%
\pgfpathmoveto{\pgfqpoint{1.089290in}{0.480619in}}%
\pgfpathlineto{\pgfqpoint{0.951674in}{0.449240in}}%
\pgfpathmoveto{\pgfqpoint{0.812716in}{0.489158in}}%
\pgfpathlineto{\pgfqpoint{0.815316in}{0.370020in}}%
\pgfpathmoveto{\pgfqpoint{0.812716in}{0.489158in}}%
\pgfpathlineto{\pgfqpoint{0.951674in}{0.449240in}}%
\pgfpathmoveto{\pgfqpoint{0.812716in}{0.489158in}}%
\pgfpathlineto{\pgfqpoint{0.672687in}{0.479220in}}%
\pgfpathmoveto{\pgfqpoint{0.523931in}{0.529297in}}%
\pgfpathlineto{\pgfqpoint{0.534657in}{0.384219in}}%
\pgfpathmoveto{\pgfqpoint{0.523931in}{0.529297in}}%
\pgfpathlineto{\pgfqpoint{0.672687in}{0.479220in}}%
\pgfpathmoveto{\pgfqpoint{0.356202in}{0.490685in}}%
\pgfpathlineto{\pgfqpoint{0.190332in}{0.716109in}}%
\pgfpathmoveto{\pgfqpoint{0.356202in}{0.490685in}}%
\pgfpathlineto{\pgfqpoint{0.352544in}{0.709066in}}%
\pgfpathmoveto{\pgfqpoint{0.356202in}{0.490685in}}%
\pgfpathlineto{\pgfqpoint{0.181023in}{0.405438in}}%
\pgfpathmoveto{\pgfqpoint{0.356202in}{0.490685in}}%
\pgfpathlineto{\pgfqpoint{0.534657in}{0.384219in}}%
\pgfpathmoveto{\pgfqpoint{0.356202in}{0.490685in}}%
\pgfpathlineto{\pgfqpoint{0.523931in}{0.529297in}}%
\pgfpathmoveto{\pgfqpoint{1.289734in}{0.563347in}}%
\pgfpathlineto{\pgfqpoint{1.300610in}{0.658103in}}%
\pgfpathmoveto{\pgfqpoint{1.289734in}{0.563347in}}%
\pgfpathlineto{\pgfqpoint{1.245113in}{0.640065in}}%
\pgfpathmoveto{\pgfqpoint{1.289734in}{0.563347in}}%
\pgfpathlineto{\pgfqpoint{1.230784in}{0.463627in}}%
\pgfpathmoveto{\pgfqpoint{1.289734in}{0.563347in}}%
\pgfpathlineto{\pgfqpoint{1.369377in}{0.514759in}}%
\pgfpathmoveto{\pgfqpoint{1.157523in}{0.542874in}}%
\pgfpathlineto{\pgfqpoint{1.187563in}{0.626883in}}%
\pgfpathmoveto{\pgfqpoint{1.157523in}{0.542874in}}%
\pgfpathlineto{\pgfqpoint{1.121983in}{0.616327in}}%
\pgfpathmoveto{\pgfqpoint{1.157523in}{0.542874in}}%
\pgfpathlineto{\pgfqpoint{1.230784in}{0.463627in}}%
\pgfpathmoveto{\pgfqpoint{1.157523in}{0.542874in}}%
\pgfpathlineto{\pgfqpoint{1.089290in}{0.480619in}}%
\pgfpathmoveto{\pgfqpoint{1.021299in}{0.531634in}}%
\pgfpathlineto{\pgfqpoint{1.054136in}{0.608017in}}%
\pgfpathmoveto{\pgfqpoint{1.021299in}{0.531634in}}%
\pgfpathlineto{\pgfqpoint{0.985700in}{0.601411in}}%
\pgfpathmoveto{\pgfqpoint{1.021299in}{0.531634in}}%
\pgfpathlineto{\pgfqpoint{0.951674in}{0.449240in}}%
\pgfpathmoveto{\pgfqpoint{1.021299in}{0.531634in}}%
\pgfpathlineto{\pgfqpoint{1.089290in}{0.480619in}}%
\pgfpathmoveto{\pgfqpoint{0.881689in}{0.534959in}}%
\pgfpathlineto{\pgfqpoint{0.916417in}{0.603252in}}%
\pgfpathmoveto{\pgfqpoint{0.881689in}{0.534959in}}%
\pgfpathlineto{\pgfqpoint{0.848834in}{0.613810in}}%
\pgfpathmoveto{\pgfqpoint{0.881689in}{0.534959in}}%
\pgfpathlineto{\pgfqpoint{0.951674in}{0.449240in}}%
\pgfpathmoveto{\pgfqpoint{0.881689in}{0.534959in}}%
\pgfpathlineto{\pgfqpoint{0.812716in}{0.489158in}}%
\pgfpathmoveto{\pgfqpoint{0.745570in}{0.554174in}}%
\pgfpathlineto{\pgfqpoint{0.783038in}{0.628109in}}%
\pgfpathmoveto{\pgfqpoint{0.745570in}{0.554174in}}%
\pgfpathlineto{\pgfqpoint{0.721820in}{0.643258in}}%
\pgfpathmoveto{\pgfqpoint{0.745570in}{0.554174in}}%
\pgfpathlineto{\pgfqpoint{0.672687in}{0.479220in}}%
\pgfpathmoveto{\pgfqpoint{0.745570in}{0.554174in}}%
\pgfpathlineto{\pgfqpoint{0.812716in}{0.489158in}}%
\pgfpathmoveto{\pgfqpoint{0.610324in}{0.585562in}}%
\pgfpathlineto{\pgfqpoint{0.661196in}{0.663049in}}%
\pgfpathmoveto{\pgfqpoint{0.610324in}{0.585562in}}%
\pgfpathlineto{\pgfqpoint{0.590312in}{0.684620in}}%
\pgfpathmoveto{\pgfqpoint{0.610324in}{0.585562in}}%
\pgfpathlineto{\pgfqpoint{0.672687in}{0.479220in}}%
\pgfpathmoveto{\pgfqpoint{0.610324in}{0.585562in}}%
\pgfpathlineto{\pgfqpoint{0.523931in}{0.529297in}}%
\pgfpathmoveto{\pgfqpoint{1.433326in}{0.603291in}}%
\pgfpathlineto{\pgfqpoint{1.438592in}{0.697903in}}%
\pgfpathmoveto{\pgfqpoint{1.433326in}{0.603291in}}%
\pgfpathlineto{\pgfqpoint{1.358413in}{0.681892in}}%
\pgfpathmoveto{\pgfqpoint{1.433326in}{0.603291in}}%
\pgfpathlineto{\pgfqpoint{1.529457in}{0.488685in}}%
\pgfpathmoveto{\pgfqpoint{1.433326in}{0.603291in}}%
\pgfpathlineto{\pgfqpoint{1.369377in}{0.514759in}}%
\pgfpathmoveto{\pgfqpoint{0.938271in}{0.294479in}}%
\pgfpathlineto{\pgfqpoint{0.995772in}{0.100000in}}%
\pgfpathmoveto{\pgfqpoint{0.938271in}{0.294479in}}%
\pgfpathlineto{\pgfqpoint{1.083002in}{0.353907in}}%
\pgfpathmoveto{\pgfqpoint{0.938271in}{0.294479in}}%
\pgfpathlineto{\pgfqpoint{0.815316in}{0.370020in}}%
\pgfpathmoveto{\pgfqpoint{0.938271in}{0.294479in}}%
\pgfpathlineto{\pgfqpoint{0.951674in}{0.449240in}}%
\pgfpathmoveto{\pgfqpoint{0.437987in}{0.601518in}}%
\pgfpathlineto{\pgfqpoint{0.517884in}{0.696169in}}%
\pgfpathmoveto{\pgfqpoint{0.437987in}{0.601518in}}%
\pgfpathlineto{\pgfqpoint{0.352544in}{0.709066in}}%
\pgfpathmoveto{\pgfqpoint{0.437987in}{0.601518in}}%
\pgfpathlineto{\pgfqpoint{0.523931in}{0.529297in}}%
\pgfpathmoveto{\pgfqpoint{0.437987in}{0.601518in}}%
\pgfpathlineto{\pgfqpoint{0.356202in}{0.490685in}}%
\pgfpathmoveto{\pgfqpoint{1.351528in}{0.601963in}}%
\pgfpathlineto{\pgfqpoint{1.358413in}{0.681892in}}%
\pgfpathmoveto{\pgfqpoint{1.351528in}{0.601963in}}%
\pgfpathlineto{\pgfqpoint{1.300610in}{0.658103in}}%
\pgfpathmoveto{\pgfqpoint{1.351528in}{0.601963in}}%
\pgfpathlineto{\pgfqpoint{1.369377in}{0.514759in}}%
\pgfpathmoveto{\pgfqpoint{1.351528in}{0.601963in}}%
\pgfpathlineto{\pgfqpoint{1.289734in}{0.563347in}}%
\pgfpathmoveto{\pgfqpoint{1.351528in}{0.601963in}}%
\pgfpathlineto{\pgfqpoint{1.433326in}{0.603291in}}%
\pgfpathmoveto{\pgfqpoint{1.088974in}{0.555599in}}%
\pgfpathlineto{\pgfqpoint{1.121983in}{0.616327in}}%
\pgfpathmoveto{\pgfqpoint{1.088974in}{0.555599in}}%
\pgfpathlineto{\pgfqpoint{1.054136in}{0.608017in}}%
\pgfpathmoveto{\pgfqpoint{1.088974in}{0.555599in}}%
\pgfpathlineto{\pgfqpoint{1.089290in}{0.480619in}}%
\pgfpathmoveto{\pgfqpoint{1.088974in}{0.555599in}}%
\pgfpathlineto{\pgfqpoint{1.157523in}{0.542874in}}%
\pgfpathmoveto{\pgfqpoint{1.088974in}{0.555599in}}%
\pgfpathlineto{\pgfqpoint{1.021299in}{0.531634in}}%
\pgfpathmoveto{\pgfqpoint{1.222194in}{0.569805in}}%
\pgfpathlineto{\pgfqpoint{1.245113in}{0.640065in}}%
\pgfpathmoveto{\pgfqpoint{1.222194in}{0.569805in}}%
\pgfpathlineto{\pgfqpoint{1.187563in}{0.626883in}}%
\pgfpathmoveto{\pgfqpoint{1.222194in}{0.569805in}}%
\pgfpathlineto{\pgfqpoint{1.230784in}{0.463627in}}%
\pgfpathmoveto{\pgfqpoint{1.222194in}{0.569805in}}%
\pgfpathlineto{\pgfqpoint{1.289734in}{0.563347in}}%
\pgfpathmoveto{\pgfqpoint{1.222194in}{0.569805in}}%
\pgfpathlineto{\pgfqpoint{1.157523in}{0.542874in}}%
\pgfpathmoveto{\pgfqpoint{0.951314in}{0.543497in}}%
\pgfpathlineto{\pgfqpoint{0.985700in}{0.601411in}}%
\pgfpathmoveto{\pgfqpoint{0.951314in}{0.543497in}}%
\pgfpathlineto{\pgfqpoint{0.916417in}{0.603252in}}%
\pgfpathmoveto{\pgfqpoint{0.951314in}{0.543497in}}%
\pgfpathlineto{\pgfqpoint{0.951674in}{0.449240in}}%
\pgfpathmoveto{\pgfqpoint{0.951314in}{0.543497in}}%
\pgfpathlineto{\pgfqpoint{1.021299in}{0.531634in}}%
\pgfpathmoveto{\pgfqpoint{0.951314in}{0.543497in}}%
\pgfpathlineto{\pgfqpoint{0.881689in}{0.534959in}}%
\pgfpathmoveto{\pgfqpoint{0.813693in}{0.562058in}}%
\pgfpathlineto{\pgfqpoint{0.848834in}{0.613810in}}%
\pgfpathmoveto{\pgfqpoint{0.813693in}{0.562058in}}%
\pgfpathlineto{\pgfqpoint{0.783038in}{0.628109in}}%
\pgfpathmoveto{\pgfqpoint{0.813693in}{0.562058in}}%
\pgfpathlineto{\pgfqpoint{0.812716in}{0.489158in}}%
\pgfpathmoveto{\pgfqpoint{0.813693in}{0.562058in}}%
\pgfpathlineto{\pgfqpoint{0.881689in}{0.534959in}}%
\pgfpathmoveto{\pgfqpoint{0.813693in}{0.562058in}}%
\pgfpathlineto{\pgfqpoint{0.745570in}{0.554174in}}%
\pgfpathmoveto{\pgfqpoint{1.192290in}{0.275687in}}%
\pgfpathlineto{\pgfqpoint{0.995772in}{0.100000in}}%
\pgfpathmoveto{\pgfqpoint{1.192290in}{0.275687in}}%
\pgfpathlineto{\pgfqpoint{1.264340in}{0.100000in}}%
\pgfpathmoveto{\pgfqpoint{1.192290in}{0.275687in}}%
\pgfpathlineto{\pgfqpoint{1.377207in}{0.371708in}}%
\pgfpathmoveto{\pgfqpoint{1.192290in}{0.275687in}}%
\pgfpathlineto{\pgfqpoint{1.083002in}{0.353907in}}%
\pgfpathmoveto{\pgfqpoint{1.192290in}{0.275687in}}%
\pgfpathlineto{\pgfqpoint{1.230784in}{0.463627in}}%
\pgfpathmoveto{\pgfqpoint{0.691426in}{0.290497in}}%
\pgfpathlineto{\pgfqpoint{0.727204in}{0.100000in}}%
\pgfpathmoveto{\pgfqpoint{0.691426in}{0.290497in}}%
\pgfpathlineto{\pgfqpoint{0.815316in}{0.370020in}}%
\pgfpathmoveto{\pgfqpoint{0.691426in}{0.290497in}}%
\pgfpathlineto{\pgfqpoint{0.534657in}{0.384219in}}%
\pgfpathmoveto{\pgfqpoint{0.691426in}{0.290497in}}%
\pgfpathlineto{\pgfqpoint{0.672687in}{0.479220in}}%
\pgfpathmoveto{\pgfqpoint{0.680324in}{0.583817in}}%
\pgfpathlineto{\pgfqpoint{0.721820in}{0.643258in}}%
\pgfpathmoveto{\pgfqpoint{0.680324in}{0.583817in}}%
\pgfpathlineto{\pgfqpoint{0.661196in}{0.663049in}}%
\pgfpathmoveto{\pgfqpoint{0.680324in}{0.583817in}}%
\pgfpathlineto{\pgfqpoint{0.672687in}{0.479220in}}%
\pgfpathmoveto{\pgfqpoint{0.680324in}{0.583817in}}%
\pgfpathlineto{\pgfqpoint{0.745570in}{0.554174in}}%
\pgfpathmoveto{\pgfqpoint{0.680324in}{0.583817in}}%
\pgfpathlineto{\pgfqpoint{0.610324in}{0.585562in}}%
\pgfpathmoveto{\pgfqpoint{1.535895in}{0.617892in}}%
\pgfpathlineto{\pgfqpoint{1.438592in}{0.697903in}}%
\pgfpathmoveto{\pgfqpoint{1.535895in}{0.617892in}}%
\pgfpathlineto{\pgfqpoint{1.611368in}{0.716211in}}%
\pgfpathmoveto{\pgfqpoint{1.535895in}{0.617892in}}%
\pgfpathlineto{\pgfqpoint{1.529457in}{0.488685in}}%
\pgfpathmoveto{\pgfqpoint{1.535895in}{0.617892in}}%
\pgfpathlineto{\pgfqpoint{1.433326in}{0.603291in}}%
\pgfpathmoveto{\pgfqpoint{0.545528in}{0.620176in}}%
\pgfpathlineto{\pgfqpoint{0.517884in}{0.696169in}}%
\pgfpathmoveto{\pgfqpoint{0.545528in}{0.620176in}}%
\pgfpathlineto{\pgfqpoint{0.590312in}{0.684620in}}%
\pgfpathmoveto{\pgfqpoint{0.545528in}{0.620176in}}%
\pgfpathlineto{\pgfqpoint{0.523931in}{0.529297in}}%
\pgfpathmoveto{\pgfqpoint{0.545528in}{0.620176in}}%
\pgfpathlineto{\pgfqpoint{0.610324in}{0.585562in}}%
\pgfpathmoveto{\pgfqpoint{0.545528in}{0.620176in}}%
\pgfpathlineto{\pgfqpoint{0.437987in}{0.601518in}}%
\pgfpathmoveto{\pgfqpoint{0.379348in}{0.283472in}}%
\pgfpathlineto{\pgfqpoint{0.190067in}{0.100000in}}%
\pgfpathmoveto{\pgfqpoint{0.379348in}{0.283472in}}%
\pgfpathlineto{\pgfqpoint{0.458635in}{0.100000in}}%
\pgfpathmoveto{\pgfqpoint{0.379348in}{0.283472in}}%
\pgfpathlineto{\pgfqpoint{0.181023in}{0.405438in}}%
\pgfpathmoveto{\pgfqpoint{0.379348in}{0.283472in}}%
\pgfpathlineto{\pgfqpoint{0.534657in}{0.384219in}}%
\pgfpathmoveto{\pgfqpoint{0.379348in}{0.283472in}}%
\pgfpathlineto{\pgfqpoint{0.356202in}{0.490685in}}%
\pgfpathmoveto{\pgfqpoint{1.533495in}{0.293969in}}%
\pgfpathlineto{\pgfqpoint{1.801477in}{0.100000in}}%
\pgfpathmoveto{\pgfqpoint{1.533495in}{0.293969in}}%
\pgfpathlineto{\pgfqpoint{1.264340in}{0.100000in}}%
\pgfpathmoveto{\pgfqpoint{1.533495in}{0.293969in}}%
\pgfpathlineto{\pgfqpoint{1.532909in}{0.100000in}}%
\pgfpathmoveto{\pgfqpoint{1.533495in}{0.293969in}}%
\pgfpathlineto{\pgfqpoint{1.796841in}{0.415884in}}%
\pgfpathmoveto{\pgfqpoint{1.533495in}{0.293969in}}%
\pgfpathlineto{\pgfqpoint{1.377207in}{0.371708in}}%
\pgfpathmoveto{\pgfqpoint{1.533495in}{0.293969in}}%
\pgfpathlineto{\pgfqpoint{1.529457in}{0.488685in}}%
\pgfpathmoveto{\pgfqpoint{1.649933in}{0.594471in}}%
\pgfpathlineto{\pgfqpoint{1.774230in}{0.729694in}}%
\pgfpathmoveto{\pgfqpoint{1.649933in}{0.594471in}}%
\pgfpathlineto{\pgfqpoint{1.796841in}{0.415884in}}%
\pgfpathmoveto{\pgfqpoint{1.649933in}{0.594471in}}%
\pgfpathlineto{\pgfqpoint{1.611368in}{0.716211in}}%
\pgfpathmoveto{\pgfqpoint{1.649933in}{0.594471in}}%
\pgfpathlineto{\pgfqpoint{1.529457in}{0.488685in}}%
\pgfpathmoveto{\pgfqpoint{1.649933in}{0.594471in}}%
\pgfpathlineto{\pgfqpoint{1.535895in}{0.617892in}}%
\pgfpathmoveto{\pgfqpoint{0.841552in}{0.212299in}}%
\pgfpathlineto{\pgfqpoint{0.727204in}{0.100000in}}%
\pgfpathmoveto{\pgfqpoint{0.841552in}{0.212299in}}%
\pgfpathlineto{\pgfqpoint{0.995772in}{0.100000in}}%
\pgfpathmoveto{\pgfqpoint{0.841552in}{0.212299in}}%
\pgfpathlineto{\pgfqpoint{0.815316in}{0.370020in}}%
\pgfpathmoveto{\pgfqpoint{0.841552in}{0.212299in}}%
\pgfpathlineto{\pgfqpoint{0.938271in}{0.294479in}}%
\pgfpathmoveto{\pgfqpoint{0.841552in}{0.212299in}}%
\pgfpathlineto{\pgfqpoint{0.691426in}{0.290497in}}%
\pgfpathmoveto{\pgfqpoint{0.556128in}{0.231085in}}%
\pgfpathlineto{\pgfqpoint{0.458635in}{0.100000in}}%
\pgfpathmoveto{\pgfqpoint{0.556128in}{0.231085in}}%
\pgfpathlineto{\pgfqpoint{0.727204in}{0.100000in}}%
\pgfpathmoveto{\pgfqpoint{0.556128in}{0.231085in}}%
\pgfpathlineto{\pgfqpoint{0.534657in}{0.384219in}}%
\pgfpathmoveto{\pgfqpoint{0.556128in}{0.231085in}}%
\pgfpathlineto{\pgfqpoint{0.691426in}{0.290497in}}%
\pgfpathmoveto{\pgfqpoint{0.556128in}{0.231085in}}%
\pgfpathlineto{\pgfqpoint{0.379348in}{0.283472in}}%
\pgfpathlineto{\pgfqpoint{0.379348in}{0.283472in}}%
\pgfusepath{stroke}%
\end{pgfscope}%
\begin{pgfscope}%
\pgfpathrectangle{\pgfqpoint{0.100000in}{0.100000in}}{\pgfqpoint{1.782500in}{1.232000in}}%
\pgfusepath{clip}%
\pgfsetrectcap%
\pgfsetroundjoin%
\pgfsetlinewidth{0.250937pt}%
\definecolor{currentstroke}{rgb}{0.835294,0.321569,0.035294}%
\pgfsetstrokecolor{currentstroke}%
\pgfsetdash{}{0pt}%
\pgfpathmoveto{\pgfqpoint{0.488920in}{0.850184in}}%
\pgfpathlineto{\pgfqpoint{0.458635in}{1.085600in}}%
\pgfpathmoveto{\pgfqpoint{1.532909in}{1.085600in}}%
\pgfpathlineto{\pgfqpoint{1.494399in}{0.840973in}}%
\pgfpathmoveto{\pgfqpoint{0.727204in}{1.085600in}}%
\pgfpathlineto{\pgfqpoint{0.995772in}{1.085600in}}%
\pgfpathmoveto{\pgfqpoint{0.727204in}{1.085600in}}%
\pgfpathlineto{\pgfqpoint{0.458635in}{1.085600in}}%
\pgfpathmoveto{\pgfqpoint{0.529202in}{0.776277in}}%
\pgfpathlineto{\pgfqpoint{0.488920in}{0.850184in}}%
\pgfpathmoveto{\pgfqpoint{0.576443in}{0.718597in}}%
\pgfpathlineto{\pgfqpoint{0.529202in}{0.776277in}}%
\pgfpathmoveto{\pgfqpoint{0.649179in}{0.669718in}}%
\pgfpathlineto{\pgfqpoint{0.576443in}{0.718597in}}%
\pgfpathmoveto{\pgfqpoint{0.742562in}{0.637869in}}%
\pgfpathlineto{\pgfqpoint{0.649179in}{0.669718in}}%
\pgfpathmoveto{\pgfqpoint{0.828308in}{0.618003in}}%
\pgfpathlineto{\pgfqpoint{0.742562in}{0.637869in}}%
\pgfpathmoveto{\pgfqpoint{0.907725in}{0.603940in}}%
\pgfpathlineto{\pgfqpoint{0.828308in}{0.618003in}}%
\pgfpathmoveto{\pgfqpoint{0.985741in}{0.601166in}}%
\pgfpathlineto{\pgfqpoint{0.907725in}{0.603940in}}%
\pgfpathmoveto{\pgfqpoint{1.062700in}{0.609075in}}%
\pgfpathlineto{\pgfqpoint{0.985741in}{0.601166in}}%
\pgfpathmoveto{\pgfqpoint{1.142662in}{0.619169in}}%
\pgfpathlineto{\pgfqpoint{1.062700in}{0.609075in}}%
\pgfpathmoveto{\pgfqpoint{1.226275in}{0.635125in}}%
\pgfpathlineto{\pgfqpoint{1.142662in}{0.619169in}}%
\pgfpathmoveto{\pgfqpoint{1.313856in}{0.663445in}}%
\pgfpathlineto{\pgfqpoint{1.226275in}{0.635125in}}%
\pgfpathmoveto{\pgfqpoint{1.401434in}{0.701100in}}%
\pgfpathlineto{\pgfqpoint{1.313856in}{0.663445in}}%
\pgfpathmoveto{\pgfqpoint{1.451289in}{0.764757in}}%
\pgfpathlineto{\pgfqpoint{1.494399in}{0.840973in}}%
\pgfpathmoveto{\pgfqpoint{1.451289in}{0.764757in}}%
\pgfpathlineto{\pgfqpoint{1.401434in}{0.701100in}}%
\pgfpathmoveto{\pgfqpoint{1.264340in}{1.085600in}}%
\pgfpathlineto{\pgfqpoint{0.995772in}{1.085600in}}%
\pgfpathmoveto{\pgfqpoint{1.264340in}{1.085600in}}%
\pgfpathlineto{\pgfqpoint{1.532909in}{1.085600in}}%
\pgfpathmoveto{\pgfqpoint{0.830943in}{0.887574in}}%
\pgfpathlineto{\pgfqpoint{0.995772in}{1.085600in}}%
\pgfpathmoveto{\pgfqpoint{0.830943in}{0.887574in}}%
\pgfpathlineto{\pgfqpoint{0.727204in}{1.085600in}}%
\pgfpathmoveto{\pgfqpoint{0.957322in}{0.756420in}}%
\pgfpathlineto{\pgfqpoint{1.117688in}{0.824660in}}%
\pgfpathmoveto{\pgfqpoint{0.957322in}{0.756420in}}%
\pgfpathlineto{\pgfqpoint{0.830943in}{0.887574in}}%
\pgfpathmoveto{\pgfqpoint{1.287084in}{0.857316in}}%
\pgfpathlineto{\pgfqpoint{1.117688in}{0.824660in}}%
\pgfpathmoveto{\pgfqpoint{0.665556in}{0.898299in}}%
\pgfpathlineto{\pgfqpoint{0.727204in}{1.085600in}}%
\pgfpathmoveto{\pgfqpoint{0.665556in}{0.898299in}}%
\pgfpathlineto{\pgfqpoint{0.830943in}{0.887574in}}%
\pgfpathmoveto{\pgfqpoint{1.090371in}{0.720244in}}%
\pgfpathlineto{\pgfqpoint{1.062700in}{0.609075in}}%
\pgfpathmoveto{\pgfqpoint{1.090371in}{0.720244in}}%
\pgfpathlineto{\pgfqpoint{1.142662in}{0.619169in}}%
\pgfpathmoveto{\pgfqpoint{1.090371in}{0.720244in}}%
\pgfpathlineto{\pgfqpoint{1.117688in}{0.824660in}}%
\pgfpathmoveto{\pgfqpoint{1.090371in}{0.720244in}}%
\pgfpathlineto{\pgfqpoint{0.957322in}{0.756420in}}%
\pgfpathmoveto{\pgfqpoint{1.219108in}{0.751293in}}%
\pgfpathlineto{\pgfqpoint{1.226275in}{0.635125in}}%
\pgfpathmoveto{\pgfqpoint{1.219108in}{0.751293in}}%
\pgfpathlineto{\pgfqpoint{1.313856in}{0.663445in}}%
\pgfpathmoveto{\pgfqpoint{1.219108in}{0.751293in}}%
\pgfpathlineto{\pgfqpoint{1.117688in}{0.824660in}}%
\pgfpathmoveto{\pgfqpoint{1.219108in}{0.751293in}}%
\pgfpathlineto{\pgfqpoint{1.287084in}{0.857316in}}%
\pgfpathmoveto{\pgfqpoint{1.219108in}{0.751293in}}%
\pgfpathlineto{\pgfqpoint{1.090371in}{0.720244in}}%
\pgfpathmoveto{\pgfqpoint{0.812724in}{0.731232in}}%
\pgfpathlineto{\pgfqpoint{0.742562in}{0.637869in}}%
\pgfpathmoveto{\pgfqpoint{0.812724in}{0.731232in}}%
\pgfpathlineto{\pgfqpoint{0.828308in}{0.618003in}}%
\pgfpathmoveto{\pgfqpoint{0.812724in}{0.731232in}}%
\pgfpathlineto{\pgfqpoint{0.830943in}{0.887574in}}%
\pgfpathmoveto{\pgfqpoint{0.812724in}{0.731232in}}%
\pgfpathlineto{\pgfqpoint{0.957322in}{0.756420in}}%
\pgfpathmoveto{\pgfqpoint{0.701489in}{0.784036in}}%
\pgfpathlineto{\pgfqpoint{0.576443in}{0.718597in}}%
\pgfpathmoveto{\pgfqpoint{0.701489in}{0.784036in}}%
\pgfpathlineto{\pgfqpoint{0.649179in}{0.669718in}}%
\pgfpathmoveto{\pgfqpoint{0.701489in}{0.784036in}}%
\pgfpathlineto{\pgfqpoint{0.830943in}{0.887574in}}%
\pgfpathmoveto{\pgfqpoint{0.701489in}{0.784036in}}%
\pgfpathlineto{\pgfqpoint{0.665556in}{0.898299in}}%
\pgfpathmoveto{\pgfqpoint{0.701489in}{0.784036in}}%
\pgfpathlineto{\pgfqpoint{0.812724in}{0.731232in}}%
\pgfpathmoveto{\pgfqpoint{1.373352in}{0.959810in}}%
\pgfpathlineto{\pgfqpoint{1.494399in}{0.840973in}}%
\pgfpathmoveto{\pgfqpoint{1.373352in}{0.959810in}}%
\pgfpathlineto{\pgfqpoint{1.532909in}{1.085600in}}%
\pgfpathmoveto{\pgfqpoint{1.373352in}{0.959810in}}%
\pgfpathlineto{\pgfqpoint{1.264340in}{1.085600in}}%
\pgfpathmoveto{\pgfqpoint{1.373352in}{0.959810in}}%
\pgfpathlineto{\pgfqpoint{1.287084in}{0.857316in}}%
\pgfpathmoveto{\pgfqpoint{0.949513in}{0.659395in}}%
\pgfpathlineto{\pgfqpoint{0.907725in}{0.603940in}}%
\pgfpathmoveto{\pgfqpoint{0.949513in}{0.659395in}}%
\pgfpathlineto{\pgfqpoint{0.985741in}{0.601166in}}%
\pgfpathmoveto{\pgfqpoint{0.949513in}{0.659395in}}%
\pgfpathlineto{\pgfqpoint{0.957322in}{0.756420in}}%
\pgfpathmoveto{\pgfqpoint{1.310269in}{0.744148in}}%
\pgfpathlineto{\pgfqpoint{1.313856in}{0.663445in}}%
\pgfpathmoveto{\pgfqpoint{1.310269in}{0.744148in}}%
\pgfpathlineto{\pgfqpoint{1.401434in}{0.701100in}}%
\pgfpathmoveto{\pgfqpoint{1.310269in}{0.744148in}}%
\pgfpathlineto{\pgfqpoint{1.287084in}{0.857316in}}%
\pgfpathmoveto{\pgfqpoint{1.310269in}{0.744148in}}%
\pgfpathlineto{\pgfqpoint{1.219108in}{0.751293in}}%
\pgfpathmoveto{\pgfqpoint{1.167827in}{0.689125in}}%
\pgfpathlineto{\pgfqpoint{1.142662in}{0.619169in}}%
\pgfpathmoveto{\pgfqpoint{1.167827in}{0.689125in}}%
\pgfpathlineto{\pgfqpoint{1.226275in}{0.635125in}}%
\pgfpathmoveto{\pgfqpoint{1.167827in}{0.689125in}}%
\pgfpathlineto{\pgfqpoint{1.090371in}{0.720244in}}%
\pgfpathmoveto{\pgfqpoint{1.167827in}{0.689125in}}%
\pgfpathlineto{\pgfqpoint{1.219108in}{0.751293in}}%
\pgfpathmoveto{\pgfqpoint{1.390634in}{0.855971in}}%
\pgfpathlineto{\pgfqpoint{1.494399in}{0.840973in}}%
\pgfpathmoveto{\pgfqpoint{1.390634in}{0.855971in}}%
\pgfpathlineto{\pgfqpoint{1.451289in}{0.764757in}}%
\pgfpathmoveto{\pgfqpoint{1.390634in}{0.855971in}}%
\pgfpathlineto{\pgfqpoint{1.287084in}{0.857316in}}%
\pgfpathmoveto{\pgfqpoint{1.390634in}{0.855971in}}%
\pgfpathlineto{\pgfqpoint{1.373352in}{0.959810in}}%
\pgfpathmoveto{\pgfqpoint{0.728960in}{0.711312in}}%
\pgfpathlineto{\pgfqpoint{0.649179in}{0.669718in}}%
\pgfpathmoveto{\pgfqpoint{0.728960in}{0.711312in}}%
\pgfpathlineto{\pgfqpoint{0.742562in}{0.637869in}}%
\pgfpathmoveto{\pgfqpoint{0.728960in}{0.711312in}}%
\pgfpathlineto{\pgfqpoint{0.812724in}{0.731232in}}%
\pgfpathmoveto{\pgfqpoint{0.728960in}{0.711312in}}%
\pgfpathlineto{\pgfqpoint{0.701489in}{0.784036in}}%
\pgfpathmoveto{\pgfqpoint{0.616584in}{0.797742in}}%
\pgfpathlineto{\pgfqpoint{0.529202in}{0.776277in}}%
\pgfpathmoveto{\pgfqpoint{0.616584in}{0.797742in}}%
\pgfpathlineto{\pgfqpoint{0.576443in}{0.718597in}}%
\pgfpathmoveto{\pgfqpoint{0.616584in}{0.797742in}}%
\pgfpathlineto{\pgfqpoint{0.665556in}{0.898299in}}%
\pgfpathmoveto{\pgfqpoint{0.616584in}{0.797742in}}%
\pgfpathlineto{\pgfqpoint{0.701489in}{0.784036in}}%
\pgfpathmoveto{\pgfqpoint{0.567574in}{0.969290in}}%
\pgfpathlineto{\pgfqpoint{0.458635in}{1.085600in}}%
\pgfpathmoveto{\pgfqpoint{0.567574in}{0.969290in}}%
\pgfpathlineto{\pgfqpoint{0.488920in}{0.850184in}}%
\pgfpathmoveto{\pgfqpoint{0.567574in}{0.969290in}}%
\pgfpathlineto{\pgfqpoint{0.727204in}{1.085600in}}%
\pgfpathmoveto{\pgfqpoint{0.567574in}{0.969290in}}%
\pgfpathlineto{\pgfqpoint{0.665556in}{0.898299in}}%
\pgfpathmoveto{\pgfqpoint{1.022401in}{0.663058in}}%
\pgfpathlineto{\pgfqpoint{0.985741in}{0.601166in}}%
\pgfpathmoveto{\pgfqpoint{1.022401in}{0.663058in}}%
\pgfpathlineto{\pgfqpoint{1.062700in}{0.609075in}}%
\pgfpathmoveto{\pgfqpoint{1.022401in}{0.663058in}}%
\pgfpathlineto{\pgfqpoint{0.957322in}{0.756420in}}%
\pgfpathmoveto{\pgfqpoint{1.022401in}{0.663058in}}%
\pgfpathlineto{\pgfqpoint{1.090371in}{0.720244in}}%
\pgfpathmoveto{\pgfqpoint{1.022401in}{0.663058in}}%
\pgfpathlineto{\pgfqpoint{0.949513in}{0.659395in}}%
\pgfpathmoveto{\pgfqpoint{0.876764in}{0.668875in}}%
\pgfpathlineto{\pgfqpoint{0.828308in}{0.618003in}}%
\pgfpathmoveto{\pgfqpoint{0.876764in}{0.668875in}}%
\pgfpathlineto{\pgfqpoint{0.907725in}{0.603940in}}%
\pgfpathmoveto{\pgfqpoint{0.876764in}{0.668875in}}%
\pgfpathlineto{\pgfqpoint{0.957322in}{0.756420in}}%
\pgfpathmoveto{\pgfqpoint{0.876764in}{0.668875in}}%
\pgfpathlineto{\pgfqpoint{0.812724in}{0.731232in}}%
\pgfpathmoveto{\pgfqpoint{0.876764in}{0.668875in}}%
\pgfpathlineto{\pgfqpoint{0.949513in}{0.659395in}}%
\pgfpathmoveto{\pgfqpoint{1.374566in}{0.778570in}}%
\pgfpathlineto{\pgfqpoint{1.401434in}{0.701100in}}%
\pgfpathmoveto{\pgfqpoint{1.374566in}{0.778570in}}%
\pgfpathlineto{\pgfqpoint{1.451289in}{0.764757in}}%
\pgfpathmoveto{\pgfqpoint{1.374566in}{0.778570in}}%
\pgfpathlineto{\pgfqpoint{1.287084in}{0.857316in}}%
\pgfpathmoveto{\pgfqpoint{1.374566in}{0.778570in}}%
\pgfpathlineto{\pgfqpoint{1.310269in}{0.744148in}}%
\pgfpathmoveto{\pgfqpoint{1.374566in}{0.778570in}}%
\pgfpathlineto{\pgfqpoint{1.390634in}{0.855971in}}%
\pgfpathmoveto{\pgfqpoint{1.016954in}{0.905905in}}%
\pgfpathlineto{\pgfqpoint{0.995772in}{1.085600in}}%
\pgfpathmoveto{\pgfqpoint{1.016954in}{0.905905in}}%
\pgfpathlineto{\pgfqpoint{1.117688in}{0.824660in}}%
\pgfpathmoveto{\pgfqpoint{1.016954in}{0.905905in}}%
\pgfpathlineto{\pgfqpoint{0.830943in}{0.887574in}}%
\pgfpathmoveto{\pgfqpoint{1.016954in}{0.905905in}}%
\pgfpathlineto{\pgfqpoint{0.957322in}{0.756420in}}%
\pgfpathmoveto{\pgfqpoint{1.178001in}{0.952692in}}%
\pgfpathlineto{\pgfqpoint{0.995772in}{1.085600in}}%
\pgfpathmoveto{\pgfqpoint{1.178001in}{0.952692in}}%
\pgfpathlineto{\pgfqpoint{1.264340in}{1.085600in}}%
\pgfpathmoveto{\pgfqpoint{1.178001in}{0.952692in}}%
\pgfpathlineto{\pgfqpoint{1.117688in}{0.824660in}}%
\pgfpathmoveto{\pgfqpoint{1.178001in}{0.952692in}}%
\pgfpathlineto{\pgfqpoint{1.287084in}{0.857316in}}%
\pgfpathmoveto{\pgfqpoint{1.178001in}{0.952692in}}%
\pgfpathlineto{\pgfqpoint{1.373352in}{0.959810in}}%
\pgfpathmoveto{\pgfqpoint{1.178001in}{0.952692in}}%
\pgfpathlineto{\pgfqpoint{1.016954in}{0.905905in}}%
\pgfpathmoveto{\pgfqpoint{0.572757in}{0.858321in}}%
\pgfpathlineto{\pgfqpoint{0.488920in}{0.850184in}}%
\pgfpathmoveto{\pgfqpoint{0.572757in}{0.858321in}}%
\pgfpathlineto{\pgfqpoint{0.529202in}{0.776277in}}%
\pgfpathmoveto{\pgfqpoint{0.572757in}{0.858321in}}%
\pgfpathlineto{\pgfqpoint{0.665556in}{0.898299in}}%
\pgfpathmoveto{\pgfqpoint{0.572757in}{0.858321in}}%
\pgfpathlineto{\pgfqpoint{0.616584in}{0.797742in}}%
\pgfpathmoveto{\pgfqpoint{0.572757in}{0.858321in}}%
\pgfpathlineto{\pgfqpoint{0.567574in}{0.969290in}}%
\pgfpathlineto{\pgfqpoint{0.567574in}{0.969290in}}%
\pgfusepath{stroke}%
\end{pgfscope}%
\begin{pgfscope}%
\pgfpathrectangle{\pgfqpoint{0.100000in}{0.100000in}}{\pgfqpoint{1.782500in}{1.232000in}}%
\pgfusepath{clip}%
\pgfsetbuttcap%
\pgfsetroundjoin%
\definecolor{currentfill}{rgb}{0.054902,0.262745,0.486275}%
\pgfsetfillcolor{currentfill}%
\pgfsetlinewidth{1.003750pt}%
\definecolor{currentstroke}{rgb}{0.054902,0.262745,0.486275}%
\pgfsetstrokecolor{currentstroke}%
\pgfsetdash{}{0pt}%
\pgfsys@defobject{currentmarker}{\pgfqpoint{-0.018373in}{-0.018373in}}{\pgfqpoint{0.018373in}{0.018373in}}{%
\pgfpathmoveto{\pgfqpoint{0.000000in}{-0.018373in}}%
\pgfpathcurveto{\pgfqpoint{0.004873in}{-0.018373in}}{\pgfqpoint{0.009546in}{-0.016437in}}{\pgfqpoint{0.012992in}{-0.012992in}}%
\pgfpathcurveto{\pgfqpoint{0.016437in}{-0.009546in}}{\pgfqpoint{0.018373in}{-0.004873in}}{\pgfqpoint{0.018373in}{0.000000in}}%
\pgfpathcurveto{\pgfqpoint{0.018373in}{0.004873in}}{\pgfqpoint{0.016437in}{0.009546in}}{\pgfqpoint{0.012992in}{0.012992in}}%
\pgfpathcurveto{\pgfqpoint{0.009546in}{0.016437in}}{\pgfqpoint{0.004873in}{0.018373in}}{\pgfqpoint{0.000000in}{0.018373in}}%
\pgfpathcurveto{\pgfqpoint{-0.004873in}{0.018373in}}{\pgfqpoint{-0.009546in}{0.016437in}}{\pgfqpoint{-0.012992in}{0.012992in}}%
\pgfpathcurveto{\pgfqpoint{-0.016437in}{0.009546in}}{\pgfqpoint{-0.018373in}{0.004873in}}{\pgfqpoint{-0.018373in}{0.000000in}}%
\pgfpathcurveto{\pgfqpoint{-0.018373in}{-0.004873in}}{\pgfqpoint{-0.016437in}{-0.009546in}}{\pgfqpoint{-0.012992in}{-0.012992in}}%
\pgfpathcurveto{\pgfqpoint{-0.009546in}{-0.016437in}}{\pgfqpoint{-0.004873in}{-0.018373in}}{\pgfqpoint{0.000000in}{-0.018373in}}%
\pgfpathlineto{\pgfqpoint{0.000000in}{-0.018373in}}%
\pgfpathclose%
\pgfusepath{stroke,fill}%
}%
\begin{pgfscope}%
\pgfsys@transformshift{1.300610in}{0.658103in}%
\pgfsys@useobject{currentmarker}{}%
\end{pgfscope}%
\begin{pgfscope}%
\pgfsys@transformshift{1.245113in}{0.640065in}%
\pgfsys@useobject{currentmarker}{}%
\end{pgfscope}%
\begin{pgfscope}%
\pgfsys@transformshift{1.187563in}{0.626883in}%
\pgfsys@useobject{currentmarker}{}%
\end{pgfscope}%
\begin{pgfscope}%
\pgfsys@transformshift{1.121983in}{0.616327in}%
\pgfsys@useobject{currentmarker}{}%
\end{pgfscope}%
\begin{pgfscope}%
\pgfsys@transformshift{1.054136in}{0.608017in}%
\pgfsys@useobject{currentmarker}{}%
\end{pgfscope}%
\begin{pgfscope}%
\pgfsys@transformshift{0.985700in}{0.601411in}%
\pgfsys@useobject{currentmarker}{}%
\end{pgfscope}%
\begin{pgfscope}%
\pgfsys@transformshift{0.916417in}{0.603252in}%
\pgfsys@useobject{currentmarker}{}%
\end{pgfscope}%
\begin{pgfscope}%
\pgfsys@transformshift{0.848834in}{0.613810in}%
\pgfsys@useobject{currentmarker}{}%
\end{pgfscope}%
\begin{pgfscope}%
\pgfsys@transformshift{0.783038in}{0.628109in}%
\pgfsys@useobject{currentmarker}{}%
\end{pgfscope}%
\begin{pgfscope}%
\pgfsys@transformshift{0.721820in}{0.643258in}%
\pgfsys@useobject{currentmarker}{}%
\end{pgfscope}%
\begin{pgfscope}%
\pgfsys@transformshift{0.661196in}{0.663049in}%
\pgfsys@useobject{currentmarker}{}%
\end{pgfscope}%
\end{pgfscope}%
\begin{pgfscope}%
\pgfpathrectangle{\pgfqpoint{0.100000in}{0.100000in}}{\pgfqpoint{1.782500in}{1.232000in}}%
\pgfusepath{clip}%
\pgfsetbuttcap%
\pgfsetroundjoin%
\definecolor{currentfill}{rgb}{0.835294,0.321569,0.035294}%
\pgfsetfillcolor{currentfill}%
\pgfsetlinewidth{1.003750pt}%
\definecolor{currentstroke}{rgb}{0.835294,0.321569,0.035294}%
\pgfsetstrokecolor{currentstroke}%
\pgfsetdash{}{0pt}%
\pgfsys@defobject{currentmarker}{\pgfqpoint{-0.018373in}{-0.018373in}}{\pgfqpoint{0.018373in}{0.018373in}}{%
\pgfpathmoveto{\pgfqpoint{0.000000in}{-0.018373in}}%
\pgfpathcurveto{\pgfqpoint{0.004873in}{-0.018373in}}{\pgfqpoint{0.009546in}{-0.016437in}}{\pgfqpoint{0.012992in}{-0.012992in}}%
\pgfpathcurveto{\pgfqpoint{0.016437in}{-0.009546in}}{\pgfqpoint{0.018373in}{-0.004873in}}{\pgfqpoint{0.018373in}{0.000000in}}%
\pgfpathcurveto{\pgfqpoint{0.018373in}{0.004873in}}{\pgfqpoint{0.016437in}{0.009546in}}{\pgfqpoint{0.012992in}{0.012992in}}%
\pgfpathcurveto{\pgfqpoint{0.009546in}{0.016437in}}{\pgfqpoint{0.004873in}{0.018373in}}{\pgfqpoint{0.000000in}{0.018373in}}%
\pgfpathcurveto{\pgfqpoint{-0.004873in}{0.018373in}}{\pgfqpoint{-0.009546in}{0.016437in}}{\pgfqpoint{-0.012992in}{0.012992in}}%
\pgfpathcurveto{\pgfqpoint{-0.016437in}{0.009546in}}{\pgfqpoint{-0.018373in}{0.004873in}}{\pgfqpoint{-0.018373in}{0.000000in}}%
\pgfpathcurveto{\pgfqpoint{-0.018373in}{-0.004873in}}{\pgfqpoint{-0.016437in}{-0.009546in}}{\pgfqpoint{-0.012992in}{-0.012992in}}%
\pgfpathcurveto{\pgfqpoint{-0.009546in}{-0.016437in}}{\pgfqpoint{-0.004873in}{-0.018373in}}{\pgfqpoint{0.000000in}{-0.018373in}}%
\pgfpathlineto{\pgfqpoint{0.000000in}{-0.018373in}}%
\pgfpathclose%
\pgfusepath{stroke,fill}%
}%
\begin{pgfscope}%
\pgfsys@transformshift{0.649179in}{0.669718in}%
\pgfsys@useobject{currentmarker}{}%
\end{pgfscope}%
\begin{pgfscope}%
\pgfsys@transformshift{0.742562in}{0.637869in}%
\pgfsys@useobject{currentmarker}{}%
\end{pgfscope}%
\begin{pgfscope}%
\pgfsys@transformshift{0.828308in}{0.618003in}%
\pgfsys@useobject{currentmarker}{}%
\end{pgfscope}%
\begin{pgfscope}%
\pgfsys@transformshift{0.907725in}{0.603940in}%
\pgfsys@useobject{currentmarker}{}%
\end{pgfscope}%
\begin{pgfscope}%
\pgfsys@transformshift{0.985741in}{0.601166in}%
\pgfsys@useobject{currentmarker}{}%
\end{pgfscope}%
\begin{pgfscope}%
\pgfsys@transformshift{1.062700in}{0.609075in}%
\pgfsys@useobject{currentmarker}{}%
\end{pgfscope}%
\begin{pgfscope}%
\pgfsys@transformshift{1.142662in}{0.619169in}%
\pgfsys@useobject{currentmarker}{}%
\end{pgfscope}%
\begin{pgfscope}%
\pgfsys@transformshift{1.226275in}{0.635125in}%
\pgfsys@useobject{currentmarker}{}%
\end{pgfscope}%
\begin{pgfscope}%
\pgfsys@transformshift{1.313856in}{0.663445in}%
\pgfsys@useobject{currentmarker}{}%
\end{pgfscope}%
\end{pgfscope}%
\begin{pgfscope}%
\pgfpathrectangle{\pgfqpoint{0.100000in}{0.100000in}}{\pgfqpoint{1.782500in}{1.232000in}}%
\pgfusepath{clip}%
\pgfsetbuttcap%
\pgfsetroundjoin%
\pgfsetlinewidth{1.003750pt}%
\definecolor{currentstroke}{rgb}{0.054902,0.262745,0.486275}%
\pgfsetstrokecolor{currentstroke}%
\pgfsetdash{}{0pt}%
\pgfpathmoveto{\pgfqpoint{0.000000in}{-0.018373in}}%
\pgfpathcurveto{\pgfqpoint{0.004873in}{-0.018373in}}{\pgfqpoint{0.009546in}{-0.016437in}}{\pgfqpoint{0.012992in}{-0.012992in}}%
\pgfpathcurveto{\pgfqpoint{0.016437in}{-0.009546in}}{\pgfqpoint{0.018373in}{-0.004873in}}{\pgfqpoint{0.018373in}{0.000000in}}%
\pgfpathcurveto{\pgfqpoint{0.018373in}{0.004873in}}{\pgfqpoint{0.016437in}{0.009546in}}{\pgfqpoint{0.012992in}{0.012992in}}%
\pgfpathcurveto{\pgfqpoint{0.009546in}{0.016437in}}{\pgfqpoint{0.004873in}{0.018373in}}{\pgfqpoint{0.000000in}{0.018373in}}%
\pgfpathcurveto{\pgfqpoint{-0.004873in}{0.018373in}}{\pgfqpoint{-0.009546in}{0.016437in}}{\pgfqpoint{-0.012992in}{0.012992in}}%
\pgfpathcurveto{\pgfqpoint{-0.016437in}{0.009546in}}{\pgfqpoint{-0.018373in}{0.004873in}}{\pgfqpoint{-0.018373in}{0.000000in}}%
\pgfpathcurveto{\pgfqpoint{-0.018373in}{-0.004873in}}{\pgfqpoint{-0.016437in}{-0.009546in}}{\pgfqpoint{-0.012992in}{-0.012992in}}%
\pgfpathcurveto{\pgfqpoint{-0.009546in}{-0.016437in}}{\pgfqpoint{-0.004873in}{-0.018373in}}{\pgfqpoint{0.000000in}{-0.018373in}}%
\pgfusepath{stroke}%
\end{pgfscope}%
\begin{pgfscope}%
\pgfpathrectangle{\pgfqpoint{0.100000in}{0.100000in}}{\pgfqpoint{1.782500in}{1.232000in}}%
\pgfusepath{clip}%
\pgfsetbuttcap%
\pgfsetroundjoin%
\pgfsetlinewidth{1.003750pt}%
\definecolor{currentstroke}{rgb}{0.835294,0.321569,0.035294}%
\pgfsetstrokecolor{currentstroke}%
\pgfsetdash{}{0pt}%
\pgfpathmoveto{\pgfqpoint{0.000000in}{-0.018373in}}%
\pgfpathcurveto{\pgfqpoint{0.004873in}{-0.018373in}}{\pgfqpoint{0.009546in}{-0.016437in}}{\pgfqpoint{0.012992in}{-0.012992in}}%
\pgfpathcurveto{\pgfqpoint{0.016437in}{-0.009546in}}{\pgfqpoint{0.018373in}{-0.004873in}}{\pgfqpoint{0.018373in}{0.000000in}}%
\pgfpathcurveto{\pgfqpoint{0.018373in}{0.004873in}}{\pgfqpoint{0.016437in}{0.009546in}}{\pgfqpoint{0.012992in}{0.012992in}}%
\pgfpathcurveto{\pgfqpoint{0.009546in}{0.016437in}}{\pgfqpoint{0.004873in}{0.018373in}}{\pgfqpoint{0.000000in}{0.018373in}}%
\pgfpathcurveto{\pgfqpoint{-0.004873in}{0.018373in}}{\pgfqpoint{-0.009546in}{0.016437in}}{\pgfqpoint{-0.012992in}{0.012992in}}%
\pgfpathcurveto{\pgfqpoint{-0.016437in}{0.009546in}}{\pgfqpoint{-0.018373in}{0.004873in}}{\pgfqpoint{-0.018373in}{0.000000in}}%
\pgfpathcurveto{\pgfqpoint{-0.018373in}{-0.004873in}}{\pgfqpoint{-0.016437in}{-0.009546in}}{\pgfqpoint{-0.012992in}{-0.012992in}}%
\pgfpathcurveto{\pgfqpoint{-0.009546in}{-0.016437in}}{\pgfqpoint{-0.004873in}{-0.018373in}}{\pgfqpoint{0.000000in}{-0.018373in}}%
\pgfusepath{stroke}%
\end{pgfscope}%
\begin{pgfscope}%
\pgfpathrectangle{\pgfqpoint{0.100000in}{0.100000in}}{\pgfqpoint{1.782500in}{1.232000in}}%
\pgfusepath{clip}%
\pgfsetbuttcap%
\pgfsetroundjoin%
\definecolor{currentfill}{rgb}{0.054902,0.262745,0.486275}%
\pgfsetfillcolor{currentfill}%
\pgfsetlinewidth{1.505625pt}%
\definecolor{currentstroke}{rgb}{0.054902,0.262745,0.486275}%
\pgfsetstrokecolor{currentstroke}%
\pgfsetdash{}{0pt}%
\pgfsys@defobject{currentmarker}{\pgfqpoint{-0.018373in}{-0.018373in}}{\pgfqpoint{0.018373in}{0.018373in}}{%
\pgfpathmoveto{\pgfqpoint{-0.018373in}{-0.018373in}}%
\pgfpathlineto{\pgfqpoint{0.018373in}{0.018373in}}%
\pgfpathmoveto{\pgfqpoint{-0.018373in}{0.018373in}}%
\pgfpathlineto{\pgfqpoint{0.018373in}{-0.018373in}}%
\pgfusepath{stroke,fill}%
}%
\begin{pgfscope}%
\pgfsys@transformshift{1.358413in}{0.681892in}%
\pgfsys@useobject{currentmarker}{}%
\end{pgfscope}%
\end{pgfscope}%
\begin{pgfscope}%
\pgfpathrectangle{\pgfqpoint{0.100000in}{0.100000in}}{\pgfqpoint{1.782500in}{1.232000in}}%
\pgfusepath{clip}%
\pgfsetbuttcap%
\pgfsetroundjoin%
\definecolor{currentfill}{rgb}{0.835294,0.321569,0.035294}%
\pgfsetfillcolor{currentfill}%
\pgfsetlinewidth{1.505625pt}%
\definecolor{currentstroke}{rgb}{0.835294,0.321569,0.035294}%
\pgfsetstrokecolor{currentstroke}%
\pgfsetdash{}{0pt}%
\pgfsys@defobject{currentmarker}{\pgfqpoint{-0.018373in}{-0.018373in}}{\pgfqpoint{0.018373in}{0.018373in}}{%
\pgfpathmoveto{\pgfqpoint{-0.018373in}{-0.018373in}}%
\pgfpathlineto{\pgfqpoint{0.018373in}{0.018373in}}%
\pgfpathmoveto{\pgfqpoint{-0.018373in}{0.018373in}}%
\pgfpathlineto{\pgfqpoint{0.018373in}{-0.018373in}}%
\pgfusepath{stroke,fill}%
}%
\begin{pgfscope}%
\pgfsys@transformshift{0.576443in}{0.718597in}%
\pgfsys@useobject{currentmarker}{}%
\end{pgfscope}%
\begin{pgfscope}%
\pgfsys@transformshift{1.401434in}{0.701100in}%
\pgfsys@useobject{currentmarker}{}%
\end{pgfscope}%
\end{pgfscope}%
\end{pgfpicture}%
\makeatother%
\endgroup%

        \caption{Iteration 2: Update interface}\label{fig:example-iter1-dumping}
    \end{subfigure}
    \begin{subfigure}[b]{.32\linewidth}
        %% Creator: Matplotlib, PGF backend
%%
%% To include the figure in your LaTeX document, write
%%   \input{<filename>.pgf}
%%
%% Make sure the required packages are loaded in your preamble
%%   \usepackage{pgf}
%%
%% Also ensure that all the required font packages are loaded; for instance,
%% the lmodern package is sometimes necessary when using math font.
%%   \usepackage{lmodern}
%%
%% Figures using additional raster images can only be included by \input if
%% they are in the same directory as the main LaTeX file. For loading figures
%% from other directories you can use the `import` package
%%   \usepackage{import}
%%
%% and then include the figures with
%%   \import{<path to file>}{<filename>.pgf}
%%
%% Matplotlib used the following preamble
%%   
%%   \usepackage{fontspec}
%%   \setmainfont{DejaVuSans.ttf}[Path=\detokenize{/home/fabio/Internodes-CM/.venv/lib/python3.8/site-packages/matplotlib/mpl-data/fonts/ttf/}]
%%   \setsansfont{DejaVuSans.ttf}[Path=\detokenize{/home/fabio/Internodes-CM/.venv/lib/python3.8/site-packages/matplotlib/mpl-data/fonts/ttf/}]
%%   \setmonofont{DejaVuSansMono.ttf}[Path=\detokenize{/home/fabio/Internodes-CM/.venv/lib/python3.8/site-packages/matplotlib/mpl-data/fonts/ttf/}]
%%   \makeatletter\@ifpackageloaded{underscore}{}{\usepackage[strings]{underscore}}\makeatother
%%
\begingroup%
\makeatletter%
\begin{pgfpicture}%
\pgfpathrectangle{\pgfpointorigin}{\pgfqpoint{1.982500in}{1.432000in}}%
\pgfusepath{use as bounding box, clip}%
\begin{pgfscope}%
\pgfsetbuttcap%
\pgfsetmiterjoin%
\definecolor{currentfill}{rgb}{1.000000,1.000000,1.000000}%
\pgfsetfillcolor{currentfill}%
\pgfsetlinewidth{0.000000pt}%
\definecolor{currentstroke}{rgb}{1.000000,1.000000,1.000000}%
\pgfsetstrokecolor{currentstroke}%
\pgfsetdash{}{0pt}%
\pgfpathmoveto{\pgfqpoint{0.000000in}{0.000000in}}%
\pgfpathlineto{\pgfqpoint{1.982500in}{0.000000in}}%
\pgfpathlineto{\pgfqpoint{1.982500in}{1.432000in}}%
\pgfpathlineto{\pgfqpoint{0.000000in}{1.432000in}}%
\pgfpathlineto{\pgfqpoint{0.000000in}{0.000000in}}%
\pgfpathclose%
\pgfusepath{fill}%
\end{pgfscope}%
\begin{pgfscope}%
\pgfpathrectangle{\pgfqpoint{0.100000in}{0.100000in}}{\pgfqpoint{1.782500in}{1.232000in}}%
\pgfusepath{clip}%
\pgfsetrectcap%
\pgfsetroundjoin%
\pgfsetlinewidth{0.250937pt}%
\definecolor{currentstroke}{rgb}{0.054902,0.262745,0.486275}%
\pgfsetstrokecolor{currentstroke}%
\pgfsetdash{}{0pt}%
\pgfpathmoveto{\pgfqpoint{1.882500in}{0.452000in}}%
\pgfpathlineto{\pgfqpoint{1.892500in}{0.452000in}}%
\pgfpathmoveto{\pgfqpoint{1.729714in}{0.452000in}}%
\pgfpathlineto{\pgfqpoint{1.882500in}{0.452000in}}%
\pgfpathmoveto{\pgfqpoint{1.576929in}{0.452000in}}%
\pgfpathlineto{\pgfqpoint{1.729714in}{0.452000in}}%
\pgfpathmoveto{\pgfqpoint{1.424143in}{0.452000in}}%
\pgfpathlineto{\pgfqpoint{1.576929in}{0.452000in}}%
\pgfpathmoveto{\pgfqpoint{1.271357in}{0.452000in}}%
\pgfpathlineto{\pgfqpoint{1.424143in}{0.452000in}}%
\pgfpathmoveto{\pgfqpoint{1.118571in}{0.452000in}}%
\pgfpathlineto{\pgfqpoint{1.271357in}{0.452000in}}%
\pgfpathmoveto{\pgfqpoint{0.965786in}{0.452000in}}%
\pgfpathlineto{\pgfqpoint{0.813000in}{0.452000in}}%
\pgfpathmoveto{\pgfqpoint{0.965786in}{0.452000in}}%
\pgfpathlineto{\pgfqpoint{1.118571in}{0.452000in}}%
\pgfpathmoveto{\pgfqpoint{0.456500in}{0.452000in}}%
\pgfpathlineto{\pgfqpoint{0.813000in}{0.452000in}}%
\pgfpathmoveto{\pgfqpoint{0.456500in}{0.452000in}}%
\pgfpathlineto{\pgfqpoint{0.100000in}{0.452000in}}%
\pgfpathmoveto{\pgfqpoint{0.100000in}{0.090000in}}%
\pgfpathlineto{\pgfqpoint{0.100000in}{0.452000in}}%
\pgfpathmoveto{\pgfqpoint{1.805208in}{0.175249in}}%
\pgfpathlineto{\pgfqpoint{1.892500in}{0.121802in}}%
\pgfpathmoveto{\pgfqpoint{1.805208in}{0.175249in}}%
\pgfpathlineto{\pgfqpoint{1.637355in}{0.090000in}}%
\pgfpathmoveto{\pgfqpoint{1.195930in}{0.177192in}}%
\pgfpathlineto{\pgfqpoint{1.374206in}{0.090000in}}%
\pgfpathmoveto{\pgfqpoint{1.195930in}{0.177192in}}%
\pgfpathlineto{\pgfqpoint{1.042511in}{0.090000in}}%
\pgfpathmoveto{\pgfqpoint{1.892500in}{0.189285in}}%
\pgfpathlineto{\pgfqpoint{1.805208in}{0.175249in}}%
\pgfpathmoveto{\pgfqpoint{1.501834in}{0.227677in}}%
\pgfpathlineto{\pgfqpoint{1.506188in}{0.090000in}}%
\pgfpathmoveto{\pgfqpoint{1.501834in}{0.227677in}}%
\pgfpathlineto{\pgfqpoint{1.805208in}{0.175249in}}%
\pgfpathmoveto{\pgfqpoint{1.501834in}{0.227677in}}%
\pgfpathlineto{\pgfqpoint{1.195930in}{0.177192in}}%
\pgfpathmoveto{\pgfqpoint{0.856997in}{0.215161in}}%
\pgfpathlineto{\pgfqpoint{0.876107in}{0.090000in}}%
\pgfpathmoveto{\pgfqpoint{0.856997in}{0.215161in}}%
\pgfpathlineto{\pgfqpoint{1.195930in}{0.177192in}}%
\pgfpathmoveto{\pgfqpoint{0.485360in}{0.139977in}}%
\pgfpathlineto{\pgfqpoint{0.100000in}{0.452000in}}%
\pgfpathmoveto{\pgfqpoint{0.485360in}{0.139977in}}%
\pgfpathlineto{\pgfqpoint{0.456500in}{0.452000in}}%
\pgfpathmoveto{\pgfqpoint{0.485360in}{0.139977in}}%
\pgfpathlineto{\pgfqpoint{0.334871in}{0.090000in}}%
\pgfpathmoveto{\pgfqpoint{0.485360in}{0.139977in}}%
\pgfpathlineto{\pgfqpoint{0.632751in}{0.090000in}}%
\pgfpathmoveto{\pgfqpoint{0.485360in}{0.139977in}}%
\pgfpathlineto{\pgfqpoint{0.856997in}{0.215161in}}%
\pgfpathmoveto{\pgfqpoint{1.892500in}{0.434693in}}%
\pgfpathlineto{\pgfqpoint{1.882500in}{0.452000in}}%
\pgfpathmoveto{\pgfqpoint{1.892500in}{0.257344in}}%
\pgfpathlineto{\pgfqpoint{1.805208in}{0.175249in}}%
\pgfpathmoveto{\pgfqpoint{1.653321in}{0.319800in}}%
\pgfpathlineto{\pgfqpoint{1.729714in}{0.452000in}}%
\pgfpathmoveto{\pgfqpoint{1.653321in}{0.319800in}}%
\pgfpathlineto{\pgfqpoint{1.576929in}{0.452000in}}%
\pgfpathmoveto{\pgfqpoint{1.653321in}{0.319800in}}%
\pgfpathlineto{\pgfqpoint{1.805208in}{0.175249in}}%
\pgfpathmoveto{\pgfqpoint{1.653321in}{0.319800in}}%
\pgfpathlineto{\pgfqpoint{1.501834in}{0.227677in}}%
\pgfpathmoveto{\pgfqpoint{1.347750in}{0.319800in}}%
\pgfpathlineto{\pgfqpoint{1.424143in}{0.452000in}}%
\pgfpathmoveto{\pgfqpoint{1.347750in}{0.319800in}}%
\pgfpathlineto{\pgfqpoint{1.271357in}{0.452000in}}%
\pgfpathmoveto{\pgfqpoint{1.347750in}{0.319800in}}%
\pgfpathlineto{\pgfqpoint{1.195930in}{0.177192in}}%
\pgfpathmoveto{\pgfqpoint{1.347750in}{0.319800in}}%
\pgfpathlineto{\pgfqpoint{1.501834in}{0.227677in}}%
\pgfpathmoveto{\pgfqpoint{1.042179in}{0.319800in}}%
\pgfpathlineto{\pgfqpoint{1.118571in}{0.452000in}}%
\pgfpathmoveto{\pgfqpoint{1.042179in}{0.319800in}}%
\pgfpathlineto{\pgfqpoint{0.965786in}{0.452000in}}%
\pgfpathmoveto{\pgfqpoint{1.042179in}{0.319800in}}%
\pgfpathlineto{\pgfqpoint{1.195930in}{0.177192in}}%
\pgfpathmoveto{\pgfqpoint{1.042179in}{0.319800in}}%
\pgfpathlineto{\pgfqpoint{0.856997in}{0.215161in}}%
\pgfpathmoveto{\pgfqpoint{1.793890in}{0.090000in}}%
\pgfpathlineto{\pgfqpoint{1.805208in}{0.175249in}}%
\pgfpathmoveto{\pgfqpoint{0.655483in}{0.306719in}}%
\pgfpathlineto{\pgfqpoint{0.813000in}{0.452000in}}%
\pgfpathmoveto{\pgfqpoint{0.655483in}{0.306719in}}%
\pgfpathlineto{\pgfqpoint{0.456500in}{0.452000in}}%
\pgfpathmoveto{\pgfqpoint{0.655483in}{0.306719in}}%
\pgfpathlineto{\pgfqpoint{0.856997in}{0.215161in}}%
\pgfpathmoveto{\pgfqpoint{0.655483in}{0.306719in}}%
\pgfpathlineto{\pgfqpoint{0.485360in}{0.139977in}}%
\pgfpathmoveto{\pgfqpoint{1.805927in}{0.343767in}}%
\pgfpathlineto{\pgfqpoint{1.882500in}{0.452000in}}%
\pgfpathmoveto{\pgfqpoint{1.805927in}{0.343767in}}%
\pgfpathlineto{\pgfqpoint{1.729714in}{0.452000in}}%
\pgfpathmoveto{\pgfqpoint{1.805927in}{0.343767in}}%
\pgfpathlineto{\pgfqpoint{1.805208in}{0.175249in}}%
\pgfpathmoveto{\pgfqpoint{1.805927in}{0.343767in}}%
\pgfpathlineto{\pgfqpoint{1.892500in}{0.330194in}}%
\pgfpathmoveto{\pgfqpoint{1.805927in}{0.343767in}}%
\pgfpathlineto{\pgfqpoint{1.653321in}{0.319800in}}%
\pgfpathmoveto{\pgfqpoint{1.500668in}{0.351969in}}%
\pgfpathlineto{\pgfqpoint{1.576929in}{0.452000in}}%
\pgfpathmoveto{\pgfqpoint{1.500668in}{0.351969in}}%
\pgfpathlineto{\pgfqpoint{1.424143in}{0.452000in}}%
\pgfpathmoveto{\pgfqpoint{1.500668in}{0.351969in}}%
\pgfpathlineto{\pgfqpoint{1.501834in}{0.227677in}}%
\pgfpathmoveto{\pgfqpoint{1.500668in}{0.351969in}}%
\pgfpathlineto{\pgfqpoint{1.653321in}{0.319800in}}%
\pgfpathmoveto{\pgfqpoint{1.500668in}{0.351969in}}%
\pgfpathlineto{\pgfqpoint{1.347750in}{0.319800in}}%
\pgfpathmoveto{\pgfqpoint{1.206988in}{0.090000in}}%
\pgfpathlineto{\pgfqpoint{1.195930in}{0.177192in}}%
\pgfpathmoveto{\pgfqpoint{1.195157in}{0.344158in}}%
\pgfpathlineto{\pgfqpoint{1.271357in}{0.452000in}}%
\pgfpathmoveto{\pgfqpoint{1.195157in}{0.344158in}}%
\pgfpathlineto{\pgfqpoint{1.118571in}{0.452000in}}%
\pgfpathmoveto{\pgfqpoint{1.195157in}{0.344158in}}%
\pgfpathlineto{\pgfqpoint{1.195930in}{0.177192in}}%
\pgfpathmoveto{\pgfqpoint{1.195157in}{0.344158in}}%
\pgfpathlineto{\pgfqpoint{1.347750in}{0.319800in}}%
\pgfpathmoveto{\pgfqpoint{1.195157in}{0.344158in}}%
\pgfpathlineto{\pgfqpoint{1.042179in}{0.319800in}}%
\pgfpathmoveto{\pgfqpoint{0.889393in}{0.349588in}}%
\pgfpathlineto{\pgfqpoint{0.813000in}{0.452000in}}%
\pgfpathmoveto{\pgfqpoint{0.889393in}{0.349588in}}%
\pgfpathlineto{\pgfqpoint{0.965786in}{0.452000in}}%
\pgfpathmoveto{\pgfqpoint{0.889393in}{0.349588in}}%
\pgfpathlineto{\pgfqpoint{0.856997in}{0.215161in}}%
\pgfpathmoveto{\pgfqpoint{0.889393in}{0.349588in}}%
\pgfpathlineto{\pgfqpoint{1.042179in}{0.319800in}}%
\pgfpathmoveto{\pgfqpoint{0.889393in}{0.349588in}}%
\pgfpathlineto{\pgfqpoint{0.655483in}{0.306719in}}%
\pgfpathmoveto{\pgfqpoint{0.494080in}{0.090000in}}%
\pgfpathlineto{\pgfqpoint{0.485360in}{0.139977in}}%
\pgfpathlineto{\pgfqpoint{0.485360in}{0.139977in}}%
\pgfusepath{stroke}%
\end{pgfscope}%
\begin{pgfscope}%
\pgfpathrectangle{\pgfqpoint{0.100000in}{0.100000in}}{\pgfqpoint{1.782500in}{1.232000in}}%
\pgfusepath{clip}%
\pgfsetrectcap%
\pgfsetroundjoin%
\pgfsetlinewidth{0.250937pt}%
\definecolor{currentstroke}{rgb}{0.835294,0.321569,0.035294}%
\pgfsetstrokecolor{currentstroke}%
\pgfsetdash{}{0pt}%
\pgfpathmoveto{\pgfqpoint{0.813000in}{0.860417in}}%
\pgfpathlineto{\pgfqpoint{0.694167in}{1.244000in}}%
\pgfpathmoveto{\pgfqpoint{1.288333in}{1.244000in}}%
\pgfpathlineto{\pgfqpoint{1.882500in}{1.244000in}}%
\pgfpathmoveto{\pgfqpoint{1.288333in}{1.244000in}}%
\pgfpathlineto{\pgfqpoint{0.694167in}{1.244000in}}%
\pgfpathmoveto{\pgfqpoint{0.909162in}{0.739160in}}%
\pgfpathlineto{\pgfqpoint{0.813000in}{0.860417in}}%
\pgfpathmoveto{\pgfqpoint{1.030178in}{0.630794in}}%
\pgfpathlineto{\pgfqpoint{0.909162in}{0.739160in}}%
\pgfpathmoveto{\pgfqpoint{1.172958in}{0.538087in}}%
\pgfpathlineto{\pgfqpoint{1.030178in}{0.630794in}}%
\pgfpathmoveto{\pgfqpoint{1.333857in}{0.463404in}}%
\pgfpathlineto{\pgfqpoint{1.172958in}{0.538087in}}%
\pgfpathmoveto{\pgfqpoint{1.508764in}{0.408655in}}%
\pgfpathlineto{\pgfqpoint{1.333857in}{0.463404in}}%
\pgfpathmoveto{\pgfqpoint{1.693215in}{0.375235in}}%
\pgfpathlineto{\pgfqpoint{1.508764in}{0.408655in}}%
\pgfpathmoveto{\pgfqpoint{1.882500in}{0.364000in}}%
\pgfpathlineto{\pgfqpoint{1.693215in}{0.375235in}}%
\pgfpathmoveto{\pgfqpoint{1.892500in}{0.364594in}}%
\pgfpathlineto{\pgfqpoint{1.882500in}{0.364000in}}%
\pgfpathmoveto{\pgfqpoint{1.892500in}{1.244000in}}%
\pgfpathlineto{\pgfqpoint{1.882500in}{1.244000in}}%
\pgfpathmoveto{\pgfqpoint{1.544975in}{0.898385in}}%
\pgfpathlineto{\pgfqpoint{1.882500in}{1.244000in}}%
\pgfpathmoveto{\pgfqpoint{1.544975in}{0.898385in}}%
\pgfpathlineto{\pgfqpoint{1.288333in}{1.244000in}}%
\pgfpathmoveto{\pgfqpoint{1.817713in}{0.645445in}}%
\pgfpathlineto{\pgfqpoint{1.892500in}{0.675390in}}%
\pgfpathmoveto{\pgfqpoint{1.817713in}{0.645445in}}%
\pgfpathlineto{\pgfqpoint{1.544975in}{0.898385in}}%
\pgfpathmoveto{\pgfqpoint{1.186381in}{0.929127in}}%
\pgfpathlineto{\pgfqpoint{1.288333in}{1.244000in}}%
\pgfpathmoveto{\pgfqpoint{1.186381in}{0.929127in}}%
\pgfpathlineto{\pgfqpoint{1.544975in}{0.898385in}}%
\pgfpathmoveto{\pgfqpoint{1.892500in}{0.628836in}}%
\pgfpathlineto{\pgfqpoint{1.817713in}{0.645445in}}%
\pgfpathmoveto{\pgfqpoint{1.511108in}{0.615050in}}%
\pgfpathlineto{\pgfqpoint{1.333857in}{0.463404in}}%
\pgfpathmoveto{\pgfqpoint{1.511108in}{0.615050in}}%
\pgfpathlineto{\pgfqpoint{1.508764in}{0.408655in}}%
\pgfpathmoveto{\pgfqpoint{1.511108in}{0.615050in}}%
\pgfpathlineto{\pgfqpoint{1.544975in}{0.898385in}}%
\pgfpathmoveto{\pgfqpoint{1.511108in}{0.615050in}}%
\pgfpathlineto{\pgfqpoint{1.817713in}{0.645445in}}%
\pgfpathmoveto{\pgfqpoint{1.278705in}{0.730225in}}%
\pgfpathlineto{\pgfqpoint{1.030178in}{0.630794in}}%
\pgfpathmoveto{\pgfqpoint{1.278705in}{0.730225in}}%
\pgfpathlineto{\pgfqpoint{1.172958in}{0.538087in}}%
\pgfpathmoveto{\pgfqpoint{1.278705in}{0.730225in}}%
\pgfpathlineto{\pgfqpoint{1.544975in}{0.898385in}}%
\pgfpathmoveto{\pgfqpoint{1.278705in}{0.730225in}}%
\pgfpathlineto{\pgfqpoint{1.186381in}{0.929127in}}%
\pgfpathmoveto{\pgfqpoint{1.278705in}{0.730225in}}%
\pgfpathlineto{\pgfqpoint{1.511108in}{0.615050in}}%
\pgfpathmoveto{\pgfqpoint{1.798690in}{0.469691in}}%
\pgfpathlineto{\pgfqpoint{1.693215in}{0.375235in}}%
\pgfpathmoveto{\pgfqpoint{1.798690in}{0.469691in}}%
\pgfpathlineto{\pgfqpoint{1.882500in}{0.364000in}}%
\pgfpathmoveto{\pgfqpoint{1.798690in}{0.469691in}}%
\pgfpathlineto{\pgfqpoint{1.817713in}{0.645445in}}%
\pgfpathmoveto{\pgfqpoint{1.334493in}{0.597404in}}%
\pgfpathlineto{\pgfqpoint{1.172958in}{0.538087in}}%
\pgfpathmoveto{\pgfqpoint{1.334493in}{0.597404in}}%
\pgfpathlineto{\pgfqpoint{1.333857in}{0.463404in}}%
\pgfpathmoveto{\pgfqpoint{1.334493in}{0.597404in}}%
\pgfpathlineto{\pgfqpoint{1.511108in}{0.615050in}}%
\pgfpathmoveto{\pgfqpoint{1.334493in}{0.597404in}}%
\pgfpathlineto{\pgfqpoint{1.278705in}{0.730225in}}%
\pgfpathmoveto{\pgfqpoint{1.095926in}{0.762297in}}%
\pgfpathlineto{\pgfqpoint{0.909162in}{0.739160in}}%
\pgfpathmoveto{\pgfqpoint{1.095926in}{0.762297in}}%
\pgfpathlineto{\pgfqpoint{1.030178in}{0.630794in}}%
\pgfpathmoveto{\pgfqpoint{1.095926in}{0.762297in}}%
\pgfpathlineto{\pgfqpoint{1.186381in}{0.929127in}}%
\pgfpathmoveto{\pgfqpoint{1.095926in}{0.762297in}}%
\pgfpathlineto{\pgfqpoint{1.278705in}{0.730225in}}%
\pgfpathmoveto{\pgfqpoint{0.960375in}{1.053168in}}%
\pgfpathlineto{\pgfqpoint{0.694167in}{1.244000in}}%
\pgfpathmoveto{\pgfqpoint{0.960375in}{1.053168in}}%
\pgfpathlineto{\pgfqpoint{0.813000in}{0.860417in}}%
\pgfpathmoveto{\pgfqpoint{0.960375in}{1.053168in}}%
\pgfpathlineto{\pgfqpoint{1.288333in}{1.244000in}}%
\pgfpathmoveto{\pgfqpoint{0.960375in}{1.053168in}}%
\pgfpathlineto{\pgfqpoint{1.186381in}{0.929127in}}%
\pgfpathmoveto{\pgfqpoint{1.892500in}{0.376929in}}%
\pgfpathlineto{\pgfqpoint{1.882500in}{0.364000in}}%
\pgfpathmoveto{\pgfqpoint{1.892500in}{0.558019in}}%
\pgfpathlineto{\pgfqpoint{1.817713in}{0.645445in}}%
\pgfpathmoveto{\pgfqpoint{1.892500in}{0.471002in}}%
\pgfpathlineto{\pgfqpoint{1.798690in}{0.469691in}}%
\pgfpathmoveto{\pgfqpoint{1.633962in}{0.491742in}}%
\pgfpathlineto{\pgfqpoint{1.508764in}{0.408655in}}%
\pgfpathmoveto{\pgfqpoint{1.633962in}{0.491742in}}%
\pgfpathlineto{\pgfqpoint{1.693215in}{0.375235in}}%
\pgfpathmoveto{\pgfqpoint{1.633962in}{0.491742in}}%
\pgfpathlineto{\pgfqpoint{1.817713in}{0.645445in}}%
\pgfpathmoveto{\pgfqpoint{1.633962in}{0.491742in}}%
\pgfpathlineto{\pgfqpoint{1.511108in}{0.615050in}}%
\pgfpathmoveto{\pgfqpoint{1.633962in}{0.491742in}}%
\pgfpathlineto{\pgfqpoint{1.798690in}{0.469691in}}%
\pgfpathmoveto{\pgfqpoint{1.892500in}{1.184050in}}%
\pgfpathlineto{\pgfqpoint{1.882500in}{1.244000in}}%
\pgfpathmoveto{\pgfqpoint{1.892500in}{0.919808in}}%
\pgfpathlineto{\pgfqpoint{1.544975in}{0.898385in}}%
\pgfpathmoveto{\pgfqpoint{1.892500in}{0.820421in}}%
\pgfpathlineto{\pgfqpoint{1.817713in}{0.645445in}}%
\pgfpathmoveto{\pgfqpoint{1.892500in}{1.238184in}}%
\pgfpathlineto{\pgfqpoint{1.882500in}{1.244000in}}%
\pgfpathmoveto{\pgfqpoint{0.992969in}{0.868834in}}%
\pgfpathlineto{\pgfqpoint{0.813000in}{0.860417in}}%
\pgfpathmoveto{\pgfqpoint{0.992969in}{0.868834in}}%
\pgfpathlineto{\pgfqpoint{0.909162in}{0.739160in}}%
\pgfpathmoveto{\pgfqpoint{0.992969in}{0.868834in}}%
\pgfpathlineto{\pgfqpoint{1.186381in}{0.929127in}}%
\pgfpathmoveto{\pgfqpoint{0.992969in}{0.868834in}}%
\pgfpathlineto{\pgfqpoint{1.095926in}{0.762297in}}%
\pgfpathmoveto{\pgfqpoint{0.992969in}{0.868834in}}%
\pgfpathlineto{\pgfqpoint{0.960375in}{1.053168in}}%
\pgfpathlineto{\pgfqpoint{0.960375in}{1.053168in}}%
\pgfusepath{stroke}%
\end{pgfscope}%
\begin{pgfscope}%
\pgfpathrectangle{\pgfqpoint{0.100000in}{0.100000in}}{\pgfqpoint{1.782500in}{1.232000in}}%
\pgfusepath{clip}%
\pgfsetbuttcap%
\pgfsetroundjoin%
\definecolor{currentfill}{rgb}{0.054902,0.262745,0.486275}%
\pgfsetfillcolor{currentfill}%
\pgfsetlinewidth{1.003750pt}%
\definecolor{currentstroke}{rgb}{0.054902,0.262745,0.486275}%
\pgfsetstrokecolor{currentstroke}%
\pgfsetdash{}{0pt}%
\pgfsys@defobject{currentmarker}{\pgfqpoint{-0.018373in}{-0.018373in}}{\pgfqpoint{0.018373in}{0.018373in}}{%
\pgfpathmoveto{\pgfqpoint{0.000000in}{-0.018373in}}%
\pgfpathcurveto{\pgfqpoint{0.004873in}{-0.018373in}}{\pgfqpoint{0.009546in}{-0.016437in}}{\pgfqpoint{0.012992in}{-0.012992in}}%
\pgfpathcurveto{\pgfqpoint{0.016437in}{-0.009546in}}{\pgfqpoint{0.018373in}{-0.004873in}}{\pgfqpoint{0.018373in}{0.000000in}}%
\pgfpathcurveto{\pgfqpoint{0.018373in}{0.004873in}}{\pgfqpoint{0.016437in}{0.009546in}}{\pgfqpoint{0.012992in}{0.012992in}}%
\pgfpathcurveto{\pgfqpoint{0.009546in}{0.016437in}}{\pgfqpoint{0.004873in}{0.018373in}}{\pgfqpoint{0.000000in}{0.018373in}}%
\pgfpathcurveto{\pgfqpoint{-0.004873in}{0.018373in}}{\pgfqpoint{-0.009546in}{0.016437in}}{\pgfqpoint{-0.012992in}{0.012992in}}%
\pgfpathcurveto{\pgfqpoint{-0.016437in}{0.009546in}}{\pgfqpoint{-0.018373in}{0.004873in}}{\pgfqpoint{-0.018373in}{0.000000in}}%
\pgfpathcurveto{\pgfqpoint{-0.018373in}{-0.004873in}}{\pgfqpoint{-0.016437in}{-0.009546in}}{\pgfqpoint{-0.012992in}{-0.012992in}}%
\pgfpathcurveto{\pgfqpoint{-0.009546in}{-0.016437in}}{\pgfqpoint{-0.004873in}{-0.018373in}}{\pgfqpoint{0.000000in}{-0.018373in}}%
\pgfpathlineto{\pgfqpoint{0.000000in}{-0.018373in}}%
\pgfpathclose%
\pgfusepath{stroke,fill}%
}%
\begin{pgfscope}%
\pgfsys@transformshift{2.646429in}{0.452000in}%
\pgfsys@useobject{currentmarker}{}%
\end{pgfscope}%
\begin{pgfscope}%
\pgfsys@transformshift{2.493643in}{0.452000in}%
\pgfsys@useobject{currentmarker}{}%
\end{pgfscope}%
\begin{pgfscope}%
\pgfsys@transformshift{2.340857in}{0.452000in}%
\pgfsys@useobject{currentmarker}{}%
\end{pgfscope}%
\begin{pgfscope}%
\pgfsys@transformshift{2.188071in}{0.452000in}%
\pgfsys@useobject{currentmarker}{}%
\end{pgfscope}%
\begin{pgfscope}%
\pgfsys@transformshift{2.035286in}{0.452000in}%
\pgfsys@useobject{currentmarker}{}%
\end{pgfscope}%
\begin{pgfscope}%
\pgfsys@transformshift{1.882500in}{0.452000in}%
\pgfsys@useobject{currentmarker}{}%
\end{pgfscope}%
\begin{pgfscope}%
\pgfsys@transformshift{1.729714in}{0.452000in}%
\pgfsys@useobject{currentmarker}{}%
\end{pgfscope}%
\begin{pgfscope}%
\pgfsys@transformshift{1.576929in}{0.452000in}%
\pgfsys@useobject{currentmarker}{}%
\end{pgfscope}%
\begin{pgfscope}%
\pgfsys@transformshift{1.424143in}{0.452000in}%
\pgfsys@useobject{currentmarker}{}%
\end{pgfscope}%
\begin{pgfscope}%
\pgfsys@transformshift{1.271357in}{0.452000in}%
\pgfsys@useobject{currentmarker}{}%
\end{pgfscope}%
\begin{pgfscope}%
\pgfsys@transformshift{1.118571in}{0.452000in}%
\pgfsys@useobject{currentmarker}{}%
\end{pgfscope}%
\end{pgfscope}%
\begin{pgfscope}%
\pgfpathrectangle{\pgfqpoint{0.100000in}{0.100000in}}{\pgfqpoint{1.782500in}{1.232000in}}%
\pgfusepath{clip}%
\pgfsetbuttcap%
\pgfsetroundjoin%
\definecolor{currentfill}{rgb}{0.835294,0.321569,0.035294}%
\pgfsetfillcolor{currentfill}%
\pgfsetlinewidth{1.003750pt}%
\definecolor{currentstroke}{rgb}{0.835294,0.321569,0.035294}%
\pgfsetstrokecolor{currentstroke}%
\pgfsetdash{}{0pt}%
\pgfsys@defobject{currentmarker}{\pgfqpoint{-0.018373in}{-0.018373in}}{\pgfqpoint{0.018373in}{0.018373in}}{%
\pgfpathmoveto{\pgfqpoint{0.000000in}{-0.018373in}}%
\pgfpathcurveto{\pgfqpoint{0.004873in}{-0.018373in}}{\pgfqpoint{0.009546in}{-0.016437in}}{\pgfqpoint{0.012992in}{-0.012992in}}%
\pgfpathcurveto{\pgfqpoint{0.016437in}{-0.009546in}}{\pgfqpoint{0.018373in}{-0.004873in}}{\pgfqpoint{0.018373in}{0.000000in}}%
\pgfpathcurveto{\pgfqpoint{0.018373in}{0.004873in}}{\pgfqpoint{0.016437in}{0.009546in}}{\pgfqpoint{0.012992in}{0.012992in}}%
\pgfpathcurveto{\pgfqpoint{0.009546in}{0.016437in}}{\pgfqpoint{0.004873in}{0.018373in}}{\pgfqpoint{0.000000in}{0.018373in}}%
\pgfpathcurveto{\pgfqpoint{-0.004873in}{0.018373in}}{\pgfqpoint{-0.009546in}{0.016437in}}{\pgfqpoint{-0.012992in}{0.012992in}}%
\pgfpathcurveto{\pgfqpoint{-0.016437in}{0.009546in}}{\pgfqpoint{-0.018373in}{0.004873in}}{\pgfqpoint{-0.018373in}{0.000000in}}%
\pgfpathcurveto{\pgfqpoint{-0.018373in}{-0.004873in}}{\pgfqpoint{-0.016437in}{-0.009546in}}{\pgfqpoint{-0.012992in}{-0.012992in}}%
\pgfpathcurveto{\pgfqpoint{-0.009546in}{-0.016437in}}{\pgfqpoint{-0.004873in}{-0.018373in}}{\pgfqpoint{0.000000in}{-0.018373in}}%
\pgfpathlineto{\pgfqpoint{0.000000in}{-0.018373in}}%
\pgfpathclose%
\pgfusepath{stroke,fill}%
}%
\begin{pgfscope}%
\pgfsys@transformshift{1.172958in}{0.538087in}%
\pgfsys@useobject{currentmarker}{}%
\end{pgfscope}%
\begin{pgfscope}%
\pgfsys@transformshift{1.333857in}{0.463404in}%
\pgfsys@useobject{currentmarker}{}%
\end{pgfscope}%
\begin{pgfscope}%
\pgfsys@transformshift{1.508764in}{0.408655in}%
\pgfsys@useobject{currentmarker}{}%
\end{pgfscope}%
\begin{pgfscope}%
\pgfsys@transformshift{1.693215in}{0.375235in}%
\pgfsys@useobject{currentmarker}{}%
\end{pgfscope}%
\begin{pgfscope}%
\pgfsys@transformshift{1.882500in}{0.364000in}%
\pgfsys@useobject{currentmarker}{}%
\end{pgfscope}%
\begin{pgfscope}%
\pgfsys@transformshift{2.071785in}{0.375235in}%
\pgfsys@useobject{currentmarker}{}%
\end{pgfscope}%
\begin{pgfscope}%
\pgfsys@transformshift{2.256236in}{0.408655in}%
\pgfsys@useobject{currentmarker}{}%
\end{pgfscope}%
\begin{pgfscope}%
\pgfsys@transformshift{2.431143in}{0.463404in}%
\pgfsys@useobject{currentmarker}{}%
\end{pgfscope}%
\begin{pgfscope}%
\pgfsys@transformshift{2.592042in}{0.538087in}%
\pgfsys@useobject{currentmarker}{}%
\end{pgfscope}%
\end{pgfscope}%
\end{pgfpicture}%
\makeatother%
\endgroup%

        \caption{Iteration 3: Find interface}\label{fig:example-iter2-interface}
    \end{subfigure}
    \begin{subfigure}[b]{.32\linewidth}
        %% Creator: Matplotlib, PGF backend
%%
%% To include the figure in your LaTeX document, write
%%   \input{<filename>.pgf}
%%
%% Make sure the required packages are loaded in your preamble
%%   \usepackage{pgf}
%%
%% Also ensure that all the required font packages are loaded; for instance,
%% the lmodern package is sometimes necessary when using math font.
%%   \usepackage{lmodern}
%%
%% Figures using additional raster images can only be included by \input if
%% they are in the same directory as the main LaTeX file. For loading figures
%% from other directories you can use the `import` package
%%   \usepackage{import}
%%
%% and then include the figures with
%%   \import{<path to file>}{<filename>.pgf}
%%
%% Matplotlib used the following preamble
%%   
%%   \usepackage{fontspec}
%%   \setmainfont{DejaVuSans.ttf}[Path=\detokenize{/home/fabio/Internodes-CM/.venv/lib/python3.8/site-packages/matplotlib/mpl-data/fonts/ttf/}]
%%   \setsansfont{DejaVuSans.ttf}[Path=\detokenize{/home/fabio/Internodes-CM/.venv/lib/python3.8/site-packages/matplotlib/mpl-data/fonts/ttf/}]
%%   \setmonofont{DejaVuSansMono.ttf}[Path=\detokenize{/home/fabio/Internodes-CM/.venv/lib/python3.8/site-packages/matplotlib/mpl-data/fonts/ttf/}]
%%   \makeatletter\@ifpackageloaded{underscore}{}{\usepackage[strings]{underscore}}\makeatother
%%
\begingroup%
\makeatletter%
\begin{pgfpicture}%
\pgfpathrectangle{\pgfpointorigin}{\pgfqpoint{1.982500in}{1.432000in}}%
\pgfusepath{use as bounding box, clip}%
\begin{pgfscope}%
\pgfsetbuttcap%
\pgfsetmiterjoin%
\definecolor{currentfill}{rgb}{1.000000,1.000000,1.000000}%
\pgfsetfillcolor{currentfill}%
\pgfsetlinewidth{0.000000pt}%
\definecolor{currentstroke}{rgb}{1.000000,1.000000,1.000000}%
\pgfsetstrokecolor{currentstroke}%
\pgfsetdash{}{0pt}%
\pgfpathmoveto{\pgfqpoint{0.000000in}{0.000000in}}%
\pgfpathlineto{\pgfqpoint{1.982500in}{0.000000in}}%
\pgfpathlineto{\pgfqpoint{1.982500in}{1.432000in}}%
\pgfpathlineto{\pgfqpoint{0.000000in}{1.432000in}}%
\pgfpathlineto{\pgfqpoint{0.000000in}{0.000000in}}%
\pgfpathclose%
\pgfusepath{fill}%
\end{pgfscope}%
\begin{pgfscope}%
\pgfpathrectangle{\pgfqpoint{0.100000in}{0.100000in}}{\pgfqpoint{1.782500in}{1.232000in}}%
\pgfusepath{clip}%
\pgfsetrectcap%
\pgfsetroundjoin%
\pgfsetlinewidth{0.250937pt}%
\definecolor{currentstroke}{rgb}{0.054902,0.262745,0.486275}%
\pgfsetstrokecolor{currentstroke}%
\pgfsetdash{}{0pt}%
\pgfpathmoveto{\pgfqpoint{0.454502in}{0.100000in}}%
\pgfpathlineto{\pgfqpoint{0.185788in}{0.100000in}}%
\pgfpathmoveto{\pgfqpoint{0.723215in}{0.100000in}}%
\pgfpathlineto{\pgfqpoint{0.454502in}{0.100000in}}%
\pgfpathmoveto{\pgfqpoint{0.991929in}{0.100000in}}%
\pgfpathlineto{\pgfqpoint{0.723215in}{0.100000in}}%
\pgfpathmoveto{\pgfqpoint{1.260642in}{0.100000in}}%
\pgfpathlineto{\pgfqpoint{0.991929in}{0.100000in}}%
\pgfpathmoveto{\pgfqpoint{1.529356in}{0.100000in}}%
\pgfpathlineto{\pgfqpoint{1.798069in}{0.100000in}}%
\pgfpathmoveto{\pgfqpoint{1.529356in}{0.100000in}}%
\pgfpathlineto{\pgfqpoint{1.260642in}{0.100000in}}%
\pgfpathmoveto{\pgfqpoint{1.801477in}{0.408760in}}%
\pgfpathlineto{\pgfqpoint{1.798069in}{0.100000in}}%
\pgfpathmoveto{\pgfqpoint{1.801477in}{0.408760in}}%
\pgfpathlineto{\pgfqpoint{1.787931in}{0.719684in}}%
\pgfpathmoveto{\pgfqpoint{1.625404in}{0.710683in}}%
\pgfpathlineto{\pgfqpoint{1.787931in}{0.719684in}}%
\pgfpathmoveto{\pgfqpoint{1.625404in}{0.710683in}}%
\pgfpathlineto{\pgfqpoint{1.458297in}{0.696339in}}%
\pgfpathmoveto{\pgfqpoint{1.385143in}{0.684752in}}%
\pgfpathlineto{\pgfqpoint{1.458297in}{0.696339in}}%
\pgfpathmoveto{\pgfqpoint{1.310324in}{0.660296in}}%
\pgfpathlineto{\pgfqpoint{1.385143in}{0.684752in}}%
\pgfpathmoveto{\pgfqpoint{1.253051in}{0.642085in}}%
\pgfpathlineto{\pgfqpoint{1.310324in}{0.660296in}}%
\pgfpathmoveto{\pgfqpoint{1.188115in}{0.625601in}}%
\pgfpathlineto{\pgfqpoint{1.253051in}{0.642085in}}%
\pgfpathmoveto{\pgfqpoint{1.127651in}{0.615595in}}%
\pgfpathlineto{\pgfqpoint{1.188115in}{0.625601in}}%
\pgfpathmoveto{\pgfqpoint{1.058490in}{0.606300in}}%
\pgfpathlineto{\pgfqpoint{1.127651in}{0.615595in}}%
\pgfpathmoveto{\pgfqpoint{0.990122in}{0.598971in}}%
\pgfpathlineto{\pgfqpoint{1.058490in}{0.606300in}}%
\pgfpathmoveto{\pgfqpoint{0.921265in}{0.600292in}}%
\pgfpathlineto{\pgfqpoint{0.990122in}{0.598971in}}%
\pgfpathmoveto{\pgfqpoint{0.853138in}{0.610721in}}%
\pgfpathlineto{\pgfqpoint{0.921265in}{0.600292in}}%
\pgfpathmoveto{\pgfqpoint{0.793331in}{0.622693in}}%
\pgfpathlineto{\pgfqpoint{0.853138in}{0.610721in}}%
\pgfpathmoveto{\pgfqpoint{0.728172in}{0.639388in}}%
\pgfpathlineto{\pgfqpoint{0.793331in}{0.622693in}}%
\pgfpathmoveto{\pgfqpoint{0.670517in}{0.657059in}}%
\pgfpathlineto{\pgfqpoint{0.728172in}{0.639388in}}%
\pgfpathmoveto{\pgfqpoint{0.597734in}{0.682628in}}%
\pgfpathlineto{\pgfqpoint{0.523844in}{0.695653in}}%
\pgfpathmoveto{\pgfqpoint{0.597734in}{0.682628in}}%
\pgfpathlineto{\pgfqpoint{0.670517in}{0.657059in}}%
\pgfpathmoveto{\pgfqpoint{0.357342in}{0.710776in}}%
\pgfpathlineto{\pgfqpoint{0.523844in}{0.695653in}}%
\pgfpathmoveto{\pgfqpoint{0.357342in}{0.710776in}}%
\pgfpathlineto{\pgfqpoint{0.194832in}{0.720303in}}%
\pgfpathmoveto{\pgfqpoint{0.181023in}{0.408839in}}%
\pgfpathlineto{\pgfqpoint{0.185788in}{0.100000in}}%
\pgfpathmoveto{\pgfqpoint{0.181023in}{0.408839in}}%
\pgfpathlineto{\pgfqpoint{0.194832in}{0.720303in}}%
\pgfpathmoveto{\pgfqpoint{1.380079in}{0.372824in}}%
\pgfpathlineto{\pgfqpoint{1.260642in}{0.100000in}}%
\pgfpathmoveto{\pgfqpoint{1.085607in}{0.353637in}}%
\pgfpathlineto{\pgfqpoint{0.991929in}{0.100000in}}%
\pgfpathmoveto{\pgfqpoint{1.236571in}{0.465285in}}%
\pgfpathlineto{\pgfqpoint{1.380079in}{0.372824in}}%
\pgfpathmoveto{\pgfqpoint{1.236571in}{0.465285in}}%
\pgfpathlineto{\pgfqpoint{1.085607in}{0.353637in}}%
\pgfpathmoveto{\pgfqpoint{0.955470in}{0.447258in}}%
\pgfpathlineto{\pgfqpoint{1.085607in}{0.353637in}}%
\pgfpathmoveto{\pgfqpoint{0.955470in}{0.447258in}}%
\pgfpathlineto{\pgfqpoint{0.816937in}{0.367970in}}%
\pgfpathmoveto{\pgfqpoint{0.675550in}{0.476601in}}%
\pgfpathlineto{\pgfqpoint{0.816937in}{0.367970in}}%
\pgfpathmoveto{\pgfqpoint{0.675550in}{0.476601in}}%
\pgfpathlineto{\pgfqpoint{0.534719in}{0.383741in}}%
\pgfpathmoveto{\pgfqpoint{1.535916in}{0.486155in}}%
\pgfpathlineto{\pgfqpoint{1.801477in}{0.408760in}}%
\pgfpathmoveto{\pgfqpoint{1.535916in}{0.486155in}}%
\pgfpathlineto{\pgfqpoint{1.380079in}{0.372824in}}%
\pgfpathmoveto{\pgfqpoint{1.376985in}{0.517176in}}%
\pgfpathlineto{\pgfqpoint{1.380079in}{0.372824in}}%
\pgfpathmoveto{\pgfqpoint{1.376985in}{0.517176in}}%
\pgfpathlineto{\pgfqpoint{1.236571in}{0.465285in}}%
\pgfpathmoveto{\pgfqpoint{1.376985in}{0.517176in}}%
\pgfpathlineto{\pgfqpoint{1.535916in}{0.486155in}}%
\pgfpathmoveto{\pgfqpoint{1.093931in}{0.479746in}}%
\pgfpathlineto{\pgfqpoint{1.085607in}{0.353637in}}%
\pgfpathmoveto{\pgfqpoint{1.093931in}{0.479746in}}%
\pgfpathlineto{\pgfqpoint{1.236571in}{0.465285in}}%
\pgfpathmoveto{\pgfqpoint{1.093931in}{0.479746in}}%
\pgfpathlineto{\pgfqpoint{0.955470in}{0.447258in}}%
\pgfpathmoveto{\pgfqpoint{0.816759in}{0.485778in}}%
\pgfpathlineto{\pgfqpoint{0.816937in}{0.367970in}}%
\pgfpathmoveto{\pgfqpoint{0.816759in}{0.485778in}}%
\pgfpathlineto{\pgfqpoint{0.955470in}{0.447258in}}%
\pgfpathmoveto{\pgfqpoint{0.816759in}{0.485778in}}%
\pgfpathlineto{\pgfqpoint{0.675550in}{0.476601in}}%
\pgfpathmoveto{\pgfqpoint{0.526682in}{0.528714in}}%
\pgfpathlineto{\pgfqpoint{0.534719in}{0.383741in}}%
\pgfpathmoveto{\pgfqpoint{0.526682in}{0.528714in}}%
\pgfpathlineto{\pgfqpoint{0.675550in}{0.476601in}}%
\pgfpathmoveto{\pgfqpoint{0.357678in}{0.492105in}}%
\pgfpathlineto{\pgfqpoint{0.194832in}{0.720303in}}%
\pgfpathmoveto{\pgfqpoint{0.357678in}{0.492105in}}%
\pgfpathlineto{\pgfqpoint{0.357342in}{0.710776in}}%
\pgfpathmoveto{\pgfqpoint{0.357678in}{0.492105in}}%
\pgfpathlineto{\pgfqpoint{0.181023in}{0.408839in}}%
\pgfpathmoveto{\pgfqpoint{0.357678in}{0.492105in}}%
\pgfpathlineto{\pgfqpoint{0.534719in}{0.383741in}}%
\pgfpathmoveto{\pgfqpoint{0.357678in}{0.492105in}}%
\pgfpathlineto{\pgfqpoint{0.526682in}{0.528714in}}%
\pgfpathmoveto{\pgfqpoint{1.298659in}{0.566272in}}%
\pgfpathlineto{\pgfqpoint{1.310324in}{0.660296in}}%
\pgfpathmoveto{\pgfqpoint{1.298659in}{0.566272in}}%
\pgfpathlineto{\pgfqpoint{1.253051in}{0.642085in}}%
\pgfpathmoveto{\pgfqpoint{1.298659in}{0.566272in}}%
\pgfpathlineto{\pgfqpoint{1.236571in}{0.465285in}}%
\pgfpathmoveto{\pgfqpoint{1.298659in}{0.566272in}}%
\pgfpathlineto{\pgfqpoint{1.376985in}{0.517176in}}%
\pgfpathmoveto{\pgfqpoint{1.163096in}{0.542097in}}%
\pgfpathlineto{\pgfqpoint{1.188115in}{0.625601in}}%
\pgfpathmoveto{\pgfqpoint{1.163096in}{0.542097in}}%
\pgfpathlineto{\pgfqpoint{1.127651in}{0.615595in}}%
\pgfpathmoveto{\pgfqpoint{1.163096in}{0.542097in}}%
\pgfpathlineto{\pgfqpoint{1.236571in}{0.465285in}}%
\pgfpathmoveto{\pgfqpoint{1.163096in}{0.542097in}}%
\pgfpathlineto{\pgfqpoint{1.093931in}{0.479746in}}%
\pgfpathmoveto{\pgfqpoint{1.026137in}{0.529850in}}%
\pgfpathlineto{\pgfqpoint{1.058490in}{0.606300in}}%
\pgfpathmoveto{\pgfqpoint{1.026137in}{0.529850in}}%
\pgfpathlineto{\pgfqpoint{0.990122in}{0.598971in}}%
\pgfpathmoveto{\pgfqpoint{1.026137in}{0.529850in}}%
\pgfpathlineto{\pgfqpoint{0.955470in}{0.447258in}}%
\pgfpathmoveto{\pgfqpoint{1.026137in}{0.529850in}}%
\pgfpathlineto{\pgfqpoint{1.093931in}{0.479746in}}%
\pgfpathmoveto{\pgfqpoint{0.886661in}{0.531913in}}%
\pgfpathlineto{\pgfqpoint{0.921265in}{0.600292in}}%
\pgfpathmoveto{\pgfqpoint{0.886661in}{0.531913in}}%
\pgfpathlineto{\pgfqpoint{0.853138in}{0.610721in}}%
\pgfpathmoveto{\pgfqpoint{0.886661in}{0.531913in}}%
\pgfpathlineto{\pgfqpoint{0.955470in}{0.447258in}}%
\pgfpathmoveto{\pgfqpoint{0.886661in}{0.531913in}}%
\pgfpathlineto{\pgfqpoint{0.816759in}{0.485778in}}%
\pgfpathmoveto{\pgfqpoint{0.750901in}{0.550725in}}%
\pgfpathlineto{\pgfqpoint{0.793331in}{0.622693in}}%
\pgfpathmoveto{\pgfqpoint{0.750901in}{0.550725in}}%
\pgfpathlineto{\pgfqpoint{0.728172in}{0.639388in}}%
\pgfpathmoveto{\pgfqpoint{0.750901in}{0.550725in}}%
\pgfpathlineto{\pgfqpoint{0.675550in}{0.476601in}}%
\pgfpathmoveto{\pgfqpoint{0.750901in}{0.550725in}}%
\pgfpathlineto{\pgfqpoint{0.816759in}{0.485778in}}%
\pgfpathmoveto{\pgfqpoint{0.614919in}{0.583110in}}%
\pgfpathlineto{\pgfqpoint{0.670517in}{0.657059in}}%
\pgfpathmoveto{\pgfqpoint{0.614919in}{0.583110in}}%
\pgfpathlineto{\pgfqpoint{0.597734in}{0.682628in}}%
\pgfpathmoveto{\pgfqpoint{0.614919in}{0.583110in}}%
\pgfpathlineto{\pgfqpoint{0.675550in}{0.476601in}}%
\pgfpathmoveto{\pgfqpoint{0.614919in}{0.583110in}}%
\pgfpathlineto{\pgfqpoint{0.526682in}{0.528714in}}%
\pgfpathmoveto{\pgfqpoint{1.445323in}{0.602316in}}%
\pgfpathlineto{\pgfqpoint{1.458297in}{0.696339in}}%
\pgfpathmoveto{\pgfqpoint{1.445323in}{0.602316in}}%
\pgfpathlineto{\pgfqpoint{1.385143in}{0.684752in}}%
\pgfpathmoveto{\pgfqpoint{1.445323in}{0.602316in}}%
\pgfpathlineto{\pgfqpoint{1.535916in}{0.486155in}}%
\pgfpathmoveto{\pgfqpoint{1.445323in}{0.602316in}}%
\pgfpathlineto{\pgfqpoint{1.376985in}{0.517176in}}%
\pgfpathmoveto{\pgfqpoint{0.938998in}{0.293567in}}%
\pgfpathlineto{\pgfqpoint{0.991929in}{0.100000in}}%
\pgfpathmoveto{\pgfqpoint{0.938998in}{0.293567in}}%
\pgfpathlineto{\pgfqpoint{1.085607in}{0.353637in}}%
\pgfpathmoveto{\pgfqpoint{0.938998in}{0.293567in}}%
\pgfpathlineto{\pgfqpoint{0.816937in}{0.367970in}}%
\pgfpathmoveto{\pgfqpoint{0.938998in}{0.293567in}}%
\pgfpathlineto{\pgfqpoint{0.955470in}{0.447258in}}%
\pgfpathmoveto{\pgfqpoint{0.441612in}{0.602117in}}%
\pgfpathlineto{\pgfqpoint{0.523844in}{0.695653in}}%
\pgfpathmoveto{\pgfqpoint{0.441612in}{0.602117in}}%
\pgfpathlineto{\pgfqpoint{0.357342in}{0.710776in}}%
\pgfpathmoveto{\pgfqpoint{0.441612in}{0.602117in}}%
\pgfpathlineto{\pgfqpoint{0.526682in}{0.528714in}}%
\pgfpathmoveto{\pgfqpoint{0.441612in}{0.602117in}}%
\pgfpathlineto{\pgfqpoint{0.357678in}{0.492105in}}%
\pgfpathmoveto{\pgfqpoint{1.364219in}{0.606819in}}%
\pgfpathlineto{\pgfqpoint{1.385143in}{0.684752in}}%
\pgfpathmoveto{\pgfqpoint{1.364219in}{0.606819in}}%
\pgfpathlineto{\pgfqpoint{1.310324in}{0.660296in}}%
\pgfpathmoveto{\pgfqpoint{1.364219in}{0.606819in}}%
\pgfpathlineto{\pgfqpoint{1.376985in}{0.517176in}}%
\pgfpathmoveto{\pgfqpoint{1.364219in}{0.606819in}}%
\pgfpathlineto{\pgfqpoint{1.298659in}{0.566272in}}%
\pgfpathmoveto{\pgfqpoint{1.364219in}{0.606819in}}%
\pgfpathlineto{\pgfqpoint{1.445323in}{0.602316in}}%
\pgfpathmoveto{\pgfqpoint{1.094158in}{0.554638in}}%
\pgfpathlineto{\pgfqpoint{1.127651in}{0.615595in}}%
\pgfpathmoveto{\pgfqpoint{1.094158in}{0.554638in}}%
\pgfpathlineto{\pgfqpoint{1.058490in}{0.606300in}}%
\pgfpathmoveto{\pgfqpoint{1.094158in}{0.554638in}}%
\pgfpathlineto{\pgfqpoint{1.093931in}{0.479746in}}%
\pgfpathmoveto{\pgfqpoint{1.094158in}{0.554638in}}%
\pgfpathlineto{\pgfqpoint{1.163096in}{0.542097in}}%
\pgfpathmoveto{\pgfqpoint{1.094158in}{0.554638in}}%
\pgfpathlineto{\pgfqpoint{1.026137in}{0.529850in}}%
\pgfpathmoveto{\pgfqpoint{1.228887in}{0.571665in}}%
\pgfpathlineto{\pgfqpoint{1.253051in}{0.642085in}}%
\pgfpathmoveto{\pgfqpoint{1.228887in}{0.571665in}}%
\pgfpathlineto{\pgfqpoint{1.188115in}{0.625601in}}%
\pgfpathmoveto{\pgfqpoint{1.228887in}{0.571665in}}%
\pgfpathlineto{\pgfqpoint{1.236571in}{0.465285in}}%
\pgfpathmoveto{\pgfqpoint{1.228887in}{0.571665in}}%
\pgfpathlineto{\pgfqpoint{1.298659in}{0.566272in}}%
\pgfpathmoveto{\pgfqpoint{1.228887in}{0.571665in}}%
\pgfpathlineto{\pgfqpoint{1.163096in}{0.542097in}}%
\pgfpathmoveto{\pgfqpoint{0.956125in}{0.541001in}}%
\pgfpathlineto{\pgfqpoint{0.990122in}{0.598971in}}%
\pgfpathmoveto{\pgfqpoint{0.956125in}{0.541001in}}%
\pgfpathlineto{\pgfqpoint{0.921265in}{0.600292in}}%
\pgfpathmoveto{\pgfqpoint{0.956125in}{0.541001in}}%
\pgfpathlineto{\pgfqpoint{0.955470in}{0.447258in}}%
\pgfpathmoveto{\pgfqpoint{0.956125in}{0.541001in}}%
\pgfpathlineto{\pgfqpoint{1.026137in}{0.529850in}}%
\pgfpathmoveto{\pgfqpoint{0.956125in}{0.541001in}}%
\pgfpathlineto{\pgfqpoint{0.886661in}{0.531913in}}%
\pgfpathmoveto{\pgfqpoint{0.819642in}{0.557313in}}%
\pgfpathlineto{\pgfqpoint{0.853138in}{0.610721in}}%
\pgfpathmoveto{\pgfqpoint{0.819642in}{0.557313in}}%
\pgfpathlineto{\pgfqpoint{0.793331in}{0.622693in}}%
\pgfpathmoveto{\pgfqpoint{0.819642in}{0.557313in}}%
\pgfpathlineto{\pgfqpoint{0.816759in}{0.485778in}}%
\pgfpathmoveto{\pgfqpoint{0.819642in}{0.557313in}}%
\pgfpathlineto{\pgfqpoint{0.886661in}{0.531913in}}%
\pgfpathmoveto{\pgfqpoint{0.819642in}{0.557313in}}%
\pgfpathlineto{\pgfqpoint{0.750901in}{0.550725in}}%
\pgfpathmoveto{\pgfqpoint{1.193096in}{0.276126in}}%
\pgfpathlineto{\pgfqpoint{0.991929in}{0.100000in}}%
\pgfpathmoveto{\pgfqpoint{1.193096in}{0.276126in}}%
\pgfpathlineto{\pgfqpoint{1.260642in}{0.100000in}}%
\pgfpathmoveto{\pgfqpoint{1.193096in}{0.276126in}}%
\pgfpathlineto{\pgfqpoint{1.380079in}{0.372824in}}%
\pgfpathmoveto{\pgfqpoint{1.193096in}{0.276126in}}%
\pgfpathlineto{\pgfqpoint{1.085607in}{0.353637in}}%
\pgfpathmoveto{\pgfqpoint{1.193096in}{0.276126in}}%
\pgfpathlineto{\pgfqpoint{1.236571in}{0.465285in}}%
\pgfpathmoveto{\pgfqpoint{0.690674in}{0.289421in}}%
\pgfpathlineto{\pgfqpoint{0.723215in}{0.100000in}}%
\pgfpathmoveto{\pgfqpoint{0.690674in}{0.289421in}}%
\pgfpathlineto{\pgfqpoint{0.816937in}{0.367970in}}%
\pgfpathmoveto{\pgfqpoint{0.690674in}{0.289421in}}%
\pgfpathlineto{\pgfqpoint{0.534719in}{0.383741in}}%
\pgfpathmoveto{\pgfqpoint{0.690674in}{0.289421in}}%
\pgfpathlineto{\pgfqpoint{0.675550in}{0.476601in}}%
\pgfpathmoveto{\pgfqpoint{0.686113in}{0.579464in}}%
\pgfpathlineto{\pgfqpoint{0.728172in}{0.639388in}}%
\pgfpathmoveto{\pgfqpoint{0.686113in}{0.579464in}}%
\pgfpathlineto{\pgfqpoint{0.670517in}{0.657059in}}%
\pgfpathmoveto{\pgfqpoint{0.686113in}{0.579464in}}%
\pgfpathlineto{\pgfqpoint{0.675550in}{0.476601in}}%
\pgfpathmoveto{\pgfqpoint{0.686113in}{0.579464in}}%
\pgfpathlineto{\pgfqpoint{0.750901in}{0.550725in}}%
\pgfpathmoveto{\pgfqpoint{0.686113in}{0.579464in}}%
\pgfpathlineto{\pgfqpoint{0.614919in}{0.583110in}}%
\pgfpathmoveto{\pgfqpoint{1.547819in}{0.613854in}}%
\pgfpathlineto{\pgfqpoint{1.458297in}{0.696339in}}%
\pgfpathmoveto{\pgfqpoint{1.547819in}{0.613854in}}%
\pgfpathlineto{\pgfqpoint{1.625404in}{0.710683in}}%
\pgfpathmoveto{\pgfqpoint{1.547819in}{0.613854in}}%
\pgfpathlineto{\pgfqpoint{1.535916in}{0.486155in}}%
\pgfpathmoveto{\pgfqpoint{1.547819in}{0.613854in}}%
\pgfpathlineto{\pgfqpoint{1.445323in}{0.602316in}}%
\pgfpathmoveto{\pgfqpoint{0.550365in}{0.619266in}}%
\pgfpathlineto{\pgfqpoint{0.523844in}{0.695653in}}%
\pgfpathmoveto{\pgfqpoint{0.550365in}{0.619266in}}%
\pgfpathlineto{\pgfqpoint{0.597734in}{0.682628in}}%
\pgfpathmoveto{\pgfqpoint{0.550365in}{0.619266in}}%
\pgfpathlineto{\pgfqpoint{0.526682in}{0.528714in}}%
\pgfpathmoveto{\pgfqpoint{0.550365in}{0.619266in}}%
\pgfpathlineto{\pgfqpoint{0.614919in}{0.583110in}}%
\pgfpathmoveto{\pgfqpoint{0.550365in}{0.619266in}}%
\pgfpathlineto{\pgfqpoint{0.441612in}{0.602117in}}%
\pgfpathmoveto{\pgfqpoint{0.377336in}{0.284113in}}%
\pgfpathlineto{\pgfqpoint{0.185788in}{0.100000in}}%
\pgfpathmoveto{\pgfqpoint{0.377336in}{0.284113in}}%
\pgfpathlineto{\pgfqpoint{0.454502in}{0.100000in}}%
\pgfpathmoveto{\pgfqpoint{0.377336in}{0.284113in}}%
\pgfpathlineto{\pgfqpoint{0.181023in}{0.408839in}}%
\pgfpathmoveto{\pgfqpoint{0.377336in}{0.284113in}}%
\pgfpathlineto{\pgfqpoint{0.534719in}{0.383741in}}%
\pgfpathmoveto{\pgfqpoint{0.377336in}{0.284113in}}%
\pgfpathlineto{\pgfqpoint{0.357678in}{0.492105in}}%
\pgfpathmoveto{\pgfqpoint{1.533730in}{0.293058in}}%
\pgfpathlineto{\pgfqpoint{1.798069in}{0.100000in}}%
\pgfpathmoveto{\pgfqpoint{1.533730in}{0.293058in}}%
\pgfpathlineto{\pgfqpoint{1.260642in}{0.100000in}}%
\pgfpathmoveto{\pgfqpoint{1.533730in}{0.293058in}}%
\pgfpathlineto{\pgfqpoint{1.529356in}{0.100000in}}%
\pgfpathmoveto{\pgfqpoint{1.533730in}{0.293058in}}%
\pgfpathlineto{\pgfqpoint{1.801477in}{0.408760in}}%
\pgfpathmoveto{\pgfqpoint{1.533730in}{0.293058in}}%
\pgfpathlineto{\pgfqpoint{1.380079in}{0.372824in}}%
\pgfpathmoveto{\pgfqpoint{1.533730in}{0.293058in}}%
\pgfpathlineto{\pgfqpoint{1.535916in}{0.486155in}}%
\pgfpathmoveto{\pgfqpoint{1.660037in}{0.588517in}}%
\pgfpathlineto{\pgfqpoint{1.787931in}{0.719684in}}%
\pgfpathmoveto{\pgfqpoint{1.660037in}{0.588517in}}%
\pgfpathlineto{\pgfqpoint{1.801477in}{0.408760in}}%
\pgfpathmoveto{\pgfqpoint{1.660037in}{0.588517in}}%
\pgfpathlineto{\pgfqpoint{1.625404in}{0.710683in}}%
\pgfpathmoveto{\pgfqpoint{1.660037in}{0.588517in}}%
\pgfpathlineto{\pgfqpoint{1.535916in}{0.486155in}}%
\pgfpathmoveto{\pgfqpoint{1.660037in}{0.588517in}}%
\pgfpathlineto{\pgfqpoint{1.547819in}{0.613854in}}%
\pgfpathmoveto{\pgfqpoint{0.840010in}{0.211728in}}%
\pgfpathlineto{\pgfqpoint{0.723215in}{0.100000in}}%
\pgfpathmoveto{\pgfqpoint{0.840010in}{0.211728in}}%
\pgfpathlineto{\pgfqpoint{0.991929in}{0.100000in}}%
\pgfpathmoveto{\pgfqpoint{0.840010in}{0.211728in}}%
\pgfpathlineto{\pgfqpoint{0.816937in}{0.367970in}}%
\pgfpathmoveto{\pgfqpoint{0.840010in}{0.211728in}}%
\pgfpathlineto{\pgfqpoint{0.938998in}{0.293567in}}%
\pgfpathmoveto{\pgfqpoint{0.840010in}{0.211728in}}%
\pgfpathlineto{\pgfqpoint{0.690674in}{0.289421in}}%
\pgfpathmoveto{\pgfqpoint{0.553809in}{0.230866in}}%
\pgfpathlineto{\pgfqpoint{0.454502in}{0.100000in}}%
\pgfpathmoveto{\pgfqpoint{0.553809in}{0.230866in}}%
\pgfpathlineto{\pgfqpoint{0.723215in}{0.100000in}}%
\pgfpathmoveto{\pgfqpoint{0.553809in}{0.230866in}}%
\pgfpathlineto{\pgfqpoint{0.534719in}{0.383741in}}%
\pgfpathmoveto{\pgfqpoint{0.553809in}{0.230866in}}%
\pgfpathlineto{\pgfqpoint{0.690674in}{0.289421in}}%
\pgfpathmoveto{\pgfqpoint{0.553809in}{0.230866in}}%
\pgfpathlineto{\pgfqpoint{0.377336in}{0.284113in}}%
\pgfpathlineto{\pgfqpoint{0.377336in}{0.284113in}}%
\pgfusepath{stroke}%
\end{pgfscope}%
\begin{pgfscope}%
\pgfpathrectangle{\pgfqpoint{0.100000in}{0.100000in}}{\pgfqpoint{1.782500in}{1.232000in}}%
\pgfusepath{clip}%
\pgfsetrectcap%
\pgfsetroundjoin%
\pgfsetlinewidth{0.250937pt}%
\definecolor{currentstroke}{rgb}{0.835294,0.321569,0.035294}%
\pgfsetstrokecolor{currentstroke}%
\pgfsetdash{}{0pt}%
\pgfpathmoveto{\pgfqpoint{0.493932in}{0.837819in}}%
\pgfpathlineto{\pgfqpoint{0.454502in}{1.085600in}}%
\pgfpathmoveto{\pgfqpoint{1.529356in}{1.085600in}}%
\pgfpathlineto{\pgfqpoint{1.489237in}{0.837484in}}%
\pgfpathmoveto{\pgfqpoint{0.723215in}{1.085600in}}%
\pgfpathlineto{\pgfqpoint{0.991929in}{1.085600in}}%
\pgfpathmoveto{\pgfqpoint{0.723215in}{1.085600in}}%
\pgfpathlineto{\pgfqpoint{0.454502in}{1.085600in}}%
\pgfpathmoveto{\pgfqpoint{0.539694in}{0.759689in}}%
\pgfpathlineto{\pgfqpoint{0.493932in}{0.837819in}}%
\pgfpathmoveto{\pgfqpoint{0.597797in}{0.697728in}}%
\pgfpathlineto{\pgfqpoint{0.539694in}{0.759689in}}%
\pgfpathmoveto{\pgfqpoint{0.666513in}{0.658314in}}%
\pgfpathlineto{\pgfqpoint{0.597797in}{0.697728in}}%
\pgfpathmoveto{\pgfqpoint{0.752247in}{0.632595in}}%
\pgfpathlineto{\pgfqpoint{0.666513in}{0.658314in}}%
\pgfpathmoveto{\pgfqpoint{0.833592in}{0.614513in}}%
\pgfpathlineto{\pgfqpoint{0.752247in}{0.632595in}}%
\pgfpathmoveto{\pgfqpoint{0.912544in}{0.600941in}}%
\pgfpathlineto{\pgfqpoint{0.833592in}{0.614513in}}%
\pgfpathmoveto{\pgfqpoint{0.990153in}{0.598729in}}%
\pgfpathlineto{\pgfqpoint{0.912544in}{0.600941in}}%
\pgfpathmoveto{\pgfqpoint{1.067159in}{0.607466in}}%
\pgfpathlineto{\pgfqpoint{0.990153in}{0.598729in}}%
\pgfpathmoveto{\pgfqpoint{1.147591in}{0.618472in}}%
\pgfpathlineto{\pgfqpoint{1.067159in}{0.607466in}}%
\pgfpathmoveto{\pgfqpoint{1.229103in}{0.635206in}}%
\pgfpathlineto{\pgfqpoint{1.147591in}{0.618472in}}%
\pgfpathmoveto{\pgfqpoint{1.314300in}{0.661595in}}%
\pgfpathlineto{\pgfqpoint{1.229103in}{0.635206in}}%
\pgfpathmoveto{\pgfqpoint{1.383298in}{0.696947in}}%
\pgfpathlineto{\pgfqpoint{1.314300in}{0.661595in}}%
\pgfpathmoveto{\pgfqpoint{1.443404in}{0.759511in}}%
\pgfpathlineto{\pgfqpoint{1.489237in}{0.837484in}}%
\pgfpathmoveto{\pgfqpoint{1.443404in}{0.759511in}}%
\pgfpathlineto{\pgfqpoint{1.383298in}{0.696947in}}%
\pgfpathmoveto{\pgfqpoint{1.260642in}{1.085600in}}%
\pgfpathlineto{\pgfqpoint{0.991929in}{1.085600in}}%
\pgfpathmoveto{\pgfqpoint{1.260642in}{1.085600in}}%
\pgfpathlineto{\pgfqpoint{1.529356in}{1.085600in}}%
\pgfpathmoveto{\pgfqpoint{0.831595in}{0.886984in}}%
\pgfpathlineto{\pgfqpoint{0.991929in}{1.085600in}}%
\pgfpathmoveto{\pgfqpoint{0.831595in}{0.886984in}}%
\pgfpathlineto{\pgfqpoint{0.723215in}{1.085600in}}%
\pgfpathmoveto{\pgfqpoint{0.960383in}{0.754944in}}%
\pgfpathlineto{\pgfqpoint{1.116418in}{0.824542in}}%
\pgfpathmoveto{\pgfqpoint{0.960383in}{0.754944in}}%
\pgfpathlineto{\pgfqpoint{0.831595in}{0.886984in}}%
\pgfpathmoveto{\pgfqpoint{1.283109in}{0.858031in}}%
\pgfpathlineto{\pgfqpoint{1.116418in}{0.824542in}}%
\pgfpathmoveto{\pgfqpoint{0.666415in}{0.893785in}}%
\pgfpathlineto{\pgfqpoint{0.723215in}{1.085600in}}%
\pgfpathmoveto{\pgfqpoint{0.666415in}{0.893785in}}%
\pgfpathlineto{\pgfqpoint{0.831595in}{0.886984in}}%
\pgfpathmoveto{\pgfqpoint{1.091714in}{0.718622in}}%
\pgfpathlineto{\pgfqpoint{1.067159in}{0.607466in}}%
\pgfpathmoveto{\pgfqpoint{1.091714in}{0.718622in}}%
\pgfpathlineto{\pgfqpoint{1.147591in}{0.618472in}}%
\pgfpathmoveto{\pgfqpoint{1.091714in}{0.718622in}}%
\pgfpathlineto{\pgfqpoint{1.116418in}{0.824542in}}%
\pgfpathmoveto{\pgfqpoint{1.091714in}{0.718622in}}%
\pgfpathlineto{\pgfqpoint{0.960383in}{0.754944in}}%
\pgfpathmoveto{\pgfqpoint{1.216169in}{0.750966in}}%
\pgfpathlineto{\pgfqpoint{1.229103in}{0.635206in}}%
\pgfpathmoveto{\pgfqpoint{1.216169in}{0.750966in}}%
\pgfpathlineto{\pgfqpoint{1.314300in}{0.661595in}}%
\pgfpathmoveto{\pgfqpoint{1.216169in}{0.750966in}}%
\pgfpathlineto{\pgfqpoint{1.116418in}{0.824542in}}%
\pgfpathmoveto{\pgfqpoint{1.216169in}{0.750966in}}%
\pgfpathlineto{\pgfqpoint{1.283109in}{0.858031in}}%
\pgfpathmoveto{\pgfqpoint{1.216169in}{0.750966in}}%
\pgfpathlineto{\pgfqpoint{1.091714in}{0.718622in}}%
\pgfpathmoveto{\pgfqpoint{0.819793in}{0.728687in}}%
\pgfpathlineto{\pgfqpoint{0.752247in}{0.632595in}}%
\pgfpathmoveto{\pgfqpoint{0.819793in}{0.728687in}}%
\pgfpathlineto{\pgfqpoint{0.833592in}{0.614513in}}%
\pgfpathmoveto{\pgfqpoint{0.819793in}{0.728687in}}%
\pgfpathlineto{\pgfqpoint{0.831595in}{0.886984in}}%
\pgfpathmoveto{\pgfqpoint{0.819793in}{0.728687in}}%
\pgfpathlineto{\pgfqpoint{0.960383in}{0.754944in}}%
\pgfpathmoveto{\pgfqpoint{0.708591in}{0.780276in}}%
\pgfpathlineto{\pgfqpoint{0.597797in}{0.697728in}}%
\pgfpathmoveto{\pgfqpoint{0.708591in}{0.780276in}}%
\pgfpathlineto{\pgfqpoint{0.666513in}{0.658314in}}%
\pgfpathmoveto{\pgfqpoint{0.708591in}{0.780276in}}%
\pgfpathlineto{\pgfqpoint{0.831595in}{0.886984in}}%
\pgfpathmoveto{\pgfqpoint{0.708591in}{0.780276in}}%
\pgfpathlineto{\pgfqpoint{0.666415in}{0.893785in}}%
\pgfpathmoveto{\pgfqpoint{0.708591in}{0.780276in}}%
\pgfpathlineto{\pgfqpoint{0.819793in}{0.728687in}}%
\pgfpathmoveto{\pgfqpoint{1.369911in}{0.959364in}}%
\pgfpathlineto{\pgfqpoint{1.489237in}{0.837484in}}%
\pgfpathmoveto{\pgfqpoint{1.369911in}{0.959364in}}%
\pgfpathlineto{\pgfqpoint{1.529356in}{1.085600in}}%
\pgfpathmoveto{\pgfqpoint{1.369911in}{0.959364in}}%
\pgfpathlineto{\pgfqpoint{1.260642in}{1.085600in}}%
\pgfpathmoveto{\pgfqpoint{1.369911in}{0.959364in}}%
\pgfpathlineto{\pgfqpoint{1.283109in}{0.858031in}}%
\pgfpathmoveto{\pgfqpoint{0.953812in}{0.656912in}}%
\pgfpathlineto{\pgfqpoint{0.912544in}{0.600941in}}%
\pgfpathmoveto{\pgfqpoint{0.953812in}{0.656912in}}%
\pgfpathlineto{\pgfqpoint{0.990153in}{0.598729in}}%
\pgfpathmoveto{\pgfqpoint{0.953812in}{0.656912in}}%
\pgfpathlineto{\pgfqpoint{0.960383in}{0.754944in}}%
\pgfpathmoveto{\pgfqpoint{1.303198in}{0.744504in}}%
\pgfpathlineto{\pgfqpoint{1.314300in}{0.661595in}}%
\pgfpathmoveto{\pgfqpoint{1.303198in}{0.744504in}}%
\pgfpathlineto{\pgfqpoint{1.383298in}{0.696947in}}%
\pgfpathmoveto{\pgfqpoint{1.303198in}{0.744504in}}%
\pgfpathlineto{\pgfqpoint{1.283109in}{0.858031in}}%
\pgfpathmoveto{\pgfqpoint{1.303198in}{0.744504in}}%
\pgfpathlineto{\pgfqpoint{1.216169in}{0.750966in}}%
\pgfpathmoveto{\pgfqpoint{1.169146in}{0.688366in}}%
\pgfpathlineto{\pgfqpoint{1.147591in}{0.618472in}}%
\pgfpathmoveto{\pgfqpoint{1.169146in}{0.688366in}}%
\pgfpathlineto{\pgfqpoint{1.229103in}{0.635206in}}%
\pgfpathmoveto{\pgfqpoint{1.169146in}{0.688366in}}%
\pgfpathlineto{\pgfqpoint{1.091714in}{0.718622in}}%
\pgfpathmoveto{\pgfqpoint{1.169146in}{0.688366in}}%
\pgfpathlineto{\pgfqpoint{1.216169in}{0.750966in}}%
\pgfpathmoveto{\pgfqpoint{1.386091in}{0.854492in}}%
\pgfpathlineto{\pgfqpoint{1.489237in}{0.837484in}}%
\pgfpathmoveto{\pgfqpoint{1.386091in}{0.854492in}}%
\pgfpathlineto{\pgfqpoint{1.443404in}{0.759511in}}%
\pgfpathmoveto{\pgfqpoint{1.386091in}{0.854492in}}%
\pgfpathlineto{\pgfqpoint{1.283109in}{0.858031in}}%
\pgfpathmoveto{\pgfqpoint{1.386091in}{0.854492in}}%
\pgfpathlineto{\pgfqpoint{1.369911in}{0.959364in}}%
\pgfpathmoveto{\pgfqpoint{0.739504in}{0.707019in}}%
\pgfpathlineto{\pgfqpoint{0.666513in}{0.658314in}}%
\pgfpathmoveto{\pgfqpoint{0.739504in}{0.707019in}}%
\pgfpathlineto{\pgfqpoint{0.752247in}{0.632595in}}%
\pgfpathmoveto{\pgfqpoint{0.739504in}{0.707019in}}%
\pgfpathlineto{\pgfqpoint{0.819793in}{0.728687in}}%
\pgfpathmoveto{\pgfqpoint{0.739504in}{0.707019in}}%
\pgfpathlineto{\pgfqpoint{0.708591in}{0.780276in}}%
\pgfpathmoveto{\pgfqpoint{0.624324in}{0.786217in}}%
\pgfpathlineto{\pgfqpoint{0.539694in}{0.759689in}}%
\pgfpathmoveto{\pgfqpoint{0.624324in}{0.786217in}}%
\pgfpathlineto{\pgfqpoint{0.597797in}{0.697728in}}%
\pgfpathmoveto{\pgfqpoint{0.624324in}{0.786217in}}%
\pgfpathlineto{\pgfqpoint{0.666415in}{0.893785in}}%
\pgfpathmoveto{\pgfqpoint{0.624324in}{0.786217in}}%
\pgfpathlineto{\pgfqpoint{0.708591in}{0.780276in}}%
\pgfpathmoveto{\pgfqpoint{0.566652in}{0.964750in}}%
\pgfpathlineto{\pgfqpoint{0.454502in}{1.085600in}}%
\pgfpathmoveto{\pgfqpoint{0.566652in}{0.964750in}}%
\pgfpathlineto{\pgfqpoint{0.493932in}{0.837819in}}%
\pgfpathmoveto{\pgfqpoint{0.566652in}{0.964750in}}%
\pgfpathlineto{\pgfqpoint{0.723215in}{1.085600in}}%
\pgfpathmoveto{\pgfqpoint{0.566652in}{0.964750in}}%
\pgfpathlineto{\pgfqpoint{0.666415in}{0.893785in}}%
\pgfpathmoveto{\pgfqpoint{1.025791in}{0.660878in}}%
\pgfpathlineto{\pgfqpoint{0.990153in}{0.598729in}}%
\pgfpathmoveto{\pgfqpoint{1.025791in}{0.660878in}}%
\pgfpathlineto{\pgfqpoint{1.067159in}{0.607466in}}%
\pgfpathmoveto{\pgfqpoint{1.025791in}{0.660878in}}%
\pgfpathlineto{\pgfqpoint{0.960383in}{0.754944in}}%
\pgfpathmoveto{\pgfqpoint{1.025791in}{0.660878in}}%
\pgfpathlineto{\pgfqpoint{1.091714in}{0.718622in}}%
\pgfpathmoveto{\pgfqpoint{1.025791in}{0.660878in}}%
\pgfpathlineto{\pgfqpoint{0.953812in}{0.656912in}}%
\pgfpathmoveto{\pgfqpoint{0.882004in}{0.665729in}}%
\pgfpathlineto{\pgfqpoint{0.833592in}{0.614513in}}%
\pgfpathmoveto{\pgfqpoint{0.882004in}{0.665729in}}%
\pgfpathlineto{\pgfqpoint{0.912544in}{0.600941in}}%
\pgfpathmoveto{\pgfqpoint{0.882004in}{0.665729in}}%
\pgfpathlineto{\pgfqpoint{0.960383in}{0.754944in}}%
\pgfpathmoveto{\pgfqpoint{0.882004in}{0.665729in}}%
\pgfpathlineto{\pgfqpoint{0.819793in}{0.728687in}}%
\pgfpathmoveto{\pgfqpoint{0.882004in}{0.665729in}}%
\pgfpathlineto{\pgfqpoint{0.953812in}{0.656912in}}%
\pgfpathmoveto{\pgfqpoint{1.367508in}{0.777446in}}%
\pgfpathlineto{\pgfqpoint{1.383298in}{0.696947in}}%
\pgfpathmoveto{\pgfqpoint{1.367508in}{0.777446in}}%
\pgfpathlineto{\pgfqpoint{1.443404in}{0.759511in}}%
\pgfpathmoveto{\pgfqpoint{1.367508in}{0.777446in}}%
\pgfpathlineto{\pgfqpoint{1.283109in}{0.858031in}}%
\pgfpathmoveto{\pgfqpoint{1.367508in}{0.777446in}}%
\pgfpathlineto{\pgfqpoint{1.303198in}{0.744504in}}%
\pgfpathmoveto{\pgfqpoint{1.367508in}{0.777446in}}%
\pgfpathlineto{\pgfqpoint{1.386091in}{0.854492in}}%
\pgfpathmoveto{\pgfqpoint{1.015772in}{0.905746in}}%
\pgfpathlineto{\pgfqpoint{0.991929in}{1.085600in}}%
\pgfpathmoveto{\pgfqpoint{1.015772in}{0.905746in}}%
\pgfpathlineto{\pgfqpoint{1.116418in}{0.824542in}}%
\pgfpathmoveto{\pgfqpoint{1.015772in}{0.905746in}}%
\pgfpathlineto{\pgfqpoint{0.831595in}{0.886984in}}%
\pgfpathmoveto{\pgfqpoint{1.015772in}{0.905746in}}%
\pgfpathlineto{\pgfqpoint{0.960383in}{0.754944in}}%
\pgfpathmoveto{\pgfqpoint{1.174985in}{0.953328in}}%
\pgfpathlineto{\pgfqpoint{0.991929in}{1.085600in}}%
\pgfpathmoveto{\pgfqpoint{1.174985in}{0.953328in}}%
\pgfpathlineto{\pgfqpoint{1.260642in}{1.085600in}}%
\pgfpathmoveto{\pgfqpoint{1.174985in}{0.953328in}}%
\pgfpathlineto{\pgfqpoint{1.116418in}{0.824542in}}%
\pgfpathmoveto{\pgfqpoint{1.174985in}{0.953328in}}%
\pgfpathlineto{\pgfqpoint{1.283109in}{0.858031in}}%
\pgfpathmoveto{\pgfqpoint{1.174985in}{0.953328in}}%
\pgfpathlineto{\pgfqpoint{1.369911in}{0.959364in}}%
\pgfpathmoveto{\pgfqpoint{1.174985in}{0.953328in}}%
\pgfpathlineto{\pgfqpoint{1.015772in}{0.905746in}}%
\pgfpathmoveto{\pgfqpoint{0.576803in}{0.848301in}}%
\pgfpathlineto{\pgfqpoint{0.493932in}{0.837819in}}%
\pgfpathmoveto{\pgfqpoint{0.576803in}{0.848301in}}%
\pgfpathlineto{\pgfqpoint{0.539694in}{0.759689in}}%
\pgfpathmoveto{\pgfqpoint{0.576803in}{0.848301in}}%
\pgfpathlineto{\pgfqpoint{0.666415in}{0.893785in}}%
\pgfpathmoveto{\pgfqpoint{0.576803in}{0.848301in}}%
\pgfpathlineto{\pgfqpoint{0.624324in}{0.786217in}}%
\pgfpathmoveto{\pgfqpoint{0.576803in}{0.848301in}}%
\pgfpathlineto{\pgfqpoint{0.566652in}{0.964750in}}%
\pgfpathlineto{\pgfqpoint{0.566652in}{0.964750in}}%
\pgfusepath{stroke}%
\end{pgfscope}%
\begin{pgfscope}%
\pgfpathrectangle{\pgfqpoint{0.100000in}{0.100000in}}{\pgfqpoint{1.782500in}{1.232000in}}%
\pgfusepath{clip}%
\pgfsetbuttcap%
\pgfsetroundjoin%
\definecolor{currentfill}{rgb}{0.054902,0.262745,0.486275}%
\pgfsetfillcolor{currentfill}%
\pgfsetlinewidth{1.003750pt}%
\definecolor{currentstroke}{rgb}{0.054902,0.262745,0.486275}%
\pgfsetstrokecolor{currentstroke}%
\pgfsetdash{}{0pt}%
\pgfsys@defobject{currentmarker}{\pgfqpoint{-0.018373in}{-0.018373in}}{\pgfqpoint{0.018373in}{0.018373in}}{%
\pgfpathmoveto{\pgfqpoint{0.000000in}{-0.018373in}}%
\pgfpathcurveto{\pgfqpoint{0.004873in}{-0.018373in}}{\pgfqpoint{0.009546in}{-0.016437in}}{\pgfqpoint{0.012992in}{-0.012992in}}%
\pgfpathcurveto{\pgfqpoint{0.016437in}{-0.009546in}}{\pgfqpoint{0.018373in}{-0.004873in}}{\pgfqpoint{0.018373in}{0.000000in}}%
\pgfpathcurveto{\pgfqpoint{0.018373in}{0.004873in}}{\pgfqpoint{0.016437in}{0.009546in}}{\pgfqpoint{0.012992in}{0.012992in}}%
\pgfpathcurveto{\pgfqpoint{0.009546in}{0.016437in}}{\pgfqpoint{0.004873in}{0.018373in}}{\pgfqpoint{0.000000in}{0.018373in}}%
\pgfpathcurveto{\pgfqpoint{-0.004873in}{0.018373in}}{\pgfqpoint{-0.009546in}{0.016437in}}{\pgfqpoint{-0.012992in}{0.012992in}}%
\pgfpathcurveto{\pgfqpoint{-0.016437in}{0.009546in}}{\pgfqpoint{-0.018373in}{0.004873in}}{\pgfqpoint{-0.018373in}{0.000000in}}%
\pgfpathcurveto{\pgfqpoint{-0.018373in}{-0.004873in}}{\pgfqpoint{-0.016437in}{-0.009546in}}{\pgfqpoint{-0.012992in}{-0.012992in}}%
\pgfpathcurveto{\pgfqpoint{-0.009546in}{-0.016437in}}{\pgfqpoint{-0.004873in}{-0.018373in}}{\pgfqpoint{0.000000in}{-0.018373in}}%
\pgfpathlineto{\pgfqpoint{0.000000in}{-0.018373in}}%
\pgfpathclose%
\pgfusepath{stroke,fill}%
}%
\begin{pgfscope}%
\pgfsys@transformshift{1.310324in}{0.660296in}%
\pgfsys@useobject{currentmarker}{}%
\end{pgfscope}%
\begin{pgfscope}%
\pgfsys@transformshift{1.253051in}{0.642085in}%
\pgfsys@useobject{currentmarker}{}%
\end{pgfscope}%
\begin{pgfscope}%
\pgfsys@transformshift{1.188115in}{0.625601in}%
\pgfsys@useobject{currentmarker}{}%
\end{pgfscope}%
\begin{pgfscope}%
\pgfsys@transformshift{1.127651in}{0.615595in}%
\pgfsys@useobject{currentmarker}{}%
\end{pgfscope}%
\begin{pgfscope}%
\pgfsys@transformshift{1.058490in}{0.606300in}%
\pgfsys@useobject{currentmarker}{}%
\end{pgfscope}%
\begin{pgfscope}%
\pgfsys@transformshift{0.990122in}{0.598971in}%
\pgfsys@useobject{currentmarker}{}%
\end{pgfscope}%
\begin{pgfscope}%
\pgfsys@transformshift{0.921265in}{0.600292in}%
\pgfsys@useobject{currentmarker}{}%
\end{pgfscope}%
\begin{pgfscope}%
\pgfsys@transformshift{0.853138in}{0.610721in}%
\pgfsys@useobject{currentmarker}{}%
\end{pgfscope}%
\begin{pgfscope}%
\pgfsys@transformshift{0.793331in}{0.622693in}%
\pgfsys@useobject{currentmarker}{}%
\end{pgfscope}%
\begin{pgfscope}%
\pgfsys@transformshift{0.728172in}{0.639388in}%
\pgfsys@useobject{currentmarker}{}%
\end{pgfscope}%
\begin{pgfscope}%
\pgfsys@transformshift{0.670517in}{0.657059in}%
\pgfsys@useobject{currentmarker}{}%
\end{pgfscope}%
\end{pgfscope}%
\begin{pgfscope}%
\pgfpathrectangle{\pgfqpoint{0.100000in}{0.100000in}}{\pgfqpoint{1.782500in}{1.232000in}}%
\pgfusepath{clip}%
\pgfsetbuttcap%
\pgfsetroundjoin%
\definecolor{currentfill}{rgb}{0.835294,0.321569,0.035294}%
\pgfsetfillcolor{currentfill}%
\pgfsetlinewidth{1.003750pt}%
\definecolor{currentstroke}{rgb}{0.835294,0.321569,0.035294}%
\pgfsetstrokecolor{currentstroke}%
\pgfsetdash{}{0pt}%
\pgfsys@defobject{currentmarker}{\pgfqpoint{-0.018373in}{-0.018373in}}{\pgfqpoint{0.018373in}{0.018373in}}{%
\pgfpathmoveto{\pgfqpoint{0.000000in}{-0.018373in}}%
\pgfpathcurveto{\pgfqpoint{0.004873in}{-0.018373in}}{\pgfqpoint{0.009546in}{-0.016437in}}{\pgfqpoint{0.012992in}{-0.012992in}}%
\pgfpathcurveto{\pgfqpoint{0.016437in}{-0.009546in}}{\pgfqpoint{0.018373in}{-0.004873in}}{\pgfqpoint{0.018373in}{0.000000in}}%
\pgfpathcurveto{\pgfqpoint{0.018373in}{0.004873in}}{\pgfqpoint{0.016437in}{0.009546in}}{\pgfqpoint{0.012992in}{0.012992in}}%
\pgfpathcurveto{\pgfqpoint{0.009546in}{0.016437in}}{\pgfqpoint{0.004873in}{0.018373in}}{\pgfqpoint{0.000000in}{0.018373in}}%
\pgfpathcurveto{\pgfqpoint{-0.004873in}{0.018373in}}{\pgfqpoint{-0.009546in}{0.016437in}}{\pgfqpoint{-0.012992in}{0.012992in}}%
\pgfpathcurveto{\pgfqpoint{-0.016437in}{0.009546in}}{\pgfqpoint{-0.018373in}{0.004873in}}{\pgfqpoint{-0.018373in}{0.000000in}}%
\pgfpathcurveto{\pgfqpoint{-0.018373in}{-0.004873in}}{\pgfqpoint{-0.016437in}{-0.009546in}}{\pgfqpoint{-0.012992in}{-0.012992in}}%
\pgfpathcurveto{\pgfqpoint{-0.009546in}{-0.016437in}}{\pgfqpoint{-0.004873in}{-0.018373in}}{\pgfqpoint{0.000000in}{-0.018373in}}%
\pgfpathlineto{\pgfqpoint{0.000000in}{-0.018373in}}%
\pgfpathclose%
\pgfusepath{stroke,fill}%
}%
\begin{pgfscope}%
\pgfsys@transformshift{0.666513in}{0.658314in}%
\pgfsys@useobject{currentmarker}{}%
\end{pgfscope}%
\begin{pgfscope}%
\pgfsys@transformshift{0.752247in}{0.632595in}%
\pgfsys@useobject{currentmarker}{}%
\end{pgfscope}%
\begin{pgfscope}%
\pgfsys@transformshift{0.833592in}{0.614513in}%
\pgfsys@useobject{currentmarker}{}%
\end{pgfscope}%
\begin{pgfscope}%
\pgfsys@transformshift{0.912544in}{0.600941in}%
\pgfsys@useobject{currentmarker}{}%
\end{pgfscope}%
\begin{pgfscope}%
\pgfsys@transformshift{0.990153in}{0.598729in}%
\pgfsys@useobject{currentmarker}{}%
\end{pgfscope}%
\begin{pgfscope}%
\pgfsys@transformshift{1.067159in}{0.607466in}%
\pgfsys@useobject{currentmarker}{}%
\end{pgfscope}%
\begin{pgfscope}%
\pgfsys@transformshift{1.147591in}{0.618472in}%
\pgfsys@useobject{currentmarker}{}%
\end{pgfscope}%
\begin{pgfscope}%
\pgfsys@transformshift{1.229103in}{0.635206in}%
\pgfsys@useobject{currentmarker}{}%
\end{pgfscope}%
\begin{pgfscope}%
\pgfsys@transformshift{1.314300in}{0.661595in}%
\pgfsys@useobject{currentmarker}{}%
\end{pgfscope}%
\end{pgfscope}%
\end{pgfpicture}%
\makeatother%
\endgroup%

        \caption{Iteration 3: Solve system}\label{fig:example-iter2-solution}
    \end{subfigure}
    \begin{subfigure}[b]{.32\linewidth}
        %% Creator: Matplotlib, PGF backend
%%
%% To include the figure in your LaTeX document, write
%%   \input{<filename>.pgf}
%%
%% Make sure the required packages are loaded in your preamble
%%   \usepackage{pgf}
%%
%% Also ensure that all the required font packages are loaded; for instance,
%% the lmodern package is sometimes necessary when using math font.
%%   \usepackage{lmodern}
%%
%% Figures using additional raster images can only be included by \input if
%% they are in the same directory as the main LaTeX file. For loading figures
%% from other directories you can use the `import` package
%%   \usepackage{import}
%%
%% and then include the figures with
%%   \import{<path to file>}{<filename>.pgf}
%%
%% Matplotlib used the following preamble
%%   
%%   \usepackage{fontspec}
%%   \setmainfont{DejaVuSans.ttf}[Path=\detokenize{/home/fabio/Internodes-CM/.venv/lib/python3.8/site-packages/matplotlib/mpl-data/fonts/ttf/}]
%%   \setsansfont{DejaVuSans.ttf}[Path=\detokenize{/home/fabio/Internodes-CM/.venv/lib/python3.8/site-packages/matplotlib/mpl-data/fonts/ttf/}]
%%   \setmonofont{DejaVuSansMono.ttf}[Path=\detokenize{/home/fabio/Internodes-CM/.venv/lib/python3.8/site-packages/matplotlib/mpl-data/fonts/ttf/}]
%%   \makeatletter\@ifpackageloaded{underscore}{}{\usepackage[strings]{underscore}}\makeatother
%%
\begingroup%
\makeatletter%
\begin{pgfpicture}%
\pgfpathrectangle{\pgfpointorigin}{\pgfqpoint{1.982500in}{1.432000in}}%
\pgfusepath{use as bounding box, clip}%
\begin{pgfscope}%
\pgfsetbuttcap%
\pgfsetmiterjoin%
\definecolor{currentfill}{rgb}{1.000000,1.000000,1.000000}%
\pgfsetfillcolor{currentfill}%
\pgfsetlinewidth{0.000000pt}%
\definecolor{currentstroke}{rgb}{1.000000,1.000000,1.000000}%
\pgfsetstrokecolor{currentstroke}%
\pgfsetdash{}{0pt}%
\pgfpathmoveto{\pgfqpoint{0.000000in}{0.000000in}}%
\pgfpathlineto{\pgfqpoint{1.982500in}{0.000000in}}%
\pgfpathlineto{\pgfqpoint{1.982500in}{1.432000in}}%
\pgfpathlineto{\pgfqpoint{0.000000in}{1.432000in}}%
\pgfpathlineto{\pgfqpoint{0.000000in}{0.000000in}}%
\pgfpathclose%
\pgfusepath{fill}%
\end{pgfscope}%
\begin{pgfscope}%
\pgfpathrectangle{\pgfqpoint{0.100000in}{0.100000in}}{\pgfqpoint{1.782500in}{1.232000in}}%
\pgfusepath{clip}%
\pgfsetrectcap%
\pgfsetroundjoin%
\pgfsetlinewidth{0.250937pt}%
\definecolor{currentstroke}{rgb}{0.054902,0.262745,0.486275}%
\pgfsetstrokecolor{currentstroke}%
\pgfsetdash{}{0pt}%
\pgfpathmoveto{\pgfqpoint{0.454502in}{0.100000in}}%
\pgfpathlineto{\pgfqpoint{0.185788in}{0.100000in}}%
\pgfpathmoveto{\pgfqpoint{0.723215in}{0.100000in}}%
\pgfpathlineto{\pgfqpoint{0.454502in}{0.100000in}}%
\pgfpathmoveto{\pgfqpoint{0.991929in}{0.100000in}}%
\pgfpathlineto{\pgfqpoint{0.723215in}{0.100000in}}%
\pgfpathmoveto{\pgfqpoint{1.260642in}{0.100000in}}%
\pgfpathlineto{\pgfqpoint{0.991929in}{0.100000in}}%
\pgfpathmoveto{\pgfqpoint{1.529356in}{0.100000in}}%
\pgfpathlineto{\pgfqpoint{1.798069in}{0.100000in}}%
\pgfpathmoveto{\pgfqpoint{1.529356in}{0.100000in}}%
\pgfpathlineto{\pgfqpoint{1.260642in}{0.100000in}}%
\pgfpathmoveto{\pgfqpoint{1.801477in}{0.408760in}}%
\pgfpathlineto{\pgfqpoint{1.798069in}{0.100000in}}%
\pgfpathmoveto{\pgfqpoint{1.801477in}{0.408760in}}%
\pgfpathlineto{\pgfqpoint{1.787931in}{0.719684in}}%
\pgfpathmoveto{\pgfqpoint{1.625404in}{0.710683in}}%
\pgfpathlineto{\pgfqpoint{1.787931in}{0.719684in}}%
\pgfpathmoveto{\pgfqpoint{1.625404in}{0.710683in}}%
\pgfpathlineto{\pgfqpoint{1.458297in}{0.696339in}}%
\pgfpathmoveto{\pgfqpoint{1.385143in}{0.684752in}}%
\pgfpathlineto{\pgfqpoint{1.458297in}{0.696339in}}%
\pgfpathmoveto{\pgfqpoint{1.310324in}{0.660296in}}%
\pgfpathlineto{\pgfqpoint{1.385143in}{0.684752in}}%
\pgfpathmoveto{\pgfqpoint{1.253051in}{0.642085in}}%
\pgfpathlineto{\pgfqpoint{1.310324in}{0.660296in}}%
\pgfpathmoveto{\pgfqpoint{1.188115in}{0.625601in}}%
\pgfpathlineto{\pgfqpoint{1.253051in}{0.642085in}}%
\pgfpathmoveto{\pgfqpoint{1.127651in}{0.615595in}}%
\pgfpathlineto{\pgfqpoint{1.188115in}{0.625601in}}%
\pgfpathmoveto{\pgfqpoint{1.058490in}{0.606300in}}%
\pgfpathlineto{\pgfqpoint{1.127651in}{0.615595in}}%
\pgfpathmoveto{\pgfqpoint{0.990122in}{0.598971in}}%
\pgfpathlineto{\pgfqpoint{1.058490in}{0.606300in}}%
\pgfpathmoveto{\pgfqpoint{0.921265in}{0.600292in}}%
\pgfpathlineto{\pgfqpoint{0.990122in}{0.598971in}}%
\pgfpathmoveto{\pgfqpoint{0.853138in}{0.610721in}}%
\pgfpathlineto{\pgfqpoint{0.921265in}{0.600292in}}%
\pgfpathmoveto{\pgfqpoint{0.793331in}{0.622693in}}%
\pgfpathlineto{\pgfqpoint{0.853138in}{0.610721in}}%
\pgfpathmoveto{\pgfqpoint{0.728172in}{0.639388in}}%
\pgfpathlineto{\pgfqpoint{0.793331in}{0.622693in}}%
\pgfpathmoveto{\pgfqpoint{0.670517in}{0.657059in}}%
\pgfpathlineto{\pgfqpoint{0.728172in}{0.639388in}}%
\pgfpathmoveto{\pgfqpoint{0.597734in}{0.682628in}}%
\pgfpathlineto{\pgfqpoint{0.523844in}{0.695653in}}%
\pgfpathmoveto{\pgfqpoint{0.597734in}{0.682628in}}%
\pgfpathlineto{\pgfqpoint{0.670517in}{0.657059in}}%
\pgfpathmoveto{\pgfqpoint{0.357342in}{0.710776in}}%
\pgfpathlineto{\pgfqpoint{0.523844in}{0.695653in}}%
\pgfpathmoveto{\pgfqpoint{0.357342in}{0.710776in}}%
\pgfpathlineto{\pgfqpoint{0.194832in}{0.720303in}}%
\pgfpathmoveto{\pgfqpoint{0.181023in}{0.408839in}}%
\pgfpathlineto{\pgfqpoint{0.185788in}{0.100000in}}%
\pgfpathmoveto{\pgfqpoint{0.181023in}{0.408839in}}%
\pgfpathlineto{\pgfqpoint{0.194832in}{0.720303in}}%
\pgfpathmoveto{\pgfqpoint{1.380079in}{0.372824in}}%
\pgfpathlineto{\pgfqpoint{1.260642in}{0.100000in}}%
\pgfpathmoveto{\pgfqpoint{1.085607in}{0.353637in}}%
\pgfpathlineto{\pgfqpoint{0.991929in}{0.100000in}}%
\pgfpathmoveto{\pgfqpoint{1.236571in}{0.465285in}}%
\pgfpathlineto{\pgfqpoint{1.380079in}{0.372824in}}%
\pgfpathmoveto{\pgfqpoint{1.236571in}{0.465285in}}%
\pgfpathlineto{\pgfqpoint{1.085607in}{0.353637in}}%
\pgfpathmoveto{\pgfqpoint{0.955470in}{0.447258in}}%
\pgfpathlineto{\pgfqpoint{1.085607in}{0.353637in}}%
\pgfpathmoveto{\pgfqpoint{0.955470in}{0.447258in}}%
\pgfpathlineto{\pgfqpoint{0.816937in}{0.367970in}}%
\pgfpathmoveto{\pgfqpoint{0.675550in}{0.476601in}}%
\pgfpathlineto{\pgfqpoint{0.816937in}{0.367970in}}%
\pgfpathmoveto{\pgfqpoint{0.675550in}{0.476601in}}%
\pgfpathlineto{\pgfqpoint{0.534719in}{0.383741in}}%
\pgfpathmoveto{\pgfqpoint{1.535916in}{0.486155in}}%
\pgfpathlineto{\pgfqpoint{1.801477in}{0.408760in}}%
\pgfpathmoveto{\pgfqpoint{1.535916in}{0.486155in}}%
\pgfpathlineto{\pgfqpoint{1.380079in}{0.372824in}}%
\pgfpathmoveto{\pgfqpoint{1.376985in}{0.517176in}}%
\pgfpathlineto{\pgfqpoint{1.380079in}{0.372824in}}%
\pgfpathmoveto{\pgfqpoint{1.376985in}{0.517176in}}%
\pgfpathlineto{\pgfqpoint{1.236571in}{0.465285in}}%
\pgfpathmoveto{\pgfqpoint{1.376985in}{0.517176in}}%
\pgfpathlineto{\pgfqpoint{1.535916in}{0.486155in}}%
\pgfpathmoveto{\pgfqpoint{1.093931in}{0.479746in}}%
\pgfpathlineto{\pgfqpoint{1.085607in}{0.353637in}}%
\pgfpathmoveto{\pgfqpoint{1.093931in}{0.479746in}}%
\pgfpathlineto{\pgfqpoint{1.236571in}{0.465285in}}%
\pgfpathmoveto{\pgfqpoint{1.093931in}{0.479746in}}%
\pgfpathlineto{\pgfqpoint{0.955470in}{0.447258in}}%
\pgfpathmoveto{\pgfqpoint{0.816759in}{0.485778in}}%
\pgfpathlineto{\pgfqpoint{0.816937in}{0.367970in}}%
\pgfpathmoveto{\pgfqpoint{0.816759in}{0.485778in}}%
\pgfpathlineto{\pgfqpoint{0.955470in}{0.447258in}}%
\pgfpathmoveto{\pgfqpoint{0.816759in}{0.485778in}}%
\pgfpathlineto{\pgfqpoint{0.675550in}{0.476601in}}%
\pgfpathmoveto{\pgfqpoint{0.526682in}{0.528714in}}%
\pgfpathlineto{\pgfqpoint{0.534719in}{0.383741in}}%
\pgfpathmoveto{\pgfqpoint{0.526682in}{0.528714in}}%
\pgfpathlineto{\pgfqpoint{0.675550in}{0.476601in}}%
\pgfpathmoveto{\pgfqpoint{0.357678in}{0.492105in}}%
\pgfpathlineto{\pgfqpoint{0.194832in}{0.720303in}}%
\pgfpathmoveto{\pgfqpoint{0.357678in}{0.492105in}}%
\pgfpathlineto{\pgfqpoint{0.357342in}{0.710776in}}%
\pgfpathmoveto{\pgfqpoint{0.357678in}{0.492105in}}%
\pgfpathlineto{\pgfqpoint{0.181023in}{0.408839in}}%
\pgfpathmoveto{\pgfqpoint{0.357678in}{0.492105in}}%
\pgfpathlineto{\pgfqpoint{0.534719in}{0.383741in}}%
\pgfpathmoveto{\pgfqpoint{0.357678in}{0.492105in}}%
\pgfpathlineto{\pgfqpoint{0.526682in}{0.528714in}}%
\pgfpathmoveto{\pgfqpoint{1.298659in}{0.566272in}}%
\pgfpathlineto{\pgfqpoint{1.310324in}{0.660296in}}%
\pgfpathmoveto{\pgfqpoint{1.298659in}{0.566272in}}%
\pgfpathlineto{\pgfqpoint{1.253051in}{0.642085in}}%
\pgfpathmoveto{\pgfqpoint{1.298659in}{0.566272in}}%
\pgfpathlineto{\pgfqpoint{1.236571in}{0.465285in}}%
\pgfpathmoveto{\pgfqpoint{1.298659in}{0.566272in}}%
\pgfpathlineto{\pgfqpoint{1.376985in}{0.517176in}}%
\pgfpathmoveto{\pgfqpoint{1.163096in}{0.542097in}}%
\pgfpathlineto{\pgfqpoint{1.188115in}{0.625601in}}%
\pgfpathmoveto{\pgfqpoint{1.163096in}{0.542097in}}%
\pgfpathlineto{\pgfqpoint{1.127651in}{0.615595in}}%
\pgfpathmoveto{\pgfqpoint{1.163096in}{0.542097in}}%
\pgfpathlineto{\pgfqpoint{1.236571in}{0.465285in}}%
\pgfpathmoveto{\pgfqpoint{1.163096in}{0.542097in}}%
\pgfpathlineto{\pgfqpoint{1.093931in}{0.479746in}}%
\pgfpathmoveto{\pgfqpoint{1.026137in}{0.529850in}}%
\pgfpathlineto{\pgfqpoint{1.058490in}{0.606300in}}%
\pgfpathmoveto{\pgfqpoint{1.026137in}{0.529850in}}%
\pgfpathlineto{\pgfqpoint{0.990122in}{0.598971in}}%
\pgfpathmoveto{\pgfqpoint{1.026137in}{0.529850in}}%
\pgfpathlineto{\pgfqpoint{0.955470in}{0.447258in}}%
\pgfpathmoveto{\pgfqpoint{1.026137in}{0.529850in}}%
\pgfpathlineto{\pgfqpoint{1.093931in}{0.479746in}}%
\pgfpathmoveto{\pgfqpoint{0.886661in}{0.531913in}}%
\pgfpathlineto{\pgfqpoint{0.921265in}{0.600292in}}%
\pgfpathmoveto{\pgfqpoint{0.886661in}{0.531913in}}%
\pgfpathlineto{\pgfqpoint{0.853138in}{0.610721in}}%
\pgfpathmoveto{\pgfqpoint{0.886661in}{0.531913in}}%
\pgfpathlineto{\pgfqpoint{0.955470in}{0.447258in}}%
\pgfpathmoveto{\pgfqpoint{0.886661in}{0.531913in}}%
\pgfpathlineto{\pgfqpoint{0.816759in}{0.485778in}}%
\pgfpathmoveto{\pgfqpoint{0.750901in}{0.550725in}}%
\pgfpathlineto{\pgfqpoint{0.793331in}{0.622693in}}%
\pgfpathmoveto{\pgfqpoint{0.750901in}{0.550725in}}%
\pgfpathlineto{\pgfqpoint{0.728172in}{0.639388in}}%
\pgfpathmoveto{\pgfqpoint{0.750901in}{0.550725in}}%
\pgfpathlineto{\pgfqpoint{0.675550in}{0.476601in}}%
\pgfpathmoveto{\pgfqpoint{0.750901in}{0.550725in}}%
\pgfpathlineto{\pgfqpoint{0.816759in}{0.485778in}}%
\pgfpathmoveto{\pgfqpoint{0.614919in}{0.583110in}}%
\pgfpathlineto{\pgfqpoint{0.670517in}{0.657059in}}%
\pgfpathmoveto{\pgfqpoint{0.614919in}{0.583110in}}%
\pgfpathlineto{\pgfqpoint{0.597734in}{0.682628in}}%
\pgfpathmoveto{\pgfqpoint{0.614919in}{0.583110in}}%
\pgfpathlineto{\pgfqpoint{0.675550in}{0.476601in}}%
\pgfpathmoveto{\pgfqpoint{0.614919in}{0.583110in}}%
\pgfpathlineto{\pgfqpoint{0.526682in}{0.528714in}}%
\pgfpathmoveto{\pgfqpoint{1.445323in}{0.602316in}}%
\pgfpathlineto{\pgfqpoint{1.458297in}{0.696339in}}%
\pgfpathmoveto{\pgfqpoint{1.445323in}{0.602316in}}%
\pgfpathlineto{\pgfqpoint{1.385143in}{0.684752in}}%
\pgfpathmoveto{\pgfqpoint{1.445323in}{0.602316in}}%
\pgfpathlineto{\pgfqpoint{1.535916in}{0.486155in}}%
\pgfpathmoveto{\pgfqpoint{1.445323in}{0.602316in}}%
\pgfpathlineto{\pgfqpoint{1.376985in}{0.517176in}}%
\pgfpathmoveto{\pgfqpoint{0.938998in}{0.293567in}}%
\pgfpathlineto{\pgfqpoint{0.991929in}{0.100000in}}%
\pgfpathmoveto{\pgfqpoint{0.938998in}{0.293567in}}%
\pgfpathlineto{\pgfqpoint{1.085607in}{0.353637in}}%
\pgfpathmoveto{\pgfqpoint{0.938998in}{0.293567in}}%
\pgfpathlineto{\pgfqpoint{0.816937in}{0.367970in}}%
\pgfpathmoveto{\pgfqpoint{0.938998in}{0.293567in}}%
\pgfpathlineto{\pgfqpoint{0.955470in}{0.447258in}}%
\pgfpathmoveto{\pgfqpoint{0.441612in}{0.602117in}}%
\pgfpathlineto{\pgfqpoint{0.523844in}{0.695653in}}%
\pgfpathmoveto{\pgfqpoint{0.441612in}{0.602117in}}%
\pgfpathlineto{\pgfqpoint{0.357342in}{0.710776in}}%
\pgfpathmoveto{\pgfqpoint{0.441612in}{0.602117in}}%
\pgfpathlineto{\pgfqpoint{0.526682in}{0.528714in}}%
\pgfpathmoveto{\pgfqpoint{0.441612in}{0.602117in}}%
\pgfpathlineto{\pgfqpoint{0.357678in}{0.492105in}}%
\pgfpathmoveto{\pgfqpoint{1.364219in}{0.606819in}}%
\pgfpathlineto{\pgfqpoint{1.385143in}{0.684752in}}%
\pgfpathmoveto{\pgfqpoint{1.364219in}{0.606819in}}%
\pgfpathlineto{\pgfqpoint{1.310324in}{0.660296in}}%
\pgfpathmoveto{\pgfqpoint{1.364219in}{0.606819in}}%
\pgfpathlineto{\pgfqpoint{1.376985in}{0.517176in}}%
\pgfpathmoveto{\pgfqpoint{1.364219in}{0.606819in}}%
\pgfpathlineto{\pgfqpoint{1.298659in}{0.566272in}}%
\pgfpathmoveto{\pgfqpoint{1.364219in}{0.606819in}}%
\pgfpathlineto{\pgfqpoint{1.445323in}{0.602316in}}%
\pgfpathmoveto{\pgfqpoint{1.094158in}{0.554638in}}%
\pgfpathlineto{\pgfqpoint{1.127651in}{0.615595in}}%
\pgfpathmoveto{\pgfqpoint{1.094158in}{0.554638in}}%
\pgfpathlineto{\pgfqpoint{1.058490in}{0.606300in}}%
\pgfpathmoveto{\pgfqpoint{1.094158in}{0.554638in}}%
\pgfpathlineto{\pgfqpoint{1.093931in}{0.479746in}}%
\pgfpathmoveto{\pgfqpoint{1.094158in}{0.554638in}}%
\pgfpathlineto{\pgfqpoint{1.163096in}{0.542097in}}%
\pgfpathmoveto{\pgfqpoint{1.094158in}{0.554638in}}%
\pgfpathlineto{\pgfqpoint{1.026137in}{0.529850in}}%
\pgfpathmoveto{\pgfqpoint{1.228887in}{0.571665in}}%
\pgfpathlineto{\pgfqpoint{1.253051in}{0.642085in}}%
\pgfpathmoveto{\pgfqpoint{1.228887in}{0.571665in}}%
\pgfpathlineto{\pgfqpoint{1.188115in}{0.625601in}}%
\pgfpathmoveto{\pgfqpoint{1.228887in}{0.571665in}}%
\pgfpathlineto{\pgfqpoint{1.236571in}{0.465285in}}%
\pgfpathmoveto{\pgfqpoint{1.228887in}{0.571665in}}%
\pgfpathlineto{\pgfqpoint{1.298659in}{0.566272in}}%
\pgfpathmoveto{\pgfqpoint{1.228887in}{0.571665in}}%
\pgfpathlineto{\pgfqpoint{1.163096in}{0.542097in}}%
\pgfpathmoveto{\pgfqpoint{0.956125in}{0.541001in}}%
\pgfpathlineto{\pgfqpoint{0.990122in}{0.598971in}}%
\pgfpathmoveto{\pgfqpoint{0.956125in}{0.541001in}}%
\pgfpathlineto{\pgfqpoint{0.921265in}{0.600292in}}%
\pgfpathmoveto{\pgfqpoint{0.956125in}{0.541001in}}%
\pgfpathlineto{\pgfqpoint{0.955470in}{0.447258in}}%
\pgfpathmoveto{\pgfqpoint{0.956125in}{0.541001in}}%
\pgfpathlineto{\pgfqpoint{1.026137in}{0.529850in}}%
\pgfpathmoveto{\pgfqpoint{0.956125in}{0.541001in}}%
\pgfpathlineto{\pgfqpoint{0.886661in}{0.531913in}}%
\pgfpathmoveto{\pgfqpoint{0.819642in}{0.557313in}}%
\pgfpathlineto{\pgfqpoint{0.853138in}{0.610721in}}%
\pgfpathmoveto{\pgfqpoint{0.819642in}{0.557313in}}%
\pgfpathlineto{\pgfqpoint{0.793331in}{0.622693in}}%
\pgfpathmoveto{\pgfqpoint{0.819642in}{0.557313in}}%
\pgfpathlineto{\pgfqpoint{0.816759in}{0.485778in}}%
\pgfpathmoveto{\pgfqpoint{0.819642in}{0.557313in}}%
\pgfpathlineto{\pgfqpoint{0.886661in}{0.531913in}}%
\pgfpathmoveto{\pgfqpoint{0.819642in}{0.557313in}}%
\pgfpathlineto{\pgfqpoint{0.750901in}{0.550725in}}%
\pgfpathmoveto{\pgfqpoint{1.193096in}{0.276126in}}%
\pgfpathlineto{\pgfqpoint{0.991929in}{0.100000in}}%
\pgfpathmoveto{\pgfqpoint{1.193096in}{0.276126in}}%
\pgfpathlineto{\pgfqpoint{1.260642in}{0.100000in}}%
\pgfpathmoveto{\pgfqpoint{1.193096in}{0.276126in}}%
\pgfpathlineto{\pgfqpoint{1.380079in}{0.372824in}}%
\pgfpathmoveto{\pgfqpoint{1.193096in}{0.276126in}}%
\pgfpathlineto{\pgfqpoint{1.085607in}{0.353637in}}%
\pgfpathmoveto{\pgfqpoint{1.193096in}{0.276126in}}%
\pgfpathlineto{\pgfqpoint{1.236571in}{0.465285in}}%
\pgfpathmoveto{\pgfqpoint{0.690674in}{0.289421in}}%
\pgfpathlineto{\pgfqpoint{0.723215in}{0.100000in}}%
\pgfpathmoveto{\pgfqpoint{0.690674in}{0.289421in}}%
\pgfpathlineto{\pgfqpoint{0.816937in}{0.367970in}}%
\pgfpathmoveto{\pgfqpoint{0.690674in}{0.289421in}}%
\pgfpathlineto{\pgfqpoint{0.534719in}{0.383741in}}%
\pgfpathmoveto{\pgfqpoint{0.690674in}{0.289421in}}%
\pgfpathlineto{\pgfqpoint{0.675550in}{0.476601in}}%
\pgfpathmoveto{\pgfqpoint{0.686113in}{0.579464in}}%
\pgfpathlineto{\pgfqpoint{0.728172in}{0.639388in}}%
\pgfpathmoveto{\pgfqpoint{0.686113in}{0.579464in}}%
\pgfpathlineto{\pgfqpoint{0.670517in}{0.657059in}}%
\pgfpathmoveto{\pgfqpoint{0.686113in}{0.579464in}}%
\pgfpathlineto{\pgfqpoint{0.675550in}{0.476601in}}%
\pgfpathmoveto{\pgfqpoint{0.686113in}{0.579464in}}%
\pgfpathlineto{\pgfqpoint{0.750901in}{0.550725in}}%
\pgfpathmoveto{\pgfqpoint{0.686113in}{0.579464in}}%
\pgfpathlineto{\pgfqpoint{0.614919in}{0.583110in}}%
\pgfpathmoveto{\pgfqpoint{1.547819in}{0.613854in}}%
\pgfpathlineto{\pgfqpoint{1.458297in}{0.696339in}}%
\pgfpathmoveto{\pgfqpoint{1.547819in}{0.613854in}}%
\pgfpathlineto{\pgfqpoint{1.625404in}{0.710683in}}%
\pgfpathmoveto{\pgfqpoint{1.547819in}{0.613854in}}%
\pgfpathlineto{\pgfqpoint{1.535916in}{0.486155in}}%
\pgfpathmoveto{\pgfqpoint{1.547819in}{0.613854in}}%
\pgfpathlineto{\pgfqpoint{1.445323in}{0.602316in}}%
\pgfpathmoveto{\pgfqpoint{0.550365in}{0.619266in}}%
\pgfpathlineto{\pgfqpoint{0.523844in}{0.695653in}}%
\pgfpathmoveto{\pgfqpoint{0.550365in}{0.619266in}}%
\pgfpathlineto{\pgfqpoint{0.597734in}{0.682628in}}%
\pgfpathmoveto{\pgfqpoint{0.550365in}{0.619266in}}%
\pgfpathlineto{\pgfqpoint{0.526682in}{0.528714in}}%
\pgfpathmoveto{\pgfqpoint{0.550365in}{0.619266in}}%
\pgfpathlineto{\pgfqpoint{0.614919in}{0.583110in}}%
\pgfpathmoveto{\pgfqpoint{0.550365in}{0.619266in}}%
\pgfpathlineto{\pgfqpoint{0.441612in}{0.602117in}}%
\pgfpathmoveto{\pgfqpoint{0.377336in}{0.284113in}}%
\pgfpathlineto{\pgfqpoint{0.185788in}{0.100000in}}%
\pgfpathmoveto{\pgfqpoint{0.377336in}{0.284113in}}%
\pgfpathlineto{\pgfqpoint{0.454502in}{0.100000in}}%
\pgfpathmoveto{\pgfqpoint{0.377336in}{0.284113in}}%
\pgfpathlineto{\pgfqpoint{0.181023in}{0.408839in}}%
\pgfpathmoveto{\pgfqpoint{0.377336in}{0.284113in}}%
\pgfpathlineto{\pgfqpoint{0.534719in}{0.383741in}}%
\pgfpathmoveto{\pgfqpoint{0.377336in}{0.284113in}}%
\pgfpathlineto{\pgfqpoint{0.357678in}{0.492105in}}%
\pgfpathmoveto{\pgfqpoint{1.533730in}{0.293058in}}%
\pgfpathlineto{\pgfqpoint{1.798069in}{0.100000in}}%
\pgfpathmoveto{\pgfqpoint{1.533730in}{0.293058in}}%
\pgfpathlineto{\pgfqpoint{1.260642in}{0.100000in}}%
\pgfpathmoveto{\pgfqpoint{1.533730in}{0.293058in}}%
\pgfpathlineto{\pgfqpoint{1.529356in}{0.100000in}}%
\pgfpathmoveto{\pgfqpoint{1.533730in}{0.293058in}}%
\pgfpathlineto{\pgfqpoint{1.801477in}{0.408760in}}%
\pgfpathmoveto{\pgfqpoint{1.533730in}{0.293058in}}%
\pgfpathlineto{\pgfqpoint{1.380079in}{0.372824in}}%
\pgfpathmoveto{\pgfqpoint{1.533730in}{0.293058in}}%
\pgfpathlineto{\pgfqpoint{1.535916in}{0.486155in}}%
\pgfpathmoveto{\pgfqpoint{1.660037in}{0.588517in}}%
\pgfpathlineto{\pgfqpoint{1.787931in}{0.719684in}}%
\pgfpathmoveto{\pgfqpoint{1.660037in}{0.588517in}}%
\pgfpathlineto{\pgfqpoint{1.801477in}{0.408760in}}%
\pgfpathmoveto{\pgfqpoint{1.660037in}{0.588517in}}%
\pgfpathlineto{\pgfqpoint{1.625404in}{0.710683in}}%
\pgfpathmoveto{\pgfqpoint{1.660037in}{0.588517in}}%
\pgfpathlineto{\pgfqpoint{1.535916in}{0.486155in}}%
\pgfpathmoveto{\pgfqpoint{1.660037in}{0.588517in}}%
\pgfpathlineto{\pgfqpoint{1.547819in}{0.613854in}}%
\pgfpathmoveto{\pgfqpoint{0.840010in}{0.211728in}}%
\pgfpathlineto{\pgfqpoint{0.723215in}{0.100000in}}%
\pgfpathmoveto{\pgfqpoint{0.840010in}{0.211728in}}%
\pgfpathlineto{\pgfqpoint{0.991929in}{0.100000in}}%
\pgfpathmoveto{\pgfqpoint{0.840010in}{0.211728in}}%
\pgfpathlineto{\pgfqpoint{0.816937in}{0.367970in}}%
\pgfpathmoveto{\pgfqpoint{0.840010in}{0.211728in}}%
\pgfpathlineto{\pgfqpoint{0.938998in}{0.293567in}}%
\pgfpathmoveto{\pgfqpoint{0.840010in}{0.211728in}}%
\pgfpathlineto{\pgfqpoint{0.690674in}{0.289421in}}%
\pgfpathmoveto{\pgfqpoint{0.553809in}{0.230866in}}%
\pgfpathlineto{\pgfqpoint{0.454502in}{0.100000in}}%
\pgfpathmoveto{\pgfqpoint{0.553809in}{0.230866in}}%
\pgfpathlineto{\pgfqpoint{0.723215in}{0.100000in}}%
\pgfpathmoveto{\pgfqpoint{0.553809in}{0.230866in}}%
\pgfpathlineto{\pgfqpoint{0.534719in}{0.383741in}}%
\pgfpathmoveto{\pgfqpoint{0.553809in}{0.230866in}}%
\pgfpathlineto{\pgfqpoint{0.690674in}{0.289421in}}%
\pgfpathmoveto{\pgfqpoint{0.553809in}{0.230866in}}%
\pgfpathlineto{\pgfqpoint{0.377336in}{0.284113in}}%
\pgfpathlineto{\pgfqpoint{0.377336in}{0.284113in}}%
\pgfusepath{stroke}%
\end{pgfscope}%
\begin{pgfscope}%
\pgfpathrectangle{\pgfqpoint{0.100000in}{0.100000in}}{\pgfqpoint{1.782500in}{1.232000in}}%
\pgfusepath{clip}%
\pgfsetrectcap%
\pgfsetroundjoin%
\pgfsetlinewidth{0.250937pt}%
\definecolor{currentstroke}{rgb}{0.835294,0.321569,0.035294}%
\pgfsetstrokecolor{currentstroke}%
\pgfsetdash{}{0pt}%
\pgfpathmoveto{\pgfqpoint{0.493932in}{0.837819in}}%
\pgfpathlineto{\pgfqpoint{0.454502in}{1.085600in}}%
\pgfpathmoveto{\pgfqpoint{1.529356in}{1.085600in}}%
\pgfpathlineto{\pgfqpoint{1.489237in}{0.837484in}}%
\pgfpathmoveto{\pgfqpoint{0.723215in}{1.085600in}}%
\pgfpathlineto{\pgfqpoint{0.991929in}{1.085600in}}%
\pgfpathmoveto{\pgfqpoint{0.723215in}{1.085600in}}%
\pgfpathlineto{\pgfqpoint{0.454502in}{1.085600in}}%
\pgfpathmoveto{\pgfqpoint{0.539694in}{0.759689in}}%
\pgfpathlineto{\pgfqpoint{0.493932in}{0.837819in}}%
\pgfpathmoveto{\pgfqpoint{0.597797in}{0.697728in}}%
\pgfpathlineto{\pgfqpoint{0.539694in}{0.759689in}}%
\pgfpathmoveto{\pgfqpoint{0.666513in}{0.658314in}}%
\pgfpathlineto{\pgfqpoint{0.597797in}{0.697728in}}%
\pgfpathmoveto{\pgfqpoint{0.752247in}{0.632595in}}%
\pgfpathlineto{\pgfqpoint{0.666513in}{0.658314in}}%
\pgfpathmoveto{\pgfqpoint{0.833592in}{0.614513in}}%
\pgfpathlineto{\pgfqpoint{0.752247in}{0.632595in}}%
\pgfpathmoveto{\pgfqpoint{0.912544in}{0.600941in}}%
\pgfpathlineto{\pgfqpoint{0.833592in}{0.614513in}}%
\pgfpathmoveto{\pgfqpoint{0.990153in}{0.598729in}}%
\pgfpathlineto{\pgfqpoint{0.912544in}{0.600941in}}%
\pgfpathmoveto{\pgfqpoint{1.067159in}{0.607466in}}%
\pgfpathlineto{\pgfqpoint{0.990153in}{0.598729in}}%
\pgfpathmoveto{\pgfqpoint{1.147591in}{0.618472in}}%
\pgfpathlineto{\pgfqpoint{1.067159in}{0.607466in}}%
\pgfpathmoveto{\pgfqpoint{1.229103in}{0.635206in}}%
\pgfpathlineto{\pgfqpoint{1.147591in}{0.618472in}}%
\pgfpathmoveto{\pgfqpoint{1.314300in}{0.661595in}}%
\pgfpathlineto{\pgfqpoint{1.229103in}{0.635206in}}%
\pgfpathmoveto{\pgfqpoint{1.383298in}{0.696947in}}%
\pgfpathlineto{\pgfqpoint{1.314300in}{0.661595in}}%
\pgfpathmoveto{\pgfqpoint{1.443404in}{0.759511in}}%
\pgfpathlineto{\pgfqpoint{1.489237in}{0.837484in}}%
\pgfpathmoveto{\pgfqpoint{1.443404in}{0.759511in}}%
\pgfpathlineto{\pgfqpoint{1.383298in}{0.696947in}}%
\pgfpathmoveto{\pgfqpoint{1.260642in}{1.085600in}}%
\pgfpathlineto{\pgfqpoint{0.991929in}{1.085600in}}%
\pgfpathmoveto{\pgfqpoint{1.260642in}{1.085600in}}%
\pgfpathlineto{\pgfqpoint{1.529356in}{1.085600in}}%
\pgfpathmoveto{\pgfqpoint{0.831595in}{0.886984in}}%
\pgfpathlineto{\pgfqpoint{0.991929in}{1.085600in}}%
\pgfpathmoveto{\pgfqpoint{0.831595in}{0.886984in}}%
\pgfpathlineto{\pgfqpoint{0.723215in}{1.085600in}}%
\pgfpathmoveto{\pgfqpoint{0.960383in}{0.754944in}}%
\pgfpathlineto{\pgfqpoint{1.116418in}{0.824542in}}%
\pgfpathmoveto{\pgfqpoint{0.960383in}{0.754944in}}%
\pgfpathlineto{\pgfqpoint{0.831595in}{0.886984in}}%
\pgfpathmoveto{\pgfqpoint{1.283109in}{0.858031in}}%
\pgfpathlineto{\pgfqpoint{1.116418in}{0.824542in}}%
\pgfpathmoveto{\pgfqpoint{0.666415in}{0.893785in}}%
\pgfpathlineto{\pgfqpoint{0.723215in}{1.085600in}}%
\pgfpathmoveto{\pgfqpoint{0.666415in}{0.893785in}}%
\pgfpathlineto{\pgfqpoint{0.831595in}{0.886984in}}%
\pgfpathmoveto{\pgfqpoint{1.091714in}{0.718622in}}%
\pgfpathlineto{\pgfqpoint{1.067159in}{0.607466in}}%
\pgfpathmoveto{\pgfqpoint{1.091714in}{0.718622in}}%
\pgfpathlineto{\pgfqpoint{1.147591in}{0.618472in}}%
\pgfpathmoveto{\pgfqpoint{1.091714in}{0.718622in}}%
\pgfpathlineto{\pgfqpoint{1.116418in}{0.824542in}}%
\pgfpathmoveto{\pgfqpoint{1.091714in}{0.718622in}}%
\pgfpathlineto{\pgfqpoint{0.960383in}{0.754944in}}%
\pgfpathmoveto{\pgfqpoint{1.216169in}{0.750966in}}%
\pgfpathlineto{\pgfqpoint{1.229103in}{0.635206in}}%
\pgfpathmoveto{\pgfqpoint{1.216169in}{0.750966in}}%
\pgfpathlineto{\pgfqpoint{1.314300in}{0.661595in}}%
\pgfpathmoveto{\pgfqpoint{1.216169in}{0.750966in}}%
\pgfpathlineto{\pgfqpoint{1.116418in}{0.824542in}}%
\pgfpathmoveto{\pgfqpoint{1.216169in}{0.750966in}}%
\pgfpathlineto{\pgfqpoint{1.283109in}{0.858031in}}%
\pgfpathmoveto{\pgfqpoint{1.216169in}{0.750966in}}%
\pgfpathlineto{\pgfqpoint{1.091714in}{0.718622in}}%
\pgfpathmoveto{\pgfqpoint{0.819793in}{0.728687in}}%
\pgfpathlineto{\pgfqpoint{0.752247in}{0.632595in}}%
\pgfpathmoveto{\pgfqpoint{0.819793in}{0.728687in}}%
\pgfpathlineto{\pgfqpoint{0.833592in}{0.614513in}}%
\pgfpathmoveto{\pgfqpoint{0.819793in}{0.728687in}}%
\pgfpathlineto{\pgfqpoint{0.831595in}{0.886984in}}%
\pgfpathmoveto{\pgfqpoint{0.819793in}{0.728687in}}%
\pgfpathlineto{\pgfqpoint{0.960383in}{0.754944in}}%
\pgfpathmoveto{\pgfqpoint{0.708591in}{0.780276in}}%
\pgfpathlineto{\pgfqpoint{0.597797in}{0.697728in}}%
\pgfpathmoveto{\pgfqpoint{0.708591in}{0.780276in}}%
\pgfpathlineto{\pgfqpoint{0.666513in}{0.658314in}}%
\pgfpathmoveto{\pgfqpoint{0.708591in}{0.780276in}}%
\pgfpathlineto{\pgfqpoint{0.831595in}{0.886984in}}%
\pgfpathmoveto{\pgfqpoint{0.708591in}{0.780276in}}%
\pgfpathlineto{\pgfqpoint{0.666415in}{0.893785in}}%
\pgfpathmoveto{\pgfqpoint{0.708591in}{0.780276in}}%
\pgfpathlineto{\pgfqpoint{0.819793in}{0.728687in}}%
\pgfpathmoveto{\pgfqpoint{1.369911in}{0.959364in}}%
\pgfpathlineto{\pgfqpoint{1.489237in}{0.837484in}}%
\pgfpathmoveto{\pgfqpoint{1.369911in}{0.959364in}}%
\pgfpathlineto{\pgfqpoint{1.529356in}{1.085600in}}%
\pgfpathmoveto{\pgfqpoint{1.369911in}{0.959364in}}%
\pgfpathlineto{\pgfqpoint{1.260642in}{1.085600in}}%
\pgfpathmoveto{\pgfqpoint{1.369911in}{0.959364in}}%
\pgfpathlineto{\pgfqpoint{1.283109in}{0.858031in}}%
\pgfpathmoveto{\pgfqpoint{0.953812in}{0.656912in}}%
\pgfpathlineto{\pgfqpoint{0.912544in}{0.600941in}}%
\pgfpathmoveto{\pgfqpoint{0.953812in}{0.656912in}}%
\pgfpathlineto{\pgfqpoint{0.990153in}{0.598729in}}%
\pgfpathmoveto{\pgfqpoint{0.953812in}{0.656912in}}%
\pgfpathlineto{\pgfqpoint{0.960383in}{0.754944in}}%
\pgfpathmoveto{\pgfqpoint{1.303198in}{0.744504in}}%
\pgfpathlineto{\pgfqpoint{1.314300in}{0.661595in}}%
\pgfpathmoveto{\pgfqpoint{1.303198in}{0.744504in}}%
\pgfpathlineto{\pgfqpoint{1.383298in}{0.696947in}}%
\pgfpathmoveto{\pgfqpoint{1.303198in}{0.744504in}}%
\pgfpathlineto{\pgfqpoint{1.283109in}{0.858031in}}%
\pgfpathmoveto{\pgfqpoint{1.303198in}{0.744504in}}%
\pgfpathlineto{\pgfqpoint{1.216169in}{0.750966in}}%
\pgfpathmoveto{\pgfqpoint{1.169146in}{0.688366in}}%
\pgfpathlineto{\pgfqpoint{1.147591in}{0.618472in}}%
\pgfpathmoveto{\pgfqpoint{1.169146in}{0.688366in}}%
\pgfpathlineto{\pgfqpoint{1.229103in}{0.635206in}}%
\pgfpathmoveto{\pgfqpoint{1.169146in}{0.688366in}}%
\pgfpathlineto{\pgfqpoint{1.091714in}{0.718622in}}%
\pgfpathmoveto{\pgfqpoint{1.169146in}{0.688366in}}%
\pgfpathlineto{\pgfqpoint{1.216169in}{0.750966in}}%
\pgfpathmoveto{\pgfqpoint{1.386091in}{0.854492in}}%
\pgfpathlineto{\pgfqpoint{1.489237in}{0.837484in}}%
\pgfpathmoveto{\pgfqpoint{1.386091in}{0.854492in}}%
\pgfpathlineto{\pgfqpoint{1.443404in}{0.759511in}}%
\pgfpathmoveto{\pgfqpoint{1.386091in}{0.854492in}}%
\pgfpathlineto{\pgfqpoint{1.283109in}{0.858031in}}%
\pgfpathmoveto{\pgfqpoint{1.386091in}{0.854492in}}%
\pgfpathlineto{\pgfqpoint{1.369911in}{0.959364in}}%
\pgfpathmoveto{\pgfqpoint{0.739504in}{0.707019in}}%
\pgfpathlineto{\pgfqpoint{0.666513in}{0.658314in}}%
\pgfpathmoveto{\pgfqpoint{0.739504in}{0.707019in}}%
\pgfpathlineto{\pgfqpoint{0.752247in}{0.632595in}}%
\pgfpathmoveto{\pgfqpoint{0.739504in}{0.707019in}}%
\pgfpathlineto{\pgfqpoint{0.819793in}{0.728687in}}%
\pgfpathmoveto{\pgfqpoint{0.739504in}{0.707019in}}%
\pgfpathlineto{\pgfqpoint{0.708591in}{0.780276in}}%
\pgfpathmoveto{\pgfqpoint{0.624324in}{0.786217in}}%
\pgfpathlineto{\pgfqpoint{0.539694in}{0.759689in}}%
\pgfpathmoveto{\pgfqpoint{0.624324in}{0.786217in}}%
\pgfpathlineto{\pgfqpoint{0.597797in}{0.697728in}}%
\pgfpathmoveto{\pgfqpoint{0.624324in}{0.786217in}}%
\pgfpathlineto{\pgfqpoint{0.666415in}{0.893785in}}%
\pgfpathmoveto{\pgfqpoint{0.624324in}{0.786217in}}%
\pgfpathlineto{\pgfqpoint{0.708591in}{0.780276in}}%
\pgfpathmoveto{\pgfqpoint{0.566652in}{0.964750in}}%
\pgfpathlineto{\pgfqpoint{0.454502in}{1.085600in}}%
\pgfpathmoveto{\pgfqpoint{0.566652in}{0.964750in}}%
\pgfpathlineto{\pgfqpoint{0.493932in}{0.837819in}}%
\pgfpathmoveto{\pgfqpoint{0.566652in}{0.964750in}}%
\pgfpathlineto{\pgfqpoint{0.723215in}{1.085600in}}%
\pgfpathmoveto{\pgfqpoint{0.566652in}{0.964750in}}%
\pgfpathlineto{\pgfqpoint{0.666415in}{0.893785in}}%
\pgfpathmoveto{\pgfqpoint{1.025791in}{0.660878in}}%
\pgfpathlineto{\pgfqpoint{0.990153in}{0.598729in}}%
\pgfpathmoveto{\pgfqpoint{1.025791in}{0.660878in}}%
\pgfpathlineto{\pgfqpoint{1.067159in}{0.607466in}}%
\pgfpathmoveto{\pgfqpoint{1.025791in}{0.660878in}}%
\pgfpathlineto{\pgfqpoint{0.960383in}{0.754944in}}%
\pgfpathmoveto{\pgfqpoint{1.025791in}{0.660878in}}%
\pgfpathlineto{\pgfqpoint{1.091714in}{0.718622in}}%
\pgfpathmoveto{\pgfqpoint{1.025791in}{0.660878in}}%
\pgfpathlineto{\pgfqpoint{0.953812in}{0.656912in}}%
\pgfpathmoveto{\pgfqpoint{0.882004in}{0.665729in}}%
\pgfpathlineto{\pgfqpoint{0.833592in}{0.614513in}}%
\pgfpathmoveto{\pgfqpoint{0.882004in}{0.665729in}}%
\pgfpathlineto{\pgfqpoint{0.912544in}{0.600941in}}%
\pgfpathmoveto{\pgfqpoint{0.882004in}{0.665729in}}%
\pgfpathlineto{\pgfqpoint{0.960383in}{0.754944in}}%
\pgfpathmoveto{\pgfqpoint{0.882004in}{0.665729in}}%
\pgfpathlineto{\pgfqpoint{0.819793in}{0.728687in}}%
\pgfpathmoveto{\pgfqpoint{0.882004in}{0.665729in}}%
\pgfpathlineto{\pgfqpoint{0.953812in}{0.656912in}}%
\pgfpathmoveto{\pgfqpoint{1.367508in}{0.777446in}}%
\pgfpathlineto{\pgfqpoint{1.383298in}{0.696947in}}%
\pgfpathmoveto{\pgfqpoint{1.367508in}{0.777446in}}%
\pgfpathlineto{\pgfqpoint{1.443404in}{0.759511in}}%
\pgfpathmoveto{\pgfqpoint{1.367508in}{0.777446in}}%
\pgfpathlineto{\pgfqpoint{1.283109in}{0.858031in}}%
\pgfpathmoveto{\pgfqpoint{1.367508in}{0.777446in}}%
\pgfpathlineto{\pgfqpoint{1.303198in}{0.744504in}}%
\pgfpathmoveto{\pgfqpoint{1.367508in}{0.777446in}}%
\pgfpathlineto{\pgfqpoint{1.386091in}{0.854492in}}%
\pgfpathmoveto{\pgfqpoint{1.015772in}{0.905746in}}%
\pgfpathlineto{\pgfqpoint{0.991929in}{1.085600in}}%
\pgfpathmoveto{\pgfqpoint{1.015772in}{0.905746in}}%
\pgfpathlineto{\pgfqpoint{1.116418in}{0.824542in}}%
\pgfpathmoveto{\pgfqpoint{1.015772in}{0.905746in}}%
\pgfpathlineto{\pgfqpoint{0.831595in}{0.886984in}}%
\pgfpathmoveto{\pgfqpoint{1.015772in}{0.905746in}}%
\pgfpathlineto{\pgfqpoint{0.960383in}{0.754944in}}%
\pgfpathmoveto{\pgfqpoint{1.174985in}{0.953328in}}%
\pgfpathlineto{\pgfqpoint{0.991929in}{1.085600in}}%
\pgfpathmoveto{\pgfqpoint{1.174985in}{0.953328in}}%
\pgfpathlineto{\pgfqpoint{1.260642in}{1.085600in}}%
\pgfpathmoveto{\pgfqpoint{1.174985in}{0.953328in}}%
\pgfpathlineto{\pgfqpoint{1.116418in}{0.824542in}}%
\pgfpathmoveto{\pgfqpoint{1.174985in}{0.953328in}}%
\pgfpathlineto{\pgfqpoint{1.283109in}{0.858031in}}%
\pgfpathmoveto{\pgfqpoint{1.174985in}{0.953328in}}%
\pgfpathlineto{\pgfqpoint{1.369911in}{0.959364in}}%
\pgfpathmoveto{\pgfqpoint{1.174985in}{0.953328in}}%
\pgfpathlineto{\pgfqpoint{1.015772in}{0.905746in}}%
\pgfpathmoveto{\pgfqpoint{0.576803in}{0.848301in}}%
\pgfpathlineto{\pgfqpoint{0.493932in}{0.837819in}}%
\pgfpathmoveto{\pgfqpoint{0.576803in}{0.848301in}}%
\pgfpathlineto{\pgfqpoint{0.539694in}{0.759689in}}%
\pgfpathmoveto{\pgfqpoint{0.576803in}{0.848301in}}%
\pgfpathlineto{\pgfqpoint{0.666415in}{0.893785in}}%
\pgfpathmoveto{\pgfqpoint{0.576803in}{0.848301in}}%
\pgfpathlineto{\pgfqpoint{0.624324in}{0.786217in}}%
\pgfpathmoveto{\pgfqpoint{0.576803in}{0.848301in}}%
\pgfpathlineto{\pgfqpoint{0.566652in}{0.964750in}}%
\pgfpathlineto{\pgfqpoint{0.566652in}{0.964750in}}%
\pgfusepath{stroke}%
\end{pgfscope}%
\begin{pgfscope}%
\pgfpathrectangle{\pgfqpoint{0.100000in}{0.100000in}}{\pgfqpoint{1.782500in}{1.232000in}}%
\pgfusepath{clip}%
\pgfsetbuttcap%
\pgfsetroundjoin%
\definecolor{currentfill}{rgb}{0.054902,0.262745,0.486275}%
\pgfsetfillcolor{currentfill}%
\pgfsetlinewidth{1.003750pt}%
\definecolor{currentstroke}{rgb}{0.054902,0.262745,0.486275}%
\pgfsetstrokecolor{currentstroke}%
\pgfsetdash{}{0pt}%
\pgfsys@defobject{currentmarker}{\pgfqpoint{-0.018373in}{-0.018373in}}{\pgfqpoint{0.018373in}{0.018373in}}{%
\pgfpathmoveto{\pgfqpoint{0.000000in}{-0.018373in}}%
\pgfpathcurveto{\pgfqpoint{0.004873in}{-0.018373in}}{\pgfqpoint{0.009546in}{-0.016437in}}{\pgfqpoint{0.012992in}{-0.012992in}}%
\pgfpathcurveto{\pgfqpoint{0.016437in}{-0.009546in}}{\pgfqpoint{0.018373in}{-0.004873in}}{\pgfqpoint{0.018373in}{0.000000in}}%
\pgfpathcurveto{\pgfqpoint{0.018373in}{0.004873in}}{\pgfqpoint{0.016437in}{0.009546in}}{\pgfqpoint{0.012992in}{0.012992in}}%
\pgfpathcurveto{\pgfqpoint{0.009546in}{0.016437in}}{\pgfqpoint{0.004873in}{0.018373in}}{\pgfqpoint{0.000000in}{0.018373in}}%
\pgfpathcurveto{\pgfqpoint{-0.004873in}{0.018373in}}{\pgfqpoint{-0.009546in}{0.016437in}}{\pgfqpoint{-0.012992in}{0.012992in}}%
\pgfpathcurveto{\pgfqpoint{-0.016437in}{0.009546in}}{\pgfqpoint{-0.018373in}{0.004873in}}{\pgfqpoint{-0.018373in}{0.000000in}}%
\pgfpathcurveto{\pgfqpoint{-0.018373in}{-0.004873in}}{\pgfqpoint{-0.016437in}{-0.009546in}}{\pgfqpoint{-0.012992in}{-0.012992in}}%
\pgfpathcurveto{\pgfqpoint{-0.009546in}{-0.016437in}}{\pgfqpoint{-0.004873in}{-0.018373in}}{\pgfqpoint{0.000000in}{-0.018373in}}%
\pgfpathlineto{\pgfqpoint{0.000000in}{-0.018373in}}%
\pgfpathclose%
\pgfusepath{stroke,fill}%
}%
\begin{pgfscope}%
\pgfsys@transformshift{1.310324in}{0.660296in}%
\pgfsys@useobject{currentmarker}{}%
\end{pgfscope}%
\begin{pgfscope}%
\pgfsys@transformshift{1.253051in}{0.642085in}%
\pgfsys@useobject{currentmarker}{}%
\end{pgfscope}%
\begin{pgfscope}%
\pgfsys@transformshift{1.188115in}{0.625601in}%
\pgfsys@useobject{currentmarker}{}%
\end{pgfscope}%
\begin{pgfscope}%
\pgfsys@transformshift{1.127651in}{0.615595in}%
\pgfsys@useobject{currentmarker}{}%
\end{pgfscope}%
\begin{pgfscope}%
\pgfsys@transformshift{1.058490in}{0.606300in}%
\pgfsys@useobject{currentmarker}{}%
\end{pgfscope}%
\begin{pgfscope}%
\pgfsys@transformshift{0.990122in}{0.598971in}%
\pgfsys@useobject{currentmarker}{}%
\end{pgfscope}%
\begin{pgfscope}%
\pgfsys@transformshift{0.921265in}{0.600292in}%
\pgfsys@useobject{currentmarker}{}%
\end{pgfscope}%
\begin{pgfscope}%
\pgfsys@transformshift{0.853138in}{0.610721in}%
\pgfsys@useobject{currentmarker}{}%
\end{pgfscope}%
\begin{pgfscope}%
\pgfsys@transformshift{0.793331in}{0.622693in}%
\pgfsys@useobject{currentmarker}{}%
\end{pgfscope}%
\begin{pgfscope}%
\pgfsys@transformshift{0.728172in}{0.639388in}%
\pgfsys@useobject{currentmarker}{}%
\end{pgfscope}%
\begin{pgfscope}%
\pgfsys@transformshift{0.670517in}{0.657059in}%
\pgfsys@useobject{currentmarker}{}%
\end{pgfscope}%
\end{pgfscope}%
\begin{pgfscope}%
\pgfpathrectangle{\pgfqpoint{0.100000in}{0.100000in}}{\pgfqpoint{1.782500in}{1.232000in}}%
\pgfusepath{clip}%
\pgfsetbuttcap%
\pgfsetroundjoin%
\definecolor{currentfill}{rgb}{0.835294,0.321569,0.035294}%
\pgfsetfillcolor{currentfill}%
\pgfsetlinewidth{1.003750pt}%
\definecolor{currentstroke}{rgb}{0.835294,0.321569,0.035294}%
\pgfsetstrokecolor{currentstroke}%
\pgfsetdash{}{0pt}%
\pgfsys@defobject{currentmarker}{\pgfqpoint{-0.018373in}{-0.018373in}}{\pgfqpoint{0.018373in}{0.018373in}}{%
\pgfpathmoveto{\pgfqpoint{0.000000in}{-0.018373in}}%
\pgfpathcurveto{\pgfqpoint{0.004873in}{-0.018373in}}{\pgfqpoint{0.009546in}{-0.016437in}}{\pgfqpoint{0.012992in}{-0.012992in}}%
\pgfpathcurveto{\pgfqpoint{0.016437in}{-0.009546in}}{\pgfqpoint{0.018373in}{-0.004873in}}{\pgfqpoint{0.018373in}{0.000000in}}%
\pgfpathcurveto{\pgfqpoint{0.018373in}{0.004873in}}{\pgfqpoint{0.016437in}{0.009546in}}{\pgfqpoint{0.012992in}{0.012992in}}%
\pgfpathcurveto{\pgfqpoint{0.009546in}{0.016437in}}{\pgfqpoint{0.004873in}{0.018373in}}{\pgfqpoint{0.000000in}{0.018373in}}%
\pgfpathcurveto{\pgfqpoint{-0.004873in}{0.018373in}}{\pgfqpoint{-0.009546in}{0.016437in}}{\pgfqpoint{-0.012992in}{0.012992in}}%
\pgfpathcurveto{\pgfqpoint{-0.016437in}{0.009546in}}{\pgfqpoint{-0.018373in}{0.004873in}}{\pgfqpoint{-0.018373in}{0.000000in}}%
\pgfpathcurveto{\pgfqpoint{-0.018373in}{-0.004873in}}{\pgfqpoint{-0.016437in}{-0.009546in}}{\pgfqpoint{-0.012992in}{-0.012992in}}%
\pgfpathcurveto{\pgfqpoint{-0.009546in}{-0.016437in}}{\pgfqpoint{-0.004873in}{-0.018373in}}{\pgfqpoint{0.000000in}{-0.018373in}}%
\pgfpathlineto{\pgfqpoint{0.000000in}{-0.018373in}}%
\pgfpathclose%
\pgfusepath{stroke,fill}%
}%
\begin{pgfscope}%
\pgfsys@transformshift{0.666513in}{0.658314in}%
\pgfsys@useobject{currentmarker}{}%
\end{pgfscope}%
\begin{pgfscope}%
\pgfsys@transformshift{0.752247in}{0.632595in}%
\pgfsys@useobject{currentmarker}{}%
\end{pgfscope}%
\begin{pgfscope}%
\pgfsys@transformshift{0.833592in}{0.614513in}%
\pgfsys@useobject{currentmarker}{}%
\end{pgfscope}%
\begin{pgfscope}%
\pgfsys@transformshift{0.912544in}{0.600941in}%
\pgfsys@useobject{currentmarker}{}%
\end{pgfscope}%
\begin{pgfscope}%
\pgfsys@transformshift{0.990153in}{0.598729in}%
\pgfsys@useobject{currentmarker}{}%
\end{pgfscope}%
\begin{pgfscope}%
\pgfsys@transformshift{1.067159in}{0.607466in}%
\pgfsys@useobject{currentmarker}{}%
\end{pgfscope}%
\begin{pgfscope}%
\pgfsys@transformshift{1.147591in}{0.618472in}%
\pgfsys@useobject{currentmarker}{}%
\end{pgfscope}%
\begin{pgfscope}%
\pgfsys@transformshift{1.229103in}{0.635206in}%
\pgfsys@useobject{currentmarker}{}%
\end{pgfscope}%
\begin{pgfscope}%
\pgfsys@transformshift{1.314300in}{0.661595in}%
\pgfsys@useobject{currentmarker}{}%
\end{pgfscope}%
\end{pgfscope}%
\begin{pgfscope}%
\pgfpathrectangle{\pgfqpoint{0.100000in}{0.100000in}}{\pgfqpoint{1.782500in}{1.232000in}}%
\pgfusepath{clip}%
\pgfsetbuttcap%
\pgfsetroundjoin%
\pgfsetlinewidth{1.003750pt}%
\definecolor{currentstroke}{rgb}{0.054902,0.262745,0.486275}%
\pgfsetstrokecolor{currentstroke}%
\pgfsetdash{}{0pt}%
\pgfpathmoveto{\pgfqpoint{0.000000in}{-0.018373in}}%
\pgfpathcurveto{\pgfqpoint{0.004873in}{-0.018373in}}{\pgfqpoint{0.009546in}{-0.016437in}}{\pgfqpoint{0.012992in}{-0.012992in}}%
\pgfpathcurveto{\pgfqpoint{0.016437in}{-0.009546in}}{\pgfqpoint{0.018373in}{-0.004873in}}{\pgfqpoint{0.018373in}{0.000000in}}%
\pgfpathcurveto{\pgfqpoint{0.018373in}{0.004873in}}{\pgfqpoint{0.016437in}{0.009546in}}{\pgfqpoint{0.012992in}{0.012992in}}%
\pgfpathcurveto{\pgfqpoint{0.009546in}{0.016437in}}{\pgfqpoint{0.004873in}{0.018373in}}{\pgfqpoint{0.000000in}{0.018373in}}%
\pgfpathcurveto{\pgfqpoint{-0.004873in}{0.018373in}}{\pgfqpoint{-0.009546in}{0.016437in}}{\pgfqpoint{-0.012992in}{0.012992in}}%
\pgfpathcurveto{\pgfqpoint{-0.016437in}{0.009546in}}{\pgfqpoint{-0.018373in}{0.004873in}}{\pgfqpoint{-0.018373in}{0.000000in}}%
\pgfpathcurveto{\pgfqpoint{-0.018373in}{-0.004873in}}{\pgfqpoint{-0.016437in}{-0.009546in}}{\pgfqpoint{-0.012992in}{-0.012992in}}%
\pgfpathcurveto{\pgfqpoint{-0.009546in}{-0.016437in}}{\pgfqpoint{-0.004873in}{-0.018373in}}{\pgfqpoint{0.000000in}{-0.018373in}}%
\pgfusepath{stroke}%
\end{pgfscope}%
\begin{pgfscope}%
\pgfpathrectangle{\pgfqpoint{0.100000in}{0.100000in}}{\pgfqpoint{1.782500in}{1.232000in}}%
\pgfusepath{clip}%
\pgfsetbuttcap%
\pgfsetroundjoin%
\pgfsetlinewidth{1.003750pt}%
\definecolor{currentstroke}{rgb}{0.835294,0.321569,0.035294}%
\pgfsetstrokecolor{currentstroke}%
\pgfsetdash{}{0pt}%
\pgfpathmoveto{\pgfqpoint{0.000000in}{-0.018373in}}%
\pgfpathcurveto{\pgfqpoint{0.004873in}{-0.018373in}}{\pgfqpoint{0.009546in}{-0.016437in}}{\pgfqpoint{0.012992in}{-0.012992in}}%
\pgfpathcurveto{\pgfqpoint{0.016437in}{-0.009546in}}{\pgfqpoint{0.018373in}{-0.004873in}}{\pgfqpoint{0.018373in}{0.000000in}}%
\pgfpathcurveto{\pgfqpoint{0.018373in}{0.004873in}}{\pgfqpoint{0.016437in}{0.009546in}}{\pgfqpoint{0.012992in}{0.012992in}}%
\pgfpathcurveto{\pgfqpoint{0.009546in}{0.016437in}}{\pgfqpoint{0.004873in}{0.018373in}}{\pgfqpoint{0.000000in}{0.018373in}}%
\pgfpathcurveto{\pgfqpoint{-0.004873in}{0.018373in}}{\pgfqpoint{-0.009546in}{0.016437in}}{\pgfqpoint{-0.012992in}{0.012992in}}%
\pgfpathcurveto{\pgfqpoint{-0.016437in}{0.009546in}}{\pgfqpoint{-0.018373in}{0.004873in}}{\pgfqpoint{-0.018373in}{0.000000in}}%
\pgfpathcurveto{\pgfqpoint{-0.018373in}{-0.004873in}}{\pgfqpoint{-0.016437in}{-0.009546in}}{\pgfqpoint{-0.012992in}{-0.012992in}}%
\pgfpathcurveto{\pgfqpoint{-0.009546in}{-0.016437in}}{\pgfqpoint{-0.004873in}{-0.018373in}}{\pgfqpoint{0.000000in}{-0.018373in}}%
\pgfusepath{stroke}%
\end{pgfscope}%
\begin{pgfscope}%
\pgfpathrectangle{\pgfqpoint{0.100000in}{0.100000in}}{\pgfqpoint{1.782500in}{1.232000in}}%
\pgfusepath{clip}%
\pgfsetbuttcap%
\pgfsetroundjoin%
\definecolor{currentfill}{rgb}{0.054902,0.262745,0.486275}%
\pgfsetfillcolor{currentfill}%
\pgfsetlinewidth{1.505625pt}%
\definecolor{currentstroke}{rgb}{0.054902,0.262745,0.486275}%
\pgfsetstrokecolor{currentstroke}%
\pgfsetdash{}{0pt}%
\pgfsys@defobject{currentmarker}{\pgfqpoint{-0.018373in}{-0.018373in}}{\pgfqpoint{0.018373in}{0.018373in}}{%
\pgfpathmoveto{\pgfqpoint{-0.018373in}{-0.018373in}}%
\pgfpathlineto{\pgfqpoint{0.018373in}{0.018373in}}%
\pgfpathmoveto{\pgfqpoint{-0.018373in}{0.018373in}}%
\pgfpathlineto{\pgfqpoint{0.018373in}{-0.018373in}}%
\pgfusepath{stroke,fill}%
}%
\end{pgfscope}%
\begin{pgfscope}%
\pgfpathrectangle{\pgfqpoint{0.100000in}{0.100000in}}{\pgfqpoint{1.782500in}{1.232000in}}%
\pgfusepath{clip}%
\pgfsetbuttcap%
\pgfsetroundjoin%
\definecolor{currentfill}{rgb}{0.835294,0.321569,0.035294}%
\pgfsetfillcolor{currentfill}%
\pgfsetlinewidth{1.505625pt}%
\definecolor{currentstroke}{rgb}{0.835294,0.321569,0.035294}%
\pgfsetstrokecolor{currentstroke}%
\pgfsetdash{}{0pt}%
\pgfsys@defobject{currentmarker}{\pgfqpoint{-0.018373in}{-0.018373in}}{\pgfqpoint{0.018373in}{0.018373in}}{%
\pgfpathmoveto{\pgfqpoint{-0.018373in}{-0.018373in}}%
\pgfpathlineto{\pgfqpoint{0.018373in}{0.018373in}}%
\pgfpathmoveto{\pgfqpoint{-0.018373in}{0.018373in}}%
\pgfpathlineto{\pgfqpoint{0.018373in}{-0.018373in}}%
\pgfusepath{stroke,fill}%
}%
\end{pgfscope}%
\end{pgfpicture}%
\makeatother%
\endgroup%

        \caption{Iteration 3: Update interface}\label{fig:example-iter2-dumping}
    \end{subfigure}
    \caption{First iteration}
    \label{fig:example-iter2}
\end{figure}

\subsection{Strong form}
\label{subsec:strong}

The strong form of the problem is formulated for the displacement field $\mathbf{u}$:
\begin{align}
    \begin{cases}
        - \text{div}(\boldsymbol{\sigma}(\mathbf{u})) = \mathbf{f} &\text{Differential equation (Cauchy stress tensor $\boldsymbol{\sigma}$)}\\
        \mathbf{u} = \mathbf{g} &\text{Dirichlet boundary conditions (displacement field $\mathbf{g}$)}\\
        \boldsymbol{\sigma}(\mathbf{u}) \mathbf{n} = \mathbf{t} &\text{Neumann boundary conditions (surface traction $\mathbf{t}$)}\\
        \boldsymbol{\sigma}(\mathbf{u}) \mathbf{n} = \boldsymbol{\lambda} &\text{Lagrange multipliers $\boldsymbol{\lambda}$ defined along interface $\Gamma$}\\
        \boldsymbol{\lambda} \cdot \mathbf{n} \leq 0 &\text{Hertz-Signorini-Moreau condition enforced along interface $\Gamma$}
    \end{cases}
\end{align}

\subsection{Weak form}
\label{subsec:weak}

The weak formulation results in a linear system for the displacements $\mathbf{u}$ and Lagrange multipliers $\boldsymbol{\lambda}$ \cite{voet}:
\begin{equation}
\underbrace{
\begin{bmatrix}
    \mathbf{K}_{\Omega_1\Omega_1} & \mathbf{K}_{\Omega_1\Gamma_1} & \boldsymbol{0} & \boldsymbol{0} & \boldsymbol{0} \\
    \mathbf{K}_{\Gamma_1\Omega_1} & \mathbf{K}_{\Gamma_1\Gamma_1} & \boldsymbol{0} & \boldsymbol{0} & - \mathbf{M}_{\Gamma_1} \\
    \boldsymbol{0} & \boldsymbol{0} & \mathbf{K}_{\Omega_2\Omega_2} & \mathbf{K}_{\Omega_2\Gamma_2} & \boldsymbol{0} \\
    \boldsymbol{0} & \boldsymbol{0} & \mathbf{K}_{\Gamma_2\Omega_2} & \mathbf{K}_{\Gamma_2\Gamma_2} & -\mathbf{M}_{\Gamma_2} \mathbf{R}_{\Gamma_2 \Gamma_1} \\
    \boldsymbol{0} & \boldsymbol{I} & \boldsymbol{0} & -\mathbf{R}_{\Gamma_1 \Gamma_2} & \boldsymbol{0}   
\end{bmatrix}
}_{\mathbf{A}~\text{(INTERNODES matrix)}}
\underbrace{
\begin{bmatrix}
    \mathbf{u}_{\Omega_1} \\
    \mathbf{u}_{\Gamma_1} \\
    \mathbf{u}_{\Omega_2} \\
    \mathbf{u}_{\Gamma_2} \\
    \boldsymbol{\lambda} \\
\end{bmatrix}
}_{\mathbf{x}}
=
\underbrace{
\begin{bmatrix}
    \mathbf{f}_{\Omega_1} \\
    \mathbf{f}_{\Gamma_1} \\
    \mathbf{f}_{\Omega_2} \\
    \mathbf{f}_{\Gamma_2} \\
    \mathbf{d} \\
\end{bmatrix}
}_{\mathbf{b}}
\label{equ:linear-system}
\end{equation}
\begin{align}
    \mathbf{M} &:~ \text{Interface mass matrices} \\
    \mathbf{K} &:~ \text{Stiffness matrices} \\
    \mathbf{f} &:~ \text{Body force} \\
    \mathbf{d} &:~ \text{Nodal gaps in configuration}
\end{align}

\subsection{Contact algorithm}
\label{subsec:contact-algorithm}

\begin{algorithm}
    \caption{Contact algorithm for interrnodes method}
    \begin{algorithmic}[1]
    \Require Positions of primary nodes $\{\boldsymbol{\xi}_1, \boldsymbol{\xi}_2, \dots\}$
    \Require Positions of secondary nodes $\{\boldsymbol{\zeta}_1, \boldsymbol{\zeta}_2, \dots\}$
    \Require Interface candidate index sets $\mathcal{I}^C$ and $\mathcal{J}^C$
    \While{$\mathcal{I}^C$ or $\mathcal{J}^C$ were modified in the previous iteration}
        \State Determine interface nodes $\mathcal{I}$ and $\mathcal{J}$ with radius parameters $r^{\boldsymbol{\xi}}_i, i \in \mathcal{I}$
        and $r^{\boldsymbol{\zeta}}_j, j \in \mathcal{J}$  using \refalg{alg:nodesearch} on the candidate index sets $\mathcal{I}^C$ and $\mathcal{J}^C$
        \State Assemble the matrix $\mathbf{A}$ and right-hand side $\mathbf{b}$ of \refequ{equ:linear-system}
        \State Solve \refequ{equ:linear-system} to obtain displacements $\mathbf{u}$ and Lagarnge multipliers $\boldsymbol{\lambda}$
        \State Update the interface candidate nodes $\mathcal{I}^C$ and $\mathcal{J}^C$ with
        $\mathcal{I}$ and $\mathcal{J}$, respectively, where all nodes in tension
        have been removed and all interpenetrating nodes have been added 
    \EndWhile
\end{algorithmic}
    \label{alg:internodes}
\end{algorithm}


\section{Experiments}
\label{sec:experiments}

\begin{figure}[ht]
   \centering
    %% Creator: Matplotlib, PGF backend
%%
%% To include the figure in your LaTeX document, write
%%   \input{<filename>.pgf}
%%
%% Make sure the required packages are loaded in your preamble
%%   \usepackage{pgf}
%%
%% Also ensure that all the required font packages are loaded; for instance,
%% the lmodern package is sometimes necessary when using math font.
%%   \usepackage{lmodern}
%%
%% Figures using additional raster images can only be included by \input if
%% they are in the same directory as the main LaTeX file. For loading figures
%% from other directories you can use the `import` package
%%   \usepackage{import}
%%
%% and then include the figures with
%%   \import{<path to file>}{<filename>.pgf}
%%
%% Matplotlib used the following preamble
%%   
%%   \usepackage{fontspec}
%%   \setmainfont{DejaVuSans.ttf}[Path=\detokenize{/home/fabio/Internodes-CM/.venv/lib/python3.8/site-packages/matplotlib/mpl-data/fonts/ttf/}]
%%   \setsansfont{DejaVuSans.ttf}[Path=\detokenize{/home/fabio/Internodes-CM/.venv/lib/python3.8/site-packages/matplotlib/mpl-data/fonts/ttf/}]
%%   \setmonofont{DejaVuSansMono.ttf}[Path=\detokenize{/home/fabio/Internodes-CM/.venv/lib/python3.8/site-packages/matplotlib/mpl-data/fonts/ttf/}]
%%   \makeatletter\@ifpackageloaded{underscore}{}{\usepackage[strings]{underscore}}\makeatother
%%
\begingroup%
\makeatletter%
\begin{pgfpicture}%
\pgfpathrectangle{\pgfpointorigin}{\pgfqpoint{5.926790in}{3.343486in}}%
\pgfusepath{use as bounding box, clip}%
\begin{pgfscope}%
\pgfsetbuttcap%
\pgfsetmiterjoin%
\definecolor{currentfill}{rgb}{1.000000,1.000000,1.000000}%
\pgfsetfillcolor{currentfill}%
\pgfsetlinewidth{0.000000pt}%
\definecolor{currentstroke}{rgb}{1.000000,1.000000,1.000000}%
\pgfsetstrokecolor{currentstroke}%
\pgfsetdash{}{0pt}%
\pgfpathmoveto{\pgfqpoint{0.000000in}{0.000000in}}%
\pgfpathlineto{\pgfqpoint{5.926790in}{0.000000in}}%
\pgfpathlineto{\pgfqpoint{5.926790in}{3.343486in}}%
\pgfpathlineto{\pgfqpoint{0.000000in}{3.343486in}}%
\pgfpathlineto{\pgfqpoint{0.000000in}{0.000000in}}%
\pgfpathclose%
\pgfusepath{fill}%
\end{pgfscope}%
\begin{pgfscope}%
\pgfsetbuttcap%
\pgfsetmiterjoin%
\definecolor{currentfill}{rgb}{1.000000,1.000000,1.000000}%
\pgfsetfillcolor{currentfill}%
\pgfsetlinewidth{0.000000pt}%
\definecolor{currentstroke}{rgb}{0.000000,0.000000,0.000000}%
\pgfsetstrokecolor{currentstroke}%
\pgfsetstrokeopacity{0.000000}%
\pgfsetdash{}{0pt}%
\pgfpathmoveto{\pgfqpoint{0.789290in}{0.548486in}}%
\pgfpathlineto{\pgfqpoint{5.826790in}{0.548486in}}%
\pgfpathlineto{\pgfqpoint{5.826790in}{3.243486in}}%
\pgfpathlineto{\pgfqpoint{0.789290in}{3.243486in}}%
\pgfpathlineto{\pgfqpoint{0.789290in}{0.548486in}}%
\pgfpathclose%
\pgfusepath{fill}%
\end{pgfscope}%
\begin{pgfscope}%
\pgfpathrectangle{\pgfqpoint{0.789290in}{0.548486in}}{\pgfqpoint{5.037500in}{2.695000in}}%
\pgfusepath{clip}%
\pgfsetbuttcap%
\pgfsetroundjoin%
\definecolor{currentfill}{rgb}{0.054902,0.262745,0.486275}%
\pgfsetfillcolor{currentfill}%
\pgfsetlinewidth{1.003750pt}%
\definecolor{currentstroke}{rgb}{0.054902,0.262745,0.486275}%
\pgfsetstrokecolor{currentstroke}%
\pgfsetdash{}{0pt}%
\pgfsys@defobject{currentmarker}{\pgfqpoint{-0.041667in}{-0.041667in}}{\pgfqpoint{0.041667in}{0.041667in}}{%
\pgfpathmoveto{\pgfqpoint{0.000000in}{-0.041667in}}%
\pgfpathcurveto{\pgfqpoint{0.011050in}{-0.041667in}}{\pgfqpoint{0.021649in}{-0.037276in}}{\pgfqpoint{0.029463in}{-0.029463in}}%
\pgfpathcurveto{\pgfqpoint{0.037276in}{-0.021649in}}{\pgfqpoint{0.041667in}{-0.011050in}}{\pgfqpoint{0.041667in}{0.000000in}}%
\pgfpathcurveto{\pgfqpoint{0.041667in}{0.011050in}}{\pgfqpoint{0.037276in}{0.021649in}}{\pgfqpoint{0.029463in}{0.029463in}}%
\pgfpathcurveto{\pgfqpoint{0.021649in}{0.037276in}}{\pgfqpoint{0.011050in}{0.041667in}}{\pgfqpoint{0.000000in}{0.041667in}}%
\pgfpathcurveto{\pgfqpoint{-0.011050in}{0.041667in}}{\pgfqpoint{-0.021649in}{0.037276in}}{\pgfqpoint{-0.029463in}{0.029463in}}%
\pgfpathcurveto{\pgfqpoint{-0.037276in}{0.021649in}}{\pgfqpoint{-0.041667in}{0.011050in}}{\pgfqpoint{-0.041667in}{0.000000in}}%
\pgfpathcurveto{\pgfqpoint{-0.041667in}{-0.011050in}}{\pgfqpoint{-0.037276in}{-0.021649in}}{\pgfqpoint{-0.029463in}{-0.029463in}}%
\pgfpathcurveto{\pgfqpoint{-0.021649in}{-0.037276in}}{\pgfqpoint{-0.011050in}{-0.041667in}}{\pgfqpoint{0.000000in}{-0.041667in}}%
\pgfpathlineto{\pgfqpoint{0.000000in}{-0.041667in}}%
\pgfpathclose%
\pgfusepath{stroke,fill}%
}%
\begin{pgfscope}%
\pgfsys@transformshift{1.293040in}{2.955747in}%
\pgfsys@useobject{currentmarker}{}%
\end{pgfscope}%
\begin{pgfscope}%
\pgfsys@transformshift{1.740818in}{2.784307in}%
\pgfsys@useobject{currentmarker}{}%
\end{pgfscope}%
\begin{pgfscope}%
\pgfsys@transformshift{2.188595in}{2.501776in}%
\pgfsys@useobject{currentmarker}{}%
\end{pgfscope}%
\begin{pgfscope}%
\pgfsys@transformshift{2.636373in}{2.247807in}%
\pgfsys@useobject{currentmarker}{}%
\end{pgfscope}%
\begin{pgfscope}%
\pgfsys@transformshift{3.084151in}{2.032879in}%
\pgfsys@useobject{currentmarker}{}%
\end{pgfscope}%
\begin{pgfscope}%
\pgfsys@transformshift{3.531929in}{1.783552in}%
\pgfsys@useobject{currentmarker}{}%
\end{pgfscope}%
\begin{pgfscope}%
\pgfsys@transformshift{3.979707in}{1.580801in}%
\pgfsys@useobject{currentmarker}{}%
\end{pgfscope}%
\begin{pgfscope}%
\pgfsys@transformshift{4.427484in}{1.362953in}%
\pgfsys@useobject{currentmarker}{}%
\end{pgfscope}%
\begin{pgfscope}%
\pgfsys@transformshift{4.875262in}{1.132455in}%
\pgfsys@useobject{currentmarker}{}%
\end{pgfscope}%
\begin{pgfscope}%
\pgfsys@transformshift{5.323040in}{0.917032in}%
\pgfsys@useobject{currentmarker}{}%
\end{pgfscope}%
\end{pgfscope}%
\begin{pgfscope}%
\pgfsetbuttcap%
\pgfsetroundjoin%
\definecolor{currentfill}{rgb}{0.000000,0.000000,0.000000}%
\pgfsetfillcolor{currentfill}%
\pgfsetlinewidth{0.803000pt}%
\definecolor{currentstroke}{rgb}{0.000000,0.000000,0.000000}%
\pgfsetstrokecolor{currentstroke}%
\pgfsetdash{}{0pt}%
\pgfsys@defobject{currentmarker}{\pgfqpoint{0.000000in}{-0.048611in}}{\pgfqpoint{0.000000in}{0.000000in}}{%
\pgfpathmoveto{\pgfqpoint{0.000000in}{0.000000in}}%
\pgfpathlineto{\pgfqpoint{0.000000in}{-0.048611in}}%
\pgfusepath{stroke,fill}%
}%
\begin{pgfscope}%
\pgfsys@transformshift{1.293040in}{0.548486in}%
\pgfsys@useobject{currentmarker}{}%
\end{pgfscope}%
\end{pgfscope}%
\begin{pgfscope}%
\definecolor{textcolor}{rgb}{0.000000,0.000000,0.000000}%
\pgfsetstrokecolor{textcolor}%
\pgfsetfillcolor{textcolor}%
\pgftext[x=1.293040in,y=0.451264in,,top]{\color{textcolor}\rmfamily\fontsize{11.000000}{13.200000}\selectfont \(\displaystyle {0.05}\)}%
\end{pgfscope}%
\begin{pgfscope}%
\pgfsetbuttcap%
\pgfsetroundjoin%
\definecolor{currentfill}{rgb}{0.000000,0.000000,0.000000}%
\pgfsetfillcolor{currentfill}%
\pgfsetlinewidth{0.803000pt}%
\definecolor{currentstroke}{rgb}{0.000000,0.000000,0.000000}%
\pgfsetstrokecolor{currentstroke}%
\pgfsetdash{}{0pt}%
\pgfsys@defobject{currentmarker}{\pgfqpoint{0.000000in}{-0.048611in}}{\pgfqpoint{0.000000in}{0.000000in}}{%
\pgfpathmoveto{\pgfqpoint{0.000000in}{0.000000in}}%
\pgfpathlineto{\pgfqpoint{0.000000in}{-0.048611in}}%
\pgfusepath{stroke,fill}%
}%
\begin{pgfscope}%
\pgfsys@transformshift{2.300540in}{0.548486in}%
\pgfsys@useobject{currentmarker}{}%
\end{pgfscope}%
\end{pgfscope}%
\begin{pgfscope}%
\definecolor{textcolor}{rgb}{0.000000,0.000000,0.000000}%
\pgfsetstrokecolor{textcolor}%
\pgfsetfillcolor{textcolor}%
\pgftext[x=2.300540in,y=0.451264in,,top]{\color{textcolor}\rmfamily\fontsize{11.000000}{13.200000}\selectfont \(\displaystyle {0.10}\)}%
\end{pgfscope}%
\begin{pgfscope}%
\pgfsetbuttcap%
\pgfsetroundjoin%
\definecolor{currentfill}{rgb}{0.000000,0.000000,0.000000}%
\pgfsetfillcolor{currentfill}%
\pgfsetlinewidth{0.803000pt}%
\definecolor{currentstroke}{rgb}{0.000000,0.000000,0.000000}%
\pgfsetstrokecolor{currentstroke}%
\pgfsetdash{}{0pt}%
\pgfsys@defobject{currentmarker}{\pgfqpoint{0.000000in}{-0.048611in}}{\pgfqpoint{0.000000in}{0.000000in}}{%
\pgfpathmoveto{\pgfqpoint{0.000000in}{0.000000in}}%
\pgfpathlineto{\pgfqpoint{0.000000in}{-0.048611in}}%
\pgfusepath{stroke,fill}%
}%
\begin{pgfscope}%
\pgfsys@transformshift{3.308040in}{0.548486in}%
\pgfsys@useobject{currentmarker}{}%
\end{pgfscope}%
\end{pgfscope}%
\begin{pgfscope}%
\definecolor{textcolor}{rgb}{0.000000,0.000000,0.000000}%
\pgfsetstrokecolor{textcolor}%
\pgfsetfillcolor{textcolor}%
\pgftext[x=3.308040in,y=0.451264in,,top]{\color{textcolor}\rmfamily\fontsize{11.000000}{13.200000}\selectfont \(\displaystyle {0.15}\)}%
\end{pgfscope}%
\begin{pgfscope}%
\pgfsetbuttcap%
\pgfsetroundjoin%
\definecolor{currentfill}{rgb}{0.000000,0.000000,0.000000}%
\pgfsetfillcolor{currentfill}%
\pgfsetlinewidth{0.803000pt}%
\definecolor{currentstroke}{rgb}{0.000000,0.000000,0.000000}%
\pgfsetstrokecolor{currentstroke}%
\pgfsetdash{}{0pt}%
\pgfsys@defobject{currentmarker}{\pgfqpoint{0.000000in}{-0.048611in}}{\pgfqpoint{0.000000in}{0.000000in}}{%
\pgfpathmoveto{\pgfqpoint{0.000000in}{0.000000in}}%
\pgfpathlineto{\pgfqpoint{0.000000in}{-0.048611in}}%
\pgfusepath{stroke,fill}%
}%
\begin{pgfscope}%
\pgfsys@transformshift{4.315540in}{0.548486in}%
\pgfsys@useobject{currentmarker}{}%
\end{pgfscope}%
\end{pgfscope}%
\begin{pgfscope}%
\definecolor{textcolor}{rgb}{0.000000,0.000000,0.000000}%
\pgfsetstrokecolor{textcolor}%
\pgfsetfillcolor{textcolor}%
\pgftext[x=4.315540in,y=0.451264in,,top]{\color{textcolor}\rmfamily\fontsize{11.000000}{13.200000}\selectfont \(\displaystyle {0.20}\)}%
\end{pgfscope}%
\begin{pgfscope}%
\pgfsetbuttcap%
\pgfsetroundjoin%
\definecolor{currentfill}{rgb}{0.000000,0.000000,0.000000}%
\pgfsetfillcolor{currentfill}%
\pgfsetlinewidth{0.803000pt}%
\definecolor{currentstroke}{rgb}{0.000000,0.000000,0.000000}%
\pgfsetstrokecolor{currentstroke}%
\pgfsetdash{}{0pt}%
\pgfsys@defobject{currentmarker}{\pgfqpoint{0.000000in}{-0.048611in}}{\pgfqpoint{0.000000in}{0.000000in}}{%
\pgfpathmoveto{\pgfqpoint{0.000000in}{0.000000in}}%
\pgfpathlineto{\pgfqpoint{0.000000in}{-0.048611in}}%
\pgfusepath{stroke,fill}%
}%
\begin{pgfscope}%
\pgfsys@transformshift{5.323040in}{0.548486in}%
\pgfsys@useobject{currentmarker}{}%
\end{pgfscope}%
\end{pgfscope}%
\begin{pgfscope}%
\definecolor{textcolor}{rgb}{0.000000,0.000000,0.000000}%
\pgfsetstrokecolor{textcolor}%
\pgfsetfillcolor{textcolor}%
\pgftext[x=5.323040in,y=0.451264in,,top]{\color{textcolor}\rmfamily\fontsize{11.000000}{13.200000}\selectfont \(\displaystyle {0.25}\)}%
\end{pgfscope}%
\begin{pgfscope}%
\definecolor{textcolor}{rgb}{0.000000,0.000000,0.000000}%
\pgfsetstrokecolor{textcolor}%
\pgfsetfillcolor{textcolor}%
\pgftext[x=3.308040in,y=0.247854in,,top]{\color{textcolor}\rmfamily\fontsize{11.000000}{13.200000}\selectfont \(\displaystyle d\)}%
\end{pgfscope}%
\begin{pgfscope}%
\pgfsetbuttcap%
\pgfsetroundjoin%
\definecolor{currentfill}{rgb}{0.000000,0.000000,0.000000}%
\pgfsetfillcolor{currentfill}%
\pgfsetlinewidth{0.803000pt}%
\definecolor{currentstroke}{rgb}{0.000000,0.000000,0.000000}%
\pgfsetstrokecolor{currentstroke}%
\pgfsetdash{}{0pt}%
\pgfsys@defobject{currentmarker}{\pgfqpoint{-0.048611in}{0.000000in}}{\pgfqpoint{-0.000000in}{0.000000in}}{%
\pgfpathmoveto{\pgfqpoint{-0.000000in}{0.000000in}}%
\pgfpathlineto{\pgfqpoint{-0.048611in}{0.000000in}}%
\pgfusepath{stroke,fill}%
}%
\begin{pgfscope}%
\pgfsys@transformshift{0.789290in}{0.621986in}%
\pgfsys@useobject{currentmarker}{}%
\end{pgfscope}%
\end{pgfscope}%
\begin{pgfscope}%
\definecolor{textcolor}{rgb}{0.000000,0.000000,0.000000}%
\pgfsetstrokecolor{textcolor}%
\pgfsetfillcolor{textcolor}%
\pgftext[x=0.303410in, y=0.563948in, left, base]{\color{textcolor}\rmfamily\fontsize{11.000000}{13.200000}\selectfont \(\displaystyle {\ensuremath{-}0.14}\)}%
\end{pgfscope}%
\begin{pgfscope}%
\pgfsetbuttcap%
\pgfsetroundjoin%
\definecolor{currentfill}{rgb}{0.000000,0.000000,0.000000}%
\pgfsetfillcolor{currentfill}%
\pgfsetlinewidth{0.803000pt}%
\definecolor{currentstroke}{rgb}{0.000000,0.000000,0.000000}%
\pgfsetstrokecolor{currentstroke}%
\pgfsetdash{}{0pt}%
\pgfsys@defobject{currentmarker}{\pgfqpoint{-0.048611in}{0.000000in}}{\pgfqpoint{-0.000000in}{0.000000in}}{%
\pgfpathmoveto{\pgfqpoint{-0.000000in}{0.000000in}}%
\pgfpathlineto{\pgfqpoint{-0.048611in}{0.000000in}}%
\pgfusepath{stroke,fill}%
}%
\begin{pgfscope}%
\pgfsys@transformshift{0.789290in}{1.013986in}%
\pgfsys@useobject{currentmarker}{}%
\end{pgfscope}%
\end{pgfscope}%
\begin{pgfscope}%
\definecolor{textcolor}{rgb}{0.000000,0.000000,0.000000}%
\pgfsetstrokecolor{textcolor}%
\pgfsetfillcolor{textcolor}%
\pgftext[x=0.303410in, y=0.955948in, left, base]{\color{textcolor}\rmfamily\fontsize{11.000000}{13.200000}\selectfont \(\displaystyle {\ensuremath{-}0.12}\)}%
\end{pgfscope}%
\begin{pgfscope}%
\pgfsetbuttcap%
\pgfsetroundjoin%
\definecolor{currentfill}{rgb}{0.000000,0.000000,0.000000}%
\pgfsetfillcolor{currentfill}%
\pgfsetlinewidth{0.803000pt}%
\definecolor{currentstroke}{rgb}{0.000000,0.000000,0.000000}%
\pgfsetstrokecolor{currentstroke}%
\pgfsetdash{}{0pt}%
\pgfsys@defobject{currentmarker}{\pgfqpoint{-0.048611in}{0.000000in}}{\pgfqpoint{-0.000000in}{0.000000in}}{%
\pgfpathmoveto{\pgfqpoint{-0.000000in}{0.000000in}}%
\pgfpathlineto{\pgfqpoint{-0.048611in}{0.000000in}}%
\pgfusepath{stroke,fill}%
}%
\begin{pgfscope}%
\pgfsys@transformshift{0.789290in}{1.405986in}%
\pgfsys@useobject{currentmarker}{}%
\end{pgfscope}%
\end{pgfscope}%
\begin{pgfscope}%
\definecolor{textcolor}{rgb}{0.000000,0.000000,0.000000}%
\pgfsetstrokecolor{textcolor}%
\pgfsetfillcolor{textcolor}%
\pgftext[x=0.303410in, y=1.347948in, left, base]{\color{textcolor}\rmfamily\fontsize{11.000000}{13.200000}\selectfont \(\displaystyle {\ensuremath{-}0.10}\)}%
\end{pgfscope}%
\begin{pgfscope}%
\pgfsetbuttcap%
\pgfsetroundjoin%
\definecolor{currentfill}{rgb}{0.000000,0.000000,0.000000}%
\pgfsetfillcolor{currentfill}%
\pgfsetlinewidth{0.803000pt}%
\definecolor{currentstroke}{rgb}{0.000000,0.000000,0.000000}%
\pgfsetstrokecolor{currentstroke}%
\pgfsetdash{}{0pt}%
\pgfsys@defobject{currentmarker}{\pgfqpoint{-0.048611in}{0.000000in}}{\pgfqpoint{-0.000000in}{0.000000in}}{%
\pgfpathmoveto{\pgfqpoint{-0.000000in}{0.000000in}}%
\pgfpathlineto{\pgfqpoint{-0.048611in}{0.000000in}}%
\pgfusepath{stroke,fill}%
}%
\begin{pgfscope}%
\pgfsys@transformshift{0.789290in}{1.797986in}%
\pgfsys@useobject{currentmarker}{}%
\end{pgfscope}%
\end{pgfscope}%
\begin{pgfscope}%
\definecolor{textcolor}{rgb}{0.000000,0.000000,0.000000}%
\pgfsetstrokecolor{textcolor}%
\pgfsetfillcolor{textcolor}%
\pgftext[x=0.303410in, y=1.739948in, left, base]{\color{textcolor}\rmfamily\fontsize{11.000000}{13.200000}\selectfont \(\displaystyle {\ensuremath{-}0.08}\)}%
\end{pgfscope}%
\begin{pgfscope}%
\pgfsetbuttcap%
\pgfsetroundjoin%
\definecolor{currentfill}{rgb}{0.000000,0.000000,0.000000}%
\pgfsetfillcolor{currentfill}%
\pgfsetlinewidth{0.803000pt}%
\definecolor{currentstroke}{rgb}{0.000000,0.000000,0.000000}%
\pgfsetstrokecolor{currentstroke}%
\pgfsetdash{}{0pt}%
\pgfsys@defobject{currentmarker}{\pgfqpoint{-0.048611in}{0.000000in}}{\pgfqpoint{-0.000000in}{0.000000in}}{%
\pgfpathmoveto{\pgfqpoint{-0.000000in}{0.000000in}}%
\pgfpathlineto{\pgfqpoint{-0.048611in}{0.000000in}}%
\pgfusepath{stroke,fill}%
}%
\begin{pgfscope}%
\pgfsys@transformshift{0.789290in}{2.189986in}%
\pgfsys@useobject{currentmarker}{}%
\end{pgfscope}%
\end{pgfscope}%
\begin{pgfscope}%
\definecolor{textcolor}{rgb}{0.000000,0.000000,0.000000}%
\pgfsetstrokecolor{textcolor}%
\pgfsetfillcolor{textcolor}%
\pgftext[x=0.303410in, y=2.131948in, left, base]{\color{textcolor}\rmfamily\fontsize{11.000000}{13.200000}\selectfont \(\displaystyle {\ensuremath{-}0.06}\)}%
\end{pgfscope}%
\begin{pgfscope}%
\pgfsetbuttcap%
\pgfsetroundjoin%
\definecolor{currentfill}{rgb}{0.000000,0.000000,0.000000}%
\pgfsetfillcolor{currentfill}%
\pgfsetlinewidth{0.803000pt}%
\definecolor{currentstroke}{rgb}{0.000000,0.000000,0.000000}%
\pgfsetstrokecolor{currentstroke}%
\pgfsetdash{}{0pt}%
\pgfsys@defobject{currentmarker}{\pgfqpoint{-0.048611in}{0.000000in}}{\pgfqpoint{-0.000000in}{0.000000in}}{%
\pgfpathmoveto{\pgfqpoint{-0.000000in}{0.000000in}}%
\pgfpathlineto{\pgfqpoint{-0.048611in}{0.000000in}}%
\pgfusepath{stroke,fill}%
}%
\begin{pgfscope}%
\pgfsys@transformshift{0.789290in}{2.581986in}%
\pgfsys@useobject{currentmarker}{}%
\end{pgfscope}%
\end{pgfscope}%
\begin{pgfscope}%
\definecolor{textcolor}{rgb}{0.000000,0.000000,0.000000}%
\pgfsetstrokecolor{textcolor}%
\pgfsetfillcolor{textcolor}%
\pgftext[x=0.303410in, y=2.523948in, left, base]{\color{textcolor}\rmfamily\fontsize{11.000000}{13.200000}\selectfont \(\displaystyle {\ensuremath{-}0.04}\)}%
\end{pgfscope}%
\begin{pgfscope}%
\pgfsetbuttcap%
\pgfsetroundjoin%
\definecolor{currentfill}{rgb}{0.000000,0.000000,0.000000}%
\pgfsetfillcolor{currentfill}%
\pgfsetlinewidth{0.803000pt}%
\definecolor{currentstroke}{rgb}{0.000000,0.000000,0.000000}%
\pgfsetstrokecolor{currentstroke}%
\pgfsetdash{}{0pt}%
\pgfsys@defobject{currentmarker}{\pgfqpoint{-0.048611in}{0.000000in}}{\pgfqpoint{-0.000000in}{0.000000in}}{%
\pgfpathmoveto{\pgfqpoint{-0.000000in}{0.000000in}}%
\pgfpathlineto{\pgfqpoint{-0.048611in}{0.000000in}}%
\pgfusepath{stroke,fill}%
}%
\begin{pgfscope}%
\pgfsys@transformshift{0.789290in}{2.973986in}%
\pgfsys@useobject{currentmarker}{}%
\end{pgfscope}%
\end{pgfscope}%
\begin{pgfscope}%
\definecolor{textcolor}{rgb}{0.000000,0.000000,0.000000}%
\pgfsetstrokecolor{textcolor}%
\pgfsetfillcolor{textcolor}%
\pgftext[x=0.303410in, y=2.915948in, left, base]{\color{textcolor}\rmfamily\fontsize{11.000000}{13.200000}\selectfont \(\displaystyle {\ensuremath{-}0.02}\)}%
\end{pgfscope}%
\begin{pgfscope}%
\definecolor{textcolor}{rgb}{0.000000,0.000000,0.000000}%
\pgfsetstrokecolor{textcolor}%
\pgfsetfillcolor{textcolor}%
\pgftext[x=0.247854in,y=1.895986in,,bottom,rotate=90.000000]{\color{textcolor}\rmfamily\fontsize{11.000000}{13.200000}\selectfont \(\displaystyle u_z\)}%
\end{pgfscope}%
\begin{pgfscope}%
\pgfpathrectangle{\pgfqpoint{0.789290in}{0.548486in}}{\pgfqpoint{5.037500in}{2.695000in}}%
\pgfusepath{clip}%
\pgfsetrectcap%
\pgfsetroundjoin%
\pgfsetlinewidth{1.505625pt}%
\definecolor{currentstroke}{rgb}{0.054902,0.262745,0.486275}%
\pgfsetstrokecolor{currentstroke}%
\pgfsetstrokeopacity{0.250000}%
\pgfsetdash{}{0pt}%
\pgfpathmoveto{\pgfqpoint{0.789290in}{3.120986in}}%
\pgfpathlineto{\pgfqpoint{0.840174in}{3.096238in}}%
\pgfpathlineto{\pgfqpoint{0.891058in}{3.071491in}}%
\pgfpathlineto{\pgfqpoint{0.941941in}{3.046743in}}%
\pgfpathlineto{\pgfqpoint{0.992825in}{3.021996in}}%
\pgfpathlineto{\pgfqpoint{1.043709in}{2.997248in}}%
\pgfpathlineto{\pgfqpoint{1.094593in}{2.972501in}}%
\pgfpathlineto{\pgfqpoint{1.145477in}{2.947754in}}%
\pgfpathlineto{\pgfqpoint{1.196361in}{2.923006in}}%
\pgfpathlineto{\pgfqpoint{1.247244in}{2.898259in}}%
\pgfpathlineto{\pgfqpoint{1.298128in}{2.873511in}}%
\pgfpathlineto{\pgfqpoint{1.349012in}{2.848764in}}%
\pgfpathlineto{\pgfqpoint{1.399896in}{2.824016in}}%
\pgfpathlineto{\pgfqpoint{1.450780in}{2.799269in}}%
\pgfpathlineto{\pgfqpoint{1.501664in}{2.774521in}}%
\pgfpathlineto{\pgfqpoint{1.552547in}{2.749774in}}%
\pgfpathlineto{\pgfqpoint{1.603431in}{2.725026in}}%
\pgfpathlineto{\pgfqpoint{1.654315in}{2.700279in}}%
\pgfpathlineto{\pgfqpoint{1.705199in}{2.675531in}}%
\pgfpathlineto{\pgfqpoint{1.756083in}{2.650784in}}%
\pgfpathlineto{\pgfqpoint{1.806967in}{2.626036in}}%
\pgfpathlineto{\pgfqpoint{1.857850in}{2.601289in}}%
\pgfpathlineto{\pgfqpoint{1.908734in}{2.576541in}}%
\pgfpathlineto{\pgfqpoint{1.959618in}{2.551794in}}%
\pgfpathlineto{\pgfqpoint{2.010502in}{2.527046in}}%
\pgfpathlineto{\pgfqpoint{2.061386in}{2.502299in}}%
\pgfpathlineto{\pgfqpoint{2.112270in}{2.477551in}}%
\pgfpathlineto{\pgfqpoint{2.163153in}{2.452804in}}%
\pgfpathlineto{\pgfqpoint{2.214037in}{2.428057in}}%
\pgfpathlineto{\pgfqpoint{2.264921in}{2.403309in}}%
\pgfpathlineto{\pgfqpoint{2.315805in}{2.378562in}}%
\pgfpathlineto{\pgfqpoint{2.366689in}{2.353814in}}%
\pgfpathlineto{\pgfqpoint{2.417573in}{2.329067in}}%
\pgfpathlineto{\pgfqpoint{2.468457in}{2.304319in}}%
\pgfpathlineto{\pgfqpoint{2.519340in}{2.279572in}}%
\pgfpathlineto{\pgfqpoint{2.570224in}{2.254824in}}%
\pgfpathlineto{\pgfqpoint{2.621108in}{2.230077in}}%
\pgfpathlineto{\pgfqpoint{2.671992in}{2.205329in}}%
\pgfpathlineto{\pgfqpoint{2.722876in}{2.180582in}}%
\pgfpathlineto{\pgfqpoint{2.773760in}{2.155834in}}%
\pgfpathlineto{\pgfqpoint{2.824643in}{2.131087in}}%
\pgfpathlineto{\pgfqpoint{2.875527in}{2.106339in}}%
\pgfpathlineto{\pgfqpoint{2.926411in}{2.081592in}}%
\pgfpathlineto{\pgfqpoint{2.977295in}{2.056844in}}%
\pgfpathlineto{\pgfqpoint{3.028179in}{2.032097in}}%
\pgfpathlineto{\pgfqpoint{3.079063in}{2.007349in}}%
\pgfpathlineto{\pgfqpoint{3.129946in}{1.982602in}}%
\pgfpathlineto{\pgfqpoint{3.180830in}{1.957855in}}%
\pgfpathlineto{\pgfqpoint{3.231714in}{1.933107in}}%
\pgfpathlineto{\pgfqpoint{3.282598in}{1.908360in}}%
\pgfpathlineto{\pgfqpoint{3.333482in}{1.883612in}}%
\pgfpathlineto{\pgfqpoint{3.384366in}{1.858865in}}%
\pgfpathlineto{\pgfqpoint{3.435249in}{1.834117in}}%
\pgfpathlineto{\pgfqpoint{3.486133in}{1.809370in}}%
\pgfpathlineto{\pgfqpoint{3.537017in}{1.784622in}}%
\pgfpathlineto{\pgfqpoint{3.587901in}{1.759875in}}%
\pgfpathlineto{\pgfqpoint{3.638785in}{1.735127in}}%
\pgfpathlineto{\pgfqpoint{3.689669in}{1.710380in}}%
\pgfpathlineto{\pgfqpoint{3.740552in}{1.685632in}}%
\pgfpathlineto{\pgfqpoint{3.791436in}{1.660885in}}%
\pgfpathlineto{\pgfqpoint{3.842320in}{1.636137in}}%
\pgfpathlineto{\pgfqpoint{3.893204in}{1.611390in}}%
\pgfpathlineto{\pgfqpoint{3.944088in}{1.586642in}}%
\pgfpathlineto{\pgfqpoint{3.994972in}{1.561895in}}%
\pgfpathlineto{\pgfqpoint{4.045856in}{1.537147in}}%
\pgfpathlineto{\pgfqpoint{4.096739in}{1.512400in}}%
\pgfpathlineto{\pgfqpoint{4.147623in}{1.487653in}}%
\pgfpathlineto{\pgfqpoint{4.198507in}{1.462905in}}%
\pgfpathlineto{\pgfqpoint{4.249391in}{1.438158in}}%
\pgfpathlineto{\pgfqpoint{4.300275in}{1.413410in}}%
\pgfpathlineto{\pgfqpoint{4.351159in}{1.388663in}}%
\pgfpathlineto{\pgfqpoint{4.402042in}{1.363915in}}%
\pgfpathlineto{\pgfqpoint{4.452926in}{1.339168in}}%
\pgfpathlineto{\pgfqpoint{4.503810in}{1.314420in}}%
\pgfpathlineto{\pgfqpoint{4.554694in}{1.289673in}}%
\pgfpathlineto{\pgfqpoint{4.605578in}{1.264925in}}%
\pgfpathlineto{\pgfqpoint{4.656462in}{1.240178in}}%
\pgfpathlineto{\pgfqpoint{4.707345in}{1.215430in}}%
\pgfpathlineto{\pgfqpoint{4.758229in}{1.190683in}}%
\pgfpathlineto{\pgfqpoint{4.809113in}{1.165935in}}%
\pgfpathlineto{\pgfqpoint{4.859997in}{1.141188in}}%
\pgfpathlineto{\pgfqpoint{4.910881in}{1.116440in}}%
\pgfpathlineto{\pgfqpoint{4.961765in}{1.091693in}}%
\pgfpathlineto{\pgfqpoint{5.012648in}{1.066945in}}%
\pgfpathlineto{\pgfqpoint{5.063532in}{1.042198in}}%
\pgfpathlineto{\pgfqpoint{5.114416in}{1.017450in}}%
\pgfpathlineto{\pgfqpoint{5.165300in}{0.992703in}}%
\pgfpathlineto{\pgfqpoint{5.216184in}{0.967956in}}%
\pgfpathlineto{\pgfqpoint{5.267068in}{0.943208in}}%
\pgfpathlineto{\pgfqpoint{5.317951in}{0.918461in}}%
\pgfpathlineto{\pgfqpoint{5.368835in}{0.893713in}}%
\pgfpathlineto{\pgfqpoint{5.419719in}{0.868966in}}%
\pgfpathlineto{\pgfqpoint{5.470603in}{0.844218in}}%
\pgfpathlineto{\pgfqpoint{5.521487in}{0.819471in}}%
\pgfpathlineto{\pgfqpoint{5.572371in}{0.794723in}}%
\pgfpathlineto{\pgfqpoint{5.623255in}{0.769976in}}%
\pgfpathlineto{\pgfqpoint{5.674138in}{0.745228in}}%
\pgfpathlineto{\pgfqpoint{5.725022in}{0.720481in}}%
\pgfpathlineto{\pgfqpoint{5.775906in}{0.695733in}}%
\pgfpathlineto{\pgfqpoint{5.826790in}{0.670986in}}%
\pgfusepath{stroke}%
\end{pgfscope}%
\begin{pgfscope}%
\pgfsetrectcap%
\pgfsetmiterjoin%
\pgfsetlinewidth{0.803000pt}%
\definecolor{currentstroke}{rgb}{0.000000,0.000000,0.000000}%
\pgfsetstrokecolor{currentstroke}%
\pgfsetdash{}{0pt}%
\pgfpathmoveto{\pgfqpoint{0.789290in}{0.548486in}}%
\pgfpathlineto{\pgfqpoint{0.789290in}{3.243486in}}%
\pgfusepath{stroke}%
\end{pgfscope}%
\begin{pgfscope}%
\pgfsetrectcap%
\pgfsetmiterjoin%
\pgfsetlinewidth{0.803000pt}%
\definecolor{currentstroke}{rgb}{0.000000,0.000000,0.000000}%
\pgfsetstrokecolor{currentstroke}%
\pgfsetdash{}{0pt}%
\pgfpathmoveto{\pgfqpoint{5.826790in}{0.548486in}}%
\pgfpathlineto{\pgfqpoint{5.826790in}{3.243486in}}%
\pgfusepath{stroke}%
\end{pgfscope}%
\begin{pgfscope}%
\pgfsetrectcap%
\pgfsetmiterjoin%
\pgfsetlinewidth{0.803000pt}%
\definecolor{currentstroke}{rgb}{0.000000,0.000000,0.000000}%
\pgfsetstrokecolor{currentstroke}%
\pgfsetdash{}{0pt}%
\pgfpathmoveto{\pgfqpoint{0.789290in}{0.548486in}}%
\pgfpathlineto{\pgfqpoint{5.826790in}{0.548486in}}%
\pgfusepath{stroke}%
\end{pgfscope}%
\begin{pgfscope}%
\pgfsetrectcap%
\pgfsetmiterjoin%
\pgfsetlinewidth{0.803000pt}%
\definecolor{currentstroke}{rgb}{0.000000,0.000000,0.000000}%
\pgfsetstrokecolor{currentstroke}%
\pgfsetdash{}{0pt}%
\pgfpathmoveto{\pgfqpoint{0.789290in}{3.243486in}}%
\pgfpathlineto{\pgfqpoint{5.826790in}{3.243486in}}%
\pgfusepath{stroke}%
\end{pgfscope}%
\end{pgfpicture}%
\makeatother%
\endgroup%

    \caption{Normal displacements}
    \label{fig:normal-displacements}
\end{figure}

\begin{figure}[ht]
    \centering
     %% Creator: Matplotlib, PGF backend
%%
%% To include the figure in your LaTeX document, write
%%   \input{<filename>.pgf}
%%
%% Make sure the required packages are loaded in your preamble
%%   \usepackage{pgf}
%%
%% Also ensure that all the required font packages are loaded; for instance,
%% the lmodern package is sometimes necessary when using math font.
%%   \usepackage{lmodern}
%%
%% Figures using additional raster images can only be included by \input if
%% they are in the same directory as the main LaTeX file. For loading figures
%% from other directories you can use the `import` package
%%   \usepackage{import}
%%
%% and then include the figures with
%%   \import{<path to file>}{<filename>.pgf}
%%
%% Matplotlib used the following preamble
%%   
%%   \usepackage{fontspec}
%%   \setmainfont{DejaVuSans.ttf}[Path=\detokenize{/home/fabio/Internodes-CM/.venv/lib/python3.8/site-packages/matplotlib/mpl-data/fonts/ttf/}]
%%   \setsansfont{DejaVuSans.ttf}[Path=\detokenize{/home/fabio/Internodes-CM/.venv/lib/python3.8/site-packages/matplotlib/mpl-data/fonts/ttf/}]
%%   \setmonofont{DejaVuSansMono.ttf}[Path=\detokenize{/home/fabio/Internodes-CM/.venv/lib/python3.8/site-packages/matplotlib/mpl-data/fonts/ttf/}]
%%   \makeatletter\@ifpackageloaded{underscore}{}{\usepackage[strings]{underscore}}\makeatother
%%
\begingroup%
\makeatletter%
\begin{pgfpicture}%
\pgfpathrectangle{\pgfpointorigin}{\pgfqpoint{5.808503in}{3.343486in}}%
\pgfusepath{use as bounding box, clip}%
\begin{pgfscope}%
\pgfsetbuttcap%
\pgfsetmiterjoin%
\definecolor{currentfill}{rgb}{1.000000,1.000000,1.000000}%
\pgfsetfillcolor{currentfill}%
\pgfsetlinewidth{0.000000pt}%
\definecolor{currentstroke}{rgb}{1.000000,1.000000,1.000000}%
\pgfsetstrokecolor{currentstroke}%
\pgfsetdash{}{0pt}%
\pgfpathmoveto{\pgfqpoint{0.000000in}{0.000000in}}%
\pgfpathlineto{\pgfqpoint{5.808503in}{0.000000in}}%
\pgfpathlineto{\pgfqpoint{5.808503in}{3.343486in}}%
\pgfpathlineto{\pgfqpoint{0.000000in}{3.343486in}}%
\pgfpathlineto{\pgfqpoint{0.000000in}{0.000000in}}%
\pgfpathclose%
\pgfusepath{fill}%
\end{pgfscope}%
\begin{pgfscope}%
\pgfsetbuttcap%
\pgfsetmiterjoin%
\definecolor{currentfill}{rgb}{1.000000,1.000000,1.000000}%
\pgfsetfillcolor{currentfill}%
\pgfsetlinewidth{0.000000pt}%
\definecolor{currentstroke}{rgb}{0.000000,0.000000,0.000000}%
\pgfsetstrokecolor{currentstroke}%
\pgfsetstrokeopacity{0.000000}%
\pgfsetdash{}{0pt}%
\pgfpathmoveto{\pgfqpoint{0.671003in}{0.548486in}}%
\pgfpathlineto{\pgfqpoint{5.708502in}{0.548486in}}%
\pgfpathlineto{\pgfqpoint{5.708502in}{3.243486in}}%
\pgfpathlineto{\pgfqpoint{0.671003in}{3.243486in}}%
\pgfpathlineto{\pgfqpoint{0.671003in}{0.548486in}}%
\pgfpathclose%
\pgfusepath{fill}%
\end{pgfscope}%
\begin{pgfscope}%
\pgfpathrectangle{\pgfqpoint{0.671003in}{0.548486in}}{\pgfqpoint{5.037500in}{2.695000in}}%
\pgfusepath{clip}%
\pgfsetbuttcap%
\pgfsetroundjoin%
\definecolor{currentfill}{rgb}{0.054902,0.262745,0.486275}%
\pgfsetfillcolor{currentfill}%
\pgfsetlinewidth{1.003750pt}%
\definecolor{currentstroke}{rgb}{0.054902,0.262745,0.486275}%
\pgfsetstrokecolor{currentstroke}%
\pgfsetdash{}{0pt}%
\pgfsys@defobject{currentmarker}{\pgfqpoint{-0.041667in}{-0.041667in}}{\pgfqpoint{0.041667in}{0.041667in}}{%
\pgfpathmoveto{\pgfqpoint{0.000000in}{-0.041667in}}%
\pgfpathcurveto{\pgfqpoint{0.011050in}{-0.041667in}}{\pgfqpoint{0.021649in}{-0.037276in}}{\pgfqpoint{0.029463in}{-0.029463in}}%
\pgfpathcurveto{\pgfqpoint{0.037276in}{-0.021649in}}{\pgfqpoint{0.041667in}{-0.011050in}}{\pgfqpoint{0.041667in}{0.000000in}}%
\pgfpathcurveto{\pgfqpoint{0.041667in}{0.011050in}}{\pgfqpoint{0.037276in}{0.021649in}}{\pgfqpoint{0.029463in}{0.029463in}}%
\pgfpathcurveto{\pgfqpoint{0.021649in}{0.037276in}}{\pgfqpoint{0.011050in}{0.041667in}}{\pgfqpoint{0.000000in}{0.041667in}}%
\pgfpathcurveto{\pgfqpoint{-0.011050in}{0.041667in}}{\pgfqpoint{-0.021649in}{0.037276in}}{\pgfqpoint{-0.029463in}{0.029463in}}%
\pgfpathcurveto{\pgfqpoint{-0.037276in}{0.021649in}}{\pgfqpoint{-0.041667in}{0.011050in}}{\pgfqpoint{-0.041667in}{0.000000in}}%
\pgfpathcurveto{\pgfqpoint{-0.041667in}{-0.011050in}}{\pgfqpoint{-0.037276in}{-0.021649in}}{\pgfqpoint{-0.029463in}{-0.029463in}}%
\pgfpathcurveto{\pgfqpoint{-0.021649in}{-0.037276in}}{\pgfqpoint{-0.011050in}{-0.041667in}}{\pgfqpoint{0.000000in}{-0.041667in}}%
\pgfpathlineto{\pgfqpoint{0.000000in}{-0.041667in}}%
\pgfpathclose%
\pgfusepath{stroke,fill}%
}%
\begin{pgfscope}%
\pgfsys@transformshift{1.174753in}{0.940695in}%
\pgfsys@useobject{currentmarker}{}%
\end{pgfscope}%
\begin{pgfscope}%
\pgfsys@transformshift{1.622530in}{1.438501in}%
\pgfsys@useobject{currentmarker}{}%
\end{pgfscope}%
\begin{pgfscope}%
\pgfsys@transformshift{2.070308in}{1.840724in}%
\pgfsys@useobject{currentmarker}{}%
\end{pgfscope}%
\begin{pgfscope}%
\pgfsys@transformshift{2.518086in}{2.019350in}%
\pgfsys@useobject{currentmarker}{}%
\end{pgfscope}%
\begin{pgfscope}%
\pgfsys@transformshift{2.965864in}{2.243552in}%
\pgfsys@useobject{currentmarker}{}%
\end{pgfscope}%
\begin{pgfscope}%
\pgfsys@transformshift{3.413641in}{2.623822in}%
\pgfsys@useobject{currentmarker}{}%
\end{pgfscope}%
\begin{pgfscope}%
\pgfsys@transformshift{3.861419in}{2.718059in}%
\pgfsys@useobject{currentmarker}{}%
\end{pgfscope}%
\begin{pgfscope}%
\pgfsys@transformshift{4.309197in}{2.705364in}%
\pgfsys@useobject{currentmarker}{}%
\end{pgfscope}%
\begin{pgfscope}%
\pgfsys@transformshift{4.756975in}{2.701932in}%
\pgfsys@useobject{currentmarker}{}%
\end{pgfscope}%
\begin{pgfscope}%
\pgfsys@transformshift{5.204752in}{2.696080in}%
\pgfsys@useobject{currentmarker}{}%
\end{pgfscope}%
\end{pgfscope}%
\begin{pgfscope}%
\pgfsetbuttcap%
\pgfsetroundjoin%
\definecolor{currentfill}{rgb}{0.000000,0.000000,0.000000}%
\pgfsetfillcolor{currentfill}%
\pgfsetlinewidth{0.803000pt}%
\definecolor{currentstroke}{rgb}{0.000000,0.000000,0.000000}%
\pgfsetstrokecolor{currentstroke}%
\pgfsetdash{}{0pt}%
\pgfsys@defobject{currentmarker}{\pgfqpoint{0.000000in}{-0.048611in}}{\pgfqpoint{0.000000in}{0.000000in}}{%
\pgfpathmoveto{\pgfqpoint{0.000000in}{0.000000in}}%
\pgfpathlineto{\pgfqpoint{0.000000in}{-0.048611in}}%
\pgfusepath{stroke,fill}%
}%
\begin{pgfscope}%
\pgfsys@transformshift{1.174753in}{0.548486in}%
\pgfsys@useobject{currentmarker}{}%
\end{pgfscope}%
\end{pgfscope}%
\begin{pgfscope}%
\definecolor{textcolor}{rgb}{0.000000,0.000000,0.000000}%
\pgfsetstrokecolor{textcolor}%
\pgfsetfillcolor{textcolor}%
\pgftext[x=1.174753in,y=0.451264in,,top]{\color{textcolor}\rmfamily\fontsize{11.000000}{13.200000}\selectfont \(\displaystyle {0.05}\)}%
\end{pgfscope}%
\begin{pgfscope}%
\pgfsetbuttcap%
\pgfsetroundjoin%
\definecolor{currentfill}{rgb}{0.000000,0.000000,0.000000}%
\pgfsetfillcolor{currentfill}%
\pgfsetlinewidth{0.803000pt}%
\definecolor{currentstroke}{rgb}{0.000000,0.000000,0.000000}%
\pgfsetstrokecolor{currentstroke}%
\pgfsetdash{}{0pt}%
\pgfsys@defobject{currentmarker}{\pgfqpoint{0.000000in}{-0.048611in}}{\pgfqpoint{0.000000in}{0.000000in}}{%
\pgfpathmoveto{\pgfqpoint{0.000000in}{0.000000in}}%
\pgfpathlineto{\pgfqpoint{0.000000in}{-0.048611in}}%
\pgfusepath{stroke,fill}%
}%
\begin{pgfscope}%
\pgfsys@transformshift{2.182253in}{0.548486in}%
\pgfsys@useobject{currentmarker}{}%
\end{pgfscope}%
\end{pgfscope}%
\begin{pgfscope}%
\definecolor{textcolor}{rgb}{0.000000,0.000000,0.000000}%
\pgfsetstrokecolor{textcolor}%
\pgfsetfillcolor{textcolor}%
\pgftext[x=2.182253in,y=0.451264in,,top]{\color{textcolor}\rmfamily\fontsize{11.000000}{13.200000}\selectfont \(\displaystyle {0.10}\)}%
\end{pgfscope}%
\begin{pgfscope}%
\pgfsetbuttcap%
\pgfsetroundjoin%
\definecolor{currentfill}{rgb}{0.000000,0.000000,0.000000}%
\pgfsetfillcolor{currentfill}%
\pgfsetlinewidth{0.803000pt}%
\definecolor{currentstroke}{rgb}{0.000000,0.000000,0.000000}%
\pgfsetstrokecolor{currentstroke}%
\pgfsetdash{}{0pt}%
\pgfsys@defobject{currentmarker}{\pgfqpoint{0.000000in}{-0.048611in}}{\pgfqpoint{0.000000in}{0.000000in}}{%
\pgfpathmoveto{\pgfqpoint{0.000000in}{0.000000in}}%
\pgfpathlineto{\pgfqpoint{0.000000in}{-0.048611in}}%
\pgfusepath{stroke,fill}%
}%
\begin{pgfscope}%
\pgfsys@transformshift{3.189753in}{0.548486in}%
\pgfsys@useobject{currentmarker}{}%
\end{pgfscope}%
\end{pgfscope}%
\begin{pgfscope}%
\definecolor{textcolor}{rgb}{0.000000,0.000000,0.000000}%
\pgfsetstrokecolor{textcolor}%
\pgfsetfillcolor{textcolor}%
\pgftext[x=3.189752in,y=0.451264in,,top]{\color{textcolor}\rmfamily\fontsize{11.000000}{13.200000}\selectfont \(\displaystyle {0.15}\)}%
\end{pgfscope}%
\begin{pgfscope}%
\pgfsetbuttcap%
\pgfsetroundjoin%
\definecolor{currentfill}{rgb}{0.000000,0.000000,0.000000}%
\pgfsetfillcolor{currentfill}%
\pgfsetlinewidth{0.803000pt}%
\definecolor{currentstroke}{rgb}{0.000000,0.000000,0.000000}%
\pgfsetstrokecolor{currentstroke}%
\pgfsetdash{}{0pt}%
\pgfsys@defobject{currentmarker}{\pgfqpoint{0.000000in}{-0.048611in}}{\pgfqpoint{0.000000in}{0.000000in}}{%
\pgfpathmoveto{\pgfqpoint{0.000000in}{0.000000in}}%
\pgfpathlineto{\pgfqpoint{0.000000in}{-0.048611in}}%
\pgfusepath{stroke,fill}%
}%
\begin{pgfscope}%
\pgfsys@transformshift{4.197253in}{0.548486in}%
\pgfsys@useobject{currentmarker}{}%
\end{pgfscope}%
\end{pgfscope}%
\begin{pgfscope}%
\definecolor{textcolor}{rgb}{0.000000,0.000000,0.000000}%
\pgfsetstrokecolor{textcolor}%
\pgfsetfillcolor{textcolor}%
\pgftext[x=4.197253in,y=0.451264in,,top]{\color{textcolor}\rmfamily\fontsize{11.000000}{13.200000}\selectfont \(\displaystyle {0.20}\)}%
\end{pgfscope}%
\begin{pgfscope}%
\pgfsetbuttcap%
\pgfsetroundjoin%
\definecolor{currentfill}{rgb}{0.000000,0.000000,0.000000}%
\pgfsetfillcolor{currentfill}%
\pgfsetlinewidth{0.803000pt}%
\definecolor{currentstroke}{rgb}{0.000000,0.000000,0.000000}%
\pgfsetstrokecolor{currentstroke}%
\pgfsetdash{}{0pt}%
\pgfsys@defobject{currentmarker}{\pgfqpoint{0.000000in}{-0.048611in}}{\pgfqpoint{0.000000in}{0.000000in}}{%
\pgfpathmoveto{\pgfqpoint{0.000000in}{0.000000in}}%
\pgfpathlineto{\pgfqpoint{0.000000in}{-0.048611in}}%
\pgfusepath{stroke,fill}%
}%
\begin{pgfscope}%
\pgfsys@transformshift{5.204752in}{0.548486in}%
\pgfsys@useobject{currentmarker}{}%
\end{pgfscope}%
\end{pgfscope}%
\begin{pgfscope}%
\definecolor{textcolor}{rgb}{0.000000,0.000000,0.000000}%
\pgfsetstrokecolor{textcolor}%
\pgfsetfillcolor{textcolor}%
\pgftext[x=5.204752in,y=0.451264in,,top]{\color{textcolor}\rmfamily\fontsize{11.000000}{13.200000}\selectfont \(\displaystyle {0.25}\)}%
\end{pgfscope}%
\begin{pgfscope}%
\definecolor{textcolor}{rgb}{0.000000,0.000000,0.000000}%
\pgfsetstrokecolor{textcolor}%
\pgfsetfillcolor{textcolor}%
\pgftext[x=3.189752in,y=0.247854in,,top]{\color{textcolor}\rmfamily\fontsize{11.000000}{13.200000}\selectfont \(\displaystyle d\)}%
\end{pgfscope}%
\begin{pgfscope}%
\pgfsetbuttcap%
\pgfsetroundjoin%
\definecolor{currentfill}{rgb}{0.000000,0.000000,0.000000}%
\pgfsetfillcolor{currentfill}%
\pgfsetlinewidth{0.803000pt}%
\definecolor{currentstroke}{rgb}{0.000000,0.000000,0.000000}%
\pgfsetstrokecolor{currentstroke}%
\pgfsetdash{}{0pt}%
\pgfsys@defobject{currentmarker}{\pgfqpoint{-0.048611in}{0.000000in}}{\pgfqpoint{-0.000000in}{0.000000in}}{%
\pgfpathmoveto{\pgfqpoint{-0.000000in}{0.000000in}}%
\pgfpathlineto{\pgfqpoint{-0.048611in}{0.000000in}}%
\pgfusepath{stroke,fill}%
}%
\begin{pgfscope}%
\pgfsys@transformshift{0.671003in}{0.559335in}%
\pgfsys@useobject{currentmarker}{}%
\end{pgfscope}%
\end{pgfscope}%
\begin{pgfscope}%
\definecolor{textcolor}{rgb}{0.000000,0.000000,0.000000}%
\pgfsetstrokecolor{textcolor}%
\pgfsetfillcolor{textcolor}%
\pgftext[x=0.303410in, y=0.501297in, left, base]{\color{textcolor}\rmfamily\fontsize{11.000000}{13.200000}\selectfont \(\displaystyle {0.10}\)}%
\end{pgfscope}%
\begin{pgfscope}%
\pgfsetbuttcap%
\pgfsetroundjoin%
\definecolor{currentfill}{rgb}{0.000000,0.000000,0.000000}%
\pgfsetfillcolor{currentfill}%
\pgfsetlinewidth{0.803000pt}%
\definecolor{currentstroke}{rgb}{0.000000,0.000000,0.000000}%
\pgfsetstrokecolor{currentstroke}%
\pgfsetdash{}{0pt}%
\pgfsys@defobject{currentmarker}{\pgfqpoint{-0.048611in}{0.000000in}}{\pgfqpoint{-0.000000in}{0.000000in}}{%
\pgfpathmoveto{\pgfqpoint{-0.000000in}{0.000000in}}%
\pgfpathlineto{\pgfqpoint{-0.048611in}{0.000000in}}%
\pgfusepath{stroke,fill}%
}%
\begin{pgfscope}%
\pgfsys@transformshift{0.671003in}{1.032296in}%
\pgfsys@useobject{currentmarker}{}%
\end{pgfscope}%
\end{pgfscope}%
\begin{pgfscope}%
\definecolor{textcolor}{rgb}{0.000000,0.000000,0.000000}%
\pgfsetstrokecolor{textcolor}%
\pgfsetfillcolor{textcolor}%
\pgftext[x=0.303410in, y=0.974258in, left, base]{\color{textcolor}\rmfamily\fontsize{11.000000}{13.200000}\selectfont \(\displaystyle {0.15}\)}%
\end{pgfscope}%
\begin{pgfscope}%
\pgfsetbuttcap%
\pgfsetroundjoin%
\definecolor{currentfill}{rgb}{0.000000,0.000000,0.000000}%
\pgfsetfillcolor{currentfill}%
\pgfsetlinewidth{0.803000pt}%
\definecolor{currentstroke}{rgb}{0.000000,0.000000,0.000000}%
\pgfsetstrokecolor{currentstroke}%
\pgfsetdash{}{0pt}%
\pgfsys@defobject{currentmarker}{\pgfqpoint{-0.048611in}{0.000000in}}{\pgfqpoint{-0.000000in}{0.000000in}}{%
\pgfpathmoveto{\pgfqpoint{-0.000000in}{0.000000in}}%
\pgfpathlineto{\pgfqpoint{-0.048611in}{0.000000in}}%
\pgfusepath{stroke,fill}%
}%
\begin{pgfscope}%
\pgfsys@transformshift{0.671003in}{1.505257in}%
\pgfsys@useobject{currentmarker}{}%
\end{pgfscope}%
\end{pgfscope}%
\begin{pgfscope}%
\definecolor{textcolor}{rgb}{0.000000,0.000000,0.000000}%
\pgfsetstrokecolor{textcolor}%
\pgfsetfillcolor{textcolor}%
\pgftext[x=0.303410in, y=1.447219in, left, base]{\color{textcolor}\rmfamily\fontsize{11.000000}{13.200000}\selectfont \(\displaystyle {0.20}\)}%
\end{pgfscope}%
\begin{pgfscope}%
\pgfsetbuttcap%
\pgfsetroundjoin%
\definecolor{currentfill}{rgb}{0.000000,0.000000,0.000000}%
\pgfsetfillcolor{currentfill}%
\pgfsetlinewidth{0.803000pt}%
\definecolor{currentstroke}{rgb}{0.000000,0.000000,0.000000}%
\pgfsetstrokecolor{currentstroke}%
\pgfsetdash{}{0pt}%
\pgfsys@defobject{currentmarker}{\pgfqpoint{-0.048611in}{0.000000in}}{\pgfqpoint{-0.000000in}{0.000000in}}{%
\pgfpathmoveto{\pgfqpoint{-0.000000in}{0.000000in}}%
\pgfpathlineto{\pgfqpoint{-0.048611in}{0.000000in}}%
\pgfusepath{stroke,fill}%
}%
\begin{pgfscope}%
\pgfsys@transformshift{0.671003in}{1.978218in}%
\pgfsys@useobject{currentmarker}{}%
\end{pgfscope}%
\end{pgfscope}%
\begin{pgfscope}%
\definecolor{textcolor}{rgb}{0.000000,0.000000,0.000000}%
\pgfsetstrokecolor{textcolor}%
\pgfsetfillcolor{textcolor}%
\pgftext[x=0.303410in, y=1.920180in, left, base]{\color{textcolor}\rmfamily\fontsize{11.000000}{13.200000}\selectfont \(\displaystyle {0.25}\)}%
\end{pgfscope}%
\begin{pgfscope}%
\pgfsetbuttcap%
\pgfsetroundjoin%
\definecolor{currentfill}{rgb}{0.000000,0.000000,0.000000}%
\pgfsetfillcolor{currentfill}%
\pgfsetlinewidth{0.803000pt}%
\definecolor{currentstroke}{rgb}{0.000000,0.000000,0.000000}%
\pgfsetstrokecolor{currentstroke}%
\pgfsetdash{}{0pt}%
\pgfsys@defobject{currentmarker}{\pgfqpoint{-0.048611in}{0.000000in}}{\pgfqpoint{-0.000000in}{0.000000in}}{%
\pgfpathmoveto{\pgfqpoint{-0.000000in}{0.000000in}}%
\pgfpathlineto{\pgfqpoint{-0.048611in}{0.000000in}}%
\pgfusepath{stroke,fill}%
}%
\begin{pgfscope}%
\pgfsys@transformshift{0.671003in}{2.451179in}%
\pgfsys@useobject{currentmarker}{}%
\end{pgfscope}%
\end{pgfscope}%
\begin{pgfscope}%
\definecolor{textcolor}{rgb}{0.000000,0.000000,0.000000}%
\pgfsetstrokecolor{textcolor}%
\pgfsetfillcolor{textcolor}%
\pgftext[x=0.303410in, y=2.393141in, left, base]{\color{textcolor}\rmfamily\fontsize{11.000000}{13.200000}\selectfont \(\displaystyle {0.30}\)}%
\end{pgfscope}%
\begin{pgfscope}%
\pgfsetbuttcap%
\pgfsetroundjoin%
\definecolor{currentfill}{rgb}{0.000000,0.000000,0.000000}%
\pgfsetfillcolor{currentfill}%
\pgfsetlinewidth{0.803000pt}%
\definecolor{currentstroke}{rgb}{0.000000,0.000000,0.000000}%
\pgfsetstrokecolor{currentstroke}%
\pgfsetdash{}{0pt}%
\pgfsys@defobject{currentmarker}{\pgfqpoint{-0.048611in}{0.000000in}}{\pgfqpoint{-0.000000in}{0.000000in}}{%
\pgfpathmoveto{\pgfqpoint{-0.000000in}{0.000000in}}%
\pgfpathlineto{\pgfqpoint{-0.048611in}{0.000000in}}%
\pgfusepath{stroke,fill}%
}%
\begin{pgfscope}%
\pgfsys@transformshift{0.671003in}{2.924140in}%
\pgfsys@useobject{currentmarker}{}%
\end{pgfscope}%
\end{pgfscope}%
\begin{pgfscope}%
\definecolor{textcolor}{rgb}{0.000000,0.000000,0.000000}%
\pgfsetstrokecolor{textcolor}%
\pgfsetfillcolor{textcolor}%
\pgftext[x=0.303410in, y=2.866103in, left, base]{\color{textcolor}\rmfamily\fontsize{11.000000}{13.200000}\selectfont \(\displaystyle {0.35}\)}%
\end{pgfscope}%
\begin{pgfscope}%
\definecolor{textcolor}{rgb}{0.000000,0.000000,0.000000}%
\pgfsetstrokecolor{textcolor}%
\pgfsetfillcolor{textcolor}%
\pgftext[x=0.247854in,y=1.895986in,,bottom,rotate=90.000000]{\color{textcolor}\rmfamily\fontsize{11.000000}{13.200000}\selectfont \(\displaystyle a\)}%
\end{pgfscope}%
\begin{pgfscope}%
\pgfpathrectangle{\pgfqpoint{0.671003in}{0.548486in}}{\pgfqpoint{5.037500in}{2.695000in}}%
\pgfusepath{clip}%
\pgfsetrectcap%
\pgfsetroundjoin%
\pgfsetlinewidth{1.505625pt}%
\definecolor{currentstroke}{rgb}{0.054902,0.262745,0.486275}%
\pgfsetstrokecolor{currentstroke}%
\pgfsetstrokeopacity{0.250000}%
\pgfsetdash{}{0pt}%
\pgfpathmoveto{\pgfqpoint{0.671003in}{0.670986in}}%
\pgfpathlineto{\pgfqpoint{0.721886in}{0.723114in}}%
\pgfpathlineto{\pgfqpoint{0.772770in}{0.772901in}}%
\pgfpathlineto{\pgfqpoint{0.823654in}{0.820636in}}%
\pgfpathlineto{\pgfqpoint{0.874538in}{0.866554in}}%
\pgfpathlineto{\pgfqpoint{0.925422in}{0.910849in}}%
\pgfpathlineto{\pgfqpoint{0.976306in}{0.953680in}}%
\pgfpathlineto{\pgfqpoint{1.027189in}{0.995184in}}%
\pgfpathlineto{\pgfqpoint{1.078073in}{1.035477in}}%
\pgfpathlineto{\pgfqpoint{1.128957in}{1.074660in}}%
\pgfpathlineto{\pgfqpoint{1.179841in}{1.112819in}}%
\pgfpathlineto{\pgfqpoint{1.230725in}{1.150031in}}%
\pgfpathlineto{\pgfqpoint{1.281609in}{1.186362in}}%
\pgfpathlineto{\pgfqpoint{1.332492in}{1.221874in}}%
\pgfpathlineto{\pgfqpoint{1.383376in}{1.256618in}}%
\pgfpathlineto{\pgfqpoint{1.434260in}{1.290642in}}%
\pgfpathlineto{\pgfqpoint{1.485144in}{1.323990in}}%
\pgfpathlineto{\pgfqpoint{1.536028in}{1.356700in}}%
\pgfpathlineto{\pgfqpoint{1.586912in}{1.388807in}}%
\pgfpathlineto{\pgfqpoint{1.637795in}{1.420344in}}%
\pgfpathlineto{\pgfqpoint{1.688679in}{1.451340in}}%
\pgfpathlineto{\pgfqpoint{1.739563in}{1.481822in}}%
\pgfpathlineto{\pgfqpoint{1.790447in}{1.511815in}}%
\pgfpathlineto{\pgfqpoint{1.841331in}{1.541340in}}%
\pgfpathlineto{\pgfqpoint{1.892215in}{1.570421in}}%
\pgfpathlineto{\pgfqpoint{1.943098in}{1.599076in}}%
\pgfpathlineto{\pgfqpoint{1.993982in}{1.627323in}}%
\pgfpathlineto{\pgfqpoint{2.044866in}{1.655179in}}%
\pgfpathlineto{\pgfqpoint{2.095750in}{1.682660in}}%
\pgfpathlineto{\pgfqpoint{2.146634in}{1.709781in}}%
\pgfpathlineto{\pgfqpoint{2.197518in}{1.736556in}}%
\pgfpathlineto{\pgfqpoint{2.248401in}{1.762997in}}%
\pgfpathlineto{\pgfqpoint{2.299285in}{1.789117in}}%
\pgfpathlineto{\pgfqpoint{2.350169in}{1.814927in}}%
\pgfpathlineto{\pgfqpoint{2.401053in}{1.840438in}}%
\pgfpathlineto{\pgfqpoint{2.451937in}{1.865659in}}%
\pgfpathlineto{\pgfqpoint{2.502821in}{1.890602in}}%
\pgfpathlineto{\pgfqpoint{2.553705in}{1.915274in}}%
\pgfpathlineto{\pgfqpoint{2.604588in}{1.939685in}}%
\pgfpathlineto{\pgfqpoint{2.655472in}{1.963842in}}%
\pgfpathlineto{\pgfqpoint{2.706356in}{1.987754in}}%
\pgfpathlineto{\pgfqpoint{2.757240in}{2.011427in}}%
\pgfpathlineto{\pgfqpoint{2.808124in}{2.034868in}}%
\pgfpathlineto{\pgfqpoint{2.859008in}{2.058085in}}%
\pgfpathlineto{\pgfqpoint{2.909891in}{2.081083in}}%
\pgfpathlineto{\pgfqpoint{2.960775in}{2.103869in}}%
\pgfpathlineto{\pgfqpoint{3.011659in}{2.126449in}}%
\pgfpathlineto{\pgfqpoint{3.062543in}{2.148827in}}%
\pgfpathlineto{\pgfqpoint{3.113427in}{2.171009in}}%
\pgfpathlineto{\pgfqpoint{3.164311in}{2.193001in}}%
\pgfpathlineto{\pgfqpoint{3.215194in}{2.214807in}}%
\pgfpathlineto{\pgfqpoint{3.266078in}{2.236432in}}%
\pgfpathlineto{\pgfqpoint{3.316962in}{2.257880in}}%
\pgfpathlineto{\pgfqpoint{3.367846in}{2.279155in}}%
\pgfpathlineto{\pgfqpoint{3.418730in}{2.300261in}}%
\pgfpathlineto{\pgfqpoint{3.469614in}{2.321204in}}%
\pgfpathlineto{\pgfqpoint{3.520497in}{2.341985in}}%
\pgfpathlineto{\pgfqpoint{3.571381in}{2.362610in}}%
\pgfpathlineto{\pgfqpoint{3.622265in}{2.383081in}}%
\pgfpathlineto{\pgfqpoint{3.673149in}{2.403401in}}%
\pgfpathlineto{\pgfqpoint{3.724033in}{2.423575in}}%
\pgfpathlineto{\pgfqpoint{3.774917in}{2.443605in}}%
\pgfpathlineto{\pgfqpoint{3.825800in}{2.463494in}}%
\pgfpathlineto{\pgfqpoint{3.876684in}{2.483245in}}%
\pgfpathlineto{\pgfqpoint{3.927568in}{2.502862in}}%
\pgfpathlineto{\pgfqpoint{3.978452in}{2.522346in}}%
\pgfpathlineto{\pgfqpoint{4.029336in}{2.541700in}}%
\pgfpathlineto{\pgfqpoint{4.080220in}{2.560927in}}%
\pgfpathlineto{\pgfqpoint{4.131104in}{2.580030in}}%
\pgfpathlineto{\pgfqpoint{4.181987in}{2.599010in}}%
\pgfpathlineto{\pgfqpoint{4.232871in}{2.617871in}}%
\pgfpathlineto{\pgfqpoint{4.283755in}{2.636614in}}%
\pgfpathlineto{\pgfqpoint{4.334639in}{2.655241in}}%
\pgfpathlineto{\pgfqpoint{4.385523in}{2.673755in}}%
\pgfpathlineto{\pgfqpoint{4.436407in}{2.692158in}}%
\pgfpathlineto{\pgfqpoint{4.487290in}{2.710451in}}%
\pgfpathlineto{\pgfqpoint{4.538174in}{2.728637in}}%
\pgfpathlineto{\pgfqpoint{4.589058in}{2.746718in}}%
\pgfpathlineto{\pgfqpoint{4.639942in}{2.764694in}}%
\pgfpathlineto{\pgfqpoint{4.690826in}{2.782569in}}%
\pgfpathlineto{\pgfqpoint{4.741710in}{2.800344in}}%
\pgfpathlineto{\pgfqpoint{4.792593in}{2.818019in}}%
\pgfpathlineto{\pgfqpoint{4.843477in}{2.835598in}}%
\pgfpathlineto{\pgfqpoint{4.894361in}{2.853082in}}%
\pgfpathlineto{\pgfqpoint{4.945245in}{2.870471in}}%
\pgfpathlineto{\pgfqpoint{4.996129in}{2.887769in}}%
\pgfpathlineto{\pgfqpoint{5.047013in}{2.904975in}}%
\pgfpathlineto{\pgfqpoint{5.097896in}{2.922092in}}%
\pgfpathlineto{\pgfqpoint{5.148780in}{2.939121in}}%
\pgfpathlineto{\pgfqpoint{5.199664in}{2.956063in}}%
\pgfpathlineto{\pgfqpoint{5.250548in}{2.972920in}}%
\pgfpathlineto{\pgfqpoint{5.301432in}{2.989692in}}%
\pgfpathlineto{\pgfqpoint{5.352316in}{3.006382in}}%
\pgfpathlineto{\pgfqpoint{5.403199in}{3.022990in}}%
\pgfpathlineto{\pgfqpoint{5.454083in}{3.039517in}}%
\pgfpathlineto{\pgfqpoint{5.504967in}{3.055965in}}%
\pgfpathlineto{\pgfqpoint{5.555851in}{3.072335in}}%
\pgfpathlineto{\pgfqpoint{5.606735in}{3.088627in}}%
\pgfpathlineto{\pgfqpoint{5.657619in}{3.104844in}}%
\pgfpathlineto{\pgfqpoint{5.708502in}{3.120986in}}%
\pgfusepath{stroke}%
\end{pgfscope}%
\begin{pgfscope}%
\pgfsetrectcap%
\pgfsetmiterjoin%
\pgfsetlinewidth{0.803000pt}%
\definecolor{currentstroke}{rgb}{0.000000,0.000000,0.000000}%
\pgfsetstrokecolor{currentstroke}%
\pgfsetdash{}{0pt}%
\pgfpathmoveto{\pgfqpoint{0.671003in}{0.548486in}}%
\pgfpathlineto{\pgfqpoint{0.671003in}{3.243486in}}%
\pgfusepath{stroke}%
\end{pgfscope}%
\begin{pgfscope}%
\pgfsetrectcap%
\pgfsetmiterjoin%
\pgfsetlinewidth{0.803000pt}%
\definecolor{currentstroke}{rgb}{0.000000,0.000000,0.000000}%
\pgfsetstrokecolor{currentstroke}%
\pgfsetdash{}{0pt}%
\pgfpathmoveto{\pgfqpoint{5.708502in}{0.548486in}}%
\pgfpathlineto{\pgfqpoint{5.708502in}{3.243486in}}%
\pgfusepath{stroke}%
\end{pgfscope}%
\begin{pgfscope}%
\pgfsetrectcap%
\pgfsetmiterjoin%
\pgfsetlinewidth{0.803000pt}%
\definecolor{currentstroke}{rgb}{0.000000,0.000000,0.000000}%
\pgfsetstrokecolor{currentstroke}%
\pgfsetdash{}{0pt}%
\pgfpathmoveto{\pgfqpoint{0.671003in}{0.548486in}}%
\pgfpathlineto{\pgfqpoint{5.708502in}{0.548486in}}%
\pgfusepath{stroke}%
\end{pgfscope}%
\begin{pgfscope}%
\pgfsetrectcap%
\pgfsetmiterjoin%
\pgfsetlinewidth{0.803000pt}%
\definecolor{currentstroke}{rgb}{0.000000,0.000000,0.000000}%
\pgfsetstrokecolor{currentstroke}%
\pgfsetdash{}{0pt}%
\pgfpathmoveto{\pgfqpoint{0.671003in}{3.243486in}}%
\pgfpathlineto{\pgfqpoint{5.708502in}{3.243486in}}%
\pgfusepath{stroke}%
\end{pgfscope}%
\end{pgfpicture}%
\makeatother%
\endgroup%

     \caption{Contact radius}
     \label{fig:contact-radius}
\end{figure}

\section{Adjacent work}
\label{sec:adjacent}

\begin{figure}[ht]
    \centering
    \begin{subfigure}[b]{.49\linewidth}
        \centering
        \begin{tikzpicture}
    \draw[darkblue, ultra thick, ->] (0,-3) to (0, -1.8);
    \draw[darkblue, ultra thick, ->] (0, -1.5) to (0, -0.25);
    
    \node[rectangle, fill=darkblue] at (0, -3) {\color{white}{ResolutionPenaltyQuadratic}};
    \node[rectangle, fill=darkblue] at (0, -1.5) {\color{white}{ResolutionPenalty}};
    \node[rectangle, fill=darkblue] at (0, 0) {\color{white}{Resolution}};
  \end{tikzpicture}
        \caption{Initial class diagram of resolutions}\label{fig:resolution-initial}
    \end{subfigure}
    \begin{subfigure}[b]{.49\linewidth}
        \centering
        \begin{tikzpicture}
    \draw[darkblue, ultra thick, ->] (0, -1.5) to (0, -0.25);

    \draw[darkblue, ultra thick] (-1.66, -1.5) rectangle (1.66, -2.5);
    \node[darkblue] at (0, -2.15) {PenaltyFunction};

    \node[rectangle, fill=darkblue] at (0, -1.5) {\color{white}{ResolutionPenalty}};
    \node[rectangle, fill=darkblue] at (0, 0) {\color{white}{Resolution}};

  \end{tikzpicture}
        \caption{Refactored class diagram of resolutions}\label{fig:resolution-refactor}
    \end{subfigure}
    \caption{Combine linear and quadratic penalty resolution by templating the classes
    with a penalty function.}
    \label{fig:resolution}
\end{figure}

\section{Discussion}
\label{sec:discussion}

Flaws when same nodes that were dumped are added back in (no convergence) 

Usually interpenetrating nodes won't be considered part of interface in next iteration

\bibliography{biblio.bib}

\end{document}